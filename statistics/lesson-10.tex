\documentclass[12pt]{article}

\usepackage[T1]{fontenc}
\usepackage[utf8]{inputenc}
\usepackage[english,russian]{babel}
\usepackage[margin=2cm]{geometry}
\usepackage{amsmath}
\usepackage{amsfonts}

% команды вывода первой частной производной
\newcommand{\fpd}[1]{\frac{\partial}{\partial #1}}
\newcommand{\fpda}[2]{\frac{\partial #1}{\partial #2}}
\newcommand{\fpdp}[2]{\fpd{#2} \left ( #1 \right )}

\newcommand{\expectation}[1]{\mathtt{M} \left [ #1 \right ]}
\newcommand{\conditionalexpectation}[2]{\expectation{ #1 \left | #2 \right .}}
\newcommand{\variance}[1]{\mathtt{D} \left [ #1 \right ]}
\newcommand{\covariance}[2]{\mathtt{cov} \left ( #1, #2 \right )}

\newcommand{\modulus}[1]{\left | #1 \right |}
\newcommand{\norm}[1]{\left \| {#1} \right \|}

\newcommand{\event}[1]{\left \{ #1 \right \} }
\newcommand{\probability}[1]{P \event{#1}}


\begin{document}

    \title{Домашнее задание №10}
    \author{Тигетов Давид Георгиевич}
    \date{}
    \maketitle

    \section*{Задача 10.1}
    Используя счётную аддитивность вероятностной меры, на основе определения цепи Маркова из п.2, доказать, что для произвольного $k$ ($1 \le k \le n$) выполняется равенство:
    \[
        \conditionalprobability{X_k = k}{X_{k-1} = x_{k-1}, \dots, X_0 = x_0} = \conditionalprobability{X_k = k}{X_{k-1} = x_{k-1}}.
    \]

    \subsection*{Решение:}
    Счётная аддитивность вероятностной меры приводит к счётной формуле полной вероятности: пусть $B$ --- произвольное событий и $A_i$ --- счётное разбиение множества всех элементарных исходов
    $\Omega$, тогда
    \[
        B
        = B \cdot \Omega
        = B \cdot \bigcup_{k=0}^{\infty} A_k
        = \bigcup_{k=0}^{\infty} B A_k
    \]
    И поскольку множества $B A_k$ не пересекаются, то в силу счётной аддитивности:
    \[
        \probability{B}
        = \probability{\bigcup_{k=0}^{\infty} B A_k}
        = \sum_{k=0}^{\infty} \probability{B A_k}
        = \sum_{k=0}^{\infty} \conditionalprobability{B}{A_k} \probability{A_k} .
    \]

    Пусть $X^{(0)}$ --- множество всех возможных значений случайной величины $X_0$, множество $X^{(0)}$ конечно либо счётно. В силу условий нормировки и счётной аддитивности меры:
    \begin{gather*}
        1 = \probability{X_0 \in X^{(0)}} , \\
        1 = \probability{\bigcup_{\widetilde{x}_0 \in X^{(0)}} X_0 = \widetilde{x}_0 }, \\
        1 = \sum_{\widetilde{x}_0 \in X^{(0)}} \probability{X_0 = \widetilde{x}_0 } ,
    \end{gather*}
    и умножая последнее равенство на $f_k(x_k, \dots, x_1, x_0)$, получим:
    \[
        f_k(x_k, \dots, x_1, x_0) = f_k(x_k, \dots, x_1, x_0) \sum_{\widetilde{x}_0 \in X^{(0)}} \probability{X_0 = \widetilde{x}_0} .
    \]
    По условию $f_k(x_k, \dots, x_1, x_0)$ зависит от $x_0$ формально, то есть не изменяет своё значение при изменении величины $x_0$, поэтому значение
    $f_k(x_k, \dots, x_1, x_0)$ можно внести в сумму:
    \[
        f_k(x_k, \dots, x_1, x_0)
        = \sum_{\widetilde{x}_0 \in X^{(0)}} f_k(x_k, \dots, x_1, x_0) \probability{X_0 = \widetilde{x}_0}
        = \sum_{\widetilde{x}_0 \in X^{(0)}} f_k(x_k, \dots, x_1, \widetilde{x}_0) \probability{X_0 = \widetilde{x}_0} .
    \]
    Далее, используя счётную формулу полной вероятности, получим:
    \begin{multline*}
        f_k(x_k, \dots, x_0)
        = \sum_{\widetilde{x}_0 \in X^{(0)}} \conditionalprobability{X_k = k}{X_{k-1} = x_{k-1}, \dots, X_1 = x_1, X_0 = \widetilde{x}_0} \probability{X_0 = \widetilde{x}_0} = \\
        %
        = \conditionalprobability{X_k = k}{X_{k-1} = x_{k-1}, \dots, X_1 = x_1} .
    \end{multline*}
    Таким образом,
    \[
        f_k(x_k, \dots, x_0) = \conditionalprobability{X_k = k}{X_{k-1} = x_{k-1}, \dots, X_1 = x_1} .
    \]
    Повторяя аналогичную процедуру для множества значений величин $X_1$, \dots, $X_{k-2}$, получим последовательность равенств:
    \begin{gather*}
        f_k(x_k, \dots, x_0) = \conditionalprobability{X_k = k}{X_{k-1} = x_{k-1}, \dots, X_2 = x_2} , \\
        \dots \\
        f_k(x_k, \dots, x_0) = \conditionalprobability{X_k = k}{X_{k-1} = x_{k-1}, X_{k-2} = x_{k-2}} , \\
        f_k(x_k, \dots, x_0) = \conditionalprobability{X_k = k}{X_{k-1} = x_{k-1}} .
    \end{gather*}
    Откуда следует:
    \[
        \conditionalprobability{X_k = k}{X_{k-1} = x_{k-1}, \dots, X_0 = x_0} = \conditionalprobability{X_k = k}{X_{k-1} = x_{k-1}} .
    \]

    \section*{Задача 10.5}
    Используя марковское свойство, доказать, что $p_{ij}(n) = \conditionalprobability{X_n = j}{X_0 = i}$ --- $(i,j)$-й элемент матрицы $P^n$.

    \subsection*{Решение:}
    Докажем по индукции.

    При $n=1$ вероятности $p_{ij}(1) = \conditionalprobability{X_1 = j}{X_0 = i}$ являются элементами матрицы $P$.

    Пусть вероятности $p_{ij}(n-1) = \conditionalprobability{X_{n-1} = j}{X_0 = i}$ являются элементами матрицы $P^{n-1}$. Пусть числа $1$, \dots, $M$ обозначают состояния цепи, тогда
    по формуле полной вероятности:
    \begin{multline*}
        p_{ij}(n)
        = \conditionalprobability{X_n = j}{X_0 = i}
        = \sum_{k=1}^M \conditionalprobability{X_n = j, X_{n-1} = k}{X_0 = i} = \\
        %
        = \sum_{k=1}^M \conditionalprobability{X_n = j}{X_{n-1} = k, X_0 = i} \conditionalprobability{X_{n-1} = k}{X_0 = i}
    \end{multline*}
    Поскольку цепь является марковской, то
    \[
        \conditionalprobability{X_n = j}{X_{n-1} = k, X_0 = i}
        = \conditionalprobability{X_n = j}{X_{n-1} = k} .
    \]
    В силу однородности цепи:
    \[
        \conditionalprobability{X_n = j}{X_{n-1} = k}
        = \conditionalprobability{X_1 = j}{X_0 = k},
    \]
    тогда
    \begin{multline*}
        p_{ij}(n)
        = \sum_{k=1}^M \conditionalprobability{X_1 = j}{X_0 = k} \conditionalprobability{X_{n-1} = k}{X_0 = i} = \\
        %
        = \sum_{k=1}^M p_{kj}(1) p_{ik}(n-1)
        = \sum_{k=1}^M p_{ik}(n-1) p_{kj}(1)
    \end{multline*}
    Согласно правилу умножения "строка на столбец"{} сумма в правой части соответствует элементу $(i,j)$ произведения $P^{n-1} \cdot P = P^n$.

\end{document}