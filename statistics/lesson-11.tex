\documentclass[12pt]{article}

\usepackage[T1]{fontenc}
\usepackage[utf8]{inputenc}
\usepackage[english,russian]{babel}
\usepackage[margin=2cm]{geometry}
\usepackage{amsmath}
\usepackage{amsfonts}

% команды вывода первой частной производной
\newcommand{\fpd}[1]{\frac{\partial}{\partial #1}}
\newcommand{\fpda}[2]{\frac{\partial #1}{\partial #2}}
\newcommand{\fpdp}[2]{\fpd{#2} \left ( #1 \right )}

\newcommand{\expectation}[1]{\mathtt{M} \left [ #1 \right ]}
\newcommand{\conditionalexpectation}[2]{\expectation{ #1 \left | #2 \right .}}
\newcommand{\variance}[1]{\mathtt{D} \left [ #1 \right ]}
\newcommand{\covariance}[2]{\mathtt{cov} \left ( #1, #2 \right )}

\newcommand{\modulus}[1]{\left | #1 \right |}
\newcommand{\norm}[1]{\left \| {#1} \right \|}

\newcommand{\event}[1]{\left \{ #1 \right \} }
\newcommand{\probability}[1]{P \event{#1}}


\begin{document}

    \title{Домашнее задание №11}
    \author{Тигетов Давид Георгиевич}
    \date{}
    \maketitle

    \section*{Задача 11.1}
    \begin{enumerate}
        \item По индукции с использованием формулы свёртки найти формулу для плотности распределения случайной величины $S_n$ из определения 1 пуассоновского процесса.
        \item Используя равенство $\probability{N_t = k} = \probability{S_k \le t, S_{k+1} > t} = \probability{S_{k+1} > t} - \probability{S_k > t}$, интегрированием по частям установить,
        что $\probability{N_t = k} = \frac{(\lambda t)^k}{k!} e^{-\lambda t}$.
    \end{enumerate}

    \subsection*{Решение:}
    \begin{enumerate}
        \item Все величины $T_i$ имеют одинаковую плотность распределения $f(x)$:
        \[
            f(x) = \derivative{x} F(x) = \lambda e^{-\lambda x} .
        \]

        Величина $S_1 = T_1$ имеет плотность распределения $f_1(x)$:
        \[
            f_1(x) = f(x) = \lambda e^{-\lambda x} .
        \]

        Величина $S_2 = S_1 + T_2$ имеет плотность распределения $f_2(x)$:
        \[
            f_2(x)
            = \int \limits_{-\infty}^{\infty} f_1(t) f(x-t) dt
            = \int \limits_0^x \lambda e^{- \lambda t} \lambda e^{- \lambda (x-t)} dt
            = \lambda^2 \int \limits_0^x e^{- \lambda x} dt
            = \lambda^2 x e^{- \lambda x} .
        \]

        Величина $S_3 = S_2 + T_3$ имеет плотность распределения $f_3(x)$:
        \[
            f_3(x)
            = \int \limits_{-\infty}^{\infty} f_2(t) f(x-t) dt
            = \int \limits_0^x \lambda^2 t e^{- \lambda t} \lambda e^{- \lambda (x-t)} dt
            = \lambda^3 \int \limits_0^x t dt e^{- \lambda x}
            = \lambda^3 \frac{x^2}{2} e^{- \lambda x} .
        \]

        Теперь видна закономерность: в результате свертки добавляется множитель $\lambda \frac{1}{n-1}$ и на единицу увеличивается степень $x$. Для доказательства по индукции предположим, что для величина $S_{n-1}$
        имеет плотность распределения:
        \[
            f_{n-1}(x) = \lambda^{n-1} \frac{x^{n-2}}{(n-2)!} e^{-x} ,
        \]
        тогда величина $S_n = S_{n-1} + T_n$ имеет плотность распредления $f_n(x)$:
        \begin{multline*}
            f_n(x)
            = \int \limits_{-\infty}^{\infty} f_{n-1}(t) f(x-t) dt
            = \int \limits_0^x \lambda^{n-1} \frac{t^{n-2}}{(n-2)!} e^{-x} \lambda e^{- \lambda (x - t )} dt = \\
            %
            = \lambda^n \frac{1}{(n-2)!} \int \limits_0^x t^{n-2} dt e^{- \lambda x}
            = \lambda^n \frac{1}{(n-2)!} \frac{x^{n-1}}{n-1} e^{- \lambda x}
            = \lambda^n \frac{x^{n-1}}{(n-1)!} e^{- \lambda x} .
        \end{multline*}
        Плотность распределения Эрланга.

        \item Необходимо вычислить вероятность вида:
        \begin{multline*}
            \probability{S_k > t}
            = \int \limits_t^{\infty} f_k(x) dx
            = \int \limits_t^{\infty} \lambda^n \frac{x^{n-1}}{(n-1)!} e^{- \lambda x} dx = \\
            %
            =
            \left . \frac{x^{n-1}}{(n-1)!} \lambda^{n-1} \left ( - e^{-\lambda x } \right ) \right |_t^{\infty}
            - \left . \frac{x^{n-2}}{(n-2)!} \lambda^{n-2} \left ( e^{-\lambda x } \right ) \right |_t^{\infty}
            + \left . \frac{x^{n-3}}{(n-3)!} \lambda^{n-3} \left ( - e^{-\lambda x } \right ) \right |_t^{\infty}
            - ...
            - \left . e^{-\lambda x} \right |_t^\infty = \\
            %
            = \frac{t^{n-1}}{(n-1)!} \lambda^{n-1} e^{-\lambda t}
            + \frac{t^{n-2}}{(n-2)!} \lambda^{n-2} e^{-\lambda t}
            + ...
            + e^{-\lambda t},
        \end{multline*}
        поскольку $\lim \limits_{x \rightarrow \infty} x^k e^{- \lambda x} = 0$.

        Согласно равенству из условия задачи:
        \begin{multline*}
            \probability{N_t = k}
            = \probability{S_{k+1} > t} - \probability{S_k > t} = \\
            %
            \shoveleft{
                = \frac{t^k}{k!} \lambda^{k} e^{-\lambda t}
                + \frac{t^{k-1}}{(k-1)!} \lambda^{k-1} e^{-\lambda t}
                + ...
                + e^{-\lambda t} - } \\
            %
            \shoveright{
                - \frac{t^{k-1}}{(k-1)!} \lambda^{k-1} e^{-\lambda t}
                - \frac{t^{k-2}}{(k-2)!} \lambda^{k-2} e^{-\lambda t}
                - ...
                - e^{-\lambda t} } \\
            %
            = \frac{t^k}{k!} \lambda^{k} e^{-\lambda t}
        \end{multline*}
    \end{enumerate}

\end{document}