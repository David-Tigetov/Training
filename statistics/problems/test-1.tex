\documentclass[12pt]{article}

\usepackage[T1]{fontenc}
\usepackage[utf8]{inputenc}
\usepackage[english,russian]{babel}
\usepackage[margin=2cm]{geometry}
\usepackage{amsmath}
\usepackage{wasysym}

% команды вывода первой частной производной
\newcommand{\fpd}[1]{\frac{\partial}{\partial #1}}
\newcommand{\fpda}[2]{\frac{\partial #1}{\partial #2}}
\newcommand{\fpdp}[2]{\fpd{#2} \left ( #1 \right )}

\newcommand{\expectation}[1]{\mathtt{M} \left [ #1 \right ]}
\newcommand{\conditionalexpectation}[2]{\expectation{ #1 \left | #2 \right .}}
\newcommand{\variance}[1]{\mathtt{D} \left [ #1 \right ]}
\newcommand{\covariance}[2]{\mathtt{cov} \left ( #1, #2 \right )}

\newcommand{\modulus}[1]{\left | #1 \right |}
\newcommand{\norm}[1]{\left \| {#1} \right \|}

\newcommand{\event}[1]{\left \{ #1 \right \} }
\newcommand{\probability}[1]{P \event{#1}}


\begin{document}
    \title{Тест 1}
    \author{Тигетов Давид}
    \date{}
    \maketitle

    \section*{Вопрос 1}
    Какими свойствами обладает произвольная функция распределения?

    \subsection*{Ответ:}
    Пусть $F(x)$ --- функция распределения, тогда
    \begin{enumerate}
        \item $\forall x: 0 \le F(x) \le 1$,
        \item $\lim \limits_{x \rightarrow - \infty}$ F(x) = 0,
        \item $\lim \limits_{x \rightarrow + \infty}$ F(x) = 1,
        \item $x_1 \le x_2 \Rightarrow F(x_1) \le F(x_2)$.
    \end{enumerate}

    \section*{Вопрос 9}
    Чему равна нижняя выборочная квартиль для выборки с элементами 1, 3, 7?

    \subsection*{Ответ:}
    $\frac{1+3}{2} = 2.$

    \section*{Вопрос 17}
    Будет ли асимптотически нормальной выборочная медиана в модели сдвига закона Коши?

    \subsection*{Ответ:}
    Да.

    По теореме (Крамер) выборочная медиана является асимпотически нормальной, поскольку плотность вероятности распределения Коши:
    \[
        f(x) = \frac{1}{\pi b \left ( 1 + \left ( \frac{x - a}{b} \right )^2 \right )} > 0
    \]
    положительна в точке медианы, равной параметру сдвига $a$.

    \section*{Вопрос 25}
    Чему равна информация Фишера в модели сдвига нормального распределения с известной дисперсией?

    \subsection*{Ответ:}
    Пусть $\xi$ --- случайная величина, имеющая нормальное распределение $\mathcal{N} \left ( m, \sigma \right)$, тогда функция плотности вероятности величины $\xi$:
    \[
        f_\xi(x;m) = \frac{1}{\sqrt{2 \pi} \sigma} e^{- \frac{1}{2} \frac{\left ( x-m \right)^2}{\sigma^2}}.
    \]
    Откуда
    \begin{gather}
        \ln f_\xi(x;m) = \ln \left ( \frac{1}{\sqrt{2 \pi} \sigma} \right ) - \frac{1}{2} \frac{\left ( x-m \right)^2}{\sigma^2} , \\
        \derivative{m} f_\xi(x;m) =  - \frac{1}{2} \frac{2 \left ( x-m \right ) (-1)}{\sigma^2} = \frac{x-m}{\sigma^2} .
    \end{gather}
    Информация Фишера $I(m)$:
    \[
        I(m)
        = \expectation{\left ( \derivative{m} f_\xi(\xi;m) \right )^2}
        = \expectation{\left ( \frac{\xi-m}{\sigma^2} \right )^2}
        = \frac{\expectation{\left ( \xi - m \right )^2}}{\sigma^4}
        = \frac{\sigma^2}{\sigma^4}
        = \frac{1}{\sigma^2}.
    \]

    \section*{Будет ли конечным математическое ожидание числа шагов в критерии Вальда?}
    \subsection*{Ответ:}
    В некоторых случаях будет.

    Например, в случае когда для случайной величины ($\xi_i$ --- последовательность наблюдений)
    \[
        Z_i = \ln \left ( \frac{f(\xi_i; \theta_0)}{f(\xi_i; \theta_1)} \right )
    \]
    выполняются условия:
    \begin{enumerate}
        \item $\expectation{Z_i} \neq 0$,
        \item $0 < \variance{Z_i} < \infty$.
    \end{enumerate}
\end{document}