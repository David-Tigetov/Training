\documentclass[12pt]{article}

\usepackage[T1]{fontenc}
\usepackage[utf8]{inputenc}
\usepackage[english,russian]{babel}
\usepackage[margin=2cm]{geometry}
\usepackage{amsmath}

% команды вывода первой частной производной
\newcommand{\fpd}[1]{\frac{\partial}{\partial #1}}
\newcommand{\fpda}[2]{\frac{\partial #1}{\partial #2}}
\newcommand{\fpdp}[2]{\fpd{#2} \left ( #1 \right )}

\newcommand{\expectation}[1]{\mathtt{M} \left [ #1 \right ]}
\newcommand{\conditionalexpectation}[2]{\expectation{ #1 \left | #2 \right .}}
\newcommand{\variance}[1]{\mathtt{D} \left [ #1 \right ]}
\newcommand{\covariance}[2]{\mathtt{cov} \left ( #1, #2 \right )}

\newcommand{\modulus}[1]{\left | #1 \right |}
\newcommand{\norm}[1]{\left \| {#1} \right \|}

\newcommand{\event}[1]{\left \{ #1 \right \} }
\newcommand{\probability}[1]{P \event{#1}}


\begin{document}

    \title{Домашнее задание №3}
    \author{Тигетов Давид Георгиевич}
    \date{}
    \maketitle

    \section*{Задача 3.2}
    Рассмотрим модель сдвига распределения Лапласа: плотность распределения элементов выборки есть $f_\theta(x) = \frac{1}{2} e^{-\modulus{x - \theta}}$, где параметр $\theta$ ---
    неизвестное действительное число. Вычислить относительную асимптотическую эффективность $e_{MED,\overline{X}}$. Какая из оценок точнее в этой модели?

    \subsection*{Решение:}
    Асимптотическая дисперсия $\sigma_{MED}^2$ медианной оценки определяется теоремой:
    \[
        \sigma_{MED}^2
        = \frac{1}{n} \frac{1}{4 f_\theta^2(x_\frac{1}{2})}
        = \frac{1}{n} \frac{1}{4 f_\theta^2(\theta)}
        = \frac{1}{n} \frac{1}{4 \left ( \frac{1}{2} \right )^2}
        = \frac{1}{n} ,
    \]
    поскольку медиана совпадает с $\theta$.

    Дисперсия среднего $\sigma_{\overline{X}}^2$:
    \[
        \sigma_{\overline{X}}^2
        = \frac{1}{n^2} \sum_{i=1}^n 2
        = \frac{1}{n^2} n 2
        = \frac{2}{n} ,
    \]
    поскольку дисперсия случайной величины с плотностью вероятности $f_\theta(x)$ равна 2.

    Асимптотическая эффективность:
    \[
        e_{MED,\overline{X}}
        = \frac{\sigma_{MED}^2}{\sigma_{\overline{X}}^2}
        = \frac{\frac{1}{n}}{\frac{2}{n}}
        = \frac{1}{2}.
    \]
    Отсюда следует, что оценка $\overline{X}$ точнее медианной оценки.

    \subsection*{Ответ:}
    $e_{MED,\overline{X}} = \frac{1}{2}$, выборочное среднее точнее медианной оценки.

    \section*{Задача 3.6}
    В нормальной модели $\mathcal{N} \left ( \theta, 1 \right )$, найти ОМП и вычислить информацию Фишера.

    \subsection*{Решение:}
    Пусть $\xi_1$, \dots, $\xi_n$ --- выборка из распределения $\mathcal{N} \left ( \theta, 1 \right )$, тогда плотность вероятности выборки:
    \[
        f(x_1, \dots, x_n)
        = \frac{1}{\left ( 2 \pi \right )^\frac{n}{2}} e^{-\frac{1}{2} \sum_{i=1}^n (x_i - \theta)^2}
    \]
    Функция правдоподобия:
    \begin{gather*}
        L(\xi_1, \dots, \xi_n) = f(\xi_1, \dots, \xi_n) , \\
        \ln L(\xi_1, \dots, \xi_n) = \ln \frac{1}{\left ( 2 \pi \right )^\frac{n}{2}} - \frac{1}{2} \sum_{i=1}^n (\xi_i - \theta)^2 , \\
        \derivative{\theta} \ln L(\xi_1, \dots, \xi_n) =  \sum_{i=1}^n (\xi_i - \theta) .
    \end{gather*}
    Отсюда ОМП оценка $\widehat{\theta}$:
    \begin{gather*}
        \left . \derivative{\theta} \ln L(\xi_1, \dots, \xi_n) \right |_{\theta = \widehat{\theta}} = 0 , \\
        \sum_{i=1}^n (\xi_i - \widehat{\theta}) = 0 , \\
        \widehat{\theta} = \frac{1}{n} \sum_{i=1}^n \xi_i .
    \end{gather*}
    Информация Фишера $I(\theta)$:
    \[
        I(\theta)
        = - \expectation{\Kderivative{2}{\theta} \ln L(\xi_1, \dots, \xi_n)}
        = - \expectation{\sum_{i=1}^n (-1)}
        = n .
    \]

    \subsection*{Ответ:}
    \begin{enumerate}
        \item ОМП оценка $\frac{1}{n} \sum_{i=1}^n \xi_i$,
        \item информация Фишера $n$.
    \end{enumerate}
\end{document}