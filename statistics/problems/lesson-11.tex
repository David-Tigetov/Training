\documentclass[12pt]{article}

\usepackage[T1]{fontenc}
\usepackage[utf8]{inputenc}
\usepackage[english,russian]{babel}
\usepackage[margin=2cm]{geometry}
\usepackage{amsmath}
\usepackage{amsfonts}

% команды вывода первой частной производной
\newcommand{\fpd}[1]{\frac{\partial}{\partial #1}}
\newcommand{\fpda}[2]{\frac{\partial #1}{\partial #2}}
\newcommand{\fpdp}[2]{\fpd{#2} \left ( #1 \right )}

\newcommand{\expectation}[1]{\mathtt{M} \left [ #1 \right ]}
\newcommand{\conditionalexpectation}[2]{\expectation{ #1 \left | #2 \right .}}
\newcommand{\variance}[1]{\mathtt{D} \left [ #1 \right ]}
\newcommand{\covariance}[2]{\mathtt{cov} \left ( #1, #2 \right )}

\newcommand{\modulus}[1]{\left | #1 \right |}
\newcommand{\norm}[1]{\left \| {#1} \right \|}

\newcommand{\event}[1]{\left \{ #1 \right \} }
\newcommand{\probability}[1]{P \event{#1}}


\begin{document}

    \title{Домашнее задание №11}
    \author{Тигетов Давид Георгиевич}
    \date{}
    \maketitle

    \section*{Задача 11.1}
    По индукции с использованием формулы свёртки найти формулу для плотности распределения случайной величины $S_n$ из определения 1 пуассоновского процесса.

    Используя равенство $\probability{N_t = k} = \probability{S_k \le t, S_{k+1} > t} = \probability{S_{k+1} > t} - \probability{S_k > t}$, интегрированием по частям установить,
    что $\probability{N_t = k} = \frac{(\lambda t)^k}{k!} e^{-\lambda t}$.

    \subsection*{Решение:}
    Все величины $T_i$ имеют одинаковую плотность распределения $f(x)$:
    \[
        f(x) = \derivative{x} F(x) = \lambda e^{-\lambda x} .
    \]

    Величина $S_1 = T_1$ имеет плотность распределения $f_1(x)$:
    \[
        f_1(x) = f(x) = \lambda e^{-\lambda x} .
    \]

    Величина $S_2 = S_1 + T_2$ имеет плотность распределения $f_2(x)$:
    \[
        f_2(x)
        = \int \limits_{-\infty}^{\infty} f_1(t) f(x-t) dt
        = \int \limits_0^x \lambda e^{- \lambda t} \lambda e^{- \lambda (x-t)} dt
        = \lambda^2 \int \limits_0^x e^{- \lambda x} dt
        = \lambda^2 x e^{- \lambda x} .
    \]

    Величина $S_3 = S_2 + T_3$ имеет плотность распределения $f_3(x)$:
    \[
        f_3(x)
        = \int \limits_{-\infty}^{\infty} f_2(t) f(x-t) dt
        = \int \limits_0^x \lambda^2 t e^{- \lambda t} \lambda e^{- \lambda (x-t)} dt
        = \lambda^3 \int \limits_0^x t dt e^{- \lambda x}
        = \lambda^3 \frac{x^2}{2} e^{- \lambda x} .
    \]

    Теперь видна закономерность: в результате свертки добавляется множитель $\lambda \frac{1}{n-1}$ и на единицу увеличивается степень $x$. Для доказательства по индукции предположим, что для величина $S_{n-1}$
    имеет плотность распределения:
    \[
        f_{n-1}(x) = \lambda^{n-1} \frac{x^{n-2}}{(n-2)!} e^{-x} ,
    \]
    тогда величина $S_n = S_{n-1} + T_n$ имеет плотность распредления $f_n(x)$:
    \begin{multline*}
        f_n(x)
        = \int \limits_{-\infty}^{\infty} f_{n-1}(t) f(x-t) dt
        = \int \limits_0^x \lambda^{n-1} \frac{t^{n-2}}{(n-2)!} e^{-x} \lambda e^{- \lambda (x - t )} dt = \\
        %
        = \lambda^n \frac{1}{(n-2)!} \int \limits_0^x t^{n-2} dt e^{- \lambda x}
        = \lambda^n \frac{1}{(n-2)!} \frac{x^{n-1}}{n-1} e^{- \lambda x}
        = \lambda^n \frac{x^{n-1}}{(n-1)!} e^{- \lambda x} .
    \end{multline*}
    Плотность распределения Эрланга.

    Необходимо вычислить вероятность вида:
    \begin{multline*}
        \probability{S_k > t}
        = \int \limits_t^{\infty} f_k(x) dx
        = \int \limits_t^{\infty} \lambda^n \frac{x^{n-1}}{(n-1)!} e^{- \lambda x} dx = \\
        %
        =
        \left . \frac{x^{n-1}}{(n-1)!} \lambda^{n-1} \left ( - e^{-\lambda x } \right ) \right |_t^{\infty}
        - \left . \frac{x^{n-2}}{(n-2)!} \lambda^{n-2} \left ( e^{-\lambda x } \right ) \right |_t^{\infty}
        + \left . \frac{x^{n-3}}{(n-3)!} \lambda^{n-3} \left ( - e^{-\lambda x } \right ) \right |_t^{\infty}
        - ...
        - \left . e^{-\lambda x} \right |_t^\infty = \\
        %
        = \frac{t^{n-1}}{(n-1)!} \lambda^{n-1} e^{-\lambda t}
        + \frac{t^{n-2}}{(n-2)!} \lambda^{n-2} e^{-\lambda t}
        + ...
        + e^{-\lambda t},
    \end{multline*}
    поскольку $\lim \limits_{x \rightarrow \infty} x^k e^{- \lambda x} = 0$.

    Согласно равенству из условия задачи:
    \begin{multline*}
        \probability{N_t = k}
        = \probability{S_{k+1} > t} - \probability{S_k > t} = \\
        %
        \shoveleft{
            = \frac{t^k}{k!} \lambda^{k} e^{-\lambda t}
            + \frac{t^{k-1}}{(k-1)!} \lambda^{k-1} e^{-\lambda t}
            + ...
            + e^{-\lambda t} - } \\
        %
        \shoveright{
            - \frac{t^{k-1}}{(k-1)!} \lambda^{k-1} e^{-\lambda t}
            - \frac{t^{k-2}}{(k-2)!} \lambda^{k-2} e^{-\lambda t}
            - ...
            - e^{-\lambda t} } \\
        %
        = \frac{t^k}{k!} \lambda^{k} e^{-\lambda t}
    \end{multline*}


    \section*{Задача 11.5}
    Используя условие ординарности, оценить в соотношении (3) сумму от 0 до $k-2$, затем с помощью формул (2), устремив $h \rightarrow 0$, получить систему дифференциальных уравнений для
    функций $p_k(t)$, $k=0,1, \dots$.

    Сделав замену $r_k(t) = e^{\lambda t} p_k(t)$ и используя начальные условия $p_0(0)=1$, $p_k(0)=0$ при $k \ge 1$, найти решение этой системы.

    \subsection*{Решение:}
    Прежде всего рассмотрим вероятность $p_0(t)$. Заметим, что отсутствие точек за время $t+h$ это произведение двух независимых событий отсутствия точек за время $t$ и $h$:
    \[
        p_0(t + h) = p_0(t) \cdot p_0(h)
    \]
    Такому равенству удовлетворяет, например, показательная функция $e^{a t}$. Из условия ординарности следует, что при малых $h$:
    \[
        p_0(h)
        = 1 - \sum_{k=1}^\infty p_k(h)
        = 1 - p_1(h) - \sum_{k=2}^\infty p_k(h)
        = 1 - \lambda h - o(h)
    \]
    откуда $a = -\lambda$ и $p_0(t) = e^{- \lambda t}$.

    Теперь в соотношении (3) используя свойство ординарности, получим:
    \begin{multline*}
        p_k(t+h)
        = \sum_{j=0}^k p_j(t) p_{k-j}(h)
        = \sum_{j=0}^{k-2} p_j(t) p_{k-j}(h) + p_{k-1}(t) p_1(h) + p_k(t) p_0(h) = \\
        %
        = \sum_{j=0}^{k-2} p_j(t) o(h) + p_{k-1}(t) p_1(h) + p_k(t) p_0(h) = \\
        %
        = \sum_{j=0}^{k-2} p_j(t) o(h) + p_{k-1}(t) \left ( \lambda h + o(h) \right ) + p_k(t) \left ( 1 - \lambda h - o(h) \right )
    \end{multline*}
    Откуда:
    \begin{gather*}
        p_k(t+h) - p_k(t) = \sum_{j=0}^{k-2} p_j(t) o(h) + p_{k-1}(t) \left ( \lambda h + o(h) \right ) - p_k(t) \left ( \lambda h + o(h) \right ) , \\
        \frac{p_k(t+h) - p_k(t)}{h} = \sum_{j=0}^{k-2} p_j(t) \frac{o(h)}{h} + p_{k-1}(t) \left ( \lambda + \frac{o(h)}{h} \right ) - p_k(t) \left ( \lambda + \frac{o(h)}{h} \right ).
    \end{gather*}
    Переходя к пределу при $h \rightarrow \infty$, получим:
    \begin{gather*}
        p_k^\prime(t) = p_{k-1}(t) \lambda - p_k(t) \lambda , \\
        p_k^\prime(t) + \lambda p_k(t) = \lambda p_{k-1}(t).
    \end{gather*}

    Известно, что $p_0(t) = e^{- \lambda t}$, пусть $p_{k-1}(t) = \frac{(\lambda t)^{k-1}}{(k-1)!} e^{-\lambda t}$, тогда для $p_k(t)$ получим уравнение:
    \[
        p_k^\prime(t) + \lambda p_k(t) = \lambda p_{k-1}(t).
    \]
    Решением однородного уравнения является функция $c_1 e^{-\lambda t}$, а решение неоднородного уравнения получим вариацией постоянной:
    \begin{gather*}
        c_1^\prime(t) e^{- \lambda t} - \lambda c_1(t) e^{- \lambda t} + \lambda c_1(t) e^{- \lambda t} = \lambda \frac{(\lambda t)^{k-1}}{(k-1)!} e^{-\lambda t} , \\
        c_1^\prime(t) = \lambda \frac{(\lambda t)^{k-1}}{(k-1)!} , \\
        c_1(t) = \frac{(\lambda t)^{k}}{k!} + c_0 .
    \end{gather*}
    Учитывая условие $p_k(0) = 0$ получим $c_0 = 0$ и:
    \[
        p_k(t) = \frac{(\lambda t)^{k}}{k!} e^{- \lambda t}.
    \]
\end{document}