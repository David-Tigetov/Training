\documentclass[12pt]{article}

\usepackage[T1]{fontenc}
\usepackage[utf8]{inputenc}
\usepackage[english,russian]{babel}
\usepackage[margin=2cm]{geometry}
\usepackage{amsmath}

% команды вывода первой частной производной
\newcommand{\fpd}[1]{\frac{\partial}{\partial #1}}
\newcommand{\fpda}[2]{\frac{\partial #1}{\partial #2}}
\newcommand{\fpdp}[2]{\fpd{#2} \left ( #1 \right )}

\newcommand{\expectation}[1]{\mathtt{M} \left [ #1 \right ]}
\newcommand{\conditionalexpectation}[2]{\expectation{ #1 \left | #2 \right .}}
\newcommand{\variance}[1]{\mathtt{D} \left [ #1 \right ]}
\newcommand{\covariance}[2]{\mathtt{cov} \left ( #1, #2 \right )}

\newcommand{\modulus}[1]{\left | #1 \right |}
\newcommand{\norm}[1]{\left \| {#1} \right \|}

\newcommand{\event}[1]{\left \{ #1 \right \} }
\newcommand{\probability}[1]{P \event{#1}}


\begin{document}

    \title{Домашнее задание №4}
    \author{Тигетов Давид Георгиевич}
    \date{}
    \maketitle

    \section*{Задача 4.1}
    Вычислить квадратичный риск байесовской оценки из п.3 с $a = b = \frac{\sqrt{n}}{2}$. Сравнить её с оценкой Ходжеса-Лемана из п.3 темы 2.2.

    \subsection*{Решение:}
    Байесовская оценка $\widehat{\theta}$ имеет вид:
    \[
        \widehat{\theta} = \frac{a}{a + b + n} + \frac{1}{a + b + n} \sum_{i=1}^n X_i,
    \]
    где $X_1, \dots, X_n$ --- выборка из распределения Бернулли.

    Функция квадратичного риска $R(\theta)$ имеет вид:
    \[
        R(\theta)
        = \expectation{\left ( \widehat{\theta} - \theta \right )^2}
        = \variance{\widehat{\theta}}
        + \left ( \expectation{\widehat{\theta}} - \theta \right )^2 ,
    \]
    где
    \begin{gather*}
        \variance{\widehat{\theta}}
        = \frac{1}{\left ( a + b + n \right )^2} \sum_{i=1}^n \variance{X_i}
        = \frac{n}{\left ( a + b + n \right )^2} \theta \left ( 1 - \theta \right ), \\
        %
        \expectation{\widehat{\theta}}
        = \frac{a}{a + b + n} + \frac{1}{a + b + n} \sum_{i=1}^n \expectation{X_i}
        = \frac{a}{a + b + n} + \frac{n}{a + b + n} \theta
    \end{gather*}
    При $a = b = \frac{\sqrt{n}}{2}$:
    \begin{gather*}
        \variance{\widehat{\theta}}
        = \frac{n}{\left ( \frac{\sqrt{n}}{2} + \frac{\sqrt{n}}{2} + n \right )^2} \theta \left ( 1 - \theta \right )
        = \frac{n}{n \left ( 1 + \sqrt{n} \right )^2} \theta \left ( 1 - \theta \right )
        = \frac{\theta \left ( 1 - \theta \right )}{\left ( 1 + \sqrt{n} \right )^2}, \\
        %
        \expectation{\widehat{\theta}}
        = \frac{\frac{\sqrt{n}}{2}}{\frac{\sqrt{n}}{2} + \frac{\sqrt{n}}{2} + n} + \frac{n}{\frac{\sqrt{n}}{2} + \frac{\sqrt{n}}{2} + n} \theta
        = \frac{\frac{1}{2}}{1 + \sqrt{n}} + \frac{\sqrt{n}}{1 + \sqrt{n}} \theta , \\
        %
        \expectation{\widehat{\theta}} - \theta
        = \frac{\frac{1}{2}}{1 + \sqrt{n}} + \frac{\sqrt{n}}{1 + \sqrt{n}} \theta - \theta
        = \frac{\frac{1}{2}}{1 + \sqrt{n}} - \frac{\theta}{1 + \sqrt{n}}
        = \frac{\frac{1}{2} - \theta}{1 + \sqrt{n}} .
    \end{gather*}

    Таким образом, функция квадратичного риска
    \[
        R(\theta)
        = \frac{\theta \left ( 1 - \theta \right )}{\left ( 1 + \sqrt{n} \right )^2}
        + \frac{\left ( \frac{1}{2} - \theta \right )^2}{\left ( 1 + \sqrt{n} \right )^2}
        = \frac{\theta - \theta^2 + \frac{1}{4} - \theta + \theta^2}{\left ( 1 + \sqrt{n} \right )^2}
        = \frac{\frac{1}{4}}{\left ( 1 + \sqrt{n} \right )^2} .
    \]

    При $a = b = \frac{\sqrt{n}}{2}$ байесовская оценка имеет вид:
    \[
        \widehat{\theta}
        = \frac{\frac{\sqrt{n}}{2}}{\frac{\sqrt{n}}{2} + \frac{\sqrt{n}}{2} + n} + \frac{1}{\frac{\sqrt{n}}{2} + \frac{\sqrt{n}}{2} + n} \sum_{i=1}^n X_i
        = \frac{\frac{1}{2}}{1 + \sqrt{n}} + \frac{1}{1 + \sqrt{n}} \frac{1}{\sqrt{n}} \sum_{i=1}^n X_i
    \]
    а оценка Ходжеса-Лемана имеет вид:
    \begin{multline*}
        \widetilde{\theta}
        = \frac{1}{n} \sum_{i=1}^n X_i + \frac{1}{1 + \sqrt{n}} \left ( \frac{1}{2} - \frac{1}{n} \sum_{i=1}^n X_i \right )
        = \frac{\frac{1}{2}}{1 + \sqrt{n}} + \frac{\sqrt{n}}{1 + \sqrt{n}} \frac{1}{n} \sum_{i=1}^n X_i = \\
        %
        = \frac{\frac{1}{2}}{1 + \sqrt{n}} + \frac{1}{1 + \sqrt{n}} \frac{1}{\sqrt{n}} \sum_{i=1}^n X_i
    \end{multline*}

    \subsection*{Ответ:}
    Функция квадратичного риска $R(\theta) = \frac{\frac{1}{4}}{\left ( 1 + \sqrt{n} \right )^2}$. Байесовкая оценка совпадает с оценкой Ходжеса--Лемана.

    \section*{Задача 4.5}
    Пусть $t > 0 $. Найти байесовскую оценку для функции надёжности $\varphi(\theta) = \probability{X_i > t} = e^{-t \theta}$ в условиях задачи 4.2.

    \subsection*{Решение:}
    Плотности вероятности $f_{X_i}(x | \theta)$ величин $X_i$:
    \[
        f_{X_i}(x | \theta) = \theta e^{- \theta x} I_{\left ( 0, + \infty\right )}(x) .
    \]
    Плотность вероятности $f_X(x_1, \dots, x_n | \theta)$ выборки $X_1, \dots, X_n$:
    \[
        f_X(x_1, \dots, x_n | \theta) = \theta^n e^{- \theta \sum_{i=1}^n x_i} \prod_{i=1}^n I_{\left ( 0, + \infty\right )}(x_i) .
    \]
    Совместная плотность $X_1, \dots, X_n$ и $\widetilde{\theta}$:
    \begin{multline*}
        f_{X,\widetilde{\theta}}(x_1, \dots, x_n, \theta)
        = f_X(x_1, \dots, x_n | \theta) \cdot q_{\widetilde{\theta}}(\theta) = \\
        %
        = \theta^n e^{- \theta \sum_{i=1}^n x_i} \prod_{i=1}^n I_{\left ( 0, + \infty\right )}(x_i) \cdot \frac{\sigma^p \theta^{p-1} e^{-\sigma \theta}}{\Gamma(p)} I_{\left ( 0, + \infty\right )}(\theta) = \\
        %
        = \frac{\sigma^p}{\Gamma(p)} \theta^{n+p-1} e^{- \left ( \sum_{i=1}^n x_i + \sigma \right ) \theta } I_{\left ( 0, + \infty\right )}(\theta) \prod_{i=1}^n I_{\left ( 0, + \infty\right )}(x_i) = \\
    \end{multline*}
    Апостериорная плотность вероятности:
    \[
        q_{\widetilde{\theta}}(\theta | x_1, \dots, x_n) = \frac{f_{X,\widetilde{\theta}}(x_1, \dots, x_n, \theta)}{\int \limits_0^{+ \infty} f_{X,\widetilde{\theta}}(x_1, \dots, x_n, \theta) d \theta} ,
    \]
    где
    \begin{multline*}
        \int \limits_0^{+ \infty} f_{X,\widetilde{\theta}}(x_1, \dots, x_n, \theta) d \theta
        = \int \limits_0^{+ \infty} \frac{\sigma^p}{\Gamma(p)} \theta^{n+p-1} e^{- \left ( \sum_{i=1}^n x_i + \sigma \right ) \theta } d \theta \prod_{i=1}^n I_{\left ( 0, + \infty\right )}(x_i) = \\
        %
        = \frac{\sigma^p}{\Gamma(p)} \frac{1}{\left ( \sum_{i=1}^n x_i + \sigma \right )^{n+p}} \int \limits_0^{+ \infty} \left ( \sum_{i=1}^n x_i + \sigma \right )^{n+p} \theta^{n+p-1} e^{- \left ( \sum_{i=1}^n x_i + \sigma \right ) \theta } d \theta \prod_{i=1}^n I_{\left ( 0, + \infty\right )}(x_i) = \\
        %
        = \frac{\sigma^p}{\Gamma(p)} \frac{\Gamma(n+p)}{\left ( \sum_{i=1}^n x_i + \sigma \right )^{n+p}} \prod_{i=1}^n I_{\left ( 0, + \infty\right )}(x_i) .
    \end{multline*}
    Тогда апостериорная плотность:
    \begin{multline*}
        q_{\widetilde{\theta}}(\theta | x_1, \dots, x_n)
        = \frac{\frac{\sigma^p}{\Gamma(p)} \theta^{n+p-1} e^{- \left ( \sum_{i=1}^n x_i + \sigma \right ) \theta } I_{\left ( 0, + \infty\right )}(\theta) \prod_{i=1}^n I_{\left ( 0, + \infty\right )}(x_i)}{\frac{\sigma^p}{\Gamma(p)} \frac{\Gamma(n+p)}{\left ( \sum_{i=1}^n x_i + \sigma \right )^{n+p}} \prod_{i=1}^n I_{\left ( 0, + \infty\right )}(x_i)} = \\
        %
        = \frac{\theta^{n+p-1} e^{- \left ( \sum_{i=1}^n x_i + \sigma \right ) \theta }}{\frac{\Gamma(n+p)}{\left ( \sum_{i=1}^n x_i + \sigma \right )^{n+p}}} I_{\left ( 0, + \infty\right )}(\theta) \prod_{i=1}^n I_{\left ( 0, + \infty\right )}(x_i) = \\
        = \frac{\left ( \sum_{i=1}^n x_i + \sigma \right )^{n+p} \theta^{n+p-1} e^{- \left ( \sum_{i=1}^n x_i + \sigma \right ) \theta }}{\Gamma(n+p)} I_{\left ( 0, + \infty\right )}(\theta) \prod_{i=1}^n I_{\left ( 0, + \infty\right )}(x_i) .
    \end{multline*}

    Байесовская оценка надёжности:
    \begin{multline*}
        \widetilde{\varphi}
        = \int \limits_0^{+\infty} e^{-t \theta} \frac{\left ( \sum_{i=1}^n x_i + \sigma \right )^{n+p} \theta^{n+p-1} e^{- \left ( \sum_{i=1}^n x_i + \sigma \right ) \theta }}{\Gamma(n+p)} I_{\left ( 0, + \infty\right )}(\theta) \prod_{i=1}^n I_{\left ( 0, + \infty\right )}(x_i) d \theta = \\
        %
        = \frac{\left ( \sum_{i=1}^n x_i + \sigma \right )^{n+p}}{\Gamma(n+p)} \frac{1}{\left ( \sum_{i=1}^n x_i + \sigma + t \right )^{n+p}} \int \limits_0^{+\infty} \left ( \sum_{i=1}^n x_i + \sigma + t \right )^{n+p} \theta^{n+p-1} e^{- \left ( \sum_{i=1}^n x_i + \sigma +t \right ) \theta} d \theta \prod_{i=1}^n I_{\left ( 0, + \infty\right )}(x_i) = \\
        %
        = \frac{\left ( \sum_{i=1}^n x_i + \sigma \right )^{n+p}}{\Gamma(n+p)} \frac{1}{\left ( \sum_{i=1}^n x_i + \sigma + t \right )^{n+p}} \Gamma(n+p) \prod_{i=1}^n I_{\left ( 0, + \infty\right )}(x_i) = \\
        %
        = \frac{\left ( \sum_{i=1}^n x_i + \sigma \right )^{n+p}}{\left ( \sum_{i=1}^n x_i + \sigma + t \right )^{n+p}} \prod_{i=1}^n I_{\left ( 0, + \infty\right )}(x_i)
        = \left ( \frac{\sum_{i=1}^n x_i + \sigma}{\sum_{i=1}^n x_i + \sigma + t} \right )^{n+p} \prod_{i=1}^n I_{\left ( 0, + \infty\right )}(x_i) .
    \end{multline*}
    Таким образом,
    \[
        \varphi(X_1, \dots, X_n)
        = \left ( \frac{\sum_{i=1}^n X_i + \sigma}{\sum_{i=1}^n X_i + \sigma + t} \right )^{n+p}
        = \left ( \frac{1}{1 + \frac{t}{\sum_{i=1}^n X_i + \sigma}} \right )^{n+p}.
    \]

    \subsection*{Ответ:}
    $\left ( \frac{1}{1 + \frac{t}{\sum_{i=1}^n X_i + \sigma}} \right )^{n+p}$.
\end{document}