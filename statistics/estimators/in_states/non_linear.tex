\section{Нелинейность состояний и наблюдений}

Представим, что состояния и наблюдения связаны рекуррентными нелинейными соотношениями:
\begin{gather*}
    \xi_{k+1} = F_k ( \xi_k, u_k ) + v_k , \\
    \eta_{k+1} = H_k ( \xi_{k+1} ) + w_k,
\end{gather*}
где
\begin{enumerate}
    \item $u_k$ --- известные детерминированные управляющие воздействия,
    \item $F_k(\cdot, \cdot)$ и $H_k(\cdot, \cdot)$ --- известные, но нелинейные функции,
    \item о начальном состоянии $\xi_0$ известны математическое ожидание $\expectation{\xi_0}$ и ковариация $P_{\xi_0, \xi_0}$,
    \item $v_k$ и $w_k$ --- шумовые случайные величины с нулевыми математическими ожиданиями $\expectation{v_k} = 0$ и $\expectation{w_k} = 0$, и известными
        ковариациями $P_{v_k, v_k}$ и $P_{w_k, w_k}$,
    \item все величины $\xi_0$, $v_k$ и $w_k$ некоррелированы.
\end{enumerate}

Как и ранее требуется построить оценку состояния $\xi_k$ по наблюдениями $\eta_1$, \dots, $\eta_k$.

Формально для решения задачи можно использовать уравнения линейной оптимальной фильтрации \eqref{equation:filtering:in_states:process:optimal_estimate} и
\eqref{equation:filtering:in_states:process:optimal_error_covariance}, но в силу нелинойности функций $F_k$ и $H_k$ вычисление математических ожиданий и ковариаций
потребует неизвестных распределений, поэтому решить задачу в её исходной постановке с помощью линейной фильтрации не получается.

Тем не менее уравнения линейной фильтрации можно использовать для построения субоптимальных оценок, если вместо точных величин использовать их приближения,
что формально приводит к замене исходной на линеаризованную задачу.