\section{Фильтр Калмана}~\label{section:filtering:in_states:kalman}

\subsection{Постановка задачи}

Представим, что в процессе $\xi_k \in L_2^m$ каждая следующая величина $\xi_{k+1}$ связана с предыдущей величиной $\xi_k$ рекуррентным соотношением:
\begin{equation} \label{equation:filtering:in_states:kalman:states}
	\xi_{k+1} = F_k \xi_k + G_k u_k + v_k ,
\end{equation}
где
\begin{enumerate}
	\item $F_k$ --- известные вещественные матрицы размера $m \times m$,
	\item $G_k$ --- известные вещественные матрицы размера $m \times r$,
	\item $u_k$ --- известные вещественные векторы размера $r$, представляющие управляющие воздействия,
	\item $v_k$ --- случайные величины из пространства $L_2^m$, представляющие случайные воздействия; величины $v_k$ попарно некоррелированны между собой,
		некоррелированы с начальным состоянием $\xi_0$ и имеют нулевое математическое ожидание:
		$$
			\expectation{v_k} = 0 .
		$$
\end{enumerate}

Для начального состояния $\xi_0$ известны:
\begin{enumerate}
	\item математическое ожидание $\expectation{\xi_0}$,
	\item ковариация $P_{\xi_0, \xi_0} = \outerexpectation{\xi_0}{\xi_0}$.
\end{enumerate}

Величины $\xi_k$ не доступны наблюдению, а наблюдается последовательность случайных величин $\eta_k \in L_2^n$, в которой величины $\eta_k$ связаны с величинами $\xi_k$
линейным равенством:
\begin{equation} \label{equation:filtering:in_states:kalman:observations}
	\eta_k = H_k \xi_k + w_k ,
\end{equation}
в котором
\begin{enumerate}
	\item $H_k$ --- известные вещественные матрицы размера $n \times m$,
	\item $w_k$ --- случайные величины из пространства $L_2^n$, представляющие ошибки измерений; величины $w_k$ попарно некоррелированы между собой и некоррелированы
		с величинами $\xi_0$ и $v_0$, \dots $v_k$, и имеют нулевое математическое ожидание:
		$$
			\expectation{w_{k+1}} = 0 .
		$$
\end{enumerate}

Требуется по наблюдениям $\eta_1$, \dots, $\eta_{k+1}$ построить линейную несмещенную и оптимальную оценку состояния $\xi_{k+1}$.

Согласно разделу \ref{section:filtering:in_states:process} такая оценка и ковариация её ошибки должны иметь вид равенств
\eqref{equation:filtering:in_states:process:optimal_estimate} и \eqref{equation:filtering:in_states:process:optimal_error_covariance} и остается лишь вычислить
необходимые оценки и ковариации.

\subsection{Прогноз состояния}

Прогноз состояния $\widehat{\xi}_{k+1,1|k}$ является линейной несмещенной и оптимальной оценкой состояния $\xi_{k+1}$ для момента времени $k+1$, построенная по
наблюдениям $\eta_1$, \dots, $\eta_k$:
\begin{equation} \label{equation:filtering:in_states:kalman:state_prediction}
	\widehat{\xi}_{k+1, 1|k} = \expectation{\xi_{k+1}} + P_{\xi_{k+1}, \eta_{1|k}} P_{\eta_{1|k}, \eta_{1|k}}^{-1} \centered{\eta_{1|k}} .
\end{equation}
С помощью равенств \eqref{equation:filtering:in_states:kalman:states_and_observations:state_expectation} и
\eqref{equation:filtering:in_states:kalman:states_and_observations:state_and_observations_covariance} выражение для оценки $\widehat{\xi}_{k+1, 1|k}$ преобразуется к виду:
\begin{multline} \label{equation:filtering:in_states:kalman:state_prediction:recursion}
	\widehat{\xi}_{k+1, 1|k}
		= F_k \expectation{\xi_k} + G_k u_k + F_k P_{\xi_k, \eta_{1|k}} P_{\eta_{1|k},\eta_{1|k}}^{-1} \centered{\eta_{1|k}} = \\
	%
	= F_k \left ( \expectation{\xi_k} + P_{\xi_k, \eta_{1|k}} P_{\eta_{1|k},\eta_{1|k}}^{-1} \centered{\eta_{1|k}} \right ) + G_k u_k
		= F_k \widehat{\xi}_{k,1|k} + G_k u_k
	,
\end{multline}
в котором $\widehat{\xi}_{k,1|k}$ --- линейная несмещенная и оптимальная оценка состояния $\xi_k$, построенная по наблюдениям $\eta_1$, \dots, $\eta_k$.

В соответствии с уравнением состояний \eqref{equation:filtering:in_states:kalman:states} и равенством
\eqref{equation:filtering:in_states:kalman:state_prediction:recursion} ошибка $\widetilde{\xi}_{k+1, 1|k}$ принимает вид:
$$
	\widetilde{\xi}_{k+1,1|k}
		= \xi_{k+1} - \widehat{\xi}_{k+1,1|k}
		= F_k \xi_k + G_k u_k + v_k - F_k \widehat{\xi}_{k,1|k} - G_k u_k
		= F_k \left ( \xi_k - \widehat{\xi}_{k,1|k} \right ) + v_k
	.
$$
В силу несмещенности оценки $\widehat{\xi}_{k,1|k}$:
\begin{equation} \label{equation:filtering:in_states:kalman:state_prediction:unbiasedness}
	\expectation{\widetilde{\xi}_{k+1,1|k}}
		= F_k \left ( \expectation{\xi_k} - \expectation{\widehat{\xi}_{k,1|k}} \right ) + \expectation{v_k}
		= F_k \left ( \expectation{\xi_k} - \expectation{\xi_k} \right )
		= 0
	,
\end{equation}
поскольку величины $v_k$ имеют нулевое математическое ожидание, поэтому ковариация ошибки:
\begin{multline*}
	P_{\widetilde{\xi}_{k+1,1|k}, \widetilde{\xi}_{k+1,1|k}}
		= \outerexpectation{\widetilde{\xi}_{k+1,1|k}}{\widetilde{\xi}_{k+1,1|k}} = \\
	%
	= \expectation{ \left ( F_k \left ( \xi_k - \widehat{\xi}_{k,1|k} \right ) + v_k \right ) \left ( F_k \left ( \xi_k - \widehat{\xi}_{k,1|k} \right ) + v_k \right )^T } = \\
	%
	\shoveleft{
		= F_k \expectation{ \left ( \xi_k - \widehat{\xi}_{k,1|k} \right ) \left ( \xi_k - \widehat{\xi}_{k,1|k} \right )^T } F_k^T
			+ F_k \expectation{ \left ( \xi_k - \widehat{\xi}_{k,1|k} \right ) v_k^T } +
	} \\
	\shoveright{
		+ \expectation{ v_k \left ( \xi_k - \widehat{\xi}_{k,1|k} \right )^T } F_k^T
			+ \expectation{ v_k v_k^T } =
	} \\
	%
	= F_k \expectation{ \left ( \xi_k - \widehat{\xi}_{k,1|k} \right ) \left ( \xi_k - \widehat{\xi}_{k,1|k} \right )^T } F_k^T + \expectation{ v_k v_k^T }
\end{multline*}
где
\begin{gather*}
	\expectation{ \left ( \xi_k - \widehat{\xi}_{k,1|k} \right ) v_k^T } = 0 , \\
	\expectation{ v_k \left ( \xi_k - \widehat{\xi}_{k,1|k} \right )^T } = 0 ,
\end{gather*}
поскольку величина $\xi_k$ зависит только от величин $\xi_0$ и $v_0$, \dots, $v_{k-1}$, которые некоррелированы с $v_k$ по условию задачи, а оценка
$\widehat{\xi}_{k,1|k}$ зависит от величин $\xi_0$, $v_0$, \dots, $v_{k-1}$ и $w_1$, \dots, $w_k$, которые также некоррелированы с $v_k$ по условию задачи. Таким образом,
\begin{equation} \label{equation:filtering:in_states:kalman:state_prediction:error_covariance_recursion}
	P_{\widetilde{\xi}_{k+1,1|k}, \widetilde{\xi}_{k+1,1|k}} = F_k P_{\widetilde{\xi}_{k,1|k}, \widetilde{\xi}_{k,1|k}} F_k^T + P_{v_k, v_k} ,
\end{equation}
где $P_{\widetilde{\xi}_{k,1|k}, \widetilde{\xi}_{k,1|k}}$ --- ковариация ошибки оценки $\widehat{\xi}_{k,1|k}$.

\subsection{Прогноз наблюдения}

Прогноз наблюдения $\widehat{\eta}_{k+1,1|k}$ является линейной несмещенной и оптимальной оценкой наблюдения $\eta_{k+1}$, построенной по наблюдениям $\eta_1$, \dots, $\eta_k$:
\begin{equation} \label{equation:filtering:in_states:kalman:observation_prediction}
	\widehat{\eta}_{k+1, 1|k} = \expectation{\eta_{k+1}} + P_{\eta_{k+1}, \eta_{1|k}} P_{\eta_{1|k}, \eta_{1|k}}^{-1} \centered{\eta_{1|k}} .
\end{equation}
Согласно выражению центрированного наблюдения \eqref{equation:filtering:in_states:kalman:states_and_observations:centered_observation}:
\begin{multline*}
	P_{\eta_{k+1}, \eta_{1|k}}
		= \outerexpectation{\eta_{k+1}}{\eta_{1|k}} = \\
	%
	= \expectation{\left ( H_{k+1} \left ( \xi_{k+1} - \expectation{\xi_{k+1}} \right ) + w_{k+1} \right ) \centered{\eta_{1|k}}^T} = \\
	%
	= H_{k+1} \outerexpectation{\xi_{k+1}}{\eta_{1|k}} + \expectation{w_{k+1} \centered{\eta_{1|k}}^T} = \\
	%
	= H_{k+1} P_{\xi_{k+1}, \eta_{1|k}}
	,
\end{multline*}
поскольку величины объединенного вектора $\eta_{1|k}$ зависят только от величин $\xi_0$, $v_0$, \dots, $v_k$ и $w_1$, \dots, $w_k$, с которыми величина $w_{k+1}$
некоррелирована по условию задачи.

С учетом равенства \eqref{equation:filtering:in_states:kalman:states_and_observations:observation_expectation} выражение
\eqref{equation:filtering:in_states:kalman:observation_prediction} для оценки $\widehat{\eta}_{k+1,1|k}$ принимает вид:
\begin{multline} \label{equation:filtering:in_states:kalman:observation_prediction:recursion}
	\widehat{\eta}_{k+1, 1|k}
		= H_{k+1} \expectation{\xi_{k+1}} + H_{k+1} P_{\xi_{k+1}, \eta_{1|k}} P_{\eta_{1|k}, \eta_{1|k}}^{-1} \centered{\eta_{1|k}} = \\
	%
	= H_{k+1} \left ( \expectation{\xi_{k+1}} + P_{\xi_{k+1}, \eta_{1|k}} P_{\eta_{1|k}, \eta_{1|k}}^{-1} \centered{\eta_{1|k}} \right )
		= H_{k+1} \widehat{\xi}_{k+1,1|k}
	,
\end{multline}
где $\widehat{\xi}_{k+1,1|k}$ --- прогноз состояния \eqref{equation:filtering:in_states:kalman:state_prediction}.

Теперь ошибке $\widetilde{\eta}_{k+1,1|k}$ в соответствии c уравнением наблюдения \eqref{equation:filtering:in_states:kalman:observations} и полученным выражением оценки
\eqref{equation:filtering:in_states:kalman:observation_prediction:recursion} можно придать вид:
\begin{multline} \label{equation:filtering:in_states:kalman:observation_prediction:error}
	\widetilde{\eta}_{k+1,1|k}
		= \eta_{k+1} - \widehat{\eta}_{k+1,1|k}
		= H_{k+1} \xi_{k+1} + w_{k+1} - H_{k+1} \widehat{\xi}_{k+1,1|k} = \\
	%
	= H_{k+1} \left ( \xi_{k+1} - \widehat{\xi}_{k+1,1|k} \right ) + w_{k+1}
	,
\end{multline}
причем в силу несмещенности оценки $\widehat{\eta}_{k+1,1|k}$:
\begin{equation} \label{equation:filtering:in_states:kalman:observation_prediction:unbiasedness}
	\expectation{\widetilde{\eta}_{k+1,1|k}}
		= \expectation{\eta_{k+1}} - \expectation{\widehat{\eta}_{k+1,1|k}}
		= \expectation{\eta_{k+1}} - \expectation{\eta_{k+1}}
		= 0
	,
\end{equation}
и ковариация ошибки:
\begin{multline*}
	P_{\widetilde{\eta}_{k+1, 1|k}, \widetilde{\eta}_{k+1, 1|k}}
		= \outerexpectation{\widetilde{\eta}_{k+1, 1|k}}{\widetilde{\eta}_{k+1, 1|k}} = \\
	%
	= \expectation{ \left ( H_{k+1} \left ( \xi_{k+1} - \widehat{\xi}_{k+1,1|k} \right ) + w_{k+1} \right ) \left ( H_{k+1} \left ( \xi_{k+1} - \widehat{\xi}_{k+1,1|k} \right ) + w_{k+1} \right )^T} = \\
	%
	\shoveleft{
		= H_{k+1} \expectation{ \left ( \xi_{k+1} - \widehat{\xi}_{k+1,1|k} \right ) \left ( \xi_{k+1} - \widehat{\xi}_{k+1,1|k} \right )^T } H_{k+1}
		+ H_{k+1} \expectation{ \left ( \xi_{k+1} - \widehat{\xi}_{k+1,1|k} \right ) w_{k+1}^T } +
	} \\
	\shoveright{
		+ \expectation{ w_{k+1} \left ( \xi_{k+1} - \widehat{\xi}_{k+1,1|k} \right )^T } H_{k+1}^T +
		\expectation{ w_{k+1}^T w_{k+1} } =
	} \\
	%
	= H_{k+1} \expectation{\left ( \xi_{k+1} - \widehat{\xi}_{k+1,1|k} \right ) \left ( \xi_{k+1} - \widehat{\xi}_{k+1,1|k} \right )^T } H_{k+1}^T
		+ \expectation{ w_{k+1}^T w_{k+1} }
	,
\end{multline*}
где
\begin{gather*}
	\expectation{ \left ( \xi_{k+1} - \widehat{\xi}_{k+1,1|k} \right ) w_{k+1}^T } = 0 , \\
	\expectation{ w_{k+1} \left ( \xi_{k+1} - \widehat{\xi}_{k+1,1|k} \right )^T } = 0 ,
\end{gather*}
поскольку величины $\xi_{k+1}$ и $w_{k+1}$ некоррелированы по условию задачи, а оценка $\widehat{\xi}_{k+1,1|k}$ зависит только от величин $\xi_0$, $v_0$, \dots, $v_{k-1}$
и $w_1$, \dots, $w_{k}$, которые некоррелированы с $w_{k+1}$ по условию задачи. Таким образом,
\begin{equation} \label{equation:filtering:in_states:kalman:observation_prediction:error_covariance_recursion}
	P_{\widetilde{\eta}_{k+1, 1|k}, \widetilde{\eta}_{k+1, 1|k}} = H_{k+1} P_{\widetilde{\xi}_{k+1,1|k}, \widetilde{\xi}_{k+1,1|k}} H_{k+1}^T + P_{w_{k+1}, w_{k+1}}
	.
\end{equation}
где $P_{\widetilde{\xi}_{k+1,1|k}, \widetilde{\xi}_{k+1,1|k}}$ --- ковариация \eqref{equation:filtering:in_states:kalman:state_prediction:error_covariance_recursion}
ошибки оценки $\widehat{\xi}_{k+1,1|k}$.

\subsection{Ковариация с ошибкой прогноза наблюдения}

При вычислении ковариации
$$
	P_{\xi_{k+1}, \widetilde{\eta}_{k+1, 1|k}}
		= \expectation{\centered{\xi_{k+1}} \centered{\widetilde{\eta}_{k+1, 1|k}}^T}
$$
первый множитель под знаком математического ожидания представим с помощью ошибки прогноза состояния:
\begin{gather*}
	\widetilde{\xi}_{k+1,1|k} = \xi_{k+1} - \widehat{\xi}_{k+1,1|k} , \\
	\xi_{k+1} = \widehat{\xi}_{k+1,1|k} + \widetilde{\xi}_{k+1,1|k} ,
\end{gather*}
в результате получим выражение:
\begin{multline*}
	\xi_{k+1} - \expectation{\xi_{k+1}}
	= \centered{\widehat{\xi}_{k+1,1|k} + \widetilde{\xi}_{k+1,1|k}} = \\
	%
	= \centered{\widehat{\xi}_{k+1,1|k}} + \centered{\widetilde{\xi}_{k+1,1|k}}
	= \centered{\widehat{\xi}_{k+1,1|k}} + \widetilde{\xi}_{k+1,1|k} ,
\end{multline*}
где последнее равенство получено в силу несмещенности \eqref{equation:filtering:in_states:kalman:state_prediction:unbiasedness}.

Второй множитель под знаком математического ожидания преобразуем в силу несмещенности \eqref{equation:filtering:in_states:kalman:observation_prediction:unbiasedness}
и представления ошибки \eqref{equation:filtering:in_states:kalman:observation_prediction:error}:
$$
	\widetilde{\eta}_{k+1, 1|k} - \expectation{\widetilde{\eta}_{k+1, 1|k}}
		= \widetilde{\eta}_{k+1, 1|k}
		= H_{k+1} \left ( \xi_{k+1} - \widehat{\xi}_{k+1,1|k} \right ) + w_{k+1}
		= H_{k+1} \widetilde{\xi}_{k+1,1|k} + w_{k+1} .
$$
Используя полученные выражения вычислим ковариацию:
\begin{multline*}
	P_{\xi_{k+1}, \widetilde{\eta}_{k+1, 1|k}}
		= \expectation{\left ( \centered{\widehat{\xi}_{k+1,1|k}} + \widetilde{\xi}_{k+1,1|k} \right ) \left ( H_{k+1} \widetilde{\xi}_{k+1,1|k} + w_{k+1} \right )^T } = \\
	%
	= \expectation{\centered{\widehat{\xi}_{k+1,1|k}} \left ( H_{k+1} \widetilde{\xi}_{k+1,1|k} + w_{k+1} \right )^T }
		+ \expectation{\widetilde{\xi}_{k+1,1|k} \left ( H_{k+1} \widetilde{\xi}_{k+1,1|k} + w_{k+1} \right )^T } = \\
	%
	\shoveleft{
		= \expectation{\centered{\widehat{\xi}_{k+1,1|k}} \widetilde{\xi}_{k+1,1|k}^T} H_{k+1}^T
		+ \expectation{\centered{\widehat{\xi}_{k+1,1|k}} w_{k+1}^T } +
	} \\
	+ \expectation{\widetilde{\xi}_{k+1,1|k} \widetilde{\xi}_{k+1,1|k}^T } H_{k+1}^T
	+ \expectation{\widetilde{\xi}_{k+1,1|k} w_{k+1}^T } .
\end{multline*}
Заметим, что
$$
	\expectation{\centered{\widehat{\xi}_{k+1,1|k}} \widetilde{\xi}_{k+1,1|k}^T} = 0
$$
в силу свойства ортогональности оптимальной оценки и её ошибки \eqref{equation:filtering:in_states:variable:error_and_estimate_orthogonality}, и
\begin{gather*}
	\expectation{\centered{\widehat{\xi}_{k+1,1|k}} w_{k+1}^T } = 0 , \\
	\expectation{\widetilde{\xi}_{k+1,1|k} w_{k+1}^T } = 0 ,
\end{gather*}
поскольку величины $\widehat{\xi}_{k+1,1|k}$ и $\widetilde{\xi}_{k+1,1|k}$ зависят от величин $\xi_0$, $v_0$, \dots, $v_k$ и $w_1$, \dots, $w_k$, а величина
$w_{k+1}$ с ними некоррелирована по условию задачи. Таким образом,
\begin{multline} \label{equation:filtering:in_states:kalman:state_and_innovation:covariance}
	P_{\xi_{k+1}, \widetilde{\eta}_{k+1, 1|k}}
		= \expectation{\widetilde{\xi}_{k+1,1|k} \widetilde{\xi}_{k+1,1|k}^T } H_{k+1}^T = \\
	%
	= \expectation{\centered{\widetilde{\xi}_{k+1,1|k}} \centered{\widetilde{\xi}_{k+1,1|k}}^T } H_{k+1}^T = \\
	%
	= P_{\widetilde{\xi}_{k+1,1|k}, \widetilde{\xi}_{k+1,1|k}} H_{k+1}^T ,
\end{multline}
в силу несмещенности \eqref{equation:filtering:in_states:kalman:state_prediction:unbiasedness}.

\subsection{Состояния и наблюдения}

Из уравнения состоянии \eqref{equation:filtering:in_states:kalman:states}:
$$
	\xi_{k+1} = F_k \xi_k + G_k u_k + v_k
$$
математическое ожидание
\begin{equation} \label{equation:filtering:in_states:kalman:states_and_observations:state_expectation}
	\expectation{\xi_{k+1}}
		= F_k \expectation{\xi_k} + G_k u_k + \expectation{v_k}
		= F_k \expectation{\xi_k} + G_k u_k
	,
\end{equation}
поскольку величины $v_k$ имеют нулевое математическое ожидание по условию задачи, центрированное состояние:
$$
	\xi_{k+1} - \expectation{\xi_{k+1}} = F_k \centered{\xi_k} + v_k .
$$
и ковариация:
\begin{multline} \label{equation:filtering:in_states:kalman:states_and_observations:state_and_observations_covariance}
	P_{\xi_{k+1}, \eta_{1|k}}
		= \outerexpectation{\xi_{k+1}}{\eta_{1|k}} = \\
	%
	= \expectation{  \left ( F_k \centered{\xi_k} + v_k \right ) \centered{\eta_{1|k}}^T } = \\
	%
	= F_k \outerexpectation{\xi_k}{\eta_{1|k}} + \expectation{ v_k \centered{\eta_{1|k}}^T }
		= F_k P_{\xi_{k}, \eta_{1|k}}
	,
\end{multline}
где
$$
	\expectation{ v_k \centered{\eta_{1|k}}^T } = 0 ,
$$
поскольку величины объединенного вектора $\eta_{1|k}$ зависят только от $\xi_0$, $v_0$, \dots, $v_{k-1}$ и $w_1$, \dots, $w_k$, которые некоррелированы с $v_k$
по условию задачи.

Аналогично, в соответствии с уравнением наблюдения \eqref{equation:filtering:in_states:kalman:observations}:
$$
	\eta_{k+1} = H_{k+1} \xi_{k+1} + w_{k+1} ,
$$
математическое ожидание
\begin{equation} \label{equation:filtering:in_states:kalman:states_and_observations:observation_expectation}
	\expectation{\eta_{k+1}}
		= H_{k+1} \expectation{\xi_{k+1}} + \expectation{w_{k+1}}
		= H_{k+1} \expectation{\xi_{k+1}}
	,
\end{equation}
поскольку величины $w_{k+1}$ имеют нулевое математическое ожидание согласно условию задачи, и центрированное наблюдение:
\begin{equation} \label{equation:filtering:in_states:kalman:states_and_observations:centered_observation}
	\eta_{k+1} - \expectation{\eta_{k+1}} = H_{k+1} \centered{\xi_{k+1}} + w_{k+1} ,
\end{equation}
тогда ковариация:
\begin{multline} \label{equation:filtering:in_states:kalman:states_and_observations:state_and_observation_covariance}
	P_{\xi_{k+1}, \eta_{k+1}}
		= \outerexpectation{\xi_{k+1}}{\eta_{k+1}} = \\
	%
	= \expectation{\centered{\xi_{k+1}} \left ( H_{k+1} \centered{\xi_{k+1}} + w_{k+1} \right )^T} = \\
	%
	= \outerexpectation{\xi_{k+1}}{\xi_{k+1}} H_{k+1}^T
		+ \expectation{\centered{\xi_{k+1}} w_{k+1}^T} = \\
	%
	= P_{\xi_{k+1}, \xi_{k+1}} H_{k+1}^T
\end{multline}
поскольку величины $\xi_{k+1}$ и $w_{k+1}$ некоррелированны по условию задачи:
$$
	\expectation{\centered{\xi_{k+1}} w_{k+1}^T} = 0 .
$$
В силу свойств ковариации:
\begin{equation} \label{equation:filtering:in_states:kalman:states_and_observations:observation_and_state_covariance}
	P_{\eta_{k+1}, \xi_{k+1}}
		= P_{\xi_{k+1}, \eta_{k+1}}^T
		= H_{k+1} P_{\xi_{k+1}, \xi_{k+1}}^T
		= H_{k+1} P_{\xi_{k+1}, \xi_{k+1}}
	.
\end{equation}

Ковариация $P_{\eta_{1|k}, \eta_{k+1}}$ в соответсвии с равенством \eqref{equation:filtering:in_states:kalman:states_and_observations:centered_observation} преобразуется к виду:
\begin{multline} \label{equation:filtering:in_states:kalman:states_and_observations:observations_and_observation_covariance}
	P_{\eta_{1|k}, \eta_{k+1}}
		= \outerexpectation{\eta_{1|k}}{\eta_{k+1}} = \\
	%
	= \expectation{\centered{\eta_{1|k}} \left ( H_{k+1} \centered{\xi_{k+1}} + w_{k+1} \right )^T} = \\
	%
	= \outerexpectation{\eta_{1|k}}{\xi_{k+1}} H_{k+1}^T
		+ \expectation{\centered{\eta_{1|k}} w_{k+1}^T} = \\
	%
	= P_{\eta_{1|k}, \xi_{k+1}} H_{k+1}^T
	,
\end{multline}
поскольку величины $\eta_1$, \dots, $\eta_k$ объединенного вектора $\eta_{1|k}$ зависят только от $\xi_0$ и $v_0$, \dots, $v_{k-1}$, а величина $w_{k+1}$ с ними
некоррелирована по условию задачи:
$$
	\expectation{\centered{\eta_{1|k}} w_{k+1}^T} = 0.
$$
В силу свойств ковариации:
\begin{equation} \label{equation:filtering:in_states:kalman:states_and_observations:observation_and_observations_covariance}
	P_{\eta_{k+1}, \eta_{1|k}}
		= P_{\eta_{1|k}, \eta_{k+1}}^T
		= H_{k+1} P_{\eta_{1|k}, \xi_{k+1}}^T
		= H_{k+1} P_{\eta_{1|k}, \xi_{k+1}}
	.
\end{equation}

\subsection{Фильтр}~\label{subsection:filtering:in_states:kalman:recursion}

Выражения для линейной несмещенной и оптимальной оценки \eqref{equation:filtering:in_states:process:optimal_estimate} и ковариаци её ошибки
\eqref{equation:filtering:in_states:process:optimal_error_covariance} в соотвествии с полученными равенствами
\eqref{equation:filtering:in_states:kalman:state_prediction:recursion},
\eqref{equation:filtering:in_states:kalman:state_prediction:error_covariance_recursion},
\eqref{equation:filtering:in_states:kalman:observation_prediction:recursion},
\eqref{equation:filtering:in_states:kalman:observation_prediction:error_covariance_recursion},
\eqref{equation:filtering:in_states:kalman:state_and_innovation:covariance}
принимают вид:
\begin{align*}
	\widehat{\xi}_{k+1, 1|k+1}
		& = \widehat{\xi}_{k+1, 1|k}
			+ P_{\widetilde{\xi}_{k+1, 1|k}, \widetilde{\xi}_{k+1, 1|k}} H_{k+1}^T P_{\widetilde{\eta}_{k+1, 1|k}, \widetilde{\eta}_{k+1, 1|k}}^{-1} \widetilde{\eta}_{k+1,1|k} , \\
	%
	P_{\widetilde{\xi}_{k+1}, \widetilde{\xi}_{k+1}}
		&= P_{\widetilde{\xi}_{k+1,1|k}, \widetilde{\xi}_{k+1,1|k}}
			- P_{\widetilde{\xi}_{k+1,1|k}, \widetilde{\xi}_{k+1,1|k}} H_{k+1}^T P_{\widetilde{\eta}_{k+1, 1|k}, \widetilde{\eta}_{k+1, 1|k}}^{-1} H_{k+1} P_{\widetilde{\xi}_{k+1,1|k}, \widetilde{\xi}_{k+1,1|k}}
\end{align*}
где
\begin{align*}
	\widehat{\xi}_{k+1, 1|k}
		& = F_k \widehat{\xi}_{k,1|k} + G_k u_k, \\
	%
	\widehat{\eta}_{k+1, 1|k}
		& = H_{k+1} \widehat{\xi}_{k+1,1|k} , \\
	%
	\widetilde{\eta}_{k+1,1|k}
		& = \eta_{k+1} - \widehat{\eta}_{k+1, 1|k} , \\
	%
	P_{\widetilde{\xi}_{k+1,1|k}, \widetilde{\xi}_{k+1,1|k}}
		& = F_k P_{\widetilde{\xi}_{k,1|k}, \widetilde{\xi}_{k,1|k}} F_k^T + P_{v_k, v_k} , \\
	%
	P_{\widetilde{\eta}_{k+1, 1|k}, \widetilde{\eta}_{k+1, 1|k}}^{-1}
		& = \left ( H_{k+1} P_{\widetilde{\xi}_{k+1,1|k}, \widetilde{\xi}_{k+1,1|k}} H_{k+1}^T + P_{w_{k+1}, w_{k+1}} \right )^{-1} .
\end{align*}

Легко видеть, что оценка $\widehat{\xi}_{k+1, 1|k+1}$ образована двумя слагаемыми: первое слагаемое $\widehat{\xi}_{k+1, 1|k}$ представляет собой прогноз состояния
для момента времени $k+1$, а во втором слагаемом величина разность $\eta_{k+1} - \widehat{\eta}_{k+1, 1|k}$ представляет собой ошибку прогноза наблюдения для момента
времени $k+1$, множитель слева от ошибки включает в себя ковариацию $P_{\widetilde{\eta}_{k+1, 1|k}, \widetilde{\eta}_{k+1, 1|k}}$, определяющую "точность"{}
прогноза наблюдения, и ковариацию $P_{\widetilde{\xi}_{k+1, 1|k}, \widetilde{\xi}_{k+1, 1|k}}$, определяющую "точность"{} прогноза состояния.