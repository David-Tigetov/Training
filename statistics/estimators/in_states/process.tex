\section{Структура фильтров} \label{section:filtering:in_states:process}

\subsection{Постановка задачи}

Пусть $\xi_k$ --- случайный процесс в дискретном времени $k = 0, 1, 2, \dots$, где все случайные величины $\xi_k \in L_2^m$ и для начального состояния $\xi_0$ заданы
математическое ожидание $\expectation{\xi_0}$ и ковариация $P_{\xi_0, \xi_0}$. Представим, что процесс $\xi_k$ не является наблюдаемым, но наблюдаемым является процесс
$\eta_k$ при $k=1, 2, \dots$, который некоторым образом связан с процессом $\xi_k$.

Требуется при всех $k$ располагать методом построения оценки $\widehat{\xi}_{k,1|k}$ величины $\xi_k$ по имеющимся на момент $k$ наблюдениям $\eta_1$, \dots, $\eta_k$,
которая:
\begin{enumerate}
	\item является несмещенной оценкой: $\expectation{\widehat{\xi}_{k,1|k}} = \expectation{\xi_k}$,
	\item является оптимальной --- наименее уклоняется от величины $\xi_k$:
$$
	\Wnorm{\xi_k - \widehat{\xi}_{k,1|k}}{W} = \min_{\xi \sim \Sigma (\eta)} \Wnorm{\xi_k - \xi}{W} ,
$$
	\item является линейной функцией наблюдений $\eta_1$, \dots, $\eta_k$.
\end{enumerate}

\subsection{Обозначения}

В дальнейшем изложении введены обозначения $\eta_{1|k}$ и $\eta_{1|k+1}$ для векторов, объединяющих величины $\eta_1$, \dots, $\eta_{k}$ и
$\eta_1$, \dots, $\eta_{k}$, $\eta_{k+1}$:
$$
	\eta_{1|k}
	=
	\begin{pmatrix}
		\eta_1 \\
		\vdots \\
		\eta_{k}
	\end{pmatrix}
	, \;
	\eta_{1|k+1}
	=
	\begin{pmatrix}
		\eta_1 \\
		\vdots \\
		\eta_{k} \\
		\eta_{k+1}
	\end{pmatrix}
	.
$$

Для ковариационных матриц случайных величин $\zeta$ и $\psi$ принято краткое обозначение $P_{\zeta, \psi}$:
$$
	P_{\zeta, \psi} = \outerexpectation{\zeta}{\psi} .
$$

\subsection{Блочная форма}

В соответствии с равенством \eqref{equation:filtering:in_states:variable:optimal_estimate} несмещенная оптимальная линейная оценка $\widehat{\xi}_{k,1|k}$ состояния $\xi_k$,
полученная по наблюдениям $\eta_1$, \dots, $\eta_k$ должна иметь вид:
\begin{equation} \label{equation:filtering:in_states:process:block:current_state_estimate}
	\widehat{\xi}_{k,1|k} = \expectation{\xi_k} + P_{\xi_k, \eta_{1|k}} P_{\eta_{1|k}, \eta_{1|k}}^{-1} \centered{\eta_{1|k}}
	.
\end{equation}
Ошибка оценивания
$$
	\widetilde{\xi}_{k,1|k} = \xi_k - \widehat{\xi}_{k,1|k}
$$
имеет ковариационную матрицу, устанавливаемую равенством \eqref{equation:filtering:in_states:variable:optimal_error_covariance}:
\begin{equation} \label{equation:filtering:in_states:process:block:current_state_estimate_error_covariance}
	P_{\widetilde{\xi}_{k,1|k}, \widetilde{\xi}_{k,1|k}} = P_{\xi_k, \xi_k} - P_{\xi_k, \eta_{1|k}} P_{\eta_{1|k}, \eta_{1|k}} P_{\eta_{1|k}, \xi_k}
	.
\end{equation}
При получении наблюдения $\eta_{k+1}$ оценка $\widehat{\xi}_{k+1, 1|k+1}$ состояния $\xi_{k+1}$ будет иметь аналогичный вид:
$$
	\widehat{\xi}_{k+1, 1|k+1} = \expectation{\xi_{k+1}} + P_{\xi_{k+1}, \eta_{1|k+1}} P_{\eta_{k+1}, \eta_{k+1}}^{-1} \centered{\eta_{1|k+1}}
	,
$$
с соответствующей ошибкой
$$
	\widetilde{\xi}_{k+1, 1|k+1} = \xi_{k+1} - \widehat{\xi}_{k+1, 1|k+1}
$$
и ковариационной матрицей ошибки:
$$
	P_{\widetilde{\xi}_{k+1,1|k+1}, \widetilde{\xi}_{k+1,1|k+1}} = P_{\xi_{k+1}, \xi_{k+1}} - P_{\xi_{k+1}, \eta_{1|k+1}} P_{\eta_{1|k+1}, \eta_{1|k+1}} P_{\eta_{1|k+1}, \xi_{k+1}}
	.
$$

Легко видеть, что в такой форме при поступлении каждого следующего наблюдения $\eta_{k+1}$ происходит большое количество вычислений. Для сокращения количества
вычислений требуется рекуррентная форма, в которой оценка $\widehat{\xi}_{k+1, 1|k+1}$ рекуррентно выражается через оценку $\widehat{\xi}_k$, а ковариационная матрица
ошибки $P_{\widetilde{\xi}_{k+1,1|k+1}, \widetilde{\xi}_{k+1,1|k+1}}$ --- через $P_{\widetilde{\xi}_{k,1|k}, \widetilde{\xi}_{k,1|k}}$.

Рекуррентные формы для оценки и ковариационной матрицы ошибки получаются после выделения последнего наблюдения $\eta_{k+1}$:
\begin{multline} \label{equation:filtering:in_states:process:block:state_estimate}
	\widehat{\xi}_{k+1, 1|k+1}
		= \expectation{\xi_{k+1}} + \\
	%
	+ \begin{pmatrix} P_{\xi_{k+1}, \eta_{1|k}} & P_{\xi_{k+1}, \eta_{k+1}} \end{pmatrix}
		\begin{pmatrix}
			P_{\eta_{1|k}, \eta_{1|k}} & P_{\eta_{1|k}, \eta_{k+1}} \\
			P_{\eta_{k+1}, \eta_{1|k}} & P_{\eta_{k+1}, \eta_{k+1}}
		\end{pmatrix}^{-1}
		\begin{pmatrix}
			\eta_{1|k} - \expectation{\eta_{1|k}} \\
			\eta_{k+1} - \expectation{\eta_{k+1}}
		\end{pmatrix}
\end{multline}
и в ковариационной матрице ошибки:
\begin{multline} \label{equation:filtering:in_states:process:block:error_covariance}
	P_{\widetilde{\xi}_{k+1,1|k+1}, \widetilde{\xi}_{k+1,1|k+1}}
		= P_{\xi_{k+1}, \xi_{k+1}} - \\
	%
	- \begin{pmatrix} P_{\xi_{k+1}, \eta_{1|k}} & P_{\xi_{k+1}, \eta_{k+1}} \end{pmatrix}
		\begin{pmatrix}
			P_{\eta_{1|k}, \eta_{1|k}} & P_{\eta_{1|k}, \eta_{k+1}} \\
			P_{\eta_{k+1}, \eta_{1|k}} & P_{\eta_{k+1}, \eta_{k+1}}
		\end{pmatrix}^{-1}
		\begin{pmatrix}
			P_{\eta_{1|k}, \xi_{k+1}} \\
			P_{\eta_{k+1}, \xi_{k+1}}
		\end{pmatrix}
	.
\end{multline}
Полученные блочные формы могут быть последовательно преобразованы к рекуррентным формам.

\subsection{Обратная матрица}

Первый шаг преобразования блочной формы связан с обращением матрицы
$$
	\begin{pmatrix}
		P_{\eta_{1|k}, \eta_{1|k}} & P_{\eta_{1|k}, \eta_{k+1}} \\
		P_{\eta_{k+1}, \eta_{1|k}} & P_{\eta_{k+1}, \eta_{k+1}}
	\end{pmatrix}^{-1}
	=
	\begin{pmatrix}
		A & B \\
		C & D
	\end{pmatrix}
$$
с помощью леммы об обратной матрице (раздел \ref{section:appendix:inversions}), которую нужно взять в форме:
\begin{equation} \label{equation:filtering:in_states:process:inversion_lemma}
	\begin{pmatrix}
		A & B \\
		C & D
	\end{pmatrix}^{-1}
	=
	\begin{pmatrix}
		A^{-1} + A^{-1} B \left ( D - C A^{-1} B \right )^{-1} C A^{-1} & - A^{-1} B \left ( D - C A^{-1} B \right )^{-1} \\
		- \left ( D - C A^{-1} B \right )^{-1} C A^{-1}                             & \left ( D - C A^{-1} B \right )^{-1}
	\end{pmatrix}
\end{equation}
Легко видеть, что во всех элементах матрицы в правой части равенства присутствует выражение $D - C A^{-1} B$, которое соответствует матрице:
$$
	D - C A^{-1} B
	\rightarrow
	P_{\eta_{k+1}, \eta_{k+1}} - P_{\eta_{k+1}, \eta_{1|k}} P_{\eta_{1|k},\eta_{1|k}}^{-1} P_{\eta_{1|k}, \eta_{k+1}}
	.
$$
В соответствии с равенством \eqref{equation:filtering:in_states:variable:optimal_error_covariance} выражение справа является ковариационной матрицей ошибки
$\widetilde{\eta}_{k+1,1|k}$ оптимальной оценки $\widehat{\eta}_{k+1,1|k}$ наблюдения $\eta_{k+1}$, построенной по наблюдениям $\eta_1$, \dots, $\eta_k$:
\begin{gather}
	\widehat{\eta}_{k+1,1|k} = \expectation{\eta_{k+1}} + P_{\eta_{k+1},\eta_{1|k}}P_{\eta_{1|k},\eta_{1|k}}^{-1} ( \eta_{1|k} - \expectation{\eta_{1|k}} ) ,
		\label{equation:filtering:in_states:process:observation_prediction} \\
	%
	\widetilde{\eta}_{k+1,1|k} = \eta_{k+1} - \widehat{\eta}_{k+1,1|k} ,
		\notag \\
	%
	P_{\widetilde{\eta}_{k+1, 1|k}, \widetilde{\eta}_{k+1, 1|k}} = P_{\eta_{k+1}, \eta_{k+1}} - P_{\eta_{k+1}, \eta_{1|k}} P_{\eta_{1|k},\eta_{1|k}}^{-1} P_{\eta_{1|k}, \eta_{k+1}} .
		\notag
\end{gather}
Используя полученное выражение, по лемме об обратной матрице в форме \eqref{equation:filtering:in_states:process:inversion_lemma} получим:
\begin{multline} \label{equation:filtering:in_states:process:observations_inverse_covariance}
	\begin{pmatrix}
		P_{\eta_{1|k}, \eta_{1|k}} & P_{\eta_{1|k}, \eta_{k+1}} \\
		P_{\eta_{k+1}, \eta_{1|k}} & P_{\eta_{k+1}, \eta_{k+1}}
	\end{pmatrix}^{-1} = \\
	%
	=
	\begin{pmatrix}
		P_{\eta_{1|k},\eta_{1|k}}^{-1} + P_{\eta_{1|k},\eta_{1|k}}^{-1} P_{\eta_{1|k}, \eta_{k+1}} P_{\widetilde{\eta}_{k+1, 1|k}, \widetilde{\eta}_{k+1, 1|k}}^{-1} P_{\eta_{k+1}, \eta_{1|k}} P_{\eta_{1|k},\eta_{1|k}}^{-1} & - P_{\eta_{1|k},\eta_{1|k}}^{-1} P_{\eta_{1|k}, \eta_{k+1}} P_{\widetilde{\eta}_{k+1, 1|k}, \widetilde{\eta}_{k+1, 1|k}}^{-1} \\
		- P_{\widetilde{\eta}_{k+1, 1|k}, \widetilde{\eta}_{k+1, 1|k}}^{-1} P_{\eta_{k+1}, \eta_{1|k}} P_{\eta_{1|k},\eta_{1|k}}^{-1}                                                                                          & P_{\widetilde{\eta}_{k+1, 1|k}, \widetilde{\eta}_{k+1, 1|k}}^{-1}
	\end{pmatrix}
\end{multline}

\subsection{Рекуррентная форма оценки}

Вычислим произведение
$$
	\begin{pmatrix}
		T_1 \\
		T_2
	\end{pmatrix}
	=
	\begin{pmatrix}
		P_{\eta_{1|k}, \eta_{1|k}} & P_{\eta_{1|k}, \eta_{k+1}} \\
		P_{\eta_{k+1}, \eta_{1|k}} & P_{\eta_{k+1}, \eta_{k+1}}
	\end{pmatrix}^{-1}
	\begin{pmatrix}
		\eta_{1|k} - \expectation{\eta_{1|k}} \\
		\eta_{k+1} - \expectation{\eta_{k+1}}
	\end{pmatrix}
$$
с использованием выражения \eqref{equation:filtering:in_states:process:observations_inverse_covariance} для обратной матрицы. Матрица $T_1$ имеет выражение:
\begin{multline*}
	T_1 = P_{\eta_{1|k},\eta_{1|k}}^{-1} \centered{\eta_{1|k}} + \\
		+ P_{\eta_{1|k},\eta_{1|k}}^{-1} P_{\eta_{1|k}, \eta_{k+1}} P_{\widetilde{\eta}_{k+1, 1|k}, \widetilde{\eta}_{k+1, 1|k}}^{-1} P_{\eta_{k+1}, \eta_{1|k}} P_{\eta_{1|k},\eta_{1|k}}^{-1} \centered{\eta_{1|k}} - \\
		\shoveright{ - P_{\eta_{1|k},\eta_{1|k}}^{-1} P_{\eta_{1|k}, \eta_{k+1}} P_{\widetilde{\eta}_{k+1, 1|k}, \widetilde{\eta}_{k+1, 1|k}}^{-1} \centered{\eta_{k+1}} = } \\
	%
	\shoveleft{ = P_{\eta_{1|k},\eta_{1|k}}^{-1} \centered{\eta_{1|k}} + } \\
		+ P_{\eta_{1|k},\eta_{1|k}}^{-1} P_{\eta_{1|k}, \eta_{k+1}} P_{\widetilde{\eta}_{k+1, 1|k}, \widetilde{\eta}_{k+1, 1|k}}^{-1} P_{\eta_{k+1}, \eta_{1|k}} P_{\eta_{1|k},\eta_{1|k}}^{-1} \centered{\eta_{1|k}} - \\
		\shoveright{ - P_{\eta_{1|k},\eta_{1|k}}^{-1} P_{\eta_{1|k}, \eta_{k+1}} P_{\widetilde{\eta}_{k+1, 1|k}, \widetilde{\eta}_{k+1, 1|k}}^{-1} \eta_{k+1} + P_{\eta_{1|k},\eta_{1|k}}^{-1} P_{\eta_{1|k}, \eta_{k+1}} P_{\widetilde{\eta}_{k+1, 1|k}, \widetilde{\eta}_{k+1, 1|k}}^{-1} \expectation{\eta_{k+1}} = } \\
	%
	\shoveleft{	= P_{\eta_{1|k},\eta_{1|k}}^{-1} \centered{\eta_{1|k}} + } \\
		+ P_{\eta_{1|k},\eta_{1|k}}^{-1} P_{\eta_{1|k}, \eta_{k+1}} P_{\widetilde{\eta}_{k+1, 1|k}, \widetilde{\eta}_{k+1, 1|k}}^{-1} \left ( \expectation{\eta_{k+1}} + P_{\eta_{k+1}, \eta_{1|k}} P_{\eta_{1|k},\eta_{1|k}}^{-1} \centered{\eta_{1|k}} \right ) - \\
		- P_{\eta_{1|k},\eta_{1|k}}^{-1} P_{\eta_{1|k}, \eta_{k+1}} P_{\widetilde{\eta}_{k+1, 1|k}, \widetilde{\eta}_{k+1, 1|k}}^{-1} \eta_{k+1} =
\end{multline*}
(в соответствии с равенством \eqref{equation:filtering:in_states:process:observation_prediction} в центральном слагаемом в скобках стоит выражение
оценки $\widehat{\eta}_{k+1, 1|k}$)
\begin{multline*}
	= P_{\eta_{1|k},\eta_{1|k}}^{-1} \centered{\eta_{1|k}}
		+ P_{\eta_{1|k},\eta_{1|k}}^{-1} P_{\eta_{1|k}, \eta_{k+1}} P_{\widetilde{\eta}_{k+1, 1|k}, \widetilde{\eta}_{k+1, 1|k}}^{-1} \widehat{\eta}_{k+1, 1|k} - \\
		\shoveright{ - P_{\eta_{1|k},\eta_{1|k}}^{-1} P_{\eta_{1|k}, \eta_{k+1}} P_{\widetilde{\eta}_{k+1, 1|k}, \widetilde{\eta}_{k+1, 1|k}}^{-1} \eta_{k+1} = } \\
	%
	= P_{\eta_{1|k},\eta_{1|k}}^{-1} \centered{\eta_{1|k}}
		- P_{\eta_{1|k},\eta_{1|k}}^{-1} P_{\eta_{1|k}, \eta_{k+1}} P_{\widetilde{\eta}_{k+1, 1|k}, \widetilde{\eta}_{k+1, 1|k}}^{-1} \left ( \eta_{k+1} - \widehat{\eta}_{k+1, 1|k} \right )
	,
\end{multline*}
и матрица $T_2$:
$$
	\shoveright{
		T_2 =
			- P_{\widetilde{\eta}_{k+1, 1|k}, \widetilde{\eta}_{k+1, 1|k}}^{-1} P_{\eta_{k+1}, \eta_{1|k}} P_{\eta_{1|k},\eta_{1|k}}^{-1} \centered{\eta_{1|k}}
			+ P_{\widetilde{\eta}_{k+1, 1|k}, \widetilde{\eta}_{k+1, 1|k}}^{-1} \centered{\eta_{k+1}}
			=
	}
$$
(преобразуем первое слагаемое в соответствии с равенством \eqref{equation:filtering:in_states:process:observation_prediction}:
$\widehat{\eta}_{k+1,1|k} - \expectation{\eta_{k+1}} = P_{\eta_{k+1},\eta_{1|k}}P_{\eta_{1|k},\eta_{1|k}}^{-1} ( \eta_{1|k} - \expectation{\eta_{1|k}} )$)
\begin{multline*}
	%
	= - P_{\widetilde{\eta}_{k+1, 1|k}, \widetilde{\eta}_{k+1, 1|k}}^{-1} \left ( \widehat{\eta}_{k+1,1|k} - \expectation{\eta_{k+1}} \right )
		+ P_{\widetilde{\eta}_{k+1, 1|k}, \widetilde{\eta}_{k+1, 1|k}}^{-1} \centered{\eta_{k+1}} = \\
	%
	= P_{\widetilde{\eta}_{k+1, 1|k}, \widetilde{\eta}_{k+1, 1|k}}^{-1}
		\left ( \eta_{k+1} - \expectation{\eta_{k+1}} - \widehat{\eta}_{k+1, 1|k} + \expectation{\eta_{k+1}} \right ) = \\
	%
	= P_{\widetilde{\eta}_{k+1, 1|k}, \widetilde{\eta}_{k+1, 1|k}}^{-1} \left ( \eta_{k+1} - \widehat{\eta}_{k+1, 1|k} \right ) .
\end{multline*}

Таким образом, в результате умножения получим:
\begin{multline*}
	\begin{pmatrix}
		P_{\eta_{1|k}, \eta_{1|k}} & P_{\eta_{1|k}, \eta_{k+1}} \\
		P_{\eta_{k+1}, \eta_{1|k}} & P_{\eta_{k+1}, \eta_{k+1}}
	\end{pmatrix}^{-1}
	\begin{pmatrix}
		\eta_{1|k} - \expectation{\eta_{1|k}} \\
		\eta_{k+1} - \expectation{\eta_{k+1}}
	\end{pmatrix}
	= \\
	%
	=
	\begin{pmatrix}
		P_{\eta_{1|k},\eta_{1|k}}^{-1} \centered{\eta_{1|k}} - P_{\eta_{1|k},\eta_{1|k}}^{-1} P_{\eta_{1|k}, \eta_{k+1}} P_{\widetilde{\eta}_{k+1, 1|k}, \widetilde{\eta}_{k+1, 1|k}}^{-1} \left ( \eta_{k+1} - \widehat{\eta}_{k+1, 1|k} \right ) \\
		P_{\widetilde{\eta}_{k+1, 1|k}, \widetilde{\eta}_{k+1, 1|k}}^{-1} \left ( \eta_{k+1} - \widehat{\eta}_{k+1, 1|k} \right )
	\end{pmatrix}
\end{multline*}
Теперь исходное выражение \eqref{equation:filtering:in_states:process:block:state_estimate} для оценки $\widehat{\xi}_{k+1,1|k+1}$ можно преобразовать:
\begin{multline*} 
	\widehat{\xi}_{k+1, 1|k+1}
		= \expectation{\xi_{k+1}} + \\
		\shoveright{
			+ \begin{pmatrix} P_{\xi_{k+1}, \eta_{1|k}} & P_{\xi_{k+1}, \eta_{k+1}} \end{pmatrix}
			\begin{pmatrix}
				P_{\eta_{1|k}, \eta_{1|k}} & P_{\eta_{1|k}, \eta_{k+1}} \\
				P_{\eta_{1|k}, \eta_{1|k}} & P_{\eta_{k+1}, \eta_{k+1}}
			\end{pmatrix}^{-1}
			\begin{pmatrix}
				\eta_{1|k} - \expectation{\eta_{1|k}} \\
				\eta_{k+1} - \expectation{\eta_{k+1}}
			\end{pmatrix}
			=
		} \\
	%
	\shoveleft{ = \expectation{\xi_{k+1}} + } \\
		+ P_{\xi_{k+1}, \eta_{1|k}} \left ( P_{\eta_{1|k},\eta_{1|k}}^{-1} \centered{\eta_{1|k}} - P_{\eta_{1|k},\eta_{1|k}}^{-1} P_{\eta_{1|k}, \eta_{k+1}} P_{\widetilde{\eta}_{k+1, 1|k}, \widetilde{\eta}_{k+1, 1|k}}^{-1} \left ( \eta_{k+1} - \widehat{\eta}_{k+1, 1|k} \right ) \right ) + \\
		\shoveright { + P_{\xi_{k+1}, \eta_{k+1}} P_{\widetilde{\eta}_{k+1, 1|k}, \widetilde{\eta}_{k+1, 1|k}}^{-1} \left ( \eta_{k+1} - \widehat{\eta}_{k+1, 1|k} \right ) = } \\
	%
	\shoveleft{ = \expectation{\xi_{k+1}} + P_{\xi_{k+1}, \eta_{1|k}} P_{\eta_{1|k},\eta_{1|k}}^{-1} \centered{\eta_{1|k}} + } \\
		+ \left ( P_{\xi_{k+1}, \eta_{k+1}} - P_{\xi_{k+1}, \eta_{1|k}} P_{\eta_{1|k},\eta_{1|k}}^{-1} P_{\eta_{1|k}, \eta_{k+1}} \right ) P_{\widetilde{\eta}_{k+1, 1|k}, \widetilde{\eta}_{k+1, 1|k}}^{-1} \left ( \eta_{k+1} - \widehat{\eta}_{k+1, 1|k} \right )
	.
\end{multline*}
Первое и второе слагаемые в силу равенства \eqref{equation:filtering:in_states:variable:optimal_estimate} образуют оптимальную оценку $\widehat{\xi}_{k+1,1|k}$ состояния
$\xi_{k+1}$, построенную по наблюдениям $\eta_1$, \dots, $\eta_k$:
\begin{equation} \label{equation:filtering:in_states:process:state_prediction}
	\widehat{\xi}_{k+1,1|k} = \expectation{\xi_{k+1}} + P_{\xi_{k+1}, \eta_{1|k}} P_{\eta_{1|k},\eta_{1|k}}^{-1} \centered{\eta_{1|k}} ,
\end{equation}
В третьем слагаемом преобразуем первый множитель, раскрывая ковариации:
\begin{multline*}
	P_{\xi_{k+1}, \eta_{k+1}} - P_{\xi_{k+1}, \eta_{1|k}} P_{\eta_{1|k},\eta_{1|k}}^{-1} P_{\eta_{1|k}, \eta_{k+1}} = \\
	%
	\shoveleft{
		= \expectation{\centered{\xi_{k+1}} \centered{\eta_{k+1}}^T}
	} \\
	\shoveright{
		- \expectation{\centered{\xi_{k+1}} \centered{\eta_{1|k}}^T} P_{\eta_{1|k},\eta_{1|k}}^{-1} P_{\eta_{1|k}, \eta_{k+1}} =
	} \\
	%
	\shoveleft{
		= \expectation{\centered{\xi_{k+1}} \centered{\eta_{k+1}}^T}
	} \\
	\shoveright{
		- \expectation{\centered{\xi_{k+1}} \centered{\eta_{1|k}}^T P_{\eta_{1|k},\eta_{1|k}}^{-1} P_{\eta_{1|k}, \eta_{k+1}} }  =
	} \\
	%
	\shoveleft{
		= \expectation{\centered{\xi_{k+1}} \centered{\eta_{k+1}}^T}
	} \\
	\shoveright{
		- \expectation{\centered{\xi_{k+1}} \left ( P_{\eta_{k+1},\eta_{1|k}} P_{\eta_{1|k},\eta_{1|k}}^{-1} \centered{\eta_{1|k}} \right )^T }  =
	} \\
	%
	= \expectation{\centered{\xi_{k+1}} \left ( \centered{\eta_{k+1}} - P_{\eta_{k+1},\eta_{1|k}} P_{\eta_{1|k},\eta_{1|k}}^{-1} \centered{\eta_{1|k}} \right )^T} = \\
	%
	= \expectation{\centered{\xi_{k+1}} \left ( \eta_{k+1} - \underline{\left ( \expectation{\eta_{k+1}} + P_{\eta_{k+1},\eta_{1|k}} P_{\eta_{1|k},\eta_{1|k}}^{-1} \centered{\eta_{1|k}} \right )} \right )^T} = \\
\end{multline*}
(в силу равенства \eqref{equation:filtering:in_states:process:observation_prediction} подчеркнутое выражение образует оценку $\widehat{\eta}_{k+1,1|k}$)
\begin{multline} \label{equation:filtering:in_states:process:next_state_and_innovation_covariance}
	%
	= \expectation{\centered{\xi_{k+1}} \left ( \eta_{k+1} - \widehat{\eta}_{k+1,1|k} \right )^T}
		= \expectation{\centered{\xi_{k+1}} \widetilde{\eta}_{k+1,1|k}} \\
	%
	= \expectation{\centered{\xi_{k+1}} \centered{\widetilde{\eta}_{k+1,1|k}}^T}
		= P_{\xi_{k+1}, \widetilde{\eta}_{k+1,1|k}}
	,
\end{multline}
поскольку в силу несмещенности оценки $\widehat{\eta}_{k+1,1|k}$ значение $\expectation{\widetilde{\eta}_{k+1,1|k}} = 0$.

Возвращаясь к выражению для оценки $\widehat{\xi}_{k+1,1|k+1}$, получим:
$$
	\widehat{\xi}_{k+1}
		= \widehat{\xi}_{k+1,1|k}
		+ P_{\xi_{k+1}, \widetilde{\eta}_{k+1,1|k}} P_{\widetilde{\eta}_{k+1, 1|k}, \widetilde{\eta}_{k+1, 1|k}}^{-1} \left ( \eta_{k+1} - \widehat{\eta}_{k+1, 1|k} \right )
	.
$$

\subsection{Рекуррентная форма ковариации ошибки}

В выражении \eqref{equation:filtering:in_states:process:block:error_covariance} для ковариационной матрицы ошибки
$P_{\widetilde{\xi}_{k+1,1|k+1}, \widetilde{\xi}_{k+1,1|k+1}}$ вычислим "квадратичную форму"{}, используя выражение для обратной матрицы
\eqref{equation:filtering:in_states:process:observations_inverse_covariance}:
\begin{multline*}
	P_{\widetilde{\xi}_{k+1,1|k+1}, \widetilde{\xi}_{k+1,1|k+1}}
		= P_{\xi_{k+1}, \xi_{k+1}} - \\
		\shoveright{
			-\begin{pmatrix} P_{\xi_{k+1}, \eta_{1|k}} & P_{\xi_{k+1}, \eta_{k+1}} \end{pmatrix}
			\begin{pmatrix}
				P_{\eta_{1|k}, \eta_{1|k}} & P_{\eta_{1|k}, \eta_{k+1}} \\
				P_{\eta_{1|k}, \eta_{1|k}} & P_{\eta_{k+1}, \eta_{k+1}}
			\end{pmatrix}^{-1}
			\begin{pmatrix}
				P_{\eta_{1|k}, \xi_{k+1}} \\
				P_{\eta_{k+1}, \xi_{k+1}}
			\end{pmatrix} =
		} \\
	%
	\shoveleft{ = P_{\xi_{k+1}, \xi_{k+1}} - \begin{pmatrix} P_{\xi_{k+1}, \eta_{1|k}} & P_{\xi_{k+1}, \eta_{k+1}} \end{pmatrix} } \\
	\begin{pmatrix}
		P_{\eta_{1|k},\eta_{1|k}}^{-1} + P_{\eta_{1|k},\eta_{1|k}}^{-1} P_{\eta_{1|k}, \eta_{k+1}} P_{\widetilde{\eta}_{k+1, 1|k}, \widetilde{\eta}_{k+1, 1|k}}^{-1} P_{\eta_{k+1}, \eta_{1|k}} P_{\eta_{1|k},\eta_{1|k}}^{-1} & - P_{\eta_{1|k},\eta_{1|k}}^{-1} P_{\eta_{1|k}, \eta_{k+1}} P_{\widetilde{\eta}_{k+1, 1|k}, \widetilde{\eta}_{k+1, 1|k}}^{-1} \\
		- P_{\widetilde{\eta}_{k+1, 1|k}, \widetilde{\eta}_{k+1, 1|k}}^{-1} P_{\eta_{k+1}, \eta_{1|k}} P_{\eta_{1|k},\eta_{1|k}}^{-1}                                                                                          & P_{\widetilde{\eta}_{k+1, 1|k}, \widetilde{\eta}_{k+1, 1|k}}^{-1}
	\end{pmatrix} \\
	\shoveright{ \begin{pmatrix} P_{\eta_{1|k}, \xi_{k+1}} \\ P_{\eta_{k+1}, \xi_{k+1}} \end{pmatrix} = } \\
	%
	\shoveleft{ = P_{\xi_{k+1}, \xi_{k+1}} - P_{\xi_{k+1}, \eta_{1|k}} P_{\eta_{1|k},\eta_{1|k}}^{-1} P_{\eta_{1|k}, \xi_{k+1}} - } \\
	- P_{\xi_{k+1}, \eta_{1|k}} P_{\eta_{1|k},\eta_{1|k}}^{-1} P_{\eta_{1|k}, \eta_{k+1}} P_{\widetilde{\eta}_{k+1, 1|k}, \widetilde{\eta}_{k+1, 1|k}}^{-1} P_{\eta_{k+1}, \eta_{1|k}} P_{\eta_{1|k},\eta_{1|k}}^{-1} P_{\eta_{1|k}, \xi_{k+1}} + \\
	+ P_{\xi_{k+1}, \eta_{1|k}} P_{\eta_{1|k},\eta_{1|k}}^{-1} P_{\eta_{1|k}, \eta_{k+1}} P_{\widetilde{\eta}_{k+1, 1|k}, \widetilde{\eta}_{k+1, 1|k}}^{-1} P_{\eta_{k+1}, \xi_{k+1}} + \\
	+ P_{\xi_{k+1}, \eta_{k+1}} P_{\widetilde{\eta}_{k+1, 1|k}, \widetilde{\eta}_{k+1, 1|k}}^{-1} P_{\eta_{k+1}, \eta_{1|k}} P_{\eta_{1|k},\eta_{1|k}}^{-1} P_{\eta_{1|k}, \xi_{k+1}} - \\
	- P_{\xi_{k+1}, \eta_{k+1}} P_{\widetilde{\eta}_{k+1, 1|k}, \widetilde{\eta}_{k+1, 1|k}}^{-1} P_{\eta_{k+1}, \xi_{k+1}} =
\end{multline*}
(в третьем и четвертом слагаемых выносим общий множитель, и в пятном и шестом слагаемых также выносим общий множитель)
\begin{multline*}
	= P_{\xi_{k+1}, \xi_{k+1}} - P_{\xi_{k+1}, \eta_{1|k}} P_{\eta_{1|k},\eta_{1|k}}^{-1} P_{\eta_{1|k}, \xi_{k+1}} + \\
	+ P_{\xi_{k+1}, \eta_{1|k}} P_{\eta_{1|k},\eta_{1|k}}^{-1} P_{\eta_{1|k}, \eta_{k+1}} P_{\widetilde{\eta}_{k+1, 1|k}, \widetilde{\eta}_{k+1, 1|k}}^{-1} \left ( P_{\eta_{k+1}, \xi_{k+1}} - P_{\eta_{k+1}, \eta_{1|k}} P_{\eta_{1|k},\eta_{1|k}}^{-1} P_{\eta_{1|k}, \xi_{k+1}} \right ) - \\
	- P_{\xi_{k+1}, \eta_{k+1}} P_{\widetilde{\eta}_{k+1, 1|k}, \widetilde{\eta}_{k+1, 1|k}}^{-1} \left ( P_{\eta_{k+1}, \xi_{k+1}} - P_{\eta_{k+1}, \eta_{1|k}} P_{\eta_{1|k},\eta_{1|k}}^{-1} P_{\eta_{1|k}, \xi_{k+1}} \right )
	.
\end{multline*}

В соответствии с равенством \eqref{equation:filtering:in_states:variable:optimal_error_covariance} первое и второе слагаемое образуют ковариацию ошибки
$\widetilde{\xi}_{k+1,1|k}$ оптимальной оценки $\widehat{\xi}_{k+1,1|k}$, определяемой равенством \eqref{equation:filtering:in_states:process:state_prediction}:
\begin{gather*}
	\widetilde{\xi}_{k+1,1|k} = \xi_{k+1} - \widehat{\xi}_{k+1,1|k} , \\
	P_{\widetilde{\xi}_{k+1,1|k}, \widetilde{\xi}_{k+1,1|k}} = P_{\xi_{k+1}, \xi_{k+1}} - P_{\xi_{k+1}, \eta_{1|k}} P_{\eta_{1|k},\eta_{1|k}}^{-1} P_{\eta_{1|k}, \xi_{k+1}}
	.
\end{gather*}
В третьем и четвертом слагаемых выражение в скобках преобразуем, раскрывая ковариацию:
\begin{multline*}
	P_{\eta_{k+1}, \xi_{k+1}} - P_{\eta_{k+1}, \eta_{1|k}} P_{\eta_{1|k},\eta_{1|k}}^{-1} P_{\eta_{1|k}, \xi_{k+1}} = \\
	%
	\shoveleft{
		= \expectation{\centered{\eta_{k+1}} \centered{\xi_{k+1}}^T}
	} \\
	\shoveright{
		- P_{\eta_{k+1}, \eta_{1|k}} P_{\eta_{1|k},\eta_{1|k}}^{-1} \expectation{\centered{\eta_{1|k}} \centered{\xi_{k+1}}^T} =
	} \\
	%
	\shoveleft{
		= \expectation{\centered{\eta_{k+1}} \centered{\xi_{k+1}}^T}
	} \\
	\shoveright{
		- \expectation{P_{\eta_{k+1}, \eta_{1|k}} P_{\eta_{1|k},\eta_{1|k}}^{-1} \centered{\eta_{1|k}} \centered{\xi_{k+1}}^T} =
	} \\
	%
	= \expectation{\left ( \centered{\eta_{k+1}} - P_{\eta_{k+1}, \eta_{1|k}} P_{\eta_{1|k},\eta_{1|k}}^{-1} \centered{\eta_{1|k}} \right ) \centered{\xi_{k+1}}^T} = \\
	%
	= \expectation{\left ( \eta_{k+1} - \underline{\left ( \expectation{\eta_{k+1}} + P_{\eta_{k+1}, \eta_{1|k}} P_{\eta_{1|k},\eta_{1|k}}^{-1} \centered{\eta_{1|k}} \right )} \right ) \centered{\xi_{k+1}}^T} = \\
\end{multline*}
(в соответствии с равенством \eqref{equation:filtering:in_states:process:observation_prediction} выражение в скобках является оценкой $\widehat{\eta}_{k+1,1|k}$)
\begin{multline*}
	= \expectation{\left ( \eta_{k+1} - \widehat{\eta}_{k+1,1|k} \right ) \centered{\xi_{k+1}}^T}
		= \expectation{\widetilde{\eta}_{k+1,1|k} \centered{\xi_{k+1}}^T} = \\
	%
	= \expectation{\centered{\widetilde{\eta}_{k+1,1|k}} \centered{\xi_{k+1}}^T}
		= P_{\widetilde{\eta}_{k+1,1|k}, \xi_{k+1}}
\end{multline*}
Возвращаясь к выражению ковариции ошибки получим:
\begin{multline*}
	P_{\widetilde{\xi}_{k+1,1|k+1}, \widetilde{\xi}_{k+1,1|k+1}} = \\
	%
	\shoveleft{= P_{\widetilde{\xi}_{k+1,1|k},\widetilde{\xi}_{k+1,1|k}} + P_{\xi_{k+1}, \eta_{1|k}} P_{\eta_{1|k},\eta_{1|k}}^{-1} P_{\eta_{1|k}, \eta_{k+1}} P_{\widetilde{\eta}_{k+1, 1|k}, \widetilde{\eta}_{k+1, 1|k}}^{-1} P_{\widetilde{\eta}_{k+1,1|k}, \xi_{k+1}} - }\\
	\shoveright{- P_{\xi_{k+1}, \eta_{k+1}} P_{\widetilde{\eta}_{k+1, 1|k}, \widetilde{\eta}_{k+1, 1|k}}^{-1} P_{\widetilde{\eta}_{k+1,1|k}, \xi_{k+1}}} = \\
	%
	= P_{\widetilde{\xi}_{k+1,1|k}}
		+ \left ( P_{\xi_{k+1}, \eta_{1|k}} P_{\eta_{1|k},\eta_{1|k}}^{-1} P_{\eta_{1|k}, \eta_{k+1}} - P_{\xi_{k+1}, \eta_{k+1}} \right ) P_{\widetilde{\eta}_{k+1, 1|k}, \widetilde{\eta}_{k+1, 1|k}}^{-1} P_{\widetilde{\eta}_{k+1,1|k}, \xi_{k+1}}
	.
\end{multline*}
Преобразуя выражение в скобках в силу равенства \eqref{equation:filtering:in_states:process:next_state_and_innovation_covariance}, окончательно получим:
$$
	P_{\widetilde{\xi}_{k+1,1|k+1}, \widetilde{\xi}_{k+1,1|k+1}}
		= P_{\widetilde{\xi}_{k+1,1|k},\widetilde{\xi}_{k+1,1|k}}
		- P_{\xi_{k+1}, \widetilde{\eta}_{k+1,1|k}} P_{\widetilde{\eta}_{k+1, 1|k}, \widetilde{\eta}_{k+1, 1|k}}^{-1} P_{\widetilde{\eta}_{k+1,1|k}, \xi_{k+1}}
	.
$$

\subsection{Выводы}

Два соотношения
\begin{gather}
	\widehat{\xi}_{k+1,1|k+1}
		= \widehat{\xi}_{k+1,1|k}
		+ P_{\xi_{k+1}, \widetilde{\eta}_{k+1,1|k}} P_{\widetilde{\eta}_{k+1, 1|k}, \widetilde{\eta}_{k+1, 1|k}}^{-1}  \widetilde{\eta}_{k+1,1|k} ,
		\label{equation:filtering:in_states:process:optimal_estimate} \\
	%
	P_{\widetilde{\xi}_{k+1,1|k+1}, \widetilde{\xi}_{k+1,1|k+1}}
		= P_{\widetilde{\xi}_{k+1,1|k},\widetilde{\xi}_{k+1,1|k}}
		- P_{\xi_{k+1}, \widetilde{\eta}_{k+1,1|k}} P_{\widetilde{\eta}_{k+1, 1|k}, \widetilde{\eta}_{k+1, 1|k}}^{-1} P_{\widetilde{\eta}_{k+1,1|k}, \xi_{k+1}} ,
		\label{equation:filtering:in_states:process:optimal_error_covariance}
\end{gather}
описывают структуру всех линейных несмещенных и оптимальных фильтров: оптимальная оценка $\widehat{\xi}_{k+1}$ образована оптимальным "прогнозом"{}
$\widehat{\xi}_{k+1,1|k}$ состояния $\xi_{k+1}$, сделанным на основе имеющихся на момент $k$ наблюдений $\eta_{1}$, \dots, $\eta_k$:
$$
	\widehat{\xi}_{k+1,1|k} = \expectation{\xi_{k+1}} + P_{\xi_{k+1}, \eta_{1|k}} P_{\eta_{1|k}, \eta_{1|k}}^{-1} \left ( \eta_{1|k} - \expectation{\eta_{1|k}} \right ) , \\
$$
и ошибкой $\widetilde{\eta}_{k+1,1|k}$:
$$
	\widetilde{\eta}_{k+1,1|k} = \eta_{k+1} - \widehat{\eta}_{k+1, 1|k} , \\
$$
оптимального "прогноза"{} $\widehat{\eta}_{k+1, 1|k}$ очередного наблюдения $\eta_{k+1}$, сделанного на основе наблюдений $\eta_{1|k}$:
$$
	\widehat{\eta}_{k+1,1|k} = \expectation{\eta_{k+1}} + P_{\eta_{k+1}, \eta_{1|k}} P_{\eta_{1|k}, \eta_{1|k}}^{-1} \left ( \eta_{1|k} - \expectation{\eta_{1|k}} \right ) .
$$
Ковариация ошибки естественным образом зависит от ошибок прогнозов состояния и наблюдения:
\begin{gather*}
	P_{\widetilde{\xi}_{k+1, 1|k}, \widetilde{\xi}_{k+1, 1|k}} = P_{\xi_{k+1}, \xi_{k+1}} - P_{\xi_{k+1}, \eta_{1|k}} P_{\eta_{1|k},\eta_{1|k}}^{-1} P_{\eta_{1|k}, \xi_{k+1}} , \\
	%
	P_{\widetilde{\eta}_{k+1, 1|k}, \widetilde{\eta}_{k+1, 1|k}} = P_{\eta_{k+1}, \eta_{k+1}} - P_{\eta_{k+1}, \eta_{1|k}} P_{\eta_{1|k},\eta_{1|k}}^{-1} P_{\eta_{1|k}, \eta_{k+1}} .
\end{gather*}

Заметим, что рассмотренная процедура фильтрации требует от состояний $\xi_k$ и наблюдений $\eta_k$ только существование требуемых моментов. Линейнойсть фильтра по наблюдениям
не требует какой-либо линейности в самом процессе $\xi_k$ или в формировании наблюдений $\eta_k$ и не предполагает рекуррентного характера смены состояний $\xi_k$.