\chapter{Дополнения}

\section{Лемма об обратной матрице} \label{section:appendix:inversions}

\subsection{Уравнения для блоков}

Пусть некоторая блочная матрица является обратимой:
$$
	\begin{pmatrix}
		A & B \\
		C & D
	\end{pmatrix}^{-1}
	=
	\begin{pmatrix}
		E & F \\
		G & H
	\end{pmatrix}
	,
$$
тогда при умножении справа должны выполняться равенства:
\begin{gather*}
	\begin{pmatrix}
		A & B \\
		C & D
	\end{pmatrix}
	\begin{pmatrix}
		E & F \\
		G & H
	\end{pmatrix}
	=
	\begin{pmatrix}
		I & 0 \\
		0 & I
	\end{pmatrix} ,
\end{gather*}
где $I$ обозначают единичные матрицы подходящих размеров. Из равенства для матриц следуют равенства для блоков:
\begin{gather}
	A \cdot E + B \cdot G = I \label{equation:filtering:appendix:inversions:1} \\
	C \cdot E + D \cdot G = 0 \label{equation:filtering:appendix:inversions:2} \\
	A \cdot F + B \cdot H = 0 \label{equation:filtering:appendix:inversions:3} \\
	C \cdot F + D \cdot H = I \label{equation:filtering:appendix:inversions:4}
	,
\end{gather}
из которых можно получить разные выражения для блоков $E$, $F$, $G$ и $H$.

\subsection{Решения для блоков $E$ и $G$}

Из равенств \eqref{equation:filtering:appendix:inversions:1} и \eqref{equation:filtering:appendix:inversions:2} можно получить следующие выражения
для блоков $E$ и $G$:
\begin{gather*}
	\left \{
	\begin{array}{cl}
		A \cdot E + B \cdot G & = I \\
		C \cdot E + D \cdot G & = 0
	\end{array}
	\right .
	\Rightarrow
	\left \{
	\begin{array}{cl}
		A \cdot E + B \cdot G               & = I \\
		D^{-1} \cdot C \cdot E + D^{-1} \cdot D \cdot G & = 0
	\end{array}
	\right .
	\Rightarrow \\
	%
	\Rightarrow
	\left \{
	\begin{array}{cl}
		A \cdot E + B \cdot G & = I \\
		D^{-1} \cdot C \cdot E + G  & = 0
	\end{array}
	\right .
	\Rightarrow
	\left \{
	\begin{array}{cl}
		A \cdot E + B \cdot G & = I \\
		-B \cdot D^{-1} \cdot C \cdot E - B \cdot G  & = 0 \\
		D^{-1} \cdot C \cdot E + G  & = 0
	\end{array}
	\right .
	\Rightarrow \\
	%
	\Rightarrow
	\left \{
	\begin{array}{cl}
		A \cdot E - B \cdot D^{-1} \cdot C \cdot E & = I \\
		D^{-1} \cdot C \cdot E + G  & = 0
	\end{array}
	\right .
	\Rightarrow
	%
	\left \{
	\begin{array}{cl}
		\left ( A - B \cdot D^{-1} \cdot C \right ) \cdot E & = I \\
		G  & = - D^{-1} \cdot C \cdot E
	\end{array}
	\right .
	\Rightarrow \\
	%
	\Rightarrow
	\left \{
	\begin{array}{cl}
		E & = \left ( A - B \cdot D^{-1} \cdot C \right )^{-1} \\
		G  & = - D^{-1} \cdot C \cdot \left ( A - B \cdot D^{-1} \cdot C \right )^{-1}
	\end{array}
	\right .
	.
\end{gather*}

Если систему равенств \eqref{equation:filtering:appendix:inversions:1} и \eqref{equation:filtering:appendix:inversions:2} решать иным способом, то можно
получить другие выражения для блоков $E$ и $G$:
\begin{gather*}
	\left \{
	\begin{array}{cl}
		A \cdot E + B \cdot G & = I \\
		C \cdot E + D \cdot G & = 0
	\end{array}
	\right .
	\Rightarrow
	\left \{
	\begin{array}{cl}
		A^{-1} \cdot A \cdot E + A^{-1} \cdot B \cdot G & = A^{-1} \\
		C \cdot E + D \cdot G & = 0
	\end{array}
	\right .
	\Rightarrow \\
	%
	\Rightarrow
	\left \{
	\begin{array}{cl}
		E + A^{-1} \cdot B \cdot G & = A^{-1} \\
		C \cdot E + D \cdot G & = 0
	\end{array}
	\right .
	\Rightarrow
	\left \{
	\begin{array}{cl}
		E + A^{-1} \cdot B \cdot G & = A^{-1} \\
		-C \cdot E - C \cdot A^{-1} \cdot B \cdot G & = -C \cdot A^{-1} \\
		C \cdot E + D \cdot G & = 0
	\end{array}
	\right .
	\Rightarrow \\
	%
	\Rightarrow
	\left \{
	\begin{array}{cl}
		E + A^{-1} \cdot B \cdot G & = A^{-1} \\
		- C \cdot A^{-1} \cdot B \cdot G + D \cdot G & = -C \cdot A^{-1}
	\end{array}
	\right .
	\Rightarrow \\
	%
	\Rightarrow
	\left \{
	\begin{array}{cl}
		E & =  A^{-1} - A^{-1} \cdot B \cdot G \\
		\left ( D - C \cdot A^{-1} \cdot B \right ) \cdot G & = -C \cdot A^{-1}
	\end{array}
	\right .
	\Rightarrow \\
	%
	\Rightarrow
	\left \{
	\begin{array}{cl}
		E & =  A^{-1} + A^{-1} \cdot B \cdot \left ( D - C \cdot A^{-1} \cdot B \right )^{-1} \cdot C \cdot A^{-1} \\
		G & = - \left ( D - C \cdot A^{-1} \cdot B \right )^{-1} \cdot C \cdot A^{-1}
	\end{array}
	\right .
	.
\end{gather*}

\subsection{Решения для блоков $F$ и $H$}

Выражения для блоков $F$ и $H$ из равенств \eqref{equation:filtering:appendix:inversions:3} и \eqref{equation:filtering:appendix:inversions:4} являются
аналогичными:
\begin{gather*}
	\left \{
	\begin{array}{cl}
		A \cdot F + B \cdot H & = 0 \\
		C \cdot F + D \cdot H & = I
	\end{array}
	\right .
	\Rightarrow
	\left \{
	\begin{array}{cl}
		A^{-1} \cdot A \cdot F + A^{-1} \cdot B \cdot H & = 0 \\
		C \cdot F + D \cdot H & = I
	\end{array}
	\right .
	\Rightarrow \\
	%
	\Rightarrow
	\left \{
	\begin{array}{cl}
		F + A^{-1} \cdot B \cdot H & = 0 \\
		C \cdot F + D \cdot H & = I
	\end{array}
	\right .
	\Rightarrow
	\left \{
	\begin{array}{cl}
		F + A^{-1} \cdot B \cdot H & = 0 \\
		-C \cdot F - C \cdot A^{-1} \cdot B \cdot H & = 0 \\
		C \cdot F + D \cdot H & = I
	\end{array}
	\right .
	\Rightarrow \\
	%
	\Rightarrow
	\left \{
	\begin{array}{cl}
		F + A^{-1} \cdot B \cdot H & = 0 \\
		D \cdot H - C \cdot A^{-1} \cdot B \cdot H & = I
	\end{array}
	\right .
	\Rightarrow
	\left \{
	\begin{array}{cl}
		F & = - A^{-1} \cdot B \cdot H \\
		\left ( D - C \cdot A^{-1} \cdot B \right ) \cdot H & = I
	\end{array}
	\right .
	\Rightarrow \\
	%
	\Rightarrow
	\left \{
	\begin{array}{cl}
		F & = - A^{-1} \cdot B \cdot \left ( D - C \cdot A^{-1} \cdot B \right )^{-1} \\
		H & = \left ( D - C \cdot A^{-1} \cdot B \right )^{-1}
	\end{array}
	\right .
\end{gather*}

При альтернативном способе решения системы:
\begin{gather*}
	\left \{
	\begin{array}{cl}
		A \cdot F + B \cdot H & = 0 \\
		C \cdot F + D \cdot H & = I
	\end{array}
	\right .
	\Rightarrow
	\left \{
	\begin{array}{cl}
		A \cdot F + B \cdot H & = 0 \\
		D^{-1} \cdot C \cdot F + D^{-1} \cdot D \cdot H & = D^{-1}
	\end{array}
	\right .
	\Rightarrow \\
	%
	\Rightarrow
	\left \{
	\begin{array}{cl}
		A \cdot F + B \cdot H & = 0 \\
		D^{-1} \cdot C \cdot F + H & = D^{-1}
	\end{array}
	\right .
	\Rightarrow
	\left \{
	\begin{array}{cl}
		A \cdot F + B \cdot H & = 0 \\
		- B \cdot D^{-1} \cdot C \cdot F - B \cdot H & = - B \cdot D^{-1} \\
		D^{-1} \cdot C \cdot F + H & = D^{-1}
	\end{array}
	\right .
	\Rightarrow \\
	%
	\Rightarrow
	\left \{
	\begin{array}{cl}
		A \cdot F - B \cdot D^{-1} \cdot C \cdot F & = - B \cdot D^{-1} \\
		D^{-1} \cdot C \cdot F + H & = D^{-1}
	\end{array}
	\right .
	\Rightarrow \\
	%
	\Rightarrow
	\left \{
	\begin{array}{cl}
		\left ( A - B \cdot D^{-1} \cdot C \right ) \cdot F & = - B \cdot D^{-1} \\
		H & = D^{-1} - D^{-1} \cdot C \cdot F
	\end{array}
	\right .
	\Rightarrow \\
	%
	\Rightarrow
	\left \{
	\begin{array}{cl}
		F & = - B \cdot D^{-1} \cdot \left ( A - B \cdot D^{-1} \cdot C \right )^{-1} \\
		H & = D^{-1} - D^{-1} \cdot C \cdot \left ( A - B \cdot D^{-1} \cdot C \right )^{-1}
	\end{array}
	\right .
\end{gather*}

\subsection{Тождества обращений}

Таким образом, получено два выражения для блоков $E$ и $G$:
$$
	\begin{pmatrix}
		\left ( A - B \cdot D^{-1} \cdot C \right )^{-1} \\
		- D^{-1} \cdot C \cdot \left ( A - B \cdot D^{-1} \cdot C \right )^{-1}
	\end{pmatrix}
	\text{ и }
	\begin{pmatrix}
		A^{-1} + A^{-1} \cdot B \cdot \left ( D - C \cdot A^{-1} \cdot B \right )^{-1} \cdot C \cdot A^{-1} \\
		- \left ( D - C \cdot A^{-1} \cdot B \right )^{-1} \cdot C \cdot A^{-1}
	\end{pmatrix}
$$
и два выражения для блоков $F$ и $H$:
$$
	\begin{pmatrix}
		- A^{-1} \cdot B \cdot \left ( D - C \cdot A^{-1} \cdot B \right )^{-1} \\
		\left ( D - C \cdot A^{-1} \cdot B \right )^{-1}
	\end{pmatrix}
	\text{ и }
	\begin{pmatrix}
		- B \cdot D^{-1} \cdot \left ( A - B \cdot D^{-1} \cdot C \right )^{-1} \\
		D^{-1} - D^{-1} \cdot C \cdot \left ( A - B \cdot D^{-1} \cdot C \right )^{-1}
	\end{pmatrix}
	,
$$
из которых можно составить четыре различных вида обратной матрицы. Например, используя первый вариант для блоков $E$ и $G$ и второй вариант для блоков $F$ и $H$,
получим следующий вид обратной матрицы:
$$
	\begin{pmatrix}
		A & B \\
		C & D
	\end{pmatrix}^{-1}
	=
	\begin{pmatrix}
		\left ( A - B \cdot D^{-1} \cdot C \right )^{-1}                        & - B \cdot D^{-1} \cdot \left ( A - B \cdot D^{-1} \cdot C \right )^{-1} \\
		- D^{-1} \cdot C \cdot \left ( A - B \cdot D^{-1} \cdot C \right )^{-1} & D^{-1} - D^{-1} \cdot C \cdot \left ( A - B \cdot D^{-1} \cdot C \right )^{-1}
	\end{pmatrix}
	.
$$
Аналогичным образом можно получить и остальные три вида обратной матрицы.

Тем не менее, поскольку обратная матрица является единственной, то все четыре вида обратной матрицы должны совпадать, поэтому выражения, полученные для блоков
$E$ и $G$ также должны совпадать:
\begin{gather*}
	\left ( A - B \cdot D^{-1} \cdot C \right )^{-1} = A^{-1} + A^{-1} \cdot B \cdot \left ( D - C \cdot A^{-1} \cdot B \right )^{-1} \cdot C \cdot A^{-1} , \\
	D^{-1} \cdot C \cdot \left ( A - B \cdot D^{-1} \cdot C \right )^{-1} = \left ( D - C \cdot A^{-1} \cdot B \right )^{-1} \cdot C \cdot A^{-1} .
\end{gather*}
Для упрощения выражений можно в левых и правых частях заменить $B$ на $-B$ и $D^{-1}$ на $D$, в результате замен равенства, очевидно, сохраняются:
\begin{gather}
	\left ( A + B \cdot D \cdot C \right )^{-1} = A^{-1} - A^{-1} \cdot B \cdot \left ( D^{-1} + C \cdot A^{-1} \cdot B \right )^{-1} \cdot C \cdot A^{-1} ,
		\label{equation:filtering:appendix:inversions:first} \\
	D \cdot C \cdot \left ( A + B \cdot D \cdot C \right )^{-1} = \left ( D^{-1} + C \cdot A^{-1} \cdot B \right )^{-1} \cdot C \cdot A^{-1}
		\label{equation:filtering:appendix:inversions:second}.
\end{gather}

В силу единственности блоков $F$ и $H$ их выражения также должны совпадать, но поскольку выражения для них аналогичны, то принципиально новых тождеств при этом
не будет получено.