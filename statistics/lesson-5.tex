\documentclass[12pt]{article}

\usepackage[T1]{fontenc}
\usepackage[utf8]{inputenc}
\usepackage[english,russian]{babel}
\usepackage[margin=2cm]{geometry}
\usepackage{amsmath}

% команды вывода первой частной производной
\newcommand{\fpd}[1]{\frac{\partial}{\partial #1}}
\newcommand{\fpda}[2]{\frac{\partial #1}{\partial #2}}
\newcommand{\fpdp}[2]{\fpd{#2} \left ( #1 \right )}

\newcommand{\expectation}[1]{\mathtt{M} \left [ #1 \right ]}
\newcommand{\conditionalexpectation}[2]{\expectation{ #1 \left | #2 \right .}}
\newcommand{\variance}[1]{\mathtt{D} \left [ #1 \right ]}
\newcommand{\covariance}[2]{\mathtt{cov} \left ( #1, #2 \right )}

\newcommand{\modulus}[1]{\left | #1 \right |}
\newcommand{\norm}[1]{\left \| {#1} \right \|}

\newcommand{\event}[1]{\left \{ #1 \right \} }
\newcommand{\probability}[1]{P \event{#1}}


\begin{document}

    \title{Домашнее задание №5}
    \author{Тигетов Давид Георгиевич}
    \date{}
    \maketitle

    \section*{Задача 5.2}
    В 10 испытаниях Бернулли случилось 8 "успехов"{} (появлений 1). Принимается или отвергается на уровне значимости 0.05 гипотеза $H: p = \frac{1}{2}$ при проверке с помощью критерия,
    статистикой которого служит число "успехов".

    \subsection*{Решение:}
    Пусть величины $X_1, \dots, X_{10}$ обозначают результаты испытаний. Статистика $S$ количества "успехов"
    \[
        S = \sum_{i=1}^n X_i \sim B(n,p)
    \]
    имеет биномиальное распределение.

    Вычислим вероятность получения 8 и более успехов при $p = \frac{1}{2}$:
    \begin{multline*}
        \probability{S \ge 8}
        = \probability{S = 8} + \probability{S = 9} + \probability{S = 10} = \\
        %
        = C_{10}^8 p^8 (1-p)^2 + C_{10}^9 p^9 (1-p) + C_{10}^{10} p^{10}
        = \frac{10 \cdot 9}{2} \frac{1}{2^{10}} + 9 \frac{1}{2^{10}} + \frac{1}{2^{10}}
        = 55 \frac{1}{2^{10}}
        \approx 0.0537 .
    \end{multline*}
    Получилось больше, чем уровень значимости 0.05, поэтому гипотеза $H$ принимается.

    \subsection*{Ответ:}
    Принимается.

\end{document}