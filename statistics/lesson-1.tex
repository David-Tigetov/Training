\documentclass[12pt]{article}

\usepackage[T1]{fontenc}
\usepackage[utf8]{inputenc}
\usepackage[english,russian]{babel}
\usepackage[margin=2cm]{geometry}
\usepackage{amsmath}

\newcommand{\scalarproduct}[2]{\left ( #1, #2 \right )}
\newcommand{\modulus}[1]{\left | #1 \right |}

\newcommand{\perpendicular}[2]{\texttt{ort}_{#1} {#2}}
\newcommand{\projection}[2]{\texttt{pr}_{#1} {#2}}

\newcommand{\set}[1]{\left \{ #1 \right \}}

\newcommand{\kernel}{\mathtt{Ker}}
\newcommand{\image}{\mathtt{Im}}

\begin{document}

    \title{Домашнее задание №1}
    \author{Тигетов Давид Георгиевич}
    \date{}
    \maketitle

    \section*{Задача 1.4}
    Пусть $\xi_1$, $\xi_2$, \dots --- независимые случайные величины, имеющие распределение Бернулли с вероятностью $p$. Положим $S_n = \xi_1 + \dots + \xi_n$. Найти $p_i = \probability{S_n = i}$
    для $i = 0, 1, \dots, n$ и вычислить $\expectation{S_n}$.

    \subsection*{Решение:}
    Сумма $S_n$ равна $i$, если среди величин $\xi_k$ ровно $i$ принимают значения 1 и $n-i$ принимают значение 0, поэтому
    \[
        p_i
        = \probability{S_n = i}
        = C_n^i p^i (1-p)^{n-i}
        .
    \]
    Математическое ожидание величин $\xi_i$ найдем по определению
    \[
        \expectation{\xi_i} = 0 \cdot (1-p) + 1 \cdot p = p,
    \]
    а для величин $S_n$, используя свойство линейности математического ожидания:
    \[
        \expectation{S_n}
        = \expectation{\sum_{i=1}^n \xi_i}
        = \sum_{i=1}^n \expectation{\xi_i}
        = \sum_{i=1}^n p
        = n \cdot p
        .
    \]

    \subsection*{Ответ:}
    \begin{enumerate}
        \item $\probability{S_n = i} = C_n^i p^i (1-p)^{m-i}$,
        \item $\expectation{S_n} = n p$.
    \end{enumerate}

\end{document}