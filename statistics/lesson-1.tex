\documentclass[12pt]{article}

\usepackage[T1]{fontenc}
\usepackage[utf8]{inputenc}
\usepackage[english,russian]{babel}
\usepackage[margin=2cm]{geometry}
\usepackage{amsmath}

% команды вывода первой частной производной
\newcommand{\fpd}[1]{\frac{\partial}{\partial #1}}
\newcommand{\fpda}[2]{\frac{\partial #1}{\partial #2}}
\newcommand{\fpdp}[2]{\fpd{#2} \left ( #1 \right )}

\newcommand{\expectation}[1]{\mathtt{M} \left [ #1 \right ]}
\newcommand{\conditionalexpectation}[2]{\expectation{ #1 \left | #2 \right .}}
\newcommand{\variance}[1]{\mathtt{D} \left [ #1 \right ]}
\newcommand{\covariance}[2]{\mathtt{cov} \left ( #1, #2 \right )}

\newcommand{\modulus}[1]{\left | #1 \right |}
\newcommand{\norm}[1]{\left \| {#1} \right \|}

\newcommand{\event}[1]{\left \{ #1 \right \} }
\newcommand{\probability}[1]{P \event{#1}}


\begin{document}

    \title{Домашнее задание №1}
    \author{Тигетов Давид Георгиевич}
    \date{}
    \maketitle

    \section*{Задача 1.4}
    Пусть $\xi_1$, $\xi_2$, \dots --- независимые случайные величины, имеющие распределение Бернулли $B(m, p)$. Положим $S_n = \xi_1 + \dots + \xi_n$. Найти $p_i = \probability{S_n = i}$
    для $i = 0, 1, \dots, n$ и вычислить $\expectation{S_n}$.

    \subsection*{Решение:}
    Каждую величину $\xi_i$ представим в виде суммы индикаторных случайных величин $\eta_{i,j}$ ($j=\overline{1,m}$), принимающих значение 1 с вероятностью $p$ и 0 с вероятностью $1-p$:
    \[
        \eta_{i,j}
        = \left \{
        \begin{array}{ll}
            0, & \text{с вероятностью } 1-p \\
            1, & \text{с вероятностью } p
        \end{array}
        \right .
        .
    \]
    Случайные величины $\xi_i$ представим в виде сумм:
    \[
        \xi_i = \sum_{j=1}^m \eta_{i,j} ,
    \]
    тогда величины $S_n$ имеют вид:
    \[
        S_n = \sum_{i=1}^n \xi_i = \sum_{i=1}^n \sum_{j=1}^m \eta_{i,j} .
    \]
    Величины $\xi_i$ по условию являются независимыми, отсюда все величины $\eta_{i,j}$ также являются независимыми, поэтому величины $S_n$ имеют распределение Бернулли $B(nm, p)$.
    Следовательно,
    \[
        \probability{S_n = i} = C_{nm}^i p^i (1-p)^{nm-i} ,
    \]
    и математическое ожидание
    \[
        \expectation{S_n} = n m p.
    \]

    \subsection*{Ответ:}
    \begin{enumerate}
        \item $\probability{S_n = i} = C_{nm}^i p^i (1-p)^{nm-i}$,
        \item $\expectation{S_n} = n m p$.
    \end{enumerate}

\end{document}