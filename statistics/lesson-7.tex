\documentclass[12pt]{article}

\usepackage[T1]{fontenc}
\usepackage[utf8]{inputenc}
\usepackage[english,russian]{babel}
\usepackage[margin=2cm]{geometry}
\usepackage{amsmath}
\usepackage{amsfonts}

% команды вывода первой частной производной
\newcommand{\fpd}[1]{\frac{\partial}{\partial #1}}
\newcommand{\fpda}[2]{\frac{\partial #1}{\partial #2}}
\newcommand{\fpdp}[2]{\fpd{#2} \left ( #1 \right )}

\newcommand{\expectation}[1]{\mathtt{M} \left [ #1 \right ]}
\newcommand{\conditionalexpectation}[2]{\expectation{ #1 \left | #2 \right .}}
\newcommand{\variance}[1]{\mathtt{D} \left [ #1 \right ]}
\newcommand{\covariance}[2]{\mathtt{cov} \left ( #1, #2 \right )}

\newcommand{\modulus}[1]{\left | #1 \right |}
\newcommand{\norm}[1]{\left \| {#1} \right \|}

\newcommand{\event}[1]{\left \{ #1 \right \} }
\newcommand{\probability}[1]{P \event{#1}}


\begin{document}

    \title{Домашнее задание №7}
    \author{Тигетов Давид Георгиевич}
    \date{}
    \maketitle

    \section*{Задача 7.1}
    Вычислите предел при $\alpha \rightarrow 0$ экономичности критерия Вальда $E(\alpha)$, определяемой формулой:
    \[
        E(\alpha) \approx \frac{1 - 2 \alpha}{2 x_\alpha^2} \ln \frac{1 - \alpha}{\alpha} .
    \]
    с помощью асимптотики $1 - \Phi(x) \sim \frac{1}{\sqrt{2 \pi} x} e^{- \frac{x^2}{2}}$ при $x \rightarrow + \infty$.

    \subsection*{Решение:}
    Выделим в экономичности два слагаемых:
    \[
        E(\alpha)
        \approx \frac{1 - 2 \alpha}{2 x_\alpha^2} \ln \frac{1 - \alpha}{\alpha}
        = \frac{1 - 2 \alpha}{2 x_\alpha^2} \left ( \ln \left ( 1 - \alpha \right ) - \ln \alpha \right )
        = \frac{1 - 2 \alpha}{2 x_\alpha^2} \ln \left ( 1 - \alpha \right ) - \frac{1 - 2 \alpha}{2 x_\alpha^2} \ln \alpha
    \]
    Заметим, что $\lim \limits_{\alpha \rightarrow 0} x_\alpha = - \infty$ поэтому предел первого слагаемого легко вычисляется:
    \[
        \lim \limits_{\alpha \rightarrow 0} \frac{1 - 2 \alpha}{2 x_\alpha^2} \ln \left ( 1 - \alpha \right ) = 0 ,
    \]
    поэтому предел экономичности определяется вторым слагаемым:
    \begin{equation} \label{7:limit}
        \lim \limits_{\alpha \rightarrow 0} E(\alpha)
        = - \lim \limits_{\alpha \rightarrow 0} \frac{1 - 2 \alpha}{2} \frac{\ln \alpha}{x_\alpha^2}
        = - \frac{1}{2} \lim \limits_{\alpha \rightarrow 0} \frac{\ln \alpha}{x_\alpha^2} .
    \end{equation}

    В силу симметрии функции Лапласа $\Phi(x)$ имеет место эквивалентность:
    \begin{gather*}
        \Phi(x) \sim \frac{1}{\sqrt{2 \pi} (-x)} e^{- \frac{x^2}{2}}, \\
        \text{при } x \rightarrow - \infty .
    \end{gather*}
    Тогда квантиль $x_\alpha$:
    \begin{gather*}
        \Phi \left ( x_\alpha \right ) = \alpha, \\
        \frac{1}{\sqrt{2 \pi} (-x_\alpha)} e^{- \frac{x_\alpha^2}{2}} \approx \alpha, \\
        - \ln \left ( \sqrt{2 \pi} (-x_\alpha) \right ) - \frac{x_\alpha^2}{2} = \ln \alpha, \\
        - \frac{\ln \left ( \sqrt{2 \pi} (-x_\alpha) \right )}{x_\alpha^2} - \frac{1}{2} = \frac{\ln \alpha}{x_\alpha^2} .
    \end{gather*}
    Подставим левую часть в равенство \eqref{7:limit}:
    \[
        \lim \limits_{\alpha \rightarrow 0} E(\alpha)
        = - \frac{1}{2} \lim \limits_{\alpha \rightarrow 0} \left ( - \frac{\ln \left ( \sqrt{2 \pi} (-x_\alpha) \right )}{x_\alpha^2} - \frac{1}{2} \right )
        = \frac{1}{2} \lim \limits_{\alpha \rightarrow 0} \frac{\ln \left ( \sqrt{2 \pi} (-x_\alpha) \right )}{x_\alpha^2} + \frac{1}{4}
        = \frac{1}{4} ,
    \]
    поскольку $\ln(-x_\alpha)$ возрастает медленней степени $x_\alpha^2$.

    \subsection*{Ответ:}
    $\frac{1}{4}$.
\end{document}