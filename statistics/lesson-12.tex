\documentclass[12pt]{article}

\usepackage[T1]{fontenc}
\usepackage[utf8]{inputenc}
\usepackage[english,russian]{babel}
\usepackage[margin=2cm]{geometry}
\usepackage{amsmath}
\usepackage{amsfonts}

% команды вывода первой частной производной
\newcommand{\fpd}[1]{\frac{\partial}{\partial #1}}
\newcommand{\fpda}[2]{\frac{\partial #1}{\partial #2}}
\newcommand{\fpdp}[2]{\fpd{#2} \left ( #1 \right )}

\newcommand{\expectation}[1]{\mathtt{M} \left [ #1 \right ]}
\newcommand{\conditionalexpectation}[2]{\expectation{ #1 \left | #2 \right .}}
\newcommand{\variance}[1]{\mathtt{D} \left [ #1 \right ]}
\newcommand{\covariance}[2]{\mathtt{cov} \left ( #1, #2 \right )}

\newcommand{\modulus}[1]{\left | #1 \right |}
\newcommand{\norm}[1]{\left \| {#1} \right \|}

\newcommand{\event}[1]{\left \{ #1 \right \} }
\newcommand{\probability}[1]{P \event{#1}}


\begin{document}

    \title{Домашнее задание №12}
    \author{Тигетов Давид Георгиевич}
    \date{}
    \maketitle

    \section*{Задача 12.1}
    а) Используя марковское свойство, вывести соотношение Колмогорова-Чепмена (формула (3)).
    б) Доказать равенство (4).

    \subsection*{Решение:}
    Согласно формуле умножения:
    \begin{multline*}
        \probability{X(0) = i, X(t) = k, X(t+h) = j} = \\
        %
        = \probability{X(0) = i, X(t) = k} \cdot \conditionalprobability{X(t+h) = j}{X(0) = i, X(t) = k} = \\
        %
        = \probability{X(0) = i} \cdot \conditionalprobability{X(t) = k}{X(0) = i} \cdot \conditionalprobability{X(t+h) = j}{X(0) = i, X(t) = k}.
    \end{multline*}
    Используем марковское свойство в последнем множителе в правой части:
    \[
        \probability{X(0) = i, X(t) = k, X(t+h) = j} = \probability{X(0) = i} \cdot \conditionalprobability{X(t) = k}{X(0) = i} \cdot \conditionalprobability{X(t+h) = j}{X(t) = k} , \\
    \]
    Суммируем по всем состояниям в момент $t$, используем свойство совместного распределения в левой части:
    \begin{multline*}
        \sum_k \probability{X(0) = i, X(t) = k, X(t+h) = j} = \\
        \shoveright{= \sum_k \probability{X(0) = i} \cdot \conditionalprobability{X(t) = k}{X(0) = i} \cdot \conditionalprobability{X(t+h) = j}{X(t) = k},} \\
        %
        \shoveleft{\probability{X(0) = i, X(t+h) = j} =} \\
        = \probability{X(0) = i} \sum_k \conditionalprobability{X(t) = k}{X(0) = i} \cdot \conditionalprobability{X(t+h) = j}{X(t) = k}.
    \end{multline*}
    Делим на $\probability{X(0) = i}$, используем определение условной вероятности:
    \begin{gather*}
        \frac{\probability{X(0) = i, X(t+h) = j}}{\probability{X(0) = i}} = \sum_k \conditionalprobability{X(t) = k}{X(0) = i} \cdot \conditionalprobability{X(t+h) = j}{X(t) = k}, \\
        \conditionalprobability{X(t+h) = j}{X(0) = i} = \sum_k \conditionalprobability{X(t) = k}{X(0) = i} \cdot \conditionalprobability{X(t+h) = j}{X(t) = k}, \\
        p_{ij}(0, t+h) = \sum_k p_{ik}(0, t) \cdot p_{kj}(t, t+h) .
    \end{gather*}
    Используем однородность
    \[
        p_{ij}(t+h) = \sum_k p_{ik}(t) \cdot p_{kj}(h) .
    \]

    Суммируем по всем возможным состояния в момент времени 0 с учётом вероятностей:
    \begin{gather*}
        \sum_i p_{ij}(t+h) p_i(0) = \sum_i \left ( \sum_k p_{ik}(t) \cdot p_{kj}(h) \right ) p_i(0) , \\
        \sum_i p_{ij}(t+h) p_i(0) = \sum_k \left ( \sum_i p_i(0) p_{ik}(t) \right ) \cdot p_{kj}(h).
    \end{gather*}
    Используем формулу полной вероятности:
    \[
        p_j(t+h) = \sum_k p_k(t) \cdot p_{kj}(h) .
    \]
    Переименовываем индекс:
    \[
        p_j(t+h) = \sum_i p_i(t) \cdot p_{ij}(h) .
    \]
\end{document}