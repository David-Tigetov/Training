\documentclass[12pt]{article}

\usepackage[T1]{fontenc}
\usepackage[utf8]{inputenc}
\usepackage[english,russian]{babel}
\usepackage[margin=2cm]{geometry}
\usepackage{amsmath}
\usepackage{amsfonts}

% команды вывода первой частной производной
\newcommand{\fpd}[1]{\frac{\partial}{\partial #1}}
\newcommand{\fpda}[2]{\frac{\partial #1}{\partial #2}}
\newcommand{\fpdp}[2]{\fpd{#2} \left ( #1 \right )}

\newcommand{\expectation}[1]{\mathtt{M} \left [ #1 \right ]}
\newcommand{\conditionalexpectation}[2]{\expectation{ #1 \left | #2 \right .}}
\newcommand{\variance}[1]{\mathtt{D} \left [ #1 \right ]}
\newcommand{\covariance}[2]{\mathtt{cov} \left ( #1, #2 \right )}

\newcommand{\modulus}[1]{\left | #1 \right |}
\newcommand{\norm}[1]{\left \| {#1} \right \|}

\newcommand{\event}[1]{\left \{ #1 \right \} }
\newcommand{\probability}[1]{P \event{#1}}


\begin{document}

    \title{Домашнее задание №6}
    \author{Тигетов Давид Георгиевич}
    \date{}
    \maketitle

    \section*{Задача 6.4}
    $X_i \sim \mathcal{N} \left ( \theta, \sigma^2 \right )$, $H_0: \theta = \theta_0$, $H_1: \theta = \theta_1$, где $\theta_1 > \theta_0$, параметр $\sigma$ известен. Найдите размер
    $n^*$ выборки, обеспечивающий заданные вероятности $\alpha$ и $\beta$ ошибок I и II рода для наиболее мощного критерия.

    \subsection*{Решение:}
    Плотность вероятности $f_X(x_1, \dots, x_n)$ выборки:
    \[
        f_X(x_1, \dots, x_n | \theta) = \frac{1}{\left ( 2 \pi \right )^{\frac{n}{2}} \sigma^n} e^{-\frac{1}{2 \sigma^2} \sum_{i=1}^n \left ( x_i - \theta \right )^2}.
    \]
    Отношение правдоподобия:
    \begin{multline*}
        S(x_1, \dots, x_n)
        = \frac{f_X(x_1, \dots, x_n | \theta_1)}{f_X(x_1, \dots, x_n | \theta_0)}
        = \frac{\frac{1}{\left ( 2 \pi \right )^{\frac{n}{2}} \sigma^n} e^{-\frac{1}{2 \sigma^2} \sum_{i=1}^n \left ( x_i - \theta_1 \right )^2}}{\frac{1}{\left ( 2 \pi \right )^{\frac{n}{2}} \sigma^n} e^{-\frac{1}{2 \sigma^2} \sum_{i=1}^n \left ( x_i - \theta_0 \right )^2}}
        = e^{- \frac{1}{2 \sigma^2} \sum_{i=1}^n \left ( \left ( x_i - \theta_1 \right )^2 - \left ( x_i - \theta_0 \right )^2 \right )} = \\
        %
        = e^{- \frac{1}{2 \sigma^2} \sum_{i=1}^n \left ( - 2 x_i \theta_1 + \theta_1^2 + 2 x_i \theta_0 - \theta_0^2 \right )}
        = e^{- \frac{1}{2 \sigma^2} \left ( - 2 \left ( \theta_1 - \theta_0 \right ) \sum_{i=1}^n x_i + n \left ( \theta_1^2 - \theta_0^2 \right ) \right )}
    \end{multline*}
    Гритическая область $G$ образована точками, в которых выполняется неравенство
    \begin{gather*}
        S(x_1, \dots, x_n) \ge c , \\
        e^{- \frac{1}{2 \sigma^2} \left ( - 2 \left ( \theta_1 - \theta_0 \right ) \sum_{i=1}^n x_i + n \left ( \theta_1^2 - \theta_0^2 \right ) \right )} \ge c , \\
        - \frac{1}{2 \sigma^2} \left ( - 2 \left ( \theta_1 - \theta_0 \right ) \sum_{i=1}^n x_i + n \left ( \theta_1^2 - \theta_0^2 \right ) \right ) \ge \ln c , \\
        \left ( \theta_1 - \theta_0 \right ) \sum_{i=1}^n x_i - \frac{n}{2} \left ( \theta_1^2 - \theta_0^2 \right ) \ge \sigma^2 \ln c , \\
        \sum_{i=1}^n x_i \ge \frac{\sigma^2 \ln c + \frac{n}{2} \left ( \theta_1^2 - \theta_0^2 \right )}{\theta_1 - \theta_0} , \\
        \sum_{i=1}^n x_i \ge \frac{\sigma^2 \ln c}{\theta_1 - \theta_0} + \frac{n}{2} \left ( \theta_1 + \theta_0 \right ) , \\
        \frac{1}{n} \sum_{i=1}^n x_i \ge \frac{\sigma^2 \ln c}{n \left ( \theta_1 - \theta_0 \right )} + \frac{1}{2} \left ( \theta_1 + \theta_0 \right ) = T(c, n).
    \end{gather*}

    Порог $c$ и количество $n$ выбираются из условий ($u_{1-\alpha}$ и $u_\beta$ --- квантили $\mathcal{N}(0,1)$ уровней $1 - \alpha$ и $\beta$):
    \begin{gather*}
        \left \{
        \begin{array}{rcl}
            \conditionalprobability{S(x_1, \dots, x_n) \ge c}{\theta_0} & = & \alpha \\
            \conditionalprobability{S(x_1, \dots, x_n) < c}{\theta_1}   & = & \beta
        \end{array}
        \right . , \\
        %
        \left \{
        \begin{array}{rcl}
            \conditionalprobability{\frac{1}{n} \sum_{i=1}^n x_i \ge T(c, n)}{\theta_0} & = & \alpha \\
            \conditionalprobability{\frac{1}{n} \sum_{i=1}^n x_i < T(c, n)}{\theta_1}   & = & \beta
        \end{array}
        \right . , \\
        %
        \left \{
        \begin{array}{rcl}
            1 - \Phi \left ( \frac{T(c,n) - \theta_0}{\frac{\sigma}{\sqrt{n}}} \right ) & = & \alpha \\
            \Phi \left ( \frac{T(c,n) - \theta_1}{\frac{\sigma}{\sqrt{n}}} \right )     & = & \beta
        \end{array}
        \right . , \\
        %
        \left \{
        \begin{array}{rcl}
            \frac{T(c,n) - \theta_0}{\frac{\sigma}{\sqrt{n}}} & = & u_{1-\alpha} \\
            \frac{T(c,n) - \theta_1}{\frac{\sigma}{\sqrt{n}}} & = & u_\beta
        \end{array}
        \right . , \\
        %
        \left \{
        \begin{array}{rcl}
            \frac{\frac{\sigma^2 \ln c}{n \left ( \theta_1 - \theta_0 \right )} + \frac{1}{2} \left ( \theta_1 + \theta_0 \right ) - \theta_0}{\frac{\sigma}{\sqrt{n}}} & = & u_{1-\alpha} \\
            \frac{\frac{\sigma^2 \ln c}{n \left ( \theta_1 - \theta_0 \right )} + \frac{1}{2} \left ( \theta_1 + \theta_0 \right ) - \theta_1}{\frac{\sigma}{\sqrt{n}}} & = & u_\beta
        \end{array}
        \right . , \\
        %
        \left \{
        \begin{array}{rcl}
            \frac{\frac{\sigma^2 \ln c}{n \left ( \theta_1 - \theta_0 \right )} + \frac{1}{2} \left ( \theta_1 - \theta_0 \right )}{\frac{\sigma}{\sqrt{n}}} & = & u_{1-\alpha} \\
            \frac{\frac{\sigma^2 \ln c}{n \left ( \theta_1 - \theta_0 \right )} - \frac{1}{2} \left ( \theta_1 - \theta_0 \right )}{\frac{\sigma}{\sqrt{n}}} & = & u_\beta
        \end{array}
        \right . , \\
        %
        \left \{
        \begin{array}{rcl}
            \frac{\sigma \ln c}{\theta_1 - \theta_0} + \frac{1}{2} \frac{\theta_1 - \theta_0}{\sigma} n & = & u_{1-\alpha} \sqrt{n} \\
            \frac{\sigma \ln c}{\theta_1 - \theta_0} - \frac{1}{2} \frac{\theta_1 - \theta_0}{\sigma} n & = & u_\beta \sqrt{n}
        \end{array}
        \right . , \\
        %
        \frac{\theta_1 - \theta_0}{\sigma} n = u_{1-\alpha} \sqrt{n} - u_\beta \sqrt{n} , \\
        \frac{\theta_1 - \theta_0}{\sigma} \sqrt{n} = u_{1-\alpha} - u_\beta, \\
        \sqrt{n} = \frac{\sigma}{\theta_1 - \theta_0} \left ( u_{1-\alpha} - u_\beta \right )
    \end{gather*}

    \subsection*{Ответ:}
    $n^* \ge \left ( \frac{\sigma}{\theta_1 - \theta_0} \left ( u_{1-\alpha} - u_\beta \right ) \right )^2$.

    \section*{Задача 6.8}
    Докажите, что при выполнении приведённых выше условий 1 и 2 критерий Неймана--Пирсона является строго несмещённым.

    \subsection*{Решение:}
    Мощность $W$ для критерия Неймана--Пирсона:
    \[
        W
        = \int \limits_{x \in G^*} p_1(x) dx
        \ge \int \limits_{x \in G^*} c_\alpha p_0(x) dx
        = c_\alpha \int \limits_{x \in G^*} p_0(x) dx
        = c_\alpha \alpha .
    \]
    Однако, вовсе не обязательно, что $c_\alpha > 1$.
\end{document}