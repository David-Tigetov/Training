\documentclass[12pt]{article}

\usepackage[T1]{fontenc}
\usepackage[utf8]{inputenc}
\usepackage[english,russian]{babel}
\usepackage[margin=2cm]{geometry}
\usepackage{amsmath}
\usepackage{amsfonts}
\usepackage{xcolor}
\usepackage{color}
\usepackage{amssymb}

% команды вывода первой частной производной
\newcommand{\fpd}[1]{\frac{\partial}{\partial #1}}
\newcommand{\fpda}[2]{\frac{\partial #1}{\partial #2}}
\newcommand{\fpdp}[2]{\fpd{#2} \left ( #1 \right )}

\newcommand{\expectation}[1]{\mathtt{M} \left [ #1 \right ]}
\newcommand{\conditionalexpectation}[2]{\expectation{ #1 \left | #2 \right .}}
\newcommand{\variance}[1]{\mathtt{D} \left [ #1 \right ]}
\newcommand{\covariance}[2]{\mathtt{cov} \left ( #1, #2 \right )}

\newcommand{\modulus}[1]{\left | #1 \right |}
\newcommand{\norm}[1]{\left \| {#1} \right \|}

\newcommand{\event}[1]{\left \{ #1 \right \} }
\newcommand{\probability}[1]{P \event{#1}}


\begin{document}

    \title{Задача 11}
    \author{Тигетов Давид Георгиевич}
    \date{}
    \maketitle

    \section*{Пункт а}
    Докажите, что подпространство $W = \left < a, b \right >$ является инвариантным для оператора, заданного матрицей $A$:
    \[
        a = \begin{pmatrix}
                -2 \\ 1 \\ 1
        \end{pmatrix},
        b = \begin{pmatrix}
                3 \\ 2 \\ -5
        \end{pmatrix},
        A = \begin{pmatrix}
                2 & 10 & 5 \\
                3 & -15 & 6 \\
                -1 & 9 & -7
        \end{pmatrix}
    \]

    \subsection*{Решение:}
    Применим оператор к векторам $a$ и $b$:
    \begin{gather*}
        A a
        =
        \begin{pmatrix}
            -4 + 10 + 5 \\
            -6 - 15 + 6 \\
            2 + 9 - 7
        \end{pmatrix}
        =
        \begin{pmatrix}
            11 \\
            -15 \\
            4
        \end{pmatrix} , \\
        %
        A b
        =
        \begin{pmatrix}
            6 + 20 - 25 \\
            9 - 30 - 30 \\
            - 3 + 18 + 35 \\
        \end{pmatrix}
        =
        \begin{pmatrix}
            1 \\
            - 51 \\
            50
        \end{pmatrix}
        .
    \end{gather*}

    Проверим принадлежность полученных векторов оболочке $W = \left < a, b \right >$:
    \begin{gather*}
        \begin{pmatrix}
            -2 & 3 & 11 & 1 \\
            1 & 2 & -15 & -51 \\
            1 & -5 & 4 & 50 \\
        \end{pmatrix}
        \rightarrow
        \begin{pmatrix}
            -2 & 7 & -19 & -101 \\
            1 & 0 & 0 & 0 \\
            1 & -7 & 19 & 101 \\
        \end{pmatrix}
        \rightarrow
        \begin{pmatrix}
            -2 & 7 & 0 & 0 \\
            1 & 0 & 0 & 0 \\
            1 & -7 & 0 & 0 \\
        \end{pmatrix}
    \end{gather*}
    Таким образом, $Aa \in W$ и $Ab \in W$, поэтому для любого вектора $x \in W$:
    \[
        x = \lambda_1 a + \lambda_2 b
        \rightarrow
        A x = \lambda_1 Aa + \lambda_2 Ab \in W.
    \]


\end{document}