\documentclass[12pt]{article}

\usepackage[T1]{fontenc}
\usepackage[utf8]{inputenc}
\usepackage[english,russian]{babel}
\usepackage[margin=2cm]{geometry}
\usepackage{amsmath}
\usepackage{amsfonts}
\usepackage{xcolor}
\usepackage{color}
\usepackage{amssymb}

% команды вывода первой частной производной
\newcommand{\fpd}[1]{\frac{\partial}{\partial #1}}
\newcommand{\fpda}[2]{\frac{\partial #1}{\partial #2}}
\newcommand{\fpdp}[2]{\fpd{#2} \left ( #1 \right )}

\newcommand{\expectation}[1]{\mathtt{M} \left [ #1 \right ]}
\newcommand{\conditionalexpectation}[2]{\expectation{ #1 \left | #2 \right .}}
\newcommand{\variance}[1]{\mathtt{D} \left [ #1 \right ]}
\newcommand{\covariance}[2]{\mathtt{cov} \left ( #1, #2 \right )}

\newcommand{\modulus}[1]{\left | #1 \right |}
\newcommand{\norm}[1]{\left \| {#1} \right \|}

\newcommand{\event}[1]{\left \{ #1 \right \} }
\newcommand{\probability}[1]{P \event{#1}}


\begin{document}

    \title{Задача 11}
    \author{Тигетов Давид Георгиевич}
    \date{}
    \maketitle

    \section*{Пункт а}
    Докажите, что подпространство $W = \left < a, b \right >$ является инвариантным для оператора, заданного матрицей $A$:
    \[
        a = \begin{pmatrix}
                -2 \\ 1 \\ 1
        \end{pmatrix},
        b = \begin{pmatrix}
                3 \\ 2 \\ -5
        \end{pmatrix},
        A = \begin{pmatrix}
                2 & 10 & 5 \\
                3 & -15 & 6 \\
                -1 & 9 & -7
        \end{pmatrix}
    \]

    \subsection*{Решение:}
    Применим оператор к векторам $a$ и $b$:
    \begin{gather*}
        A a
        =
        \begin{pmatrix}
            -4 + 10 + 5 \\
            -6 - 15 + 6 \\
            2 + 9 - 7
        \end{pmatrix}
        =
        \begin{pmatrix}
            11 \\
            -15 \\
            4
        \end{pmatrix} , \\
        %
        A b
        =
        \begin{pmatrix}
            6 + 20 - 25 \\
            9 - 30 - 30 \\
            - 3 + 18 + 35 \\
        \end{pmatrix}
        =
        \begin{pmatrix}
            1 \\
            - 51 \\
            50
        \end{pmatrix}
        .
    \end{gather*}

    Проверим принадлежность полученных векторов оболочке $W = \left < a, b \right >$:
    \begin{gather*}
        \begin{pmatrix}
            -2 & 3 & 11 & 1 \\
            1 & 2 & -15 & -51 \\
            1 & -5 & 4 & 50 \\
        \end{pmatrix}
        \rightarrow
        \begin{pmatrix}
            -2 & 7 & -19 & -101 \\
            1 & 0 & 0 & 0 \\
            1 & -7 & 19 & 101 \\
        \end{pmatrix}
        \rightarrow
        \begin{pmatrix}
            -2 & 7 & 0 & 0 \\
            1 & 0 & 0 & 0 \\
            1 & -7 & 0 & 0 \\
        \end{pmatrix}
    \end{gather*}
    Таким образом, $Aa \in W$ и $Ab \in W$, поэтому для любого вектора $x \in W$:
    \[
        x = \lambda_1 a + \lambda_2 b
        \rightarrow
        A x = \lambda_1 Aa + \lambda_2 Ab \in W.
    \]

    \section*{Пункт б}
    Найдите спектр и характеристический многочлен оператора, заданного матрицей
    \[
        \begin{pmatrix}
            -5 & -9 & 2 & 7 \\
            -3 & -2 & 1 & 3 \\
            9 & 9 & -5 & -12 \\
            -9 & -9 & 3 & 10
        \end{pmatrix}
    \]

    \subsection*{Решение:}
    Преобразования подобия сохраняют спектр и характеристический многочлен оператора:
    \begin{multline*}
        \begin{pmatrix}
            1 & 0 & 0 & 0 \\
            0 & 1 & 0 & 0 \\
            0 & 0 & 1 & 1 \\
            0 & 0 & 0 & 1
        \end{pmatrix}
        \begin{pmatrix}
            -5 & -9 & 2 & 7 \\
            -3 & -2 & 1 & 3 \\
            9 & 9 & -5 & -12 \\
            -9 & -9 & 3 & 10
        \end{pmatrix}
        \begin{pmatrix}
            1 & 0 & 0 & 0 \\
            0 & 1 & 0 & 0 \\
            0 & 0 & 1 & -1 \\
            0 & 0 & 0 & 1
        \end{pmatrix} = \\
        %
        =
        \begin{pmatrix}
            -5 & -9 & 2 & 7 \\
            -3 & -2 & 1 & 3 \\
            0 & 0 & -2 & -2 \\
            -9 & -9 & 3 & 10
        \end{pmatrix}
        \begin{pmatrix}
            1 & 0 & 0 & 0 \\
            0 & 1 & 0 & 0 \\
            0 & 0 & 1 & -1 \\
            0 & 0 & 0 & 1
        \end{pmatrix}
        =
        \begin{pmatrix}
            -5 & -9 & 2 & 5 \\
            -3 & -2 & 1 & 2 \\
            0 & 0 & -2 & 0 \\
            -9 & -9 & 3 & 7
        \end{pmatrix}.
    \end{multline*}

    Характеристический многочлен:
    \begin{multline*}
        P(\lambda)
        = \begin{vmatrix}
              -5 - \lambda & -9 & 2 & 5 \\
              -3 & -2 - \lambda & 1 & 2 \\
              0 & 0 & -2 - \lambda & 0 \\
              -9 & -9 & 3 & 7 - \lambda
        \end{vmatrix}
        = ( -2 - \lambda)
        \begin{vmatrix}
            -5 - \lambda & -9 & 5 \\
            -3 & -2 - \lambda & 2 \\
            -9 & -9 & 7 - \lambda
        \end{vmatrix} = \\
        %
        = (-2 - \lambda) ( (-5 - \lambda)(-2 - \lambda)(7 - \lambda) + 135 + 162 + 45(-2 - \lambda) + 18 (-5 - \lambda) - 27(7 - \lambda) ) = \\
        %
        = (-2 - \lambda) ( (-5 - \lambda)(-2 - \lambda)(7 - \lambda) + 297 - 90 - 45 \lambda - 90 - 18 \lambda - 189 + 27 \lambda ) = \\
        %
        = (-2 - \lambda) ( (-5 - \lambda)(-2 - \lambda)(7 - \lambda) - 72 - 36 \lambda ) = \\
        %
        = (-2 - \lambda) ( (-5 - \lambda)(-2 - \lambda)(7 - \lambda) + 36 (- 2 - \lambda ) ) = \\
        %
        = (-2 - \lambda)^2 ( (-5 - \lambda)(7 - \lambda) + 36 )
        = (\lambda + 2)^2 (-35 - 7 \lambda + 5 \lambda + \lambda^2 + 36) = \\
        %
        = (\lambda + 2)^2 (\lambda^2 - 2 \lambda + 1)
        = (\lambda + 2)^2 (\lambda - 1)^2 .
    \end{multline*}

    \subsection*{Ответ:}
    Характеристический многочлен: $(\lambda + 2)^2 (\lambda - 1)^2$, спектр $\set{-2, 1}$.

    \section*{Пункт в}
    Докажите, что подпространство $W = \left < a, b, c \right >$ является инвариантным для оператора, заданного матрицей $A$:
    \[
        a  = \begin{pmatrix}
                 -3 \\ 1 \\ 2 \\ 3 \\ -3
        \end{pmatrix} ,
        b = \begin{pmatrix}
                0 \\ 2 \\ -3 \\ 0 \\ 1
        \end{pmatrix},
        c = \begin{pmatrix}
                2 \\ -1 \\ 0 \\ -2 \\ 1
        \end{pmatrix},
        A =
        \begin{pmatrix}
            1 & 2 & -1 & -2 & 0 \\
            1 & 3 & 2 & -2 & 2 \\
            1 & 0 & -1 & 4 & 3 \\
            -1 & -2 & 1 & 2 & 0 \\
            1 & 0 & 2 & 1 & -2
        \end{pmatrix}
        .
    \]
    и постройте базис, в котором матрица оператора будет иметь блочно-диагональный вид с блоками 3x3 и 2x2.

    \subsection*{Решение:}
    Действие оператора на векторы $a$, $b$, $c$:
    \begin{gather*}
        A a
        =
        \begin{pmatrix}
            -3 + 2 - 2 - 6 + 0 \\
            -3 + 3 + 4 - 6 - 6 \\
            -3 + 0 - 2 + 12 - 9 \\
            3 - 2 + 2 + 6 + 0 \\
            -3 + 0 + 4 + 3 + 6
        \end{pmatrix}
        =
        \begin{pmatrix}
            -9 \\
            -8 \\
            -2 \\
            9 \\
            10
        \end{pmatrix} , \\
        %
        A b
        =
        \begin{pmatrix}
            0 + 4 + 3 + 0 + 0 \\
            0 + 6 - 6 + 0 + 2 \\
            0 + 0 + 3 + 0 + 3 \\
            0 - 4 - 3 + 0 + 0 \\
            0 + 0 - 6 + 0 - 2 \\
        \end{pmatrix}
        =
        \begin{pmatrix}
            7 \\
            2 \\
            6 \\
            -7 \\
            - 8
        \end{pmatrix} , \\
        %
        A c
        =
        \begin{pmatrix}
            2 - 2 + 0 + 4 + 0 \\
            2 - 3 + 0 + 4 + 2 \\
            2 + 0 + 0 - 8 + 3 \\
            -2 + 2 + 0 - 4 + 0 \\
            2 + 0 + 0 - 2 - 2
        \end{pmatrix}
        =
        \begin{pmatrix}
            4 \\
            5 \\
            -3 \\
            -4 \\
            -2
        \end{pmatrix}
    \end{gather*}

    Проверим принадлежность подпространству $W$:
    \begin{gather*}
        \begin{pmatrix}
            -3 & 0 & 2 & -9 & 7 & 4 \\
            1 & 2 & -1 & -8 & 2 & 5 \\
            2 & -3 & 0 & -2 & 6 & -3 \\
            3 & 0 & -2 & 9 & -7 & -4 \\
            -3 & 1 & 1 & 10 & -8 & -2
        \end{pmatrix}
    \end{gather*}
    \section*{Пункт г}
    Найдите все инвариантные подпространства оператора, заданного матрицей:
    \[
        A
        = \begin{pmatrix}
              1 & 0 & 0 \\
              -20 & -15 & 0 \\
              8 & -7 & 4
        \end{pmatrix}
    \]

    \subsection*{Решение:}
    Найдем собственные числа --- корни характеристического многочлена:
    \[
        \left | A - \lambda E \right |
        = \begin{vmatrix}
              1 - \lambda & 0 & 0 \\
              -20 & -15 - \lambda & 0 \\
              8 & -7 & 4 - \lambda
        \end{vmatrix}
        = 0 .
    \]
    Собстенные числа: $\lambda_1 = 1$, $\lambda_2 = -15$, $\lambda_3 = 4$, каждому соответствует собственный вектор $a_i$, являющийся решением
    соответствующей ОСЛУ:
    \begin{gather*}
        \left ( A - \lambda_1 E \right ) a_1
        = \begin{pmatrix}
              0 & 0 & 0 \\
              -20 & -16 & 0 \\
              8 & -7 & 3
        \end{pmatrix}
        a_1 = 0
        \rightarrow
        a_1
        = \begin{pmatrix}
              1 \\ - \frac{5}{4} \\ - \frac{8 + \frac{35}{4}}{3}
        \end{pmatrix}
        = \begin{pmatrix}
              1 \\ - \frac{5}{4} \\ - \frac{67}{12}
        \end{pmatrix} , \\
        %
        \left ( A - \lambda_2 E \right ) a_2
        = \begin{pmatrix}
              16 & 0 & 0 \\
              -20 & 0 & 0 \\
              8 & -7 & 19
        \end{pmatrix}
        a_2 = 0
        \rightarrow
        a_2
        = \begin{pmatrix}
              0 \\ 1 \\ \frac{7}{19}
        \end{pmatrix} , \\
        %
        \left ( A - \lambda_3 E \right ) a_3
        = \begin{pmatrix}
              -3 & 0 & 0 \\
              -20 & -19 & 0 \\
              8 & -7 & 0
        \end{pmatrix}
        a_3 = 0
        \rightarrow
        a_3
        = \begin{pmatrix}
              0 \\ 0 \\ 1
        \end{pmatrix}
        %
    \end{gather*}

    \subsection*{Ответ:}
    Всего три инвариантных подпространства $W_i = \left < a_i \right >$, где
    \[
        a_1 = \begin{pmatrix}
                  1 \\ - \frac{5}{4} \\ - \frac{67}{12}
        \end{pmatrix} ,
        a_2 = \begin{pmatrix}
                  0 \\ 1 \\ \frac{7}{19}
        \end{pmatrix} ,
        a_3 = \begin{pmatrix}
                  0 \\ 0 \\ 1
        \end{pmatrix} .
    \]
\end{document}