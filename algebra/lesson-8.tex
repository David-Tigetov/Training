\documentclass[12pt]{article}

\usepackage[T1]{fontenc}
\usepackage[utf8]{inputenc}
\usepackage[english,russian]{babel}
\usepackage[margin=2cm]{geometry}
\usepackage{amsmath}
\usepackage{amsfonts}
\usepackage{xcolor}
\usepackage{color}

% команды вывода первой частной производной
\newcommand{\fpd}[1]{\frac{\partial}{\partial #1}}
\newcommand{\fpda}[2]{\frac{\partial #1}{\partial #2}}
\newcommand{\fpdp}[2]{\fpd{#2} \left ( #1 \right )}

\newcommand{\expectation}[1]{\mathtt{M} \left [ #1 \right ]}
\newcommand{\conditionalexpectation}[2]{\expectation{ #1 \left | #2 \right .}}
\newcommand{\variance}[1]{\mathtt{D} \left [ #1 \right ]}
\newcommand{\covariance}[2]{\mathtt{cov} \left ( #1, #2 \right )}

\newcommand{\modulus}[1]{\left | #1 \right |}
\newcommand{\norm}[1]{\left \| {#1} \right \|}

\newcommand{\event}[1]{\left \{ #1 \right \} }
\newcommand{\probability}[1]{P \event{#1}}


\begin{document}

    \title{Задача 8}
    \author{Тигетов Давид Георгиевич}
    \date{}
    \maketitle

    \section*{Пункт а}
    Методом наименьших квадратов найти псевдорешение следующей несовместной системы линейных уравнений.
    \[
        \left \{
        \begin{array}{rrrrrcl}
            2 x_1 &   &     & - & x_3 & = & 1  \\
            &   & x_2 & + & x_3 & = & -1 \\
            x_1   & - & x_2 & + & x_3 & = & 0  \\
            x_1   &   &     & - & x_3 & = & -1
        \end{array}
        \right .
    \]

    \subsection*{Решение:}
    Матрица нормальной системы:
    \[
        \begin{pmatrix}
            2  & 0 & 1  & 1  \\
            0  & 1 & -1 & 0  \\
            -1 & 1 & 1  & -1
        \end{pmatrix}
        \begin{pmatrix}
            2 & 0  & -1 \\
            0 & 1  & 1  \\
            1 & -1 & 1  \\
            1 & 0  & -1
        \end{pmatrix}
        =
        \begin{pmatrix}
            4 + 1 + 1 & -1    & -2 + 1 -1     \\
            -1        & 1 + 1 & 1 - 1         \\
            -2 + 1 -1 & 1 - 1 & 1 + 1 + 1 + 1
        \end{pmatrix}
        =
        \begin{pmatrix}
            6  & -1 & -2 \\
            -1 & 2  & 0  \\
            -2 & 0  & 4
        \end{pmatrix}
    \]
    Правая часть:
    \[
        \begin{pmatrix}
            2  & 0 & 1  & 1  \\
            0  & 1 & -1 & 0  \\
            -1 & 1 & 1  & -1
        \end{pmatrix}
        \begin{pmatrix}
            1  \\
            -1 \\
            0  \\
            -1
        \end{pmatrix}
        =
        \begin{pmatrix}
            2 - 1 \\
            -1    \\
            -1 -1 + 1
        \end{pmatrix}
        =
        \begin{pmatrix}
            1  \\
            -1 \\
            -1
        \end{pmatrix}
    \]
    Решение системы:
    \begin{gather*}
        \begin{pmatrix}
            6  & -1 & -2 \\
            -1 & 2  & 0  \\
            -2 & 0  & 4
        \end{pmatrix}
        \left |
        \begin{pmatrix}
            1  \\
            -1 \\
            -1
        \end{pmatrix}
        \right .
        %
        \rightarrow
        \begin{pmatrix}
            6             & -1 & -2 \\
            - \frac{1}{2} & 1  & 0  \\
            -1            & 0  & 2
        \end{pmatrix}
        \left |
        \begin{pmatrix}
            1             \\
            - \frac{1}{2} \\
            - \frac{1}{2}
        \end{pmatrix}
        \right .
        %
        \rightarrow
        \begin{pmatrix}
            5             & -1 & 0 \\
            - \frac{1}{2} & 1  & 0 \\
            -1            & 0  & 2
        \end{pmatrix}
        \left |
        \begin{pmatrix}
            \frac{1}{2}   \\
            - \frac{1}{2} \\
            - \frac{1}{2}
        \end{pmatrix}
        \right .
        \rightarrow \\
        %
        \rightarrow
        \begin{pmatrix}
            \frac{9}{2}   & 0 & 0 \\
            - \frac{1}{2} & 1 & 0 \\
            -1            & 0 & 2
        \end{pmatrix}
        \left |
        \begin{pmatrix}
            0             \\
            - \frac{1}{2} \\
            - \frac{1}{2}
        \end{pmatrix}
        \right .
        \rightarrow
        \begin{pmatrix}
            x_1 \\
            x_2 \\
            x_3
        \end{pmatrix}
        =
        \begin{pmatrix}
            0             \\
            - \frac{1}{2} \\
            - \frac{1}{4}
        \end{pmatrix}
    \end{gather*}

    \subsection*{Ответ:}
    $x_1 = 0$, $x_2 = - \frac{1}{2}$, $x_3 = - \frac{1}{4}$.

    \section*{Пункт б}
    Составьте матрицу Грама следующей системы векторов
    \[
        a =
        \begin{pmatrix}
            1 \\ -i \\ 1 + i
        \end{pmatrix},
        b =
        \begin{pmatrix}
            i \\ 0 \\ -2
        \end{pmatrix},
        c =
        \begin{pmatrix}
            1 - i \\ -2 \\ -2
        \end{pmatrix}.
    \]

    \subsection*{Решение:}
    Скалярные произведения:
    \begin{align*}
        \scalarproduct{a}{a} & = 1 + i (-i) + (1+i)(1-i) = 1 - i^2 + 1 - i^2 = 4 , \\
        \scalarproduct{a}{b} & = i + (1-i)(-2) = i - 2 + 2i = -2 + 3i , \\
        \scalarproduct{a}{c} & = (1-i) + i(-2) + (1-i)(-2) = 1 - i - 2i - 2 + 2i = -1 - i, \\
        \scalarproduct{b}{b} & = (-i)i + (-2)(-2) = -i^2 + 4 = 5, \\
        \scalarproduct{b}{c} & = (-i)(1-i) + (-2)(-2) = -i + i^2 + 4 = -i + 3, \\
        \scalarproduct{c}{c} & = (1+i)(1-i) + (-2)(-2) + (-2)(-2) = 1 - i^2 + 4 + 4 = 10.
    \end{align*}

    \subsection*{Ответ:}
    $
    \begin{pmatrix}
        4       & -2 + 3i & -1 - i \\
        -2 - 3i & 5       & 3 - i  \\
        -1 + i  & 3 + i   & 10
    \end{pmatrix}
    $

    \section*{Пункт в}
    Методом наименьших квадратов найти \textcolor{magenta}{аппроксимацию} (наилучшее среднеквадратическое приближение) функции $f$, заданной значениями $f(0) = 1$, $f(1)=2$, $f(2)=3$, $f(3)=5$ квадратичным
    многочленом.

    \subsection*{Решение:}
    Пусть $p(x)$ --- искомый квадратичный многочлен:
    \[
        p(x) = a + b x + c x^2 ,
    \]
    коэффициенты которого удовлетворяют СЛУ:
    \[
        \begin{pmatrix}
            1 & 1 & 1 & 1 \\
            0 & 1 & 2 & 3 \\
            0 & 1 & 4 & 9
        \end{pmatrix}
        \begin{pmatrix}
            1 & 0 & 0 \\
            1 & 1 & 1 \\
            1 & 2 & 4 \\
            1 & 3 & 9
        \end{pmatrix}
        \begin{pmatrix}
            a \\
            b \\
            c
        \end{pmatrix}
        =
        \begin{pmatrix}
            1 & 1 & 1 & 1 \\
            0 & 1 & 2 & 3 \\
            0 & 1 & 4 & 9
        \end{pmatrix}
        \begin{pmatrix}
            1 \\
            2 \\
            3 \\
            5
        \end{pmatrix}
    \]
    Решаем СЛУ
    \begin{gather*}
        \begin{pmatrix}
            1 + 1 + 1 + 1 & 1 + 2 + 3 & 1 + 4 + 9   \\
            & 1 + 4 + 9 & 1 + 8 + 27  \\
            &           & 1 + 16 + 81
        \end{pmatrix}
        \left |
        \begin{pmatrix}
            1 + 2 + 3 + 5 \\
            2 + 6 + 15    \\
            2 + 12 + 45
        \end{pmatrix}
        \right .
        \rightarrow
        \begin{pmatrix}
            4  & 6  & 14 \\
            6  & 14 & 36 \\
            14 & 36 & 98
        \end{pmatrix}
        \left |
        \begin{pmatrix}
            11 \\
            23 \\
            59
        \end{pmatrix}
        \right .
        \rightarrow \\
        %
        \rightarrow
        \begin{pmatrix}
            4  & 6  & 14 \\
            6  & 14 & 36 \\
            10 & 30 & 84
        \end{pmatrix}
        \left |
        \begin{pmatrix}
            11 \\
            23 \\
            48
        \end{pmatrix}
        \right .
        \rightarrow
        \begin{pmatrix}
            4 & 6  & 14 \\
            6 & 14 & 36 \\
            6 & 24 & 70
        \end{pmatrix}
        \left |
        \begin{pmatrix}
            11 \\
            23 \\
            37
        \end{pmatrix}
        \right .
        \rightarrow
        \begin{pmatrix}
            4 & 6  & 14 \\
            6 & 14 & 36 \\
            0 & 10 & 34
        \end{pmatrix}
        \left |
        \begin{pmatrix}
            11 \\
            23 \\
            14
        \end{pmatrix}
        \right .
        \rightarrow \\
        %
        \rightarrow
        \begin{pmatrix}
            4 & 6  & 14 \\
            2 & 8  & 22 \\
            0 & 10 & 34
        \end{pmatrix}
        \left |
        \begin{pmatrix}
            11 \\
            12 \\
            14
        \end{pmatrix}
        \right .
        \rightarrow
        \begin{pmatrix}
            0 & -10 & -30 \\
            2 & 8   & 22  \\
            0 & 10  & 34
        \end{pmatrix}
        \left |
        \begin{pmatrix}
            -13 \\
            12  \\
            14
        \end{pmatrix}
        \right .
        \rightarrow
        \begin{pmatrix}
            0 & 0  & 4  \\
            2 & 8  & 22 \\
            0 & 10 & 34
        \end{pmatrix}
        \left |
        \begin{pmatrix}
            1  \\
            12 \\
            14
        \end{pmatrix}
        \right .
    \end{gather*}
    Таким образом,
    \begin{gather*}
        c
        = \frac{1}{4}
        = 0.25, \\
        %
        b
        = \frac{14 - 34 \cdot \frac{1}{4}}{10}
        = \frac{56 - 34}{40}
        = \frac{22}{40}
        = \frac{11}{20}
        = 0.55, \\
        %
        a
        = \frac{12 - 8 \cdot \frac{11}{20} - 22 \cdot \frac{1}{4}}{2}
        = \frac{240 - 88 - 110}{40}
        = \frac{42}{40}
        = \frac{21}{20}
        = 1.05
    \end{gather*}

    \subsection*{Ответ:}
    $1.05 + 0.55 x + 0.25 x^2$.

    \section*{Пункт г}
    Может ли эта матрица $G$ быть матрицей Грама какой-то системы векторов
    \[
        G
        =
        \begin{pmatrix}
            4 & 4 & 2 & 1 \\
            4 & 4 & 2 & 1 \\
            2 & 2 & 2 & 1 \\
            1 & 1 & 1 & 1
        \end{pmatrix}
    \]
    в $\mathbb{R}^4$ со стандартным скалярным произведением?

    \subsection*{Решение:}
    Найдем четыре вектора $a$, $b$, $c$, $d$, для которых матрица Грама совпадает с $G$.

    Первые две строки совпадают, следовательно $a$ = $b$. Убираем вектор $a$, остаётся сокращенная матрица Грама:
    \[
        \widetilde{G}
        =
        \begin{pmatrix}
            4 & 2 & 1 \\
            2 & 2 & 1 \\
            1 & 1 & 1
        \end{pmatrix}
    \]
    Заметим, что $\scalarproduct{d}{d} = 1$, поэтому положим:
    \[
        d
        =
        \begin{pmatrix}
            0 \\
            0 \\
            0 \\
            1
        \end{pmatrix}
        .
    \]
    $\scalarproduct{b}{d} = \scalarproduct{c}{d} = 1$, поэтому векторы $b$ и $c$ имеют вид:
    \[
        b
        =
        \begin{pmatrix}
            b_1 \\
            b_2 \\
            b_3 \\
            1
        \end{pmatrix},
        c
        =
        \begin{pmatrix}
            c_1 \\
            c_2 \\
            c_3 \\
            1
        \end{pmatrix}.
    \]
    $\scalarproduct{c}{c} = 2$, поэтому положим:
    \[
        c
        =
        \begin{pmatrix}
            0 \\
            0 \\
            1 \\
            1
        \end{pmatrix},
    \]
    $\scalarproduct{b}{c} = 2$:
    \[
        b
        =
        \begin{pmatrix}
            b_1 \\
            b_2 \\
            1   \\
            1
        \end{pmatrix},
    \]
    $\scalarproduct{b}{b} = 4$:
    \[
        b
        =
        \begin{pmatrix}
            1 \\
            1 \\
            1 \\
            1
        \end{pmatrix}.
    \]
    Таким образом, векторы
    \[
        a =
        \begin{pmatrix}
            1 \\ 1 \\ 1 \\ 1
        \end{pmatrix} ,
        b =
        \begin{pmatrix}
            1 \\ 1 \\ 1 \\ 1
        \end{pmatrix} ,
        c =
        \begin{pmatrix}
            0 \\ 0 \\ 1 \\ 1
        \end{pmatrix} ,
        d =
        \begin{pmatrix}
            0 \\ 0 \\ 0 \\ 1
        \end{pmatrix}
    \]
    имеют матрицу Грама $G$.

    \subsection*{Ответ:}
    Да.
\end{document}