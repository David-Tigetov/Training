\documentclass[12pt]{article}

\usepackage[T1]{fontenc}
\usepackage[utf8]{inputenc}
\usepackage[english,russian]{babel}
\usepackage[margin=2cm]{geometry}
\usepackage{amsmath}
\usepackage{amsfonts}
\usepackage{xcolor}
\usepackage{color}
\usepackage{amssymb}

% команды вывода первой частной производной
\newcommand{\fpd}[1]{\frac{\partial}{\partial #1}}
\newcommand{\fpda}[2]{\frac{\partial #1}{\partial #2}}
\newcommand{\fpdp}[2]{\fpd{#2} \left ( #1 \right )}

\newcommand{\expectation}[1]{\mathtt{M} \left [ #1 \right ]}
\newcommand{\conditionalexpectation}[2]{\expectation{ #1 \left | #2 \right .}}
\newcommand{\variance}[1]{\mathtt{D} \left [ #1 \right ]}
\newcommand{\covariance}[2]{\mathtt{cov} \left ( #1, #2 \right )}

\newcommand{\modulus}[1]{\left | #1 \right |}
\newcommand{\norm}[1]{\left \| {#1} \right \|}

\newcommand{\event}[1]{\left \{ #1 \right \} }
\newcommand{\probability}[1]{P \event{#1}}


\begin{document}

    \title{Задача 10}
    \author{Тигетов Давид Георгиевич}
    \date{}
    \maketitle

    \section*{Пункт а}
    Линейный оператор в некотором базисе $e_1$, $e_2$, $e_3$, $e_4$ имеет матрицу:
    \[
        A
        =
        \begin{pmatrix}
            1 & 2 & 0  & 1 \\
            3 & 0 & -1 & 2 \\
            2 & 5 & 3  & 1 \\
            1 & 2 & 1  & 3
        \end{pmatrix} .
    \]
    Найдите его матрицу в базисе $f_1 = e_1$, $f_1 = e_1 + e_2$, $f_1 = e_1 + e_2 + e_3$, $f_1 = e_1 + e_2 + e_3 + e_4$.

    \subsection*{Решение:}
    Матрица перехода от базиса $e_i$ к базису $f_i$:
    \[
        C
        =
        \begin{pmatrix}
            1 & 1 & 1 & 1 \\
            0 & 1 & 1 & 1 \\
            0 & 0 & 1 & 1 \\
            0 & 0 & 0 & 1
        \end{pmatrix}.
    \]
    Обратная матрица:
    \[
        C^{-1}
        =
        \begin{pmatrix}
            1 & -1 & 0  & 0  \\
            0 & 1  & -1 & 0  \\
            0 & 0  & 1  & -1 \\
            0 & 0  & 0  & 1
        \end{pmatrix}.
    \]
    Матрица оператора в базисе $f_i$:
    \begin{multline*}
        \widetilde{A}
        = C^{-1} A C
        =
        \begin{pmatrix}
            1 & -1 & 0  & 0  \\
            0 & 1  & -1 & 0  \\
            0 & 0  & 1  & -1 \\
            0 & 0  & 0  & 1
        \end{pmatrix}
        \begin{pmatrix}
            1 & 2 & 0  & 1 \\
            3 & 0 & -1 & 2 \\
            2 & 5 & 3  & 1 \\
            1 & 2 & 1  & 3
        \end{pmatrix}
        \begin{pmatrix}
            1 & 1 & 1 & 1 \\
            0 & 1 & 1 & 1 \\
            0 & 0 & 1 & 1 \\
            0 & 0 & 0 & 1
        \end{pmatrix} = \\
        %
        =
        \begin{pmatrix}
            1 & -1 & 0  & 0  \\
            0 & 1  & -1 & 0  \\
            0 & 0  & 1  & -1 \\
            0 & 0  & 0  & 1
        \end{pmatrix}
        \begin{pmatrix}
            1 & 3 & 3  & 4  \\
            3 & 4 & 2  & 4  \\
            2 & 7 & 10 & 11 \\
            1 & 3 & 4  & 7
        \end{pmatrix}
        =
        \begin{pmatrix}
            -2 & 0  & 1  & 0  \\
            1  & -4 & -8 & -7 \\
            1  & 4  & 6  & 4  \\
            1  & 3  & 4  & 7
        \end{pmatrix} .
    \end{multline*}

    \subsection*{Ответ:}
    $
    \begin{pmatrix}
        -2 & 0  & 1  & 0  \\
        1  & -4 & -8 & -7 \\
        1  & 4  & 6  & 4  \\
        1  & 3  & 4  & 7
    \end{pmatrix} .
    $

    \section*{Пункт б}
    Пусть $V = \mathbb{R}^3$ со стандартным скалярным произведением. Пусть $\mathcal{A}$ --- оператор ортогональной проекции на вектор $u = (1, -2, 1)$, $\mathcal{B}$ --- оператор,
    сопоставляющий вектору $v$ его векторное произведение с вектором $w = (1, 1, 1)$. Найдите матрицы операторов $\mathcal{A} \circ \mathcal{B}$ и $\mathcal{B} \circ \mathcal{A}$
    в стандартном базисе.

    \subsection*{Решение:}
    Оператор проекции $\mathcal{A}$ имеет выражение:
    \[
        \mathcal{A}(x)
        = \frac{\scalarproduct{x}{u}}{\scalarproduct{u}{u}} u
        = \frac{\scalarproduct{x}{u}}{6} u
    \]
    и матрицу $A$:
    \[
        A
        =
        \begin{pmatrix}
            1  & 1  & 1  \\
            -2 & -2 & -2 \\
            1  & 1  & 1
        \end{pmatrix}
        \begin{pmatrix}
            \frac{1}{6} & 0             & 0           \\
            0           & - \frac{2}{6} & 0           \\
            0           & 0             & \frac{1}{6}
        \end{pmatrix}
    \]

    Действие оператора $\mathcal{B}$ на векторы стандартного базиса:
    \begin{gather}
        \mathcal{B}(e_1)
        =
        \begin{vmatrix}
            e_1 & e_2 & e_3 \\
            1   & 0   & 0   \\
            1   & 1   & 1
        \end{vmatrix}
        = - e_2 + e_3 , \\
        %
        \mathcal{B}(e_2)
        =
        \begin{vmatrix}
            e_1 & e_2 & e_3 \\
            0   & 1   & 0   \\
            1   & 1   & 1
        \end{vmatrix}
        = e_1 - e_3 , \\
        %
        \mathcal{B}(e_3)
        =
        \begin{vmatrix}
            e_1 & e_2 & e_3 \\
            0   & 0   & 1   \\
            1   & 1   & 1
        \end{vmatrix}
        = - e_1 + e_2 ,
    \end{gather}
    откуда матрица оператора $\mathcal{B}$
    \[
        B
        =
        \begin{pmatrix}
            0  & 1  & -1 \\
            -1 & 0  & 1  \\
            1  & -1 & 0
        \end{pmatrix}
        .
    \]

    Матрица оператора $\mathcal{A} \circ \mathcal{B}$:
    \begin{multline*}
        A B =
        \begin{pmatrix}
            1  & 1  & 1  \\
            -2 & -2 & -2 \\
            1  & 1  & 1
        \end{pmatrix}
        \begin{pmatrix}
            \frac{1}{6} & 0             & 0           \\
            0           & - \frac{2}{6} & 0           \\
            0           & 0             & \frac{1}{6}
        \end{pmatrix}
        \begin{pmatrix}
            0  & 1  & -1 \\
            -1 & 0  & 1  \\
            1  & -1 & 0
        \end{pmatrix} = \\
        %
        =
        \begin{pmatrix}
            1  & 1  & 1  \\
            -2 & -2 & -2 \\
            1  & 1  & 1
        \end{pmatrix}
        \begin{pmatrix}
            0           & \frac{1}{6}  & - \frac{1}{6} \\
            \frac{2}{6} & 0            & -\frac{2}{6}  \\
            \frac{1}{6} & -\frac{1}{6} & 0
        \end{pmatrix}
        =
        \begin{pmatrix}
            \frac{1}{2} & 0 & - \frac{1}{2} \\
            - 1         & 0 & 1             \\
            \frac{1}{2} & 0 & - \frac{1}{2}
        \end{pmatrix} .
    \end{multline*}

    Матрица оператора $\mathcal{B} \circ \mathcal{A}$:
    \begin{multline*}
        B A =
        \begin{pmatrix}
            0  & 1  & -1 \\
            -1 & 0  & 1  \\
            1  & -1 & 0
        \end{pmatrix}
        \begin{pmatrix}
            1  & 1  & 1  \\
            -2 & -2 & -2 \\
            1  & 1  & 1
        \end{pmatrix}
        \begin{pmatrix}
            \frac{1}{6} & 0             & 0           \\
            0           & - \frac{2}{6} & 0           \\
            0           & 0             & \frac{1}{6}
        \end{pmatrix} = \\
        %
        =
        \begin{pmatrix}
            -3 & -3 & -3 \\
            0  & 0  & 0  \\
            3  & 3  & 3
        \end{pmatrix}
        \begin{pmatrix}
            \frac{1}{6} & 0             & 0           \\
            0           & - \frac{2}{6} & 0           \\
            0           & 0             & \frac{1}{6}
        \end{pmatrix}
        =
        \begin{pmatrix}
            -\frac{1}{2} & 1   & - \frac{1}{2} \\
            0            & 0   & 0             \\
            \frac{1}{2}  & - 1 & \frac{1}{2}
        \end{pmatrix} .
    \end{multline*}

    \subsection*{Ответ:}
    Матрица оператора $\mathcal{A} \circ \mathcal{B}$:
    $
    \begin{pmatrix}
        \frac{1}{2} & 0 & - \frac{1}{2} \\
        - 1         & 0 & 1             \\
        \frac{1}{2} & 0 & - \frac{1}{2}
    \end{pmatrix}
    $.
    Матрица оператора $\mathcal{B} \circ \mathcal{A}$:
    $
    \begin{pmatrix}
        -\frac{1}{2} & 1   & - \frac{1}{2} \\
        0            & 0   & 0             \\
        \frac{1}{2}  & - 1 & \frac{1}{2}
    \end{pmatrix}
    $.

    \section*{Пункт в}
    Линейный оператор в базисе $e_1 = ( 8, -6, 7 )$, $e_2 = ( -16, 7, -13 )$, $e_3 = ( 9, -3, 7 )$ имеет матрицу:
    \[
        A_e =
        \begin{pmatrix}
            1  & - 18 & 15 \\
            -1 & -22  & 20 \\
            1  & -25  & 22
        \end{pmatrix}
        .
    \]
    Найдите его матрицу в базисе $f_1 = ( 1, -2, 1 )$, $f_2 = ( 3, -1, 2 )$, $f_3 = ( 2, 1, 2 )$.

    \subsection*{Решение:}
    Матрица $C$ перехода от базиса $e_i$ к базису $f_i$ удовлетворяет матричному уравнению:
    \begin{gather*}
        \begin{pmatrix}
            1  & 3  & 2 \\
            -2 & -1 & 1 \\
            1  & 2  & 2
        \end{pmatrix}
        =
        \begin{pmatrix}
            8  & -16 & 9  \\
            -6 & 7   & -3 \\
            9  & -3  & 7
        \end{pmatrix}
        C
    \end{gather*}
    откуда
    \[
        C =
        \frac{1}{65}
        \begin{pmatrix}
            29  & -1 & -27 \\
            -7  & -2 & 11  \\
            -31 & 19 & 58
        \end{pmatrix}
    \]
    Матрица $A_f$ оператора в базисе $f_i$ имеет выражение:
    \begin{multline*}
        A_f
        = C^{-1} A_e C = \\
        %
        = \begin{pmatrix}
              5  & 7   & 1 \\
              -1 & -13 & 2 \\
              3  & 8   & 1
        \end{pmatrix}
        \begin{pmatrix}
            1  & - 18 & 15 \\
            -1 & -22  & 20 \\
            1  & -25  & 22
        \end{pmatrix}
        \frac{1}{65}
        \begin{pmatrix}
            29  & -1 & -27 \\
            -7  & -2 & 11  \\
            -31 & 19 & 58
        \end{pmatrix} = \\
        %
        =
        \begin{pmatrix}
            -8  & 10 & 19  \\
            27  & 14 & -11 \\
            -12 & 3  & 16
        \end{pmatrix}
    \end{multline*}

    \subsection*{Ответ:}
    $
    \begin{pmatrix}
        -8  & 10 & 19  \\
        27  & 14 & -11 \\
        -12 & 3  & 16
    \end{pmatrix}
    $.

    \section*{Пункт г}
    Пусть $V_1 = \left < a_1, b_1 \right >$, $V_2 = \left < a_2, b_2 \right >$ --- линейные подпространства в $\mathbb{R}^4$.
    $\mathcal{A}_1, \mathcal{A}_2: \mathbb{R}^4 \rightarrow \mathbb{R}^4$ --- операторы ортогональной проекции на подпространства $V_1$ и $V_2$ соответственно, а именно
    $\mathcal{A}_i = \projection{V_i}{v}$ для любого вектора $v \in \mathbb{R}_4$. Найдите матрицы операторов $\mathcal{A}_1$, $\mathcal{A}_2$, $\mathcal{A}_1 \circ \mathcal{A}_2$,
    $\mathcal{A}_2 \circ \mathcal{A}_1$, если
    \[
        a_1 = \begin{pmatrix}
                  -1 \\ 1 \\ 2 \\ 3
        \end{pmatrix} ,
        b_1 = \begin{pmatrix}
                  2 \\ 0 \\ 1 \\ 1
        \end{pmatrix} ,
        a_2 = \begin{pmatrix}
                  1 \\ 4 \\ -3 \\ 1
        \end{pmatrix} ,
        b_2 = \begin{pmatrix}
                  0 \\ 1 \\ 1 \\ -1
        \end{pmatrix} .
    \]

    \subsection*{Решение:}
    Найдем проекции векторов стандартного базиса на подпространство $V_1$, координаты проекций $c_{ij}$ удовлетворяют системе:
    \begin{gather*}
        \begin{pmatrix}
            -1 & 1 & 2 & 3 \\
            2  & 0 & 1 & 1
        \end{pmatrix}
        \begin{pmatrix}
            -1 & 2 \\
            1  & 0 \\
            2  & 1 \\
            3  & 1
        \end{pmatrix}
        \begin{pmatrix}
            c_{11} & c_{12} & c_{13} & c_{14} \\
            c_{21} & c_{22} & c_{23} & c_{14}
        \end{pmatrix}
        =
        \begin{pmatrix}
            -1 & 1 & 2 & 3 \\
            2  & 0 & 1 & 1
        \end{pmatrix}
        \begin{pmatrix}
            1 & 0 & 0 & 0 \\
            0 & 1 & 0 & 0 \\
            0 & 0 & 1 & 0 \\
            0 & 0 & 0 & 1
        \end{pmatrix} , \\
        %
        \begin{pmatrix}
            1 + 1 + 4 + 9 & -2 + 2 + 3 \\
            -2 + 2 + 3    & 4 + 1 + 1
        \end{pmatrix}
        \begin{pmatrix}
            c_{11} & c_{12} & c_{13} & c_{14} \\
            c_{21} & c_{22} & c_{23} & c_{14}
        \end{pmatrix}
        =
        \begin{pmatrix}
            -1 & 1 & 2 & 3 \\
            2  & 0 & 1 & 1
        \end{pmatrix} , \\
        %
        \begin{pmatrix}
            15 & 3 \\
            3  & 6
        \end{pmatrix}
        \begin{pmatrix}
            c_{11} & c_{12} & c_{13} & c_{14} \\
            c_{21} & c_{22} & c_{23} & c_{14}
        \end{pmatrix}
        =
        \begin{pmatrix}
            -1 & 1 & 2 & 3 \\
            2  & 0 & 1 & 1
        \end{pmatrix} , \\
        %
        \begin{pmatrix}
            15  & 3 \\
            -27 & 0
        \end{pmatrix}
        \begin{pmatrix}
            c_{11} & c_{12} & c_{13} & c_{14} \\
            c_{21} & c_{22} & c_{23} & c_{14}
        \end{pmatrix}
        =
        \begin{pmatrix}
            -1 & 1  & 2  & 3  \\
            4  & -2 & -3 & -5
        \end{pmatrix} , \\
        %
        \begin{pmatrix}
            c_{11} & c_{12} & c_{13} & c_{14} \\
            c_{21} & c_{22} & c_{23} & c_{14}
        \end{pmatrix}
        =
        \begin{pmatrix}
            -\frac{4}{27} & \frac{2}{27}   & \frac{3}{27} & \frac{5}{27} \\
            \frac{11}{27} & - \frac{1}{27} & \frac{3}{27} & \frac{2}{27}
        \end{pmatrix}
    \end{gather*}
    Матрица $A_1$ оператора $\mathcal{A}_1$ в стандартном базисе:
    \[
        A_1
        =
        \begin{pmatrix}
            -1 & 2 \\
            1  & 0 \\
            2  & 1 \\
            3  & 1
        \end{pmatrix}
        \begin{pmatrix}
            -4 & 2  & 3 & 5 \\
            11 & -1 & 3 & 2
        \end{pmatrix}
        \frac{1}{27}
        =
        \frac{1}{27}
        \begin{pmatrix}
            26 & -4 & 3  & -1 \\
            -4 & 2  & 3  & 5  \\
            3  & 3  & 9  & 12 \\
            -1 & 5  & 12 & 17
        \end{pmatrix}
    \]

    Аналогично найдем проекции векторов стандартного базиса на подпространство $V_2$:
    \begin{gather*}
        \begin{pmatrix}
            1 & 4 & -3 & 1  \\
            0 & 1 & 1  & -1
        \end{pmatrix}
        \begin{pmatrix}
            1  & 0  \\
            4  & 1  \\
            -3 & 1  \\
            1  & -1
        \end{pmatrix}
        \begin{pmatrix}
            c_{11} & c_{12} & c_{13} & c_{14} \\
            c_{21} & c_{22} & c_{23} & c_{14}
        \end{pmatrix}
        =
        \begin{pmatrix}
            1 & 4 & -3 & 1  \\
            0 & 1 & 1  & -1
        \end{pmatrix}
        \begin{pmatrix}
            1 & 0 & 0 & 0 \\
            0 & 1 & 0 & 0 \\
            0 & 0 & 1 & 0 \\
            0 & 0 & 0 & 1
        \end{pmatrix} , \\
        %
        \begin{pmatrix}
            1 + 16 + 9 + 1 & 4 - 3 + 1 \\
            4 - 3 + 1      & 1 + 1 + 1
        \end{pmatrix}
        \begin{pmatrix}
            c_{11} & c_{12} & c_{13} & c_{14} \\
            c_{21} & c_{22} & c_{23} & c_{14}
        \end{pmatrix}
        =
        \begin{pmatrix}
            1 & 4 & -3 & 1  \\
            0 & 1 & 1  & -1
        \end{pmatrix} , \\
        %
        \begin{pmatrix}
            27 & 0 \\
            0  & 3
        \end{pmatrix}
        \begin{pmatrix}
            c_{11} & c_{12} & c_{13} & c_{14} \\
            c_{21} & c_{22} & c_{23} & c_{14}
        \end{pmatrix}
        =
        \begin{pmatrix}
            1 & 4 & -3 & 1  \\
            0 & 1 & 1  & -1
        \end{pmatrix} , \\
        %
        \begin{pmatrix}
            c_{11} & c_{12} & c_{13} & c_{14} \\
            c_{21} & c_{22} & c_{23} & c_{14}
        \end{pmatrix}
        =
        \begin{pmatrix}
            \frac{1}{27} & \frac{4}{27} & - \frac{3}{27} & \frac{1}{27}  \\
            0            & \frac{1}{3}  & \frac{1}{3}    & - \frac{1}{3}
        \end{pmatrix} .
    \end{gather*}
    Матрица $A_2$ оператора $\mathcal{A}_2$ в стандартном базисе:
    \[
        A_2
        =
        \begin{pmatrix}
            1  & 0  \\
            4  & 1  \\
            -3 & 1  \\
            1  & -1
        \end{pmatrix}
        \begin{pmatrix}
            1 & 4 & -3 & 1  \\
            0 & 9 & 9  & -9
        \end{pmatrix}
        \frac{1}{27}
        =
        \frac{1}{27}
        \begin{pmatrix}
            1  & 4  & -3  & 1   \\
            4  & 25 & -3  & -5  \\
            -3 & -3 & 18  & -12 \\
            1  & -5 & -12 & 10
        \end{pmatrix}
    \]

    Матрицы $A_{12}$ оператора $\mathcal{A}_1 \circ \mathcal{A}_2$ и матрица $A_{21}$ $\mathcal{A}_2 \circ \mathcal{A}_1$ оказывается нулевыми:
    \[
        \frac{1}{27}
        \begin{pmatrix}
            26 & -4 & 3  & -1 \\
            -4 & 2  & 3  & 5  \\
            3  & 3  & 9  & 12 \\
            -1 & 5  & 12 & 17
        \end{pmatrix}
        \frac{1}{27}
        \begin{pmatrix}
            1  & 4  & -3  & 1   \\
            4  & 25 & -3  & -5  \\
            -3 & -3 & 18  & -12 \\
            1  & -5 & -12 & 10
        \end{pmatrix}
        = 0 ,
    \]
    потому что $V_1 \perp V_2$:
    \begin{gather*}
        \scalarproduct{a_1}{a_2} = -1 + 4 - 6 + 3 = 0, \\
        \scalarproduct{a_1}{b_2} = 1 + 2 - 3 = 0, \\
        \scalarproduct{b_1}{a_2} = 2 - 3 + 1 = 0, \\
        \scalarproduct{b_1}{b_2} = 1 - 1 = 0 .
    \end{gather*}

    \subsection*{Ответ:}
    Матрица оператора $\mathcal{A}_1$:
    $
    \frac{1}{27}
    \begin{pmatrix}
        26 & -4 & 3  & -1 \\
        -4 & 2  & 3  & 5  \\
        3  & 3  & 9  & 12 \\
        -1 & 5  & 12 & 17
    \end{pmatrix}
    $.

    Матрица оператора $\mathcal{A}_2$:
    $
    \frac{1}{27}
    \begin{pmatrix}
        1  & 4  & -3  & 1   \\
        4  & 25 & -3  & -5  \\
        -3 & -3 & 18  & -12 \\
        1  & -5 & -12 & 10
    \end{pmatrix}
    $.

    Матрицы операторов $\mathcal{A}_1 \circ \mathcal{A}_2$ и $\mathcal{A}_2 \circ \mathcal{A}_1$ нулевые.
\end{document}