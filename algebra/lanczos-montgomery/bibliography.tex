\begin{thebibliography}{4}
	\bibitem{Montgomery} Montgomery P. A block Lanczos algorithm for finding dependencies over GF(2)

	\bibitem{Zamarashkin} Замарашкин Н. Л. Алгоритмы для разреженных систем линейных уравнений в $GF(2)$ / Москва: Издательство Московского
		Университета, 2013.
\end{thebibliography}

