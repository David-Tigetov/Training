\documentclass[12pt]{article}

\usepackage[T1]{fontenc}
\usepackage[utf8]{inputenc}
\usepackage[english,russian]{babel}
\usepackage[margin=2cm]{geometry}
\usepackage{amsmath}
\usepackage{amsfonts}
\usepackage{xcolor}
\usepackage{color}
\usepackage{amssymb}

% команды вывода первой частной производной
\newcommand{\fpd}[1]{\frac{\partial}{\partial #1}}
\newcommand{\fpda}[2]{\frac{\partial #1}{\partial #2}}
\newcommand{\fpdp}[2]{\fpd{#2} \left ( #1 \right )}

\newcommand{\expectation}[1]{\mathtt{M} \left [ #1 \right ]}
\newcommand{\conditionalexpectation}[2]{\expectation{ #1 \left | #2 \right .}}
\newcommand{\variance}[1]{\mathtt{D} \left [ #1 \right ]}
\newcommand{\covariance}[2]{\mathtt{cov} \left ( #1, #2 \right )}

\newcommand{\modulus}[1]{\left | #1 \right |}
\newcommand{\norm}[1]{\left \| {#1} \right \|}

\newcommand{\event}[1]{\left \{ #1 \right \} }
\newcommand{\probability}[1]{P \event{#1}}


\begin{document}

    \title{Задача 12}
    \author{Тигетов Давид Георгиевич}
    \date{}
    \maketitle

    \section*{Пункт а}
    Найдите базисы в собственных подпространствах для оператора над $\mathbb{C}$, заданного матрицей:
    \[
        A =
        \begin{pmatrix}
            7  & -2 & -14 & 10 \\
            -1 & 0  & 4   & -5 \\
            6  & -2 & -13 & 10 \\
            3  & -1 & -6  & 4
        \end{pmatrix}
    \]

    \subsection*{Решение:}
    Собственные значения оператора $\lambda_1 = -1$, $\lambda_2 = 1$.

    Собственные векторы собственного значения $\lambda_1$:
    \begin{gather*}
        \left ( A - \lambda_1 E \right ) u_i = 0, \\
        u_1 = \begin{pmatrix}
                  1 \\ -3 \\ 1 \\ 0
        \end{pmatrix},
        u_2 = \begin{pmatrix}
                  0 \\ 5 \\ 0 \\ 1
        \end{pmatrix}
    \end{gather*}

    Собственные векторы собственного значения $\lambda_2$
    \begin{gather*}
        \left ( A - \lambda_2 E \right ) v_i = 0, \\
        v_1 = \begin{pmatrix}
                  3 \\ 0 \\ 2 \\ 1
        \end{pmatrix} .
    \end{gather*}

    \subsection*{Ответ:}
    Собственное подпространство $U = \left < u_1, u_2 \right >$:
    \[
        u_1 = \begin{pmatrix}
                  1 \\ -3 \\ 1 \\ 0
        \end{pmatrix},
        u_2 = \begin{pmatrix}
                  0 \\ 5 \\ 0 \\ 1
        \end{pmatrix}.
    \]

    Собственное подпространство $V = \left < v_1 \right >$:
    \[
        v_1 = \begin{pmatrix}
                  3 \\ 0 \\ 2 \\ 1
        \end{pmatrix} .
    \]

    \section*{Пункт б}
    Рассмотрим оператор, заданный матрицей
    \[
        \begin{pmatrix}
            1 & 1  & 1  & 1  \\
            1 & 1  & -1 & -1 \\
            1 & -1 & 1  & -1 \\
            1 & -1 & -1 & 1
        \end{pmatrix}.
    \]
    Найдите базис, в котором матрица оператора диагональна и укажите соответствующий диагональный вид.

    \subsection*{Решение:}
    Собственные значения оператора $\lambda_1 = -2$, $\lambda_2 = 2$.

    Собственные векторы собственного значения $\lambda_1$:
    \[
        u_1 = \begin{pmatrix}
                  -1 \\ 1 \\ 1 \\ 1
        \end{pmatrix}
        .
    \]

    Cобственные векторы собственного значения $\lambda_2$:
    \[
        v_1 = \begin{pmatrix}
                  1 \\ 1 \\ 0 \\ 0
        \end{pmatrix},
        v_2 = \begin{pmatrix}
                  1 \\ 0 \\ 1 \\ 0
        \end{pmatrix},
        v_3 = \begin{pmatrix}
                  1 \\ 0 \\ 0 \\ 1
        \end{pmatrix}.
    \]

    \subsection*{Ответ:}
    В базисе
    \[
        u_1 = \begin{pmatrix}
                  -1 \\ 1 \\ 1 \\ 1
        \end{pmatrix},
        v_1 = \begin{pmatrix}
                  1 \\ 1 \\ 0 \\ 0
        \end{pmatrix},
        v_2 = \begin{pmatrix}
                  1 \\ 0 \\ 1 \\ 0
        \end{pmatrix},
        v_3 = \begin{pmatrix}
                  1 \\ 0 \\ 0 \\ 1
        \end{pmatrix}
    \]
    матрица оператора имеет диагональный вид:
    \[
        \begin{pmatrix}
            -2 & 0 & 0 & 0 \\
            0  & 2 & 0 & 0 \\
            0  & 0 & 2 & 0 \\
            0  & 0 & 0 & 2
        \end{pmatrix}
        .
    \]

    \section*{Пункт в}
    Выясните, можно ли диагонализировать оператор над $\mathbb{R}$ и $\mathbb{C}$, заданным матрицей:
    \[
        \begin{pmatrix}
            4  & 7 & -5  \\
            -4 & 5 & 0   \\
            1  & 9 & - 4
        \end{pmatrix}
    \]
    Если нет, то объясните почему. Если да, то приведите базис, в котором матрица диагонализуется и соответствующий базис.

    \subsection*{Решение:}
    Собственные значения оператора $\lambda_1 = 1$, $\lambda_2 = 2 - 3i$, $\lambda_3 = 2 + 3i$.

    Есть комплесные собственные значения, поэтому на полем действительных чисел $\mathbb{R}$ матрица оператора не может быть приведена к диагональному виду.

    Тем не менее, матрица диагонализуема над полем комплексных чисел $\mathbb{C}$.

    Собственный вектор собственного значения $\lambda_1$:
    \[
        u_1 = \begin{pmatrix}
                  1 \\ 1 \\ 2
        \end{pmatrix}.
    \]

    Собственные векторы собственных значений $\lambda_2$ и $\lambda_3$ соответственно:
    \[
        v_1 = \begin{pmatrix}
                  12 + 3i \\ 10 - 6i \\ 17
        \end{pmatrix},
        v_2 = \begin{pmatrix}
                  12 - 3i \\ 10 + 6i \\ 17
        \end{pmatrix}.
    \]

    \subsection*{Ответ:}
    Над полем $\mathbb{R}$ матрица не может быть диагонализирована.

    Над полем $\mathbb{C}$ в базисе
    \[
        u_1 = \begin{pmatrix}
                  1 \\ 1 \\ 2
        \end{pmatrix},
        v_1 = \begin{pmatrix}
                  12 + 3i \\ 10 - 6i \\ 17
        \end{pmatrix},
        v_2 = \begin{pmatrix}
                  12 - 3i \\ 10 + 6i \\ 17
        \end{pmatrix}
    \]
    матрица оператора имеет диагональный вид:
    \[
        \begin{pmatrix}
            1 & 0      & 0      \\
            0 & 2 - 3i & 0      \\
            0 & 0      & 2 + 3i
        \end{pmatrix}
        .
    \]

    \section*{Пункт г}
    Приведите матрицу оператора $\mathcal{A}: \mathbb{C}^{\textcolor{magenta}{4}} \rightarrow \mathbb{C}^{\textcolor{magenta}{4}}$ к верхнетреугольному виду и найдите соответствующий базис
    \[
        \begin{pmatrix}
            -3 & -5 & 2  & 5  \\
            -2 & 0  & 1  & 2  \\
            6  & 3  & -4 & -8 \\
            -6 & -3 & 3  & 7
        \end{pmatrix}
        .
    \]

    \subsection*{Решение:}
    Собственные значения оператора $\lambda_1 = -1$ и $\lambda_2 = 1$.

    Собственный вектор собственного значения $\lambda_1$:
    \[
        u_1 = \begin{pmatrix}
                  2 \\ 1 \\ -3 \\ 3
        \end{pmatrix}
        .
    \]

    Собственный вектор собственного значения $\lambda_2$:
    \[
        v_1 = \begin{pmatrix}
                  1 \\ 1 \\ -3 \\ 3
        \end{pmatrix}
        .
    \]

    Собственных векторов не хватает для формирования базиса.

    Подпространство $\kernel{\left ( A - \lambda_1 E\right )^2}$ оказалось размерности 2, в дополнение к вектору $u_1$ можно взять вектор:
    \[
        u_2 = \begin{pmatrix}
                  0 \\ 0 \\ 1 \\ 0
        \end{pmatrix}
        .
    \]

    Подпространство $\kernel{\left ( A - \lambda_2 E\right )^2}$ также размерности 2, в дополнение к вектору $v_1$ можно взять вектор:
    \[
        v_2 = \begin{pmatrix}
                  -2 \\ 1 \\ 0 \\ 0
        \end{pmatrix}
        .
    \]

    Векторы $u_1$, $u_2$, $v_1$, $v_2$ линейно независимы:
    \[
        \begin{pmatrix}
            2  & 0 & 1  & -2 \\
            1  & 0 & 1  & 1  \\
            -3 & 1 & -3 & 0  \\
            3  & 0 & 3  & 0
        \end{pmatrix}
        \rightarrow
        \begin{pmatrix}
            2 & 0 & 1 & -2 \\
            1 & 0 & 1 & 1  \\
            0 & 1 & 0 & 0  \\
            3 & 0 & 3 & 0
        \end{pmatrix}
        \rightarrow
        \begin{pmatrix}
            1 & 0 & 1 & -2 \\
            0 & 0 & 1 & 1  \\
            0 & 1 & 0 & 0  \\
            0 & 0 & 3 & 0
        \end{pmatrix}
        \rightarrow
        \begin{pmatrix}
            1 & 0 & 0 & 0 \\
            0 & 0 & 1 & 1 \\
            0 & 1 & 0 & 0 \\
            0 & 0 & 3 & 0
        \end{pmatrix}
        \rightarrow
        \begin{pmatrix}
            1 & 0 & 0 & 0 \\
            0 & 0 & 0 & 1 \\
            0 & 1 & 0 & 0 \\
            0 & 0 & 3 & 0
        \end{pmatrix}
        .
    \]
    Итак, найден базис $u_1$, $u_2$, $v_1$, $v_2$.

    Найдём разложение вектора $\mathcal{A} u_2$ по векторам $u_1$, $u_2$:
    \begin{gather*}
        \left .
        \begin{pmatrix}
            2  & 0 \\
            1  & 0 \\
            -3 & 1 \\
            3  & 0
        \end{pmatrix}
        \right |
        \begin{pmatrix}
            A u_2
        \end{pmatrix}
        =
        \begin{pmatrix}
            2  \\
            1  \\
            -4 \\
            3
        \end{pmatrix} , \\
        %
        A u_2 = u_1 - u_2 .
    \end{gather*}

    Найдём разложение вектора $\mathcal{A} v_2$ по векторам $v_1$, $v_2$:
    \begin{gather*}
        \left .
        \begin{pmatrix}
            1  & -2 \\
            1  & 1  \\
            -3 & 0  \\
            3  & 0
        \end{pmatrix}
        \right |
        \begin{pmatrix}
            A v_2
        \end{pmatrix}
        =
        \begin{pmatrix}
            1  \\
            3  \\
            -9 \\
            9
        \end{pmatrix} , \\
        %
        A v_2 = 3 v_1 + v_2 .
    \end{gather*}

    Таким образом, в базисе $u_1$, $u_2$, $v_1$, $v_2$ матрица оператора $\mathcal{A}$ будет иметь вид:
    \[
        \begin{pmatrix}
            1 & 1  & 0 & 0 \\
            0 & -1 & 0 & 0 \\
            0 & 0  & 1 & 3 \\
            0 & 0  & 0 & 1
        \end{pmatrix}
    \]

    \subsection*{Ответ:}
    В базисе $u_1$, $u_2$, $v_1$, $v_2$:
    \[
        u_1 = \begin{pmatrix}
                  2 \\ 1 \\ -3 \\ 3
        \end{pmatrix},
        u_2 = \begin{pmatrix}
                  0 \\ 0 \\ 1 \\ 0
        \end{pmatrix},
        v_1 = \begin{pmatrix}
                  1 \\ 1 \\ -3 \\ 3
        \end{pmatrix},
        v_2 = \begin{pmatrix}
                  -2 \\ 1 \\ 0 \\ 0
        \end{pmatrix}.
    \]
    матрица оператора будет иметь верхне треугольный вид:
    \[
        \begin{pmatrix}
            1 & 1  & 0 & 0 \\
            0 & -1 & 0 & 0 \\
            0 & 0  & 1 & 3 \\
            0 & 0  & 0 & 1
        \end{pmatrix}
        .
    \]


\end{document}