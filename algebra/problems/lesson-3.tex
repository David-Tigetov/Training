\documentclass[12pt]{article}

\usepackage[T1]{fontenc}
\usepackage[utf8]{inputenc}
\usepackage[english,russian]{babel}
\usepackage[margin=2cm]{geometry}
\usepackage{amsmath}
\usepackage{amsfonts}
\usepackage{xcolor}
\usepackage{color}

\begin{document}

    \title{Задача 3}
    \author{Тигетов Давид Георгиевич}
    \date{}
    \maketitle

    \section*{Пункт г}
    $V_2 = \left < a, b, c \right >$:
    \[
        a =
        \begin{pmatrix}
            1 \\ -1 \\ 1 \\ -1 \\ 0 \\ 0
        \end{pmatrix},
        b =
        \begin{pmatrix}
            -5 \\ 1 \\ -1 \\ -1 \\ 2 \\ 1
        \end{pmatrix},
        c =
        \begin{pmatrix}
            2 \\ 2 \\ 0 \\ 2 \\ -1 \\ 0
        \end{pmatrix} .
    \]
    Проверьте, что $V_2$ содержится в $V$ и $a$, $b$, $c$ --- базис в $V_2$. Дополните $a$, $b$, $c$ до базиса в $V$:
    \[
        V: 2 x_1 + 2 x_2 - 5 x_3 - 5 x_4 - 2 x_5 + 2 x_6 = 0
    \]

    \subsection*{Решение:}
    \begin{enumerate}
        \item
        Проверим, что $V_2 \subseteq V$ --- подставим векторы $a$, $b$, $c$ в уравнение для $V$:
        \begin{align*}
            a: & \; 2 \cdot 1 + 2 \cdot (-1) - 5 \cdot 1 - 5 \cdot (-1) = 2 - 2 - 5 + 5 = 0, \\
            b: & \; 2 \cdot (-5) + 2 \cdot 1 - 5 \cdot (-1) - 5 \cdot (-1) - 2 \cdot 2 + 2 \cdot 1 = -10 + 2 + 5 + 5 - 4 + 2 = 0, \\
            c: & \; 2 \cdot 2 + 2 \cdot 2 - 5 \cdot 2 - 2 \cdot (-1) = 4 + 4 - 10 + 2 = 0 .
        \end{align*}

        \item
        Покажем, что $a$, $b$, $c$ --- базис в $V_2$: $V_2$ --- линейная оболочка, поэтому любой вектор в $V_2$ выражается через $a$, $b$, $c$, и эти векторы линейно независимы,
        это можно увидеть, если обратить внимание на последний и предпоследний компоненты векторов.

        \item
        Базис в $V$ образуют векторы ФСР, поэтому к векторам $a$, $b$, $c$ добавим векторы ФСР и уберем те векторы ФCР, которые линейно зависимы от $a$, $b$, $c$:
        \begin{gather*}
            \begin{pmatrix}
                -5 & 2  & 1  & 5 & 1  & 5 & 1 & 1  \\
                1  & 2  & -1 & 0 & -1 & 0 & 0 & 0  \\
                -1 & 0  & 1  & 2 & 0  & 0 & 0 & 0  \\
                -1 & 2  & -1 & 0 & 0  & 2 & 0 & 0  \\
                2  & -1 & 0  & 0 & 0  & 0 & 1 & 0  \\
                1  & 0  & 0  & 0 & 0  & 0 & 0 & -1
            \end{pmatrix}
            \rightarrow
            \begin{pmatrix}
                -5 & 2  & 1  & 5 & 1  & 7  & 3 & -4 \\
                1  & 2  & -1 & 0 & -1 & -2 & 2 & 1  \\
                -1 & 0  & 1  & 2 & 0  & 2  & 0 & -1 \\
                -1 & 2  & -1 & 0 & 0  & 0  & 2 & -1 \\
                2  & -1 & 0  & 0 & 0  & 0  & 0 & 2  \\
                1  & 0  & 0  & 0 & 0  & 0  & 0 & 0
            \end{pmatrix}
            \rightarrow \\
            %
            \rightarrow
            \begin{pmatrix}
                -5 & 2  & 1  & 5 & 1  & 2  & 5                   & \textcolor{blue}{0}  \\
                1  & 2  & -1 & 0 & -1 & -2 & 0                   & \textcolor{blue}{5}  \\
                -1 & 0  & 1  & 2 & 0  & 0  & \textcolor{blue}{2} & \textcolor{blue}{-1} \\
                -1 & 2  & -1 & 0 & 0  & 0  & 0                   & \textcolor{blue}{3}  \\
                2  & -1 & 0  & 0 & 0  & 0  & 0                   & \textcolor{blue}{0}  \\
                1  & 0  & 0  & 0 & 0  & 0  & 0                   & 0
            \end{pmatrix}
            \rightarrow
            \begin{pmatrix}
                -5 & 2  & 1  & 5 & 1  & 2  & 5                   & \textcolor{blue}{3} \\
                1  & 2  & -1 & 0 & -1 & -2 & 0                   & \textcolor{blue}{2} \\
                -1 & 0  & 1  & 2 & 0  & 0  & \textcolor{blue}{2} & \textcolor{blue}{2} \\
                -1 & 2  & -1 & 0 & 0  & 0  & 0                   & \textcolor{blue}{0} \\
                2  & -1 & 0  & 0 & 0  & 0  & 0                   & \textcolor{blue}{0} \\
                1  & 0  & 0  & 0 & 0  & 0  & 0                   & 0
            \end{pmatrix}
            \rightarrow \\
            %
            \begin{pmatrix}
                -5 & 2  & 1  & 5 & 1  & 2  & 5                   & \textcolor{blue}{-2} \\
                1  & 2  & -1 & 0 & -1 & -2 & 0                   & \textcolor{blue}{2}  \\
                -1 & 0  & 1  & 2 & 0  & 0  & \textcolor{blue}{2} & \textcolor{blue}{0}  \\
                -1 & 2  & -1 & 0 & 0  & 0  & 0                   & \textcolor{blue}{0}  \\
                2  & -1 & 0  & 0 & 0  & 0  & 0                   & \textcolor{blue}{0}  \\
                1  & 0  & 0  & 0 & 0  & 0  & 0                   & 0
            \end{pmatrix}
            \rightarrow
            \begin{pmatrix}
                -5 & 2  & 1  & 5 & 1  \\
                1  & 2  & -1 & 0 & -1 \\
                -1 & 0  & 1  & 2 & 0  \\
                -1 & 2  & -1 & 0 & 0  \\
                2  & -1 & 0  & 0 & 0  \\
                1  & 0  & 0  & 0 & 0
            \end{pmatrix}
        \end{gather*}
        Векторы-столбцы последней матрицы образуют базис в пространстве $V$.
    \end{enumerate}

    \subsection*{Ответ:}
    Базис $V$
    \[
        \color{blue}
        a, b, c ,
        \begin{pmatrix}
            5 \\ 0 \\ 2 \\ 0 \\ 0 \\ 0
        \end{pmatrix} ,
        \begin{pmatrix}
            1 \\ -1 \\ 0 \\ 0 \\ 0 \\ 0
        \end{pmatrix} .
    \]

    \section*{Пункт а \textcolor{red}{[верно]}}
    Построить базис суммы и пересечения линейных подпространств:
    \[
        V_1 :
        \left \{
        \begin{array}{rcl}
            x_2 + x_3 - x_4 - x_6                         & = & 0 \\
            2 x_1 + 2 x_2 - 5 x_3 - 5 x_4 - 2 x_5 + 2 x_6 & = & 0
        \end{array}
        \right .
    \]
    и $V_2 = \left < a, b, c \right >$:
    \[
        a =
        \begin{pmatrix}
            1 \\ -1 \\ 1 \\ -1 \\ 0 \\ 0
        \end{pmatrix},
        b =
        \begin{pmatrix}
            -5 \\ 1 \\ -1 \\ -1 \\ 2 \\ 1
        \end{pmatrix},
        c =
        \begin{pmatrix}
            2 \\ 2 \\ 0 \\ 2 \\ -1 \\ 0
        \end{pmatrix} .
    \]

    \subsection*{Решение:}
    \begin{enumerate}
        \item
        Из пункта г известно, что векторы $a$, $b$ и $c$ образуют базис в $V_2$, поэтому добавим к ним векторы ФСР системы для $V_1$ --- базис в $V_1$ --- и уберем линейно зависимые:
        \begin{gather*}
            \begin{pmatrix}
                -5 & 2  & 1  & 7  & 3 & 1 & -4 \\
                1  & 2  & -1 & -2 & 2 & 0 & 2  \\
                -1 & 0  & 1  & 2  & 0 & 0 & 0  \\
                -1 & 2  & -1 & 0  & 2 & 0 & 0  \\
                2  & -1 & 0  & 0  & 0 & 1 & 0  \\
                1  & 0  & 0  & 0  & 0 & 0 & 2
            \end{pmatrix}
            \rightarrow
            \begin{pmatrix}
                -5 & 2  & 1  & 7  & 5 & 3 & 6  \\
                1  & 2  & -1 & -2 & 0 & 2 & 0  \\
                -1 & 0  & 1  & 2  & 2 & 0 & 2  \\
                -1 & 2  & -1 & 0  & 0 & 2 & 2  \\
                2  & -1 & 0  & 0  & 0 & 0 & -4 \\
                1  & 0  & 0  & 0  & 0 & 0 & 0
            \end{pmatrix}
            \rightarrow \\
            %
            \rightarrow
            \begin{pmatrix}
                -5 & 2  & 1  & 7  & -2 & 5 & -2 \\
                1  & 2  & -1 & -2 & 2  & 0 & -8 \\
                -1 & 0  & 1  & 2  & 0  & 2 & 2  \\
                -1 & 2  & -1 & 0  & 0  & 0 & -6 \\
                2  & -1 & 0  & 0  & 0  & 0 & 0  \\
                1  & 0  & 0  & 0  & 0  & 0 & 0
            \end{pmatrix}
            \rightarrow
            \begin{pmatrix}
                -5 & 2  & 1  & 7  & -2 & -2 & -8 \\
                1  & 2  & -1 & -2 & 2  & 2  & -2 \\
                -1 & 0  & 1  & 2  & 0  & 0  & -4 \\
                -1 & 2  & -1 & 0  & 0  & 0  & 0  \\
                2  & -1 & 0  & 0  & 0  & 0  & 0  \\
                1  & 0  & 0  & 0  & 0  & 0  & 0
            \end{pmatrix}
            \rightarrow \\
            %
            \rightarrow
            \begin{pmatrix}
                -5 & 2  & 1  & 7  & -2 & 0 & 6  \\
                1  & 2  & -1 & -2 & 2  & 0 & -6 \\
                -1 & 0  & 1  & 2  & 0  & 0 & 0  \\
                -1 & 2  & -1 & 0  & 0  & 0 & 0  \\
                2  & -1 & 0  & 0  & 0  & 0 & 0  \\
                1  & 0  & 0  & 0  & 0  & 0 & 0
            \end{pmatrix}
            \rightarrow
            \begin{pmatrix}
                -5 & 2  & 1  & 7  & -2 \\
                1  & 2  & -1 & -2 & 2  \\
                -1 & 0  & 1  & 2  & 0  \\
                -1 & 2  & -1 & 0  & 0  \\
                2  & -1 & 0  & 0  & 0  \\
                1  & 0  & 0  & 0  & 0
            \end{pmatrix}
        \end{gather*}
        Векторы-столбы последней матрицы образуют базис суммы $V_1 + V_2$.

        \item Поскольку $V_1 \cap V_2 \subseteq V_2$, то достаточно среди базисных векторов $a$, $b$, $c$ оболочки $V_2$ проверить какие из них удовлетворяют системе, задающей $V_1$.
        Из пункта г известно, что все три вектора удовлетворяют второму уравнению, поэтому проверяем первое уравнение:
        \begin{align*}
            a: & \; -1 - 1 - (-1) \neq 0 , \\
            b: & \; 1 -1 - (-1) - 1 = 0 , \\
            c: & \; 2 - 2 = 0
        \end{align*}
        Таким образом, векторы $b$, $c$ образуют базис в $V_1 \cap V_2$.
    \end{enumerate}

    \subsection*{Ответ:}
    \begin{enumerate}
        \item
        Базис $V_1 + V_2$
        \[
            a, b, c ,
            \begin{pmatrix}
                7 \\ -2 \\ 2 \\ 0 \\ 0 \\ 0
            \end{pmatrix} ,
            \begin{pmatrix}
                -2 \\ 2 \\ 0 \\ 0 \\ 0 \\ 0
            \end{pmatrix} .
        \]

        \item
        Базис $V_1 \cap V_2$ --- векторы $b$ и $c$.
    \end{enumerate}

    \section*{Пункт б \textcolor{red}{[верно]}}
    Найдите базис суммы и пересечения двух векторных подпространств $U_1 = \left < a_1, b_1, c_1 \right >$, $U_2 = \left < a_2, b_2, c_2 \right >$.
    \[
        a_1 =
        \begin{pmatrix}
            1 \\ 1 \\ -1 \\ 1 \\ 1
        \end{pmatrix},
        b_1 =
        \begin{pmatrix}
            0 \\ 1 \\ -1 \\ 1 \\ 0
        \end{pmatrix},
        c_1 =
        \begin{pmatrix}
            1 \\ 2 \\ -3 \\ 2 \\ 0
        \end{pmatrix},
        a_2 =
        \begin{pmatrix}
            1 \\ 0 \\ -2 \\ 1 \\ 1
        \end{pmatrix},
        b_2 =
        \begin{pmatrix}
            1 \\ 1 \\ -2 \\ 1 \\ 0
        \end{pmatrix},
        c_2 =
        \begin{pmatrix}
            2 \\ 1 \\ 0 \\ 0 \\ 1
        \end{pmatrix}.
    \]

    \subsection*{Решение:}
    \begin{enumerate}
        \item
        Найдем базис суммы $U_1 + U_2$:
        \begin{gather*}
            \begin{pmatrix}
                1  & 0  & 1  & 1  & 1  & 2 \\
                1  & 1  & 2  & 0  & 1  & 1 \\
                -1 & -1 & -3 & -2 & -2 & 0 \\
                1  & 1  & 2  & 1  & 1  & 0 \\
                1  & 0  & 0  & 1  & 0  & 1
            \end{pmatrix}
            \rightarrow
            \begin{pmatrix}
                -1 & 0  & 1  & -1 & 1  & 2 \\
                0  & 1  & 2  & -1 & 1  & 1 \\
                -1 & -1 & -3 & -2 & -2 & 0 \\
                1  & 1  & 2  & 1  & 1  & 0 \\
                0  & 0  & 0  & 0  & 0  & 1
            \end{pmatrix}
            \rightarrow \\
            %
            \rightarrow
            \begin{pmatrix}
                -1 & 0  & 1  & -1 & 1  & 2 \\
                -1 & 1  & 0  & -2 & 0  & 1 \\
                0  & -1 & -1 & -1 & -1 & 0 \\
                0  & 1  & 0  & 0  & 0  & 0 \\
                0  & 0  & 0  & 0  & 0  & 1
            \end{pmatrix}
            \rightarrow
            \begin{pmatrix}
                -1 & 0  & 1  & -2 & 0 & 2 \\
                -1 & 1  & 0  & -2 & 0 & 1 \\
                0  & -1 & -1 & 0  & 0 & 0 \\
                0  & 1  & 0  & 0  & 0 & 0 \\
                0  & 0  & 0  & 0  & 0 & 1
            \end{pmatrix}
            \rightarrow \\
            %
            \rightarrow
            \begin{pmatrix}
                -1 & 0  & 1  & & 2 \\
                -1 & 1  & 0  & & 1 \\
                0  & -1 & -1 & & 0 \\
                0  & 1  & 0  & & 0 \\
                0  & 0  & 0  & & 1
            \end{pmatrix}
        \end{gather*}
        Векторы-столбцы последней матрицы образуют базис в пространстве $U_1 + U_2$.

        \item
        Найдем базис пересечения $U_1 \cap U_2$:
        \begin{gather*}
            \begin{pmatrix}
                1  & 0  & 1  \\
                1  & 1  & 2  \\
                -1 & -1 & -3 \\
                1  & 1  & 2  \\
                1  & 0  & 0
            \end{pmatrix}
            \begin{pmatrix}
                x_1 \\ x_2 \\ x_3
            \end{pmatrix}
            =
            \begin{pmatrix}
                1  & 1  & 2 \\
                0  & 1  & 1 \\
                -2 & -2 & 0 \\
                1  & 1  & 0 \\
                1  & 0  & 1
            \end{pmatrix}
            \begin{pmatrix}
                y_1 \\ y_2 \\ y_3
            \end{pmatrix} , \\
            %
            \begin{pmatrix}
                1 & 0 & 1  \\
                1 & 1 & 2  \\
                0 & 0 & -1 \\
                0 & 0 & 0  \\
                1 & 0 & 0
            \end{pmatrix}
            \begin{pmatrix}
                x_1 \\ x_2 \\ x_3
            \end{pmatrix}
            =
            \begin{pmatrix}
                1  & 1  & 2  \\
                0  & 1  & 1  \\
                -2 & -1 & 1  \\
                1  & 0  & -1 \\
                1  & 0  & 1
            \end{pmatrix}
            \begin{pmatrix}
                y_1 \\ y_2 \\ y_3
            \end{pmatrix} , \\
            %
            \begin{pmatrix}
                0 & 0 & 1  \\
                0 & 1 & 2  \\
                0 & 0 & -1 \\
                0 & 0 & 0  \\
                1 & 0 & 0
            \end{pmatrix}
            \begin{pmatrix}
                x_1 \\ x_2 \\ x_3
            \end{pmatrix}
            =
            \begin{pmatrix}
                0  & 1  & 1  \\
                -1 & 1  & 0  \\
                -2 & -1 & 1  \\
                1  & 0  & -1 \\
                1  & 0  & 1
            \end{pmatrix}
            \begin{pmatrix}
                y_1 \\ y_2 \\ y_3
            \end{pmatrix} , \\
            %
            \begin{pmatrix}
                0 & 0 & 0  \\
                0 & 1 & 0  \\
                0 & 0 & -1 \\
                0 & 0 & 0  \\
                1 & 0 & 0
            \end{pmatrix}
            \begin{pmatrix}
                x_1 \\ x_2 \\ x_3
            \end{pmatrix}
            =
            \begin{pmatrix}
                -2 & 0  & 2  \\
                -5 & -1 & 2  \\
                -2 & -1 & 1  \\
                1  & 0  & -1 \\
                1  & 0  & 1
            \end{pmatrix}
            \begin{pmatrix}
                y_1 \\ y_2 \\ y_3
            \end{pmatrix} , \\
            %
            \begin{pmatrix}
                0 & 0 & 0  \\
                0 & 1 & 0  \\
                0 & 0 & -1 \\
                0 & 0 & 0  \\
                1 & 0 & 0
            \end{pmatrix}
            \begin{pmatrix}
                x_1 \\ x_2 \\ x_3
            \end{pmatrix}
            =
            \begin{pmatrix}
                0  & 0  & 0  \\
                -5 & -1 & 2  \\
                -2 & -1 & 1  \\
                1  & 0  & -1 \\
                1  & 0  & 1
            \end{pmatrix}
            \begin{pmatrix}
                y_1 \\ y_2 \\ y_3
            \end{pmatrix} , \\
            %
            \begin{pmatrix}
                1 & 0 & 0 \\
                0 & 1 & 0 \\
                0 & 0 & 1 \\
                0 & 0 & 0 \\
            \end{pmatrix}
            \begin{pmatrix}
                x_1 \\ x_2 \\ x_3
            \end{pmatrix}
            =
            \begin{pmatrix}
                1  & 0  & 1  \\
                -5 & -1 & 2  \\
                2  & 1  & -1 \\
                1  & 0  & -1 \\
            \end{pmatrix}
            \begin{pmatrix}
                y_1 \\ y_2 \\ y_3
            \end{pmatrix} , \\
        \end{gather*}

        Получили два вектора ФСР:
        \begin{gather*}
            \begin{array}{c}
                % x
                \begin{pmatrix}
                    x_1 \\ x_2 \\ x_3
                \end{pmatrix}
                = \begin{pmatrix}
                      0 \\ -1 \\ 1
                \end{pmatrix}             \\
                % y
                \begin{pmatrix}
                    y_1 \\ y_2 \\ y_3
                \end{pmatrix}
                = \begin{pmatrix}
                      0 \\ 1 \\ 0
                \end{pmatrix}
            \end{array},
            \begin{array}{c}
                % x
                \begin{pmatrix}
                    x_1 \\ x_2 \\ x_3
                \end{pmatrix}
                = \begin{pmatrix}
                      2 \\ -3 \\ 1
                \end{pmatrix}             \\
                % y
                \begin{pmatrix}
                    y_1 \\ y_2 \\ y_3
                \end{pmatrix}
                = \begin{pmatrix}
                      1 \\ 0 \\ 1
                \end{pmatrix}
            \end{array}
        \end{gather*}
        Значит базис образуют два вектора:
        \[
            b_2 =
            \begin{pmatrix}
                1 \\ 1 \\ -2 \\ 1 \\ 0
            \end{pmatrix} ,
            a_2 + c_2 =
            \begin{pmatrix}
                3 \\ 1 \\ -2 \\ 1 \\ 2
            \end{pmatrix} .
        \]
    \end{enumerate}

    \subsection*{Ответ:}
    \begin{enumerate}
        \item
        Базис $U_1 + U_2$
        \[
            \begin{pmatrix}
                -1 & 0  & 1  & & 2 \\
                -1 & 1  & 0  & & 1 \\
                0  & -1 & -1 & & 0 \\
                0  & 1  & 0  & & 0 \\
                0  & 0  & 0  & & 1
            \end{pmatrix}
        \]
        \item
        Базис $U_1 \cap U_2$
        \[
            \begin{pmatrix}
                1  & 3  \\
                1  & 1  \\
                -2 & -2 \\
                1  & 1  \\
                0  & 2
            \end{pmatrix}
        \]
    \end{enumerate}

    \section*{Пункт в \textcolor{red}{[верно]}}
    По подпространствам какой размерности могут пересекаться 6-мерное и 16-мерное подпространства в $\mathbb{C}^{20}$.

    \subsection*{Решение:}
    6-мерное пространство может полностью лежать внутри 16-мерного пространства. Можно "вынести"{} один вектора базиса 6-мерного пространства из 16-мерного. Продолжая, можно "вынести"{}
    два, три и четыре вектора базиса 6-мерного пространства, но больше "вынести"{} нельзя, потому что размерность пространства $\mathbb{C}^{20}$ всего 20.

    \subsection*{Ответ:}
    По подпространствам размерностей 2, 3, 4, 5, 6.

\end{document}