\documentclass[12pt]{article}

\usepackage[T1]{fontenc}
\usepackage[utf8]{inputenc}
\usepackage[english,russian]{babel}
\usepackage[margin=2cm]{geometry}
\usepackage{amsmath}
\usepackage{amsfonts}
\usepackage{xcolor}
\usepackage{color}

% команды вывода первой частной производной
\newcommand{\fpd}[1]{\frac{\partial}{\partial #1}}
\newcommand{\fpda}[2]{\frac{\partial #1}{\partial #2}}
\newcommand{\fpdp}[2]{\fpd{#2} \left ( #1 \right )}

\newcommand{\expectation}[1]{\mathtt{M} \left [ #1 \right ]}
\newcommand{\conditionalexpectation}[2]{\expectation{ #1 \left | #2 \right .}}
\newcommand{\variance}[1]{\mathtt{D} \left [ #1 \right ]}
\newcommand{\covariance}[2]{\mathtt{cov} \left ( #1, #2 \right )}

\newcommand{\modulus}[1]{\left | #1 \right |}
\newcommand{\norm}[1]{\left \| {#1} \right \|}

\newcommand{\event}[1]{\left \{ #1 \right \} }
\newcommand{\probability}[1]{P \event{#1}}


\begin{document}

    \title{Задача 9}
    \author{Тигетов Давид Георгиевич}
    \date{}
    \maketitle

    \section*{Пункт а}
    Пусть $V = \mathbb{R}^3$ со стандартным скалярным произведением, а $W \in V$ является линейной оболочкой вектора $a$:
    \[
        a = \begin{pmatrix}
                1 \\ 2 \\ 3
        \end{pmatrix} .
    \]
    Рассмотрим отображение $\mathcal{A}: V \rightarrow W$, заданное формулой $\mathcal{A}(x) = \projection{W}{x}$. Докажите, что $\mathcal{A}$ --- линейное отображение и найдите его
    матрицу $A$ по отношению к базисам $\epsilon$, $\varphi$ пространств $V$ и $W$ соответственно, где:
    \begin{enumerate}
        \item $\epsilon$ стандартный базис в $\mathbb{R}^3$, $\varphi = \set{a}$,
        \item $\epsilon = \set{b_1, b_2, b_3}$, $\varphi = \set{a}$, где
        \[
            b_1 = \begin{pmatrix}
                      1 \\ 0 \\ 1
            \end{pmatrix} ,
            b_2 = \begin{pmatrix}
                      2 \\ 0 \\ -1
            \end{pmatrix} ,
            b_3 = \begin{pmatrix}
                      1 \\ 1 \\ 0
            \end{pmatrix} .
        \]
    \end{enumerate}

    \subsection*{Решение:}
    По свойству проекции для любых векторов $x$ и $y$ и постоянной $\lambda$:
    \begin{gather*}
        \mathcal{A}(x + y)
        = \projection{W}{(x + y)}
        = \projection{W}{x} + \projection{W}{y}
        = \mathcal{A}(x) + \mathcal{A}(y), \\
        %
        \mathcal{A}(\lambda x)
        = \projection{W}{\lambda x}
        = \lambda \projection{W}{x}
        = \lambda \mathcal{A}(x) .
    \end{gather*}
    Это означает, что $\mathcal{A}$ --- линейное отображение.

    Отображение $\mathcal{A}$ можно представить в виде:
    \[
        \mathcal{A}(x)
        = \frac{\scalarproduct{x}{a}}{\scalarproduct{a}{a}} a ,
    \]
    где
    \[
        \scalarproduct{a}{a} = 1 + 4 + 9 = 14.
    \]
    Действие отображения $\mathcal{A}$ на векторы стандартного базиса:
    \begin{gather*}
        \mathcal{A}(e_1)
        = \frac{1}{14} a, \\
        %
        \mathcal{A}(e_2)
        = \frac{2}{14} a, \\
        %
        \mathcal{A}(e_3)
        = \frac{3}{14} a,
    \end{gather*}
    откуда матрица $A$
    \[
        A =
        \begin{pmatrix}
            \frac{1}{14} & \frac{2}{14} & \frac{3}{14}
        \end{pmatrix}
    \]

    Действие отображения $\mathcal{A}$ на векторы $b_i$:
    \begin{gather*}
        \mathcal{A}(b_1)
        = \frac{1 + 3}{14} a, \\
        %
        \mathcal{A}(b_2)
        = \frac{2 - 6}{14} a, \\
        %
        \mathcal{A}(b_3)
        = \frac{1 + 2}{14} a,
    \end{gather*}
    откуда матрица $A$
    \[
        A =
        \begin{pmatrix}
            \frac{4}{14} & - \frac{4}{14} & \frac{3}{14}
        \end{pmatrix}
        .
    \]

\end{document}