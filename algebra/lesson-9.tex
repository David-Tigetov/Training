\documentclass[12pt]{article}

\usepackage[T1]{fontenc}
\usepackage[utf8]{inputenc}
\usepackage[english,russian]{babel}
\usepackage[margin=2cm]{geometry}
\usepackage{amsmath}
\usepackage{amsfonts}
\usepackage{xcolor}
\usepackage{color}

% команды вывода первой частной производной
\newcommand{\fpd}[1]{\frac{\partial}{\partial #1}}
\newcommand{\fpda}[2]{\frac{\partial #1}{\partial #2}}
\newcommand{\fpdp}[2]{\fpd{#2} \left ( #1 \right )}

\newcommand{\expectation}[1]{\mathtt{M} \left [ #1 \right ]}
\newcommand{\conditionalexpectation}[2]{\expectation{ #1 \left | #2 \right .}}
\newcommand{\variance}[1]{\mathtt{D} \left [ #1 \right ]}
\newcommand{\covariance}[2]{\mathtt{cov} \left ( #1, #2 \right )}

\newcommand{\modulus}[1]{\left | #1 \right |}
\newcommand{\norm}[1]{\left \| {#1} \right \|}

\newcommand{\event}[1]{\left \{ #1 \right \} }
\newcommand{\probability}[1]{P \event{#1}}


\begin{document}

    \title{Задача 9}
    \author{Тигетов Давид Георгиевич}
    \date{}
    \maketitle

    \section*{Пункт а}
    Пусть $V = \mathbb{R}^3$ со стандартным скалярным произведением, а $W \in V$ является линейной оболочкой вектора $a$:
    \[
        a = \begin{pmatrix}
                1 \\ 2 \\ 3
        \end{pmatrix} .
    \]
    Рассмотрим отображение $\mathcal{A}: V \rightarrow W$, заданное формулой $\mathcal{A}(x) = \projection{W}{x}$. Докажите, что $\mathcal{A}$ --- линейное отображение и найдите его
    матрицу $A$ по отношению к базисам $\epsilon$, $\varphi$ пространств $V$ и $W$ соответственно, где:
    \begin{enumerate}
        \item $\epsilon$ стандартный базис в $\mathbb{R}^3$, $\varphi = \set{a}$,
        \item $\epsilon = \set{b_1, b_2, b_3}$, $\varphi = \set{a}$, где
        \[
            b_1 = \begin{pmatrix}
                      1 \\ 0 \\ 1
            \end{pmatrix} ,
            b_2 = \begin{pmatrix}
                      2 \\ 0 \\ -1
            \end{pmatrix} ,
            b_3 = \begin{pmatrix}
                      1 \\ 1 \\ 0
            \end{pmatrix} .
        \]
    \end{enumerate}

    \subsection*{Решение:}
    Отображение $\mathcal{A}$ можно представить в виде:
    \[
        \mathcal{A}(x)
        = \frac{\scalarproduct{x}{a}}{\scalarproduct{a}{a}} a ,
    \]
    откуда следует, что для любых векторов $x$ и $y$ и постоянной $\lambda$:
    \begin{gather*}
        \mathcal{A}(x + y)
        = \frac{\scalarproduct{x + y}{a}}{\scalarproduct{a}{a}} a
        = \frac{\scalarproduct{x}{a} + \scalarproduct{y}{a}}{\scalarproduct{a}{a}} a
        = \frac{\scalarproduct{x}{a}}{\scalarproduct{a}{a}} a + \frac{\scalarproduct{y}{a}}{\scalarproduct{a}{a}} a
        = \mathcal{A}(x) + \mathcal{A}(y), \\
        %
        \mathcal{A}(\lambda x)
        = \frac{\scalarproduct{\lambda x}{a}}{\scalarproduct{a}{a}} a
        = \frac{\lambda  \scalarproduct{x}{a}}{\scalarproduct{a}{a}} a
        = \lambda \mathcal{A}(x) .
    \end{gather*}
    Это означает, что $\mathcal{A}$ --- линейное отображение.

    Заметим, что
    \[
        \scalarproduct{a}{a} = 1 + 4 + 9 = 14,
    \]
    и действие отображения $\mathcal{A}$ на векторы стандартного базиса:
    \begin{gather*}
        \mathcal{A}(e_1)
        = \frac{1}{14} a, \\
        %
        \mathcal{A}(e_2)
        = \frac{2}{14} a, \\
        %
        \mathcal{A}(e_3)
        = \frac{3}{14} a,
    \end{gather*}
    откуда матрица $A$
    \[
        A =
        \begin{pmatrix}
            \frac{1}{14} & \frac{2}{14} & \frac{3}{14}
        \end{pmatrix}.
    \]

    Действие отображения $\mathcal{A}$ на векторы $b_i$:
    \begin{gather*}
        \mathcal{A}(b_1)
        = \frac{1 + 3}{14} a, \\
        %
        \mathcal{A}(b_2)
        = \frac{2 - 3}{14} a, \\
        %
        \mathcal{A}(b_3)
        = \frac{1 + 2}{14} a,
    \end{gather*}
    откуда матрица $A$
    \[
        A =
        \begin{pmatrix}
            \frac{4}{14} & - \frac{1}{14} & \frac{3}{14}
        \end{pmatrix}
        .
    \]

    \subsection*{Ответ:}
    \begin{enumerate}
        \item $\epsilon$ стандартный базис в $\mathbb{R}^3$, $\varphi = \set{a}$: матрица
        $A =
        \begin{pmatrix}
            \frac{1}{14} & \frac{2}{14} & \frac{3}{14}
        \end{pmatrix}$,

        \item $\epsilon = \set{b_1, b_2, b_3}$, $\varphi = \set{a}$: матрица
        $A =
        \begin{pmatrix}
            \frac{4}{14} & - \frac{1}{14} & \frac{3}{14}
        \end{pmatrix}$.
    \end{enumerate}

    \section*{Пункт б}
    Пусть линейное отображение $\mathcal{A}: V \rightarrow W$ в базисах $\set{e_1, e_2, e_3}$ и $\set{f_1, f_2}$ пространств $V$ и $W$ соответственно имеет матрицу
    \[
        A = \begin{pmatrix}
                0 & 1 & 2 \\
                3 & 4 & 5
        \end{pmatrix} .
    \]
    Найти матрицу $A^\prime$ отображения $\mathcal{A}$ в базисах $\set{e_1, e_1 + e_2, e_1 + e_2 + e_3}$ и $\set{f_1, f_1 + f_2}$.

    \subsection*{Решение:}
    В соответствии с матрицей $A$:
    \begin{align*}
        \mathcal{A}(e_1) &
        = 3 f_2
        = - 3 f_1 + 3 ( f_1 + f_2 ), \\
        %
        \mathcal{A}(e_2) &
        = f_1 + 4 f_2
        = f_1 - 4 f_1 + 4 ( f_1 + f_2 )
        = -3 f_1 + 4 ( f_1 + f_2 ), \\
        %
        \mathcal{A}(e_3) &
        = 2 f_1 + 5 f_2
        = 2 f_1 - 5 f_1 + 5 ( f_1 + f_2 )
        = -3 f_1 + 5 ( f_1 + f_2 ) .
    \end{align*}
    Откуда
    \begin{align*}
        \mathcal{A}(e_1 + e_2) &
        = \mathcal{A}(e_1) + \mathcal{A}(e_2)
        = - 3 f_1 + 3 ( f_1 + f_2 ) - 3 f_1 + 4 ( f_1 + f_2 )
        = - 6 f_1 + 7 ( f_1 + f_2 ), \\
        %
        \mathcal{A}(e_1 + e_2 + e_3) &
        = \mathcal{A}(e_1 + e_2) + \mathcal{A}(e_3)
        = - 6 f_1 + 7 ( f_1 + f_2 ) - 3 f_1 + 5 ( f_1 + f_2 )
        = - 9 f_1 + 12 ( f_1 + f_2 ) .
    \end{align*}
    Таким образом,
    \[
        A^{\prime}
        =
        \begin{pmatrix}
            -3 & -6 & -9 \\
            3  & 7  & 12
        \end{pmatrix}
        .
    \]

    \subsection*{Ответ:}
    $
    A^{\prime}
    =
    \begin{pmatrix}
        -3 & -6 & -9 \\
        3  & 7  & 12
    \end{pmatrix}
    $.

    \section*{Пункт в}
    Пусть $V = \mathbb{R}^3$ со стандартным произведением, а $W \in V$ и является линейной оболочкой векторов $a_1$ и $a_2$:
    \[
        a_1 = \begin{pmatrix}
                  1 \\ 0 \\ -1
        \end{pmatrix},
        a_2 = \begin{pmatrix}
                  1 \\ 0 \\ 1
        \end{pmatrix}
        .
    \]
    Рассмотрим отображение $\mathcal{A}: V \rightarrow W$, заданное формулой $\mathcal{A}(x) = \projection{W}{x}$.

    Докажите, что $\mathcal{A}$ --- линейное отображение и найдите его матрицу по отношению к базисам $\epsilon$, $\varphi$ пространств $V$ и $W$ соответственно, где $\epsilon$ ---
    стандартный базис в $\mathbb{R}^3$.

    \subsection*{Решение:}
    К счастью векторы $a_1$ и $a_2$ ортогональны:
    \[
        \scalarproduct{a_1}{a_2} = 1 + 0 - 1 = 0,
    \]
    что существенно упрощает вычисление проекции:
    \begin{gather*}
        \mathcal{A}(x)
        = \frac{\scalarproduct{x}{a_1}}{\scalarproduct{a_1}{a_1}} a_1 + \frac{\scalarproduct{x}{a_2}}{\scalarproduct{a_2}{a_2}} a_2
        = \mathcal{A}_1(x) + \mathcal{A}_2(x) , \\
        %
        \mathcal{A}_i(x) = \frac{\scalarproduct{x}{a_i}}{\scalarproduct{a_i}{a_i}} a_i .
    \end{gather*}
    Из пункта а мы уже знаем, что $\mathcal{A}_i$ --- линейные отображения, поэтому для любых векторов $x$ и $y$ и числа $\lambda$:
    \begin{gather*}
        \mathcal{A}(x + y)
        = \mathcal{A}_1(x + y) + \mathcal{A}_2(x + y)
        = \mathcal{A}_1(x) + \mathcal{A}_1(y) + \mathcal{A}_2(x) + \mathcal{A}_2(y)
        = \mathcal{A}(x) + \mathcal{A}(y), \\
        %
        \mathcal{A}(\lambda x)
        = \mathcal{A}_1(\lambda x) + \mathcal{A}_2(\lambda x)
        = \lambda \mathcal{A}_1(x) + \lambda \mathcal{A}_2(x)
        = \lambda \left ( \mathcal{A}_1(x) + \mathcal{A}_2(x) \right )
        = \lambda \mathcal{A}(x),
    \end{gather*}
    и $\mathcal{A}$ тоже линейное отображение.

    Произведения:
    \begin{gather*}
        \scalarproduct{a_1}{a_1} = 1 + 1 = 2, \\
        \scalarproduct{a_2}{a_2} = 1 + 1 = 2 ,
    \end{gather*}
    действие отображения $\mathcal{A}$ на векторы стандартного базиса:
    \begin{align*}
        \mathcal{A}(e_1) & = \frac{1}{2} a_1 + \frac{1}{2} a_2 , \\
        \mathcal{A}(e_2) & = \frac{0}{2} a_1 + \frac{0}{2} a_2  = 0 , \\
        \mathcal{A}(e_3) & = -\frac{1}{2} a_1 + \frac{1}{2} a_2 .
    \end{align*}
    откуда матрица $A$ отображения:
    \[
        A
        =
        \begin{pmatrix}
            \frac{1}{2} & 0 & - \frac{1}{2} \\
            \frac{1}{2} & 0 & \frac{1}{2}
        \end{pmatrix}
        .
    \]

    \subsection*{Ответ:}
    Матрица отображения:
    $
    \begin{pmatrix}
        \frac{1}{2} & 0 & - \frac{1}{2} \\
        \frac{1}{2} & 0 & \frac{1}{2}
    \end{pmatrix}
    $.
\end{document}