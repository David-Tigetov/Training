\documentclass[12pt]{article}

\usepackage[T1]{fontenc}
\usepackage[utf8]{inputenc}
\usepackage[english,russian]{babel}
\usepackage[margin=2cm]{geometry}
\usepackage{amsmath}
\usepackage{amsfonts}
\usepackage{xcolor}
\usepackage{color}

\begin{document}

    \title{Задача 2}
    \author{Тигетов Давид Георгиевич}
    \date{}
    \maketitle

    \section*{Пункт а}
    Лежит ли вектор $x$:
    \[
        x =
        \begin{pmatrix}
            1 \\ -7 \\ 5 \\ 3
        \end{pmatrix}
    \]
    в линейной оболочке векторов:
    \[
        a_1 =
        \begin{pmatrix}
            1 \\ 2 \\ 1 \\ 3
        \end{pmatrix}, \;
        a_2 =
        \begin{pmatrix}
            1 \\ 1 \\ 1 \\ 3
        \end{pmatrix}, \;
        a_3 =
        \begin{pmatrix}
            1 \\ 0 \\ 1 \\ 3
        \end{pmatrix}.
    \]

    \subsection*{Решение:}
    Составим матрицу из векторов $a_i$ и вектора $x$ и линейными преобразованиями \textcolor{blue}{над столбцами} попробуем занулить столбец вектора $x$:
    \[
        \begin{pmatrix}
            1 & 1 & 1 & 1  \\
            2 & 1 & 0 & -7 \\
            1 & 1 & 1 & 5  \\
            3 & 3 & 3 & 3
        \end{pmatrix}
        \rightarrow
        \begin{pmatrix}
            0 & 0 & 1 & 0  \\
            2 & 1 & 0 & -7 \\
            0 & 0 & 1 & 4  \\
            0 & 0 & 3 & 0
        \end{pmatrix}
        \rightarrow
        \begin{pmatrix}
            0 & 0 & 1 & 0 \\
            0 & 1 & 0 & 0 \\
            0 & 0 & 1 & 4 \\
            0 & 0 & 3 & 0
        \end{pmatrix}
    \]
    \subsection*{Ответ:}
    Нет.

    \section*{Пункт б \textcolor{red}{[верно]}}
    Докажите, что система векторов $e_1$, $e_2$, $e_3$, $e_4$ --- это базис пространства и найдите координаты вектора $x$ в этом базисе
    \[
        e_1 =
        \begin{pmatrix}
            1 \\ 2 \\ -1 \\ 2
        \end{pmatrix}, \;
        e_2 =
        \begin{pmatrix}
            2 \\ 3 \\ 0 \\ -1
        \end{pmatrix}, \;
        e_3 =
        \begin{pmatrix}
            1 \\ 2 \\ 1 \\ 4
        \end{pmatrix}, \;
        e_4 =
        \begin{pmatrix}
            1 \\ 3 \\ -1 \\ 0
        \end{pmatrix}, \;
        x =
        \begin{pmatrix}
            7 \\ 14 \\ -1 \\ 2
        \end{pmatrix} .
    \]

    \subsection*{Решение:}
    Прежде всего проверим линейную независимость векторов $e_i$ и попутно определим координаты вектора $x$ линейными преобразованиями над строками матрицы:
    \begin{gather*}
        \begin{pmatrix}
            1  & 2  & 1 & 1  & 7  \\
            2  & 3  & 2 & 3  & 14 \\
            -1 & 0  & 1 & -1 & -1 \\
            2  & -1 & 4 & 0  & 2
        \end{pmatrix}
        \rightarrow
        \begin{pmatrix}
            2  & 2  & 0 & 2  & 8  \\
            4  & 3  & 0 & 5  & 16 \\
            -1 & 0  & 1 & -1 & -1 \\
            6  & -1 & 0 & 4  & 6
        \end{pmatrix}
        \rightarrow
        \begin{pmatrix}
            1  & 1  & 0 & 1 & 4   \\
            -1 & -2 & 0 & 0 & -4  \\
            0  & 1  & 1 & 0 & 3   \\
            2  & -5 & 0 & 0 & -10
        \end{pmatrix}
        \rightarrow \\
        \rightarrow
        \begin{pmatrix}
            0 & -1 & 0 & 1 & 0   \\
            1 & 2  & 0 & 0 & 4   \\
            0 & 1  & 1 & 0 & 3   \\
            0 & -9 & 0 & 0 & -18
        \end{pmatrix}
        \rightarrow
        \begin{pmatrix}
            0 & 0 & 0 & 1 & 2 \\
            1 & 0 & 0 & 0 & 0 \\
            0 & 0 & 1 & 0 & 1 \\
            0 & 1 & 0 & 0 & 2
        \end{pmatrix}
    \end{gather*}
    Координаты вектора $x$ --- $\left ( 0, 2, 1, 2 \right )$.

    Векторы $e_i$ линейно независимы. Любой вектор выражается через $e_i$, поскольку в противном случае имеем пять линейно назависимых векторов, следовательно, в базисе, построенном
    на основе $e_i$ будет уже точно больше четырех векторов, а это противоречит наличию стандартного базиса $\varphi_i = \left ( 0, \dots, 0, 1, 0, \dots, 0 \right )$, в котором ровно четыре
    вектора и одинаковому количеству векторов в любом базисе.

    \subsection*{Ответ:}
    Координаты вектора $x$ --- $\left ( 0, 2, 1, 2 \right )$.

    \section*{Пункт в}
    Исследуйте систему и найдите общее решение в зависимости от значения параметра $\lambda$.
    \[
        \begin{pmatrix}
            2 & 5  & 1 & 3       \\
            4 & 6  & 3 & 5       \\
            4 & 14 & 1 & 7       \\
            2 & -3 & 3 & \lambda
        \end{pmatrix}
        \begin{pmatrix}
            x_1 \\ x_2 \\ x_3 \\ x_4
        \end{pmatrix}
        =
        \begin{pmatrix}
            2 \\ 4 \\ 4 \\ 7
        \end{pmatrix}
    \]

    \subsection*{Решение:}
    Составим расширенную матрицу и преобразуем её линейными операциями над строками:
    \begin{gather*}
        \begin{pmatrix}
            2 & 5  & 1 & 3       & 2 \\
            4 & 6  & 3 & 5       & 4 \\
            4 & 14 & 1 & 7       & 4 \\
            2 & -3 & 3 & \lambda & 7
        \end{pmatrix}
        \rightarrow
        \begin{pmatrix}
            2  & 5   & 1 & 3         & 2  \\
            -2 & -9  & 0 & -4        & -2 \\
            2  & 9   & 0 & 4         & 2  \\
            -4 & -18 & 0 & \lambda-9 & 1
        \end{pmatrix}
        \rightarrow
        \begin{pmatrix}
            0 & -4 & 1 & -1          & 0 \\
            0 & 0  & 0 & 0           & 0 \\
            2 & 9  & 0 & 4           & 2 \\
            0 & 0  & 0 & \lambda - 1 & 5
        \end{pmatrix}
    \end{gather*}

    Если $\lambda = 1$, то система не имеет решений.

    Если $\lambda \neq 1$, то частное решение СЛУ:
    \[
        a =
        \begin{pmatrix}
            1 - \frac{10}{\lambda - 1}              \\
            0                                       \\
            \textcolor{blue}{\frac{5}{\lambda - 1}} \\
            \frac{5}{\lambda - 1}
        \end{pmatrix}
    \]
        {\color{blue} Проверка частного решения:
    \[
        \begin{pmatrix}
            2 & 5  & 1 & 3       \\
            4 & 6  & 3 & 5       \\
            4 & 14 & 1 & 7       \\
            2 & -3 & 3 & \lambda
        \end{pmatrix}
        \begin{pmatrix}
            1 - \frac{10}{\lambda - 1} \\
            0                          \\
            \frac{5}{\lambda - 1}      \\
            \frac{5}{\lambda - 1}
        \end{pmatrix}
        =
        \begin{pmatrix}
            2 - \frac{20}{\lambda-1} + \frac{5}{\lambda-1} + \frac{15}{\lambda-1}  \\
            4 - \frac{40}{\lambda-1} + \frac{15}{\lambda-1} + \frac{25}{\lambda-1} \\
            4 - \frac{40}{\lambda-1} + \frac{5}{\lambda-1} + \frac{35}{\lambda-1}  \\
            2 - \frac{20}{\lambda-1} + \frac{15}{\lambda-1} + \frac{5 \lambda}{\lambda-1}
        \end{pmatrix}
        =
        \begin{pmatrix}
            2 \\
            4 \\
            4 \\
            \frac{2 \lambda - 2 - 5 + 5 \lambda}{\lambda-1}
        \end{pmatrix}
        =
        \begin{pmatrix}
            2 \\
            4 \\
            4 \\
            \frac{7 (\lambda - 1)}{\lambda-1}
        \end{pmatrix}
        =
        \begin{pmatrix}
            2 \\
            4 \\
            4 \\
            7
        \end{pmatrix}
        .
    \]
    }

    Базис пространства решений ОСЛУ:
    \[
        b =
        \begin{pmatrix}
            - 9 \\ 2 \\ 8 \\ 0
        \end{pmatrix}
        .
    \]

    \subsection*{Ответ:}
    При $\lambda = 1$ решения нет.

    При $\lambda \neq 1$ общее решение имеет вид:
    \[
        \begin{pmatrix}
            1 - \frac{10}{\lambda - 1}              \\
            0                                       \\
            \textcolor{blue}{\frac{5}{\lambda - 1}} \\
            \frac{5}{\lambda - 1}
        \end{pmatrix}
        + \alpha
        \begin{pmatrix}
            - 9 \\ 2 \\ 8 \\ 0
        \end{pmatrix}
        ,
    \]
    где $\alpha$ --- произвольная постоянная.

    \subsection*{Пункт г}
    Составьте какую-нибудь СЛУ, для которой решениями будут
    \[
        A_1 =
        \begin{pmatrix}
            0 \\ 1 \\ 0 \\ 1 \\ 5
        \end{pmatrix},
        A_2 =
        \begin{pmatrix}
            3 \\ -1 \\ 3 \\ 4 \\ 0
        \end{pmatrix},
        A_3 =
        \begin{pmatrix}
            2 \\ 2 \\ 7 \\ 6 \\ 1
        \end{pmatrix},
    \]
    и при этом ранг соответствующей матрицы будет равен 3.

    \subsection*{Решение:}
    Строки матрицы $\left ( a_1, a_2, a_3, a_4, a_5 \right )$ \textcolor{blue}{СЛУ} должны удовлетворять системе:
        {\color{blue}
    \begin{gather*}
        \left \{
        \begin{array}{lll}
            a_1 \cdot 0 + a_2 \cdot 1 + a_3 \cdot 0 + a_4 \cdot 1 + a_5 \cdot 5 & = & a_1 \cdot 3 + a_2 \cdot (-1) + a_3 \cdot 3 + a_4 \cdot 4 + a_5 \cdot 0 \\
            a_1 \cdot 0 + a_2 \cdot 1 + a_3 \cdot 0 + a_4 \cdot 1 + a_5 \cdot 5 & = & a_1 \cdot 2 + a_2 \cdot 2 + a_3 \cdot 7 + a_4 \cdot 6 + a_5 \cdot 1    \\
        \end{array}
        \right . , \\
        %
        \left \{
        \begin{array}{lll}
            -3 a_1 + 2 a_2 - 3 a_3 - 3 a_4 + 5 a_5 & = & 0 \\
            -2 a_1 - 1 a_2 - 7 a_3 - 5 a_4 + 4 a_5 & = & 0 \\
        \end{array}
        \right . , \\
        %
        \left \{
        \begin{array}{cccccccccc}
            -7 a_1 &   &     & - & 17 a_3 & - & 13 a_4 & + & 13 a_5 & = 0 \\
            -2 a_1 & - & a_2 & - & 7 a_3  & - & 5 a_4  & + & 4 a_5  & = 0
        \end{array}
        \right . , \\
        %
        \left \{
        \begin{array}{cccccccccc}
            -7 a_1 &   &     & = & 17 a_3 & + & 13 a_4 & - & 13 a_5 \\
            -2 a_1 & - & a_2 & = & 7 a_3  & + & 5 a_4  & - & 4 a_5
        \end{array}
        \right .
        .
    \end{gather*}

    Отсюда матрица СЛУ:
    \[
        \begin{pmatrix}
            - 17 & - 15 & 7 & 0 & 0 \\
            -13  & - 9  & 0 & 7 & 0 \\
            13   & 2   & 0 & 0 & 7
        \end{pmatrix}
    \]
    Ранг матрицы системы равен 3.

    Столбец правой части получим умножением справа на вектор $A_1$:
    \[
        \begin{pmatrix}
            - 17 & - 15 & 7 & 0 & 0 \\
            -13  & - 9  & 0 & 7 & 0 \\
            13   & 2    & 0 & 0 & 7
        \end{pmatrix}
        \begin{pmatrix}
            0 \\ 1 \\ 0 \\ 1 \\ 5
        \end{pmatrix}
        =
        \begin{pmatrix}
            -15 \\
            -2  \\
            37
        \end{pmatrix}
    \]
    }

    \subsection*{Ответ:}
    {\color{blue}
        $
        \begin{pmatrix}
            - 17 & - 15 & 7 & 0 & 0 \\
            -13  & - 9  & 0 & 7 & 0 \\
            13   & 2    & 0 & 0 & 7
        \end{pmatrix}
        \begin{pmatrix}
            x_1 \\ x_2 \\ x_3 \\ x_4 \\ x_5
        \end{pmatrix}
        =
        \begin{pmatrix}
            -15 \\
            -2  \\
            37
        \end{pmatrix}
        $
    }
\end{document}