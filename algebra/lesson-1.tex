\documentclass[12pt]{article}

\usepackage[T1]{fontenc}
\usepackage[utf8]{inputenc}
\usepackage[english,russian]{babel}
\usepackage[margin=2cm]{geometry}
\usepackage{amsmath}
\usepackage{xcolor}

\begin{document}

    \title{Задача 1}
    \author{Тигетов Давид Георгиевич}
    \date{}
    \maketitle

    \section*{Пункт а}
    Найдите все максимальные линейно независимые подсистемы следующей системы векторов
    \[
        a_1 = \begin{pmatrix}
                  2 \\ 1 \\ -3 \\ 1
        \end{pmatrix}, \;
        a_2 = \begin{pmatrix}
                  2 \\ 2 \\ -6 \\ 2
        \end{pmatrix}, \;
        a_3 = \begin{pmatrix}
                  6 \\ 3 \\ -9 \\ 3
        \end{pmatrix}, \;
        a_4 = \begin{pmatrix}
                  1 \\ 1 \\ 1 \\ 1
        \end{pmatrix}, \;
    \]

    \subsection*{Решение:}
    Для краткости соберем векторы в матрицу и будем выполнять линейные преобразования над столбцами:
    \[
        \begin{array}{c}
            \text{исходные векторы} \\
            \begin{pmatrix}
                2  & 2  & 6  & 1 \\
                1  & 2  & 3  & 1 \\
                -3 & -6 & -9 & 1 \\
                1  & 2  & 3  & 1
            \end{pmatrix}
        \end{array}
        \rightarrow
        \begin{array}{c}
            a_3 \leftarrow a_3 - 3 a_1 \\
            \begin{pmatrix}
                2  & 2  & 0 & 1 \\
                1  & 2  & 0 & 1 \\
                -3 & -6 & 0 & 1 \\
                1  & 2  & 0 & 1
            \end{pmatrix}
        \end{array}
        \rightarrow
        \begin{array}{c}
            a_2 \leftarrow \left ( a_2 - 2 a_1 \right ) \left ( - \frac{1}{2} \right ) \\
            \begin{pmatrix}
                2  & 1 & 0 & 1 \\
                1  & 0 & 0 & 1 \\
                -3 & 0 & 0 & 1 \\
                1  & 0 & 0 & 1
            \end{pmatrix}
        \end{array}
    \]

    \subsection*{Ответ:}
    $a_1, a_2, a_4$ и $\textcolor{blue}{a_2}, a_3, a_4$.

    \subsection*{Пункт б}
    Найдите какую-нибудь ФСР следующей ОСЛУ:
    \[
        \begin{pmatrix}
            2 & -1 & 3 & -2 & 4 \\
            4 & -2 & 5 & 1  & 7 \\
            2 & -1 & 1 & 8  & 2
        \end{pmatrix}
        \begin{pmatrix}
            x_1 \\
            x_2 \\
            x_3 \\
            x_4 \\
            x_5
        \end{pmatrix}
        = 0
    \]

    \subsection*{Решение:}
    Линейными преобразованиями над строками приводим матрицу системы в верхне-треугольному виду:
    \[
        \begin{pmatrix}
            2 & -1 & 3 & -2 & 4 \\
            4 & -2 & 5 & 1  & 7 \\
            2 & -1 & 1 & 8  & 2
        \end{pmatrix}
        \rightarrow
        \begin{pmatrix}
            2 & -1 & 3  & -2 & 4  \\
            0 & 0  & -6 & 5  & -1 \\
            0 & 0  & -2 & 10 & -2
        \end{pmatrix}
        \rightarrow
        \begin{pmatrix}
            2 & -1 & 3  & -2 & 4  \\
            0 & 0  & -6 & 5  & -1 \\
            0 & 0  & 10 & 0  & 0
        \end{pmatrix}
    \]
    Пусть переменные $x_1$ и $x_4$ являются свободными, а $x_2$, $x_3$, $x_5$ зависимыми, причём из последней строки следует, что $x_3 = 0$. Векторы $v_1$ и $v_2$, образующие ФСР, имеют
    вид:
    \[
        v_1 =
        \begin{pmatrix}
            1 \\
            2 \\
            0 \\
            0 \\
            0
        \end{pmatrix}
        , \;
        v_2 =
        \begin{pmatrix}
            0  \\
            18 \\
            0  \\
            1  \\
            5
        \end{pmatrix}
    \]

    \subsection*{Ответ:}
    $\left ( 1, 2, 0, 0, 0 \right )^T$, $\left ( 0, 18, 0, 1, 5 \right )^T$.

    \section*{Пункт в}
    Найдите ранг матрицы при различных значениях параметра $\lambda$:
    \[
        \begin{pmatrix}
            1 & \lambda & -1      & 2 \\
            2 & -1      & \lambda & 5 \\
            1 & 10      & -6      & 1
        \end{pmatrix}
    \]

    \subsection*{Решение:}
    Преобразуем матрицу вычитанием первого столбца из всех остальных
    \[
        \begin{pmatrix}
            1 & \lambda - 10 & 5            & 1 \\
            2 & -21          & \lambda + 12 & 3 \\
            1 & 0            & 0            & 0
        \end{pmatrix}
    \]
    и вычитанием четвертого столбца из второго и третьего
    \[
        \begin{pmatrix}
            1 & \lambda - 3 & 0           & 1 \\
            2 & 0           & \lambda - 3 & 3 \\
            1 & 0           & 0           & 0
        \end{pmatrix}
    \]

    \subsection*{Ответ:}
    Если $\lambda = 3$, то ранг равен 2, если $\lambda \neq 3$, то ранг равен 3.


    \section*{Пункт г}
    Составьте какую-нибудь ОСЛУ, для которой множество решений представляется в виде линейной оболочки следующих векторов:
    \[
        a_1 = \begin{pmatrix}
                  1 \\ 1 \\ 1 \\ 1
        \end{pmatrix}, \;
        a_2 = \begin{pmatrix}
                  1 \\ 1 \\ 1 \\ 3
        \end{pmatrix}, \;
        a_3 = \begin{pmatrix}
                  3 \\ -5 \\ 7 \\ 2
        \end{pmatrix}, \;
        a_4 = \begin{pmatrix}
                  1 \\ -7 \\ 5 \\ -2
        \end{pmatrix}, \;
    \]

    \subsection*{Решение:}
    Каждая строка $\left ( x_1, x_2, x_3, x_4 \right )$ матрицы искомой ОСЛУ должна удовлетворять равенству:
    \[
        \left ( x_1, x_2, x_3, x_4 \right )
        \begin{pmatrix}
            1 & 1 & 3  & 1  \\
            1 & 1 & -5 & -7 \\
            1 & 1 & 7  & 5  \\
            1 & 3 & 2  & -2
        \end{pmatrix}
        = 0
    \]

    Преобразуем матрицу в левой части линейными операциями над столбцами:
    \[
        \begin{pmatrix}
            1 & 1 & 3  & 1  \\
            1 & 1 & -5 & -7 \\
            1 & 1 & 7  & 5  \\
            1 & 3 & 2  & -2
        \end{pmatrix}
        \rightarrow
        \begin{pmatrix}
            1 & 0 & 0  & 0  \\
            1 & 0 & -8 & -8 \\
            1 & 0 & 4  & 4  \\
            1 & 2 & -1 & -3
        \end{pmatrix}
        \rightarrow
        \begin{pmatrix}
            1 & 0 & 0  & 0  \\
            1 & 0 & -8 & 0  \\
            1 & 0 & 4  & 0  \\
            1 & 2 & -1 & -2
        \end{pmatrix}
        \rightarrow
        \begin{pmatrix}
            1 & 0 & 0  & 0 \\
            1 & 0 & -8 & 0 \\
            1 & 0 & 4  & 0 \\
            1 & 2 & -1 & 0
        \end{pmatrix}
    \]
    Пусть $x_3$ --- свободная переменная, $x_1$, $x_2$, $x_3$ --- зависимые. Из последней строки следует, что $x_4 = 0$, поэтому строки матрицы искомой ОСЛУ необходимо формировать
    по правилу:
    \[
        \left ( x_1, x_2, x_3, x_4 \right ) = \left ( - 3 c, c, 2 c, 0 \right ) ,
    \]
    где $c$ --- произвольная постоянная.

    \textcolor{blue}{Самая простая ОСЛУ состоит из одного уравнения.}

    \subsection*{Ответ:}
    $\color{blue} -3 y_1 + y_2 + 2 y_3 = 0 $.
\end{document}

