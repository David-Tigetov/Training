\documentclass[12pt]{article}

\usepackage[T1]{fontenc}
\usepackage[utf8]{inputenc}
\usepackage[english,russian]{babel}
\usepackage[margin=2cm]{geometry}
\usepackage{amsmath}
\usepackage{amsfonts}
\usepackage{xcolor}
\usepackage{color}
\usepackage{amssymb}

% команды вывода первой частной производной
\newcommand{\fpd}[1]{\frac{\partial}{\partial #1}}
\newcommand{\fpda}[2]{\frac{\partial #1}{\partial #2}}
\newcommand{\fpdp}[2]{\fpd{#2} \left ( #1 \right )}

\newcommand{\expectation}[1]{\mathtt{M} \left [ #1 \right ]}
\newcommand{\conditionalexpectation}[2]{\expectation{ #1 \left | #2 \right .}}
\newcommand{\variance}[1]{\mathtt{D} \left [ #1 \right ]}
\newcommand{\covariance}[2]{\mathtt{cov} \left ( #1, #2 \right )}

\newcommand{\modulus}[1]{\left | #1 \right |}
\newcommand{\norm}[1]{\left \| {#1} \right \|}

\newcommand{\event}[1]{\left \{ #1 \right \} }
\newcommand{\probability}[1]{P \event{#1}}


\begin{document}

    \title{Задача 13}
    \author{Тигетов Давид Георгиевич}
    \date{}
    \maketitle

    \section*{Пункт а}
    Найдите жорданову нормальную форму и жорданов базис для операторов, заданных матрицами:
    \[
        A =
        \begin{pmatrix}
            -2 & -3 & -4 \\
            -1 & 0  & 0  \\
            1  & 1  & 1
        \end{pmatrix},
        B =
        \begin{pmatrix}
            -5 & -9 & 2  & 7   \\
            -3 & -2 & 1  & 3   \\
            9  & 9  & -5 & -12 \\
            -9 & -9 & 3  & 10
        \end{pmatrix}
    \]

    \subsection*{Решение:}
    У матрицы $A$ два собственных значения $\lambda_1 = -1$ и $\lambda_2 = 1$.

    Для собственного значения $\lambda_1$ есть только один собственный вектор, поэтому строим базис в корневом подпространстве $\kernel{\left ( A - \lambda_1 E \right)^2}$, в котором
    выберем вектор
    \begin{gather*}
        v_2 \in \kernel{\left ( A - \lambda_1 E \right)^2}, \\
        v_2 \notin \kernel{\left ( A - \lambda_1 E \right)}.
    \end{gather*}
    Таким вектором является:
    \[
        v_2 = \begin{pmatrix}
                  1 \\ 0 \\ 0
        \end{pmatrix}.
    \]
    Под действием $\left ( A - \lambda_1 E \right )$ вектор $v_1$ переходит во второй базисный вектор $v_2$:
    \[
        v_1
        = \left ( A - \lambda_1 E \right ) v_2
        = \begin{pmatrix}
              -1 \\ -1 \\ 1
        \end{pmatrix}.
    \]

    Для собственного значения $\lambda_2$ есть собственный вектор:
    \[
        v_3 = \begin{pmatrix}
                  -1 \\ 1 \\ 0
        \end{pmatrix} .
    \]

    Матрица перехода к базису $v_1$, $v_2$, $v_3$:
    \[
        C_A =
        \begin{pmatrix}
            -1 & 1 & -1 \\
            -1 & 0 & 1  \\
            1  & 0 & 0
        \end{pmatrix}.
    \]

    Матрица оператора в базисе $v_1$, $v_2$, $v_3$:
    \[
        J_A
        = C_A^{-1} A C_A
        = \begin{pmatrix}
              -1 & 1  & 0 \\
              0  & -1 & 0 \\
              0  & 0  & 1
        \end{pmatrix}.
    \]

    У матрицы $B$ два собственных значения $\lambda_1 = -2$ и $\lambda_2 = 1$.

    Для $\lambda_1$ всего один собственный вектор --- требуется рассмотреть корневое подпространство. Кратность корня равна 2, поэтому достаточно рассмотреть
    $\kernel{\left ( B - \lambda_1 E \right )^2}$. Возьмем вектор
    \begin{gather*}
        v_2 \in \kernel{\left ( B - \lambda_1 E \right)^2}, \\
        v_2 \notin \kernel{\left ( B - \lambda_1 E \right)}, \\
        v_2 = \begin{pmatrix}
                  0 \\ 0 \\ 1 \\ 0
        \end{pmatrix}.
    \end{gather*}
    В качестве второго базисного вектора из $\kernel{\left ( B - \lambda_1 E \right )^2}$ возьмем
    \[
        v_1
        = \left ( B - \lambda_1 E \right ) v_2
        = \begin{pmatrix}
              2 \\ 1 \\ -3 \\ 3
        \end{pmatrix} .
    \]

    Для $\lambda_2$ также один собственный вектор, рассматриваем корневое подпространство $\kernel{\left ( B - \lambda_1 E \right )^2}$:
    \begin{gather*}
        v_4 \in \kernel{\left ( B - \lambda_2 E \right)^2}, \\
        v_4 \notin \kernel{\left ( B - \lambda_2 E \right)}, \\
        v_4 = \begin{pmatrix}
                  -2 \\ 1 \\ 0 \\ 0
        \end{pmatrix}, \\
        %
        v_3
        = \left ( B - \lambda_2 E \right) v_4
        = \begin{pmatrix}
              3 \\ 3 \\ -9 \\ 9
        \end{pmatrix}.
    \end{gather*}

    Матрица перехода к базису $v_1$, $v_2$, $v_3$, $v_4$:
    \[
        C_B
        = \begin{pmatrix}
              2  & 0 & 3  & -2 \\
              1  & 0 & 3  & 1  \\
              -3 & 1 & -9 & 0  \\
              3  & 0 & 9  & 0
        \end{pmatrix}.
    \]
    Матрица оператора в базисе $v_1$, $v_2$, $v_3$, $v_4$:
    \[
        J_B
        = C_B^{-1} B C_B
        = \begin{pmatrix}
              -2 & 1  & 0 & 0 \\
              0  & -2 & 0 & 0 \\
              0  & 0  & 1 & 1 \\
              0  & 0  & 0 & 1
        \end{pmatrix}.
    \]

    \subsection*{Ответ:}
    Для матрицы $A$ жорданова форма:
    \[
        \begin{pmatrix}
            -1 & 1  & 0 \\
            0  & -1 & 0 \\
            0  & 0  & 1
        \end{pmatrix}.
    \]
    в жордановом базисе:
    \begin{gather*}
        v_1 = \begin{pmatrix}
                  -1 \\ -1 \\ 1
        \end{pmatrix},
        v_2 = \begin{pmatrix}
                  1 \\ 0 \\ 0
        \end{pmatrix},
        v_3 = \begin{pmatrix}
                  -1 \\ 1 \\ 0
        \end{pmatrix} .
    \end{gather*}

    Для матрицы $B$ жорданова форма:
    \[
        \begin{pmatrix}
            -2 & 1  & 0 & 0 \\
            0  & -2 & 0 & 0 \\
            0  & 0  & 1 & 1 \\
            0  & 0  & 0 & 1
        \end{pmatrix}.
    \]
    в жордановом базисе:
    \begin{gather*}
        v_1 = \begin{pmatrix}
                  2 \\ 1 \\ -3 \\ 3
        \end{pmatrix} ,
        v_2 = \begin{pmatrix}
                  0 \\ 0 \\ 1 \\ 0
        \end{pmatrix},
        v_3 = \begin{pmatrix}
                  3 \\ 3 \\ -9 \\ 9
        \end{pmatrix},
        v_4 = \begin{pmatrix}
                  -2 \\ 1 \\ 0 \\ 0
        \end{pmatrix}.
    \end{gather*}

\end{document}