\section{Постановка задачи нахождения проекционно аннулирующего полинома} \label{section:TFPAP:task_for_finding_projection_annihilating_polynomials}

Для решения неоднородных и однородных систем уравнений необходимо располагать способом построения аннулирущего полинома для заданного
вектора из пространства $\gfpvector{n}$.

Как уже отмечалось в подразделе \ref{section:KS:krylov_spaces}, один из способов построения аннулирущего полинома $\polynomial{\varphi}$:

	$$ \varphi(\lambda) =
		\varphi_0 + \varphi_1 \lambda + \varphi_2 \lambda^2 + \dots + \varphi_r \lambda^r $$

некоторого вектора $z \in \gfpvector{n}$ заключается в нахождении наименьшего количества линейно зависимых векторов Крылова 

	$$ z, Az, A^2z, \dots, A^rz $$

и определении коэффициентов $\varphi_i$ линейной комбинации этих векторов, дающих нулевой вектор:

	$$ \varphi_0 z + \varphi_1 Az + \varphi_2 A^2 z + \dots + \varphi_r A^r z = 0, $$
	\begin{equation} \label{equation:TFPAP:vector_annihilating_polynomial_condition}
		\varphi(A) z = 0.
	\end{equation}

К сожалению, задача нахождения коэффициентов $\varphi_i$ приводит к задаче нахождения решения однородной системы уравнений, которую
призвана решать, поэтому задачу поиска коэффициентов $\varphi_i$ приходится заменять другой задачей, более простой в решении.
Упрощение исходной задачи приводит к снижению вероятности успеха: не всякое решение упрощенной задачи оказывается решением исходной
задачи, тем не менее, ничего другого по-видимому не остается.

Упрощенная задача, о которой пойдет речь, и метод её решения рассматриваются работе Видемана \cite{Wiedemann}.

В упомянутой упрощенной задаче условие для аннулирующего полинома $\polynomial{\varphi}$
\eqref{equation:TFPAP:vector_annihilating_polynomial_condition}, приводящих линейную комбинацию векторов Крылова $z$, \dots, $A^rz$
к нулевому вектору заменено более простым условием. Возьмем произвольным образом вектор $y \in \gfpvector{n}$ и построим подпространство
Крылова для вектора $y$ и сопряженного оператора:

	$$ \mathcal K_n(y,A^T) = \mathcal L \left \{ y, A^T y, \left ( A^T \right ) ^2 y, \dots, \left ( A^T \right )^{n-1} y \right \}. $$

Далее будем рассматривать задачу в смысле проекционных методов: будем искать такой вектор $w$ подпространства
Крылова $\mathcal K_{r+1}(z, A)$, который является ортогональным подпространству Крылова $\mathcal K_n(y, A^T)$:

	\begin{equation} \label{equation:TFPAP:w_orthogonality_to_conjugated_space_condition}
		w \perp \mathcal K_n(y, A^T),
	\end{equation}
	$$ w \in \mathcal K_{r+1}(z, A). $$

Поскольку вектор $w$ принадлежит подпространству Крылова $\mathcal K_{r+1}(z, A)$, а подпространство $\mathcal K_{r+1}(z, A)$ является
линейной оболочкой векторов Крылова $z$, $Az$, $A^2z$, \dots, $A^rz$, то существуют коэффициенты $\rho_0$, $\rho_1$, $\rho_2$, \dots,
$\rho_r$ такие, что вектор $w$ является линейной комбинацией векторов Крылова $z$, $Az$, $A^2z$, \dots, $A^rz$
с коэффициентами $\rho_0$, $\rho_1$, $\rho_2$, \dots, $\rho_r$:

	$$ w = \rho_0 z + \rho_1 Az + \rho_2 A^2 z + \dots + \rho_r A^r z. $$

Откуда, вынося $z$ получим, равенство:

	$$ w = ( \rho_0 + \rho_1 A + \rho_2 A^2 + \dots + \rho_r A^r ) z, $$

и если представить себе некоторый полином $\polynomial{\rho}$:

	$$ \rho(\lambda) = \rho_0 + \rho_1 \lambda + \rho_2 \lambda^2 + \dots + \rho_r \lambda^r, $$

то

	$$ w = \rho(A)z, $$
	
и из условия \eqref{equation:TFPAP:w_orthogonality_to_conjugated_space_condition} следует:

	\begin{equation} \label{equation:TFPAP:polynomial_orthogonality_to_conjugated_space_condition}
		\rho(A)z \perp \mathcal K_n(y, A^T).
	\end{equation}

Для удобства последущего изложения введем для полиномов $\polynomial{\rho}$, удовлетворяющих условию 
\eqref{equation:TFPAP:polynomial_orthogonality_to_conjugated_space_condition} специальный термин.

\begin{definition}
	Полином $\polynomial{\rho} \in \gfppolynomial$ называется \definedterm{проекционно аннулирущим полиномом вектора $x$ и
	пространства $\mathcal F$ относительно оператора $\mathcal A$}, если:

	$$ \rho(A) x \perp \mathcal F. $$
\end{definition}

Как и ранее для краткости слова "относительно оператора $\mathcal A$"{} будем опускать.
 
Слово "процекционно"{} в приведенном определении указывает на то, что полином $\polynomial{\rho}$ зануляет не вектор $\rho(A)x$, а
его проекцию на пространство $\mathcal F$.

\begin{definition}
	Полином $\polynomial{\tilde\rho} \in \gfppolynomial$ называется \definedterm{минимальным проекционно аннулирующим полином
	вектора $x \in \gfpvector{n}$ и пространства $\mathcal F$ отностительно оператора $\mathcal A$}, если его степень не больше степени
	любого другого проекционно аннулирующего полинома $\polynomial{\rho}$ вектора $x \in \gfpvector{n}$ и пространства $\mathcal F$
	отностительно оператора $\mathcal A$:
		$$ \polynomialdegree{\tilde\rho} \le \polynomialdegree{\rho}. $$
\end{definition}

Таким образом, задача нахождения аннулирующего полинома $\polynomial{\varphi}$ вектора $z$ заменяется более простой задачей
нахождения проекционно аннулирующего полинома $\polynomial{\rho}$ вектора $z$ и подпространства Крылова $K_n(y, A^T)$
\eqref{equation:TFPAP:polynomial_orthogonality_to_conjugated_space_condition}.

Поскольку вектор $\rho(A) x$ ортогонален пространству $\mathcal K_n(y, A^T)$, то вектор $\rho(A) x$ должен быть ортогонален всем
векторам Крылова $y$, $A^T y$, $\left ( A^T \right ) ^2 y$, \dots, $\left ( A^T \right )^{n-1} y$. Верно и обратное, если вектор
$\rho(A) x$ ортогонален всем векторам Крылова $y$, $A^T y$, $\left ( A^T \right ) ^2 y$, \dots, $\left ( A^T \right )^{n-1} y$,
то вектор $\rho(A) x$ ортогонален каждой линейной комбинации векторов Крылова $y$, $A^T y$, $\left ( A^T \right ) ^2 y$, \dots,
$\left ( A^T \right )^{n-1} y$ и следовательно ортогонален всему подпространству Крылова $\mathcal K_n(y, A^T)$, представляющему собой
линейную оболочку векторов Крылова $y$, $A^T y$, $\left ( A^T \right ) ^2 y$, \dots, $\left ( A^T \right )^{n-1} y$. Таким образом,
условие \eqref{equation:TFPAP:polynomial_orthogonality_to_conjugated_space_condition} эквивалентно ортогональности вектора $\rho(A) x$ 
каждому из векторов Крылова $y$, $A^T y$, $\left ( A^T \right ) ^2 y$, \dots, $\left ( A^T \right )^{n-1} y$:

	$$
		\left \{
			\begin{array}{ccc}
				\rho(A) z & \perp & y \\
				\rho(A) z & \perp & A^T y \\
				\rho(A) z & \perp & \left ( A^T \right )^2 y \\
				\vdots \\
				\rho(A) z & \perp & \left ( A^T \right )^{n-1} y \\
			\end{array}
		\right .
		.
	$$

Условия ортогональности эквивалентны равенствам нулю скалярных произведения вектора $\rho(A)z$ и каждого из векторов $y$, $A^T y$,
$\left ( A^T \right ) ^2 y$, \dots, $\left ( A^T \right )^{n-1} y$:

	$$
		\left \{
			\begin{array}{ccc}
				( \rho(A) z , & y                        & ) = 0\\
				( \rho(A) z , & A^T y                    & ) = 0\\
				( \rho(A) z , & \left ( A^T \right )^2 y & ) = 0\\
				\vdots \\
				( \rho(A) z , & \left ( A^T \right )^{n-1} y & ) = 0\\
			\end{array}
		\right .
		,
	$$

или в развернутом виде:

	$$
		\left \{
			\begin{array}{ccc}
				( \rho_0 z + \rho_1 A z + \rho_2 A^2 z + \dots + \rho_r A^r z , & y & ) = 0 \\
				( \rho_0 z + \rho_1 A z + \rho_2 A^2 z + \dots + \rho_r A^r z , & A^T y & ) = 0 \\
				( \rho_0 z + \rho_1 A z + \rho_2 A^2 z + \dots + \rho_r A^r z , & \left ( A^T \right )^2 y & ) = 0 \\
				\vdots \\
				( \rho_0 z + \rho_1 A z + \rho_2 A^2 z + \dots + \rho_r A^r z , & \left ( A^T \right )^{n-1} y & ) = 0 \\
			\end{array}
		\right .
		.
	$$

Преобразуем полученную систему к следующему виду:

	$$
		\left \{
			\begin{array}{ccc}
				y^T         & \left ( \rho_0 z + \rho_1 A z + \rho_2 A^2 z + \dots + \rho_r A^r z \right ) & = 0 \\
				y^T A       & \left ( \rho_0 z + \rho_1 A z + \rho_2 A^2 z + \dots + \rho_r A^r z \right ) & = 0 \\
				y^T A^2     & \left ( \rho_0 z + \rho_1 A z + \rho_2 A^2 z + \dots + \rho_r A^r z \right ) & = 0 \\
				\vdots \\
				y^T A^{n-1} & \left ( \rho_0 z + \rho_1 A z + \rho_2 A^2 z + \dots + \rho_r A^r z \right ) & = 0 \\
			\end{array}
		\right .
		.
	$$

Раскрывая скобки и выполняя умножение, получим систему:

	$$
		\left \{
			\begin{array}{ccccccccccccc}
				y^T z         & \rho_0 & + & y^T A z         & \rho_1 & + & y^T A^2 z       & \rho_2 & + & \dots & + & y^T A^r z & \rho_r = 0 \\
				y^T A z       & \rho_0 & + & y^T A^2 z       & \rho_1 & + & y^T A^3 z       & \rho_2 & + & \dots & + & y^T A^{1+r} z & \rho_r = 0 \\
				y^T A^2 z     & \rho_0 & + & y^T A^3 z       & \rho_1 & + & y^T A^4 z       & \rho_2 & + & \dots & + & y^T A^{2+r} z & \rho_r = 0 \\
				\vdots \\
				y^T A^{n-1} z & \rho_0 & + & y^T A^{n-1+1} z & \rho_1 & + & y^T A^{n-1+2} z & \rho_2 & + & \dots & + & y^T A^{n-1+r} z & \rho_r = 0 \\
			\end{array}
		\right .
	$$

Таким образом, вектор коэффициентов $( \rho_0, \rho_1, \rho_2, \dots, \rho_r )$ удовлетворяет однородной системе:

	$$
		\begin{pmatrix}
			y^T z         & y^T A z         & y^T A^2 z       & \ldots & y^T A^n z \\
			y^T A z       & y^T A^2 z       & y^T A^3 z       & \ldots & y^T A^{n+1} z \\
			y^T A^2 z     & y^T A^3 z       & y^T A^4 z       & \ldots & y^T A^{n+2} z \\
			\vdots        & \vdots          & \vdots          & \ddots & \vdots \\
			y^T A^{n-1} z & y^T A^{n-1+1} z & y^T A^{n-1+2} z & \ldots & y^T A^{n-1+r} z \\
		\end{pmatrix}
		\begin{pmatrix}
			\rho_0 \\
			\rho_1 \\
			\rho_2 \\
			\vdots \\
			\rho_r \\
		\end{pmatrix}
		=
		\begin{pmatrix}
			0 \\
			0 \\
			0 \\
			\vdots \\
			0 \\
		\end{pmatrix}
		.
	$$

Обозначим

	$$ \alpha_{i+j} = y^T A^{i+j} z, $$
	$$ 0 \le i+j \le n-1+r, $$

тогда

	$$
		\begin{pmatrix}
			\alpha_0     & \alpha_1 & \alpha_2     & \ldots & \alpha_r \\
			\alpha_1     & \alpha_2 & \alpha_3     & \ldots & \alpha_{1+r} \\
			\alpha_2     & \alpha_3 & \alpha_4     & \ldots & \alpha_{2+r} \\
			\vdots       & \vdots   & \vdots       & \ddots & \vdots \\
			\alpha_{n-1} & \alpha_n & \alpha_{n+1} & \ldots & \alpha_{n-1+r} \\
		\end{pmatrix}
		\begin{pmatrix}
			\rho_0 \\
			\rho_1 \\
			\rho_2 \\
			\vdots \\
			\rho_r \\
		\end{pmatrix}
		=
		\begin{pmatrix}
			0 \\
			0 \\
			0 \\
			\vdots \\
			0 \\
		\end{pmatrix}
		.
	$$

Легко видеть, что матрица системы имеет ганкелеву структуру. Если же переставить строки в обратном порядке, то получится система

	$$
		\begin{pmatrix}
			\alpha_{n-1} & \alpha_n & \alpha_{n+1} & \ldots & \alpha_{n-1+r} \\
			\vdots       & \vdots   & \vdots       & \ddots & \vdots \\
			\alpha_2     & \alpha_3 & \alpha_4     & \ldots & \alpha_{n+1} \\
			\alpha_1     & \alpha_2 & \alpha_3     & \ldots & \alpha_n \\
			\alpha_0     & \alpha_1 & \alpha_2     & \ldots & \alpha_{n-1} \\
		\end{pmatrix}
		\begin{pmatrix}
			\rho_0 \\
			\rho_1 \\
			\rho_2 \\
			\vdots \\
			\rho_r \\
		\end{pmatrix}
		=
		\begin{pmatrix}
			0 \\
			0 \\
			0 \\
			\vdots \\
			0 \\
		\end{pmatrix}
	$$

матрица, которой является теплицевой.

Для систем с ганкелевыми и теплицевыми матрицами известны эффективные методы нахождения решений, применение которых приводит к вектору
$( \rho_0, \rho_1, \rho_2, \ldots, \rho_r )$ коэффициентов проекционно аннулирующего полинома $\polynomial{\rho}$.

В своей работе \cite{Wiedemann} Видеман предлагает использовать алгоритм Берлекемпа--Месси, обсуждаемый в разделе
\ref{section:FPAP:finding_projection_annihilating_polynomials}, для решения системы с ганкелевой матрицей.

\section{Решение задачи нахождения проекционно аннулирущего полинома} \label{section:FPAP:finding_projection_annihilating_polynomials}

В предыдущем разделе было показано, что коэффициенты $\rho_i$ проекционно аннулирующего полинома $\polynomial{\rho}$ удовлетворяют системе
уравнений:

	\begin{equation} \label{equation:FPAP:equations_for_rho}
		\begin{pmatrix}
			\alpha_0     & \alpha_1 & \alpha_2     & \ldots & \alpha_r \\
			\alpha_1     & \alpha_2 & \alpha_3     & \ldots & \alpha_{1+r} \\
			\alpha_2     & \alpha_3 & \alpha_4     & \ldots & \alpha_{2+r} \\
			\vdots       & \vdots   & \vdots       & \ddots & \vdots \\
			\alpha_{n-1} & \alpha_n & \alpha_{n+1} & \ldots & \alpha_{n-1+r} \\
		\end{pmatrix}
		\begin{pmatrix}
			\rho_0 \\
			\rho_1 \\
			\rho_2 \\
			\vdots \\
			\rho_r \\
		\end{pmatrix}
		=
		\begin{pmatrix}
			0 \\
			0 \\
			0 \\
			\vdots \\
			0 \\
		\end{pmatrix}
		.
	\end{equation}

Дополнительно будем предполагать, что число $r$ выбрано таким образом, что $\rho_r \neq 0$. Сделанное предположение оправдывается тем, что
если вектор $(\rho_0, \rho_1, \rho_2, \dots, \rho_r)$ является нетривиальным решением системы \eqref{equation:FPAP:equations_for_rho}, то
не ограничивая общности, можно считать, что коэффициент $\rho_r \neq 0$, поскольку в случае $\rho_r = 0$ всегда можно для $r$ выбрать
меньшее значение, при котором $\rho_r \neq 0$, отбросив при этом несколько последних столбцов матрицы системы
\eqref{equation:FPAP:equations_for_rho} и несколько последних коэффициентов $\rho_i$. Далее, поскольку $\rho_r \neq 0$, то в поле $\gfp$
существует обратный по умножению элемент $\rho_r^{-1}$, и тогда если вектор $(\rho_0, \rho_1, \rho_2, \dots, \rho_{r-1}, \rho_r)$ является
решением однородной системы \eqref{equation:FPAP:equations_for_rho}, то вектор:

	$$
		\begin{pmatrix}
			\rho_r^{-1} \rho_0 \\
			\rho_r^{-1} \rho_1 \\
			\rho_r^{-1} \rho_2 \\
			\vdots \\
			\rho_r^{-1} \rho_{r-1} \\
			\rho_r^{-1} \rho_{r} \\
		\end{pmatrix}
		=
		\begin{pmatrix}
			\rho_r^{-1} \rho_0 \\
			\rho_r^{-1} \rho_1 \\
			\rho_r^{-1} \rho_2 \\
			\vdots \\
			\rho_r^{-1} \rho_{r-1} \\
			1 \\
		\end{pmatrix}
	$$

также удовлетворяет системе \eqref{equation:FPAP:equations_for_rho}.

Таким образом, если система \eqref{equation:FPAP:equations_for_rho}
имеет нетривиальное решение $(\rho_0, \rho_1,$ $\rho_2, \dots, \rho_{r-1}, \rho_r)$, то существует нетривиальное решение, в котором
$\rho_r = 1$, и именно такие решения будем искать в дальнейшем.

Выбор именно последнего коэффициента $\rho_r$ в качестве ненулевого не является случайным: ведь чем больше $r$, тем больше коэффициентов
$\alpha_i$ потребуется вычислить, а вычисление каждого коэффициента $\alpha_i$ требует определенных вычислительных затрат.
Если представить ситуацию, в которой несколько последних коэффициентов $\rho_i$ являются нулевыми, включая и $\rho_r$, то окажется, что
столько же последних коэффициентов $\alpha_i$ вычислены напрасно, поскольку не используются. Если же в полученном решении $\rho_r \neq 0$,
то можно быть абсолютно уверенным в том, что коэффициент $\alpha_{n-1+r}$ используются и должен быть вычислен.

Введем полином $\polynomial{\xi}$, коэффициенты $\xi_i$ которого являются коэффициентами $\rho_i$ полинома $\polynomial{\rho}$ взятыми
в обратном порядке:

	$$ \xi(\lambda) = \lambda^r \rho \left ( \frac{1}{\lambda} \right ) = $$
	$$ = \lambda^r \left ( \rho_0 + \rho_1 \frac{1}{\lambda} + \rho_2 \frac{1}{\lambda^2} + \dots + \rho_r \frac{1}{\lambda^r} \right ) = $$
	$$ = \rho_0 \lambda^r + \rho_1 \lambda^{r-1} + \rho_2 \lambda^{r-2} + \dots + \rho_r = $$
	$$ = \xi_r \lambda^r + \xi_{r-1} \lambda^{r-1} + \xi_{r-2} \lambda^{r-2} + \dots + \xi_0. $$

Таким образом,

	\begin{equation} \label{equation:FPAP:rho_and_xi_equation}
		\begin{pmatrix}
			\rho_0 \\
			\rho_1 \\
			\rho_2 \\
			\vdots \\
			\rho_r
		\end{pmatrix}
		=
		\begin{pmatrix}
			\xi_r \\
			\xi_{r-1} \\
			\xi_{r-2} \\
			\vdots \\
			\xi_0
		\end{pmatrix}
		.
	\end{equation}

Причем согласно предположению $\rho_r = 1$, поэтому $\xi_0 = 1$ и полином $\polynomial{\xi}$ имеет вид:

	$$ \xi(\lambda) = 1 + \dots + \xi_{r-2} \lambda^{r-2} + \xi_{r-1} \lambda^{r-1} + \xi_r \lambda^r. $$

Для множества всех полиномов с коэффициентом 1 при нулевой степени введем специальное обозначение $\gfpunitypolynomial$:

	$$ \gfpunitypolynomial = \left \{ \polynomial{\gamma} \in \gfppolynomial : \degreecoefficient{0}{\polynomial{\gamma}} = 1 \right \}. $$

Таким образом,

	$$ \polynomial{\xi} \in \gfpunitypolynomial. $$

Из равенства \eqref{equation:FPAP:equations_for_rho} следует, что коэффициенты $\xi_i$ полинома $\polynomial{\xi}$ удовлетворяют
системе уравнений:

	\begin{equation} \label{equation:FPAP:equations_for_xi}
		\begin{pmatrix}
			\alpha_0     & \alpha_1 & \alpha_2     & \ldots & \alpha_r \\
			\alpha_1     & \alpha_2 & \alpha_3     & \ldots & \alpha_{1+r} \\
			\alpha_2     & \alpha_3 & \alpha_4     & \ldots & \alpha_{2+r} \\
			\vdots       & \vdots   & \vdots       & \ddots & \vdots \\
			\alpha_{n-1} & \alpha_n & \alpha_{n+1} & \ldots & \alpha_{n-1+r} \\
		\end{pmatrix}
		\begin{pmatrix}
			\xi_r \\
			\xi_{r-1} \\
			\xi_{r-2} \\
			\vdots \\
			\xi_0 \\
		\end{pmatrix}
		=
		\begin{pmatrix}
			0 \\
			0 \\
			0 \\
			\vdots \\
			0 \\
		\end{pmatrix}
		.
	\end{equation}

Системе \eqref{equation:FPAP:equations_for_xi} можно придать следующую интерпретацию: представим, что числа $\alpha_0$, \dots, $\alpha_{n-1+r}$
являются коэффициентами полинома $\polynomial{\alpha}$:

	$$ \alpha(\lambda) = \alpha_0 + \alpha_1 \lambda + \alpha_2 \lambda^2 + \dots + \alpha_{n-1+r} \lambda^{n-1+r}, $$

тогда первая строка в системе \eqref{equation:FPAP:equations_for_xi} означает, что коэффициент при степени $r$ в произведении полиномов
$\polynomial{\alpha} \polynomial{\xi}$ равен нулю:

	$$ \degreecoefficient{r}{\polynomial{\alpha} \polynomial{\xi}}
		= \alpha_0 \xi_r + \alpha_1 \xi_{r-1} + \alpha_2 \xi_{r-2} + \dots + \alpha_r \xi_0 = 0 $$

вторая строка в системе \eqref{equation:FPAP:equations_for_xi} означает, что коэффициент при степени $r+1$ в произведении полиномов
$\polynomial{\alpha} \polynomial{\xi}$ равен нулю:

	$$ \degreecoefficient{r+1}{\polynomial{\alpha} \polynomial{\xi}}
		= \alpha_1 \xi_r + \alpha_2 \xi_{r-1} + \alpha_3 \xi_{r-2} + \dots + \alpha_{r+1} \xi_0 = 0 $$

а все последующие строки означают равенство нулю всех коэффициентов при степенях до степени $n-1+r$ включительно в произведении полиномов
$\polynomial{\alpha} \polynomial{\xi}$.

Таким образом, система \eqref{equation:FPAP:equations_for_xi} эквивалентна системе равенств:

	\begin{equation} \label{equation:FPAP:equations_for_product_coefficients}
		\left \{
			\begin{array}{c}
				\degreecoefficient{r}{\polynomial{\alpha} \polynomial{\xi}} = 0 \\
				\degreecoefficient{r+1}{\polynomial{\alpha} \polynomial{\xi}} = 0 \\
				\vdots \\
				\degreecoefficient{n-1+r}{\polynomial{\alpha} \polynomial{\xi}} = 0 \\
			\end{array}
		\right .
	\end{equation}

Систему равенств \eqref{equation:FPAP:equations_for_product_coefficients} можно коротко записать в виде равенства:

	\begin{equation} \label{equation:FPAP:alpha_xi_product}
		\begin{array}{c}
			\alpha(\lambda) \xi(\lambda) = \delta(\lambda) + \varepsilon(\lambda) \lambda^{n+r}, \\
			\polynomialdegree{\delta} \le r-1, \\
		\end{array}
	\end{equation}

в котором $\polynomial{\delta}$ --- некоторый полином степени не выше $r-1$ и $\polynomial{\varepsilon}$ --- некоторый полином
произвольной степени. Иногда полученное равенство записывают в виде:

	$$ \alpha(\lambda) \xi(\lambda) = \delta(\lambda) \; mod \; \lambda^{n+r}. $$

Таким образом, исходная задача нахождения решения системы \eqref{equation:FPAP:equations_for_rho} сводится к задаче нахождения полинома
$\polynomial{\xi}$ такого, что в произведении $\polynomial{\alpha} \polynomial{\xi}$ отсутствуют степени $r$, $r+1$, \dots, $n-1+r$.

Для наглядности последующего изложения, прежде всего, введем графические схемы вычисления коэффициентов $\delta_i$ произведения полиномов
из равенства \eqref{equation:FPAP:alpha_xi_product}: пусть имеется три полосы --- верхняя, средняя и нижняя, на верхней полосе записаны
коэффициенты $\alpha_i$ полинома $\polynomial{\alpha}$, на средней --- коэффициенты $\xi_i$ полинома $\polynomial{\xi}$, а на нижней
будут записываться коэффициенты $\delta_i$ полинома $\polynomial{\delta}$. Сверху для удобства отсчета раставлены степени $\lambda^k$.

	$$
		\begin{array}{ccccccccccc}
			                      &       &       &       &       &       & \lambda^0 & \lambda^1 & \lambda^2 & \dots & \lambda^{n-1+r} \\
			\hline
			\text{верхняя полоса} &       &       &       &       &       & \alpha_0  & \alpha_1  & \alpha_2  & \dots & \alpha_{n-1+r} \\
			                      &       &       &       &       &       &           &           &           &       & \\
			\text{средняя полоса} & \xi_r & \dots & \xi_2 & \xi_1 & \xi_0 &           &           &           &       & \\
			                      &       &       &       &       &       &           &           &           &       & \\
			\text{нижняя полоса}  &       &       &       &       &       &           &           &           &       & \\
		\end{array}
	$$

Теперь представим как средняя полоса смещается вправо на одну позицию: коэффициенты $\alpha_0$ и $\xi_0$ оказывают напротив друг друга,
перемножаются и результат (коэффициент $\delta_0$) записывается на нижнюю полосу:

	$$
		\begin{array}{ccccccccccc}
			                      &       &       &       &       &       & \lambda^0  & \lambda^1 & \lambda^2 & \dots & \lambda^{n-1+r} \\
			\hline
			\text{верхняя полоса} &       &       &       &       &       & \alpha_0   & \alpha_1  & \alpha_2  & \dots & \alpha_{n-1+r} \\
			                      &       &       &       &       &       & |          &           &           &       & \\
			\text{средняя полоса} &       & \xi_r & \dots & \xi_2 & \xi_1 & \xi_0      &           &           &       & \\
			                      &       &       &       &       &       & \downarrow &           &           &       & \\
			\text{нижняя полоса}  &       &       &       &       &       & \delta_0   &           &           &       & \\
		\end{array}
	$$

Далее средняя полоса смещается еще на одну позицию вправо: коэффициенты $\alpha_0$ и $\xi_1$, а также $\alpha_1$ и $\xi_0$
оказываются друг напротив друга, перемножаются, произведения складываются и результат (коэффициент $\delta_1$) записывается на нижнюю полосу:

	$$
		\begin{array}{ccccccccccc}
			                      &       &       &       &       &       & \lambda^0 & \lambda^1  & \lambda^2 & \dots & \lambda^{n-1+r} \\
			\hline
			\text{верхняя полоса} &       &       &       &       &       & \alpha_0  & \alpha_1   & \alpha_2  & \dots & \alpha_{n-1+r} \\
			                      &       &       &       &       &       & |         & |          &           &       & \\
			\text{средняя полоса} &       &       & \xi_r & \dots & \xi_2 & \xi_1     & \xi_0      &           &       & \\
			                      &       &       &       &       &       &           & \downarrow &           &       & \\
			\text{нижняя полоса}  &       &       &       &       &       & \delta_0  & \delta_1   &           &       & \\
		\end{array}
	$$

Аналогичным образом, для коэффициента $\delta_2$ получим схему:

	$$
		\begin{array}{ccccccccccc}
			                      &       &       &       &       &       & \lambda^0 & \lambda^1  & \lambda^2  & \dots & \lambda^{n-1+r} \\
			\hline
			\text{верхняя полоса} &       &       &       &       &       & \alpha_0  & \alpha_1   & \alpha_2   & \dots & \alpha_{n-1+r} \\
			                      &       &       &       &       &       & |         & |          & |          &       & \\
			\text{средняя полоса} &       &       &       & \xi_r & \dots & \xi_2     & \xi_1      & \xi_0      &       & \\
			                      &       &       &       &       &       &           &            & \downarrow &       & \\
			\text{нижняя полоса}  &       &       &       &       &       & \delta_0  & \delta_1   & \delta_2   &       & \\
		\end{array}
	$$

Далее средняя полоса продолжает движение вправо до тех пор пока коэффициент $\xi_0$ не окажется напротив коэффициента $\alpha_{n-1+r}$.
В общем случае схема вычисления коэффициента $\delta_k$ имеет вид:

	$$
		\begin{array}{ccccccccccc}
			                      & \lambda^0 & \lambda^1 & \dots & \lambda^{k-r} & \dots & \lambda^{k-2} & \lambda^{k-1} & \lambda^k  & \dots & \lambda^{n-1+r} \\
			\hline
			\text{верхняя полоса} & \alpha_0  & \alpha_1  & \dots & \alpha_{k-r}  & \dots & \alpha_{k-2}  & \alpha_{k-1}  & \alpha_k   & \dots & \alpha_{n-1+r} \\
			                      &           &           &       & |             &       & |             & |             & |          &       & \\
			\text{средняя полоса} &           &           &       & \xi_r         & \dots & \xi_2         & \xi_1         & \xi_0      &       & \\
			                      &           &           &       &               &       &               &               & \downarrow &       & \\
			\text{нижняя полоса}  & \delta_0  & \delta_1  &       & \delta_{k-r}  & \dots & \delta_{k-2}  & \delta_{k-1}  & \delta_k   &       & \\
		\end{array}
	$$

От полученных схем вычисления коэффициентов $\delta_i$ произведения $\polynomial{\alpha} \polynomial{\xi}$ легко перейти к наглядному графическому
представлению самого произведения $\polynomial{\alpha} \polynomial{\xi}$, которое используется в работе Томе \cite{Thome}. В этом представлении
вместо $\lambda^k$ будем записывать только степени $k$, верхнюю полосу не будем записывать вовсе, а вместо средней и нижней полосы будем писать
полином средней полосы $\polynomial{\xi}$ и напротив коэффициенты произведения --- $\delta_i$:

	$$
		\begin{array}{cccccccc}
			                 & 0        & 1        & \dots & r-1          & r        & \dots & n-1+r \\
			\hline
			\polynomial{\xi} & \delta_0 & \delta_1 & \dots & \delta_{r-1} & \delta_r & \dots & \delta_{n-1+r} \\
		\end{array}
	$$

Далее еще более упростим полученное представление введя специальные обозначения для коэффициентов $\delta_i$:

\begin{itemize}
	\item [0] --- означает, что $\delta_i = 0$;
	\item [$\diamond$] --- означает, что $\delta_i \neq 0$;
	\item [$\bullet$] --- означает произвольное значение $\delta_i$, которое нас не интересует.
\end{itemize}

С учетом принятых обозначенений равенство \eqref{equation:FPAP:alpha_xi_product} означает, что необходимо найти такой полином
$\polynomial{\xi}$, который имеет схему произведения:

	$$
		\begin{array}{ccccccccc}
			                 & 0       & 1       & \dots & r-1     & r & r+1 & \dots & n-1+r \\
			\hline
			\polynomial{\xi} & \bullet & \bullet & \dots & \bullet & 0 & 0   & \dots & 0 \\
		\end{array}
	$$

Представленная далее процедура построения полинома $\polynomial{\xi}$ является итерационной и основана на известном алгоритме Берлекемпа--Месси,
излагаемого в работе Месси \cite{Massey}.

В процессе выполнения итераций процедура старается находить такие полиномы $\polynomial{\xi^{(k)}} \in \gfpunitypolynomial$,
что в произведениии $\polynomial{\alpha} \polynomial{\xi^{(k)}}$ отсутствуют степени, начиная от степени самого полинома $\polynomial{\xi^{(k)}}$
и до степени $k$ включительно. Цель процедуры, таким образом, заключается в построении такого полинома $\polynomial{\xi^{(q)}}$, который
зануляет в произведении $\polynomial{\alpha} \polynomial{\xi^{(q)}}$ не меньше $n$ степеней, начиная со степени полинома $\polynomial{\xi^{(q)}}$,
и именно такой полином $\polynomial{\xi^{(q)}}$ является одним из возможных полиномов $\polynomial{\xi}$, коэффициенты которого удовлетворяют
системе \eqref{equation:FPAP:equations_for_xi}.

Принадлежность полиномов $\polynomial{\xi^{(k)}}$ множеству $\gfpunitypolynomial$ является существенным ограничением, поскольку оставаясь
в рамках множества $\gfpunitypolynomial$, всегда можно быть уверенным, что построенный полином $\polynomial{\xi^{(k)}}$ не является нулевым
(поскольку по крайней мере коэффициент при нулевой степени у полинома $\polynomial{\xi^{(k)}}$ равен 1), а это значит, что получаемые в результате
обратного преобразования \eqref{equation:FPAP:rho_and_xi_equation} векторы $(\rho_0, \dots, \rho_r)$ имеют ненулевой элемент $\rho_r$ и
следовательно являются нетривиальными решениями исходной системы \eqref{equation:FPAP:equations_for_rho}.

Поскольку полиномы $\polynomial{\xi^{(k)}}$ будут часто встречаться в дальнейшем, введем для них специальное определение.

\begin{definition}
	Полином $\polynomial{\xi}$ называется \definedterm{полиномом, аннулирущим до степени $k$}, если в произведении
	$\polynomial{\alpha} \polynomial{\xi}$ равны нулю коэффициенты при степенях, начиная от степени самого полинома $\polynomial{\xi}$
	и заканчивая степенью $k$ включительно:

		$$ \degreecoefficient{j}{\polynomial{\alpha} \polynomial{\xi}} = 0, $$
		$$ j = \polynomialdegree{\xi}, \dots, k. $$
\end{definition}

Коэффициенты в произведении $\polynomial{\alpha} \polynomial{\xi}$ при степенях меньших степени полинома $\polynomial{\xi}$ не
представляют интереса, поскольку в соответствии уравнениями системы \eqref{equation:FPAP:equations_for_xi} на коэффициенты в произведении
$\polynomial{\alpha} \polynomial{\xi}$ при степенях меньших $r$ никаких условий не налагается, и они могут быть произвольными, поэтому и в
произведении $\polynomial{\alpha} \polynomial{\xi}$ коэффициенты до степени полинома $\polynomial{\xi}$ (не включительно) также могут
быть произвольными.

Заметим, что степень полинома $\polynomial{\xi}$ может оказаться больше $k$ и первые $k$ коэффициентов при степенях с 0 по $k-1$ в произведении
$\polynomial{\alpha} \polynomial{\xi}$ могут быть и не равны нулю, но поскольку они не представляют интереса, то такой полином $\polynomial{\xi}$
будем все таки считать аннулирующим до степени $k-1$ и примем следующее соглашение.

\begin{agreement} \label{agreement:FPAP:ignoring_nonzero_product_coefficients}
	Любой полином $\polynomial{\xi}$ степени $d$ считается аннулирущим до степени $d-1$.
\end{agreement}

В излагаемой далее процедуре построения аннулирущих до различных степеней $k$ полиномов $\polynomial{\xi^{(k)}}$ используются два вспомогательных
полинома: $\polynomial{\xi_p}$, который будем называть предыдущим (индекс $p$ соответствует слову "previous"{}), и $\polynomial{\xi_c}$, который
будем называть текущим (индекс $c$ соответствует слову "current"{}). Степени полиномов $\polynomial{\xi_p}$ и $\polynomial{\xi_c}$ будем обозначать
$d_p$ и $d_c$ соответсвенно.

Первые шаги процедуры связаны с инициализацией этих двух полиномов: шаг, связанный с инициализацией предыдущего полинома $\polynomial{\xi_p}$,
будем обозначать Н.1 (где литера Н обозначает "Начало"{} или "иНициализация"{}, а цифра 1 указывает, что предыдущий полином инициализируется
первым), а шаг, связанный с инициализацией текущего полинома $\polynomial{\xi_c}$, будем обозначать Н.2.

Итак, на шаге Н.1 возьмем в качестве предыдущего полинома $\polynomial{\xi_p}$ константу $1$:

	\begin{equation} \label{equation:FPAP:previous_polynomial_initialization}
		\polynomial{\xi_p} = 1,
	\end{equation}
	$$ d_p = 0, $$

и будем вычислять коэффициенты в произведении $\polynomial{\alpha} \polynomial{\xi_p}$ до тех пор, пока не встретиться ненулевой коэффициент.
В общем случае несколько первых коэффициентов в произведении $\polynomial{\alpha} \polynomial{\xi_p}$ могут оказаться равными нулю,
пусть $m$ обозначает самую младшую степень при которой коэффициент не равен нулю. В этом случае схема произведения
$\polynomial{\alpha} \polynomial{\xi_p}$ (например, для $m=2$) имеет вид:

	$$
		\begin{array}{cccccccc}
			                   & d_p &   & m        &         &         &         & \\
			                   & 0   & 1 & 2        & 3       & 4       & 5       & \dots \\
			\hline
			\polynomial{\xi_p} & 0   & 0 & \diamond & \bullet & \bullet & \bullet & \dots \\
		\end{array}
	$$

Если $m>0$, то полином $\polynomial{\xi^{(0)}}$, аннулирующий до степени 0, совпадает с предыдущим полиномом $\polynomial{\xi_p}$:

	$$ \polynomial{\xi^{(0)}} = \polynomial{\xi_p}. $$

То же самое касается и всех полиномов $\polynomial{\xi^{(k)}}$ для $k \le m-1$:

	$$ \polynomial{\xi^{(k)}} = \polynomial{\xi_p}, $$
	$$ k \le m-1. $$

Заметим, что, как и требовалось, все построенные полиномы $\polynomial{\xi^{(k)}}$ принадлежат множеству $\gfpunitypolynomial$:

	$$ \polynomial{\xi^{(k)}} \in \gfpunitypolynomial, $$
	$$ k \le m-1, $$

поскольку $\polynomial{\xi_p} = 1$.

Если $n-1 \le m$, то процедура построения завершается и в качестве искомого полинома $\polynomial{\xi}$ следует использовать полином
$\polynomial{\xi_p}$:

	$$ \polynomial{\xi} = \polynomial{\xi_p} = 1. $$

Этим завершается шаг Н.1 и прежде чем переходить к следующему шагу Н.2 проанализируем результаты, полученные на шаге Н.1.

Итак, если $m > 1$, то удалось построить аннулирующие полиномы $\polynomial{\xi^{(0)}}$, \dots, $\polynomial{\xi^{(m-1)}}$, если же $m=0$, то ни
одного аннулирущего полинома построить не удалось.

Кроме того следует сделать важное замечание, которое касается коэффициентов $\alpha_i$: $m$ --- наименьшая степень в произведении
$\polynomial{\alpha} \polynomial{\xi_p}$ имеющая ненулевой коэффициент, следовательно, все предыдущие коэффициенты при
степенях 0, \dots, $m-1$ в произведении $\polynomial{\alpha} \polynomial{\xi_p}$ равны нулю, а коэффициент при степени $m$ нулю не равен:

	$$
		\left \{
			\begin{array}{c}
				\degreecoefficient{0}{\polynomial{\alpha} \polynomial{\xi_p}} = 0, \\
				\degreecoefficient{1}{\polynomial{\alpha} \polynomial{\xi_p}} = 0, \\
				\vdots \\
				\degreecoefficient{m-1}{\polynomial{\alpha} \polynomial{\xi_p}} = 0\\
				\degreecoefficient{m}{\polynomial{\alpha} \polynomial{\xi_p}} \neq 0\\
			\end{array}
		\right .
		.
	$$

Поскольку предыдущий полином $\polynomial{\xi_p} = 1$ в соответствии с равенством \eqref{equation:FPAP:previous_polynomial_initialization}, то

	\begin{equation} \label{equation:FPAP:first_m_alpha_coefficients}
		\left \{
			\begin{array}{c}
				\degreecoefficient{0}{\polynomial{\alpha}} = 0, \\
				\degreecoefficient{1}{\polynomial{\alpha}} = 0, \\
				\vdots \\
				\degreecoefficient{m-1}{\polynomial{\alpha}} = 0\\
				\degreecoefficient{m}{\polynomial{\alpha}} \neq 0\\
			\end{array}
		\right .
		\Leftrightarrow
		\left \{
			\begin{array}{c}
				\alpha_0 = 0, \\
				\alpha_1 = 0, \\
				\vdots \\
				\alpha_{m-1} = 0\\
				\alpha_{m} \neq 0\\
			\end{array}
		\right .
	\end{equation}

Полученные соотношения для коэффициентов $\alpha_0$, \dots, $\alpha_m$ играют важную роль на шаге Н.2 инициализации текущего полинома
$\polynomial{\xi_c}$.

По завершению шага Н.1 процедура построения остановилась и не смогла построить полином $\polynomial{\xi^{(m)}} \in \gfpunitypolynomial$,
аннулирующий до степени $m$, поэтому основной целью  шага Н.2 является построение этого полинома. Способ построения аналогичен шагу Н.1:
текущий полином $\polynomial{\xi_c}$ инициализируется (подбирается) таким образом, чтобы он являлся аннулирущим до степени $m$ и имел
коэффициент при нулевой степени равный 1, и затем используется в качестве полинома $\polynomial{\xi^{(m)}}$.

При подборе текущего полинома $\polynomial{\xi_c}$ существенную роль играют равенства \eqref{equation:FPAP:first_m_alpha_coefficients}:
из-за того, что среди коэффициентов $\alpha_0$, \dots, $\alpha_m$ лишь один коэффициент $\alpha_m$ не равен нулю, его нечем "занулить"{}
при вычислении коэффициента при степени $m$ в произведении $\polynomial{\alpha} \polynomial{\xi_c}$, поэтому приходится искусственным образом
повышать степень текущего полинома $\polynomial{\xi_c}$ до $m+1$, чтобы текущий полином $\polynomial{\xi_c}$ стал аннулирующим до степени $m$
по соглашению \ref{agreement:FPAP:ignoring_nonzero_product_coefficients}.

Формальное доказательство того факта, что аннулирующий до степени $m$ текущий полином $\polynomial{\xi_c} \in \gfpunitypolynomial$ должен
обязательно иметь степень больше $m$ можно провести от противного: пусть текущий полином $\polynomial{\xi_c}$, аннулирущий до степени $m$
и имеющий коэффициент при нулевой степени равный 1, имеет степень меньшую или равную $m$. Представим текущий полином $\polynomial{\xi_c}$
в виде:

	$$ \xi_c(\lambda) = 1 + \xi_{c,1} \lambda + \xi_{c,2} \lambda^2 + \dots \xi_{c,m} \lambda^m, $$

где некоторые коэффициенты из $\xi_{c,1}$, $\xi_{c,2}$, \dots, $\xi_{c,m}$ могут быть нулевыми. Попробуем вычислить коэффициент при степени
$m$ в произведении $\polynomial{\alpha} \polynomial{\xi_c}$:

	$$
		\begin{array}{c}
			\degreecoefficient{m}{\polynomial{\alpha} \polynomial{\xi_c}} = \\
			= \alpha_m \cdot 1 + \alpha_{m-1} \cdot \xi_{c,1} + \alpha_{m-2} \cdot \xi_{c,2} + \dots + \alpha_0 \cdot \xi_{c,m} = \\
			= \alpha_m \cdot 1 + 0 \cdot \xi_{c,1} + 0 \cdot \xi_{c,2} + \dots + 0 \cdot \xi_{c,m} = \\
			= \alpha_m \\
		\end{array}
	$$

поскольку в силу равенств \eqref{equation:FPAP:first_m_alpha_coefficients}:

	$$ \alpha_0 = \alpha_1 = \dots = \alpha_{m-1} = 0. $$

Таким образом, оказывается, что коэффициент при степени $m$ не равен нулю, поскольку:

	$$ \degreecoefficient{m}{\polynomial{\alpha} \polynomial{\xi^{(m)}}} = \alpha_m \neq 0 $$

опять же в силу равенств \eqref{equation:FPAP:first_m_alpha_coefficients}.

Получилось противоречие: аннулирующий до степени $m$ полином $\polynomial{\xi_c}$ имеет ненулевой коэффициент при степени $m$ в произведении
$\polynomial{\alpha} \polynomial{\xi_c}$. Cтало быть, исходное предположение о степени полинома $\polynomial{\xi^{(m)}}$ является неверным, и
потому полином $\polynomial{\xi_c}$ обязательно должен иметь степень больше $m$.

Раз степень текущего полинома $\polynomial{\xi_c}$ должна быть больше $m$, то постараемся в качестве текущего полинома $\polynomial{\xi_c}$
подобрать такой полином степени $m+1$, который являлся бы аннулирущим до степени $m+1$. Оказывается, что подобрать такой полином не представляет
большого труда:

\begin{itemize}
	\item если $\alpha_{m+1} = 0$, то в качестве текущего полинома возьмем полином:

		$$ \xi_c(\lambda) = 1 + \lambda^{m+1}, $$

	\item если $\alpha_{m+1} \neq 0$,  то в качестве текущего полинома будем использовать полином:

		$$ \xi_c(\lambda) = 1 - \alpha_{m+1}\alpha_m^{-1} \lambda + \lambda^{m+1} $$
\end{itemize}

Заметим, что приведенное определение для текущего полинома является корректным, поскольку $\alpha_m \neq 0$, и в поле $\gfp$ существует
обратный по умножению элемент $\alpha_m^{-1}$.

В первом и втором случаях степень $d_c$ текущего полинома $\polynomial{\xi_c}$:

	$$ d_c = m+1. $$

Кроме того, в том и другом случаях коэффициент при нулевой степени равен 1, то есть

	$$ \polynomial{\xi_c} \in \gfpunitypolynomial, $$

и в том и другом случаях коэффициент при степени $m+1$ в произведении $\polynomial{\alpha} \polynomial{\xi_c}$ равен 0:
действительно, в первом случае,

	$$ \degreecoefficient{m}{\polynomial{\alpha} \polynomial{\xi_c}} = \alpha_{m+1} \cdot 1 + \alpha_m \cdot 0 = \alpha_{m+1} = 0, $$

а во втором случае,

	$$ \degreecoefficient{m}{\polynomial{\alpha} \polynomial{\xi_c}}
		= \alpha_{m+1} \cdot 1 + \alpha_m \cdot ( - \alpha_{m+1} \alpha_m^{-1} )
		= \alpha_{m+1} - \alpha_{m+1} = 0. $$

Следовательно полином $\polynomial{\xi_c}$ по построению является аннулирующим до степени $m+1$ полиномом из множества $\gfpunitypolynomial$,
и можно в качестве $\polynomial{\xi^{(m+1)}}$ взять полином $\polynomial{\xi_c}$:

	$$ \polynomial{\xi^{(m+1)}} = \polynomial{\xi_c}. $$

Коэффициент при степени $m$ в произведении $\polynomial{\alpha} \polynomial{\xi_c}$ нулю не равен, но поскольку степень текущего
полинома $\polynomial{\xi_c}$, равная $m+1$, больше $m$, то по соглашению \ref{agreement:FPAP:ignoring_nonzero_product_coefficients} о том,
что коэффициенты при меньших степенях нас не интересуют, полином $\polynomial{\xi_c}$ степени $m+1$ считается аннулирующим до степени $m$,
поэтому в качестве полинома $\polynomial{\xi^{(m)}}$ можно использовать текущий полином $\polynomial{\xi_c}$:

	$$ \polynomial{\xi^{(m)}} = \polynomial{\xi_c}. $$

Ничего другого не остается, поскольку в множестве $\gfpunitypolynomial$, как уже было установлено ранее, не существует аннулирующих до
степени $m$ полиномов, степень которых меньше или равна $m$.

Заметим, что по построению текущий полином $\polynomial{\xi_c} \in \gfpunitypolynomial$, поэтому, как и требовалось:

	$$ \polynomial{\xi^{(m)}} \in \gfpunitypolynomial, $$
	$$ \polynomial{\xi^{(m+1)}} \in \gfpunitypolynomial. $$

На этом заканчивается шаг Н.2, и по совокупным итогам шагов Н.1 и Н.2 построены все аннулирующие полиномы $\polynomial{\xi^{(k)}}$ вплоть
до $k = m+1$.

Обратим внимание, что для полинома $\polynomial{\xi_c}$ схема произведения $\polynomial{\alpha} \polynomial{\xi_c}$ имеет вид
(например, для $m=2$):

	$$
		\begin{array}{cccccccc}
			                   &         &         & m       & m+1 &         &         & \\
			                   & 0       & 1       & 2       & 3   & 4       & 5       & \dots \\
			\hline
			\polynomial{\xi_c} & \bullet & \bullet & \bullet & 0   & \bullet & \bullet & \dots \\
		\end{array}
	$$

а общая схема с предыдущим полиномом $\polynomial{\xi_p}$ имеет вид:

	$$
		\begin{array}{cccccccc}
			                   &         &         & m        & m+1     &         &         & \\
			                   & 0       & 1       & 2        & 3       & 4       & 5       & \dots \\
			\hline
			\polynomial{\xi_p} & 0       & 0       & \diamond & \bullet & \bullet & \bullet & \dots \\
			\polynomial{\xi_c} & \bullet & \bullet & \bullet  & 0       & \bullet & \bullet & \dots \\
		\end{array}
	$$

Последняя схема является типичной для выполнения итерационных шагов алгоритма, поэтому шаг инициализации на этом заканчивается и начинается
итерационный шаг.

В начале итерационного шага рассматривается некоторая степень $k$ при общей схеме произведения, имеющий, например, вид:

	$$
		\begin{array}{cccccccccccc}
			                   &         & d_p     &         & m            &         & d_c     &         &         & k       &         & \\
			                   & 0       & 1       & 2       & 3            & 4        & 5       & 6       & 7       & 8       & 9       & \dots \\
			\hline
			\polynomial{\xi_p} & \bullet & 0       & 0       & \delta_{p,m} & \bullet & \bullet & \bullet & \bullet & \bullet & \bullet & \dots \\
			\polynomial{\xi_c} & \bullet & \bullet & \bullet & \bullet      & \bullet & 0       & 0       & 0       & \bullet & \bullet & \dots \\
		\end{array}
	$$

в которой выполняется следующие условия:

\begin{conditions} \label{conditions:FPAP:iteration_entrance}
	\begin{enumerate} 
		\item предыдущий полином $\polynomial{\xi_p}$, имеющий степень $d_p$, зануляет коэффициенты в произведении
			$\polynomial{\alpha} \polynomial{\xi_p}$ от степени $d_p$ до степени $m-1$ включительно, а коэффициент $\delta_{p,m}$
			при степени $m$ в произведении $\polynomial{\alpha} \polynomial{\xi_p}$ нулю не равен:

				$$ \delta_{p,m} \neq 0; $$

		\item текущий полином $\polynomial{\xi_c}$, имеющий степень $d_c$, зануляет коэффициенты в произведении
			$\polynomial{\alpha} \polynomial{\xi_c}$ от степени $d_c$ до степени $k-1$ включительно;

		\item текущий полином $\polynomial{\xi_c}$ принадлежит множеству $\gfpunitypolynomial$:

			$$ \polynomial{\xi_c} \in \gfpunitypolynomial. $$

		\item $k > m$
	\end{enumerate}
\end{conditions}

Вполне допустимо, что $d_p = m$ и между $d_p$ и $m$ нет нулей в схеме умножения для предыдущего полинома $\polynomial{\xi_p}$ (такая
ситуация может возникнуть, например, если на шаге Н.1 $m=0$). Принципиально важным является лишь то, что коэффициент $\delta_{p,m}$ не равен нулю.

Легко видеть, что в описываемую схему укладывается и схема, полученная после двух шагов инициализации Н.1 и Н.2: действительно, достаточно считать,
что степень предыдущего полинома $d_p = 0$, $m$ --- произвольное, но не меньше 0 ($m \ge d_p$), степень текущего полинома $d_c = m+1$ и
$k$ --- следующее за $m+1$ значение ( $k = m+2$ ), кроме того по построению текущий полином $\polynomial{\xi_c} \in \gfpunitypolynomial$.

Целью итерационного шага является построение полинома $\polynomial{\xi^{(k)}}$, аннулирующего до степени $k$ и принадлежащего
множеству $\gfpunitypolynomial$, и первое действие итерационного шага заключается в вычислении коэффициента
$\degreecoefficient{k}{\polynomial{\alpha} \polynomial{\xi_c}}$ при степени $k$ в произведении $\polynomial{\alpha} \polynomial{\xi_c}$.
Далее итерационный шаг разделяется на два взаимоисключающих варианта, которые будем обозначать И.1 и И.2 (от слова "Итерация"{}):

\begin{itemize}
	\item [И.1] выполняется, если $\degreecoefficient{m}{\polynomial{\alpha} \polynomial{\xi_c}} = 0$;
	\item [И.2] выполняется, если $\degreecoefficient{m}{\polynomial{\alpha} \polynomial{\xi_c}} \neq 0$.
\end{itemize}

Вариант И.1 является очень простым: если коэффициент $\degreecoefficient{m}{\polynomial{\alpha} \polynomial{\xi_c}} = 0$, то схема умножения
имеет вид:

	$$
		\begin{array}{cccccccccccc}
			                   &         & d_p     &         & m            &         & d_c     &         &         & k       &         & \\
			                   & 0       & 1       & 2       & 3            & 4        & 5       & 6       & 7       & 8       & 9       & \dots \\
			\hline
			\polynomial{\xi_p} & \bullet & 0       & 0       & \delta_{p,m} & \bullet & \bullet & \bullet & \bullet & \bullet & \bullet & \dots \\
			\polynomial{\xi_c} & \bullet & \bullet & \bullet & \bullet      & \bullet & 0       & 0       & 0       & 0       & \bullet & \dots \\
		\end{array}
	$$

и текущий полином $\polynomial{\xi_c}$ является аннулирущим до степени $k$, поэтому в качестве полинома $\polynomial{\xi^{(k)}}$ следует взять
полином $\polynomial{\xi_c}$ :

	$$ \polynomial{\xi^{(k)}} = \polynomial{\xi_c} $$

и на этом вариант И.1 итерационного шага окончен и необходимо перейти к проверке критерия завершения.

Легко видеть, что схема произведения по окончании варианта И.1 итерационного шага имеет тип, соответствующий началу итерационного шага,
достаточно лишь вместо $k$ взять $k+1$.

Вариант И.2 оказывается чуть более сложным: если $\degreecoefficient{m}{\polynomial{\alpha} \polynomial{\xi_c}} \neq 0$, то схема умножения
имеет вид:

	$$
		\begin{array}{cccccccccccc}
			                   &         & d_p     &         & m            &         & d_c     &         &         & k            &         & \\
			                   & 0       & 1       & 2       & 3            & 4       & 5       & 6       & 7       & 8            & 9       & \dots \\
			\hline
			\polynomial{\xi_p} & \bullet & 0       & 0       & \delta_{p,m} & \bullet & \bullet & \bullet & \bullet & \bullet      & \bullet & \dots \\
			\polynomial{\xi_c} & \bullet & \bullet & \bullet & \bullet      & \bullet & 0       & 0       & 0       & \delta_{c,k} & \bullet & \dots \\
		\end{array}
	$$

где

	$$ \delta_{c,k} = \degreecoefficient{k}{\polynomial{\alpha} \polynomial{\xi_c}} \neq 0. $$

В данном случае необходимо коэффициент $\delta_{c,k}$ занулить с помощью ненулевого коэффициента $\delta_{p,m}$. Прежде всего, умножим произведение
$\polynomial{\alpha} \polynomial{\xi_p}$ на $\lambda^{k-m}$: в результате такого умножения все степени увеличаться на $k-m$, таким образом,
все коэффициенты произведения $\polynomial{\alpha} \polynomial{\xi_p}$ сдвинуться на $k-m$, а младших степеней до $k-m$ не будет
вовсе (все коэффициенты при этих степенях будут равны нулю):

	$$
		\begin{array}{rccccccccccc}
			                                 &         & d_p     &         & m            &         & d_c     &         &         & k            & \\
			                                 & 0       & 1       & 2       & 3            & 4       & 5       & 6       & 7       & 8            & 9       & \dots \\
			\hline
			\polynomial{\xi_p}               & \bullet & 0       & 0       & \delta_{p,m} & \bullet & \bullet & \bullet & \bullet & \bullet      & \bullet & \dots \\
			\lambda^{k-m} \polynomial{\xi_p} & 0       & 0       & 0       & 0            & 0       & \bullet & 0       & 0       & \delta_{p,m} & \bullet & \dots \\
			\polynomial{\xi_c}               & \bullet & \bullet & \bullet & \bullet      & \bullet & 0       & 0       & 0       & \delta_{c,k} & \bullet & \dots \\
		\end{array}
	$$

В полученной схеме коэффициенты $\delta_{p,m}$ и $\delta_{c,k}$ являются коэффициентами при одной и той же степени $k$.
Остается лишь вычислить такую комбинацию сдвинутого предыдущего полинома $\lambda^{k-m} \polynomial{\xi_p}$ и текущего полинома
$\polynomial{\xi_c}$, в которой коэффициент при степени $k$ является нулевым. Такой комбинацией является разность $\polynomial{\xi_d}$:

	$$ \xi_d(\lambda) = \xi_c(\lambda) - \delta_{c,k} \delta_{p,m}^{-1} \lambda^{k-m} \xi_p(\lambda), $$

и как не трудно заметить, для представленной разности $\polynomial{\xi_d}$ коэффициент при степени $k$ в произведении
$\polynomial{\alpha} \polynomial{\xi_d}$ равен 0, действительно:

	$$ \degreecoefficient{k}{\polynomial{\alpha} \polynomial{\xi_d}} = $$
	$$ = \degreecoefficient{k}{\polynomial{\alpha} \left ( \polynomial{\xi_c} - \delta_{c,k} \delta_{p,m}^{-1} \lambda^{k-m} \polynomial{\xi_p} \right ) } = $$
	$$ = \degreecoefficient{k}{\polynomial{\alpha} \polynomial{\xi_c}
		- \delta_{c,k} \delta_{p,m}^{-1} \polynomial{\alpha} \lambda^{k-m} \polynomial{\xi_p}} = $$
	$$ = \degreecoefficient{k}{\polynomial{\alpha} \polynomial{\xi_c}}
		- \delta_{c,k} \delta_{p,m}^{-1} \degreecoefficient{k}{\polynomial{\alpha} \lambda^{k-m} \polynomial{\xi_p}} = $$
	\begin{equation} \label{equation:FPAP:annihilating_of_kth_degree_in_difference_polynomial_product}
		= \delta_{c,k} - \delta_{c,k} \delta_{p,m}^{-1} \delta_{p,m} = \delta_{c,k} - \delta_{c,k} = 0.
	\end{equation}

Кроме того, полином разности $\polynomial{\xi_d}$ принадлежит множеству $\gfpunitypolynomial$, действительно, коэффициент при нулевой степени
полинома разности $\polynomial{\xi_d}$ складывается из коэффициентов при нулевой степени текущего полинома $\polynomial{\xi_c}$ и
сдвинутого предыдущего полинома $\lambda^{k-m} \polynomial{\xi_p}$:

	$$ \degreecoefficient{0}{\polynomial{\xi_d}}
		= \degreecoefficient{0}{\polynomial{\xi_c}}
			- \delta_{c,k} \delta_{p,m}^{-1} \degreecoefficient{0}{\lambda^{k-m} \polynomial{\xi_p}}, $$

поскольку $k > m$ (по условиям \ref{conditions:FPAP:iteration_entrance} начала итеационного шага), то сдвинутый предыдущий полином
$\lambda^{k-m} \polynomial{\xi_p}$ "начинается"{} со степени $k-m > 0$, то есть нулевой степени, в сдвинутом предыдущем полиноме
$\lambda^{k-m} \polynomial{\xi_p}$ нет и коэффициент при нулевой степени равен нулю:

	$$ \degreecoefficient{0}{\xi_d} = \degreecoefficient{0}{\xi_c} - \delta_{c,k} \delta_{p,m}^{-1} \cdot 0 = \degreecoefficient{0}{\xi_c} = 1, $$

поскольку $\polynomial{\xi_c} \in \gfpunitypolynomial$ в соответствии с условиями \ref{conditions:FPAP:iteration_entrance} начала итерационного шага.
Таким образом

	\begin{equation} \label{equation:FPAP:difference_polynomial_belongs_to_unity_polynomials}
		\polynomial{\xi_d} \in \gfpunitypolynomial.
	\end{equation}

Общая схема произведений с полиномом разности $\polynomial{\xi_d}$ имеет вид:

	$$
		\begin{array}{rccccccccccc}
			                                 &         & d_p     &         & m            &         & d_c     &         &         & k            & \\
			                                 & 0       & 1       & 2       & 3            & 4       & 5       & 6       & 7       & 8            & 9       & \dots \\
			\hline
			\polynomial{\xi_p}               & \bullet & 0       & 0       & \delta_{p,m} & \bullet & \bullet & \bullet & \bullet & \bullet      & \bullet & \dots \\
			\lambda^{k-m} \polynomial{\xi_p} & 0       & 0       & 0       & 0            & 0       & \bullet & 0       & 0       & \delta_{p,m} & \bullet & \dots \\
			\polynomial{\xi_c}               & \bullet & \bullet & \bullet & \bullet      & \bullet & 0       & 0       & 0       & \delta_{c,k} & \bullet & \dots \\
			\polynomial{\xi_d}               & \bullet & \bullet & \bullet & \bullet      & \bullet & \bullet & 0       & 0       & 0            & \bullet & \dots \\
		\end{array}
	$$

По завершению варианта И.2 итерационного шага необходимо из четырех полиномов: $\polynomial{\xi_p}$, $\lambda^{k-m} \polynomial{\xi_p}$,
$\polynomial{\xi_c}$ и $\polynomial{\xi_d}$ оставить только два. Вполне очевидно, что разность $\polynomial{\xi_d}$ следует сделать новым
текущим полиномом. Выбор предыдущего полинома оказывается чуть более сложным: если степень сдвинутого полинома $\lambda^{k-m} \polynomial{\xi_p}$
оказывается больше степени текущего полинома $\polynomial{\xi_c}$, то предыдущий полином $\polynomial{\xi_p}$ следует заменить текущим полиномом
$\polynomial{\xi_c}$, в противном случае предыдущий полином $\polynomial{\xi_p}$ следует оставить без изменений.

В настоящий момент, конечно, не вполне ясно почему вообще следует менять предыдущий полином $\polynomial{\xi_p}$, ведь его можно было
использовать аналогичным образом и дальше, зануляя ненулевые коэффициенты в произведениях с текущим полиномом $\polynomial{\xi_c}$. Чуть позже
в подразделе \ref{section:MDAP:minimal_degree_of_annihilating_polynomial} будет показано, что это требуется для того, чтобы степени
полиномов $\polynomial{\xi^{(k)}}$ были минимальными.

Таким образом, на заключительном этапе варианта И.2 итерационного шага, следует выполнить присваивания:

	\begin{equation} \label{equation:FPAP:previous_polynomial_assignment}
		\polynomial{\xi_p}
		\leftarrow
		\left \{
			\begin{array}{ccc}
				\polynomial{\xi_p}, & \text{если} & d_p + k - m \le d_c \\
				\polynomial{\xi_c}, & \text{если} & d_p + k - m > d_c \\
			\end{array}
		\right .
		,
	\end{equation}
	\begin{equation} \label{equation:FPAP:current_polynomial_assignment}
		\polynomial{\xi_c} \leftarrow \polynomial{\xi_d}.
	\end{equation}

Соответствующим образом изменяются и степени полиномов:

	$$ d_p = \polynomialdegree{\xi_p}, $$
	$$ d_c = \polynomialdegree{\xi_c}. $$

После присваивания текущий полином $\polynomial{\xi_c}$ согласно равенству
\eqref{equation:FPAP:annihilating_of_kth_degree_in_difference_polynomial_product} становится аннулирущим до степени $k$. Таким образом,

	$$ \polynomial{\xi^{(k)}} = \polynomial{\xi_c}. $$

Кроме того, поскольку после присваивания \eqref{equation:FPAP:current_polynomial_assignment} новый текущий полином $\polynomial{\xi_c}$ ---
это полином разности $\polynomial{\xi_d}$, который принадлежит множеству $\gfpunitypolynomial$ согласно
\eqref{equation:FPAP:difference_polynomial_belongs_to_unity_polynomials}, то

	$$ \polynomial{\xi^{(k)}} \in \gfpunitypolynomial, $$

и основная цель итерационного шага выполнена --- построен полином $\polynomial{\xi^{(k)}}$, аннулирующий до степени $k$, из множества
$\gfpunitypolynomial$.

Схема произведения по завершению варианта И.2 итерационного шага имеет вид:

	$$
		\begin{array}{rccccccccccc}
			                                 &         &         &         &              &         & d_p     & d_c     &         & m            & \\
			                                 & 0       & 1       & 2       & 3            & 4       & 5       & 6       & 7       & 8            & 9       & \dots \\
			\hline
			\polynomial{\xi_p}               & \bullet & \bullet & \bullet & \bullet      & \bullet & 0       & 0       & 0       & \delta_{c,k} & \bullet & \dots \\
			\polynomial{\xi_c}               & \bullet & \bullet & \bullet & \bullet      & \bullet & \bullet & 0       & 0       & 0            & \bullet & \dots \\
		\end{array}
	$$

и, как не трудно заметить, удовлетворяются все условия \ref{conditions:FPAP:iteration_entrance} с учетом того, что для следущего итерационного
шага необходимо увеличить $k$ на 1, а значит над полученными полиномами $\polynomial{\xi_p}$ и $\polynomial{\xi_c}$ можно опять выполнить
итерационный шаг алгоритма. Таким образом, на этом вариант И.2 итерационного шага завершается и необходимо перейти к шагу проверки
критерия завершения.

Критерий завершения заключается в том, чтобы проверить является ли очередной построенный полином $\polynomial{\xi^{(k)}}$ искомым, другими
словами необходимо определить сколько степеней в произведении $\polynomial{\alpha} \polynomial{\xi^{(k)}}$ зануляет полином
$\polynomial{\xi^{(k)}}$, начиная от степени самого полинома $\polynomialdegree{\xi^{(k)}}$, которая, кстати, совпадает со степенью текущего
полинома $d_c$. Если разность $k - d_c \ge n-1$, то полином $\polynomial{\xi^{(k)}}$ является искомым полиномом $\polynomial{\xi}$ из
множества $\gfpunitypolynomial$. Если же $k - d_c < n-1$, то необходимо продолжить выполнение итерационных шагов алгоритма.

По завершению работы алгоритма последний построенный полином $\polynomial{\xi^{(k)}}$ будет являтся искомым полиномом $\polynomial{\xi}$,
коэффициенты которого удовлетворяют системе равенств \eqref{equation:FPAP:equations_for_xi}:

	$$ \polynomial{\xi} = \polynomial{\xi^{(k)}}, $$

и для получения проекционно аннулирующего полинома $\polynomial{\rho}$ необходимо будет взять коэффициенты полинома $\polynomial{\xi}$
в обратном порядке, в соответствии с равенством \eqref{equation:FPAP:rho_and_xi_equation}.

\subsection{Минимальность проекционно аннулирущего полинома} \label{section:MDAP:minimal_degree_of_annihilating_polynomial}

В этом подразделе будет доказано, что каждый из аннулирующих до степени $k$ полиномов $\polynomial{\xi^{(k)}}$, которые строит алгоритм,
представленный в предыдущем разделе, имеет наименьшую возможную степень из всех аннулирующих до степени $k$ полиномов.

Прежде всего, следуя работе Месси \cite{Massey}, докажем вспомогательное утверждение.

\begin{statement} \label{statement:MDAP:annihilating_polynomial_degrees_inequality}

	Пусть полином $\polynomial{\gamma} \in \gfpunitypolynomial$ является аннулирущим до степени $k-1$, но не
	является аннулирующим до степени $k$, и полином $\polynomial{\hat\gamma} \in \gfpunitypolynomial$ является аннулирующим до степени $k$,
	тогда степени полиномов $\polynomial{\gamma}$ и $\polynomial{\hat\gamma}$ удовлетворяют неравенству:

		$$ \polynomialdegree{\hat\gamma} \ge k + 1 - \polynomialdegree{\gamma}. $$

	\proof

	Приводимое далее доказательство имеет следующую схему:

	\begin{tabular}{|p{1cm}|p{4cm}|p{4cm}|p{5cm}|}
	\hline
	№  & ограничение для $\polynomialdegree{\gamma}$ & ограничение для $\polynomialdegree{\hat\gamma}$ & Доказательство \\
	\hline
	\hline
	1  & $\polynomialdegree{\gamma} \ge k+1$         & нет ограничения                                 & этот случай оказывается невозможным, поскольку противоречит условиям утверждения \\
	\hline
	2  & $\polynomialdegree{\gamma} = k$             & нет ограничения                                 & этот случай оказывается также невозможным, поскольку противоречит условиям утверждения \\
	\hline
	3a & $\polynomialdegree{\gamma} \le k-1$         & $\polynomialdegree{\hat\gamma} \ge k+1$         & доказываемое неравенство является тривиальным \\
	\hline
	3b & $\polynomialdegree{\gamma} \le k-1$         & $\polynomialdegree{\hat\gamma} \le k$           & доказывается от противного \\
	\hline
	\end{tabular}

	Пусть $d$ обозначает степень полинома $\polynomial{\gamma}$:

		$$ d = \polynomialdegree{\gamma}. $$

	Для величины степени $d$ возможными являются три различных случая:

	\begin{enumerate}
		\item $d \ge k+1$;
		\item $d = k$;
		\item $d \le k-1$.
	\end{enumerate}

	Рассмотрим все случаи по порядку.

	\begin{enumerate}
		\item
			Рассмотрим первый случай, пусть $d \ge k + 1$, тогда по соглашению \ref{agreement:FPAP:ignoring_nonzero_product_coefficients} полином
			$\polynomial{\gamma}$ является аннулирующим до степени $k$, что противоречит условию утверждения --- полином $\polynomial{\gamma}$ не
			является аннулирующим до степени $k$. Таким образом, первый случай оказывается невозможным.

		\item
			Рассмотрим второй случай, пусть $d = k$. В этом случае неравенство утверждения

				$$ \polynomialdegree{\hat\gamma} \ge k + 1 - \polynomialdegree{\gamma}. $$

			вырождается в неравенство

				$$ \polynomialdegree{\hat\gamma} \ge k + 1 - d. $$
				$$ \polynomialdegree{\hat\gamma} \ge k + 1 - k. $$
				$$ \polynomialdegree{\hat\gamma} \ge 1. $$

			Таким образом, необходимо доказать, что степень полинома $\polynomial{\hat\gamma}$, аннулирующего до степени $k$, не может быть
			меньше 1. Проведем доказательство от противного и предположим, что полином $\polynomial{\hat\gamma}$ имеет степень меньше 1, то есть
			имеет степень 0, другими словами полином $\polynomial{\hat\gamma}$ является константой, и поскольку
			$\polynomial{\hat\gamma} \in \gfpunitypolynomial$, то эта константа равна 1:

				$$ \polynomial{\hat\gamma} = 1. $$

			Если полином $\polynomial{\hat\gamma}$ является аннулирующим до степени $k$, то значит в произведении
			$\polynomial{\alpha} \polynomial{\hat\gamma}$ равны нулю коэффициенты при степенях, начиная со степени самого полинома
			$\polynomial{\hat\gamma}$, то есть с 0, и до степени $k$ включительно:

				$$
					\left \{
					\begin{array}{c}
						\degreecoefficient{0}{\polynomial{\alpha} \polynomial{\hat\gamma}} = 0 \\
						\vdots \\
						\degreecoefficient{k}{\polynomial{\alpha} \polynomial{\hat\gamma}} = 0 \\
					\end{array}
					\right .
					,
				$$

				но полином $\polynomial{\hat\gamma} = 1$, тогда равны нулю коэффициенты в полиноме $\polynomial{\alpha}$:

				$$
					\left \{
					\begin{array}{c}
						\degreecoefficient{0}{\polynomial{\alpha}} = 0 \\
						\vdots \\
						\degreecoefficient{k}{\polynomial{\alpha}} = 0 \\
					\end{array}
					\right .
					,
				$$

				$$
					\left \{
					\begin{array}{c}
						\alpha_0 = 0 \\
						\vdots \\
						\alpha_k = 0 \\
					\end{array}
					\right .
					.
				$$

			В этом случае, очевидно, любой полином степени не больше $k$ окажется аннулирущим, поскольку в произведении до степени $k$
			участвуют только коэффициенты $\alpha_0$, \dots, $\alpha_k$ и все они равны 0.

			Отсюда следует, что и полином $\polynomial{\gamma}$, имеющий степень $d = k$, также является аннулирущим до степени $k$, что
			опять же противоречит условию утверждения --- полином $\polynomial{\gamma}$ не является аннулирующим до степени $k$. Таким образом,
			второй случай также оказывается невозможным.

		\item
			Остается только третий случай, в котором $d \le k-1$. Пусть полином $\polynomial{\gamma}$ имеет вид:

				$$ \gamma(\lambda) = \gamma_0 + \gamma_1 \lambda + \dots + \gamma_{d-1} \lambda^{d-1} + \gamma_d \lambda^d. $$

			Поскольку $\polynomial{\gamma} \in \gfpunitypolynomial$, то коэффициент при нулевой степени $\gamma_0$ равен 1:

				$$ \gamma(\lambda) = 1 + \gamma_1 \lambda + \dots + \gamma_{d-1} \lambda^{d-1} + \gamma_{d} \lambda^d. $$

			Согласно условию утверждения полином $\polynomial{\gamma}$ является аннулирующим до степени $k-1$, следовательно выполняются
			равенства (или по крайней мере одно равенство, если $d = k-1$):

				$$
					\left \{
					\begin{array}{c}
						\degreecoefficient{d}{\polynomial{\alpha} \polynomial{\gamma}} = 0 \\
						\vdots \\
						\degreecoefficient{k-1}{\polynomial{\alpha} \polynomial{\gamma}} = 0 \\
					\end{array}
					\right .
					,
				$$

				$$
					\left \{
					\begin{array}{c}
						\alpha_d + \gamma_1 \alpha_{d-1} + \dots + \gamma_{d-1} \alpha_1 + \gamma_d \alpha_0 = 0 \\
						\vdots \\
						\alpha_{k-1} + \gamma_1 \alpha_{k-1-1} + \dots + \gamma_{d-1} \alpha_{k-1-(d-1)} + \gamma_d \alpha_{k-1-d} = 0 \\
					\end{array}
					\right .
				$$

			Отсюда

				$$
					\left \{
					\begin{array}{c}
						\alpha_d = - \left ( \gamma_1 \alpha_{d-1} + \dots + \gamma_{d-1} \alpha_1 + \gamma_d \alpha_0 \right ) \\
						\vdots \\
						\alpha_{k-1} = - \left ( \gamma_1 \alpha_{k-1-1} + \dots + \gamma_{d-1} \alpha_{k-1-(d-1)} + \gamma_d \alpha_{k-1-d} \right ) \\
					\end{array}
					\right .
				$$

			или в более короткой форме:

				\begin{equation} \label{equation:MDAP:alpha_through_gamma}
					\alpha_j = - \sum_{i=1}^d \gamma_i \alpha_{k-i},
				\end{equation}
				$$ j = d, \dots, k-1. $$

			Теперь рассмотрим полином $\polynomial{\hat\gamma}$, обозначим $\hat d$ степень полинома $\polynomial{\hat\gamma}$:

				$$ \hat d = \polynomialdegree{\hat\gamma}. $$

			Для величины $\hat d$ возможны два случая:

			\begin{enumerate}
				\item $\hat d \ge k+1$;
				\item $\hat d \le k$.
			\end{enumerate}

			Рассмотрим оба случая.

			\begin{enumerate}
				\item
					Если $\hat d \ge k+1$, то

						$$ \polynomialdegree{\hat\gamma} = \hat d \ge k + 1 \ge k + 1 - \polynomialdegree{\gamma}, $$

					поскольку $\polynomialdegree{\gamma} \ge 0$. Таким образом, для случая $\hat d \ge k+1$ утверждение доказано.

				\item
					Пусть $\hat d \le k$, и полином $\polynomial{\hat\gamma} \in \gfpunitypolynomial$ имеет вид:

						$$ \hat\gamma(\lambda)
							= 1 + \hat\gamma_1 \lambda + \dots + \hat\gamma_{\hat d - 1} \lambda^{\hat d - 1} + \hat\gamma_{\hat d - 1} \lambda^{\hat d - 1}. $$

					Согласно условию утверждения полином $\polynomial{\hat\gamma}$ является аннулирующим до степени $k$, поэтому должны
					выполнятся равенства:

						$$
							\left \{
							\begin{array}{c}
								\degreecoefficient{\hat d}{\polynomial{\alpha} \polynomial{\hat\gamma}} = 0 \\
								\vdots \\
								\degreecoefficient{k}{\polynomial{\alpha} \polynomial{\hat\gamma}} = 0 \\
							\end{array}
							\right .
							,
						$$

						$$
							\left \{
							\begin{array}{c}
								\alpha_{\hat d} + \hat\gamma_1 \alpha_{\hat d - 1} + \dots + \hat\gamma_{\hat d - 1} \alpha_1 + \hat\gamma_{\hat d} \alpha_0 = 0 \\
								\vdots \\
								\alpha_k + \hat\gamma_1 \alpha_{k-1} + \dots + \hat\gamma_{\hat d - 1} \alpha_{k - \hat d - 1} + \hat\gamma_{\hat d} \alpha_{k - \hat d} = 0 \\
							\end{array}
							\right .
						$$

						$$
							\left \{
							\begin{array}{c}
								\alpha_{\hat d} = - \left ( \hat\gamma_1 \alpha_{\hat d - 1} + \dots + \hat\gamma_{\hat d - 1} \alpha_1 + \hat\gamma_{\hat d} \alpha_0 \right ) \\
								\vdots \\
								\alpha_k = - \left ( \hat\gamma_1 \alpha_{k-1} + \dots + \hat\gamma_{\hat d - 1} \alpha_{k - \hat d - 1} + \hat\gamma_{\hat d} \alpha_{k - \hat d} \right ) \\
							\end{array}
							\right .
						$$

						\begin{equation} \label{equation:MDAP:alpha_through_hat_gamma}
							\alpha_j = - \sum_{l=1}^{\hat d} \hat\gamma_l \alpha_{j-l},
						\end{equation}
						$$ j = \hat d, \dots, k. $$

					Докажем утверждение от противного: предположим, что

						$$ \polynomialdegree{\hat\gamma} \le k - \polynomialdegree{\gamma}, $$

					или в принятых обозначениях

						\begin{equation} \label{equation:MDAP:contrary_inequality_for_hat_gamma_degree}
							\hat d \le k - d.
						\end{equation}

					По условию утверждения полином $\polynomial{\gamma}$ не является аннулирущиюм до степени $k$, поэтому коэффициент при
					степени $k$ в произведении $\polynomial{\alpha} \polynomial{\gamma}$ не равен 0:

						$$ \degreecoefficient{k}{\polynomial{\alpha} \polynomial{\gamma}} \neq 0, $$
						$$ \alpha_{k} + \gamma_1 \alpha_{k-1} + \dots + \gamma_{d-1} \alpha_{k-(d-1)} + \gamma_d \alpha_{k-d} \neq 0 $$
						$$ \alpha_{k} \neq - \left ( \gamma_1 \alpha_{k-1} + \dots + \gamma_{d-1} \alpha_{k-(d-1)} + \gamma_d \alpha_{k-d} \right ) $$
						\begin{equation} \label{equation:MDAP:alpha_k_inequality_for_gamma}
							\alpha_k \neq \sum_{i=1}^{d} \gamma_i \alpha_{k-i},
						\end{equation}

					где в сумме правой части используются коэффициенты $\alpha_{k-1}$, \dots, $\alpha_{k-d}$. Поскольку по предположению
					\eqref{equation:MDAP:contrary_inequality_for_hat_gamma_degree} $\hat d \le k - d$, то каждый из коэффициентов $\alpha_{k-1}$,
					\dots, $\alpha_{k-d}$ имеет рекуррентное выражение с коэффициентами $\hat \gamma_l$ согласно
					\eqref{equation:MDAP:alpha_through_hat_gamma}. Подставляя рекуррентные соотношения из
					\eqref{equation:MDAP:alpha_through_hat_gamma} в сумму правой части неравенства
					\eqref{equation:MDAP:alpha_k_inequality_for_gamma} получим:

						$$ \sum_{i=1}^{d} \gamma_i \alpha_{k-i} = \sum_{i=1}^{d} \gamma_i \left ( - \sum_{l=1}^{\hat d} \hat\gamma_l \alpha_{k-i-l} \right ). $$

					Изменяя порядок суммирования в правой части, получим равенство:

						$$ \sum_{i=1}^{d} \gamma_i \alpha_{k-i} = \sum_{l=1}^{\hat d} \hat\gamma_l \left ( - \sum_{i=1}^{d} \gamma_i  \alpha_{k-l-i} \right ). $$

					Теперь из равенства \eqref{equation:MDAP:contrary_inequality_for_hat_gamma_degree} следует, что $d \le k - \hat d$, но
					тогда согласно равенствам \eqref{equation:MDAP:alpha_through_gamma} суммы в скобках в правой части дают коэффициенты
					$\alpha_{k-l}$:
		
						$$ \sum_{i=1}^{d} \gamma_i \alpha_{k-i} = \sum_{l=1}^{\hat d} \hat\gamma_l \alpha_{k-l}. $$

					Правая часть полученного равенства в соответствии с равенствами \eqref{equation:MDAP:alpha_through_hat_gamma} равна $\alpha_k$:

						$$ \sum_{i=1}^{d} \gamma_i \alpha_{k-i} = \sum_{l=1}^{\hat d} \hat\gamma_l \alpha_{k-l} = \alpha_k, $$

					что противоречит неравенству \eqref{equation:MDAP:alpha_k_inequality_for_gamma}:

						$$ \alpha_k \neq \sum_{i=1}^{d} \gamma_i \alpha_{k-i} = \alpha_k, $$
						$$ \alpha_k \neq \alpha_k, $$

					Таким образом, исходное предположение \eqref{equation:MDAP:contrary_inequality_for_hat_gamma_degree} является неверным,
					следовательно:

						$$ \hat d \ge k + 1 - d, $$
						$$ \polynomialdegree{\hat\gamma} \ge k + 1 - \polynomialdegree{\gamma}. $$
			\end{enumerate}
	\end{enumerate}

	Таким образом, были рассмотрены все возможные случаи, и в тех случаях, которые не вступают в противоречие с условиями утверждения, неравенство

		$$ \polynomialdegree{\hat\gamma} \ge k + 1 - \polynomialdegree{\gamma}. $$

	оказывается справедливым.
\end{statement}

С использованием утверждения \ref{statement:MDAP:annihilating_polynomial_degrees_inequality} можно доказать, что аннулирующие до степени
$k$ полиномы $\polynomial{\xi^{(k)}}$, которые строит алгоритм, представленный в разделе
\ref{section:FPAP:finding_projection_annihilating_polynomials}, имеют минимальные степени среди аннулирущих до степени $k$ полиномов из
множества $\gfpunitypolynomial$.

Начнем с шага Н.1 инициализации предыдущего полинома $\polynomial{\xi_p}$: поскольку на шаге Н.1 предыдущий полином $\polynomial{\xi_p} = 1$,
то его степень равна 0. Если предыдущий полином $\polynomial{\xi_p}$ оказывается аннулирущим до степени $m-1$ (при $m > 0$), то алгоритм
строит аннулирущие полиномы:

	$$ \polynomial{\xi^{(k)}} = \polynomial{\xi_p} $$
	$$ k=0,\dots,m-1, $$

имеющие степень 0. Поскольку полиномов степени меньше 0 в множестве $\gfpunitypolynomial$ нет, то все полиномы $\polynomial{\xi^{(k)}}$
при $k=0,\dots,m-1$ являются полиномами минимальной возможной степени.

Далее, предыдущий полином $\polynomial{\xi_p}$ не является аннулирущим до степени $m$, и на шаге Н.2 производится инициализация текущего
полинома $\polynomial{\xi_c}$: текущий полином $\polynomial{\xi_c}$ имеет степень $m+1$ и является аннулирущим до степени $m+1$, поэтому
алгоритм строит еще два полинома:

	$$ \polynomial{\xi^{(m)}} = \polynomial{\xi_c}, $$
	$$ \polynomial{\xi^{(m+1)}} = \polynomial{\xi_c}. $$

При описании алгоритма было показано, что полином из множества $\gfpunitypolynomial$, аннулирущий до степени $m$, обязательно имеет степень
больше $m$. Степень полинома $\polynomial{\xi^{(m)}}$, который построил алгоритм, совпадает со степенью текущего полинома $\polynomial{\xi_c}$,
равной $m+1$, поэтому построенный полином $\polynomial{\xi^{(m)}}$ является аннулирущим до степени $m$ полиномом минимальной степени
(аннулирущего до степени $m$ полинома степени меньше чем $m+1$ в множестве $\gfpunitypolynomial$ не существует).

Аннулирующий полином $\polynomial{\xi^{(m+1)}}$ тоже имеет минимальную возможную степень среди аннулищих до степени $m+1$ полиномом из множества
$\gfpunitypolynomial$. В этом легко убедиться, если предположить обратное: пусть существует такой полином
$\polynomial{\hat\xi} \in \gfpunitypolynomial$, который является аннулирующим до степени $m+1$ и имеет степень меньшую или равную $m$, тогда
полином $\polynomial{\hat\xi}$ является также аннулирующим до степени $m$, но в множестве $\gfpunitypolynomial$ не существует полиномов,
аннулирущих до степени $m$, имеющих степень меньшую или равную $m$, следовательно имеет место противоречие --- такой полином $\polynomial{\hat\xi}$
не существует. Таким образом, если в множестве $\gfpunitypolynomial$ и существуют полиномы, аннулирующие до степени $m+1$, то они обязательно
имеют степень больше или равную $m+1$, следовательно построенный полином $\polynomial{\xi^{(m+1)}}$, имеющий степень $m+1$, является одним
из таких полиномов минимальной степени.

Поскольку по окончании шага Н.2 аннулирущий полином $\polynomial{\xi^{(m+1)}}$ --- это текущий полином $\polynomial{\xi_c}$, то перед началом
итерационного шага текущий полином $\polynomial{\xi_c}$ является аннулирующим до степени $m+1$ полиномом минимальной степени из множества
$\gfpunitypolynomial$.

Кроме того, как нетрудно заметить, степень текущего полинома $d_c = m+1$, а степень предыдущего полинома $d_p = 0$, поэтому степени
$d_c$ и $d_p$ удовлетворяют равенству:

	$$ d_c = m + 1 - d_p. $$

Таким образом, после шагов инициализации Н.1 и Н.2 выполняются два дополнительных условия (с учетом того, что $k$ --- число следующее
за $m+1$, то есть $k=m+2$):

\begin{conditions} \label{conditions:MDAP:iteration_entrance}
	\begin{enumerate}
		\item текущий полином $\polynomial{\xi_c}$ является аннулирущим до степени $k-1$ полиномом, имеющим минимальную степень среди всех
			аннулирущих до степени $k-1$ полиномов в множестве $\gfpunitypolynomial$;

		\item степени предыдущего и текущего полиномов удовлетворяют равенству:

			$$ d_c = m + 1 - d_p ; $$
	\end{enumerate}
\end{conditions}

Если на итерационном шаге реализуется вариант И.1, то очередной полином $\polynomial{\xi^{(k)}}$ совпадает с текущим полиномом $\polynomial{\xi_c}$
и имеет наименьшую возможную степень среди всех аннулирующих до степени $k$ полиномов из множества $\gfpunitypolynomial$. Действительно, предположим
противное: пусть существует полином $\polynomial{\hat\xi}$, который является аннулирущим до степени $k$ полиномом из $\gfpunitypolynomial$ со
степенью меньше, чем степень полинома $\polynomial{\xi^{(k)}}$, равной степени текущего полинома $\polynomial{\xi_c}$ (поскольку
$\polynomial{\xi^{(k)}}$ совпадает с $\polynomial{\xi_c}$), тогда полином $\polynomial{\hat\xi}$ является аннулирущим и до степени $k-1$ полиномом
и имеет степень меньшей, чем степень текущего полинома $\polynomial{\xi_c}$, но это невозможно, поскольку текущий полином в соответствии с
условиями \ref{conditions:MDAP:iteration_entrance} имеет минимальную степень среди всех аннулирущих до степени $k-1$ полиномов в множестве
$\gfpunitypolynomial$. В силу полученного противоречия такого полинома $\polynomial{\hat\xi}$ существовать не может.

По завершению варианта И.1 итерационного шага текущий полином становится также и аннулирующим до степени $k$ полиномом, имеющим минимальную
степень среди всех аннулирущих до степени $k$ полиномов в множестве $\gfpunitypolynomial$, и следовательно выполняется первое из условий
\ref{conditions:MDAP:iteration_entrance} перед началом следующей итерации при увеличенном на 1 значении $k$. Кроме того, поскольку текущий
и предыдущий полиномы не изменяются в варианте И.1, то их степени также не изменяются, и следовательно выполнено и второе из условий
\ref{conditions:MDAP:iteration_entrance} перед началом следующей итерации.

В случае реализации варианта И.2 итерационного шага строится полином разности $\polynomial{\xi_d}$, который является аннулирущим до степени
$k$ полиномом со степенью, которая определяется текущим полиномом $\polynomial{\xi_c}$ и сдвинутым предыдущим полиномом
$\lambda^{k-m} \polynomial{\xi_p}$ (в зависимости от того, степень какого полинома окажется больше):

	$$ \polynomialdegree{\xi_d} = max \left \{ \polynomialdegree{\xi_c} , \polynomialdegree{\lambda^{k-m}\xi_p} \right \}, $$
	\begin{equation} \label{equation:MDAP:initial_difference_polynomial_degree_equality}
		\polynomialdegree{\xi_d} = max \left \{ d_c , k - m + d_p \right \}.
	\end{equation}

В силу второго из условий \ref{conditions:MDAP:iteration_entrance}:

	$$ d_p = m + 1 - d_c, $$

откуда

	$$ \polynomialdegree{\xi_d} = max \left \{ d_c , k - m + m + 1 - d_c \right \}, $$
	\begin{equation} \label{equation:MDAP:final_difference_polynomial_degree_equality}
		\polynomialdegree{\xi_d} = max \left \{ d_c , k + 1 - d_c \right \}.
	\end{equation}

Покажем, что степень полинома разности $\polynomialdegree{\xi_d}$, определяемая последним равенством, является минимальной.

Пусть полином $\polynomial{\hat\xi}$ является любым аннулирущим до степени $k$ полиномом из множества $\gfpunitypolynomial$. Во-первых,
степень полинома $\polynomial{\hat\xi}$ не может быть меньше степени текущего полинома $\polynomial{\xi_c}$, в противном случае полином
$\polynomial{\hat\xi}$ также является аннулирущим до степени $k-1$ полиномом со степенью меньше, чем степень текущего полинома
$\polynomial{\xi_c}$, а это противоречит первому из условий \ref{conditions:MDAP:iteration_entrance}, поэтому:

	\begin{equation} \label{equation:MDAP:any_kth_annihilating_polynomial_monotony_threshold}
		\polynomialdegree{\hat\xi} \ge \polynomialdegree{\xi_c}.
	\end{equation}

Во-вторых, в варианте И.2 итерационного шага текущий полином является аннулирущим до степени $k-1$, но не является аннулирущим до степени
$k$, а полином $\polynomial{\hat\xi}$ является аннулирущим до степени $k$, поэтому в силу утверждения
\ref{statement:MDAP:annihilating_polynomial_degrees_inequality}:

	\begin{equation} \label{equation:MDAP:any_kth_annihilating_polynomial_statement_threshold}
		\polynomialdegree{\hat\xi} \ge k + 1 - \polynomialdegree{\xi_c}.
	\end{equation}

Таким образом из неравенств \eqref{equation:MDAP:any_kth_annihilating_polynomial_monotony_threshold} и
\eqref{equation:MDAP:any_kth_annihilating_polynomial_statement_threshold}:

	$$ \polynomialdegree{\hat\xi} \ge max \left \{ \polynomialdegree{\xi_c} , k + 1 - \polynomialdegree{\xi_c} \right \}. $$

Откуда с учетом равенства \eqref{equation:MDAP:final_difference_polynomial_degree_equality}:

	$$ \polynomialdegree{\hat\xi} \ge max \left \{ d_c , k + 1 - d_c \right \} = \polynomialdegree{\xi_d}, $$

и следовательно любой аннулирущий до степени $k$ полином из множества $\gfpunitypolynomial$ имеет степень большую или равную степени полинома
разности $\polynomialdegree{\xi_d}$, поэтому полином $\polynomialdegree{\xi_d}$ имеет минимальную степень среди аннулирующих до степени $k$
полиномов из множества $\gfpunitypolynomial$.

Далее в варианте И.2 итерационного шага текущий полином $\polynomial{\xi_c}$ становится полиномом разности $\polynomial{\xi_d}$, поэтому
выполняется первое из условий \ref{conditions:MDAP:iteration_entrance} перед началом следующего итерационного шага с числом $k$, увеличенным
на 1. Затем в качестве аннулирущего полинома $\polynomial{\xi^{(k)}}$ также используется полином разности $\polynomial{\xi_d}$, поэтому
построенный алгоритмом полином $\polynomial{\xi^{(k)}}$ оказывается аннулирущим до степени $k$ полиномом с минимальной степенью в множестве
$\gfpunitypolynomial$.

Остается лишь проверить выполнение второго из условий \ref{conditions:MDAP:iteration_entrance} по завершению варианта И.2 итерационного шага.

Если до присваиваний \eqref{equation:FPAP:previous_polynomial_assignment} и \eqref{equation:FPAP:current_polynomial_assignment} выполняется
условие $d_p + k - m \le d_c$, тогда предыдущий полином $\polynomial{\xi_p}$ не изменяется и его степень $d_p$ остается прежней, текущий
полином $\polynomial{\xi_c}$ заменяется на полиномом разности $\polynomial{\xi_d}$, но в силу выполнения неравенства $d_p + k - m \le d_c$
полином $\polynomial{\xi_d}$ согласно равенству \eqref{equation:MDAP:initial_difference_polynomial_degree_equality} имеет степень $d_c$,
поэтому сам текущий полином $\polynomial{\xi_c}$ изменяется, но степень его остается прежней и, следовательно, неравенство $d_p + k - m \le d_c$
выполняется и после присваиваний \eqref{equation:FPAP:previous_polynomial_assignment} и \eqref{equation:FPAP:current_polynomial_assignment}.

Если же до присваиваний \eqref{equation:FPAP:previous_polynomial_assignment} и \eqref{equation:FPAP:current_polynomial_assignment} выполняется
условие $d_p + k - m > d_c$, тогда предыдущий полином $\polynomial{\xi_p}$ заменяется на текущий полином $\polynomial{\xi_c}$ и степень
$d_p$ заменяется степенью $d_c$, сам текущий полином $\polynomial{\xi_c}$ заменяется на полином разности $\polynomial{\xi_d}$,
который теперь согласно равенству \eqref{equation:MDAP:final_difference_polynomial_degree_equality} имеет степень $k + 1 - d_c$, а число
$m$ --- наименьшая степень, коэффициент при которой в произведении $\polynomial{\alpha} \polynomial{\xi_p}$ не является нулевым, --- становится
равной $k$ (это та степень, коэффициент при которой в произведении $\polynomial{\alpha} \polynomial{\xi_c}$ не являлся нулевым
до замены текущего полинома $\polynomial{\xi_c}$ на полином разности $\polynomial{\xi_d}$). Таким образом

	$$
		\begin{array}{ccc}
			d_p & \leftarrow & d_c \\
			d_c & \leftarrow & k + 1 - d_c \\
			m   & \leftarrow & k \\
		\end{array}
	$$

Откуда следует выполнение равенства:

	$$ d_c = m + 1 - d_p, $$

после присваиваний \eqref{equation:FPAP:previous_polynomial_assignment} и \eqref{equation:FPAP:current_polynomial_assignment}.

Таким образом, после выполнения любого из вариантов И.1 либо И.2 итерационного шага выполняются условия \ref{conditions:MDAP:iteration_entrance}
перед выполнением следующего итерационного шага при увеличенном на единицу значении $k$, поэтому следующий аннулирущий полином
$\polynomial{\xi^{(k+1)}}$ так же будет иметь минимальную степень среди аннулирущих уже до степени $k+1$ полиномов из множества $\gfpunitypolynomial$.

Отсюда по индукции следует, что и вообще любой построенный алгоритмом полином $\polynomial{\xi^{(k)}}$ при любом $k$ имеет минимальную степень
среди всех аннулирущих до степени $k$ полиномов из множества $\gfpunitypolynomial$.
