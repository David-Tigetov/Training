\section{Аннулирующие полиномы}

Определения, используемые в данном разделе, взяты из книги \cite[с.~171--172]{Gantmacher}.

\subsection{Аннулирующие полиномы пространства $\gfpvector{n}$}

\begin{definition} \label{definition:AP:space:annihilating_polynomial}
	Полином $\polynomial{\psi} \in \gfppolynomial$ называется \definedterm{аннулирующим полиномом пространства $\gfpvector{n}$
	относительно оператора $\mathcal A$}, если матрица $\psi(A)$ является нулевой :
		$$ \psi(A) = 0. $$
\end{definition}

\begin{definition}
	Полином $\polynomial{\tilde\psi} \in \gfppolynomial$ называется \definedterm{минимальным аннулирующим полином пространства
	$\gfpvector{n}$ отностительно оператора $\mathcal A$}, если его степень не больше степени любого другого аннулирующего полинома
	$\polynomial{\psi}$ пространства $\gfpvector{n}$ относительно оператора $\mathcal A$:
		$$ \polynomialdegree{\tilde\psi} \le \polynomialdegree{\psi}. $$
\end{definition}

Везде далее для краткости слова "относительно оператора $\mathcal A$"{} опущены.

\begin{statement} \label{statement:AP:space:polynomials_division}
	Пусть
	\begin{enumerate}
		\item $\polynomial{\psi}$ --- аннулирущий полином пространства $\gfpvector{n}$,
		\item $\polynomial{\tilde\psi}$ --- минимальный аннулирущий полином пространства $\gfpvector{n}$,
	\end{enumerate}
	тогда полином $\polynomial{\psi}$ делится без остатка на полином $\polynomial{\tilde \psi}$:

		$$ \polynomial{\psi} = \polynomial{\delta} \polynomial{\tilde\psi} $$

	где $\polynomial{\delta} \in \gfppolynomial$ --- некоторый полином.

	\proof

	Поскольку $\polynomial{\tilde\psi}$ по условию является минимальным аннулирующим полиномом, то его степень не больше степени полинома
	$\polynomial{\psi}$. В таком случае полином $\polynomial{\psi}$ можно разделить на полином $\polynomial{\tilde\psi}$, получив полиномы
	целой части $\polynomial{\delta}$ и остатка $\polynomial{\varepsilon}$:

		\begin{equation} \label{equation:AP:space:polynomials_division}
			\polynomial{\psi} = \polynomial{\delta} \polynomial{\tilde \psi} + \polynomial{\varepsilon}
		\end{equation}

	где степень полинома $\polynomial{\varepsilon}$ меньше степени полинома $\polynomial{\tilde\psi}$:

		\begin{equation} \label{equation:AP:space:remainder_polynomial_degree}
			\polynomialdegree{\varepsilon} < \polynomialdegree{\tilde\psi}.
		\end{equation}

	Подставляя в равенство \eqref{equation:AP:space:polynomials_division} матрицу $A$, получим равенство для матриц:

		$$ \psi(A) = \delta(A) \tilde\psi(A) + \varepsilon(A), $$

	в котором $\psi(A) = 0$ и $\tilde\psi(A) = 0$ поскольку полиномы $\polynomial{\psi}$ и $\polynomial{\tilde\psi}$ являются аннулирущими
	полиномами пространства $\gfpvector{n}$. Таким образом

		$$ 0 = \varepsilon(A) $$

	и следовательно полином $\polynomial{\varepsilon}$ тоже является аннулирующим полиномом пространства $\gfpvector{n}$, и к тому же имеет
	степень меньше степени минимального аннулирущего полинома $\polynomial{\tilde\psi}$ в соответствии с неравенством
	\eqref{equation:AP:space:remainder_polynomial_degree}. Это возможно только в том случае, если $\polynomial{\varepsilon} \equiv 0$, в
	противном случае, если $\polynomial{\varepsilon}$ --- ненулевой полином, то полином $\polynomial{\tilde\psi}$ не является минимальным
	аннулирующим полиномом, что противоречит условию утверждения.

	Таким образом, $\polynomial{\varepsilon} \equiv 0$ и равенство \eqref{equation:AP:space:polynomials_division} принимает вид:

		$$ \polynomial{\psi} = \polynomial{\delta} \polynomial{\tilde\psi} $$
\end{statement}

\subsection{Аннулирующие полиномы векторов из $\gfpvector{n}$}

\begin{definition}
	Полином $\polynomial{\varphi} \in \gfppolynomial$ называется \definedterm{аннулирующим полиномом вектора $x \in \gfpvector{n}$
	относительно оператора $\mathcal A$}, если вектор $\varphi(A) x$ является нулевым:
		$$ \varphi(A) x = 0. $$
\end{definition}

\begin{definition}
	Полином $\polynomial{\tilde\varphi} \in \gfppolynomial$ называется \definedterm{минимальным аннулирующим полином вектора
	$x \in \gfpvector{n}$ отностительно оператора $\mathcal A$}, если его степень не больше степени любого другого аннулирующего полинома
	$\polynomial{\varphi}$ вектора $x \in \gfpvector{n}$ относительно оператора $\mathcal A$:
		$$ \polynomialdegree{\tilde\varphi} \le \polynomialdegree{\varphi}. $$
\end{definition}

Опять же для краткости везде далее слова "относительно оператора $\mathcal A$"{} опущены.

Совершенно аналогично утверждению \ref{statement:AP:space:polynomials_division} можно показать, что всякий аннулирующий полином вектора
делится без остатка на минимальный аннулирующий полином вектора.

\begin{statement} \label{statement:AP:vector:polynomials_division}
	Пусть
	\begin{enumerate}
		\item $x \in \gfpvector{n}$ --- произвольный вектор,
		\item $\polynomial{\varphi}$ --- аннулирущий полином вектора $x$,
		\item $\polynomial{\tilde\varphi}$ --- минимальный аннулирущий полином вектора $x$,
	\end{enumerate}
	тогда полином $\polynomial{\varphi}$ делится без остатка на полином $\polynomial{\tilde\varphi}$:

		$$ \polynomial{\psi} = \polynomial{\delta} \polynomial{\tilde\psi} $$

	где $\polynomial{\delta} \in \gfppolynomial$ --- некоторый полином.

	\proof

	Степень полинома $\polynomial{\tilde\varphi}$ не больше степени полинома $\polynomial{\varphi}$, поскольку $\polynomial{\tilde\varphi}$
	по условию является минимальным аннулирующим полиномом. Разделив полином $\polynomial{\varphi}$ на полином $\polynomial{\tilde\varphi}$
	получим полиномы целой части $\polynomial{\delta}$ и остатка $\polynomial{\varepsilon}$:

		\begin{equation} \label{equation:AP:vector:polynomials_division}
			\polynomial{\varphi} = \polynomial{\delta} \polynomial{\tilde\varphi} + \polynomial{\varepsilon}
		\end{equation}

	где степень полинома $\polynomial{\varepsilon}$ меньше степени полинома $\polynomial{\tilde\varphi}$:

		\begin{equation} \label{equation:AP:vector:remainder_polynomial_degree}
			\polynomialdegree{\varepsilon} < \polynomialdegree{\tilde\varphi}.
		\end{equation}

	Подставляя в равенство \eqref{equation:AP:vector:polynomials_division} матрицу $A$, получим равенство для матриц:

		$$ \varphi(A) = \delta(A) \tilde\varphi(A) + \varepsilon(A), $$

	умножая которое на $x$ справа, получим аналогичное равенство для векторов:

		$$ \varphi(A) x = \delta(A) \tilde\varphi(A) x + \varepsilon(A) x, $$

	в котором $\varphi(A) x = 0$ и $\tilde\varphi(A) x = 0$ поскольку полиномы $\polynomial{\varphi}$ и $\polynomial{\tilde\varphi}$
	являются аннулирущими полиномами вектора $x$. Таким образом

		$$ 0 = \varepsilon(A) x $$

	и следовательно полином $\polynomial{\varepsilon}$ является аннулирующим полиномом вектора $x$, имеющим степень меньше степени минимального
	аннулирущего полинома $\polynomial{\tilde\varphi}$ в соответствии с неравенством \eqref{equation:AP:vector:remainder_polynomial_degree}.
	Это возможно только в том случае, если $\polynomial{\varepsilon} \equiv 0$, в противном случае, если $\polynomial{\varepsilon}$ ---
	ненулевой полином, то полином $\polynomial{\tilde\varphi}$ не является минимальным аннулирующим полиномом вектора $x$, что противоречит
	условию утверждения.

	Таким образом, $\polynomial{\varepsilon} \equiv 0$ и равенство \eqref{equation:AP:vector:polynomials_division} принимает вид:

		$$ \polynomial{\varphi} = \polynomial{\delta} \polynomial{\tilde\varphi} $$
\end{statement}

Согласно определению \ref{definition:AP:space:annihilating_polynomial} аннулирующий полином всего пространства $\gfpvector{n}$ является
аннилурищим полиномом любого вектора $x$ из пространства $\gfpvector{n}$. Отсюда в частности следует делимость аннулирущих полиномов
пространства на все минимальные аннулирующие полиномы всех векторов пространства $\gfpvector{n}$.

\begin{statement} \label{statement:AP:vector:space_polynomials_division}
	Пусть
	\begin{enumerate}
		\item $\polynomial{\psi}$ --- аннулирующий полином пространства $\gfpvector{n}$,
		\item $\polynomial{\tilde\varphi}$ --- минимальный аннулирующий полином некоторого вектора $x \in \gfpvector{n}$,
	\end{enumerate}
	тогда полином $\polynomial{\psi}$ делится без остатка на полином $\polynomial{\tilde\varphi}$:

		$$ \polynomial{\psi} = \polynomial{\delta} \polynomial{\tilde\varphi}, $$

	где $\polynomial{\delta} \in \gfppolynomial$ --- некоторый полином.

	\proof

	Поскольку $\polynomial{\psi}$ является аннулирующим полиномом пространства $\gfpvector{n}$, то:

		$$ \psi(A) = 0. $$

	и следовательно

		$$ \psi(A) x = 0. $$

	Таким образом, полином $\polynomial{\psi}$ является аннулирущим полиномом вектора $x$ и в соответствии с утверждением
	\ref{statement:AP:vector:polynomials_division} полином $\polynomial{\psi}$ делится без остатка на полином
	$\polynomial{\tilde\varphi}$
\end{statement}

\subsection{Делимость аннулирующих полиномов}

Согласно доказанным утверждениям имеет место следующая делимость полиномов: любой аннулирующий полином пространства $\gfpvector{n}$
делится без остатка на минимальные аннулирующие полиномы пространства $\gfpvector{n}$, каждый из которых в свою очередь делится без
остатка на любой минимальный аннулирующий полином любого вектора из $\gfpvector{n}$.

Из приведенной делимости следует, что аннулирующие полиномы всего пространства $\gfpvector{n}$ и отдельных векторов получаются из
минимальных аннулирующих полиномов пространства $\gfpvector{n}$ и векторов путем умножения на различные полиномы.

\subsection{Аннулирующие полиномы и вырожденность оператора}

Структура аннулирущих полиномов существенно зависит от оператора $\mathcal A$, например, если оператор является вырожденным, то
аннулирующие полиномы пространства начинаются с некоторой степени $\lambda^k$ и не содержат несколько младших степеней аргумента
(по крайней мере нулевую).

Наиболее сильным образом связан с оператором характеристический полином матрицы $A$ оператора: если оператор не вырожденный, то
характеристический полином обязательно имеет ненулевой коэффициент при нулевой степени аргумента. Это свойство характеристического полинома
переносится на минимальные аннулирущие полиномы за счет делимости характеристического полинома на минимальные аннулирущие
полиномы всего пространства и отдельных векторов.

\begin{statement} \label{statement:APD:space:zero_determinant_and_space_annihilating_polynomials}
	Пусть $\polynomial{\psi}$ --- аннулирущий полином пространства $\gfpvector{n}$, тогда

		$$
			\left \{ \left | A \right | = 0 \right \}
			\Rightarrow
			\left \{
				\begin{array}{c}
					\psi(\lambda) = \psi_k \lambda^k + \psi_{k+1} \lambda^{k+1} + \dots + \psi_d \lambda^{d},\\
					k \ge 1
				\end{array}
			\right \}
		$$

	\proof

	\begin{enumerate}

		\item Пусть полином $\polynomial{\psi}$ имеет вид:

			$$ \psi(\lambda) = \psi_0 \lambda + \psi_1 \lambda + \psi_2 \lambda^2 + \ldots + \psi_d \lambda^d. $$

			Поскольку $\polynomial{\psi}$ --- аннулирущий полином пространства, то

				$$ \psi(A) = 0, $$
				$$ \psi_0 E + \psi_1 A + \psi_2 A^2 + \ldots + \psi_d A^d = 0, $$
				$$ \psi_0 E = -\psi_1 A - \psi_2 A^2 - \ldots - \psi_d A^d, $$
				$$ \psi_0 E = \left ( -\psi_1 E - \psi_2 A^1 - \ldots - \psi_d A^{d-1} \right ) \cdot A, $$
				$$ \left | \psi_0 E \right | = \left | -\psi_1 E - \psi_2 A^1 - \ldots - \psi_d A^{d-1} \right | \cdot \left | A \right |, $$
				$$ \psi_0^n = \left | -\psi_1 E - \psi_2 A^1 - \ldots - \psi_d A^{d-1} \right | \cdot \left | A \right |. $$

			Поскольку по условию утверждения $\left | A \right | = 0$, то

				$$ \psi_0^n = \left | -\psi_1 E - \psi_2 A^1 - \ldots - \psi_d A^{d-1} \right | \cdot 0 = 0, $$
				$$ \psi_0^n = 0. $$

			Поскольку во всяком поле нет делителей нуля, то из последнего равенства следует

				$$ \psi_0 = 0. $$

			и полином $\polynomial{\psi}$ должен иметь вид:

				$$ \polynomial{\psi} = \psi_1\lambda + \psi_2\lambda^2 + \ldots + \psi_d\lambda^d, $$

		\item Из последнего равенства следует, что

				$$ \polynomial{\psi} = \lambda \left ( \psi_1 + \psi_2\lambda + \ldots + \psi_d\lambda^{d-1} \right ). $$

			и вполне возможно, что полином $\polynomial{\psi^{(1)}}$ стоящий в скобках:

				$$ \polynomial{\psi^{(1)}} = \psi_1 + \psi_2\lambda + \ldots + \psi_d\lambda^{d-1} $$

			так же является аннулирующим полиномом пространства $\gfpvector{n}$. В этом случае аналогично пункту 1 доказательства:

				$$ \psi_1 = 0. $$

			Продолжая рассуждения подобным образом, в конечном счете можно придти к тому, что полином $\polynomial{\psi}$ обязательно
			имеет вид:

				$$ \psi(\lambda) = \psi_k \lambda^k + \psi_{k+1} \lambda^{k+1} + \dots + \psi_d \lambda^{d}, $$

			где $k \ge 1$.

	\end{enumerate}
\end{statement}

\begin{definition}
	\definedterm{Характеристическим полиномом} матрицы $A \in \gfpmatrix{n}{n}$ называется полином $\polynomial{c_A}$:
	$$ \polynomial{c_A} = \left | A - \lambda E \right |. $$
\end{definition}

\begin{statement} \label{statement:APD:determinant_and_characteristic_polynomial}
		$$
			\left \{
				\left | A \right | \neq 0
			\right \}
			\Rightarrow
			\left \{
				\begin{array}{c}
					c_A(\lambda) = c_0 + c_1 \lambda + \dots + c_n \lambda^n,\\
					c_0 \neq 0
				\end{array}
			\right \}
		$$
		$$
			\left \{
				\left | A \right | = 0
			\right \}
			\Rightarrow
			\left \{
				\begin{array}{c}
					c_A(\lambda) = c_k \lambda^k + c_{k+1} \lambda^{k+1} + \dots + c_n \lambda^n,\\
					k \ge 1
				\end{array}
			\right \}
		$$

	\proof

	Пусть характеристический полином $\polynomial{c_A}$ имеет вид

		$$c_A(\lambda) = c_0 + c_1 \lambda + \dots + c_n \lambda^n. $$

	Легко видеть, что

		$$ c_0 = c_A ( 0 ) = \left | A - 0 \cdot E \right | = \left | A \right |.$$

	Отсюда сразу следует, что если $\left | A \right | \neq 0$, то $c_0 \neq 0$.

	Если $\left | A \right | = 0$, то $c_0 = 0$. Если $0$ является корнем кратности $k$, то в этом случае $k$ коэффициентов при младших
	степенях аргумента в характеристическом полиноме равны $0$,	следовательно характеристический полином $\polynomial{c_A}$ имеет вид

		$$ c_A(\lambda) = c_k \lambda^k + c_{k+1} \lambda^{k+1} + \dots + c_n \lambda^n, $$

	где $k \ge 1$.

\end{statement}

\begin{corollary} \label{corollary:APD:determinant_and_characteristic_polynomial}
	Тем самым доказаны два критерия:
		$$
			\left \{
				\left | A \right | \neq 0
			\right \}
			\Leftrightarrow
			\left \{
				\begin{array}{c}
					c_A(\lambda) = c_0 + c_1 \lambda + \dots + c_n \lambda^n,\\
					c_0 \neq 0
				\end{array}
			\right \}
		$$
		$$
			\left \{
				\left | A \right | = 0
			\right \}
			\Leftrightarrow
			\left \{
				\begin{array}{c}
					c_A(\lambda) = c_k \lambda^k + c_{k+1} \lambda^{k+1} + \dots + c_n \lambda^n,\\
					k \ge 1
				\end{array}
			\right \}
		$$

	\proof

	Докажем первый критерий от противного: пусть коэффициент $c_0 \neq 0$, а $\left | A \right | = 0$. В таком случае, в соответствии
	с утверждением \ref{statement:APD:determinant_and_characteristic_polynomial} характеристический полином $\polynomial{c_A}$ должен иметь
	вид

		$$ c_A(\lambda) = c_k \lambda^k + c_{k+1} \lambda^{k+1} + \dots + c_n \lambda^n, $$
		$$ k \ge 1, $$

	в котором $c_0 = 0$, что противоречит исходному предположению.

	Второй критерий доказывается аналогичным образом от противного.
\end{corollary}

\begin{statement} \label{statement:APD:determinant_and_minimal_space_annihilating_polynomial}
	Пусть $\polynomial{\tilde\psi}$ --- минимальный аннулирущий полином пространства $\gfpvector{n}$, тогда

		$$
			\left \{
				\left | A \right | \neq 0
			\right \}
			\Rightarrow
			\left \{
				\begin{array}{c}
					\tilde\psi(\lambda) = \tilde\psi_0 + \tilde\psi_1 \lambda + \dots + \tilde\psi_d \lambda^d,\\
					\tilde\psi_0 \neq 0
				\end{array}
			\right \}
		$$

	\proof

	По теореме Кели--Гамильтона \cite[с.~93]{Gantmacher} характеристический полином $\polynomial{c_A}$ матрицы $A$ является аннулирующим
	полиномом пространства $\gfpvector{n}$:

		$$ c_A ( A ) = 0. $$

	В таком случае согласно утверждению \ref{statement:AP:space:polynomials_division} характеристический полином $\polynomial{c_A}$
	делится без остатка на минимальный аннулирующий полином	$\polynomial{\tilde\psi}$, то есть для некоторого полинома
	$\polynomial{\delta} \in \gfppolynomial$:

		$$ \delta(\lambda) = \delta_0 + \delta_1 \lambda + \dots + \delta_{n-d} \lambda^{n-d} $$

	 имеет место разложение:

		\begin{equation} \label{equation:APD:characteristic_polynomial_factorization}
			\polynomial{c_A} = \polynomial{\delta} \polynomial{\tilde\psi}.
		\end{equation}

	В соответствии с утверждением \ref{statement:APD:determinant_and_characteristic_polynomial} в левой части коэффициент при нулевой
	степени $c_0 \neq 0$, а в правой части коэффициент при нулевой степени получается умножением $\delta_0 \cdot \tilde\psi_0$:

		$$ 0 \neq c_0 = \delta_0 \cdot \tilde\psi_0 $$

	Откуда в частности следует, что

		$$ \tilde\psi_0 \neq 0. $$
\end{statement}

\begin{statement} \label{statement:APD:determinant_and_minimal_vector_annihilating_polynomial}
	Пусть $\polynomial{\tilde\varphi}$ --- минимальный аннулирущий полином некоторого вектора $x \in \gfpvector{n}$, тогда

		$$
			\left \{
				\left | A \right | \neq 0
			\right \}
			\Rightarrow
			\left \{
				\begin{array}{c}
					\tilde\varphi(\lambda) = \tilde\varphi_0 + \tilde\varphi_1 \lambda + \dots + \tilde\varphi_d \lambda^d,\\
					\tilde\varphi_0 \neq 0
				\end{array}
			\right \}
		$$

	\proof

	Доказательство этого утверждения аналогично доказательству утверждения
	\ref{statement:APD:determinant_and_minimal_space_annihilating_polynomial}.

	По теореме Кели--Гамильтона \cite[с.~93]{Gantmacher} характеристический полином $\polynomial{c_A}$ матрицы $A$ является аннулирующим
	полиномом пространства $\gfpvector{n}$, и в соответствии с утверждением \ref{statement:AP:vector:space_polynomials_division} делится без
	остатка на минимальный аннулирующий полином $\polynomial{\tilde\varphi}$ вектора $x$, то есть для некоторого полинома
	$\polynomial{\delta} \in \gfppolynomial$:

		$$ \delta(\lambda) = \delta_0 + \delta_1 \lambda + \dots + \delta_{n-d} \lambda^{n-d} $$

	имеет место разложение:

		$$ \polynomial{c_A} = \polynomial{\delta} \polynomial{\tilde\varphi}, $$

	в котором коэффициент при нулевой степени слева $c_0 \neq 0$ в соответствии с утверждением
	\ref{statement:APD:determinant_and_characteristic_polynomial}, а справа равен $\delta_0 \cdot \tilde\varphi_0$:

		$$ 0 \neq c_0 = \delta_0 \cdot \tilde\varphi_0, $$

	следовательно

		$$ \tilde\varphi_0 \neq 0. $$
\end{statement}

\subsection{Аннулирущие полиномы и подпространства Крылова} \label{section:KS:krylov_spaces}

\begin{definition}
	\definedterm{Подпространством Крылова $\mathcal K_r$ размера $r$ для вектора $x \in \gfpvector{n}$ и матрицы $A \in \gfpmatrix{n}{n}$}
	называется линейная оболочка векторов $x$, $Ax$, $A^2x$, \dots, $A^{r-1}x$:
		$$ \mathcal K_r (x,A) = \mathcal L \left \{ x, Ax, A^2x, \dots, A^{r-1}x \right \} $$
\end{definition}

\begin{definition}
	Векторы $A^k x$ называются \definedterm{векторами Крылова}.
\end{definition}

Легко видеть, что при любом $r$ подпространство $\mathcal K_r(x,A) \subseteq \gfpvector{n}$, и поскольку размерность $\gfpvector{n}$ равна $n$,
то размерности подпространств $\mathcal K_r(x,A)$ не могут быть больше $n$. Отсюда следует, что среди векторов Крылова
$x$, $Ax$, $A^2x$, \dots, $A^{r-1}x$ не может быть больше $n$ линейно независимых векторов, и следовательно при некоторых $r$ векторы Крылова
оказываются линейно зависимыми (во всяком случае при $r > n$).

Пусть вектор $x$ выбран произвольным образом и зафиксирован, и пусть при некотором $r$ векторы Крылова $x$, $Ax$, $A^2x$, \dots, $A^rx$
в количестве $r+1$ являются линейно зависимыми, тогда их линейная комбинация с некоторыми коэффициентами $\varphi_i$ равна нулевому вектору:

	$$ \varphi_0 x + \varphi_1 Ax + \varphi_2 A^2x + \dots + \varphi_{r-1} A^{r-1}x + \varphi_r A^rx = 0. $$

Отсюда, вынося вектор $x$, получим равенство

	\begin{equation} \label{equation:KS:annihilating_x}
		\left ( \varphi_0  + \varphi_1 A + \varphi_2 A^2 + \dots + \varphi_{r-1} A^{r-1} + \varphi_r A^r \right ) x = 0,
	\end{equation}

из которого следует, что полином $\polynomial{\varphi}$:

	$$ \varphi ( \lambda ) = \varphi_0  + \varphi_1 \lambda + \varphi_2 \lambda^2 + \dots + \varphi_{r-1} \lambda^{r-1} + \varphi_r \lambda^r $$

является аннулирующим полиномом вектора $x$, поскольку из равенства \eqref{equation:KS:annihilating_x}:

	$$ \varphi ( A ) x = 0. $$

Таким образом, известная линейная зависимость векторов Крылова $x$, $Ax$, $A^2x$, \dots, $A^rx$ приводит к аннулирующему полиному вектора $x$.

Пусть теперь число $\tilde r$ является наибольшим количеством линейно независимых векторов Крылова $x$, $Ax$, $A^2x$, \dots, $A^{\tilde r-1}x$,
тогда векторы Крылова $x$, $Ax$, $A^2x$, \dots, $A^{\tilde r-1}x$, $A^{\tilde r}x$ в количестве $\tilde r + 1$ уже являются линейно
зависимыми (поскольку $\tilde r$ --- это наибольшее количество линейно независимых векторов Крылова) и следовательно, как и ранее, существует
такая линейная комбинация этих векторов с коэффициентами $\tilde \varphi_i$, которая является нулевым вектором, откуда полином
$\polynomial{\tilde \varphi}$ с коэффициентами $\tilde \varphi_i$:

$$ \tilde \varphi ( \lambda ) =
	\tilde \varphi_0  + \tilde \varphi_1 \lambda + \tilde \varphi_2 \lambda^2 + \dots
		+ \tilde \varphi_{\tilde r - 1} \lambda^{\tilde r - 1}
		+ \tilde \varphi_{\tilde r} \lambda^{\tilde r} $$

является аннулирующим полиномом вектора $x$. Более того, полином $\polynomial{\tilde \varphi}$ является минимальным аннулирующим полиномом
вектора $x$: если допустить существование аннулирующего полинома некоторой меньшей степени $k < \tilde r$, то тогда окажется, что линейно
зависимыми являются векторы Крылова $x, Ax, A^2x, \dots, A^{k-1}x, A^kx$ в количестве $k+1$, что невозможно, поскольку в силу определения
числа $\tilde r$ векторы Крылова $x, Ax, A^2x, \dots, A^{k-1}x, A^kx, \dots, A^{\tilde r-1}x$ в количестве $\tilde r \ge k+1$ являются
линейно независимыми.

Таким образом, построение векторов Крылова $x, Ax, A^2x, \dots, A^{r-1}x$ и анализ их линейной зависимости приводит к построению
аннулирующих полиномов вектора $x$, в том числе и минимальных.

Легко видеть, что для каждого вектора $x$ соответствующее наибольшее количество линейно независимых векторов Крылова $r$ определяется
однозначным образом и вообще говоря может быть разным для разных векторов. Например, нетрудно себе представить вектор $x \neq 0$
из ядра оператора $\mathcal A$, для которого $Ax = 0$ и вообще для любого $k \ge 2$ $A^kx = A^{k-1} \cdot Ax = A^{k-1} \cdot 0 = 0$,
так что число $r$ линейно независимых векторов Крылова для такого вектора $x$ равно 1 (это сам вектор $x$).
