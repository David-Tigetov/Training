\documentclass[a4paper,12pt]{book}
\usepackage[utf8]{inputenc}
\usepackage[english,russian]{babel}
\usepackage{amsmath}
\usepackage{amsfonts}
\usepackage{latexsym}

\newcommand{\gftwo}{\mathbb{F}_2}
\newcommand{\gftwovector}[1]{\mathbb{F}^{#1}_2}
\newcommand{\gftwomatrix}[2]{\mathbb{F}^{#1 \times #2}_2}

\newcommand{\gfpmatrix}[2]{\mathbb{F}^{#1 \times #2}_p}
\newcommand{\gfpvector}[1]{\mathbb{F}^{#1}_p}
\newcommand{\gfp}{\mathbb{F}_p}
\newcommand{\gfppolynomial}{\gfp \left [ \lambda \right ]}
\newcommand{\gfpunitypolynomial}{\gfp^1 \left [ \lambda \right ]}

% определитель
\newcommand{\determinant}[1]{\left | #1 \right |}

% обозначение полинома
\newcommand{\polynomial}[1]{ #1 (\lambda)}
% обозначение степени полинома
\newcommand{\polynomialdegree}[1]{ deg \left [ \polynomial{#1} \right ] }
% коэффициент при степени полинома
\newcommand{\degreecoefficient}[2]{ \left [ \lambda^{#1} \right ] \left \{ #2 \right \} }

% cчетчик определений, утверждений и прочих
\newcounter{itemcounter}[chapter]
% формат выводы счетчика определений, утверждений и прочих
\renewcommand{\theitemcounter}{\thechapter.\arabic{itemcounter}}

% окружение определений
\newenvironment{definition}
{
	\vspace{0.3\baselineskip}
	\refstepcounter{itemcounter}
	\textbf{Определение \theitemcounter}.
	\par
}
{
	\vspace{0.3\baselineskip}
}
% стиль определяемого термина
\newcommand{\definedterm}[1]{ \textit{#1} }

% окружение утверждений
\newenvironment{statement}
{
	\vspace{0.3\baselineskip}
	\refstepcounter{itemcounter}
	\textbf{Утверждение \theitemcounter}.
	\par
}
{
	\par
	{\raggedleft $\Box$ \par}
	\vspace{0.3\baselineskip}
}
% доказательство
\newcommand{\proof}{\par Доказательство: \par}

% окружение следствий
\newenvironment{corollary}
{
	\vspace{0.3\baselineskip}
	\refstepcounter{itemcounter}
	\textbf{Следствие \theitemcounter}.
	\par
}
{
	\par
	{\raggedleft $\Box$ \par}
	\vspace{0.3\baselineskip}
}

% окружение соглашений
\newenvironment{agreement}
{
	\vspace{0.3\baselineskip}
	\refstepcounter{itemcounter}
	\textbf{Соглашение \theitemcounter}.
	\par
}
{
	\par
	{\raggedleft $\Box$ \par}
	\vspace{0.3\baselineskip}
}

% окружение условий
\newenvironment{conditions}
{
	\vspace{0.3\baselineskip}
	\refstepcounter{itemcounter}
	\textbf{Условия \theitemcounter}.
	\par
}
{
	\par
	{\raggedleft $\Box$ \par}
	\vspace{0.3\baselineskip}
}

% окружение условия
\newenvironment{condition}
{
	\vspace{0.3\baselineskip}
	\refstepcounter{itemcounter}
	\textbf{Условие \theitemcounter}.
	\par
}
{
	\par
	{\raggedleft $\Box$ \par}
	\vspace{0.3\baselineskip}
}

\begin{document}

\title{Решение систем линейный уравнений в конечных полях}
\author{Тигетов Давид}
\maketitle

\tableofcontents

\section{Обозначения}

$\gftwo$ --- $GF(2)$.

$\gftwovector{n}$ --- линейное пространство векторов порядка $n$ над полем $\gftwo$

$\gftwomatrix{n}{m}$ --- множество матриц с $n$ строками и $m$ столбцами и элементами из $GF(2)$.

$\gfp$ --- $GF(p)$.

$\gfpvector{n}$ --- линейное пространство векторов порядка $n$ над полем $\gfp$.

$\gfpmatrix{n}{m}$ --- множество матриц с $n$ строками и $m$ столбцами и элементами из $GF(p)$.

$\gfppolynomial$ --- поле многочленов над полем $\gfp$.

$\polynomialdegree{\varphi}$ --- степень полинома $\polynomial{\varphi}$.

$\degreecoefficient{k}{\polynomial{\varphi}}$ --- коэффициент при степени $k$ полинома $\polynomial{\varphi}$.

$E$ --- единичная матрица (порядок которой определяется из контекста).

\part{Решение систем в $\gftwovector{n}$}

\chapter{Метод Ланцоша-Монтгомери} \label{chapter:Lancsoz_Montgomery}

\section{Построение решения} \label{section:LM:finding_solution}

Данный раздел является расширением соответствующего раздела из книги \cite[с.~44--50]{Zamarashkin}.

Пусть задана матрица $A \in \gftwomatrix{n}{n}$ и вектор $b \in \gftwovector{n}$ и требуется найти решение системы:

$$
	Ax = b,
$$

где матрица $A$ является симметричной:

$$
	A = A^T,
$$

и вектор $b$ принадлежит линейной оболочке столбцов матрицы $A$.

Согласно методу Ланцоша-Монтгомери \cite{Montgomery} необходимо построить последовательность подпространств
$ \left \{ \mathcal W_i \right \} _{i=1}^m$ специального вида, прямая сумма которых образует подпространство $\mathcal W$:

$$
	\mathcal W = \mathcal W_1 + \mathcal W_2 + \dots \mathcal W_m.
$$

Пусть столбцы матриц $W_i$ являются векторами базиса соответствующих подпространств $\mathcal W_i$. Подпространства $\mathcal W$
и $\mathcal W_i$ должны удовлетворять следующим свойствам:

\begin{itemize}

	\item [M-1] апроксимация: вектор $b$ принадлежит линейной оболочке векторов из пространства $A \mathcal W$,
		$$
			b \in A \mathcal W;
		$$

	\item [M-2] А-обратимость подпространств $\mathcal W_i$: матрицы $W_i^T A W_i$ являются невырожденными,
		$$
			\determinant{W_i^T A W_i} \neq 0;
		$$

	\item [M-3] А-ортогональность подпространств $\mathcal W_i$: матрицы $W_i^T A W_j$ являются нулевыми,
		$$
			\begin{array}{c}
				W_i^T A W_j = 0, \\
				i \neq j; \\
			\end{array}
		$$

	\item [M-4] А-инвариантность: пространство $\mathcal W$ является инвариантным пространством матрицы $A$,
		$$
			A \mathcal W \subseteq \mathcal W.
		$$
\end{itemize}

Предположим, что пространства $\mathcal W_i$ и $\mathcal W$, удовлетворяющие свойствам M-1 -- M-4, построены. Постараемся в
пространстве $\mathcal W$ найти такой вектор $x$, чтобы невязка $Ax - b$ была орогональна пространству $W$:

\begin{equation} \label{equation:LM:FS:solution_conditions}
	\begin{array}{c}
	Ax - b \perp \mathcal W, \\
	x \in \mathcal W.
	\end{array}
\end{equation}

Поскольку $x \in \mathcal W$ и $\mathcal W$ является прямой суммой подпространств $\mathcal W_i$, то вектор $x$ раскладывает
по базисным векторам пространств $\mathcal W_i$:

\begin{equation} \label{equation:LM:FS:solution_representation}
	x = W_1 \alpha_1 + W_2 \alpha_2 + \dots + W_m \alpha_m = \sum_{i=1}^m W_i \alpha_i,
\end{equation}

где $\alpha_i$ --- векторы-столбцы коэффициентов.

Согласно условию \ref{equation:LM:FS:solution_conditions} невязка $Ax - b$ должна быть ортогональна пространству $\mathcal W$, следовательно
невязка должна быть ортогональна каждому из подпространств $\mathcal W_i$:

$$
	Ax - b \perp \mathcal W_i, \\
$$

Отсюда следует, что все скалярные произведения векторов базисов пространств $\mathcal W_i$ и вектора невязки $Ax-b$ должны быть равны нулю:

$$
	W_i^T ( Ax - b ) = 0, \\
$$

Откуда

$$
	W_i^T Ax = W_i^T b, \\
$$

Согласно представлению \ref{equation:LM:FS:solution_representation} из последнего равенства получим:

$$
	W_i^T A \sum_{i=1}^m W_i \alpha_i = W_i^T b, \\
$$

$$
	\sum_{j=1}^m W_i^T A W_j \alpha_j = W_i^T b,
$$

Согласно условию M-3 $W_i^T A W_j = 0$ при $i \neq j$, поэтому в сумме в правой части остается только одно слагаемое:

$$
	W_i^T A W_i \alpha_i = W_i^T b,
$$

В силу условия M-2 определитель $\determinant{W_i^T A W_i} \neq 0$, поэтому существует обратная матрица
$ \left ( W_i^T A W_i \right ) ^{-1} $:

$$
	\alpha_i = \left ( W_i^T A W_i \right ) ^{-1} W_i^T b,
$$

Подставляя полученные выражения для $\alpha_i$ в представление \ref{equation:LM:FS:solution_representation} получим:

\begin{equation} \label{equation:LM:FS:solution}
	x = \sum_{i=1}^m W_i \left ( W_i^T A W_i \right ) ^{-1} W_i^T b.
\end{equation}

Теперь покажем, что полученный вектор $x$ из соотношения \ref{equation:LM:FS:solution} является решением исходной системы $Ax = b$,
то есть невязка $Ax - b$ не только ортогональна пространству $\mathcal W$, но и равна нулю:

$$
	Ax - b \perp \mathcal W \Rightarrow Ax - b = 0.
$$

Прежде всего заметим, что $x \in \mathcal W$ по построению согласно условиям \ref{equation:LM:FS:solution_conditions} и в силу условия
M-4 $A \mathcal W \subseteq \mathcal W$, поэтому вектор $Ax$ является вектором из пространства $\mathcal W$:

$$
	Ax \in \mathcal W.
$$

Согласно условиям M-1 и M-4 $b \in A \mathcal W \subseteq \mathcal W$, поэтому вектор $b$ тоже принадлежит пространству $\mathcal W$:

$$
	b \in \mathcal W.
$$

Таким образом, невязка $Ax - b$ принадлежит пространству $\mathcal W$:

$$
	Ax - b \in \mathcal W.
$$

Отсюда следует, что существуют векторы коэффициентов $\beta_i$, такие что:

\begin{equation} \label{equation:LM:FS:residual_representation}
	Ax - b = W_1 \beta_1 + W_2 \beta_2 + \dots + W_m \beta_m = \sum_{j=1}^m W_j \beta_j.
\end{equation}

Домножим левую и правую части слева на $W_i^T A$:

$$
	W_i^T A ( Ax - b ) = W_i^T A \sum_{j=1}^m W_j \beta_j.
$$
$$
	W_i^T A ( Ax - b ) = \sum_{j=1}^m W_i^T A W_j \beta_j.
$$

Согласно условию M-2 $W_i^T A W_j = 0$ при $i \neq j$, поэтому в сумме в правой части остается только одно слагаемое:

$$
	W_i^T A ( Ax - b ) = W_i^T A W_i \beta_i.
$$

В левой части в силу симметричности матрицы $A$:

$$
	W_i^T A ( Ax - b ) = W_i^T A^T ( Ax - b ) = ( A W_i )^T ( Ax - b ).
$$

Последнее выражение является результатом скалярного умножения векторов-столбцов матрицы $A W_i$ и невязки $Ax - b$. В силу условия
M-4 $A \mathcal W \subseteq \mathcal W$, поэтому векторы-столбцы матрицы $A W_i$ являются векторами пространства $\mathcal W$,
а невязка $Ax - b$ по построению в силу условий \ref{equation:LM:FS:solution_conditions} ортогональна всем векторам пространства
$\mathcal W$ и следовательно все скалярные произведения равны нулю:

$$
	( A W_i )^T ( Ax - b ) = 0.
$$

Таким образом,

$$
	0 = ( A W_i )^T ( Ax - b ) = W_i^T A ( Ax - b ) = W_i^T A W_i \beta_i,
$$

и коэффициенты вектора $\beta_i$ удовлетворяют однородной системе с матрицей $W_i^T A W_i$:

$$
	W_i^T A W_i \beta_i = 0.
$$

Поскольку по условию M-3 определитель $\determinant{W_i^T A W_i} \ne 0$, то однородная система имеет только тривиальное решение и
следовательно вектор $\beta_i$ является нулевым:

$$
	\beta_i = 0.
$$

Таким образом, в разложении \ref{equation:LM:FS:residual_representation} все векторы $\beta_i$ являются нулевыми, поэтому невязка
$Ax - b$ является нулевым вектором:

$$
	Ax - b = W_1 \beta_1 + W_2 \beta_2 + \dots + W_m \beta_m = 0,
$$

и следовательно построенный вектор $x$ из соотношения \ref{equation:LM:FS:solution} является решением исходной системы:

$$
	\begin{array}{c}
		Ax - b = 0, \\
		Ax = b.
	\end{array}
$$

Остается только вопрос каким образом можно построить подпространства $\mathcal W_i$, удовлетворяющие условиям M-1 -- M-4, который 
рассматривается в следующем разделе.

\section{Построение подпространств} \label{section:LM:finding_subspaces}

Данный раздел является расширением соответствующего раздела из книги \cite[с.~44--50]{Zamarashkin}.

Один из возможных способов построения пространств $\mathcal W_i$ приводит в своей статье Монтгомери \cite{Montgomery}. Первым шагом
этого способа является формирование матрицы $V_1$ порядка $n \times n_b$, где $n_b$ - произвольное натурально число: в качестве одного
из столбцов матрицы $V_1$ необходимо взять вектор правой части $b$, а в качестве остальных столбцов выбрать произвольные ненулевые векторы.

Полученную матрицу $V_1$ нельзя использовать в качестве матрицы $W_1$, поскольку матрица $V_1^T A V_1$ может оказаться вырожденной и тем самым
будет нарушено условие M-2 для $W_1$. Тем не менее, в матрице $V_1$ можно выбрать наибольшее количество столбцов, из которых составить
матрицу $W_1$ так, чтобы матрица $W_1^T A W_1$ оказалась невырожденной. Из оставшихся столбцов, которые не вошли в состав $W_1$, необходимо
сформировать матрицу $\widehat{W}_1$:

$$
	V_1 =
		\begin{pmatrix}
			W_1 & \widehat{W}_1
		\end{pmatrix}
$$

Линейная оболочка векторов-столцов матрицы $W_1$ образует подпространство $\mathcal W_1$.

Для построения матрицы $W_2$, столбцы которой образуют векторы базиса пространства $\mathcal W_2$, необходимо сформировать матрицу
$\widehat{V}_2$ путем присоединения к матрице $\widehat{W}_1$ столбцов матрицы $A W_1$:

$$
	\widehat{V}_2 =
		\begin{pmatrix}
			\widehat{W}_1 & A W_1
		\end{pmatrix}
	.
$$

Векторы-столбцы матрицы $\widehat{V}_2$ необходимо сделать $A$-ортогональными подпространству $\mathcal W_1$, для этого нужно
из векторов-столбцов матрицы $\widehat{V}_2$ "вычесть их проекции"{} на пространство $\mathcal W_1$, то есть необходимо из столбцов матрицы
$\widehat{V}_2$ "вычесть"{} некоторую линейную комбинацию векторов-столбцов матрицы $W_1$, образующих базис подпространства
$\mathcal W_1$. Пусть матрица $V_2$ обозначает набор вектор-столбцов, полученных в результате $A$-ортогонализации векторов-столбцов
матрицы $\widehat{V}_2$:

\begin{equation} \label{equation:LM:FS:V_2_orthogonalization}
	V_2 = \widehat{V}_2 + W_1 \gamma_{2,1},
\end{equation}

где $\gamma_{2,1}$ --- искомая матрица коэффициентов, при которой:

$$
	W_1^T A V_2 = 0.
$$

Из последнего условия получим уравнение для матрицы $\gamma_{2,1}$:

$$
	\begin{array}{c}
		W_1^T A ( \widehat{V}_2 + W_1 \gamma_{2,1} ) = 0, \\
		W_1^T A \widehat{V}_2 + W_1^T A W_1 \gamma_{2,1} = 0, \\
		W_1^T A \widehat{V}_2 + W_1^T A \widehat{V}_2 + W_1^T A W_1 \gamma_{2,1} = W_1^T A \widehat{V}_2.
	\end{array}
$$

Поскольку в группе матриц с операцией сложения, порожденной сложением элементов в поле $\gftwo$, сама матрица является своей обратной
по сложению матрицей, то из последнего равенства следует:

$$
	\begin{array}{c}
	 	W_1^T A W_1 \gamma_{2,1} = W_1^T A \widehat{V}_2.
	\end{array}
$$

По построению матрицы $W_1$ матрица $W_1^T A W_1$ является невырожденной, поэтому существует обратная матрица
$\left ( W_1^T A W_1 \right ) ^{-1}$, умножая на которую слева левую и правую части последнего равенства получим:

$$
	\gamma_{2,1} = \left ( W_1^T A W_1 \right ) ^ {-1} W_1^T A \widehat{V}_2.
$$

Подставляя полученное выражение для $\gamma_{2,1}$ в соотношение \ref{equation:LM:FS:V_2_orthogonalization}, получим:

$$
	V_2 = \widehat{V}_2 + W_1 \left ( W_1^T A W_1 \right ) ^ {-1} W_1^T A \widehat{V}_2.
$$

Далее необходимо в матрице $V_2$ выбрать наибольшее количество столбцов, из которых сформировать матрицу $W_2$ таким образом, чтобы
матрица $W_2^T A W_2$ оказалась невырожденной, при этом необходимо чтобы все столбцы матрицы $\widehat{W}_1$ (после $A$-ортогонализации)
вошли в состав матрицы $W_2$. Оставшиеся столбцы, не вошедшие в состав матрицы $W_2$, необходимо поместить в матрицу $\widehat{W_2}$:

$$
	V_2 =
		\begin{pmatrix}
			W_2 & \widehat{W_2}
		\end{pmatrix}
	.
$$

Векторы-столбцы матрицы $W_2$ являются $A$-ортогональными векторам-столбцам $W_1$, поскольку все векторы-столбцы $V_2$
$A$-ортогональны векторам-столбцам $W_1$, и кроме того матрица $ W_2^T A W_2 $ является невырожденной, поэтому линейную оболочку
векторов-столбцов можно считать подпространством $\mathcal W_2$ и при этом выполняются условия M-2 и M-3.

Далее для построения матрицы $W_3$ аналогичным образом формируется матрица $\widehat{V}_3$:

$$
	\widehat{V}_3 =
		\begin{pmatrix}
		\widehat{W}_2 & A W_2 \\
		\end{pmatrix}
	.
$$

Вектор-столбцы матрицы $\widehat{V}_3$ необходимо сделать $A$-ортогональными подпространствам $\mathcal W_1$ и $\mathcal W_2$,
поэтому из матрицы $\widehat{V}_3$ необходимо "вычесть проекции"{} векторов-столбцов и на подпространство $\mathcal W_1$, и на
подпространство $\mathcal W_2$:

\begin{equation} \label{equation:LM:FS:V_3_orthogonalization}
	V_3 = \widehat{V}_3 + W_1 \gamma_{3,1} + W_2 \gamma_{3,2}.
\end{equation}

Из требования $A$-ортогональности подпространству $W_1$ следует:

$$
	\begin{array}{c}
		W_1^T A V_3 = 0, \\
 		W_1^T A ( \widehat{V}_3 + W_1 \gamma_{3,1} + W_2 \gamma_{3,2} ) = 0, \\
 		W_1^T A \widehat{V}_3 + W_1^T A W_1 \gamma_{3,1} + W_1^T A W_2 \gamma_{3,2} = 0.
	\end{array}
$$

В силу $A$-ортогональности подпространств $\mathcal W_1$ и $\mathcal W_2$ матрица $W_1^T A W_2$ является нулевой:

$$
	W_1^T A W_2 = 0.
$$

Таким образом,

$$
	\begin{array}{c}
 		W_1^T A \widehat{V}_3 + W_1^T A W_1 \gamma_{3,1} = 0, \\
 		W_1^T A W_1 \gamma_{3,1} = W_1^T A \widehat{V}_3, \\
 		\gamma_{3,1} = \left ( W_1^T A W_1 \right ) ^{-1} W_1^T A \widehat{V}_3. \\
	\end{array}
$$

Аналогиным образом, из требования $A$-ортогональности подпространству $W_2$ следует:

$$
	\begin{array}{c}
		W_2^T A V_3 = 0, \\
 		W_2^T A ( \widehat{V}_3 + W_1 \gamma_{3,1} + W_2 \gamma_{3,2} ) = 0, \\
 		W_2^T A \widehat{V}_3 + W_2^T A W_1 \gamma_{3,1} + W_2^T A W_2 \gamma_{3,2} = 0.
	\end{array}
$$

В силу $A$-ортогональности подпространств $\mathcal W_1$ и $\mathcal W_2$ матрица $W_2^T A W_1$ является нулевой:

$$
	W_2^T A W_1 = 0,
$$

следовательно,

$$
	\begin{array}{c}
 		W_2^T A \widehat{V}_3 + W_2^T A W_2 \gamma_{3,2} = 0, \\
 		W_2^T A W_2 \gamma_{3,2} = W_2^T A \widehat{V}_3, \\
 		\gamma_{3,2} = \left ( W_2^T A W_2 \right ) ^{-1} W_2^T A \widehat{V}_3. \\
	\end{array}
$$

Подставляя полученные выражения для $\gamma_{3,1}$ и $\gamma_{3,2}$ в соотношение \ref{equation:LM:FS:V_3_orthogonalization}, получим:

$$
	V_3 = \widehat{V}_3 + \sum_{i=1}^2 W_i \left ( W_i^T A W_i \right ) ^{-1} W_i^T A \widehat{V}_3.
$$

Далее в матрице $V_3$ необходимо найти наибольшее количество столбцов, из которых сформировать матрицу $W_3$ так, чтобы матрица
$W_3^T A W_3$ оказалась невырожденной, причем опять же все столбцы матрицы $\widehat{W}_2$ (после $A$-ортогонализации) обязательно должны
войти в состав матрицы $W_3$. Оставшиеся столбцы помещаются в матрицу $\widehat{W}_3$:

$$
	V_3 =
		\begin{pmatrix}
			W_3 & \widehat{W}_3
		\end{pmatrix}
$$

По построению вектор-столбцы матрицы $W_3$ являются $A$-ортогональными вектор-столбцам матриц $W_1$ и $W_2$ и матрица $W_3^T A W_3$
является невырожденной, поэтому линейная оболочка вектор-столбцов $W_3$ можно считать подпространством $\mathcal W_3$ и при этом выполняются
свойства M-2 и M-3.

Построение подпространств необходимо продолжать аналогичным образом. На шаге $s$ с предыдущего шага $s-1$ будет получена матрица
$\widehat{V}_s$:

$$
	\widehat{V}_s =
		\begin{pmatrix}
			\widehat{W}_{s-1} & A W_{s-1}
		\end{pmatrix}
$$

В результате $A$-ортогонализации столбцов матрицы $\widehat{V}_s$ будет получена матрица $V_s$:

\begin{equation} \label{equation:LM:FS:V_s_orthogonalization}
	V_s = \widehat{V}_s + \sum_{i=1}^{s-1} W_i \left ( W_i^T A W_i \right ) W_i^T A \widehat{V}_s.
\end{equation}

Можно заметить, что в сумме, стоящей справа, отличными от нулевых матриц могут оказаться только три "последних"{} слагаемых
с номерами $s-1$, $s-2$ и $s-3$. Действительно, рассмотрим произведение $W_i^T A \widehat{V}_s$ для произвольного номера $i < s-3$:

$$
	W_i^T A \widehat{V}_s
		= W_i^T A \begin{pmatrix} \widehat{W}_{s-1} & A W_{s-1} \end{pmatrix}
		= \begin{pmatrix} W_i^T A \widehat{W}_{s-1} & W_i^T A A W_{s-1} \end{pmatrix}.
$$

Столбцы матрицы $\widehat{W}_{s-1}$ входят в состав столбцов матрицы $V_{s-1}$, которые по построению являются $A$-ортогональными всем
столбцам матриц $W_i$ с номерами $i = s-2, s-3, \dots, 1$, поэтому первое произведение является нулевой матрицей:

$$
	W_i^T A \widehat{W}_{s-1} = 0.
$$

Второе произведение удобней рассматривать в транспонированном виде:

$$
	\left ( W_i^T A A W_{s-1} \right ) ^T = W_{s-1}^T A A W_i,
$$

поскольку матрица $A$ является симметричной ($A^T = A$). Столбцы матрицы $A W_i$ совместно со столбцами матрицы $\widehat{W}_i$
образуют матрицу $\widehat{V}_{i+1}$, подвергаются $A$-ортогонализации, в результате которой получается матрица $V_{i+1}$,
из столбцов которой формируются две матрицы $W_{i+1}$ и $\widehat{W}_{i+1}$:

$$
	\begin{pmatrix}
		W_{i+1} & \widehat{W}_{i+1}
	\end{pmatrix}
	P_{i+1}
	=
	\begin{pmatrix}
		A W_i & \widehat{W}_i
	\end{pmatrix}
	+
	\sum_{j=1}^i W_j \left ( W_j^T A W_j \right ) ^{-1} W_j^T \begin{pmatrix}	A W_i & \widehat{W}_i \end{pmatrix},
$$

где $P_{i+1}$ --- некоторая перестановочная матрица, выполняющая перестановку столбцов матрицы
$\begin{pmatrix} W_{i+1} & \widehat{W}_{i+1} \end{pmatrix}$. Полученное соотношение можно записать в ином виде:

$$
	\begin{pmatrix}
		A W_i & \widehat{W}_i
	\end{pmatrix}
	=
	\begin{pmatrix}
		W_{i+1} & \widehat{W}_{i+1}
	\end{pmatrix}
	P_{i+1}
	+
	\sum_{j=1}^i W_j \left ( W_j^T A W_j \right ) ^{-1} W_j^T \begin{pmatrix} A W_i & \widehat{W}_i \end{pmatrix},
$$

из которого видно, что столбцы матрицы $A W_i$ являются линейной комбинацией столбцов матриц $W_{i+1}$, $W_i$, \dots, $W_1$ и матрицы
$\widehat{W}_{i+1}$.

Матрица $\widehat{W}_{i+1}$ совместно с матрицей $A W_{i+1}$ образует матрицу $\widehat{V}_{i+2}$, над столцами которой будет выполнена
$A$-ортогонализация и формирование двух матриц $W_{i+2}$ и $\widehat{W}_{i+2}$, поэтому для матриц $A W_{i+1}$ и $\widehat{W}_{i+1}$
справедливо аналогичное соотношение:

$$
	\begin{pmatrix}
		A W_{i+1} & \widehat{W}_{i+1}
	\end{pmatrix}
	=
	\begin{pmatrix}
		W_{i+2} & \widehat{W}_{i+2}
	\end{pmatrix}
	P_{i+2}
	+
	\sum_{j=1}^{i+1} W_j \left ( W_j^T A W_j \right ) ^{-1} W_j^T \begin{pmatrix} A W_{i+1} & \widehat{W}_{i+1} \end{pmatrix}
	.
$$

В соответствии со способом формирования матрицы $W_{i+2}$ все столбцы матрицы $\widehat{W}_{i+1}$ (после $A$-ортогонализации) входят в
состав матрицы $W_{i+2}$, поэтому столбцы матрицы $\widehat{W}_{i+1}$ являются линейной комбинацией столбцов матриц $W_{i+2}$,
$W_{i+1}$, \dots, $W_1$, но не матрицы $\widehat{W}_{i+2}$.

Таким образом, столбцы матрицы $A W_i$ являются линейной комбинацией столбцов матриц $W_{i+2}$, $W_{i+1}$, \dots, $W_1$. Поскольку номер
$i < s-3$, то $i+2 < s-1$ и матрица $W_{s-1}$ имеет номер $s-1$ превосходящий номера $i+2$, $i+1$, \dots, $1$, поэтому матрица
$W_{s-1}$ по построению $A$-ортогональна матрицам $W_{i+2}$, $W_{i+1}$, \dots, $W_1$ и следовательно матрица $W_{s-1}$
$A$-ортогональна и матрице $A W_i$, столбцы которой являются линейной комбинацией столбцов матриц $W_{i+2}$, $W_{i+1}$, \dots, $W_1$:

$$
	\left ( W_i^T A A W_{s-1} \right ) ^T = W_{s-1}^T A A W_i = 0,
$$

для всех $i < s-3$.

Отсюда следует, что выражение \ref{equation:LM:FS:V_s_orthogonalization} имеет более короткий вид:

$$
	\begin{array}{ccl}
		V_s & = & \widehat{V}_s + \\
	        &   & + W_{s-1} \left ( W_{s-1}^T A W_{s-1} \right ) ^{-1} W_{s-1}^T A \widehat{V}_s + \\
	        &   & + W_{s-2} \left ( W_{s-2}^T A W_{s-2} \right ) ^{-1} W_{s-2}^T A \widehat{V}_s + \\
	        &   & + W_{s-3} \left ( W_{s-3}^T A W_{s-3} \right ) ^{-1} W_{s-3}^T A \widehat{V}_s.
	\end{array}
$$

Далее из столбцов матрицы $V_s$ набирается матрица $W_s$ так, чтобы матрица $W_s^T A W_s$ была невырожденной, а остальным столбцы образуют
матрицу $\widehat{W}_s$.

Процесс построения матриц $W_s$, вектор-столбцы которых являются базисами соответствующих подпространст $\mathcal W_s$, завершается 
на некотором шаге $s$ в двух случаях:

\begin{itemize}
	\item некоторые столбцы из матрицы $\widehat{W}_{s-1}$ "не попали"{} в матрицу $W_s$;
	\item матрица $V_s$ оказалась нулевой.
\end{itemize}

В первом случае решение возможно построить не удастся, а во втором случае решение будет найдено.


\part{Решение систем в $\gfpvector{n}$}

\chapter{Методы Видемана-Копперсмита}

\section{Предварительные соглашения}

Везде далее будем считать, что в линейном пространстве $\gfpvector{n}$ задан линейный оператор $\mathcal A$, действие которого
определяется матрицей $A \in \gfpmatrix{n}{n}$.

\chapter{Решение систем уравнений с помощью аннулирующих полиномов}

\section{Аннулирующие полиномы}

Определения, используемые в данном разделе, взяты из книги \cite[с.~171--172]{Gantmacher}.

\subsection{Аннулирующие полиномы пространства $\gfpvector{n}$}

\begin{definition} \label{definition:AP:space:annihilating_polynomial}
	Полином $\polynomial{\psi} \in \gfppolynomial$ называется \definedterm{аннулирующим полиномом пространства $\gfpvector{n}$
	относительно оператора $\mathcal A$}, если матрица $\psi(A)$ является нулевой :
		$$ \psi(A) = 0. $$
\end{definition}

\begin{definition}
	Полином $\polynomial{\tilde\psi} \in \gfppolynomial$ называется \definedterm{минимальным аннулирующим полином пространства
	$\gfpvector{n}$ отностительно оператора $\mathcal A$}, если его степень не больше степени любого другого аннулирующего полинома
	$\polynomial{\psi}$ пространства $\gfpvector{n}$ относительно оператора $\mathcal A$:
		$$ \polynomialdegree{\tilde\psi} \le \polynomialdegree{\psi}. $$
\end{definition}

Везде далее для краткости слова "относительно оператора $\mathcal A$"{} опущены.

\begin{statement} \label{statement:AP:space:polynomials_division}
	Пусть
	\begin{enumerate}
		\item $\polynomial{\psi}$ --- аннулирущий полином пространства $\gfpvector{n}$,
		\item $\polynomial{\tilde\psi}$ --- минимальный аннулирущий полином пространства $\gfpvector{n}$,
	\end{enumerate}
	тогда полином $\polynomial{\psi}$ делится без остатка на полином $\polynomial{\tilde \psi}$:

		$$ \polynomial{\psi} = \polynomial{\delta} \polynomial{\tilde\psi} $$

	где $\polynomial{\delta} \in \gfppolynomial$ --- некоторый полином.

	\proof

	Поскольку $\polynomial{\tilde\psi}$ по условию является минимальным аннулирующим полиномом, то его степень не больше степени полинома
	$\polynomial{\psi}$. В таком случае полином $\polynomial{\psi}$ можно разделить на полином $\polynomial{\tilde\psi}$, получив полиномы
	целой части $\polynomial{\delta}$ и остатка $\polynomial{\varepsilon}$:

		\begin{equation} \label{equation:AP:space:polynomials_division}
			\polynomial{\psi} = \polynomial{\delta} \polynomial{\tilde \psi} + \polynomial{\varepsilon}
		\end{equation}

	где степень полинома $\polynomial{\varepsilon}$ меньше степени полинома $\polynomial{\tilde\psi}$:

		\begin{equation} \label{equation:AP:space:remainder_polynomial_degree}
			\polynomialdegree{\varepsilon} < \polynomialdegree{\tilde\psi}.
		\end{equation}

	Подставляя в равенство \eqref{equation:AP:space:polynomials_division} матрицу $A$, получим равенство для матриц:

		$$ \psi(A) = \delta(A) \tilde\psi(A) + \varepsilon(A), $$

	в котором $\psi(A) = 0$ и $\tilde\psi(A) = 0$ поскольку полиномы $\polynomial{\psi}$ и $\polynomial{\tilde\psi}$ являются аннулирущими
	полиномами пространства $\gfpvector{n}$. Таким образом

		$$ 0 = \varepsilon(A) $$

	и следовательно полином $\polynomial{\varepsilon}$ тоже является аннулирующим полиномом пространства $\gfpvector{n}$, и к тому же имеет
	степень меньше степени минимального аннулирущего полинома $\polynomial{\tilde\psi}$ в соответствии с неравенством
	\eqref{equation:AP:space:remainder_polynomial_degree}. Это возможно только в том случае, если $\polynomial{\varepsilon} \equiv 0$, в
	противном случае, если $\polynomial{\varepsilon}$ --- ненулевой полином, то полином $\polynomial{\tilde\psi}$ не является минимальным
	аннулирующим полиномом, что противоречит условию утверждения.

	Таким образом, $\polynomial{\varepsilon} \equiv 0$ и равенство \eqref{equation:AP:space:polynomials_division} принимает вид:

		$$ \polynomial{\psi} = \polynomial{\delta} \polynomial{\tilde\psi} $$
\end{statement}

\subsection{Аннулирующие полиномы векторов из $\gfpvector{n}$}

\begin{definition}
	Полином $\polynomial{\varphi} \in \gfppolynomial$ называется \definedterm{аннулирующим полиномом вектора $x \in \gfpvector{n}$
	относительно оператора $\mathcal A$}, если вектор $\varphi(A) x$ является нулевым:
		$$ \varphi(A) x = 0. $$
\end{definition}

\begin{definition}
	Полином $\polynomial{\tilde\varphi} \in \gfppolynomial$ называется \definedterm{минимальным аннулирующим полином вектора
	$x \in \gfpvector{n}$ отностительно оператора $\mathcal A$}, если его степень не больше степени любого другого аннулирующего полинома
	$\polynomial{\varphi}$ вектора $x \in \gfpvector{n}$ относительно оператора $\mathcal A$:
		$$ \polynomialdegree{\tilde\varphi} \le \polynomialdegree{\varphi}. $$
\end{definition}

Опять же для краткости везде далее слова "относительно оператора $\mathcal A$"{} опущены.

Совершенно аналогично утверждению \ref{statement:AP:space:polynomials_division} можно показать, что всякий аннулирующий полином вектора
делится без остатка на минимальный аннулирующий полином вектора.

\begin{statement} \label{statement:AP:vector:polynomials_division}
	Пусть
	\begin{enumerate}
		\item $x \in \gfpvector{n}$ --- произвольный вектор,
		\item $\polynomial{\varphi}$ --- аннулирущий полином вектора $x$,
		\item $\polynomial{\tilde\varphi}$ --- минимальный аннулирущий полином вектора $x$,
	\end{enumerate}
	тогда полином $\polynomial{\varphi}$ делится без остатка на полином $\polynomial{\tilde\varphi}$:

		$$ \polynomial{\psi} = \polynomial{\delta} \polynomial{\tilde\psi} $$

	где $\polynomial{\delta} \in \gfppolynomial$ --- некоторый полином.

	\proof

	Степень полинома $\polynomial{\tilde\varphi}$ не больше степени полинома $\polynomial{\varphi}$, поскольку $\polynomial{\tilde\varphi}$
	по условию является минимальным аннулирующим полиномом. Разделив полином $\polynomial{\varphi}$ на полином $\polynomial{\tilde\varphi}$
	получим полиномы целой части $\polynomial{\delta}$ и остатка $\polynomial{\varepsilon}$:

		\begin{equation} \label{equation:AP:vector:polynomials_division}
			\polynomial{\varphi} = \polynomial{\delta} \polynomial{\tilde\varphi} + \polynomial{\varepsilon}
		\end{equation}

	где степень полинома $\polynomial{\varepsilon}$ меньше степени полинома $\polynomial{\tilde\varphi}$:

		\begin{equation} \label{equation:AP:vector:remainder_polynomial_degree}
			\polynomialdegree{\varepsilon} < \polynomialdegree{\tilde\varphi}.
		\end{equation}

	Подставляя в равенство \eqref{equation:AP:vector:polynomials_division} матрицу $A$, получим равенство для матриц:

		$$ \varphi(A) = \delta(A) \tilde\varphi(A) + \varepsilon(A), $$

	умножая которое на $x$ справа, получим аналогичное равенство для векторов:

		$$ \varphi(A) x = \delta(A) \tilde\varphi(A) x + \varepsilon(A) x, $$

	в котором $\varphi(A) x = 0$ и $\tilde\varphi(A) x = 0$ поскольку полиномы $\polynomial{\varphi}$ и $\polynomial{\tilde\varphi}$
	являются аннулирущими полиномами вектора $x$. Таким образом

		$$ 0 = \varepsilon(A) x $$

	и следовательно полином $\polynomial{\varepsilon}$ является аннулирующим полиномом вектора $x$, имеющим степень меньше степени минимального
	аннулирущего полинома $\polynomial{\tilde\varphi}$ в соответствии с неравенством \eqref{equation:AP:vector:remainder_polynomial_degree}.
	Это возможно только в том случае, если $\polynomial{\varepsilon} \equiv 0$, в противном случае, если $\polynomial{\varepsilon}$ ---
	ненулевой полином, то полином $\polynomial{\tilde\varphi}$ не является минимальным аннулирующим полиномом вектора $x$, что противоречит
	условию утверждения.

	Таким образом, $\polynomial{\varepsilon} \equiv 0$ и равенство \eqref{equation:AP:vector:polynomials_division} принимает вид:

		$$ \polynomial{\varphi} = \polynomial{\delta} \polynomial{\tilde\varphi} $$
\end{statement}

Согласно определению \ref{definition:AP:space:annihilating_polynomial} аннулирующий полином всего пространства $\gfpvector{n}$ является
аннилурищим полиномом любого вектора $x$ из пространства $\gfpvector{n}$. Отсюда в частности следует делимость аннулирущих полиномов
пространства на все минимальные аннулирующие полиномы всех векторов пространства $\gfpvector{n}$.

\begin{statement} \label{statement:AP:vector:space_polynomials_division}
	Пусть
	\begin{enumerate}
		\item $\polynomial{\psi}$ --- аннулирующий полином пространства $\gfpvector{n}$,
		\item $\polynomial{\tilde\varphi}$ --- минимальный аннулирующий полином некоторого вектора $x \in \gfpvector{n}$,
	\end{enumerate}
	тогда полином $\polynomial{\psi}$ делится без остатка на полином $\polynomial{\tilde\varphi}$:

		$$ \polynomial{\psi} = \polynomial{\delta} \polynomial{\tilde\varphi}, $$

	где $\polynomial{\delta} \in \gfppolynomial$ --- некоторый полином.

	\proof

	Поскольку $\polynomial{\psi}$ является аннулирующим полиномом пространства $\gfpvector{n}$, то:

		$$ \psi(A) = 0. $$

	и следовательно

		$$ \psi(A) x = 0. $$

	Таким образом, полином $\polynomial{\psi}$ является аннулирущим полиномом вектора $x$ и в соответствии с утверждением
	\ref{statement:AP:vector:polynomials_division} полином $\polynomial{\psi}$ делится без остатка на полином
	$\polynomial{\tilde\varphi}$
\end{statement}

\subsection{Делимость аннулирующих полиномов}

Согласно доказанным утверждениям имеет место следующая делимость полиномов: любой аннулирующий полином пространства $\gfpvector{n}$
делится без остатка на минимальные аннулирующие полиномы пространства $\gfpvector{n}$, каждый из которых в свою очередь делится без
остатка на любой минимальный аннулирующий полином любого вектора из $\gfpvector{n}$.

Из приведенной делимости следует, что аннулирующие полиномы всего пространства $\gfpvector{n}$ и отдельных векторов получаются из
минимальных аннулирующих полиномов пространства $\gfpvector{n}$ и векторов путем умножения на различные полиномы.

\subsection{Аннулирующие полиномы и вырожденность оператора}

Структура аннулирущих полиномов существенно зависит от оператора $\mathcal A$, например, если оператор является вырожденным, то
аннулирующие полиномы пространства начинаются с некоторой степени $\lambda^k$ и не содержат несколько младших степеней аргумента
(по крайней мере нулевую).

Наиболее сильным образом связан с оператором характеристический полином матрицы $A$ оператора: если оператор не вырожденный, то
характеристический полином обязательно имеет ненулевой коэффициент при нулевой степени аргумента. Это свойство характеристического полинома
переносится на минимальные аннулирущие полиномы за счет делимости характеристического полинома на минимальные аннулирущие
полиномы всего пространства и отдельных векторов.

\begin{statement} \label{statement:APD:space:zero_determinant_and_space_annihilating_polynomials}
	Пусть $\polynomial{\psi}$ --- аннулирущий полином пространства $\gfpvector{n}$, тогда

		$$
			\left \{ \left | A \right | = 0 \right \}
			\Rightarrow
			\left \{
				\begin{array}{c}
					\psi(\lambda) = \psi_k \lambda^k + \psi_{k+1} \lambda^{k+1} + \dots + \psi_d \lambda^{d},\\
					k \ge 1
				\end{array}
			\right \}
		$$

	\proof

	\begin{enumerate}

		\item Пусть полином $\polynomial{\psi}$ имеет вид:

			$$ \psi(\lambda) = \psi_0 \lambda + \psi_1 \lambda + \psi_2 \lambda^2 + \ldots + \psi_d \lambda^d. $$

			Поскольку $\polynomial{\psi}$ --- аннулирущий полином пространства, то

				$$ \psi(A) = 0, $$
				$$ \psi_0 E + \psi_1 A + \psi_2 A^2 + \ldots + \psi_d A^d = 0, $$
				$$ \psi_0 E = -\psi_1 A - \psi_2 A^2 - \ldots - \psi_d A^d, $$
				$$ \psi_0 E = \left ( -\psi_1 E - \psi_2 A^1 - \ldots - \psi_d A^{d-1} \right ) \cdot A, $$
				$$ \left | \psi_0 E \right | = \left | -\psi_1 E - \psi_2 A^1 - \ldots - \psi_d A^{d-1} \right | \cdot \left | A \right |, $$
				$$ \psi_0^n = \left | -\psi_1 E - \psi_2 A^1 - \ldots - \psi_d A^{d-1} \right | \cdot \left | A \right |. $$

			Поскольку по условию утверждения $\left | A \right | = 0$, то

				$$ \psi_0^n = \left | -\psi_1 E - \psi_2 A^1 - \ldots - \psi_d A^{d-1} \right | \cdot 0 = 0, $$
				$$ \psi_0^n = 0. $$

			Поскольку во всяком поле нет делителей нуля, то из последнего равенства следует

				$$ \psi_0 = 0. $$

			и полином $\polynomial{\psi}$ должен иметь вид:

				$$ \polynomial{\psi} = \psi_1\lambda + \psi_2\lambda^2 + \ldots + \psi_d\lambda^d, $$

		\item Из последнего равенства следует, что

				$$ \polynomial{\psi} = \lambda \left ( \psi_1 + \psi_2\lambda + \ldots + \psi_d\lambda^{d-1} \right ). $$

			и вполне возможно, что полином $\polynomial{\psi^{(1)}}$ стоящий в скобках:

				$$ \polynomial{\psi^{(1)}} = \psi_1 + \psi_2\lambda + \ldots + \psi_d\lambda^{d-1} $$

			так же является аннулирующим полиномом пространства $\gfpvector{n}$. В этом случае аналогично пункту 1 доказательства:

				$$ \psi_1 = 0. $$

			Продолжая рассуждения подобным образом, в конечном счете можно придти к тому, что полином $\polynomial{\psi}$ обязательно
			имеет вид:

				$$ \psi(\lambda) = \psi_k \lambda^k + \psi_{k+1} \lambda^{k+1} + \dots + \psi_d \lambda^{d}, $$

			где $k \ge 1$.

	\end{enumerate}
\end{statement}

\begin{definition}
	\definedterm{Характеристическим полиномом} матрицы $A \in \gfpmatrix{n}{n}$ называется полином $\polynomial{c_A}$:
	$$ \polynomial{c_A} = \left | A - \lambda E \right |. $$
\end{definition}

\begin{statement} \label{statement:APD:determinant_and_characteristic_polynomial}
		$$
			\left \{
				\left | A \right | \neq 0
			\right \}
			\Rightarrow
			\left \{
				\begin{array}{c}
					c_A(\lambda) = c_0 + c_1 \lambda + \dots + c_n \lambda^n,\\
					c_0 \neq 0
				\end{array}
			\right \}
		$$
		$$
			\left \{
				\left | A \right | = 0
			\right \}
			\Rightarrow
			\left \{
				\begin{array}{c}
					c_A(\lambda) = c_k \lambda^k + c_{k+1} \lambda^{k+1} + \dots + c_n \lambda^n,\\
					k \ge 1
				\end{array}
			\right \}
		$$

	\proof

	Пусть характеристический полином $\polynomial{c_A}$ имеет вид

		$$c_A(\lambda) = c_0 + c_1 \lambda + \dots + c_n \lambda^n. $$

	Легко видеть, что

		$$ c_0 = c_A ( 0 ) = \left | A - 0 \cdot E \right | = \left | A \right |.$$

	Отсюда сразу следует, что если $\left | A \right | \neq 0$, то $c_0 \neq 0$.

	Если $\left | A \right | = 0$, то $c_0 = 0$. Если $0$ является корнем кратности $k$, то в этом случае $k$ коэффициентов при младших
	степенях аргумента в характеристическом полиноме равны $0$,	следовательно характеристический полином $\polynomial{c_A}$ имеет вид

		$$ c_A(\lambda) = c_k \lambda^k + c_{k+1} \lambda^{k+1} + \dots + c_n \lambda^n, $$

	где $k \ge 1$.

\end{statement}

\begin{corollary} \label{corollary:APD:determinant_and_characteristic_polynomial}
	Тем самым доказаны два критерия:
		$$
			\left \{
				\left | A \right | \neq 0
			\right \}
			\Leftrightarrow
			\left \{
				\begin{array}{c}
					c_A(\lambda) = c_0 + c_1 \lambda + \dots + c_n \lambda^n,\\
					c_0 \neq 0
				\end{array}
			\right \}
		$$
		$$
			\left \{
				\left | A \right | = 0
			\right \}
			\Leftrightarrow
			\left \{
				\begin{array}{c}
					c_A(\lambda) = c_k \lambda^k + c_{k+1} \lambda^{k+1} + \dots + c_n \lambda^n,\\
					k \ge 1
				\end{array}
			\right \}
		$$

	\proof

	Докажем первый критерий от противного: пусть коэффициент $c_0 \neq 0$, а $\left | A \right | = 0$. В таком случае, в соответствии
	с утверждением \ref{statement:APD:determinant_and_characteristic_polynomial} характеристический полином $\polynomial{c_A}$ должен иметь
	вид

		$$ c_A(\lambda) = c_k \lambda^k + c_{k+1} \lambda^{k+1} + \dots + c_n \lambda^n, $$
		$$ k \ge 1, $$

	в котором $c_0 = 0$, что противоречит исходному предположению.

	Второй критерий доказывается аналогичным образом от противного.
\end{corollary}

\begin{statement} \label{statement:APD:determinant_and_minimal_space_annihilating_polynomial}
	Пусть $\polynomial{\tilde\psi}$ --- минимальный аннулирущий полином пространства $\gfpvector{n}$, тогда

		$$
			\left \{
				\left | A \right | \neq 0
			\right \}
			\Rightarrow
			\left \{
				\begin{array}{c}
					\tilde\psi(\lambda) = \tilde\psi_0 + \tilde\psi_1 \lambda + \dots + \tilde\psi_d \lambda^d,\\
					\tilde\psi_0 \neq 0
				\end{array}
			\right \}
		$$

	\proof

	По теореме Кели--Гамильтона \cite[с.~93]{Gantmacher} характеристический полином $\polynomial{c_A}$ матрицы $A$ является аннулирующим
	полиномом пространства $\gfpvector{n}$:

		$$ c_A ( A ) = 0. $$

	В таком случае согласно утверждению \ref{statement:AP:space:polynomials_division} характеристический полином $\polynomial{c_A}$
	делится без остатка на минимальный аннулирующий полином	$\polynomial{\tilde\psi}$, то есть для некоторого полинома
	$\polynomial{\delta} \in \gfppolynomial$:

		$$ \delta(\lambda) = \delta_0 + \delta_1 \lambda + \dots + \delta_{n-d} \lambda^{n-d} $$

	 имеет место разложение:

		\begin{equation} \label{equation:APD:characteristic_polynomial_factorization}
			\polynomial{c_A} = \polynomial{\delta} \polynomial{\tilde\psi}.
		\end{equation}

	В соответствии с утверждением \ref{statement:APD:determinant_and_characteristic_polynomial} в левой части коэффициент при нулевой
	степени $c_0 \neq 0$, а в правой части коэффициент при нулевой степени получается умножением $\delta_0 \cdot \tilde\psi_0$:

		$$ 0 \neq c_0 = \delta_0 \cdot \tilde\psi_0 $$

	Откуда в частности следует, что

		$$ \tilde\psi_0 \neq 0. $$
\end{statement}

\begin{statement} \label{statement:APD:determinant_and_minimal_vector_annihilating_polynomial}
	Пусть $\polynomial{\tilde\varphi}$ --- минимальный аннулирущий полином некоторого вектора $x \in \gfpvector{n}$, тогда

		$$
			\left \{
				\left | A \right | \neq 0
			\right \}
			\Rightarrow
			\left \{
				\begin{array}{c}
					\tilde\varphi(\lambda) = \tilde\varphi_0 + \tilde\varphi_1 \lambda + \dots + \tilde\varphi_d \lambda^d,\\
					\tilde\varphi_0 \neq 0
				\end{array}
			\right \}
		$$

	\proof

	Доказательство этого утверждения аналогично доказательству утверждения
	\ref{statement:APD:determinant_and_minimal_space_annihilating_polynomial}.

	По теореме Кели--Гамильтона \cite[с.~93]{Gantmacher} характеристический полином $\polynomial{c_A}$ матрицы $A$ является аннулирующим
	полиномом пространства $\gfpvector{n}$, и в соответствии с утверждением \ref{statement:AP:vector:space_polynomials_division} делится без
	остатка на минимальный аннулирующий полином $\polynomial{\tilde\varphi}$ вектора $x$, то есть для некоторого полинома
	$\polynomial{\delta} \in \gfppolynomial$:

		$$ \delta(\lambda) = \delta_0 + \delta_1 \lambda + \dots + \delta_{n-d} \lambda^{n-d} $$

	имеет место разложение:

		$$ \polynomial{c_A} = \polynomial{\delta} \polynomial{\tilde\varphi}, $$

	в котором коэффициент при нулевой степени слева $c_0 \neq 0$ в соответствии с утверждением
	\ref{statement:APD:determinant_and_characteristic_polynomial}, а справа равен $\delta_0 \cdot \tilde\varphi_0$:

		$$ 0 \neq c_0 = \delta_0 \cdot \tilde\varphi_0, $$

	следовательно

		$$ \tilde\varphi_0 \neq 0. $$
\end{statement}

\subsection{Аннулирущие полиномы и подпространства Крылова} \label{section:KS:krylov_spaces}

\begin{definition}
	\definedterm{Подпространством Крылова $\mathcal K_r$ размера $r$ для вектора $x \in \gfpvector{n}$ и матрицы $A \in \gfpmatrix{n}{n}$}
	называется линейная оболочка векторов $x$, $Ax$, $A^2x$, \dots, $A^{r-1}x$:
		$$ \mathcal K_r (x,A) = \mathcal L \left \{ x, Ax, A^2x, \dots, A^{r-1}x \right \} $$
\end{definition}

\begin{definition}
	Векторы $A^k x$ называются \definedterm{векторами Крылова}.
\end{definition}

Легко видеть, что при любом $r$ подпространство $\mathcal K_r(x,A) \subseteq \gfpvector{n}$, и поскольку размерность $\gfpvector{n}$ равна $n$,
то размерности подпространств $\mathcal K_r(x,A)$ не могут быть больше $n$. Отсюда следует, что среди векторов Крылова
$x$, $Ax$, $A^2x$, \dots, $A^{r-1}x$ не может быть больше $n$ линейно независимых векторов, и следовательно при некоторых $r$ векторы Крылова
оказываются линейно зависимыми (во всяком случае при $r > n$).

Пусть вектор $x$ выбран произвольным образом и зафиксирован, и пусть при некотором $r$ векторы Крылова $x$, $Ax$, $A^2x$, \dots, $A^rx$
в количестве $r+1$ являются линейно зависимыми, тогда их линейная комбинация с некоторыми коэффициентами $\varphi_i$ равна нулевому вектору:

	$$ \varphi_0 x + \varphi_1 Ax + \varphi_2 A^2x + \dots + \varphi_{r-1} A^{r-1}x + \varphi_r A^rx = 0. $$

Отсюда, вынося вектор $x$, получим равенство

	\begin{equation} \label{equation:KS:annihilating_x}
		\left ( \varphi_0  + \varphi_1 A + \varphi_2 A^2 + \dots + \varphi_{r-1} A^{r-1} + \varphi_r A^r \right ) x = 0,
	\end{equation}

из которого следует, что полином $\polynomial{\varphi}$:

	$$ \varphi ( \lambda ) = \varphi_0  + \varphi_1 \lambda + \varphi_2 \lambda^2 + \dots + \varphi_{r-1} \lambda^{r-1} + \varphi_r \lambda^r $$

является аннулирующим полиномом вектора $x$, поскольку из равенства \eqref{equation:KS:annihilating_x}:

	$$ \varphi ( A ) x = 0. $$

Таким образом, известная линейная зависимость векторов Крылова $x$, $Ax$, $A^2x$, \dots, $A^rx$ приводит к аннулирующему полиному вектора $x$.

Пусть теперь число $\tilde r$ является наибольшим количеством линейно независимых векторов Крылова $x$, $Ax$, $A^2x$, \dots, $A^{\tilde r-1}x$,
тогда векторы Крылова $x$, $Ax$, $A^2x$, \dots, $A^{\tilde r-1}x$, $A^{\tilde r}x$ в количестве $\tilde r + 1$ уже являются линейно
зависимыми (поскольку $\tilde r$ --- это наибольшее количество линейно независимых векторов Крылова) и следовательно, как и ранее, существует
такая линейная комбинация этих векторов с коэффициентами $\tilde \varphi_i$, которая является нулевым вектором, откуда полином
$\polynomial{\tilde \varphi}$ с коэффициентами $\tilde \varphi_i$:

$$ \tilde \varphi ( \lambda ) =
	\tilde \varphi_0  + \tilde \varphi_1 \lambda + \tilde \varphi_2 \lambda^2 + \dots
		+ \tilde \varphi_{\tilde r - 1} \lambda^{\tilde r - 1}
		+ \tilde \varphi_{\tilde r} \lambda^{\tilde r} $$

является аннулирующим полиномом вектора $x$. Более того, полином $\polynomial{\tilde \varphi}$ является минимальным аннулирующим полиномом
вектора $x$: если допустить существование аннулирующего полинома некоторой меньшей степени $k < \tilde r$, то тогда окажется, что линейно
зависимыми являются векторы Крылова $x, Ax, A^2x, \dots, A^{k-1}x, A^kx$ в количестве $k+1$, что невозможно, поскольку в силу определения
числа $\tilde r$ векторы Крылова $x, Ax, A^2x, \dots, A^{k-1}x, A^kx, \dots, A^{\tilde r-1}x$ в количестве $\tilde r \ge k+1$ являются
линейно независимыми.

Таким образом, построение векторов Крылова $x, Ax, A^2x, \dots, A^{r-1}x$ и анализ их линейной зависимости приводит к построению
аннулирующих полиномов вектора $x$, в том числе и минимальных.

Легко видеть, что для каждого вектора $x$ соответствующее наибольшее количество линейно независимых векторов Крылова $r$ определяется
однозначным образом и вообще говоря может быть разным для разных векторов. Например, нетрудно себе представить вектор $x \neq 0$
из ядра оператора $\mathcal A$, для которого $Ax = 0$ и вообще для любого $k \ge 2$ $A^kx = A^{k-1} \cdot Ax = A^{k-1} \cdot 0 = 0$,
так что число $r$ линейно независимых векторов Крылова для такого вектора $x$ равно 1 (это сам вектор $x$).

\section{Решение систем уравнений над полем $\gfp$}

\subsection{Неоднородная система с невырожденной матрицей}\label{subsection:SAP:nonuniform}

Пусть требуется найти решение неоднородной системы:

	\begin{equation} \label{equation:SAP:nonuniform_system}
		A x = b,
	\end{equation}
	$$ \left| A \right| \neq 0, $$

где $A \in \gfpmatrix{n}{n}$ и $b \in \gfpvector{n}$.

Представим, что некоторым образом найден аннулирующий полином $\polynomial{\varphi}$ вектора $b$ степени $d$:

	$$ \varphi(\lambda) =
		\varphi_0 + \varphi_1\lambda + \varphi_2\lambda^2 + \ldots + \varphi_d\lambda^d = \sum_{i=0}^d \varphi_i\lambda^i, $$

в котором

	$$ \varphi_0 \neq 0. $$

Поскольку полином $\polynomial{\varphi}$ является аннулирующим для $b$, то:

	$$ \varphi(A)b = 0, $$
	$$ \varphi_0b + \varphi_1Ab + \varphi_2A^2b + \ldots + \varphi_dA^db = 0, $$
	$$ \varphi_0b = -\varphi_1Ab - \varphi_2A^2b - \ldots - \varphi_dA^db, $$

Поскольку коэффициент $ \varphi_0 \neq 0 $, то в поле $\gfp$ существует обратный по умножению элемент $\varphi_0^{-1}$, тогда:

	$$ b = \varphi_0^{-1} \left(-\varphi_1Ab - \varphi_2A^2b - \ldots - \varphi_dA^db \right), $$
	$$ b = \varphi_0^{-1} A \left(-\varphi_1b - \varphi_2Ab - \ldots - \varphi_dA^{d-1}b \right), $$
	$$ b = A \left\{ -\varphi_0^{-1} \left(\varphi_1b + \varphi_2Ab + \ldots + \varphi_dA^{d-1}b \right) \right\}, $$
	$$ b = A \left\{ -\varphi_0^{-1} \sum_{i=1}^d \varphi_iA^{i-1}b \right\}, $$

откуда следует, что решением системы \eqref{equation:SAP:nonuniform_system} является вектор:

	$$ x=-\varphi_0^{-1} \sum_{i=1}^d \varphi_iA^{i-1}b, $$

который принадлежит пространству Крылова $\mathcal K_d ( b, A)$.

Таким образом, для нахождения решения системы~\eqref{equation:SAP:nonuniform_system} достаточно найти аннулирующий полином
$\polynomial{\varphi}$ вектора $b$ с коэффициентом $\varphi_0 \neq 0$. Одним из таких полиномов является например минимальный аннулирущий
полином $\polynomial{\tilde \varphi}$ вектора $b$, поскольку из условия $\left | A \right | \neq 0$ согласно утверждению
\ref{statement:APD:determinant_and_minimal_vector_annihilating_polynomial} следует $\tilde \varphi_0 \neq 0$.

\subsection{Однородная система с вырожденной матрицей}\label{subsection:SAP:uniform}

Пусть требуется найти нетривиальное решение однородной системы:

	\begin{equation}\label{equation:SAP:uniform_system}
		A x = 0,
	\end{equation}
	$$ \left | A \right | = 0, $$

где $A \in \gfpmatrix{n}{n}$.

\subsubsection{Метод нахождения решения}

Выберем произвольным образом вектор $z \in \gfpvector{n}$ и представим, что некоторым образом найден аннулирущий полином
$\polynomial{\varphi}$ вектора z:

	$$ \varphi(\lambda) =
		\varphi_k \lambda^k + \varphi_{k+1} \lambda^{k+1} + \ldots + \varphi_d \lambda^d = \sum_{i=k}^d \varphi_i\lambda^i, $$

в котором

	$$ k \ge 1. $$

Поскольку полином $\polynomial{\varphi}$ является аннулирующим для вектора $z$, то

	$$ \varphi(A) z = 0, $$
	$$ \left ( \varphi_k A^k + \varphi_{k+1} A^{k+1} + \ldots + \varphi_d A^d \right ) z = 0, $$
	$$ A^k \left ( \varphi_k + \varphi_{k+1} A + \ldots + \varphi_d A^{d-k} \right ) z = 0. $$

Обозначим в качестве $\polynomial{\varphi_k}$ полином, стоящий в скобках,

	$$ \varphi_k(\lambda) = \varphi_k + \varphi_{k+1} \lambda + \ldots + \varphi_d \lambda^{d-k}, $$

тогда

	\begin{equation} \label{equation:SAP:vector_z_annihilation}
		A^k \varphi_k(A) z = 0.
	\end{equation}

Если вектор $\varphi_k(A) z$ является нулевым, то, к сожалению, решение системы \eqref{equation:SAP:uniform_system} найти не удастся, и
придется повторить попытку с другим вектором $z$ и аннулирущим полиномом $\polynomial{\varphi}$.

Если вектор $\varphi_k(A) z$ не является нулевым, то необходимо вычислять векторы $A^s \varphi_k(A) z$ для $s=1,\dots,k$, до тех пор пока
не будет получен нулевой вектор. Рано или поздно нулевой вектор будет получен, поскольку по крайней мере для $s=k$ имеет место равенство
\eqref{equation:SAP:vector_z_annihilation}. Предположим, что при некотором наименьшем возможном $s$ имеют место равенства:

	$$ A^{s-1} \varphi_k(A) z \neq 0 $$
	$$ A^s \varphi_k(A) z = 0 $$

тогда из последнего равенства

	$$ A \left \{ A^{s-1} \varphi_k(A) z \right \} = 0 $$

и следовательно ненулевой вектор $x$:

	$$ x = A^{s-1} \varphi_k(A) z \neq 0 $$

является нетривиальным решением системы \eqref{equation:SAP:uniform_system}.

Легко видеть, что в данном случае вектор--решение $x$ принадлежит пространству Крылова $\mathcal K_{d-k+s}(z,A)$:

	$$ x = A^{s-1} \sum_{i=k}^d \varphi_i A^{i-k} z = \sum_{i=k}^d \varphi_i A^{i-k+s-1} z $$

\subsubsection{Анализ метода нахождения решения}

Успех в нахождении нетривиального решения системы \eqref{equation:SAP:uniform_system} зависит от того, удастся ли найти вектор $z$ и
аннулирущий полином $\polynomial{\varphi}$ такие, что:

	$$ \varphi(A) z = 0, $$
	$$ \polynomial{\varphi} =
		\lambda^k \left ( \varphi_k + \varphi_{k+1} \lambda + \ldots + \varphi_d \lambda^{d-k} \right ), $$
	$$ \left ( \varphi_k + \varphi_{k+1} A + \ldots + \varphi_d A^{d-k} \right ) z \neq 0 $$

Одним из полиномов $\polynomial{\varphi}$, удовлетворяющим все трем условиям для некоторых векторов $z$, является в частности
минимальный аннулирущий полином $\polynomial{\tilde\psi}$ пространства $\gfpvector{n}$. Действительно, поскольку 
$\polynomial{\tilde\psi}$ является аннулирущим полиномом пространства $\gfpvector{n}$, то

	$$ \tilde\psi(A) z = 0. $$

Далее, согласно утверждению \ref{statement:APD:space:zero_determinant_and_space_annihilating_polynomials} полином $\polynomial{\tilde\psi}$ имеет вид:

	$$ \tilde\psi(\lambda) = \tilde\psi_k \lambda^k + \tilde\psi_{k+1} \lambda^{k+1} + \ldots + \tilde\psi_d \lambda^d. $$

И наконец полином $\polynomial{\tilde\psi_k}$, получаемый вынесением множителя $\lambda^k$:

	$$ \tilde\psi_k(\lambda) = \tilde\psi_k + \tilde\psi_{k+1} \lambda + \dots + \tilde\psi_d \lambda^{d-k} $$

уже не является аннулирущим полиномом пространства $\gfpvector{n}$. Действительно, степень полинома $\polynomial{\tilde\psi_k}$
равна $d-k < d$, и если предположить, что полином $\polynomial{\tilde\psi_k}$ является аннулирущим полиномом пространства $\gfpvector{n}$,
то это предположение противоречит тому, что полином $\polynomial{\tilde\psi}$ является минимальным аннулирущим полиномом
пространства $\gfpvector{n}$.

Поскольку полином $\polynomial{\tilde\psi_k}$ не является аннулирущим для всего пространства $\gfpvector{n}$, то найдется и ненулевой
вектор $z$ такой, что

	$$ \tilde\psi_k(A) z \neq 0. $$

Отсюда следует, что размерность образа оператора $\tilde\psi_k(\mathcal A)$ не меньше 1. Размерность всего поля векторов $\gfpvector{n}$
равна $n$, и по известной теореме о сумме размерностей ядра и образа, ядро оператора $\tilde\psi_k(\mathcal A)$ имеет размерность не более,
чем $n-1$. В пространстве $\gfpvector{n}$ всего $p^n$ различных векторов, поскольку размерность ядра $\tilde\psi_k(\mathcal A)$ не более $n-1$,
то в ядре оператора $\tilde\psi_k(\mathcal A)$ не более $p^{n-1}$ векторов.

Если вектор $z$ выбирается случайным образом (с равномерным распределением), то вектор $z$ оказывается вектором из ядра оператора
$\tilde\psi_k(\mathcal A)$ с вероятностью не более $\frac{p^{n-1}}{p^n} = \frac{1}{p}$. Вероятность выбрать $r > 1$ векторов из ядра
оказывается не больше, чем $\left ( \frac{1}{p} \right ) ^r$, а вероятность того, что среди $r$ векторов окажется хотя бы один вектор не
из ядра оператора $\tilde\psi_k(\mathcal A)$ не меньше, чем $1 - \frac{1}{p^r}$.

Таким образом, за $r$ попыток можно расчитывать на получение нетривиального решения с вероятностью не менее, чем $1 - \frac{1}{p^r}$.


\chapter{Метод Видемана с алгоритмом Берлекемпа--Месси}

\section{Постановка задачи нахождения проекционно аннулирующего полинома} \label{section:TFPAP:task_for_finding_projection_annihilating_polynomials}

Для решения неоднородных и однородных систем уравнений необходимо располагать способом построения аннулирущего полинома для заданного
вектора из пространства $\gfpvector{n}$.

Как уже отмечалось в подразделе \ref{section:KS:krylov_spaces}, один из способов построения аннулирущего полинома $\polynomial{\varphi}$:

	$$ \varphi(\lambda) =
		\varphi_0 + \varphi_1 \lambda + \varphi_2 \lambda^2 + \dots + \varphi_r \lambda^r $$

некоторого вектора $z \in \gfpvector{n}$ заключается в нахождении наименьшего количества линейно зависимых векторов Крылова 

	$$ z, Az, A^2z, \dots, A^rz $$

и определении коэффициентов $\varphi_i$ линейной комбинации этих векторов, дающих нулевой вектор:

	$$ \varphi_0 z + \varphi_1 Az + \varphi_2 A^2 z + \dots + \varphi_r A^r z = 0, $$
	\begin{equation} \label{equation:TFPAP:vector_annihilating_polynomial_condition}
		\varphi(A) z = 0.
	\end{equation}

К сожалению, задача нахождения коэффициентов $\varphi_i$ приводит к задаче нахождения решения однородной системы уравнений, которую
призвана решать, поэтому задачу поиска коэффициентов $\varphi_i$ приходится заменять другой задачей, более простой в решении.
Упрощение исходной задачи приводит к снижению вероятности успеха: не всякое решение упрощенной задачи оказывается решением исходной
задачи, тем не менее, ничего другого по-видимому не остается.

Упрощенная задача, о которой пойдет речь, и метод её решения рассматриваются работе Видемана \cite{Wiedemann}.

В упомянутой упрощенной задаче условие для аннулирующего полинома $\polynomial{\varphi}$
\eqref{equation:TFPAP:vector_annihilating_polynomial_condition}, приводящих линейную комбинацию векторов Крылова $z$, \dots, $A^rz$
к нулевому вектору заменено более простым условием. Возьмем произвольным образом вектор $y \in \gfpvector{n}$ и построим подпространство
Крылова для вектора $y$ и сопряженного оператора:

	$$ \mathcal K_n(y,A^T) = \mathcal L \left \{ y, A^T y, \left ( A^T \right ) ^2 y, \dots, \left ( A^T \right )^{n-1} y \right \}. $$

Далее будем рассматривать задачу в смысле проекционных методов: будем искать такой вектор $w$ подпространства
Крылова $\mathcal K_{r+1}(z, A)$, который является ортогональным подпространству Крылова $\mathcal K_n(y, A^T)$:

	\begin{equation} \label{equation:TFPAP:w_orthogonality_to_conjugated_space_condition}
		w \perp \mathcal K_n(y, A^T),
	\end{equation}
	$$ w \in \mathcal K_{r+1}(z, A). $$

Поскольку вектор $w$ принадлежит подпространству Крылова $\mathcal K_{r+1}(z, A)$, а подпространство $\mathcal K_{r+1}(z, A)$ является
линейной оболочкой векторов Крылова $z$, $Az$, $A^2z$, \dots, $A^rz$, то существуют коэффициенты $\rho_0$, $\rho_1$, $\rho_2$, \dots,
$\rho_r$ такие, что вектор $w$ является линейной комбинацией векторов Крылова $z$, $Az$, $A^2z$, \dots, $A^rz$
с коэффициентами $\rho_0$, $\rho_1$, $\rho_2$, \dots, $\rho_r$:

	$$ w = \rho_0 z + \rho_1 Az + \rho_2 A^2 z + \dots + \rho_r A^r z. $$

Откуда, вынося $z$ получим, равенство:

	$$ w = ( \rho_0 + \rho_1 A + \rho_2 A^2 + \dots + \rho_r A^r ) z, $$

и если представить себе некоторый полином $\polynomial{\rho}$:

	$$ \rho(\lambda) = \rho_0 + \rho_1 \lambda + \rho_2 \lambda^2 + \dots + \rho_r \lambda^r, $$

то

	$$ w = \rho(A)z, $$
	
и из условия \eqref{equation:TFPAP:w_orthogonality_to_conjugated_space_condition} следует:

	\begin{equation} \label{equation:TFPAP:polynomial_orthogonality_to_conjugated_space_condition}
		\rho(A)z \perp \mathcal K_n(y, A^T).
	\end{equation}

Для удобства последущего изложения введем для полиномов $\polynomial{\rho}$, удовлетворяющих условию 
\eqref{equation:TFPAP:polynomial_orthogonality_to_conjugated_space_condition} специальный термин.

\begin{definition}
	Полином $\polynomial{\rho} \in \gfppolynomial$ называется \definedterm{проекционно аннулирущим полиномом вектора $x$ и
	пространства $\mathcal F$ относительно оператора $\mathcal A$}, если:

	$$ \rho(A) x \perp \mathcal F. $$
\end{definition}

Как и ранее для краткости слова "относительно оператора $\mathcal A$"{} будем опускать.
 
Слово "процекционно"{} в приведенном определении указывает на то, что полином $\polynomial{\rho}$ зануляет не вектор $\rho(A)x$, а
его проекцию на пространство $\mathcal F$.

\begin{definition}
	Полином $\polynomial{\tilde\rho} \in \gfppolynomial$ называется \definedterm{минимальным проекционно аннулирующим полином
	вектора $x \in \gfpvector{n}$ и пространства $\mathcal F$ отностительно оператора $\mathcal A$}, если его степень не больше степени
	любого другого проекционно аннулирующего полинома $\polynomial{\rho}$ вектора $x \in \gfpvector{n}$ и пространства $\mathcal F$
	отностительно оператора $\mathcal A$:
		$$ \polynomialdegree{\tilde\rho} \le \polynomialdegree{\rho}. $$
\end{definition}

Таким образом, задача нахождения аннулирующего полинома $\polynomial{\varphi}$ вектора $z$ заменяется более простой задачей
нахождения проекционно аннулирующего полинома $\polynomial{\rho}$ вектора $z$ и подпространства Крылова $K_n(y, A^T)$
\eqref{equation:TFPAP:polynomial_orthogonality_to_conjugated_space_condition}.

Поскольку вектор $\rho(A) x$ ортогонален пространству $\mathcal K_n(y, A^T)$, то вектор $\rho(A) x$ должен быть ортогонален всем
векторам Крылова $y$, $A^T y$, $\left ( A^T \right ) ^2 y$, \dots, $\left ( A^T \right )^{n-1} y$. Верно и обратное, если вектор
$\rho(A) x$ ортогонален всем векторам Крылова $y$, $A^T y$, $\left ( A^T \right ) ^2 y$, \dots, $\left ( A^T \right )^{n-1} y$,
то вектор $\rho(A) x$ ортогонален каждой линейной комбинации векторов Крылова $y$, $A^T y$, $\left ( A^T \right ) ^2 y$, \dots,
$\left ( A^T \right )^{n-1} y$ и следовательно ортогонален всему подпространству Крылова $\mathcal K_n(y, A^T)$, представляющему собой
линейную оболочку векторов Крылова $y$, $A^T y$, $\left ( A^T \right ) ^2 y$, \dots, $\left ( A^T \right )^{n-1} y$. Таким образом,
условие \eqref{equation:TFPAP:polynomial_orthogonality_to_conjugated_space_condition} эквивалентно ортогональности вектора $\rho(A) x$ 
каждому из векторов Крылова $y$, $A^T y$, $\left ( A^T \right ) ^2 y$, \dots, $\left ( A^T \right )^{n-1} y$:

	$$
		\left \{
			\begin{array}{ccc}
				\rho(A) z & \perp & y \\
				\rho(A) z & \perp & A^T y \\
				\rho(A) z & \perp & \left ( A^T \right )^2 y \\
				\vdots \\
				\rho(A) z & \perp & \left ( A^T \right )^{n-1} y \\
			\end{array}
		\right .
		.
	$$

Условия ортогональности эквивалентны равенствам нулю скалярных произведения вектора $\rho(A)z$ и каждого из векторов $y$, $A^T y$,
$\left ( A^T \right ) ^2 y$, \dots, $\left ( A^T \right )^{n-1} y$:

	$$
		\left \{
			\begin{array}{ccc}
				( \rho(A) z , & y                        & ) = 0\\
				( \rho(A) z , & A^T y                    & ) = 0\\
				( \rho(A) z , & \left ( A^T \right )^2 y & ) = 0\\
				\vdots \\
				( \rho(A) z , & \left ( A^T \right )^{n-1} y & ) = 0\\
			\end{array}
		\right .
		,
	$$

или в развернутом виде:

	$$
		\left \{
			\begin{array}{ccc}
				( \rho_0 z + \rho_1 A z + \rho_2 A^2 z + \dots + \rho_r A^r z , & y & ) = 0 \\
				( \rho_0 z + \rho_1 A z + \rho_2 A^2 z + \dots + \rho_r A^r z , & A^T y & ) = 0 \\
				( \rho_0 z + \rho_1 A z + \rho_2 A^2 z + \dots + \rho_r A^r z , & \left ( A^T \right )^2 y & ) = 0 \\
				\vdots \\
				( \rho_0 z + \rho_1 A z + \rho_2 A^2 z + \dots + \rho_r A^r z , & \left ( A^T \right )^{n-1} y & ) = 0 \\
			\end{array}
		\right .
		.
	$$

Преобразуем полученную систему к следующему виду:

	$$
		\left \{
			\begin{array}{ccc}
				y^T         & \left ( \rho_0 z + \rho_1 A z + \rho_2 A^2 z + \dots + \rho_r A^r z \right ) & = 0 \\
				y^T A       & \left ( \rho_0 z + \rho_1 A z + \rho_2 A^2 z + \dots + \rho_r A^r z \right ) & = 0 \\
				y^T A^2     & \left ( \rho_0 z + \rho_1 A z + \rho_2 A^2 z + \dots + \rho_r A^r z \right ) & = 0 \\
				\vdots \\
				y^T A^{n-1} & \left ( \rho_0 z + \rho_1 A z + \rho_2 A^2 z + \dots + \rho_r A^r z \right ) & = 0 \\
			\end{array}
		\right .
		.
	$$

Раскрывая скобки и выполняя умножение, получим систему:

	$$
		\left \{
			\begin{array}{ccccccccccccc}
				y^T z         & \rho_0 & + & y^T A z         & \rho_1 & + & y^T A^2 z       & \rho_2 & + & \dots & + & y^T A^r z & \rho_r = 0 \\
				y^T A z       & \rho_0 & + & y^T A^2 z       & \rho_1 & + & y^T A^3 z       & \rho_2 & + & \dots & + & y^T A^{1+r} z & \rho_r = 0 \\
				y^T A^2 z     & \rho_0 & + & y^T A^3 z       & \rho_1 & + & y^T A^4 z       & \rho_2 & + & \dots & + & y^T A^{2+r} z & \rho_r = 0 \\
				\vdots \\
				y^T A^{n-1} z & \rho_0 & + & y^T A^{n-1+1} z & \rho_1 & + & y^T A^{n-1+2} z & \rho_2 & + & \dots & + & y^T A^{n-1+r} z & \rho_r = 0 \\
			\end{array}
		\right .
	$$

Таким образом, вектор коэффициентов $( \rho_0, \rho_1, \rho_2, \dots, \rho_r )$ удовлетворяет однородной системе:

	$$
		\begin{pmatrix}
			y^T z         & y^T A z         & y^T A^2 z       & \ldots & y^T A^n z \\
			y^T A z       & y^T A^2 z       & y^T A^3 z       & \ldots & y^T A^{n+1} z \\
			y^T A^2 z     & y^T A^3 z       & y^T A^4 z       & \ldots & y^T A^{n+2} z \\
			\vdots        & \vdots          & \vdots          & \ddots & \vdots \\
			y^T A^{n-1} z & y^T A^{n-1+1} z & y^T A^{n-1+2} z & \ldots & y^T A^{n-1+r} z \\
		\end{pmatrix}
		\begin{pmatrix}
			\rho_0 \\
			\rho_1 \\
			\rho_2 \\
			\vdots \\
			\rho_r \\
		\end{pmatrix}
		=
		\begin{pmatrix}
			0 \\
			0 \\
			0 \\
			\vdots \\
			0 \\
		\end{pmatrix}
		.
	$$

Обозначим

	$$ \alpha_{i+j} = y^T A^{i+j} z, $$
	$$ 0 \le i+j \le n-1+r, $$

тогда

	$$
		\begin{pmatrix}
			\alpha_0     & \alpha_1 & \alpha_2     & \ldots & \alpha_r \\
			\alpha_1     & \alpha_2 & \alpha_3     & \ldots & \alpha_{1+r} \\
			\alpha_2     & \alpha_3 & \alpha_4     & \ldots & \alpha_{2+r} \\
			\vdots       & \vdots   & \vdots       & \ddots & \vdots \\
			\alpha_{n-1} & \alpha_n & \alpha_{n+1} & \ldots & \alpha_{n-1+r} \\
		\end{pmatrix}
		\begin{pmatrix}
			\rho_0 \\
			\rho_1 \\
			\rho_2 \\
			\vdots \\
			\rho_r \\
		\end{pmatrix}
		=
		\begin{pmatrix}
			0 \\
			0 \\
			0 \\
			\vdots \\
			0 \\
		\end{pmatrix}
		.
	$$

Легко видеть, что матрица системы имеет ганкелеву структуру. Если же переставить строки в обратном порядке, то получится система

	$$
		\begin{pmatrix}
			\alpha_{n-1} & \alpha_n & \alpha_{n+1} & \ldots & \alpha_{n-1+r} \\
			\vdots       & \vdots   & \vdots       & \ddots & \vdots \\
			\alpha_2     & \alpha_3 & \alpha_4     & \ldots & \alpha_{n+1} \\
			\alpha_1     & \alpha_2 & \alpha_3     & \ldots & \alpha_n \\
			\alpha_0     & \alpha_1 & \alpha_2     & \ldots & \alpha_{n-1} \\
		\end{pmatrix}
		\begin{pmatrix}
			\rho_0 \\
			\rho_1 \\
			\rho_2 \\
			\vdots \\
			\rho_r \\
		\end{pmatrix}
		=
		\begin{pmatrix}
			0 \\
			0 \\
			0 \\
			\vdots \\
			0 \\
		\end{pmatrix}
	$$

матрица, которой является теплицевой.

Для систем с ганкелевыми и теплицевыми матрицами известны эффективные методы нахождения решений, применение которых приводит к вектору
$( \rho_0, \rho_1, \rho_2, \ldots, \rho_r )$ коэффициентов проекционно аннулирующего полинома $\polynomial{\rho}$.

В своей работе \cite{Wiedemann} Видеман предлагает использовать алгоритм Берлекемпа--Месси, обсуждаемый в разделе
\ref{section:FPAP:finding_projection_annihilating_polynomials}, для решения системы с ганкелевой матрицей.

\section{Решение задачи нахождения проекционно аннулирущего полинома} \label{section:FPAP:finding_projection_annihilating_polynomials}

В предыдущем разделе было показано, что коэффициенты $\rho_i$ проекционно аннулирующего полинома $\polynomial{\rho}$ удовлетворяют системе
уравнений:

	\begin{equation} \label{equation:FPAP:equations_for_rho}
		\begin{pmatrix}
			\alpha_0     & \alpha_1 & \alpha_2     & \ldots & \alpha_r \\
			\alpha_1     & \alpha_2 & \alpha_3     & \ldots & \alpha_{1+r} \\
			\alpha_2     & \alpha_3 & \alpha_4     & \ldots & \alpha_{2+r} \\
			\vdots       & \vdots   & \vdots       & \ddots & \vdots \\
			\alpha_{n-1} & \alpha_n & \alpha_{n+1} & \ldots & \alpha_{n-1+r} \\
		\end{pmatrix}
		\begin{pmatrix}
			\rho_0 \\
			\rho_1 \\
			\rho_2 \\
			\vdots \\
			\rho_r \\
		\end{pmatrix}
		=
		\begin{pmatrix}
			0 \\
			0 \\
			0 \\
			\vdots \\
			0 \\
		\end{pmatrix}
		.
	\end{equation}

Дополнительно будем предполагать, что число $r$ выбрано таким образом, что $\rho_r \neq 0$. Сделанное предположение оправдывается тем, что
если вектор $(\rho_0, \rho_1, \rho_2, \dots, \rho_r)$ является нетривиальным решением системы \eqref{equation:FPAP:equations_for_rho}, то
не ограничивая общности, можно считать, что коэффициент $\rho_r \neq 0$, поскольку в случае $\rho_r = 0$ всегда можно для $r$ выбрать
меньшее значение, при котором $\rho_r \neq 0$, отбросив при этом несколько последних столбцов матрицы системы
\eqref{equation:FPAP:equations_for_rho} и несколько последних коэффициентов $\rho_i$. Далее, поскольку $\rho_r \neq 0$, то в поле $\gfp$
существует обратный по умножению элемент $\rho_r^{-1}$, и тогда если вектор $(\rho_0, \rho_1, \rho_2, \dots, \rho_{r-1}, \rho_r)$ является
решением однородной системы \eqref{equation:FPAP:equations_for_rho}, то вектор:

	$$
		\begin{pmatrix}
			\rho_r^{-1} \rho_0 \\
			\rho_r^{-1} \rho_1 \\
			\rho_r^{-1} \rho_2 \\
			\vdots \\
			\rho_r^{-1} \rho_{r-1} \\
			\rho_r^{-1} \rho_{r} \\
		\end{pmatrix}
		=
		\begin{pmatrix}
			\rho_r^{-1} \rho_0 \\
			\rho_r^{-1} \rho_1 \\
			\rho_r^{-1} \rho_2 \\
			\vdots \\
			\rho_r^{-1} \rho_{r-1} \\
			1 \\
		\end{pmatrix}
	$$

также удовлетворяет системе \eqref{equation:FPAP:equations_for_rho}.

Таким образом, если система \eqref{equation:FPAP:equations_for_rho}
имеет нетривиальное решение $(\rho_0, \rho_1,$ $\rho_2, \dots, \rho_{r-1}, \rho_r)$, то существует нетривиальное решение, в котором
$\rho_r = 1$, и именно такие решения будем искать в дальнейшем.

Выбор именно последнего коэффициента $\rho_r$ в качестве ненулевого не является случайным: ведь чем больше $r$, тем больше коэффициентов
$\alpha_i$ потребуется вычислить, а вычисление каждого коэффициента $\alpha_i$ требует определенных вычислительных затрат.
Если представить ситуацию, в которой несколько последних коэффициентов $\rho_i$ являются нулевыми, включая и $\rho_r$, то окажется, что
столько же последних коэффициентов $\alpha_i$ вычислены напрасно, поскольку не используются. Если же в полученном решении $\rho_r \neq 0$,
то можно быть абсолютно уверенным в том, что коэффициент $\alpha_{n-1+r}$ используются и должен быть вычислен.

Введем полином $\polynomial{\xi}$, коэффициенты $\xi_i$ которого являются коэффициентами $\rho_i$ полинома $\polynomial{\rho}$ взятыми
в обратном порядке:

	$$ \xi(\lambda) = \lambda^r \rho \left ( \frac{1}{\lambda} \right ) = $$
	$$ = \lambda^r \left ( \rho_0 + \rho_1 \frac{1}{\lambda} + \rho_2 \frac{1}{\lambda^2} + \dots + \rho_r \frac{1}{\lambda^r} \right ) = $$
	$$ = \rho_0 \lambda^r + \rho_1 \lambda^{r-1} + \rho_2 \lambda^{r-2} + \dots + \rho_r = $$
	$$ = \xi_r \lambda^r + \xi_{r-1} \lambda^{r-1} + \xi_{r-2} \lambda^{r-2} + \dots + \xi_0. $$

Таким образом,

	\begin{equation} \label{equation:FPAP:rho_and_xi_equation}
		\begin{pmatrix}
			\rho_0 \\
			\rho_1 \\
			\rho_2 \\
			\vdots \\
			\rho_r
		\end{pmatrix}
		=
		\begin{pmatrix}
			\xi_r \\
			\xi_{r-1} \\
			\xi_{r-2} \\
			\vdots \\
			\xi_0
		\end{pmatrix}
		.
	\end{equation}

Причем согласно предположению $\rho_r = 1$, поэтому $\xi_0 = 1$ и полином $\polynomial{\xi}$ имеет вид:

	$$ \xi(\lambda) = 1 + \dots + \xi_{r-2} \lambda^{r-2} + \xi_{r-1} \lambda^{r-1} + \xi_r \lambda^r. $$

Для множества всех полиномов с коэффициентом 1 при нулевой степени введем специальное обозначение $\gfpunitypolynomial$:

	$$ \gfpunitypolynomial = \left \{ \polynomial{\gamma} \in \gfppolynomial : \degreecoefficient{0}{\polynomial{\gamma}} = 1 \right \}. $$

Таким образом,

	$$ \polynomial{\xi} \in \gfpunitypolynomial. $$

Из равенства \eqref{equation:FPAP:equations_for_rho} следует, что коэффициенты $\xi_i$ полинома $\polynomial{\xi}$ удовлетворяют
системе уравнений:

	\begin{equation} \label{equation:FPAP:equations_for_xi}
		\begin{pmatrix}
			\alpha_0     & \alpha_1 & \alpha_2     & \ldots & \alpha_r \\
			\alpha_1     & \alpha_2 & \alpha_3     & \ldots & \alpha_{1+r} \\
			\alpha_2     & \alpha_3 & \alpha_4     & \ldots & \alpha_{2+r} \\
			\vdots       & \vdots   & \vdots       & \ddots & \vdots \\
			\alpha_{n-1} & \alpha_n & \alpha_{n+1} & \ldots & \alpha_{n-1+r} \\
		\end{pmatrix}
		\begin{pmatrix}
			\xi_r \\
			\xi_{r-1} \\
			\xi_{r-2} \\
			\vdots \\
			\xi_0 \\
		\end{pmatrix}
		=
		\begin{pmatrix}
			0 \\
			0 \\
			0 \\
			\vdots \\
			0 \\
		\end{pmatrix}
		.
	\end{equation}

Системе \eqref{equation:FPAP:equations_for_xi} можно придать следующую интерпретацию: представим, что числа $\alpha_0$, \dots, $\alpha_{n-1+r}$
являются коэффициентами полинома $\polynomial{\alpha}$:

	$$ \alpha(\lambda) = \alpha_0 + \alpha_1 \lambda + \alpha_2 \lambda^2 + \dots + \alpha_{n-1+r} \lambda^{n-1+r}, $$

тогда первая строка в системе \eqref{equation:FPAP:equations_for_xi} означает, что коэффициент при степени $r$ в произведении полиномов
$\polynomial{\alpha} \polynomial{\xi}$ равен нулю:

	$$ \degreecoefficient{r}{\polynomial{\alpha} \polynomial{\xi}}
		= \alpha_0 \xi_r + \alpha_1 \xi_{r-1} + \alpha_2 \xi_{r-2} + \dots + \alpha_r \xi_0 = 0 $$

вторая строка в системе \eqref{equation:FPAP:equations_for_xi} означает, что коэффициент при степени $r+1$ в произведении полиномов
$\polynomial{\alpha} \polynomial{\xi}$ равен нулю:

	$$ \degreecoefficient{r+1}{\polynomial{\alpha} \polynomial{\xi}}
		= \alpha_1 \xi_r + \alpha_2 \xi_{r-1} + \alpha_3 \xi_{r-2} + \dots + \alpha_{r+1} \xi_0 = 0 $$

а все последующие строки означают равенство нулю всех коэффициентов при степенях до степени $n-1+r$ включительно в произведении полиномов
$\polynomial{\alpha} \polynomial{\xi}$.

Таким образом, система \eqref{equation:FPAP:equations_for_xi} эквивалентна системе равенств:

	\begin{equation} \label{equation:FPAP:equations_for_product_coefficients}
		\left \{
			\begin{array}{c}
				\degreecoefficient{r}{\polynomial{\alpha} \polynomial{\xi}} = 0 \\
				\degreecoefficient{r+1}{\polynomial{\alpha} \polynomial{\xi}} = 0 \\
				\vdots \\
				\degreecoefficient{n-1+r}{\polynomial{\alpha} \polynomial{\xi}} = 0 \\
			\end{array}
		\right .
	\end{equation}

Систему равенств \eqref{equation:FPAP:equations_for_product_coefficients} можно коротко записать в виде равенства:

	\begin{equation} \label{equation:FPAP:alpha_xi_product}
		\begin{array}{c}
			\alpha(\lambda) \xi(\lambda) = \delta(\lambda) + \varepsilon(\lambda) \lambda^{n+r}, \\
			\polynomialdegree{\delta} \le r-1, \\
		\end{array}
	\end{equation}

в котором $\polynomial{\delta}$ --- некоторый полином степени не выше $r-1$ и $\polynomial{\varepsilon}$ --- некоторый полином
произвольной степени. Иногда полученное равенство записывают в виде:

	$$ \alpha(\lambda) \xi(\lambda) = \delta(\lambda) \; mod \; \lambda^{n+r}. $$

Таким образом, исходная задача нахождения решения системы \eqref{equation:FPAP:equations_for_rho} сводится к задаче нахождения полинома
$\polynomial{\xi}$ такого, что в произведении $\polynomial{\alpha} \polynomial{\xi}$ отсутствуют степени $r$, $r+1$, \dots, $n-1+r$.

Для наглядности последующего изложения, прежде всего, введем графические схемы вычисления коэффициентов $\delta_i$ произведения полиномов
из равенства \eqref{equation:FPAP:alpha_xi_product}: пусть имеется три полосы --- верхняя, средняя и нижняя, на верхней полосе записаны
коэффициенты $\alpha_i$ полинома $\polynomial{\alpha}$, на средней --- коэффициенты $\xi_i$ полинома $\polynomial{\xi}$, а на нижней
будут записываться коэффициенты $\delta_i$ полинома $\polynomial{\delta}$. Сверху для удобства отсчета раставлены степени $\lambda^k$.

	$$
		\begin{array}{ccccccccccc}
			                      &       &       &       &       &       & \lambda^0 & \lambda^1 & \lambda^2 & \dots & \lambda^{n-1+r} \\
			\hline
			\text{верхняя полоса} &       &       &       &       &       & \alpha_0  & \alpha_1  & \alpha_2  & \dots & \alpha_{n-1+r} \\
			                      &       &       &       &       &       &           &           &           &       & \\
			\text{средняя полоса} & \xi_r & \dots & \xi_2 & \xi_1 & \xi_0 &           &           &           &       & \\
			                      &       &       &       &       &       &           &           &           &       & \\
			\text{нижняя полоса}  &       &       &       &       &       &           &           &           &       & \\
		\end{array}
	$$

Теперь представим как средняя полоса смещается вправо на одну позицию: коэффициенты $\alpha_0$ и $\xi_0$ оказывают напротив друг друга,
перемножаются и результат (коэффициент $\delta_0$) записывается на нижнюю полосу:

	$$
		\begin{array}{ccccccccccc}
			                      &       &       &       &       &       & \lambda^0  & \lambda^1 & \lambda^2 & \dots & \lambda^{n-1+r} \\
			\hline
			\text{верхняя полоса} &       &       &       &       &       & \alpha_0   & \alpha_1  & \alpha_2  & \dots & \alpha_{n-1+r} \\
			                      &       &       &       &       &       & |          &           &           &       & \\
			\text{средняя полоса} &       & \xi_r & \dots & \xi_2 & \xi_1 & \xi_0      &           &           &       & \\
			                      &       &       &       &       &       & \downarrow &           &           &       & \\
			\text{нижняя полоса}  &       &       &       &       &       & \delta_0   &           &           &       & \\
		\end{array}
	$$

Далее средняя полоса смещается еще на одну позицию вправо: коэффициенты $\alpha_0$ и $\xi_1$, а также $\alpha_1$ и $\xi_0$
оказываются друг напротив друга, перемножаются, произведения складываются и результат (коэффициент $\delta_1$) записывается на нижнюю полосу:

	$$
		\begin{array}{ccccccccccc}
			                      &       &       &       &       &       & \lambda^0 & \lambda^1  & \lambda^2 & \dots & \lambda^{n-1+r} \\
			\hline
			\text{верхняя полоса} &       &       &       &       &       & \alpha_0  & \alpha_1   & \alpha_2  & \dots & \alpha_{n-1+r} \\
			                      &       &       &       &       &       & |         & |          &           &       & \\
			\text{средняя полоса} &       &       & \xi_r & \dots & \xi_2 & \xi_1     & \xi_0      &           &       & \\
			                      &       &       &       &       &       &           & \downarrow &           &       & \\
			\text{нижняя полоса}  &       &       &       &       &       & \delta_0  & \delta_1   &           &       & \\
		\end{array}
	$$

Аналогичным образом, для коэффициента $\delta_2$ получим схему:

	$$
		\begin{array}{ccccccccccc}
			                      &       &       &       &       &       & \lambda^0 & \lambda^1  & \lambda^2  & \dots & \lambda^{n-1+r} \\
			\hline
			\text{верхняя полоса} &       &       &       &       &       & \alpha_0  & \alpha_1   & \alpha_2   & \dots & \alpha_{n-1+r} \\
			                      &       &       &       &       &       & |         & |          & |          &       & \\
			\text{средняя полоса} &       &       &       & \xi_r & \dots & \xi_2     & \xi_1      & \xi_0      &       & \\
			                      &       &       &       &       &       &           &            & \downarrow &       & \\
			\text{нижняя полоса}  &       &       &       &       &       & \delta_0  & \delta_1   & \delta_2   &       & \\
		\end{array}
	$$

Далее средняя полоса продолжает движение вправо до тех пор пока коэффициент $\xi_0$ не окажется напротив коэффициента $\alpha_{n-1+r}$.
В общем случае схема вычисления коэффициента $\delta_k$ имеет вид:

	$$
		\begin{array}{ccccccccccc}
			                      & \lambda^0 & \lambda^1 & \dots & \lambda^{k-r} & \dots & \lambda^{k-2} & \lambda^{k-1} & \lambda^k  & \dots & \lambda^{n-1+r} \\
			\hline
			\text{верхняя полоса} & \alpha_0  & \alpha_1  & \dots & \alpha_{k-r}  & \dots & \alpha_{k-2}  & \alpha_{k-1}  & \alpha_k   & \dots & \alpha_{n-1+r} \\
			                      &           &           &       & |             &       & |             & |             & |          &       & \\
			\text{средняя полоса} &           &           &       & \xi_r         & \dots & \xi_2         & \xi_1         & \xi_0      &       & \\
			                      &           &           &       &               &       &               &               & \downarrow &       & \\
			\text{нижняя полоса}  & \delta_0  & \delta_1  &       & \delta_{k-r}  & \dots & \delta_{k-2}  & \delta_{k-1}  & \delta_k   &       & \\
		\end{array}
	$$

От полученных схем вычисления коэффициентов $\delta_i$ произведения $\polynomial{\alpha} \polynomial{\xi}$ легко перейти к наглядному графическому
представлению самого произведения $\polynomial{\alpha} \polynomial{\xi}$, которое используется в работе Томе \cite{Thome}. В этом представлении
вместо $\lambda^k$ будем записывать только степени $k$, верхнюю полосу не будем записывать вовсе, а вместо средней и нижней полосы будем писать
полином средней полосы $\polynomial{\xi}$ и напротив коэффициенты произведения --- $\delta_i$:

	$$
		\begin{array}{cccccccc}
			                 & 0        & 1        & \dots & r-1          & r        & \dots & n-1+r \\
			\hline
			\polynomial{\xi} & \delta_0 & \delta_1 & \dots & \delta_{r-1} & \delta_r & \dots & \delta_{n-1+r} \\
		\end{array}
	$$

Далее еще более упростим полученное представление введя специальные обозначения для коэффициентов $\delta_i$:

\begin{itemize}
	\item [0] --- означает, что $\delta_i = 0$;
	\item [$\diamond$] --- означает, что $\delta_i \neq 0$;
	\item [$\bullet$] --- означает произвольное значение $\delta_i$, которое нас не интересует.
\end{itemize}

С учетом принятых обозначенений равенство \eqref{equation:FPAP:alpha_xi_product} означает, что необходимо найти такой полином
$\polynomial{\xi}$, который имеет схему произведения:

	$$
		\begin{array}{ccccccccc}
			                 & 0       & 1       & \dots & r-1     & r & r+1 & \dots & n-1+r \\
			\hline
			\polynomial{\xi} & \bullet & \bullet & \dots & \bullet & 0 & 0   & \dots & 0 \\
		\end{array}
	$$

Представленная далее процедура построения полинома $\polynomial{\xi}$ является итерационной и основана на известном алгоритме Берлекемпа--Месси,
излагаемого в работе Месси \cite{Massey}.

В процессе выполнения итераций процедура старается находить такие полиномы $\polynomial{\xi^{(k)}} \in \gfpunitypolynomial$,
что в произведениии $\polynomial{\alpha} \polynomial{\xi^{(k)}}$ отсутствуют степени, начиная от степени самого полинома $\polynomial{\xi^{(k)}}$
и до степени $k$ включительно. Цель процедуры, таким образом, заключается в построении такого полинома $\polynomial{\xi^{(q)}}$, который
зануляет в произведении $\polynomial{\alpha} \polynomial{\xi^{(q)}}$ не меньше $n$ степеней, начиная со степени полинома $\polynomial{\xi^{(q)}}$,
и именно такой полином $\polynomial{\xi^{(q)}}$ является одним из возможных полиномов $\polynomial{\xi}$, коэффициенты которого удовлетворяют
системе \eqref{equation:FPAP:equations_for_xi}.

Принадлежность полиномов $\polynomial{\xi^{(k)}}$ множеству $\gfpunitypolynomial$ является существенным ограничением, поскольку оставаясь
в рамках множества $\gfpunitypolynomial$, всегда можно быть уверенным, что построенный полином $\polynomial{\xi^{(k)}}$ не является нулевым
(поскольку по крайней мере коэффициент при нулевой степени у полинома $\polynomial{\xi^{(k)}}$ равен 1), а это значит, что получаемые в результате
обратного преобразования \eqref{equation:FPAP:rho_and_xi_equation} векторы $(\rho_0, \dots, \rho_r)$ имеют ненулевой элемент $\rho_r$ и
следовательно являются нетривиальными решениями исходной системы \eqref{equation:FPAP:equations_for_rho}.

Поскольку полиномы $\polynomial{\xi^{(k)}}$ будут часто встречаться в дальнейшем, введем для них специальное определение.

\begin{definition}
	Полином $\polynomial{\xi}$ называется \definedterm{полиномом, аннулирущим до степени $k$}, если в произведении
	$\polynomial{\alpha} \polynomial{\xi}$ равны нулю коэффициенты при степенях, начиная от степени самого полинома $\polynomial{\xi}$
	и заканчивая степенью $k$ включительно:

		$$ \degreecoefficient{j}{\polynomial{\alpha} \polynomial{\xi}} = 0, $$
		$$ j = \polynomialdegree{\xi}, \dots, k. $$
\end{definition}

Коэффициенты в произведении $\polynomial{\alpha} \polynomial{\xi}$ при степенях меньших степени полинома $\polynomial{\xi}$ не
представляют интереса, поскольку в соответствии уравнениями системы \eqref{equation:FPAP:equations_for_xi} на коэффициенты в произведении
$\polynomial{\alpha} \polynomial{\xi}$ при степенях меньших $r$ никаких условий не налагается, и они могут быть произвольными, поэтому и в
произведении $\polynomial{\alpha} \polynomial{\xi}$ коэффициенты до степени полинома $\polynomial{\xi}$ (не включительно) также могут
быть произвольными.

Заметим, что степень полинома $\polynomial{\xi}$ может оказаться больше $k$ и первые $k$ коэффициентов при степенях с 0 по $k-1$ в произведении
$\polynomial{\alpha} \polynomial{\xi}$ могут быть и не равны нулю, но поскольку они не представляют интереса, то такой полином $\polynomial{\xi}$
будем все таки считать аннулирующим до степени $k-1$ и примем следующее соглашение.

\begin{agreement} \label{agreement:FPAP:ignoring_nonzero_product_coefficients}
	Любой полином $\polynomial{\xi}$ степени $d$ считается аннулирущим до степени $d-1$.
\end{agreement}

В излагаемой далее процедуре построения аннулирущих до различных степеней $k$ полиномов $\polynomial{\xi^{(k)}}$ используются два вспомогательных
полинома: $\polynomial{\xi_p}$, который будем называть предыдущим (индекс $p$ соответствует слову "previous"{}), и $\polynomial{\xi_c}$, который
будем называть текущим (индекс $c$ соответствует слову "current"{}). Степени полиномов $\polynomial{\xi_p}$ и $\polynomial{\xi_c}$ будем обозначать
$d_p$ и $d_c$ соответсвенно.

Первые шаги процедуры связаны с инициализацией этих двух полиномов: шаг, связанный с инициализацией предыдущего полинома $\polynomial{\xi_p}$,
будем обозначать Н.1 (где литера Н обозначает "Начало"{} или "иНициализация"{}, а цифра 1 указывает, что предыдущий полином инициализируется
первым), а шаг, связанный с инициализацией текущего полинома $\polynomial{\xi_c}$, будем обозначать Н.2.

Итак, на шаге Н.1 возьмем в качестве предыдущего полинома $\polynomial{\xi_p}$ константу $1$:

	\begin{equation} \label{equation:FPAP:previous_polynomial_initialization}
		\polynomial{\xi_p} = 1,
	\end{equation}
	$$ d_p = 0, $$

и будем вычислять коэффициенты в произведении $\polynomial{\alpha} \polynomial{\xi_p}$ до тех пор, пока не встретиться ненулевой коэффициент.
В общем случае несколько первых коэффициентов в произведении $\polynomial{\alpha} \polynomial{\xi_p}$ могут оказаться равными нулю,
пусть $m$ обозначает самую младшую степень при которой коэффициент не равен нулю. В этом случае схема произведения
$\polynomial{\alpha} \polynomial{\xi_p}$ (например, для $m=2$) имеет вид:

	$$
		\begin{array}{cccccccc}
			                   & d_p &   & m        &         &         &         & \\
			                   & 0   & 1 & 2        & 3       & 4       & 5       & \dots \\
			\hline
			\polynomial{\xi_p} & 0   & 0 & \diamond & \bullet & \bullet & \bullet & \dots \\
		\end{array}
	$$

Если $m>0$, то полином $\polynomial{\xi^{(0)}}$, аннулирующий до степени 0, совпадает с предыдущим полиномом $\polynomial{\xi_p}$:

	$$ \polynomial{\xi^{(0)}} = \polynomial{\xi_p}. $$

То же самое касается и всех полиномов $\polynomial{\xi^{(k)}}$ для $k \le m-1$:

	$$ \polynomial{\xi^{(k)}} = \polynomial{\xi_p}, $$
	$$ k \le m-1. $$

Заметим, что, как и требовалось, все построенные полиномы $\polynomial{\xi^{(k)}}$ принадлежат множеству $\gfpunitypolynomial$:

	$$ \polynomial{\xi^{(k)}} \in \gfpunitypolynomial, $$
	$$ k \le m-1, $$

поскольку $\polynomial{\xi_p} = 1$.

Если $n-1 \le m$, то процедура построения завершается и в качестве искомого полинома $\polynomial{\xi}$ следует использовать полином
$\polynomial{\xi_p}$:

	$$ \polynomial{\xi} = \polynomial{\xi_p} = 1. $$

Этим завершается шаг Н.1 и прежде чем переходить к следующему шагу Н.2 проанализируем результаты, полученные на шаге Н.1.

Итак, если $m > 1$, то удалось построить аннулирующие полиномы $\polynomial{\xi^{(0)}}$, \dots, $\polynomial{\xi^{(m-1)}}$, если же $m=0$, то ни
одного аннулирущего полинома построить не удалось.

Кроме того следует сделать важное замечание, которое касается коэффициентов $\alpha_i$: $m$ --- наименьшая степень в произведении
$\polynomial{\alpha} \polynomial{\xi_p}$ имеющая ненулевой коэффициент, следовательно, все предыдущие коэффициенты при
степенях 0, \dots, $m-1$ в произведении $\polynomial{\alpha} \polynomial{\xi_p}$ равны нулю, а коэффициент при степени $m$ нулю не равен:

	$$
		\left \{
			\begin{array}{c}
				\degreecoefficient{0}{\polynomial{\alpha} \polynomial{\xi_p}} = 0, \\
				\degreecoefficient{1}{\polynomial{\alpha} \polynomial{\xi_p}} = 0, \\
				\vdots \\
				\degreecoefficient{m-1}{\polynomial{\alpha} \polynomial{\xi_p}} = 0\\
				\degreecoefficient{m}{\polynomial{\alpha} \polynomial{\xi_p}} \neq 0\\
			\end{array}
		\right .
		.
	$$

Поскольку предыдущий полином $\polynomial{\xi_p} = 1$ в соответствии с равенством \eqref{equation:FPAP:previous_polynomial_initialization}, то

	\begin{equation} \label{equation:FPAP:first_m_alpha_coefficients}
		\left \{
			\begin{array}{c}
				\degreecoefficient{0}{\polynomial{\alpha}} = 0, \\
				\degreecoefficient{1}{\polynomial{\alpha}} = 0, \\
				\vdots \\
				\degreecoefficient{m-1}{\polynomial{\alpha}} = 0\\
				\degreecoefficient{m}{\polynomial{\alpha}} \neq 0\\
			\end{array}
		\right .
		\Leftrightarrow
		\left \{
			\begin{array}{c}
				\alpha_0 = 0, \\
				\alpha_1 = 0, \\
				\vdots \\
				\alpha_{m-1} = 0\\
				\alpha_{m} \neq 0\\
			\end{array}
		\right .
	\end{equation}

Полученные соотношения для коэффициентов $\alpha_0$, \dots, $\alpha_m$ играют важную роль на шаге Н.2 инициализации текущего полинома
$\polynomial{\xi_c}$.

По завершению шага Н.1 процедура построения остановилась и не смогла построить полином $\polynomial{\xi^{(m)}} \in \gfpunitypolynomial$,
аннулирующий до степени $m$, поэтому основной целью  шага Н.2 является построение этого полинома. Способ построения аналогичен шагу Н.1:
текущий полином $\polynomial{\xi_c}$ инициализируется (подбирается) таким образом, чтобы он являлся аннулирущим до степени $m$ и имел
коэффициент при нулевой степени равный 1, и затем используется в качестве полинома $\polynomial{\xi^{(m)}}$.

При подборе текущего полинома $\polynomial{\xi_c}$ существенную роль играют равенства \eqref{equation:FPAP:first_m_alpha_coefficients}:
из-за того, что среди коэффициентов $\alpha_0$, \dots, $\alpha_m$ лишь один коэффициент $\alpha_m$ не равен нулю, его нечем "занулить"{}
при вычислении коэффициента при степени $m$ в произведении $\polynomial{\alpha} \polynomial{\xi_c}$, поэтому приходится искусственным образом
повышать степень текущего полинома $\polynomial{\xi_c}$ до $m+1$, чтобы текущий полином $\polynomial{\xi_c}$ стал аннулирующим до степени $m$
по соглашению \ref{agreement:FPAP:ignoring_nonzero_product_coefficients}.

Формальное доказательство того факта, что аннулирующий до степени $m$ текущий полином $\polynomial{\xi_c} \in \gfpunitypolynomial$ должен
обязательно иметь степень больше $m$ можно провести от противного: пусть текущий полином $\polynomial{\xi_c}$, аннулирущий до степени $m$
и имеющий коэффициент при нулевой степени равный 1, имеет степень меньшую или равную $m$. Представим текущий полином $\polynomial{\xi_c}$
в виде:

	$$ \xi_c(\lambda) = 1 + \xi_{c,1} \lambda + \xi_{c,2} \lambda^2 + \dots \xi_{c,m} \lambda^m, $$

где некоторые коэффициенты из $\xi_{c,1}$, $\xi_{c,2}$, \dots, $\xi_{c,m}$ могут быть нулевыми. Попробуем вычислить коэффициент при степени
$m$ в произведении $\polynomial{\alpha} \polynomial{\xi_c}$:

	$$
		\begin{array}{c}
			\degreecoefficient{m}{\polynomial{\alpha} \polynomial{\xi_c}} = \\
			= \alpha_m \cdot 1 + \alpha_{m-1} \cdot \xi_{c,1} + \alpha_{m-2} \cdot \xi_{c,2} + \dots + \alpha_0 \cdot \xi_{c,m} = \\
			= \alpha_m \cdot 1 + 0 \cdot \xi_{c,1} + 0 \cdot \xi_{c,2} + \dots + 0 \cdot \xi_{c,m} = \\
			= \alpha_m \\
		\end{array}
	$$

поскольку в силу равенств \eqref{equation:FPAP:first_m_alpha_coefficients}:

	$$ \alpha_0 = \alpha_1 = \dots = \alpha_{m-1} = 0. $$

Таким образом, оказывается, что коэффициент при степени $m$ не равен нулю, поскольку:

	$$ \degreecoefficient{m}{\polynomial{\alpha} \polynomial{\xi^{(m)}}} = \alpha_m \neq 0 $$

опять же в силу равенств \eqref{equation:FPAP:first_m_alpha_coefficients}.

Получилось противоречие: аннулирующий до степени $m$ полином $\polynomial{\xi_c}$ имеет ненулевой коэффициент при степени $m$ в произведении
$\polynomial{\alpha} \polynomial{\xi_c}$. Cтало быть, исходное предположение о степени полинома $\polynomial{\xi^{(m)}}$ является неверным, и
потому полином $\polynomial{\xi_c}$ обязательно должен иметь степень больше $m$.

Раз степень текущего полинома $\polynomial{\xi_c}$ должна быть больше $m$, то постараемся в качестве текущего полинома $\polynomial{\xi_c}$
подобрать такой полином степени $m+1$, который являлся бы аннулирущим до степени $m+1$. Оказывается, что подобрать такой полином не представляет
большого труда:

\begin{itemize}
	\item если $\alpha_{m+1} = 0$, то в качестве текущего полинома возьмем полином:

		$$ \xi_c(\lambda) = 1 + \lambda^{m+1}, $$

	\item если $\alpha_{m+1} \neq 0$,  то в качестве текущего полинома будем использовать полином:

		$$ \xi_c(\lambda) = 1 - \alpha_{m+1}\alpha_m^{-1} \lambda + \lambda^{m+1} $$
\end{itemize}

Заметим, что приведенное определение для текущего полинома является корректным, поскольку $\alpha_m \neq 0$, и в поле $\gfp$ существует
обратный по умножению элемент $\alpha_m^{-1}$.

В первом и втором случаях степень $d_c$ текущего полинома $\polynomial{\xi_c}$:

	$$ d_c = m+1. $$

Кроме того, в том и другом случаях коэффициент при нулевой степени равен 1, то есть

	$$ \polynomial{\xi_c} \in \gfpunitypolynomial, $$

и в том и другом случаях коэффициент при степени $m+1$ в произведении $\polynomial{\alpha} \polynomial{\xi_c}$ равен 0:
действительно, в первом случае,

	$$ \degreecoefficient{m}{\polynomial{\alpha} \polynomial{\xi_c}} = \alpha_{m+1} \cdot 1 + \alpha_m \cdot 0 = \alpha_{m+1} = 0, $$

а во втором случае,

	$$ \degreecoefficient{m}{\polynomial{\alpha} \polynomial{\xi_c}}
		= \alpha_{m+1} \cdot 1 + \alpha_m \cdot ( - \alpha_{m+1} \alpha_m^{-1} )
		= \alpha_{m+1} - \alpha_{m+1} = 0. $$

Следовательно полином $\polynomial{\xi_c}$ по построению является аннулирующим до степени $m+1$ полиномом из множества $\gfpunitypolynomial$,
и можно в качестве $\polynomial{\xi^{(m+1)}}$ взять полином $\polynomial{\xi_c}$:

	$$ \polynomial{\xi^{(m+1)}} = \polynomial{\xi_c}. $$

Коэффициент при степени $m$ в произведении $\polynomial{\alpha} \polynomial{\xi_c}$ нулю не равен, но поскольку степень текущего
полинома $\polynomial{\xi_c}$, равная $m+1$, больше $m$, то по соглашению \ref{agreement:FPAP:ignoring_nonzero_product_coefficients} о том,
что коэффициенты при меньших степенях нас не интересуют, полином $\polynomial{\xi_c}$ степени $m+1$ считается аннулирующим до степени $m$,
поэтому в качестве полинома $\polynomial{\xi^{(m)}}$ можно использовать текущий полином $\polynomial{\xi_c}$:

	$$ \polynomial{\xi^{(m)}} = \polynomial{\xi_c}. $$

Ничего другого не остается, поскольку в множестве $\gfpunitypolynomial$, как уже было установлено ранее, не существует аннулирующих до
степени $m$ полиномов, степень которых меньше или равна $m$.

Заметим, что по построению текущий полином $\polynomial{\xi_c} \in \gfpunitypolynomial$, поэтому, как и требовалось:

	$$ \polynomial{\xi^{(m)}} \in \gfpunitypolynomial, $$
	$$ \polynomial{\xi^{(m+1)}} \in \gfpunitypolynomial. $$

На этом заканчивается шаг Н.2, и по совокупным итогам шагов Н.1 и Н.2 построены все аннулирующие полиномы $\polynomial{\xi^{(k)}}$ вплоть
до $k = m+1$.

Обратим внимание, что для полинома $\polynomial{\xi_c}$ схема произведения $\polynomial{\alpha} \polynomial{\xi_c}$ имеет вид
(например, для $m=2$):

	$$
		\begin{array}{cccccccc}
			                   &         &         & m       & m+1 &         &         & \\
			                   & 0       & 1       & 2       & 3   & 4       & 5       & \dots \\
			\hline
			\polynomial{\xi_c} & \bullet & \bullet & \bullet & 0   & \bullet & \bullet & \dots \\
		\end{array}
	$$

а общая схема с предыдущим полиномом $\polynomial{\xi_p}$ имеет вид:

	$$
		\begin{array}{cccccccc}
			                   &         &         & m        & m+1     &         &         & \\
			                   & 0       & 1       & 2        & 3       & 4       & 5       & \dots \\
			\hline
			\polynomial{\xi_p} & 0       & 0       & \diamond & \bullet & \bullet & \bullet & \dots \\
			\polynomial{\xi_c} & \bullet & \bullet & \bullet  & 0       & \bullet & \bullet & \dots \\
		\end{array}
	$$

Последняя схема является типичной для выполнения итерационных шагов алгоритма, поэтому шаг инициализации на этом заканчивается и начинается
итерационный шаг.

В начале итерационного шага рассматривается некоторая степень $k$ при общей схеме произведения, имеющий, например, вид:

	$$
		\begin{array}{cccccccccccc}
			                   &         & d_p     &         & m            &         & d_c     &         &         & k       &         & \\
			                   & 0       & 1       & 2       & 3            & 4        & 5       & 6       & 7       & 8       & 9       & \dots \\
			\hline
			\polynomial{\xi_p} & \bullet & 0       & 0       & \delta_{p,m} & \bullet & \bullet & \bullet & \bullet & \bullet & \bullet & \dots \\
			\polynomial{\xi_c} & \bullet & \bullet & \bullet & \bullet      & \bullet & 0       & 0       & 0       & \bullet & \bullet & \dots \\
		\end{array}
	$$

в которой выполняется следующие условия:

\begin{conditions} \label{conditions:FPAP:iteration_entrance}
	\begin{enumerate} 
		\item предыдущий полином $\polynomial{\xi_p}$, имеющий степень $d_p$, зануляет коэффициенты в произведении
			$\polynomial{\alpha} \polynomial{\xi_p}$ от степени $d_p$ до степени $m-1$ включительно, а коэффициент $\delta_{p,m}$
			при степени $m$ в произведении $\polynomial{\alpha} \polynomial{\xi_p}$ нулю не равен:

				$$ \delta_{p,m} \neq 0; $$

		\item текущий полином $\polynomial{\xi_c}$, имеющий степень $d_c$, зануляет коэффициенты в произведении
			$\polynomial{\alpha} \polynomial{\xi_c}$ от степени $d_c$ до степени $k-1$ включительно;

		\item текущий полином $\polynomial{\xi_c}$ принадлежит множеству $\gfpunitypolynomial$:

			$$ \polynomial{\xi_c} \in \gfpunitypolynomial. $$

		\item $k > m$
	\end{enumerate}
\end{conditions}

Вполне допустимо, что $d_p = m$ и между $d_p$ и $m$ нет нулей в схеме умножения для предыдущего полинома $\polynomial{\xi_p}$ (такая
ситуация может возникнуть, например, если на шаге Н.1 $m=0$). Принципиально важным является лишь то, что коэффициент $\delta_{p,m}$ не равен нулю.

Легко видеть, что в описываемую схему укладывается и схема, полученная после двух шагов инициализации Н.1 и Н.2: действительно, достаточно считать,
что степень предыдущего полинома $d_p = 0$, $m$ --- произвольное, но не меньше 0 ($m \ge d_p$), степень текущего полинома $d_c = m+1$ и
$k$ --- следующее за $m+1$ значение ( $k = m+2$ ), кроме того по построению текущий полином $\polynomial{\xi_c} \in \gfpunitypolynomial$.

Целью итерационного шага является построение полинома $\polynomial{\xi^{(k)}}$, аннулирующего до степени $k$ и принадлежащего
множеству $\gfpunitypolynomial$, и первое действие итерационного шага заключается в вычислении коэффициента
$\degreecoefficient{k}{\polynomial{\alpha} \polynomial{\xi_c}}$ при степени $k$ в произведении $\polynomial{\alpha} \polynomial{\xi_c}$.
Далее итерационный шаг разделяется на два взаимоисключающих варианта, которые будем обозначать И.1 и И.2 (от слова "Итерация"{}):

\begin{itemize}
	\item [И.1] выполняется, если $\degreecoefficient{m}{\polynomial{\alpha} \polynomial{\xi_c}} = 0$;
	\item [И.2] выполняется, если $\degreecoefficient{m}{\polynomial{\alpha} \polynomial{\xi_c}} \neq 0$.
\end{itemize}

Вариант И.1 является очень простым: если коэффициент $\degreecoefficient{m}{\polynomial{\alpha} \polynomial{\xi_c}} = 0$, то схема умножения
имеет вид:

	$$
		\begin{array}{cccccccccccc}
			                   &         & d_p     &         & m            &         & d_c     &         &         & k       &         & \\
			                   & 0       & 1       & 2       & 3            & 4        & 5       & 6       & 7       & 8       & 9       & \dots \\
			\hline
			\polynomial{\xi_p} & \bullet & 0       & 0       & \delta_{p,m} & \bullet & \bullet & \bullet & \bullet & \bullet & \bullet & \dots \\
			\polynomial{\xi_c} & \bullet & \bullet & \bullet & \bullet      & \bullet & 0       & 0       & 0       & 0       & \bullet & \dots \\
		\end{array}
	$$

и текущий полином $\polynomial{\xi_c}$ является аннулирущим до степени $k$, поэтому в качестве полинома $\polynomial{\xi^{(k)}}$ следует взять
полином $\polynomial{\xi_c}$ :

	$$ \polynomial{\xi^{(k)}} = \polynomial{\xi_c} $$

и на этом вариант И.1 итерационного шага окончен и необходимо перейти к проверке критерия завершения.

Легко видеть, что схема произведения по окончании варианта И.1 итерационного шага имеет тип, соответствующий началу итерационного шага,
достаточно лишь вместо $k$ взять $k+1$.

Вариант И.2 оказывается чуть более сложным: если $\degreecoefficient{m}{\polynomial{\alpha} \polynomial{\xi_c}} \neq 0$, то схема умножения
имеет вид:

	$$
		\begin{array}{cccccccccccc}
			                   &         & d_p     &         & m            &         & d_c     &         &         & k            &         & \\
			                   & 0       & 1       & 2       & 3            & 4       & 5       & 6       & 7       & 8            & 9       & \dots \\
			\hline
			\polynomial{\xi_p} & \bullet & 0       & 0       & \delta_{p,m} & \bullet & \bullet & \bullet & \bullet & \bullet      & \bullet & \dots \\
			\polynomial{\xi_c} & \bullet & \bullet & \bullet & \bullet      & \bullet & 0       & 0       & 0       & \delta_{c,k} & \bullet & \dots \\
		\end{array}
	$$

где

	$$ \delta_{c,k} = \degreecoefficient{k}{\polynomial{\alpha} \polynomial{\xi_c}} \neq 0. $$

В данном случае необходимо коэффициент $\delta_{c,k}$ занулить с помощью ненулевого коэффициента $\delta_{p,m}$. Прежде всего, умножим произведение
$\polynomial{\alpha} \polynomial{\xi_p}$ на $\lambda^{k-m}$: в результате такого умножения все степени увеличаться на $k-m$, таким образом,
все коэффициенты произведения $\polynomial{\alpha} \polynomial{\xi_p}$ сдвинуться на $k-m$, а младших степеней до $k-m$ не будет
вовсе (все коэффициенты при этих степенях будут равны нулю):

	$$
		\begin{array}{rccccccccccc}
			                                 &         & d_p     &         & m            &         & d_c     &         &         & k            & \\
			                                 & 0       & 1       & 2       & 3            & 4       & 5       & 6       & 7       & 8            & 9       & \dots \\
			\hline
			\polynomial{\xi_p}               & \bullet & 0       & 0       & \delta_{p,m} & \bullet & \bullet & \bullet & \bullet & \bullet      & \bullet & \dots \\
			\lambda^{k-m} \polynomial{\xi_p} & 0       & 0       & 0       & 0            & 0       & \bullet & 0       & 0       & \delta_{p,m} & \bullet & \dots \\
			\polynomial{\xi_c}               & \bullet & \bullet & \bullet & \bullet      & \bullet & 0       & 0       & 0       & \delta_{c,k} & \bullet & \dots \\
		\end{array}
	$$

В полученной схеме коэффициенты $\delta_{p,m}$ и $\delta_{c,k}$ являются коэффициентами при одной и той же степени $k$.
Остается лишь вычислить такую комбинацию сдвинутого предыдущего полинома $\lambda^{k-m} \polynomial{\xi_p}$ и текущего полинома
$\polynomial{\xi_c}$, в которой коэффициент при степени $k$ является нулевым. Такой комбинацией является разность $\polynomial{\xi_d}$:

	$$ \xi_d(\lambda) = \xi_c(\lambda) - \delta_{c,k} \delta_{p,m}^{-1} \lambda^{k-m} \xi_p(\lambda), $$

и как не трудно заметить, для представленной разности $\polynomial{\xi_d}$ коэффициент при степени $k$ в произведении
$\polynomial{\alpha} \polynomial{\xi_d}$ равен 0, действительно:

	$$ \degreecoefficient{k}{\polynomial{\alpha} \polynomial{\xi_d}} = $$
	$$ = \degreecoefficient{k}{\polynomial{\alpha} \left ( \polynomial{\xi_c} - \delta_{c,k} \delta_{p,m}^{-1} \lambda^{k-m} \polynomial{\xi_p} \right ) } = $$
	$$ = \degreecoefficient{k}{\polynomial{\alpha} \polynomial{\xi_c}
		- \delta_{c,k} \delta_{p,m}^{-1} \polynomial{\alpha} \lambda^{k-m} \polynomial{\xi_p}} = $$
	$$ = \degreecoefficient{k}{\polynomial{\alpha} \polynomial{\xi_c}}
		- \delta_{c,k} \delta_{p,m}^{-1} \degreecoefficient{k}{\polynomial{\alpha} \lambda^{k-m} \polynomial{\xi_p}} = $$
	\begin{equation} \label{equation:FPAP:annihilating_of_kth_degree_in_difference_polynomial_product}
		= \delta_{c,k} - \delta_{c,k} \delta_{p,m}^{-1} \delta_{p,m} = \delta_{c,k} - \delta_{c,k} = 0.
	\end{equation}

Кроме того, полином разности $\polynomial{\xi_d}$ принадлежит множеству $\gfpunitypolynomial$, действительно, коэффициент при нулевой степени
полинома разности $\polynomial{\xi_d}$ складывается из коэффициентов при нулевой степени текущего полинома $\polynomial{\xi_c}$ и
сдвинутого предыдущего полинома $\lambda^{k-m} \polynomial{\xi_p}$:

	$$ \degreecoefficient{0}{\polynomial{\xi_d}}
		= \degreecoefficient{0}{\polynomial{\xi_c}}
			- \delta_{c,k} \delta_{p,m}^{-1} \degreecoefficient{0}{\lambda^{k-m} \polynomial{\xi_p}}, $$

поскольку $k > m$ (по условиям \ref{conditions:FPAP:iteration_entrance} начала итеационного шага), то сдвинутый предыдущий полином
$\lambda^{k-m} \polynomial{\xi_p}$ "начинается"{} со степени $k-m > 0$, то есть нулевой степени, в сдвинутом предыдущем полиноме
$\lambda^{k-m} \polynomial{\xi_p}$ нет и коэффициент при нулевой степени равен нулю:

	$$ \degreecoefficient{0}{\xi_d} = \degreecoefficient{0}{\xi_c} - \delta_{c,k} \delta_{p,m}^{-1} \cdot 0 = \degreecoefficient{0}{\xi_c} = 1, $$

поскольку $\polynomial{\xi_c} \in \gfpunitypolynomial$ в соответствии с условиями \ref{conditions:FPAP:iteration_entrance} начала итерационного шага.
Таким образом

	\begin{equation} \label{equation:FPAP:difference_polynomial_belongs_to_unity_polynomials}
		\polynomial{\xi_d} \in \gfpunitypolynomial.
	\end{equation}

Общая схема произведений с полиномом разности $\polynomial{\xi_d}$ имеет вид:

	$$
		\begin{array}{rccccccccccc}
			                                 &         & d_p     &         & m            &         & d_c     &         &         & k            & \\
			                                 & 0       & 1       & 2       & 3            & 4       & 5       & 6       & 7       & 8            & 9       & \dots \\
			\hline
			\polynomial{\xi_p}               & \bullet & 0       & 0       & \delta_{p,m} & \bullet & \bullet & \bullet & \bullet & \bullet      & \bullet & \dots \\
			\lambda^{k-m} \polynomial{\xi_p} & 0       & 0       & 0       & 0            & 0       & \bullet & 0       & 0       & \delta_{p,m} & \bullet & \dots \\
			\polynomial{\xi_c}               & \bullet & \bullet & \bullet & \bullet      & \bullet & 0       & 0       & 0       & \delta_{c,k} & \bullet & \dots \\
			\polynomial{\xi_d}               & \bullet & \bullet & \bullet & \bullet      & \bullet & \bullet & 0       & 0       & 0            & \bullet & \dots \\
		\end{array}
	$$

По завершению варианта И.2 итерационного шага необходимо из четырех полиномов: $\polynomial{\xi_p}$, $\lambda^{k-m} \polynomial{\xi_p}$,
$\polynomial{\xi_c}$ и $\polynomial{\xi_d}$ оставить только два. Вполне очевидно, что разность $\polynomial{\xi_d}$ следует сделать новым
текущим полиномом. Выбор предыдущего полинома оказывается чуть более сложным: если степень сдвинутого полинома $\lambda^{k-m} \polynomial{\xi_p}$
оказывается больше степени текущего полинома $\polynomial{\xi_c}$, то предыдущий полином $\polynomial{\xi_p}$ следует заменить текущим полиномом
$\polynomial{\xi_c}$, в противном случае предыдущий полином $\polynomial{\xi_p}$ следует оставить без изменений.

В настоящий момент, конечно, не вполне ясно почему вообще следует менять предыдущий полином $\polynomial{\xi_p}$, ведь его можно было
использовать аналогичным образом и дальше, зануляя ненулевые коэффициенты в произведениях с текущим полиномом $\polynomial{\xi_c}$. Чуть позже
в подразделе \ref{section:MDAP:minimal_degree_of_annihilating_polynomial} будет показано, что это требуется для того, чтобы степени
полиномов $\polynomial{\xi^{(k)}}$ были минимальными.

Таким образом, на заключительном этапе варианта И.2 итерационного шага, следует выполнить присваивания:

	\begin{equation} \label{equation:FPAP:previous_polynomial_assignment}
		\polynomial{\xi_p}
		\leftarrow
		\left \{
			\begin{array}{ccc}
				\polynomial{\xi_p}, & \text{если} & d_p + k - m \le d_c \\
				\polynomial{\xi_c}, & \text{если} & d_p + k - m > d_c \\
			\end{array}
		\right .
		,
	\end{equation}
	\begin{equation} \label{equation:FPAP:current_polynomial_assignment}
		\polynomial{\xi_c} \leftarrow \polynomial{\xi_d}.
	\end{equation}

Соответствующим образом изменяются и степени полиномов:

	$$ d_p = \polynomialdegree{\xi_p}, $$
	$$ d_c = \polynomialdegree{\xi_c}. $$

После присваивания текущий полином $\polynomial{\xi_c}$ согласно равенству
\eqref{equation:FPAP:annihilating_of_kth_degree_in_difference_polynomial_product} становится аннулирущим до степени $k$. Таким образом,

	$$ \polynomial{\xi^{(k)}} = \polynomial{\xi_c}. $$

Кроме того, поскольку после присваивания \eqref{equation:FPAP:current_polynomial_assignment} новый текущий полином $\polynomial{\xi_c}$ ---
это полином разности $\polynomial{\xi_d}$, который принадлежит множеству $\gfpunitypolynomial$ согласно
\eqref{equation:FPAP:difference_polynomial_belongs_to_unity_polynomials}, то

	$$ \polynomial{\xi^{(k)}} \in \gfpunitypolynomial, $$

и основная цель итерационного шага выполнена --- построен полином $\polynomial{\xi^{(k)}}$, аннулирующий до степени $k$, из множества
$\gfpunitypolynomial$.

Схема произведения по завершению варианта И.2 итерационного шага имеет вид:

	$$
		\begin{array}{rccccccccccc}
			                                 &         &         &         &              &         & d_p     & d_c     &         & m            & \\
			                                 & 0       & 1       & 2       & 3            & 4       & 5       & 6       & 7       & 8            & 9       & \dots \\
			\hline
			\polynomial{\xi_p}               & \bullet & \bullet & \bullet & \bullet      & \bullet & 0       & 0       & 0       & \delta_{c,k} & \bullet & \dots \\
			\polynomial{\xi_c}               & \bullet & \bullet & \bullet & \bullet      & \bullet & \bullet & 0       & 0       & 0            & \bullet & \dots \\
		\end{array}
	$$

и, как не трудно заметить, удовлетворяются все условия \ref{conditions:FPAP:iteration_entrance} с учетом того, что для следущего итерационного
шага необходимо увеличить $k$ на 1, а значит над полученными полиномами $\polynomial{\xi_p}$ и $\polynomial{\xi_c}$ можно опять выполнить
итерационный шаг алгоритма. Таким образом, на этом вариант И.2 итерационного шага завершается и необходимо перейти к шагу проверки
критерия завершения.

Критерий завершения заключается в том, чтобы проверить является ли очередной построенный полином $\polynomial{\xi^{(k)}}$ искомым, другими
словами необходимо определить сколько степеней в произведении $\polynomial{\alpha} \polynomial{\xi^{(k)}}$ зануляет полином
$\polynomial{\xi^{(k)}}$, начиная от степени самого полинома $\polynomialdegree{\xi^{(k)}}$, которая, кстати, совпадает со степенью текущего
полинома $d_c$. Если разность $k - d_c \ge n-1$, то полином $\polynomial{\xi^{(k)}}$ является искомым полиномом $\polynomial{\xi}$ из
множества $\gfpunitypolynomial$. Если же $k - d_c < n-1$, то необходимо продолжить выполнение итерационных шагов алгоритма.

По завершению работы алгоритма последний построенный полином $\polynomial{\xi^{(k)}}$ будет являтся искомым полиномом $\polynomial{\xi}$,
коэффициенты которого удовлетворяют системе равенств \eqref{equation:FPAP:equations_for_xi}:

	$$ \polynomial{\xi} = \polynomial{\xi^{(k)}}, $$

и для получения проекционно аннулирующего полинома $\polynomial{\rho}$ необходимо будет взять коэффициенты полинома $\polynomial{\xi}$
в обратном порядке, в соответствии с равенством \eqref{equation:FPAP:rho_and_xi_equation}.

\subsection{Минимальность проекционно аннулирущего полинома} \label{section:MDAP:minimal_degree_of_annihilating_polynomial}

В этом подразделе будет доказано, что каждый из аннулирующих до степени $k$ полиномов $\polynomial{\xi^{(k)}}$, которые строит алгоритм,
представленный в предыдущем разделе, имеет наименьшую возможную степень из всех аннулирующих до степени $k$ полиномов.

Прежде всего, следуя работе Месси \cite{Massey}, докажем вспомогательное утверждение.

\begin{statement} \label{statement:MDAP:annihilating_polynomial_degrees_inequality}

	Пусть полином $\polynomial{\gamma} \in \gfpunitypolynomial$ является аннулирущим до степени $k-1$, но не
	является аннулирующим до степени $k$, и полином $\polynomial{\hat\gamma} \in \gfpunitypolynomial$ является аннулирующим до степени $k$,
	тогда степени полиномов $\polynomial{\gamma}$ и $\polynomial{\hat\gamma}$ удовлетворяют неравенству:

		$$ \polynomialdegree{\hat\gamma} \ge k + 1 - \polynomialdegree{\gamma}. $$

	\proof

	Приводимое далее доказательство имеет следующую схему:

	\begin{tabular}{|p{1cm}|p{4cm}|p{4cm}|p{5cm}|}
	\hline
	№  & ограничение для $\polynomialdegree{\gamma}$ & ограничение для $\polynomialdegree{\hat\gamma}$ & Доказательство \\
	\hline
	\hline
	1  & $\polynomialdegree{\gamma} \ge k+1$         & нет ограничения                                 & этот случай оказывается невозможным, поскольку противоречит условиям утверждения \\
	\hline
	2  & $\polynomialdegree{\gamma} = k$             & нет ограничения                                 & этот случай оказывается также невозможным, поскольку противоречит условиям утверждения \\
	\hline
	3a & $\polynomialdegree{\gamma} \le k-1$         & $\polynomialdegree{\hat\gamma} \ge k+1$         & доказываемое неравенство является тривиальным \\
	\hline
	3b & $\polynomialdegree{\gamma} \le k-1$         & $\polynomialdegree{\hat\gamma} \le k$           & доказывается от противного \\
	\hline
	\end{tabular}

	Пусть $d$ обозначает степень полинома $\polynomial{\gamma}$:

		$$ d = \polynomialdegree{\gamma}. $$

	Для величины степени $d$ возможными являются три различных случая:

	\begin{enumerate}
		\item $d \ge k+1$;
		\item $d = k$;
		\item $d \le k-1$.
	\end{enumerate}

	Рассмотрим все случаи по порядку.

	\begin{enumerate}
		\item
			Рассмотрим первый случай, пусть $d \ge k + 1$, тогда по соглашению \ref{agreement:FPAP:ignoring_nonzero_product_coefficients} полином
			$\polynomial{\gamma}$ является аннулирующим до степени $k$, что противоречит условию утверждения --- полином $\polynomial{\gamma}$ не
			является аннулирующим до степени $k$. Таким образом, первый случай оказывается невозможным.

		\item
			Рассмотрим второй случай, пусть $d = k$. В этом случае неравенство утверждения

				$$ \polynomialdegree{\hat\gamma} \ge k + 1 - \polynomialdegree{\gamma}. $$

			вырождается в неравенство

				$$ \polynomialdegree{\hat\gamma} \ge k + 1 - d. $$
				$$ \polynomialdegree{\hat\gamma} \ge k + 1 - k. $$
				$$ \polynomialdegree{\hat\gamma} \ge 1. $$

			Таким образом, необходимо доказать, что степень полинома $\polynomial{\hat\gamma}$, аннулирующего до степени $k$, не может быть
			меньше 1. Проведем доказательство от противного и предположим, что полином $\polynomial{\hat\gamma}$ имеет степень меньше 1, то есть
			имеет степень 0, другими словами полином $\polynomial{\hat\gamma}$ является константой, и поскольку
			$\polynomial{\hat\gamma} \in \gfpunitypolynomial$, то эта константа равна 1:

				$$ \polynomial{\hat\gamma} = 1. $$

			Если полином $\polynomial{\hat\gamma}$ является аннулирующим до степени $k$, то значит в произведении
			$\polynomial{\alpha} \polynomial{\hat\gamma}$ равны нулю коэффициенты при степенях, начиная со степени самого полинома
			$\polynomial{\hat\gamma}$, то есть с 0, и до степени $k$ включительно:

				$$
					\left \{
					\begin{array}{c}
						\degreecoefficient{0}{\polynomial{\alpha} \polynomial{\hat\gamma}} = 0 \\
						\vdots \\
						\degreecoefficient{k}{\polynomial{\alpha} \polynomial{\hat\gamma}} = 0 \\
					\end{array}
					\right .
					,
				$$

				но полином $\polynomial{\hat\gamma} = 1$, тогда равны нулю коэффициенты в полиноме $\polynomial{\alpha}$:

				$$
					\left \{
					\begin{array}{c}
						\degreecoefficient{0}{\polynomial{\alpha}} = 0 \\
						\vdots \\
						\degreecoefficient{k}{\polynomial{\alpha}} = 0 \\
					\end{array}
					\right .
					,
				$$

				$$
					\left \{
					\begin{array}{c}
						\alpha_0 = 0 \\
						\vdots \\
						\alpha_k = 0 \\
					\end{array}
					\right .
					.
				$$

			В этом случае, очевидно, любой полином степени не больше $k$ окажется аннулирущим, поскольку в произведении до степени $k$
			участвуют только коэффициенты $\alpha_0$, \dots, $\alpha_k$ и все они равны 0.

			Отсюда следует, что и полином $\polynomial{\gamma}$, имеющий степень $d = k$, также является аннулирущим до степени $k$, что
			опять же противоречит условию утверждения --- полином $\polynomial{\gamma}$ не является аннулирующим до степени $k$. Таким образом,
			второй случай также оказывается невозможным.

		\item
			Остается только третий случай, в котором $d \le k-1$. Пусть полином $\polynomial{\gamma}$ имеет вид:

				$$ \gamma(\lambda) = \gamma_0 + \gamma_1 \lambda + \dots + \gamma_{d-1} \lambda^{d-1} + \gamma_d \lambda^d. $$

			Поскольку $\polynomial{\gamma} \in \gfpunitypolynomial$, то коэффициент при нулевой степени $\gamma_0$ равен 1:

				$$ \gamma(\lambda) = 1 + \gamma_1 \lambda + \dots + \gamma_{d-1} \lambda^{d-1} + \gamma_{d} \lambda^d. $$

			Согласно условию утверждения полином $\polynomial{\gamma}$ является аннулирующим до степени $k-1$, следовательно выполняются
			равенства (или по крайней мере одно равенство, если $d = k-1$):

				$$
					\left \{
					\begin{array}{c}
						\degreecoefficient{d}{\polynomial{\alpha} \polynomial{\gamma}} = 0 \\
						\vdots \\
						\degreecoefficient{k-1}{\polynomial{\alpha} \polynomial{\gamma}} = 0 \\
					\end{array}
					\right .
					,
				$$

				$$
					\left \{
					\begin{array}{c}
						\alpha_d + \gamma_1 \alpha_{d-1} + \dots + \gamma_{d-1} \alpha_1 + \gamma_d \alpha_0 = 0 \\
						\vdots \\
						\alpha_{k-1} + \gamma_1 \alpha_{k-1-1} + \dots + \gamma_{d-1} \alpha_{k-1-(d-1)} + \gamma_d \alpha_{k-1-d} = 0 \\
					\end{array}
					\right .
				$$

			Отсюда

				$$
					\left \{
					\begin{array}{c}
						\alpha_d = - \left ( \gamma_1 \alpha_{d-1} + \dots + \gamma_{d-1} \alpha_1 + \gamma_d \alpha_0 \right ) \\
						\vdots \\
						\alpha_{k-1} = - \left ( \gamma_1 \alpha_{k-1-1} + \dots + \gamma_{d-1} \alpha_{k-1-(d-1)} + \gamma_d \alpha_{k-1-d} \right ) \\
					\end{array}
					\right .
				$$

			или в более короткой форме:

				\begin{equation} \label{equation:MDAP:alpha_through_gamma}
					\alpha_j = - \sum_{i=1}^d \gamma_i \alpha_{k-i},
				\end{equation}
				$$ j = d, \dots, k-1. $$

			Теперь рассмотрим полином $\polynomial{\hat\gamma}$, обозначим $\hat d$ степень полинома $\polynomial{\hat\gamma}$:

				$$ \hat d = \polynomialdegree{\hat\gamma}. $$

			Для величины $\hat d$ возможны два случая:

			\begin{enumerate}
				\item $\hat d \ge k+1$;
				\item $\hat d \le k$.
			\end{enumerate}

			Рассмотрим оба случая.

			\begin{enumerate}
				\item
					Если $\hat d \ge k+1$, то

						$$ \polynomialdegree{\hat\gamma} = \hat d \ge k + 1 \ge k + 1 - \polynomialdegree{\gamma}, $$

					поскольку $\polynomialdegree{\gamma} \ge 0$. Таким образом, для случая $\hat d \ge k+1$ утверждение доказано.

				\item
					Пусть $\hat d \le k$, и полином $\polynomial{\hat\gamma} \in \gfpunitypolynomial$ имеет вид:

						$$ \hat\gamma(\lambda)
							= 1 + \hat\gamma_1 \lambda + \dots + \hat\gamma_{\hat d - 1} \lambda^{\hat d - 1} + \hat\gamma_{\hat d - 1} \lambda^{\hat d - 1}. $$

					Согласно условию утверждения полином $\polynomial{\hat\gamma}$ является аннулирующим до степени $k$, поэтому должны
					выполнятся равенства:

						$$
							\left \{
							\begin{array}{c}
								\degreecoefficient{\hat d}{\polynomial{\alpha} \polynomial{\hat\gamma}} = 0 \\
								\vdots \\
								\degreecoefficient{k}{\polynomial{\alpha} \polynomial{\hat\gamma}} = 0 \\
							\end{array}
							\right .
							,
						$$

						$$
							\left \{
							\begin{array}{c}
								\alpha_{\hat d} + \hat\gamma_1 \alpha_{\hat d - 1} + \dots + \hat\gamma_{\hat d - 1} \alpha_1 + \hat\gamma_{\hat d} \alpha_0 = 0 \\
								\vdots \\
								\alpha_k + \hat\gamma_1 \alpha_{k-1} + \dots + \hat\gamma_{\hat d - 1} \alpha_{k - \hat d - 1} + \hat\gamma_{\hat d} \alpha_{k - \hat d} = 0 \\
							\end{array}
							\right .
						$$

						$$
							\left \{
							\begin{array}{c}
								\alpha_{\hat d} = - \left ( \hat\gamma_1 \alpha_{\hat d - 1} + \dots + \hat\gamma_{\hat d - 1} \alpha_1 + \hat\gamma_{\hat d} \alpha_0 \right ) \\
								\vdots \\
								\alpha_k = - \left ( \hat\gamma_1 \alpha_{k-1} + \dots + \hat\gamma_{\hat d - 1} \alpha_{k - \hat d - 1} + \hat\gamma_{\hat d} \alpha_{k - \hat d} \right ) \\
							\end{array}
							\right .
						$$

						\begin{equation} \label{equation:MDAP:alpha_through_hat_gamma}
							\alpha_j = - \sum_{l=1}^{\hat d} \hat\gamma_l \alpha_{j-l},
						\end{equation}
						$$ j = \hat d, \dots, k. $$

					Докажем утверждение от противного: предположим, что

						$$ \polynomialdegree{\hat\gamma} \le k - \polynomialdegree{\gamma}, $$

					или в принятых обозначениях

						\begin{equation} \label{equation:MDAP:contrary_inequality_for_hat_gamma_degree}
							\hat d \le k - d.
						\end{equation}

					По условию утверждения полином $\polynomial{\gamma}$ не является аннулирущиюм до степени $k$, поэтому коэффициент при
					степени $k$ в произведении $\polynomial{\alpha} \polynomial{\gamma}$ не равен 0:

						$$ \degreecoefficient{k}{\polynomial{\alpha} \polynomial{\gamma}} \neq 0, $$
						$$ \alpha_{k} + \gamma_1 \alpha_{k-1} + \dots + \gamma_{d-1} \alpha_{k-(d-1)} + \gamma_d \alpha_{k-d} \neq 0 $$
						$$ \alpha_{k} \neq - \left ( \gamma_1 \alpha_{k-1} + \dots + \gamma_{d-1} \alpha_{k-(d-1)} + \gamma_d \alpha_{k-d} \right ) $$
						\begin{equation} \label{equation:MDAP:alpha_k_inequality_for_gamma}
							\alpha_k \neq \sum_{i=1}^{d} \gamma_i \alpha_{k-i},
						\end{equation}

					где в сумме правой части используются коэффициенты $\alpha_{k-1}$, \dots, $\alpha_{k-d}$. Поскольку по предположению
					\eqref{equation:MDAP:contrary_inequality_for_hat_gamma_degree} $\hat d \le k - d$, то каждый из коэффициентов $\alpha_{k-1}$,
					\dots, $\alpha_{k-d}$ имеет рекуррентное выражение с коэффициентами $\hat \gamma_l$ согласно
					\eqref{equation:MDAP:alpha_through_hat_gamma}. Подставляя рекуррентные соотношения из
					\eqref{equation:MDAP:alpha_through_hat_gamma} в сумму правой части неравенства
					\eqref{equation:MDAP:alpha_k_inequality_for_gamma} получим:

						$$ \sum_{i=1}^{d} \gamma_i \alpha_{k-i} = \sum_{i=1}^{d} \gamma_i \left ( - \sum_{l=1}^{\hat d} \hat\gamma_l \alpha_{k-i-l} \right ). $$

					Изменяя порядок суммирования в правой части, получим равенство:

						$$ \sum_{i=1}^{d} \gamma_i \alpha_{k-i} = \sum_{l=1}^{\hat d} \hat\gamma_l \left ( - \sum_{i=1}^{d} \gamma_i  \alpha_{k-l-i} \right ). $$

					Теперь из равенства \eqref{equation:MDAP:contrary_inequality_for_hat_gamma_degree} следует, что $d \le k - \hat d$, но
					тогда согласно равенствам \eqref{equation:MDAP:alpha_through_gamma} суммы в скобках в правой части дают коэффициенты
					$\alpha_{k-l}$:
		
						$$ \sum_{i=1}^{d} \gamma_i \alpha_{k-i} = \sum_{l=1}^{\hat d} \hat\gamma_l \alpha_{k-l}. $$

					Правая часть полученного равенства в соответствии с равенствами \eqref{equation:MDAP:alpha_through_hat_gamma} равна $\alpha_k$:

						$$ \sum_{i=1}^{d} \gamma_i \alpha_{k-i} = \sum_{l=1}^{\hat d} \hat\gamma_l \alpha_{k-l} = \alpha_k, $$

					что противоречит неравенству \eqref{equation:MDAP:alpha_k_inequality_for_gamma}:

						$$ \alpha_k \neq \sum_{i=1}^{d} \gamma_i \alpha_{k-i} = \alpha_k, $$
						$$ \alpha_k \neq \alpha_k, $$

					Таким образом, исходное предположение \eqref{equation:MDAP:contrary_inequality_for_hat_gamma_degree} является неверным,
					следовательно:

						$$ \hat d \ge k + 1 - d, $$
						$$ \polynomialdegree{\hat\gamma} \ge k + 1 - \polynomialdegree{\gamma}. $$
			\end{enumerate}
	\end{enumerate}

	Таким образом, были рассмотрены все возможные случаи, и в тех случаях, которые не вступают в противоречие с условиями утверждения, неравенство

		$$ \polynomialdegree{\hat\gamma} \ge k + 1 - \polynomialdegree{\gamma}. $$

	оказывается справедливым.
\end{statement}

С использованием утверждения \ref{statement:MDAP:annihilating_polynomial_degrees_inequality} можно доказать, что аннулирующие до степени
$k$ полиномы $\polynomial{\xi^{(k)}}$, которые строит алгоритм, представленный в разделе
\ref{section:FPAP:finding_projection_annihilating_polynomials}, имеют минимальные степени среди аннулирущих до степени $k$ полиномов из
множества $\gfpunitypolynomial$.

Начнем с шага Н.1 инициализации предыдущего полинома $\polynomial{\xi_p}$: поскольку на шаге Н.1 предыдущий полином $\polynomial{\xi_p} = 1$,
то его степень равна 0. Если предыдущий полином $\polynomial{\xi_p}$ оказывается аннулирущим до степени $m-1$ (при $m > 0$), то алгоритм
строит аннулирущие полиномы:

	$$ \polynomial{\xi^{(k)}} = \polynomial{\xi_p} $$
	$$ k=0,\dots,m-1, $$

имеющие степень 0. Поскольку полиномов степени меньше 0 в множестве $\gfpunitypolynomial$ нет, то все полиномы $\polynomial{\xi^{(k)}}$
при $k=0,\dots,m-1$ являются полиномами минимальной возможной степени.

Далее, предыдущий полином $\polynomial{\xi_p}$ не является аннулирущим до степени $m$, и на шаге Н.2 производится инициализация текущего
полинома $\polynomial{\xi_c}$: текущий полином $\polynomial{\xi_c}$ имеет степень $m+1$ и является аннулирущим до степени $m+1$, поэтому
алгоритм строит еще два полинома:

	$$ \polynomial{\xi^{(m)}} = \polynomial{\xi_c}, $$
	$$ \polynomial{\xi^{(m+1)}} = \polynomial{\xi_c}. $$

При описании алгоритма было показано, что полином из множества $\gfpunitypolynomial$, аннулирущий до степени $m$, обязательно имеет степень
больше $m$. Степень полинома $\polynomial{\xi^{(m)}}$, который построил алгоритм, совпадает со степенью текущего полинома $\polynomial{\xi_c}$,
равной $m+1$, поэтому построенный полином $\polynomial{\xi^{(m)}}$ является аннулирущим до степени $m$ полиномом минимальной степени
(аннулирущего до степени $m$ полинома степени меньше чем $m+1$ в множестве $\gfpunitypolynomial$ не существует).

Аннулирующий полином $\polynomial{\xi^{(m+1)}}$ тоже имеет минимальную возможную степень среди аннулищих до степени $m+1$ полиномом из множества
$\gfpunitypolynomial$. В этом легко убедиться, если предположить обратное: пусть существует такой полином
$\polynomial{\hat\xi} \in \gfpunitypolynomial$, который является аннулирующим до степени $m+1$ и имеет степень меньшую или равную $m$, тогда
полином $\polynomial{\hat\xi}$ является также аннулирующим до степени $m$, но в множестве $\gfpunitypolynomial$ не существует полиномов,
аннулирущих до степени $m$, имеющих степень меньшую или равную $m$, следовательно имеет место противоречие --- такой полином $\polynomial{\hat\xi}$
не существует. Таким образом, если в множестве $\gfpunitypolynomial$ и существуют полиномы, аннулирующие до степени $m+1$, то они обязательно
имеют степень больше или равную $m+1$, следовательно построенный полином $\polynomial{\xi^{(m+1)}}$, имеющий степень $m+1$, является одним
из таких полиномов минимальной степени.

Поскольку по окончании шага Н.2 аннулирущий полином $\polynomial{\xi^{(m+1)}}$ --- это текущий полином $\polynomial{\xi_c}$, то перед началом
итерационного шага текущий полином $\polynomial{\xi_c}$ является аннулирующим до степени $m+1$ полиномом минимальной степени из множества
$\gfpunitypolynomial$.

Кроме того, как нетрудно заметить, степень текущего полинома $d_c = m+1$, а степень предыдущего полинома $d_p = 0$, поэтому степени
$d_c$ и $d_p$ удовлетворяют равенству:

	$$ d_c = m + 1 - d_p. $$

Таким образом, после шагов инициализации Н.1 и Н.2 выполняются два дополнительных условия (с учетом того, что $k$ --- число следующее
за $m+1$, то есть $k=m+2$):

\begin{conditions} \label{conditions:MDAP:iteration_entrance}
	\begin{enumerate}
		\item текущий полином $\polynomial{\xi_c}$ является аннулирущим до степени $k-1$ полиномом, имеющим минимальную степень среди всех
			аннулирущих до степени $k-1$ полиномов в множестве $\gfpunitypolynomial$;

		\item степени предыдущего и текущего полиномов удовлетворяют равенству:

			$$ d_c = m + 1 - d_p ; $$
	\end{enumerate}
\end{conditions}

Если на итерационном шаге реализуется вариант И.1, то очередной полином $\polynomial{\xi^{(k)}}$ совпадает с текущим полиномом $\polynomial{\xi_c}$
и имеет наименьшую возможную степень среди всех аннулирующих до степени $k$ полиномов из множества $\gfpunitypolynomial$. Действительно, предположим
противное: пусть существует полином $\polynomial{\hat\xi}$, который является аннулирущим до степени $k$ полиномом из $\gfpunitypolynomial$ со
степенью меньше, чем степень полинома $\polynomial{\xi^{(k)}}$, равной степени текущего полинома $\polynomial{\xi_c}$ (поскольку
$\polynomial{\xi^{(k)}}$ совпадает с $\polynomial{\xi_c}$), тогда полином $\polynomial{\hat\xi}$ является аннулирущим и до степени $k-1$ полиномом
и имеет степень меньшей, чем степень текущего полинома $\polynomial{\xi_c}$, но это невозможно, поскольку текущий полином в соответствии с
условиями \ref{conditions:MDAP:iteration_entrance} имеет минимальную степень среди всех аннулирущих до степени $k-1$ полиномов в множестве
$\gfpunitypolynomial$. В силу полученного противоречия такого полинома $\polynomial{\hat\xi}$ существовать не может.

По завершению варианта И.1 итерационного шага текущий полином становится также и аннулирующим до степени $k$ полиномом, имеющим минимальную
степень среди всех аннулирущих до степени $k$ полиномов в множестве $\gfpunitypolynomial$, и следовательно выполняется первое из условий
\ref{conditions:MDAP:iteration_entrance} перед началом следующей итерации при увеличенном на 1 значении $k$. Кроме того, поскольку текущий
и предыдущий полиномы не изменяются в варианте И.1, то их степени также не изменяются, и следовательно выполнено и второе из условий
\ref{conditions:MDAP:iteration_entrance} перед началом следующей итерации.

В случае реализации варианта И.2 итерационного шага строится полином разности $\polynomial{\xi_d}$, который является аннулирущим до степени
$k$ полиномом со степенью, которая определяется текущим полиномом $\polynomial{\xi_c}$ и сдвинутым предыдущим полиномом
$\lambda^{k-m} \polynomial{\xi_p}$ (в зависимости от того, степень какого полинома окажется больше):

	$$ \polynomialdegree{\xi_d} = max \left \{ \polynomialdegree{\xi_c} , \polynomialdegree{\lambda^{k-m}\xi_p} \right \}, $$
	\begin{equation} \label{equation:MDAP:initial_difference_polynomial_degree_equality}
		\polynomialdegree{\xi_d} = max \left \{ d_c , k - m + d_p \right \}.
	\end{equation}

В силу второго из условий \ref{conditions:MDAP:iteration_entrance}:

	$$ d_p = m + 1 - d_c, $$

откуда

	$$ \polynomialdegree{\xi_d} = max \left \{ d_c , k - m + m + 1 - d_c \right \}, $$
	\begin{equation} \label{equation:MDAP:final_difference_polynomial_degree_equality}
		\polynomialdegree{\xi_d} = max \left \{ d_c , k + 1 - d_c \right \}.
	\end{equation}

Покажем, что степень полинома разности $\polynomialdegree{\xi_d}$, определяемая последним равенством, является минимальной.

Пусть полином $\polynomial{\hat\xi}$ является любым аннулирущим до степени $k$ полиномом из множества $\gfpunitypolynomial$. Во-первых,
степень полинома $\polynomial{\hat\xi}$ не может быть меньше степени текущего полинома $\polynomial{\xi_c}$, в противном случае полином
$\polynomial{\hat\xi}$ также является аннулирущим до степени $k-1$ полиномом со степенью меньше, чем степень текущего полинома
$\polynomial{\xi_c}$, а это противоречит первому из условий \ref{conditions:MDAP:iteration_entrance}, поэтому:

	\begin{equation} \label{equation:MDAP:any_kth_annihilating_polynomial_monotony_threshold}
		\polynomialdegree{\hat\xi} \ge \polynomialdegree{\xi_c}.
	\end{equation}

Во-вторых, в варианте И.2 итерационного шага текущий полином является аннулирущим до степени $k-1$, но не является аннулирущим до степени
$k$, а полином $\polynomial{\hat\xi}$ является аннулирущим до степени $k$, поэтому в силу утверждения
\ref{statement:MDAP:annihilating_polynomial_degrees_inequality}:

	\begin{equation} \label{equation:MDAP:any_kth_annihilating_polynomial_statement_threshold}
		\polynomialdegree{\hat\xi} \ge k + 1 - \polynomialdegree{\xi_c}.
	\end{equation}

Таким образом из неравенств \eqref{equation:MDAP:any_kth_annihilating_polynomial_monotony_threshold} и
\eqref{equation:MDAP:any_kth_annihilating_polynomial_statement_threshold}:

	$$ \polynomialdegree{\hat\xi} \ge max \left \{ \polynomialdegree{\xi_c} , k + 1 - \polynomialdegree{\xi_c} \right \}. $$

Откуда с учетом равенства \eqref{equation:MDAP:final_difference_polynomial_degree_equality}:

	$$ \polynomialdegree{\hat\xi} \ge max \left \{ d_c , k + 1 - d_c \right \} = \polynomialdegree{\xi_d}, $$

и следовательно любой аннулирущий до степени $k$ полином из множества $\gfpunitypolynomial$ имеет степень большую или равную степени полинома
разности $\polynomialdegree{\xi_d}$, поэтому полином $\polynomialdegree{\xi_d}$ имеет минимальную степень среди аннулирующих до степени $k$
полиномов из множества $\gfpunitypolynomial$.

Далее в варианте И.2 итерационного шага текущий полином $\polynomial{\xi_c}$ становится полиномом разности $\polynomial{\xi_d}$, поэтому
выполняется первое из условий \ref{conditions:MDAP:iteration_entrance} перед началом следующего итерационного шага с числом $k$, увеличенным
на 1. Затем в качестве аннулирущего полинома $\polynomial{\xi^{(k)}}$ также используется полином разности $\polynomial{\xi_d}$, поэтому
построенный алгоритмом полином $\polynomial{\xi^{(k)}}$ оказывается аннулирущим до степени $k$ полиномом с минимальной степенью в множестве
$\gfpunitypolynomial$.

Остается лишь проверить выполнение второго из условий \ref{conditions:MDAP:iteration_entrance} по завершению варианта И.2 итерационного шага.

Если до присваиваний \eqref{equation:FPAP:previous_polynomial_assignment} и \eqref{equation:FPAP:current_polynomial_assignment} выполняется
условие $d_p + k - m \le d_c$, тогда предыдущий полином $\polynomial{\xi_p}$ не изменяется и его степень $d_p$ остается прежней, текущий
полином $\polynomial{\xi_c}$ заменяется на полиномом разности $\polynomial{\xi_d}$, но в силу выполнения неравенства $d_p + k - m \le d_c$
полином $\polynomial{\xi_d}$ согласно равенству \eqref{equation:MDAP:initial_difference_polynomial_degree_equality} имеет степень $d_c$,
поэтому сам текущий полином $\polynomial{\xi_c}$ изменяется, но степень его остается прежней и, следовательно, неравенство $d_p + k - m \le d_c$
выполняется и после присваиваний \eqref{equation:FPAP:previous_polynomial_assignment} и \eqref{equation:FPAP:current_polynomial_assignment}.

Если же до присваиваний \eqref{equation:FPAP:previous_polynomial_assignment} и \eqref{equation:FPAP:current_polynomial_assignment} выполняется
условие $d_p + k - m > d_c$, тогда предыдущий полином $\polynomial{\xi_p}$ заменяется на текущий полином $\polynomial{\xi_c}$ и степень
$d_p$ заменяется степенью $d_c$, сам текущий полином $\polynomial{\xi_c}$ заменяется на полином разности $\polynomial{\xi_d}$,
который теперь согласно равенству \eqref{equation:MDAP:final_difference_polynomial_degree_equality} имеет степень $k + 1 - d_c$, а число
$m$ --- наименьшая степень, коэффициент при которой в произведении $\polynomial{\alpha} \polynomial{\xi_p}$ не является нулевым, --- становится
равной $k$ (это та степень, коэффициент при которой в произведении $\polynomial{\alpha} \polynomial{\xi_c}$ не являлся нулевым
до замены текущего полинома $\polynomial{\xi_c}$ на полином разности $\polynomial{\xi_d}$). Таким образом

	$$
		\begin{array}{ccc}
			d_p & \leftarrow & d_c \\
			d_c & \leftarrow & k + 1 - d_c \\
			m   & \leftarrow & k \\
		\end{array}
	$$

Откуда следует выполнение равенства:

	$$ d_c = m + 1 - d_p, $$

после присваиваний \eqref{equation:FPAP:previous_polynomial_assignment} и \eqref{equation:FPAP:current_polynomial_assignment}.

Таким образом, после выполнения любого из вариантов И.1 либо И.2 итерационного шага выполняются условия \ref{conditions:MDAP:iteration_entrance}
перед выполнением следующего итерационного шага при увеличенном на единицу значении $k$, поэтому следующий аннулирущий полином
$\polynomial{\xi^{(k+1)}}$ так же будет иметь минимальную степень среди аннулирущих уже до степени $k+1$ полиномов из множества $\gfpunitypolynomial$.

Отсюда по индукции следует, что и вообще любой построенный алгоритмом полином $\polynomial{\xi^{(k)}}$ при любом $k$ имеет минимальную степень
среди всех аннулирущих до степени $k$ полиномов из множества $\gfpunitypolynomial$.


\chapter{Разное}

\section{Сравнение аннулирующих полиномов}

Характеристический полином матрицы не является единственным аннулирующим полиномом. Аннулирующие полиномы могут различаться степенью, и
чем меньше степень полинома, тем меньше требуется вычислений (векторов пространств Крылова), поэтому интерес представляют минимальные (по
степени) аннулирующие полиномы.

\section{Заметки}

Вообще говоря система \eqref{equation:system_for_polynomial_coefficients} представляет собой необходимое условие: если полином
$\polynomial{\varphi}$ является аннулирующим для матрицы $A$, то его коэффициенты удовлетворяют однородной системе
\eqref{equation:system_for_polynomial_coefficients}. Это необходимое условие по-видимому не является достаточным: если найден вектор
$ \left ( \tilde \varphi_0, \tilde \varphi_1, \tilde \varphi_2, ..., \tilde \varphi_d \right ) $, удовлетворяющей системе
\eqref{equation:system_for_polynomial_coefficients}, то полином $\polynomial{\tilde \varphi}$:

$$ \polynomial{\tilde \varphi} = \tilde \varphi_0 + \tilde \varphi_1\lambda + \tilde \varphi_2\lambda^2 + \ldots + \tilde \varphi_d\lambda^d $$

не обязательно является аннулирующим для $A$.

\begin{enumerate}
	\item При вычислении векторов пространства Крылова может оказаться, что $A^ix = 0$, а $A^{i-1}x \neq 0$, тогда $A^{i-1}x$ является
		нетривиальным решение и дальнейших вычислений не потребуется.
	\item Система для коэффициентов характеристического полинома является необходимым условием: 
\end{enumerate}

Во-первых, с целью упрощения придется отказаться от наименьшего количества $r$ векторов Крылова, поскольку $r$ не известно и вообще говоря
может оказаться разным для разных векторов $z$, и рассматривать $n+1$ векторов Крылова:

	$$ z, Az, A^2z, \dots, A^rz, \dots, A^nz, $$

поскольку $n+1$ векторов в пространстве $\gfpvector{n}$, имеющим размерность $n$, обязательно являются линейно зависимыми.

\begin{thebibliography}{4}
	\bibitem{Gantmacher} Гантмахер Ф. Р., Теория матриц / Изд. "Наука"{}, Москва, 1966.

	\bibitem{Massey} Massey J. L. Shift-Register Synthesis and BCH Decoding / IEEE Transactions on Information Theory,
		vol. IT-15, No. 1, January 1969.

	\bibitem{Montgomery} Montgomery P. A block Lanczos algorithm for finding dependencies over GF(2)

	\bibitem{Thome} Thome E. Fast Computation of Linear Generators for Matrix Sequences and Application to the Block Weidemann Algorithm /
		International Conference on Symbolic and Algebraic Computation, 2001.

	\bibitem{Wiedemann} Wiedemann D. H., Solving Sparse Linear Equations Over Finite Fields / Transactions of Information Theory,
		vol. IT-32, No. 1, January, 1986.

	\bibitem{Zamarashkin} Замарашкин Н. Л. Алгоритмы для разреженных систем линейных уравнений в $GF(2)$ / Москва: Издательство Московского
		Университета, 2013.
\end{thebibliography}



\end{document}
