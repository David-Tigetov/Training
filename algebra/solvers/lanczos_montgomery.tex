\section{Построение решения} \label{section:LM:finding_solution}

Данный раздел является расширением соответствующего раздела из книги \cite[с.~44--50]{Zamarashkin}.

Пусть задана матрица $A \in \gftwomatrix{n}{n}$ и вектор $b \in \gftwovector{n}$ и требуется найти решение системы:

$$
	Ax = b,
$$

где матрица $A$ является симметричной:

$$
	A = A^T,
$$

и вектор $b$ принадлежит линейной оболочке столбцов матрицы $A$.

Согласно методу Ланцоша-Монтгомери \cite{Montgomery} необходимо построить последовательность подпространств
$ \left \{ \mathcal W_i \right \} _{i=1}^m$ специального вида, прямая сумма которых образует подпространство $\mathcal W$:

$$
	\mathcal W = \mathcal W_1 + \mathcal W_2 + \dots \mathcal W_m.
$$

Пусть столбцы матриц $W_i$ являются векторами базиса соответствующих подпространств $\mathcal W_i$. Подпространства $\mathcal W$
и $\mathcal W_i$ должны удовлетворять следующим свойствам:

\begin{itemize}

	\item [M-1] апроксимация: вектор $b$ принадлежит линейной оболочке векторов из пространства $A \mathcal W$,
		$$
			b \in A \mathcal W;
		$$

	\item [M-2] А-обратимость подпространств $\mathcal W_i$: матрицы $W_i^T A W_i$ являются невырожденными,
		$$
			\determinant{W_i^T A W_i} \neq 0;
		$$

	\item [M-3] А-ортогональность подпространств $\mathcal W_i$: матрицы $W_i^T A W_j$ являются нулевыми,
		$$
			\begin{array}{c}
				W_i^T A W_j = 0, \\
				i \neq j; \\
			\end{array}
		$$

	\item [M-4] А-инвариантность: пространство $\mathcal W$ является инвариантным пространством матрицы $A$,
		$$
			A \mathcal W \subseteq \mathcal W.
		$$
\end{itemize}

Предположим, что пространства $\mathcal W_i$ и $\mathcal W$, удовлетворяющие свойствам M-1 -- M-4, построены. Постараемся в
пространстве $\mathcal W$ найти такой вектор $x$, чтобы невязка $Ax - b$ была орогональна пространству $W$:

\begin{equation} \label{equation:LM:FS:solution_conditions}
	\begin{array}{c}
	Ax - b \perp \mathcal W, \\
	x \in \mathcal W.
	\end{array}
\end{equation}

Поскольку $x \in \mathcal W$ и $\mathcal W$ является прямой суммой подпространств $\mathcal W_i$, то вектор $x$ раскладывает
по базисным векторам пространств $\mathcal W_i$:

\begin{equation} \label{equation:LM:FS:solution_representation}
	x = W_1 \alpha_1 + W_2 \alpha_2 + \dots + W_m \alpha_m = \sum_{i=1}^m W_i \alpha_i,
\end{equation}

где $\alpha_i$ --- векторы-столбцы коэффициентов.

Согласно условию \ref{equation:LM:FS:solution_conditions} невязка $Ax - b$ должна быть ортогональна пространству $\mathcal W$, следовательно
невязка должна быть ортогональна каждому из подпространств $\mathcal W_i$:

$$
	Ax - b \perp \mathcal W_i, \\
$$

Отсюда следует, что все скалярные произведения векторов базисов пространств $\mathcal W_i$ и вектора невязки $Ax-b$ должны быть равны нулю:

$$
	W_i^T ( Ax - b ) = 0, \\
$$

Откуда

$$
	W_i^T Ax = W_i^T b, \\
$$

Согласно представлению \ref{equation:LM:FS:solution_representation} из последнего равенства получим:

$$
	W_i^T A \sum_{i=1}^m W_i \alpha_i = W_i^T b, \\
$$

$$
	\sum_{j=1}^m W_i^T A W_j \alpha_j = W_i^T b,
$$

Согласно условию M-3 $W_i^T A W_j = 0$ при $i \neq j$, поэтому в сумме в правой части остается только одно слагаемое:

$$
	W_i^T A W_i \alpha_i = W_i^T b,
$$

В силу условия M-2 определитель $\determinant{W_i^T A W_i} \neq 0$, поэтому существует обратная матрица
$ \left ( W_i^T A W_i \right ) ^{-1} $:

$$
	\alpha_i = \left ( W_i^T A W_i \right ) ^{-1} W_i^T b,
$$

Подставляя полученные выражения для $\alpha_i$ в представление \ref{equation:LM:FS:solution_representation} получим:

\begin{equation} \label{equation:LM:FS:solution}
	x = \sum_{i=1}^m W_i \left ( W_i^T A W_i \right ) ^{-1} W_i^T b.
\end{equation}

Теперь покажем, что полученный вектор $x$ из соотношения \ref{equation:LM:FS:solution} является решением исходной системы $Ax = b$,
то есть невязка $Ax - b$ не только ортогональна пространству $\mathcal W$, но и равна нулю:

$$
	Ax - b \perp \mathcal W \Rightarrow Ax - b = 0.
$$

Прежде всего заметим, что $x \in \mathcal W$ по построению согласно условиям \ref{equation:LM:FS:solution_conditions} и в силу условия
M-4 $A \mathcal W \subseteq \mathcal W$, поэтому вектор $Ax$ является вектором из пространства $\mathcal W$:

$$
	Ax \in \mathcal W.
$$

Согласно условиям M-1 и M-4 $b \in A \mathcal W \subseteq \mathcal W$, поэтому вектор $b$ тоже принадлежит пространству $\mathcal W$:

$$
	b \in \mathcal W.
$$

Таким образом, невязка $Ax - b$ принадлежит пространству $\mathcal W$:

$$
	Ax - b \in \mathcal W.
$$

Отсюда следует, что существуют векторы коэффициентов $\beta_i$, такие что:

\begin{equation} \label{equation:LM:FS:residual_representation}
	Ax - b = W_1 \beta_1 + W_2 \beta_2 + \dots + W_m \beta_m = \sum_{j=1}^m W_j \beta_j.
\end{equation}

Домножим левую и правую части слева на $W_i^T A$:

$$
	W_i^T A ( Ax - b ) = W_i^T A \sum_{j=1}^m W_j \beta_j.
$$
$$
	W_i^T A ( Ax - b ) = \sum_{j=1}^m W_i^T A W_j \beta_j.
$$

Согласно условию M-2 $W_i^T A W_j = 0$ при $i \neq j$, поэтому в сумме в правой части остается только одно слагаемое:

$$
	W_i^T A ( Ax - b ) = W_i^T A W_i \beta_i.
$$

В левой части в силу симметричности матрицы $A$:

$$
	W_i^T A ( Ax - b ) = W_i^T A^T ( Ax - b ) = ( A W_i )^T ( Ax - b ).
$$

Последнее выражение является результатом скалярного умножения векторов-столбцов матрицы $A W_i$ и невязки $Ax - b$. В силу условия
M-4 $A \mathcal W \subseteq \mathcal W$, поэтому векторы-столбцы матрицы $A W_i$ являются векторами пространства $\mathcal W$,
а невязка $Ax - b$ по построению в силу условий \ref{equation:LM:FS:solution_conditions} ортогональна всем векторам пространства
$\mathcal W$ и следовательно все скалярные произведения равны нулю:

$$
	( A W_i )^T ( Ax - b ) = 0.
$$

Таким образом,

$$
	0 = ( A W_i )^T ( Ax - b ) = W_i^T A ( Ax - b ) = W_i^T A W_i \beta_i,
$$

и коэффициенты вектора $\beta_i$ удовлетворяют однородной системе с матрицей $W_i^T A W_i$:

$$
	W_i^T A W_i \beta_i = 0.
$$

Поскольку по условию M-3 определитель $\determinant{W_i^T A W_i} \ne 0$, то однородная система имеет только тривиальное решение и
следовательно вектор $\beta_i$ является нулевым:

$$
	\beta_i = 0.
$$

Таким образом, в разложении \ref{equation:LM:FS:residual_representation} все векторы $\beta_i$ являются нулевыми, поэтому невязка
$Ax - b$ является нулевым вектором:

$$
	Ax - b = W_1 \beta_1 + W_2 \beta_2 + \dots + W_m \beta_m = 0,
$$

и следовательно построенный вектор $x$ из соотношения \ref{equation:LM:FS:solution} является решением исходной системы:

$$
	\begin{array}{c}
		Ax - b = 0, \\
		Ax = b.
	\end{array}
$$

Остается только вопрос каким образом можно построить подпространства $\mathcal W_i$, удовлетворяющие условиям M-1 -- M-4, который 
рассматривается в следующем разделе.

\section{Построение подпространств} \label{section:LM:finding_subspaces}

Данный раздел является расширением соответствующего раздела из книги \cite[с.~44--50]{Zamarashkin}.

Один из возможных способов построения пространств $\mathcal W_i$ приводит в своей статье Монтгомери \cite{Montgomery}. Первым шагом
этого способа является формирование матрицы $V_1$ порядка $n \times n_b$, где $n_b$ - произвольное натурально число: в качестве одного
из столбцов матрицы $V_1$ необходимо взять вектор правой части $b$, а в качестве остальных столбцов выбрать произвольные ненулевые векторы.

Полученную матрицу $V_1$ нельзя использовать в качестве матрицы $W_1$, поскольку матрица $V_1^T A V_1$ может оказаться вырожденной и тем самым
будет нарушено условие M-2 для $W_1$. Тем не менее, в матрице $V_1$ можно выбрать наибольшее количество столбцов, из которых составить
матрицу $W_1$ так, чтобы матрица $W_1^T A W_1$ оказалась невырожденной. Из оставшихся столбцов, которые не вошли в состав $W_1$, необходимо
сформировать матрицу $\widehat{W}_1$:

$$
	V_1 =
		\begin{pmatrix}
			W_1 & \widehat{W}_1
		\end{pmatrix}
$$

Линейная оболочка векторов-столцов матрицы $W_1$ образует подпространство $\mathcal W_1$.

Для построения матрицы $W_2$, столбцы которой образуют векторы базиса пространства $\mathcal W_2$, необходимо сформировать матрицу
$\widehat{V}_2$ путем присоединения к матрице $\widehat{W}_1$ столбцов матрицы $A W_1$:

$$
	\widehat{V}_2 =
		\begin{pmatrix}
			\widehat{W}_1 & A W_1
		\end{pmatrix}
	.
$$

Векторы-столбцы матрицы $\widehat{V}_2$ необходимо сделать $A$-ортогональными подпространству $\mathcal W_1$, для этого нужно
из векторов-столбцов матрицы $\widehat{V}_2$ "вычесть их проекции"{} на пространство $\mathcal W_1$, то есть необходимо из столбцов матрицы
$\widehat{V}_2$ "вычесть"{} некоторую линейную комбинацию векторов-столбцов матрицы $W_1$, образующих базис подпространства
$\mathcal W_1$. Пусть матрица $V_2$ обозначает набор вектор-столбцов, полученных в результате $A$-ортогонализации векторов-столбцов
матрицы $\widehat{V}_2$:

\begin{equation} \label{equation:LM:FS:V_2_orthogonalization}
	V_2 = \widehat{V}_2 + W_1 \gamma_{2,1},
\end{equation}

где $\gamma_{2,1}$ --- искомая матрица коэффициентов, при которой:

$$
	W_1^T A V_2 = 0.
$$

Из последнего условия получим уравнение для матрицы $\gamma_{2,1}$:

$$
	\begin{array}{c}
		W_1^T A ( \widehat{V}_2 + W_1 \gamma_{2,1} ) = 0, \\
		W_1^T A \widehat{V}_2 + W_1^T A W_1 \gamma_{2,1} = 0, \\
		W_1^T A \widehat{V}_2 + W_1^T A \widehat{V}_2 + W_1^T A W_1 \gamma_{2,1} = W_1^T A \widehat{V}_2.
	\end{array}
$$

Поскольку в группе матриц с операцией сложения, порожденной сложением элементов в поле $\gftwo$, сама матрица является своей обратной
по сложению матрицей, то из последнего равенства следует:

$$
	\begin{array}{c}
	 	W_1^T A W_1 \gamma_{2,1} = W_1^T A \widehat{V}_2.
	\end{array}
$$

По построению матрицы $W_1$ матрица $W_1^T A W_1$ является невырожденной, поэтому существует обратная матрица
$\left ( W_1^T A W_1 \right ) ^{-1}$, умножая на которую слева левую и правую части последнего равенства получим:

$$
	\gamma_{2,1} = \left ( W_1^T A W_1 \right ) ^ {-1} W_1^T A \widehat{V}_2.
$$

Подставляя полученное выражение для $\gamma_{2,1}$ в соотношение \ref{equation:LM:FS:V_2_orthogonalization}, получим:

$$
	V_2 = \widehat{V}_2 + W_1 \left ( W_1^T A W_1 \right ) ^ {-1} W_1^T A \widehat{V}_2.
$$

Далее необходимо в матрице $V_2$ выбрать наибольшее количество столбцов, из которых сформировать матрицу $W_2$ таким образом, чтобы
матрица $W_2^T A W_2$ оказалась невырожденной, при этом необходимо чтобы все столбцы матрицы $\widehat{W}_1$ (после $A$-ортогонализации)
вошли в состав матрицы $W_2$. Оставшиеся столбцы, не вошедшие в состав матрицы $W_2$, необходимо поместить в матрицу $\widehat{W_2}$:

$$
	V_2 =
		\begin{pmatrix}
			W_2 & \widehat{W_2}
		\end{pmatrix}
	.
$$

Векторы-столбцы матрицы $W_2$ являются $A$-ортогональными векторам-столбцам $W_1$, поскольку все векторы-столбцы $V_2$
$A$-ортогональны векторам-столбцам $W_1$, и кроме того матрица $ W_2^T A W_2 $ является невырожденной, поэтому линейную оболочку
векторов-столбцов можно считать подпространством $\mathcal W_2$ и при этом выполняются условия M-2 и M-3.

Далее для построения матрицы $W_3$ аналогичным образом формируется матрица $\widehat{V}_3$:

$$
	\widehat{V}_3 =
		\begin{pmatrix}
		\widehat{W}_2 & A W_2 \\
		\end{pmatrix}
	.
$$

Вектор-столбцы матрицы $\widehat{V}_3$ необходимо сделать $A$-ортогональными подпространствам $\mathcal W_1$ и $\mathcal W_2$,
поэтому из матрицы $\widehat{V}_3$ необходимо "вычесть проекции"{} векторов-столбцов и на подпространство $\mathcal W_1$, и на
подпространство $\mathcal W_2$:

\begin{equation} \label{equation:LM:FS:V_3_orthogonalization}
	V_3 = \widehat{V}_3 + W_1 \gamma_{3,1} + W_2 \gamma_{3,2}.
\end{equation}

Из требования $A$-ортогональности подпространству $W_1$ следует:

$$
	\begin{array}{c}
		W_1^T A V_3 = 0, \\
 		W_1^T A ( \widehat{V}_3 + W_1 \gamma_{3,1} + W_2 \gamma_{3,2} ) = 0, \\
 		W_1^T A \widehat{V}_3 + W_1^T A W_1 \gamma_{3,1} + W_1^T A W_2 \gamma_{3,2} = 0.
	\end{array}
$$

В силу $A$-ортогональности подпространств $\mathcal W_1$ и $\mathcal W_2$ матрица $W_1^T A W_2$ является нулевой:

$$
	W_1^T A W_2 = 0.
$$

Таким образом,

$$
	\begin{array}{c}
 		W_1^T A \widehat{V}_3 + W_1^T A W_1 \gamma_{3,1} = 0, \\
 		W_1^T A W_1 \gamma_{3,1} = W_1^T A \widehat{V}_3, \\
 		\gamma_{3,1} = \left ( W_1^T A W_1 \right ) ^{-1} W_1^T A \widehat{V}_3. \\
	\end{array}
$$

Аналогиным образом, из требования $A$-ортогональности подпространству $W_2$ следует:

$$
	\begin{array}{c}
		W_2^T A V_3 = 0, \\
 		W_2^T A ( \widehat{V}_3 + W_1 \gamma_{3,1} + W_2 \gamma_{3,2} ) = 0, \\
 		W_2^T A \widehat{V}_3 + W_2^T A W_1 \gamma_{3,1} + W_2^T A W_2 \gamma_{3,2} = 0.
	\end{array}
$$

В силу $A$-ортогональности подпространств $\mathcal W_1$ и $\mathcal W_2$ матрица $W_2^T A W_1$ является нулевой:

$$
	W_2^T A W_1 = 0,
$$

следовательно,

$$
	\begin{array}{c}
 		W_2^T A \widehat{V}_3 + W_2^T A W_2 \gamma_{3,2} = 0, \\
 		W_2^T A W_2 \gamma_{3,2} = W_2^T A \widehat{V}_3, \\
 		\gamma_{3,2} = \left ( W_2^T A W_2 \right ) ^{-1} W_2^T A \widehat{V}_3. \\
	\end{array}
$$

Подставляя полученные выражения для $\gamma_{3,1}$ и $\gamma_{3,2}$ в соотношение \ref{equation:LM:FS:V_3_orthogonalization}, получим:

$$
	V_3 = \widehat{V}_3 + \sum_{i=1}^2 W_i \left ( W_i^T A W_i \right ) ^{-1} W_i^T A \widehat{V}_3.
$$

Далее в матрице $V_3$ необходимо найти наибольшее количество столбцов, из которых сформировать матрицу $W_3$ так, чтобы матрица
$W_3^T A W_3$ оказалась невырожденной, причем опять же все столбцы матрицы $\widehat{W}_2$ (после $A$-ортогонализации) обязательно должны
войти в состав матрицы $W_3$. Оставшиеся столбцы помещаются в матрицу $\widehat{W}_3$:

$$
	V_3 =
		\begin{pmatrix}
			W_3 & \widehat{W}_3
		\end{pmatrix}
$$

По построению вектор-столбцы матрицы $W_3$ являются $A$-ортогональными вектор-столбцам матриц $W_1$ и $W_2$ и матрица $W_3^T A W_3$
является невырожденной, поэтому линейная оболочка вектор-столбцов $W_3$ можно считать подпространством $\mathcal W_3$ и при этом выполняются
свойства M-2 и M-3.

Построение подпространств необходимо продолжать аналогичным образом. На шаге $s$ с предыдущего шага $s-1$ будет получена матрица
$\widehat{V}_s$:

$$
	\widehat{V}_s =
		\begin{pmatrix}
			\widehat{W}_{s-1} & A W_{s-1}
		\end{pmatrix}
$$

В результате $A$-ортогонализации столбцов матрицы $\widehat{V}_s$ будет получена матрица $V_s$:

\begin{equation} \label{equation:LM:FS:V_s_orthogonalization}
	V_s = \widehat{V}_s + \sum_{i=1}^{s-1} W_i \left ( W_i^T A W_i \right ) W_i^T A \widehat{V}_s.
\end{equation}

Можно заметить, что в сумме, стоящей справа, отличными от нулевых матриц могут оказаться только три "последних"{} слагаемых
с номерами $s-1$, $s-2$ и $s-3$. Действительно, рассмотрим произведение $W_i^T A \widehat{V}_s$ для произвольного номера $i < s-3$:

$$
	W_i^T A \widehat{V}_s
		= W_i^T A \begin{pmatrix} \widehat{W}_{s-1} & A W_{s-1} \end{pmatrix}
		= \begin{pmatrix} W_i^T A \widehat{W}_{s-1} & W_i^T A A W_{s-1} \end{pmatrix}.
$$

Столбцы матрицы $\widehat{W}_{s-1}$ входят в состав столбцов матрицы $V_{s-1}$, которые по построению являются $A$-ортогональными всем
столбцам матриц $W_i$ с номерами $i = s-2, s-3, \dots, 1$, поэтому первое произведение является нулевой матрицей:

$$
	W_i^T A \widehat{W}_{s-1} = 0.
$$

Второе произведение удобней рассматривать в транспонированном виде:

$$
	\left ( W_i^T A A W_{s-1} \right ) ^T = W_{s-1}^T A A W_i,
$$

поскольку матрица $A$ является симметричной ($A^T = A$). Столбцы матрицы $A W_i$ совместно со столбцами матрицы $\widehat{W}_i$
образуют матрицу $\widehat{V}_{i+1}$, подвергаются $A$-ортогонализации, в результате которой получается матрица $V_{i+1}$,
из столбцов которой формируются две матрицы $W_{i+1}$ и $\widehat{W}_{i+1}$:

$$
	\begin{pmatrix}
		W_{i+1} & \widehat{W}_{i+1}
	\end{pmatrix}
	P_{i+1}
	=
	\begin{pmatrix}
		A W_i & \widehat{W}_i
	\end{pmatrix}
	+
	\sum_{j=1}^i W_j \left ( W_j^T A W_j \right ) ^{-1} W_j^T \begin{pmatrix}	A W_i & \widehat{W}_i \end{pmatrix},
$$

где $P_{i+1}$ --- некоторая перестановочная матрица, выполняющая перестановку столбцов матрицы
$\begin{pmatrix} W_{i+1} & \widehat{W}_{i+1} \end{pmatrix}$. Полученное соотношение можно записать в ином виде:

$$
	\begin{pmatrix}
		A W_i & \widehat{W}_i
	\end{pmatrix}
	=
	\begin{pmatrix}
		W_{i+1} & \widehat{W}_{i+1}
	\end{pmatrix}
	P_{i+1}
	+
	\sum_{j=1}^i W_j \left ( W_j^T A W_j \right ) ^{-1} W_j^T \begin{pmatrix} A W_i & \widehat{W}_i \end{pmatrix},
$$

из которого видно, что столбцы матрицы $A W_i$ являются линейной комбинацией столбцов матриц $W_{i+1}$, $W_i$, \dots, $W_1$ и матрицы
$\widehat{W}_{i+1}$.

Матрица $\widehat{W}_{i+1}$ совместно с матрицей $A W_{i+1}$ образует матрицу $\widehat{V}_{i+2}$, над столцами которой будет выполнена
$A$-ортогонализация и формирование двух матриц $W_{i+2}$ и $\widehat{W}_{i+2}$, поэтому для матриц $A W_{i+1}$ и $\widehat{W}_{i+1}$
справедливо аналогичное соотношение:

$$
	\begin{pmatrix}
		A W_{i+1} & \widehat{W}_{i+1}
	\end{pmatrix}
	=
	\begin{pmatrix}
		W_{i+2} & \widehat{W}_{i+2}
	\end{pmatrix}
	P_{i+2}
	+
	\sum_{j=1}^{i+1} W_j \left ( W_j^T A W_j \right ) ^{-1} W_j^T \begin{pmatrix} A W_{i+1} & \widehat{W}_{i+1} \end{pmatrix}
	.
$$

В соответствии со способом формирования матрицы $W_{i+2}$ все столбцы матрицы $\widehat{W}_{i+1}$ (после $A$-ортогонализации) входят в
состав матрицы $W_{i+2}$, поэтому столбцы матрицы $\widehat{W}_{i+1}$ являются линейной комбинацией столбцов матриц $W_{i+2}$,
$W_{i+1}$, \dots, $W_1$, но не матрицы $\widehat{W}_{i+2}$.

Таким образом, столбцы матрицы $A W_i$ являются линейной комбинацией столбцов матриц $W_{i+2}$, $W_{i+1}$, \dots, $W_1$. Поскольку номер
$i < s-3$, то $i+2 < s-1$ и матрица $W_{s-1}$ имеет номер $s-1$ превосходящий номера $i+2$, $i+1$, \dots, $1$, поэтому матрица
$W_{s-1}$ по построению $A$-ортогональна матрицам $W_{i+2}$, $W_{i+1}$, \dots, $W_1$ и следовательно матрица $W_{s-1}$
$A$-ортогональна и матрице $A W_i$, столбцы которой являются линейной комбинацией столбцов матриц $W_{i+2}$, $W_{i+1}$, \dots, $W_1$:

$$
	\left ( W_i^T A A W_{s-1} \right ) ^T = W_{s-1}^T A A W_i = 0,
$$

для всех $i < s-3$.

Отсюда следует, что выражение \ref{equation:LM:FS:V_s_orthogonalization} имеет более короткий вид:

$$
	\begin{array}{ccl}
		V_s & = & \widehat{V}_s + \\
	        &   & + W_{s-1} \left ( W_{s-1}^T A W_{s-1} \right ) ^{-1} W_{s-1}^T A \widehat{V}_s + \\
	        &   & + W_{s-2} \left ( W_{s-2}^T A W_{s-2} \right ) ^{-1} W_{s-2}^T A \widehat{V}_s + \\
	        &   & + W_{s-3} \left ( W_{s-3}^T A W_{s-3} \right ) ^{-1} W_{s-3}^T A \widehat{V}_s.
	\end{array}
$$

Далее из столбцов матрицы $V_s$ набирается матрица $W_s$ так, чтобы матрица $W_s^T A W_s$ была невырожденной, а остальным столбцы образуют
матрицу $\widehat{W}_s$.

Процесс построения матриц $W_s$, вектор-столбцы которых являются базисами соответствующих подпространст $\mathcal W_s$, завершается 
на некотором шаге $s$ в двух случаях:

\begin{itemize}
	\item некоторые столбцы из матрицы $\widehat{W}_{s-1}$ "не попали"{} в матрицу $W_s$;
	\item матрица $V_s$ оказалась нулевой.
\end{itemize}

В первом случае решение возможно построить не удастся, а во втором случае решение будет найдено.
