\section{Решение систем уравнений над полем $\gfp$}

\subsection{Неоднородная система с невырожденной матрицей}\label{subsection:SAP:nonuniform}

Пусть требуется найти решение неоднородной системы:

	\begin{equation} \label{equation:SAP:nonuniform_system}
		A x = b,
	\end{equation}
	$$ \left| A \right| \neq 0, $$

где $A \in \gfpmatrix{n}{n}$ и $b \in \gfpvector{n}$.

Представим, что некоторым образом найден аннулирующий полином $\polynomial{\varphi}$ вектора $b$ степени $d$:

	$$ \varphi(\lambda) =
		\varphi_0 + \varphi_1\lambda + \varphi_2\lambda^2 + \ldots + \varphi_d\lambda^d = \sum_{i=0}^d \varphi_i\lambda^i, $$

в котором

	$$ \varphi_0 \neq 0. $$

Поскольку полином $\polynomial{\varphi}$ является аннулирующим для $b$, то:

	$$ \varphi(A)b = 0, $$
	$$ \varphi_0b + \varphi_1Ab + \varphi_2A^2b + \ldots + \varphi_dA^db = 0, $$
	$$ \varphi_0b = -\varphi_1Ab - \varphi_2A^2b - \ldots - \varphi_dA^db, $$

Поскольку коэффициент $ \varphi_0 \neq 0 $, то в поле $\gfp$ существует обратный по умножению элемент $\varphi_0^{-1}$, тогда:

	$$ b = \varphi_0^{-1} \left(-\varphi_1Ab - \varphi_2A^2b - \ldots - \varphi_dA^db \right), $$
	$$ b = \varphi_0^{-1} A \left(-\varphi_1b - \varphi_2Ab - \ldots - \varphi_dA^{d-1}b \right), $$
	$$ b = A \left\{ -\varphi_0^{-1} \left(\varphi_1b + \varphi_2Ab + \ldots + \varphi_dA^{d-1}b \right) \right\}, $$
	$$ b = A \left\{ -\varphi_0^{-1} \sum_{i=1}^d \varphi_iA^{i-1}b \right\}, $$

откуда следует, что решением системы \eqref{equation:SAP:nonuniform_system} является вектор:

	$$ x=-\varphi_0^{-1} \sum_{i=1}^d \varphi_iA^{i-1}b, $$

который принадлежит пространству Крылова $\mathcal K_d ( b, A)$.

Таким образом, для нахождения решения системы~\eqref{equation:SAP:nonuniform_system} достаточно найти аннулирующий полином
$\polynomial{\varphi}$ вектора $b$ с коэффициентом $\varphi_0 \neq 0$. Одним из таких полиномов является например минимальный аннулирущий
полином $\polynomial{\tilde \varphi}$ вектора $b$, поскольку из условия $\left | A \right | \neq 0$ согласно утверждению
\ref{statement:APD:determinant_and_minimal_vector_annihilating_polynomial} следует $\tilde \varphi_0 \neq 0$.

\subsection{Однородная система с вырожденной матрицей}\label{subsection:SAP:uniform}

Пусть требуется найти нетривиальное решение однородной системы:

	\begin{equation}\label{equation:SAP:uniform_system}
		A x = 0,
	\end{equation}
	$$ \left | A \right | = 0, $$

где $A \in \gfpmatrix{n}{n}$.

\subsubsection{Метод нахождения решения}

Выберем произвольным образом вектор $z \in \gfpvector{n}$ и представим, что некоторым образом найден аннулирущий полином
$\polynomial{\varphi}$ вектора z:

	$$ \varphi(\lambda) =
		\varphi_k \lambda^k + \varphi_{k+1} \lambda^{k+1} + \ldots + \varphi_d \lambda^d = \sum_{i=k}^d \varphi_i\lambda^i, $$

в котором

	$$ k \ge 1. $$

Поскольку полином $\polynomial{\varphi}$ является аннулирующим для вектора $z$, то

	$$ \varphi(A) z = 0, $$
	$$ \left ( \varphi_k A^k + \varphi_{k+1} A^{k+1} + \ldots + \varphi_d A^d \right ) z = 0, $$
	$$ A^k \left ( \varphi_k + \varphi_{k+1} A + \ldots + \varphi_d A^{d-k} \right ) z = 0. $$

Обозначим в качестве $\polynomial{\varphi_k}$ полином, стоящий в скобках,

	$$ \varphi_k(\lambda) = \varphi_k + \varphi_{k+1} \lambda + \ldots + \varphi_d \lambda^{d-k}, $$

тогда

	\begin{equation} \label{equation:SAP:vector_z_annihilation}
		A^k \varphi_k(A) z = 0.
	\end{equation}

Если вектор $\varphi_k(A) z$ является нулевым, то, к сожалению, решение системы \eqref{equation:SAP:uniform_system} найти не удастся, и
придется повторить попытку с другим вектором $z$ и аннулирущим полиномом $\polynomial{\varphi}$.

Если вектор $\varphi_k(A) z$ не является нулевым, то необходимо вычислять векторы $A^s \varphi_k(A) z$ для $s=1,\dots,k$, до тех пор пока
не будет получен нулевой вектор. Рано или поздно нулевой вектор будет получен, поскольку по крайней мере для $s=k$ имеет место равенство
\eqref{equation:SAP:vector_z_annihilation}. Предположим, что при некотором наименьшем возможном $s$ имеют место равенства:

	$$ A^{s-1} \varphi_k(A) z \neq 0 $$
	$$ A^s \varphi_k(A) z = 0 $$

тогда из последнего равенства

	$$ A \left \{ A^{s-1} \varphi_k(A) z \right \} = 0 $$

и следовательно ненулевой вектор $x$:

	$$ x = A^{s-1} \varphi_k(A) z \neq 0 $$

является нетривиальным решением системы \eqref{equation:SAP:uniform_system}.

Легко видеть, что в данном случае вектор--решение $x$ принадлежит пространству Крылова $\mathcal K_{d-k+s}(z,A)$:

	$$ x = A^{s-1} \sum_{i=k}^d \varphi_i A^{i-k} z = \sum_{i=k}^d \varphi_i A^{i-k+s-1} z $$

\subsubsection{Анализ метода нахождения решения}

Успех в нахождении нетривиального решения системы \eqref{equation:SAP:uniform_system} зависит от того, удастся ли найти вектор $z$ и
аннулирущий полином $\polynomial{\varphi}$ такие, что:

	$$ \varphi(A) z = 0, $$
	$$ \polynomial{\varphi} =
		\lambda^k \left ( \varphi_k + \varphi_{k+1} \lambda + \ldots + \varphi_d \lambda^{d-k} \right ), $$
	$$ \left ( \varphi_k + \varphi_{k+1} A + \ldots + \varphi_d A^{d-k} \right ) z \neq 0 $$

Одним из полиномов $\polynomial{\varphi}$, удовлетворяющим все трем условиям для некоторых векторов $z$, является в частности
минимальный аннулирущий полином $\polynomial{\tilde\psi}$ пространства $\gfpvector{n}$. Действительно, поскольку 
$\polynomial{\tilde\psi}$ является аннулирущим полиномом пространства $\gfpvector{n}$, то

	$$ \tilde\psi(A) z = 0. $$

Далее, согласно утверждению \ref{statement:APD:space:zero_determinant_and_space_annihilating_polynomials} полином $\polynomial{\tilde\psi}$ имеет вид:

	$$ \tilde\psi(\lambda) = \tilde\psi_k \lambda^k + \tilde\psi_{k+1} \lambda^{k+1} + \ldots + \tilde\psi_d \lambda^d. $$

И наконец полином $\polynomial{\tilde\psi_k}$, получаемый вынесением множителя $\lambda^k$:

	$$ \tilde\psi_k(\lambda) = \tilde\psi_k + \tilde\psi_{k+1} \lambda + \dots + \tilde\psi_d \lambda^{d-k} $$

уже не является аннулирущим полиномом пространства $\gfpvector{n}$. Действительно, степень полинома $\polynomial{\tilde\psi_k}$
равна $d-k < d$, и если предположить, что полином $\polynomial{\tilde\psi_k}$ является аннулирущим полиномом пространства $\gfpvector{n}$,
то это предположение противоречит тому, что полином $\polynomial{\tilde\psi}$ является минимальным аннулирущим полиномом
пространства $\gfpvector{n}$.

Поскольку полином $\polynomial{\tilde\psi_k}$ не является аннулирущим для всего пространства $\gfpvector{n}$, то найдется и ненулевой
вектор $z$ такой, что

	$$ \tilde\psi_k(A) z \neq 0. $$

Отсюда следует, что размерность образа оператора $\tilde\psi_k(\mathcal A)$ не меньше 1. Размерность всего поля векторов $\gfpvector{n}$
равна $n$, и по известной теореме о сумме размерностей ядра и образа, ядро оператора $\tilde\psi_k(\mathcal A)$ имеет размерность не более,
чем $n-1$. В пространстве $\gfpvector{n}$ всего $p^n$ различных векторов, поскольку размерность ядра $\tilde\psi_k(\mathcal A)$ не более $n-1$,
то в ядре оператора $\tilde\psi_k(\mathcal A)$ не более $p^{n-1}$ векторов.

Если вектор $z$ выбирается случайным образом (с равномерным распределением), то вектор $z$ оказывается вектором из ядра оператора
$\tilde\psi_k(\mathcal A)$ с вероятностью не более $\frac{p^{n-1}}{p^n} = \frac{1}{p}$. Вероятность выбрать $r > 1$ векторов из ядра
оказывается не больше, чем $\left ( \frac{1}{p} \right ) ^r$, а вероятность того, что среди $r$ векторов окажется хотя бы один вектор не
из ядра оператора $\tilde\psi_k(\mathcal A)$ не меньше, чем $1 - \frac{1}{p^r}$.

Таким образом, за $r$ попыток можно расчитывать на получение нетривиального решения с вероятностью не менее, чем $1 - \frac{1}{p^r}$.
