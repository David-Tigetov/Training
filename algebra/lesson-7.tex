\documentclass[12pt]{article}

\usepackage[T1]{fontenc}
\usepackage[utf8]{inputenc}
\usepackage[english,russian]{babel}
\usepackage[margin=2cm]{geometry}
\usepackage{amsmath}
\usepackage{amsfonts}
\usepackage{xcolor}
\usepackage{color}

% команды вывода первой частной производной
\newcommand{\fpd}[1]{\frac{\partial}{\partial #1}}
\newcommand{\fpda}[2]{\frac{\partial #1}{\partial #2}}
\newcommand{\fpdp}[2]{\fpd{#2} \left ( #1 \right )}

\newcommand{\expectation}[1]{\mathtt{M} \left [ #1 \right ]}
\newcommand{\conditionalexpectation}[2]{\expectation{ #1 \left | #2 \right .}}
\newcommand{\variance}[1]{\mathtt{D} \left [ #1 \right ]}
\newcommand{\covariance}[2]{\mathtt{cov} \left ( #1, #2 \right )}

\newcommand{\modulus}[1]{\left | #1 \right |}
\newcommand{\norm}[1]{\left \| {#1} \right \|}

\newcommand{\event}[1]{\left \{ #1 \right \} }
\newcommand{\probability}[1]{P \event{#1}}


\begin{document}

    \title{Задача 7}
    \author{Тигетов Давид Георгиевич}
    \date{}
    \maketitle

    \section*{Пункт а}
    Найти ортогональную проекцию и ортогональную составляющую вектора $x$ при проекции на подпространство $V = \left < a, b, c \right >$
    \[
        x =
        \begin{pmatrix}
            2 \\ 5 \\ -2 \\ 3
        \end{pmatrix},
        a =
        \begin{pmatrix}
            -3 \\ 1 \\ 1 \\ 1
        \end{pmatrix},
        b =
        \begin{pmatrix}
            1 \\ 0 \\ 1 \\ 0
        \end{pmatrix},
        c =
        \begin{pmatrix}
            -1 \\ 0 \\ 1 \\ 1
        \end{pmatrix} .
    \]

    \subsection*{Решение:}
    Пусть
    \[
        \projection{V}{x} = y_1 a + y_2 b + y_3 c,
    \]
    тогда
    \begin{gather*}
        \begin{pmatrix}
            a^T \\
            b^T \\
            c^T
        \end{pmatrix}
        \projection{V}{x}
        =
        \begin{pmatrix}
            a^T \\
            b^T \\
            c^T
        \end{pmatrix}
        x , \\
        %
        \begin{pmatrix}
            a^T \\
            b^T \\
            c^T
        \end{pmatrix}
        \begin{pmatrix}
            a & b & c
        \end{pmatrix}
        \begin{pmatrix}
            y_1 \\
            y_2 \\
            y_3
        \end{pmatrix}
        =
        \begin{pmatrix}
            a^T \\
            b^T \\
            c^T
        \end{pmatrix}
        x , \\
        %
        \begin{pmatrix}
            \scalarproduct{a}{a} & \scalarproduct{a}{b} & \scalarproduct{a}{c} \\
            \scalarproduct{b}{a} & \scalarproduct{b}{b} & \scalarproduct{b}{c} \\
            \scalarproduct{c}{a} & \scalarproduct{c}{b} & \scalarproduct{c}{c} \\
        \end{pmatrix}
        \begin{pmatrix}
            y_1 \\
            y_2 \\
            y_3
        \end{pmatrix}
        =
        \begin{pmatrix}
            \scalarproduct{a}{x} \\
            \scalarproduct{b}{x} \\
            \scalarproduct{c}{x}
        \end{pmatrix}
    \end{gather*}
    Вычислим скалярные произведения:
    \begin{align*}
        \scalarproduct{a}{a} & = 9 + 1 + 1 + 1 = 12, \\
        \scalarproduct{a}{b} & = -3 + 1 = -2, \\
        \scalarproduct{a}{c} & = 3 + 1 + 1 = 5, \\
        \scalarproduct{b}{b} & = 1 + 1 = 2, \\
        \scalarproduct{c}{b} & = -1 + 1 = 0, \\
        \scalarproduct{c}{c} & = 1 + 1 + 1 = 3, \\
        \scalarproduct{a}{x} & = -6 + 5 - 2 + 3 = 0, \\
        \scalarproduct{b}{x} & = 2 - 2 = 0, \\
        \scalarproduct{c}{x} & = -2 - 2 + 3 = -1 .
    \end{align*}
    Таким образом, необходимо решить СЛУ
    \begin{gather*}
        \begin{pmatrix}
            12 & -2 & 5 \\
            -2 & 2  & 0 \\
            5  & 0  & 3
        \end{pmatrix}
        \begin{pmatrix}
            y_1 \\
            y_2 \\
            y_3
        \end{pmatrix}
        =
        \begin{pmatrix}
            0 \\
            0 \\
            -1
        \end{pmatrix}, \\
        %
        \begin{pmatrix}
            10 & 0 & 5 \\
            -2 & 2 & 0 \\
            5  & 0 & 3
        \end{pmatrix}
        \begin{pmatrix}
            y_1 \\
            y_2 \\
            y_3
        \end{pmatrix}
        =
        \begin{pmatrix}
            0 \\
            0 \\
            -1
        \end{pmatrix}, \\
        %
        \begin{pmatrix}
            2  & 0 & 1 \\
            -2 & 2 & 0 \\
            5  & 0 & 3
        \end{pmatrix}
        \begin{pmatrix}
            y_1 \\
            y_2 \\
            y_3
        \end{pmatrix}
        =
        \begin{pmatrix}
            0 \\
            0 \\
            -1
        \end{pmatrix}, \\
        %
        \begin{pmatrix}
            2  & 0 & 1 \\
            -2 & 2 & 0 \\
            -1 & 0 & 0
        \end{pmatrix}
        \begin{pmatrix}
            y_1 \\
            y_2 \\
            y_3
        \end{pmatrix}
        =
        \begin{pmatrix}
            0 \\
            0 \\
            -1
        \end{pmatrix}, \\
        %
        \begin{pmatrix}
            y_1 \\
            y_2 \\
            y_3
        \end{pmatrix}
        =
        \begin{pmatrix}
            1 \\
            1 \\
            -2
        \end{pmatrix}
    \end{gather*}
    Откуда
    \begin{gather*}
        \projection{V}{x}
        =
        \begin{pmatrix}
            -3 + 1 + 2 \\
            1 + 0 + 0  \\
            1 + 1 - 2  \\
            1 + 0 - 2
        \end{pmatrix}
        =
        \begin{pmatrix}
            0 \\
            1 \\
            0 \\
            -1
        \end{pmatrix}, \\
        %
        \perpendicular{V}{x}
        = x - \projection{V}{x}
        =
        \begin{pmatrix}
            2 - 0  \\
            5 - 1  \\
            -2 - 0 \\
            3 + 1
        \end{pmatrix}
        =
        \begin{pmatrix}
            2  \\
            4  \\
            -2 \\
            4
        \end{pmatrix}
    \end{gather*}

    \subsection*{Ответ:}
    $
    \perpendicular{V}{x}
    =
    \begin{pmatrix}
        2  \\
        4  \\
        -2 \\
        4
    \end{pmatrix},
    %
    \projection{V}{x}
    =
    \begin{pmatrix}
        0 \\
        1 \\
        0 \\
        -1
    \end{pmatrix}
    $.

    \section*{Пункт б}
    Найти расстояние и угол между вектором $x$ и подпространством $V = \left < a, b \right >$
    \[
        x =
        \begin{pmatrix}
            4 \\ -8 \\ 0 \\ 1
        \end{pmatrix},
        a =
        \begin{pmatrix}
            -1 \\ 1 \\ 2 \\ 3
        \end{pmatrix},
        b =
        \begin{pmatrix}
            2 \\ 0 \\ 1 \\ 1
        \end{pmatrix},
    \]

    \subsection*{Решение:}
    Пусть
    \[
        \projection{V}{x} = y_1 a + y_2 b,
    \]
    тогда
    \begin{gather*}
        \begin{pmatrix}
            \scalarproduct{a}{a} & \scalarproduct{a}{b} \\
            \scalarproduct{b}{a} & \scalarproduct{b}{b}
        \end{pmatrix}
        \begin{pmatrix}
            y_1 \\
            y_2
        \end{pmatrix}
        =
        \begin{pmatrix}
            -4 - 8 + 3 \\
            8 + 1
        \end{pmatrix}, \\
        %
        \begin{pmatrix}
            1 + 1 + 4 + 9 & -2 + 2 + 3 \\
            -2 + 2 + 3    & 4 + 1 + 1
        \end{pmatrix}
        \begin{pmatrix}
            y_1 \\
            y_2
        \end{pmatrix}
        =
        \begin{pmatrix}
            -4 - 8 + 3 \\
            8 + 1
        \end{pmatrix}, \\
        %
        \begin{pmatrix}
            15 & 3 \\
            3  & 6
        \end{pmatrix}
        \begin{pmatrix}
            y_1 \\
            y_2
        \end{pmatrix}
        =
        \begin{pmatrix}
            -9 \\
            9
        \end{pmatrix}, \\
        %
        \begin{pmatrix}
            5 & 1 \\
            1 & 2
        \end{pmatrix}
        \begin{pmatrix}
            y_1 \\
            y_2
        \end{pmatrix}
        =
        \begin{pmatrix}
            -3 \\
            3
        \end{pmatrix}, \\
        %
        \begin{pmatrix}
            6 & 3 \\
            1 & 2
        \end{pmatrix}
        \begin{pmatrix}
            y_1 \\
            y_2
        \end{pmatrix}
        =
        \begin{pmatrix}
            0 \\
            3
        \end{pmatrix}, \\
        %
        \begin{pmatrix}
            2 & 1 \\
            1 & 2
        \end{pmatrix}
        \begin{pmatrix}
            y_1 \\
            y_2
        \end{pmatrix}
        =
        \begin{pmatrix}
            0 \\
            3
        \end{pmatrix}, \\
        %
        \begin{pmatrix}
            y_1 \\
            y_2
        \end{pmatrix}
        =
        \begin{pmatrix}
            -1 \\
            2
        \end{pmatrix}.
    \end{gather*}
    Таким образом,
    \begin{gather*}
        \projection{V}{x}
        = -a + 2 b
        =
        \begin{pmatrix}
            1 + 4  \\
            -1 + 0 \\
            -2 + 2 \\
            -3 + 2
        \end{pmatrix}
        =
        \begin{pmatrix}
            5  \\
            -1 \\
            0  \\
            -1
        \end{pmatrix}, \\
        %
        \perpendicular{V}{x}
        = x - \projection{V}{x}
        =
        \begin{pmatrix}
            4 - 5  \\
            -8 + 1 \\
            0 - 0  \\
            1 + 1
        \end{pmatrix}
        =
        \begin{pmatrix}
            -1 \\
            -7 \\
            0  \\
            2
        \end{pmatrix}
    \end{gather*}
    Расстояние
    \[
        \modulus{\perpendicular{V}{x}}
        = \sqrt{1 + 49 + 4}
        = \sqrt{54}
        = 3 \sqrt{6}
    \]
    и угол
    \begin{multline*}
        \angle \left ( x, V \right )
        = \arccos \frac{\scalarproduct{x}{\projection{V}{x}}}{\modulus{x} \cdot \modulus{\projection{V}{x}}}
        = \arccos \frac{20 + 8 - 1}{\sqrt{16 + 64 + 1} \cdot \sqrt{25 + 1 + 1}}
        = \arccos \frac{27}{\sqrt{81} \cdot \sqrt{27}} = \\
        %
        = \arccos \sqrt{\frac{27}{81}}
        = \arccos \frac{1}{\sqrt{3}}.
    \end{multline*}

    \subsection*{Ответ:}
    $\modulus{\perpendicular{V}{x}} = 3 \sqrt{6}, \angle \left ( x, V \right ) = \arccos \frac{1}{\sqrt{3}}$.

    \section*{Пункт в}
    Найти расстояние между вектором $x$ и подпространством $V$, заданным ОСЛУ:
    \[
        x =
        \begin{pmatrix}
            -1 \\ 3 \\ -3 \\ 5
        \end{pmatrix}, \\
        V:
        \begin{pmatrix}
            1 & 2  & 1 & 1  \\
            5 & -2 & 1 & -9
        \end{pmatrix}
        \begin{pmatrix}
            x_1 \\ x_2 \\ x_3 \\ x_4
        \end{pmatrix}
        = 0 .
    \]

    \subsection*{Решение:}
    Пусть
    \[
        a =
        \begin{pmatrix}
            1 \\ 2 \\ 1 \\ 1
        \end{pmatrix},
        b =
        \begin{pmatrix}
            5 \\ -2 \\ 1 \\ -9
        \end{pmatrix}.
    \]
    Заметим, что $\left < a, b \right > = V^\perp$, поэтому
    \[
        \perpendicular{V}{x} = \projection{V^\perp}{x} = y_1 a + y_2 b,
    \]
    где $y_1$ и $y_2$ удовлетворяют СЛУ:
    \begin{gather*}
        \begin{pmatrix}
            \scalarproduct{a}{a} & \scalarproduct{a}{b} \\
            \scalarproduct{b}{a} & \scalarproduct{b}{b}
        \end{pmatrix}
        \begin{pmatrix}
            y_1 \\
            y_2
        \end{pmatrix}
        =
        \begin{pmatrix}
            \scalarproduct{a}{x} \\
            \scalarproduct{b}{x}
        \end{pmatrix}, \\
        %
        \begin{pmatrix}
            1 + 4 + 1 + 1 & 5 - 4 + 1 - 9   \\
            5 - 4 + 1 - 9 & 25 + 4 + 1 + 81
        \end{pmatrix}
        \begin{pmatrix}
            y_1 \\
            y_2
        \end{pmatrix}
        =
        \begin{pmatrix}
            -1 + 6 - 3 + 5 \\
            -5 - 6 - 3 - 45
        \end{pmatrix}, \\
        %
        \begin{pmatrix}
            7  & -7  \\
            -7 & 111
        \end{pmatrix}
        \begin{pmatrix}
            y_1 \\
            y_2
        \end{pmatrix}
        =
        \begin{pmatrix}
            7 \\
            -59
        \end{pmatrix}, \\
        %
        \begin{pmatrix}
            7 & -7  \\
            0 & 104
        \end{pmatrix}
        \begin{pmatrix}
            y_1 \\
            y_2
        \end{pmatrix}
        =
        \begin{pmatrix}
            7 \\
            -52
        \end{pmatrix}, \\
        %
        \begin{pmatrix}
            y_1 \\
            y_2
        \end{pmatrix}
        =
        \begin{pmatrix}
            \frac{1}{2} \\
            -\frac{1}{2}
        \end{pmatrix}
    \end{gather*}
    Таким образом,
    \[
        \projection{V^\perp}{x}
        = \frac{1}{2} a + \frac{1}{2} b
        =
        \begin{pmatrix}
            \frac{1}{2} - \frac{5}{2} \\
            \frac{2}{2} + \frac{2}{2} \\
            \frac{1}{2} - \frac{1}{2} \\
            \frac{1}{2} + \frac{9}{2}
        \end{pmatrix}
        =
        \begin{pmatrix}
            -2 \\
            2  \\
            0  \\
            5
        \end{pmatrix},
    \]
    и расстояние
    \[
        \modulus{\perpendicular{V}{x}}
        = \modulus{\projection{V^\perp}{x}}
        = \sqrt{4 + 4 + 25}
        = \sqrt{33} .
    \]

    \subsection*{Ответ:}
    $\sqrt{33}$.

    \section*{Пункт г}
    Найти угол между вектором $x$ и подпространством $V$, заданным ОСЛУ:
    \[
        x =
        \begin{pmatrix}
            -1 \\ 5 \\ 3 \\ 5
        \end{pmatrix}, \\
        V:
        \begin{pmatrix}
            -2  & 4 & -2 & 0 \\
            -10 & 5 & 8  & 3
        \end{pmatrix}
        \begin{pmatrix}
            x_1 \\ x_2 \\ x_3 \\ x_4
        \end{pmatrix}
        = 0 .
    \]

    \subsection*{Решение:}
    Аналогично пункту в находим ортогональную составляющую:
    \begin{gather*}
        a =
        \begin{pmatrix}
            -2 \\ 4 \\ -2 \\ 0
        \end{pmatrix},
        b =
        \begin{pmatrix}
            -10 \\ 5 \\ 8 \\ 3
        \end{pmatrix}, \\
        %
        V^{\perp} = \left < a, b \right >, \\
        %
        \perpendicular{V}{x} = \projection{V^\perp}{x} = y_1 a + y_2 b,
    \end{gather*}
    где $y_1$ и $y_2$ удовлетворяют СЛУ:
    \begin{gather*}
        \begin{pmatrix}
            \scalarproduct{a}{a} & \scalarproduct{a}{b} \\
            \scalarproduct{b}{a} & \scalarproduct{b}{b}
        \end{pmatrix}
        \begin{pmatrix}
            y_1 \\
            y_2
        \end{pmatrix}
        =
        \begin{pmatrix}
            \scalarproduct{a}{x} \\
            \scalarproduct{b}{x}
        \end{pmatrix}, \\
        %
        \begin{pmatrix}
            4 + 16 + 4   & 20 + 20 - 16      \\
            20 + 20 - 16 & 100 + 25 + 64 + 9
        \end{pmatrix}
        \begin{pmatrix}
            y_1 \\
            y_2
        \end{pmatrix}
        =
        \begin{pmatrix}
            2 + 20 - 6 \\
            10 + 25 + 24 + 15
        \end{pmatrix}, \\
        %
        \begin{pmatrix}
            24 & 24  \\
            24 & 198
        \end{pmatrix}
        \begin{pmatrix}
            y_1 \\
            y_2
        \end{pmatrix}
        =
        \begin{pmatrix}
            16 \\
            74
        \end{pmatrix}, \\
        %
        \begin{pmatrix}
            24 & 24  \\
            0  & 174
        \end{pmatrix}
        \begin{pmatrix}
            y_1 \\
            y_2
        \end{pmatrix}
        =
        \begin{pmatrix}
            16 \\
            58
        \end{pmatrix}, \\
        %
        \begin{pmatrix}
            3 & 3 \\
            0 & 3
        \end{pmatrix}
        \begin{pmatrix}
            y_1 \\
            y_2
        \end{pmatrix}
        =
        \begin{pmatrix}
            2 \\
            1
        \end{pmatrix}, \\
        %
        \begin{pmatrix}
            y_1 \\
            y_2
        \end{pmatrix}
        =
        \begin{pmatrix}
            \frac{1}{3} \\
            \frac{1}{3}
        \end{pmatrix}
    \end{gather*}
    Таким образом,
    \[
        \projection{V^\perp}{x}
        = \frac{1}{3} a + \frac{1}{3} b
        =
        \begin{pmatrix}
            - \frac{2}{3} - \frac{10}{3} \\
            \frac{4}{3} + \frac{5}{3}    \\
            - \frac{2}{3} + \frac{8}{3}  \\
            0 + \frac{3}{3}
        \end{pmatrix}
        =
        \begin{pmatrix}
            -4 \\
            3  \\
            2  \\
            1
        \end{pmatrix},
    \]
    Проекция
    \[
        \projection{V}{x}
        = x - \perpendicular{V}{x}
        = x - \projection{V^\perp}{x}
        =
        \begin{pmatrix}
            -1 + 4 \\
            5 - 3  \\
            3 - 2  \\
            5 - 1
        \end{pmatrix}
        =
        \begin{pmatrix}
            3 \\
            2 \\
            1 \\
            4
        \end{pmatrix}
    \]
    и угол
    \begin{multline*}
        \angle \left ( x, V \right )
        = \arccos \frac{\scalarproduct{x}{\projection{V}{x}}}{\modulus{x} \cdot \modulus{\projection{V}{x}}}
        = \arccos \frac{-3 + 10 + 3 + 20}{\sqrt{1 + 25 + 9 + 25} \cdot \sqrt{9 + 4 + 1 + 16}} = \\
        %
        = \arccos \frac{30}{\sqrt{60} \cdot \sqrt{30}}
        = \arccos \sqrt{\frac{30}{60}}
        = \arccos \frac{1}{\sqrt{2}}
        = \frac{\pi}{4} .
    \end{multline*}

    \subsection*{Ответ:}
    $\frac{\pi}{4}$.
\end{document}