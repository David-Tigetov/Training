\documentclass[a4paper,12pt]{article}
\usepackage[T1]{fontenc}
\usepackage[utf8]{inputenc}
\usepackage[english,russian]{babel}
\usepackage[margin=2cm]{geometry}
\usepackage{amsmath}

\begin{document}

\title{Задача Коши}
\author{Тигетов Давид}
\maketitle

\section{Задача Коши}

Задача Коши
\begin{align}
    y^\prime(t) & = f(t, y(t)) ,
    \label{cauchy:equation}      \\
    y(t_0)      & = y_0 .
    \notag
\end{align}

Если функция $f(t,y(t))$ дифференцируема по $t$, тогда дифференцируя левую и правую части уравнения \eqref{cauchy:equation}, получим:
\begin{equation}~\label{cauchy:section_derivative}
    y^{\prime\prime}
    = f_t^\prime(t,y(t)) + f_y^\prime(t,y(t)) y^\prime(t)
    = f_t^\prime(t,y(t)) + f_y^\prime(t,y(t)) f(t,y(t))
\end{equation}

Дискретная задача Коши для сеточной функции $y_k$:
\[
    \frac{y_{k+1} - y_k}{h} = \Phi(t_k, y_k, y_{k+1}; h).
\]

\section{Методы Рунге-Кутты 2-го порядка}

Пусть
\[
    \Phi(t_k, y_k, y_{k+1}; h) = \sum_{k=1}^2 \alpha_k f(t_k + \beta_k h, y_k + \gamma_k h f(t_k,y_k)).
\]
Как выбрать $\alpha_k$, $\beta_k$ и $\gamma_k$?

Порядок аппроксимации, с использованием равенства \eqref{cauchy:section_derivative}:
\begin{multline*}
    \psi_k = \frac{y(t_{k+1}) - y(t_k)}{h} - \sum_{k=1}^2 \alpha_k f(t_k + \beta_k h, y(t_k) + \gamma_k h f(t_k,y(t_k))) = \\
    %
    \shoveleft{= \frac{y(t_k) + y^\prime(t_k) h + \frac{1}{2} y^{\prime\prime}(t_k) h^2 + A h^3 - y(t_k)}{h} -} \\
    \shoveright{ - \sum_{k=1}^2 \alpha_k \left( f(t_k, y(t_k)) + f_t^\prime(t_k, y(t_k)) \beta_k h + f_y^\prime(t_k, y(t_k)) \gamma_k h f(t_k,y(t_k)) + B_k h^2 \right) = } \\
    %
    \shoveleft{= y^\prime(t_k) + \frac{1}{2} y^{\prime\prime}(t_k) h + A h^2 -} \\
    \shoveright{ - \sum_{k=1}^2 \alpha_k \left( f(t_k, y(t_k)) + f_t^\prime(t_k, y(t_k)) \beta_k h + f_y^\prime(t_k, y(t_k)) \gamma_k h f(t_k,y(t_k)) + B_k h^2 \right) = } \\
    %
    \shoveleft{= f(t_k, y(t_k)) + \frac{1}{2} \left( f_t^\prime(t_k, y(t_k)) + f_y^\prime(t_k, y(t_k)) f^(t_k, y(t_k)) \right) h + C_1 h^2 -} \\
    \shoveright{ - \sum_{k=1}^2 \alpha_k \left( f(t_k, y(t_k)) + f_t^\prime(t_k, y(t_k)) \beta_k h + f_y^\prime(t_k, y(t_k)) \gamma_k h f(t_k,y(t_k)) + B_k h^2 \right) = } \\
    %
    \shoveleft{= f(t_k, y(t_k)) \left( 1 - \sum_{k=1}^2 \alpha_k \right ) + } \\
    + f_t^\prime(t_k, y(t_k)) \left( \frac{1}{2} - \sum_{k=1}^2 \alpha_k \beta_k \right) + \\
    + f_y^\prime(t_k, y(t_k)) f(t_k,y(t_k)) \left( \frac{1}{2} - \sum_{k=1}^2 \alpha_k \gamma_k \right)
    + ( A + B_1 + B_2 ) h^2
\end{multline*}
Для получения наибольшего порядка аппроксимации:
\begin{align*}
\end{align*}

\[
    \left \{
    \begin{array}{rcl}
        \alpha_1 + \alpha_2                   & = & 1 ,           \\
        \alpha_1 \beta_1 + \alpha_2 \beta_2   & = & \frac{1}{2} , \\
        \alpha_1 \gamma_1 + \alpha_2 \gamma_2 & = & \frac{1}{2} .
    \end{array}
    \right .
    \Leftrightarrow
    \left \{
    \begin{array}{rcl}
        \alpha_2                                    & = & 1 - \alpha_1 , \\
        \alpha_1 \beta_1 + (1 - \alpha_1) \beta_2   & = & \frac{1}{2} ,  \\
        \alpha_1 \gamma_1 + (1 - \alpha_1) \gamma_2 & = & \frac{1}{2} .
    \end{array}
    \right .
    \Leftrightarrow
    \left \{
    \begin{array}{rcc}
        \alpha_1 & = & \alpha ,                                  \\
        \alpha_2 & = & 1 - \alpha ,                              \\
        \beta_1  & = & \frac{1}{2} \frac{\beta}{\alpha} ,        \\
        \beta_2  & = & \frac{1}{2} \frac{1 - \beta}{1-\alpha} ,  \\
        \gamma_1 & = & \frac{1}{2} \frac{\gamma}{\alpha} ,       \\
        \gamma_2 & = & \frac{1}{2} \frac{1 - \gamma}{1-\alpha} , \\
    \end{array}
    \right .
\]
Таким образом,
\begin{align*}
    \Phi(t_k, y_k, y_{k+1}; h) & = \alpha f(t_k + \frac{1}{2} \frac{\beta}{\alpha} h, y_k + \frac{1}{2} \frac{\gamma}{\alpha} h f(t_k,y_k)) +                                      \\
                               & + (1 - \alpha) f \left( t_k + \frac{1}{2} \frac{1 - \beta}{1 - \alpha} h, y_k + \frac{1}{2} \frac{1 - \gamma}{1 - \alpha} h f(t_k, y_k) \right) .
\end{align*}

Какие значение может принимать $\beta$? Наверное,
\begin{gather*}
    \left \{
    \begin{array}{ccccc}
        t_k & \le & t_k + \frac{1}{2} \frac{\beta}{\alpha} h         & \le & t_k + h \\
        t_k & \le & t_k + \frac{1}{2} \frac{1 - \beta}{1 - \alpha} h & \le & t_k + h
    \end{array}
    \right .
    \Leftrightarrow
    \left \{
    \begin{array}{ccccc}
        0 & \le & \frac{1}{2} \frac{\beta}{\alpha} h         & \le & h \\
        0 & \le & \frac{1}{2} \frac{1 - \beta}{1 - \alpha} h & \le & h
    \end{array}
    \right .
    \Leftrightarrow \\
    %
    \Leftrightarrow
    \left \{
    \begin{array}{ccccc}
        0 & \le & \frac{\beta}{\alpha}         & \le & 2 \\
        0 & \le & \frac{1 - \beta}{1 - \alpha} & \le & 2
    \end{array}
    \right .
    \Leftrightarrow
    \left \{
    \begin{array}{ccccc}
        0 & \le & \beta     & \le & 2 \alpha     \\
        0 & \le & 1 - \beta & \le & 2 - 2 \alpha
    \end{array}
    \right .
    \Leftrightarrow \\
    %
    \Leftrightarrow
    \left \{
    \begin{array}{ccccc}
        0             & \le & \beta & \le & 2 \alpha \\
        -1 + 2 \alpha & \le & \beta & \le & 1
    \end{array}
    \right .
\end{gather*}

\begin{itemize}
    \item усовершенствованный метод Эйлера при $\alpha = 1$, $\beta = 1$, $\gamma = 1$.
    \item метод Эйлера-Коши при $\alpha = \frac{1}{2}$, $\beta = 0$, $\gamma = 0$.
\end{itemize}

\section{Метод трапеций}

Пусть
\[
    \Phi(t_k, y_k, y_{k+1}; h) = \alpha_1 f(t_k,y_k) + \alpha_2 f(t_{k+1}, y_{k+1}).
\]
Как выбрать $\alpha_1$ и $\alpha_2$?

Порядок аппроксимации
\begin{multline*}
    \psi_k = \frac{y(t_{k+1}) - y(t_k)}{h} - \Phi(t_k, y(t_k), y(t_{k+1}); h) = \\
    %
    = \frac{y(t_{k+1}) - y(t_k)}{h} - \alpha_1 f(t_k,y(t_k)) - \alpha_2 f(t_{k+1}, y(t_{k+1})) = \\
    %
    \shoveleft{
        = \frac{y(t_k) + y^\prime(t_k) h + \frac{1}{2} y^{\prime\prime}(t_k) h^2 + \overline{o}(h^3) - y(t_k)}{h}
        - \alpha_1 f(t_k,y(t_k)) - } \\
    \shoveright{
        - \alpha_2 \left( f(t_k, y(t_k)) + \left ( f_t^\prime(t_k, y(t_k)) + f_y^\prime(t_k, y(t_k)) \right) h + \overline{o}(h^2) \right ) =} \\
    %
    \shoveleft{
        = y^\prime(t_k) + \frac{1}{2} y^{\prime\prime}(t_k) h + \overline{o}(h^2)
        - \alpha_1 f(t_k,y(t_k)) - } \\
    \shoveright{
        - \alpha_2 f(t_k, y(t_k) - \alpha_2 \left ( f_t^\prime(t_k, y(t_k)) + f_y^\prime(t_k, y(t_k))  y^\prime(t_k) \right) h + \overline{o}(h^2) =} \\
    %
    = y^\prime(t_k) - \left ( \alpha_1 f(t_k,y(t_k)) + \alpha_2 f(t_k, y(t_k) \right )
    + \frac{1}{2} y^{\prime\prime}(t_k) h - \alpha_2 y^{\prime\prime}(t_k) h + \overline{o}(h^2) ,
\end{multline*}
поскольку, дифференцируя левую и правую части уравнения \eqref{cauchy:equation}, получим:
\[
    y^{\prime\prime}(t) = f_t^\prime(t, y(t)) + f_y^\prime(t, y(t))  y^\prime(t) .
\]
Для $\alpha_1$ и $\alpha_2$ получим систему:
\[
    \left \{
    \begin{array}{lcll}
        \alpha_1 & + & \alpha_2 & = 1           \\
                 &   & \alpha_2 & = \frac{1}{2}
    \end{array}
    \right .
\]
Откуда
\begin{align*}
    \alpha_1 & = \frac{1}{2} , \\
    \alpha_2 & = \frac{1}{2} ,
\end{align*}
и метод имеет вид:
\[
    \frac{y_{k+1} - y_k}{h} = \frac{f(t_k,y_k) + f(t_{k+1}, y_{k+1})}{2}.
\]

\end{document}