\documentclass[a4paper,12pt]{article}
\usepackage[T1]{fontenc}
\usepackage[utf8]{inputenc}
\usepackage[english,russian]{babel}
\usepackage[margin=2cm]{geometry}
\usepackage{amsmath}
\usepackage{array}

\newcommand{\variance}[1]{\mathtt{D} \left[ #1 \right]}

\newcolumntype{P}[1]{>{\centering\arraybackslash}p{#1}}

\begin{document}

\title{Математическая статистика: домашние задания}
\author{Тигетов Давид Георгиевич}
\maketitle

\section*{Домашнее задание 1}

\subsection*{Задача 1 \cite[108]{Efimov}}

Докажите, что выборочное среднее, вычисленное по выборке из распределения Пуассона $\mathcal{P}(\lambda)$, будет несмещённой и состоятельной оценкой
параметра $\lambda$.

\subsection*{Задача 2 \cite[103]{Efimov}}

Пусть $m_1$, $m_2$ и $s_1^2$, $s_2^2$ --- несмещенные и состоятельные оценки параметров $\mu$ и $\sigma^2$, построенные для двух выборок
объёмами $n_1$ и $n_2$. Покажите, что статистики:
\begin{gather*}
    m = \frac{n_1 m_1 + n_2 m_2}{n_1 + n_2} , \\
    s^2 = \frac{(n_1 - 1) s_1^2 + (n_2 - 1) s_2^2}{n_1 + n_2 - 2} .
\end{gather*}
будут несмещёнными и состоятельными оценками $\mu$ и $\sigma^2$.

\subsection*{Задача 3 \cite[107]{Efimov}}

Пусть $\widetilde{\theta}$ --- несмещенная оценка параметра $\theta$, $\variance{\widetilde{\theta}} < \infty$. Докажите, что
$\widetilde{\theta}^2$ является смещенной оценкой $\theta^2$, и вычислите смещение.

\subsection*{Задача 4 \cite[110]{Efimov}}

Пусть $\widetilde{\theta}$ --- состоятельная оценка параметра $\theta$, a $\psi(x)$ --- непрерывная функция в области изменения $\widetilde{\theta}$.
Докажите, что $\psi(\widetilde{\theta})$ является состоятельной оценкой $\psi(\theta)$.

\section*{Домашнее задание 2}

\subsection*{Задача 1 \cite[124]{Efimov}}

Случайные величины $\xi_1$, \dots, $\xi_m$ независимы и имеют биномиальные распределения $Bi(n_1, p)$, \dots, $Bi(n_m, p)$, все $n_k$ известны.
Найдите МП-оценку параметра $p$. Покажите, что МП-оценка является несмещённой и вычислите её дисперсию.

\subsection*{Задача 2}

$\left( \xi_1, \dots, \xi_n \right)$ --- выборка из нормального распределения $\mathcal{N} \left( m, \sigma^2 \right)$. Найдите МП-оценку двумерного
параметра $\left( m, \sigma^2 \right)$.

\subsection*{Задача 3 \cite[126]{Efimov}}

$\left( \xi_1, \dots, \xi_n \right)$ --- выборка из распределения с плотностью:
\[
    p_\xi(x|\theta)
    \left\{
    \begin{array}{ll}
        \theta x & x \in \left[ 0, \sqrt{\frac{2}{\theta}}\right]    \\
        0        & x \notin \left[ 0, \sqrt{\frac{2}{\theta}}\right]
    \end{array}
    \right.
\]
Найти МП-оценку математического ожидания этого распределения.

\subsection*{Задача 4}

$\left( \xi_1, \dots, \xi_n \right)$ --- выборка из распределения Рэлея $\mathcal{R}(\theta)$. Постройте оценки параметра $\theta$ методами моментов
и порядковых статистик. Сравните асимптотические дисперсии оценок.

\section*{Домашнее задание 3}

\subsection*{Задача 1 \cite[164]{Efimov}}

По двум выборкам из нормального распределения $\mathcal{N} \left( \mu, \sigma^2 \right)$ объёмами $n_1$ и $n_2$ вычислены выборочные средние
$m_1$, $m_2$, исправленные выборочные дисперсии $s_1^2$, $s_2^2$ и статистики:
\begin{gather*}
    m = \frac{n_1 m_1 + n_2 m_2}{n_1 + n_2} , \\
    s^2 = \frac{(n_1 - 1) s_1^2 + (n_2 - 1) s_2^2}{n_1 + n_2 - 2} .
\end{gather*}

Покажите, что при известной дисперсии $\sigma^2$ доверительный интервал для $\mu$ уровня доверия $\alpha$ имеет вид:
\[
    m - u(\alpha) \frac{\sigma}{\sqrt{n_1 + n_2}}
    < \mu
    < m + u(\alpha) \frac{\sigma}{\sqrt{n_1 + n_2}}
\]
и укажите $u(\alpha)$, а при неизвестной дисперсии доверительный интервал для $\mu$ уровня доверия $\alpha$ имеет вид:
\[
    m - t(\alpha) \frac{s}{\sqrt{n_1 + n_2}}
    < \mu
    < m + t(\alpha) \frac{s}{\sqrt{n_1 + n_2}}
\]
и укажите $t(\alpha)$.

\subsection*{Задача 2 \cite[176, 177]{Efimov}}

По двум выборкам из нормальных распределений $\mathcal{N} \left( \mu_1, \sigma_1^2 \right)$ и $\mathcal{N} \left( \mu_2, \sigma_2^2 \right)$ объёмами
$n_1$ и $n_2$ вычислены выборочные средние $m_1$, $m_2$, исправленные выборочные дисперсии $s_1^2$, $s_2^2$ и статистика:
\[
    s^2 = \frac{(n_1 - 1) s_1^2 + (n_2 - 1) s_2^2}{n_1 + n_2 - 2} .
\]

Покажите, что если дисперсии $\sigma_1^2$ и $\sigma_2^2$ известны, то доверительный интервал
для разности $\mu_1 - \mu_2$ с уровнем доверия $\alpha$ имеет вид:
\[
    (m_1 - m_2) - u(\alpha) \sqrt{\frac{\sigma_1^2}{n_1} + \frac{\sigma_2^2}{n_2}}
    < \mu_1 - \mu_2
    < (m_1 - m_2) + u(\alpha) \sqrt{\frac{\sigma_1^2}{n_1} + \frac{\sigma_2^2}{n_2}} ,
\]
и укажите $u(\alpha)$. Если дисперсии неизвестны, но равны $\sigma_1^2 = \sigma_2^2$, то доверительный интервал для разности $\mu_1 - \mu_2$
с уровнем доверия $\alpha$ имеет вид:
\[
    (m_1 - m_2) - t(\alpha) s \sqrt{\frac{1}{n_1} + \frac{1}{n_2}}
    < \mu_1 - \mu_2
    < (m_1 - m_2) + t(\alpha) s \sqrt{\frac{1}{n_1} + \frac{1}{n_2}} ,
\]
и укажите $t(\alpha)$.

\subsection*{Задача 3 \cite[187]{Efimov}}

Из урны, содержащей черные и белые шары в неизвестной пропорции, случайным образом извлекается 100 шаров с возвращением. Найти доверительный интервал
доли чёрных шаров с уровнем доверия $\alpha=0.95$, если среди вынутых оказалось 30 чёрных шаров. Сколько извлечений нужно сделать, чтобы относительная
частота отличалась от доли чёрных шаров не более чем на 0.01 с вероятностью 0.99?

\subsection*{Задача 4 \cite[191]{Efimov}}

На каждой из 36 АТС города в период с двух до трех часов было зафиксировано в среднем 2 вызова. Считая, что число вызовов для каждой АТС имеет
распределение Пуассона с одним и тем же параметром $\lambda$, приближённо найти нижнюю и верхнюю \textbf{доверительные границы} с уровнем
доверия $\alpha = 0.9$.

\section*{Домашнее задание 4}

\subsection*{Задача 1 \cite[291]{Efimov}}

При испытании радиоэлектронной аппаратуры фиксировалось число отказов. Результаты 59 испытаний:

\begin{center}
    \begin{tabular}{|P{2cm}|P{1cm}|P{1cm}|P{1cm}|P{1cm}|}
        \hline
        Число \par отказов   & 0  & 1  & 2 & 3 \\
        \hline
        Число \par испытаний & 42 & 10 & 4 & 3 \\
        \hline
    \end{tabular}
\end{center}

С помощью критерия $\chi^2$ определите уровни значимости, при которых принимаются гипотезы о том, что число отказов имеет:
\begin{enumerate}
    \item распределение Пуассона,
    \item биномиальное распределение.
\end{enumerate}
Какое из двух распределений числа отказов лучше согласуется с данными?

\subsection*{Задача 2}

По результатам КМ-1 студенты четвёртого курса получили следующие количества оценок:

\begin{center}
    \begin{tabular}{|P{2cm}|P{1cm}|P{1cm}|P{1cm}|P{1cm}|}
        \hline
        Группа  & 2  & 3 & 4  & 5  \\
        \hline
        A-05-21 & 13 & 2 & 6  & 3  \\
        \hline
        A-13-21 & 0  & 1 & 13 & 11 \\
        \hline
        A-14-21 & 1  & 2 & 7  & 3  \\
        \hline
        A-16-21 & 5  & 2 & 4  & 4  \\
        \hline
        A-18-21 & 10 & 3 & 6  & 4  \\
        \hline
    \end{tabular}
\end{center}

При каком наименьшем уровне значимости отклоняется гипотеза об однородности успеваемости в группах?

\subsection*{Задача 3 \cite[230]{Efimov}}

Давление в камере контролируется двумя манометрами. Считается, что показания манометров независимы и имеют нормальное распределение с неизвестными
математическими ожиданиями и дисперсиями. По результатам $n = 10$ замерам каждого из манометров вычислены: выборочные средние
$\widehat{m}_1^{(1)} = 15.3$, $\widehat{m}_1^{(2)} = 16.1$ и исправленные выборочные дисперсии $\widetilde{\mu}_{2}^{(1)} = 0.2$,
$\widetilde{\mu}_{2}^{(2)} = 0.15$. Для уровня значимости $\alpha = 0.1$ проверьте гипотезы:
\begin{enumerate}
    \item о равенстве дисперсий показаний с двусторонней критической областью,
    \item о равенстве математических ожиданий показаний с правосторонней критической областью.
\end{enumerate}

\subsection*{Задача 4}

Получена реализации выборки: 0.768, 1.31, 2.11, 1.29, 0.868, 1.59, 1.49, 3.24, 7.17, 2.1, 2.1, 0.0434, 1.82, 0.278, 1.06, 7.95, 1.69, 0.809, 1.63, 1.15.
Используя критерий согласия Колмогорова определите из какого распределения получена выборка из равномерного или показательного.

Получена ещё одна реализация выборки: 3.72, 0.49, 1.59, 1.21, 0.811, 2.63, 2.42, 0.265, 2.24, 0.596, 0.725, 3.5, 1.38, 0.0537, 3.86.
Проверьте, что распределения двух выборок совпадают, по критерию Колмогорова--Смирнова.

\subsection*{Задача 5 \cite[206, 207, 210]{Efimov}}

Получена выборка из нормального распределения $\mathcal{N} \left( m, \sigma^2 \right)$, где $m$ --- неизвестное, а $\sigma = 1$.
\begin{enumerate}
    \item Можно ли утверждать, что $m$ не имеет положительного смещения относительно $m_0 = 40$ при уровне значимости $\alpha = 0.01$, если
          выборочное среднее $\widehat{m}_1 = 40.2$?
    \item Каковы будут ошибки первого и второго рода для основной гипотезы $m = 40$ и альтернативной гипотезы $m = 40.3$ при $n=36$?
    \item Пусть основная гипотеза $m = 40$, а альтернативная гипотеза $m = 40.3$. При каком минимальном объёме выборки $n$ ошибка первого рода не
          превосходит 0.05, а ошибка второго рода не превосходит 0.1? Какова при этом критическая область?
    \item Пусть $n=36$ и критическая область $\Gamma = (40.2, \infty)$. Постройте график функции мощности на отрезке $[39, 41]$.
\end{enumerate}

\begin{thebibliography}{1}
    \bibitem{Efimov} Сборник задач по математике, часть 4 // Под редакцией Ефимова А.В. и Поспелова А.С. - М.: Издательство Физико-математической
    литературы, 2003. - 432 с.
\end{thebibliography}
\end{document}