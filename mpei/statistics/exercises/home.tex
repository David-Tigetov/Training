\documentclass[a4paper,12pt]{article}
\usepackage[T1]{fontenc}
\usepackage[utf8]{inputenc}
\usepackage[english,russian]{babel}
\usepackage[margin=2cm]{geometry}
\usepackage{amsmath}

\newcommand{\variance}[1]{\mathtt{D} \left[ #1 \right]}

\begin{document}

\title{Математическая статистика: домашние задания}
\author{Тигетов Давид Георгиевич}
\maketitle

\section*{Домашнее задание 1}

\subsection*{Задача 1 \cite[108]{Efimov}}

Докажите, что выборочное среднее, вычисленное по выборке из распределения Пуассона $\mathcal{P}(\lambda)$, будет несмещённой и состоятельной оценкой
параметра $\lambda$.

\subsection*{Задача 2 \cite[103]{Efimov}}

Рассмотрим две выборки объёмов $n_1$ и $n_2$ из одной генеральной совокупности со средним $\mu$ и дисперсией $\sigma^2$.
Пусть $m_1$ и $m_2$, $s_1^2$ и $s_2^2$ --- несмещенный оценки математических ожиданий и дисперсий, определённые по выборкам. Докажите, что
объединённые оценки, вычисляемые по формулам:
\begin{gather*}
    \overline{m} = \frac{n_1 m_1 + n_2 m_2}{n_1 + n_2} , \\
    s^2 = \frac{(n_1 - 1) s_1^2 + (n_2 - 1) s_2^2}{n_1 + n_2 - 2} .
\end{gather*}
будут несмещёнными и состоятельными оценками $\mu$ и $\sigma^2$.

\subsection*{Задача 3 \cite[107]{Efimov}}

Пусть $\widetilde{\theta}$ --- несмещенная оценка параметра $\theta$, $\variance{\widetilde{\theta}} < \infty$. Докажите, что
$\widetilde{\theta}^2$ является смещенной оценкой $\theta^2$, и вычислите смещение.

\subsection*{Задача 4 \cite[110]{Efimov}}

Пусть $\widetilde{\theta}$ --- состоятельная оценка параметра $\theta$, a $\psi(x)$ --- непрерывная функция в области изменения $\widetilde{\theta}$.
Докажите, что $\psi(\widetilde{\theta})$ является состоятельной оценкой $\psi(\theta)$.

\section*{Домашнее задание 2}

\subsection*{Задача 1 \cite[124]{Efimov}}

Случайные величины $\xi_1$, \dots, $\xi_m$ независимы и имеют биномиальные распределения $Bi(n_1, p)$, \dots, $Bi(n_m, p)$, все $n_k$ известны.
Найдите МП-оценку параметра $p$. Покажите, что МП-оценка является несмещённой и вычислите её дисперсию.

\subsection*{Задача 2}

$\left( \xi_1, \dots, \xi_n \right)$ --- выборка из нормального распределения $\mathcal{N} \left( m, \sigma^2 \right)$. Найдите МП-оценку двумерного
параметра $\left( m, \sigma^2 \right)$.

\subsection*{Задача 3 \cite[126]{Efimov}}

$\left( \xi_1, \dots, \xi_n \right)$ --- выборка из распределения с плотностью:
\[
    p_\xi(x|\theta)
    \left\{
    \begin{array}{ll}
        \theta x & x \in \left[ 0, \sqrt{\frac{2}{\theta}}\right]    \\
        0        & x \notin \left[ 0, \sqrt{\frac{2}{\theta}}\right]
    \end{array}
    \right.
\]
Найти МП-оценку математического ожидания этого распределения.

\subsection*{Задача 4}

$\left( \xi_1, \dots, \xi_n \right)$ --- выборка из распределения Рэлея $\mathcal{R}(\theta)$. Постройте оценки параметра $\theta$ методами моментов
и порядковых статистик. Сравните асимптотические дисперсии оценок.

\section*{Домашнее задание 3}

\subsection*{Задача 1 \cite[164]{Efimov}}

Величины первой выборки объёма $n_1$ имеют математическое ожидание $\mu$ и дисперсию $\sigma_1^2$, величины второй выборки объёма
$n_2$ --- $\mu$ и $\sigma_2^2$. Пусть
\begin{gather*}
    m = \frac{n_1 m_1 + n_2 m_2}{n_1 + n_2} , \\
    s^2 = \frac{(n_1 - 1) s_1^2 + (n_2 - 1) s_2^2}{n_1 + n_2 - 2} .
\end{gather*}

Покажите, что если дисперсии $\sigma_1^2$ и $\sigma_2^2$ известны, то доверительный интервал для $\mu$ уровня доверия $\alpha$ имеет вид:
\[
    m - u(\alpha) \sqrt{\frac{\sigma_1^2}{n_1} + \frac{\sigma_2^2}{n_2}}
    < \mu
    < m + u(\alpha) \sqrt{\frac{\sigma_1^2}{n_1} + \frac{\sigma_2^2}{n_2}}
\]
и укажите $u(\alpha)$. Если же дисперсии неизвестны, но равны $\sigma_1^2 = \sigma_2^2$, то доверительный интервал для $\mu$ имеет вид:
\[
    m - t(\alpha) s \sqrt{\frac{1}{n_1} + \frac{1}{n_2}}
    < \mu
    < m + t(\alpha) s \sqrt{\frac{1}{n_1} + \frac{1}{n_2}}
\]
и укажите $t(\alpha)$.

\subsection*{Задача 2 \cite[176, 177]{Efimov}}

Величины первой выборки объёма $n_1$ имеют математическое ожидание $\mu_1$ и дисперсию $\sigma_1^2$, величины второй выборки объёма
$n_2$ --- $\mu_2$ и $\sigma_2^2$. Покажите, что если дисперсии $\sigma_1^2$ и $\sigma_2^2$ известны, то доверительный интервал
для разности $\mu_1 - \mu_2$ с уровнем доверия $\alpha$ имеет вид:
\[
    (m_1 - m_2) - u(\alpha) \sqrt{\frac{\sigma_1^2}{n_1} + \frac{\sigma_2^2}{n_2}}
    < \mu_1 - \mu_2
    < (m_1 - m_2) + u(\alpha) \sqrt{\frac{\sigma_1^2}{n_1} + \frac{\sigma_2^2}{n_2}} ,
\]
где $\widehat{m}_1$ и $\widehat{m}_2$ --- выборочные средние первой и второй выборок, и укажите $u(\alpha)$. Если дисперсии
неизвестны, но равны $\sigma_1^2 = \sigma_2^2$, то доверительный интервал для разности $\mu_1 - \mu_2$ с уровнем доверия $\alpha$
имеет вид:
\[
    (m_1 - m_2) - t(\alpha) s \sqrt{\frac{1}{n_1} + \frac{1}{n_2}}
    < m_1 - m_2
    < (m_1 - m_2) + t(\alpha) s \sqrt{\frac{1}{n_1} + \frac{1}{n_2}} ,
\]
где
\[
    s^2 = \frac{(n_1 - 1) s_1^2 + (n_2 - 1) s_2^2}{n_1 + n_2 - 2} ,
\]
где $s_1^2$ и $s_2^2$ --- исправленные выборочные дисперсии выборок, и укажите $t(\alpha)$.

\subsection*{Задача 3 \cite[187]{Efimov}}

Из урны, содержащей черные и белые шары в неизвестной пропорции, случайным образом извлекается 100 шаров с возвращением. Найти доверительный интервал
доли чёрных шаров с уровнем доверия $\alpha=0.95$, если среди вынутых оказалось 30 чёрных шаров. Сколько извлечений нужно сделать, чтобы относительная
частота отличалась от доли чёрных шаров не более чем на 0.01 с вероятностью 0.99?

\subsection*{Задача 4 \cite[191]{Efimov}}

На каждой из 36 АТС города в период с двух до трех часов было зафиксировано в среднем 2 вызова. Считая, что число вызовов для каждой АТС имеет
распределение Пуассона с одним и тем же параметром $\lambda$, приближённо найти нижнюю и верхнюю \textbf{доверительные границы} с уровнем
доверия $\alpha = 0.9$.

\begin{thebibliography}{1}
    \bibitem{Efimov} Сборник задач по математике, часть 4 // Под редакцией Ефимова А.В. и Поспелова А.С. - М.: Издательство Физико-математической
    литературы, 2003. - 432 с.
\end{thebibliography}
\end{document}