\documentclass[a4paper,12pt]{article}
\usepackage[T1]{fontenc}
\usepackage[utf8]{inputenc}
\usepackage[english,russian]{babel}
\usepackage[margin=2cm]{geometry}
\usepackage{amsmath}
\usepackage{xcolor}

\begin{document}

\title{Контрольная работа}
\author{Тигетов Давид Георгиевич}
\maketitle

\section*{Вариант 1}

\subsection*{Задача 1}

Проверить простую гипотезу о вероятностях при уровне значимости $\alpha = 0.05$.

\begin{tabular}{|c|c|c|c|}
    \hline
    Количества  & 25  & 45  & 30  \\
    \hline
    Вероятности & 0.3 & 0.5 & 0.2 \\
    \hline
\end{tabular}

Решение:

\begin{tabular}{|c|c|c|c|}
    \hline
    $\nu_i$   & $\widehat{p}_i$ & $n \widehat{p}_i$ & $\frac{(\nu_i - n \widehat{p}_i)^2}{n \widehat{p}_i}$ \\
    \hline
    25        & 0.300           & 30.000            & 0.833                                                 \\
    45        & 0.500           & 50.000            & 0.500                                                 \\
    30        & 0.200           & 20.000            & 5.000                                                 \\
    \hline
    $n = 100$ & 1.000           &                   & $X^2 = 6.333$                                         \\
    \hline
\end{tabular}

Степени свободы 3 - 1 = 2, уровень значимости 0.04214, критическая область $(5.99146, \infty)$.



\subsection*{Задача 2}

Две выборки из нормального распределения. В реализации первой выборки: объём $n_1 = 8$, выборочное среднее $m_1 = 9$, исправленная выборочная дисперсия
$s_1^2 = 15$. В реализации второй выборки: объём $n_2 = 8$, выборочное среднее $m_2 = 3$, исправленная выборочная дисперсия $s_2^2 = 17$.
Проверить гипотезу о равенстве математических ожиданий при уровне значимости $\alpha = 0.1$.

Решение:

Статистика
\[
    \sqrt{\frac{8 \cdot 8}{8 + 8}} \frac{9 - 3}{\sqrt{\frac{7 \cdot 15 + 7 \cdot 17}{8 + 8 - 2}}}
    = \sqrt{\frac{8}{2}} \frac{6}{\sqrt{\frac{15 + 17}{2}}}
    = 2 \frac{6}{\sqrt{\frac{16 + 16}{2}}}
    = 2 \frac{6}{\sqrt{16}}
    = 2 \frac{6}{4}
    = 3
\]
Критическая область: $(- \infty, -1.7613) \cup (1.7613, \infty)$.

\subsection*{Задача 3}

Проверьте по критерию согласия Колмогорова, что $(1, 0, 0, 1, 2, 1, 1, 0, 1, 1)$ является реализацией выборки из биномиального распределения при
уровне значимости $\alpha = 0.01$.

Решение:

Оценка вероятности: $\widehat{p} = \frac{6+2}{2 \cdot 10} = \frac{8}{2 \cdot 10} = 0.4$.

\begin{tabular}{|c|c|c|c|}
    \hline
    $x$       & 0 & 1              & 2                                                   \\
    \hline
    $F^*$     & 0 & 0.3            & 0.9                                                 \\
    \hline
    $F$       & 0 & $0.6^2 = 0.36$ & $0.36 + 2 \cdot 0.4 \cdot 0.6 = 0.36 + 0.48 = 0.84$ \\
    \hline
    $F^* - F$ & 0 & -0.06          & 0.06                                                \\
    \hline
\end{tabular}

Статистика: $D = \sqrt{10} \cdot 0.06$, критическая область $(0.488932, \infty)$.

\subsection*{Задача 4}

В цирке.

\begin{tabular}{|c|c|c|c|}
    \hline
          & Мясо & Брокколи & Мороженое \\
    \hline
    Львы  & 32   & 23       & 12        \\
    \hline
    Тигры & 18   & 7        & 8         \\
    \hline
\end{tabular}

Дрессировщик Епифан Зверев жалуется, что тигров кормят хуже, чем львов. А прав ли он при уровне значимости $\alpha = 0.1$?

Решение:

Оценки вероятностей $\widehat{\theta} = (0.5, 0.3, 0.2)$.

\begin{tabular}{|c|c|c|}
    \hline
    32    33.500     0.067 & 23    20.100     0.418 & 12    13.400     0.146 \\
    \hline
    18    16.500     0.136 & 7     9.900     0.849  & 8     6.600     0.297  \\
    \hline
\end{tabular}

Статистика 1.914669, степени свободы (2-1)(3-1) = 2, критическая область: $(4.60517, \infty)$

\subsection*{Задача 5}

В реализации выборки объёма 100 встречается 30 нулей, 40 единиц и 30 двоек. По критерию хи-квадрат проверьте, что распределение величин выборки
биномиальное при уровне значимости $\alpha = 0.05$.

Решение:

Оценка вероятности $\widehat{p} = \frac{40 + 30 \cdot 2}{2 \cdot 100} = \frac{40 + 60}{2 \cdot 100} = 0.5$.

\begin{tabular}{|c|c|c|c|}
    \hline
    $\nu_i$   & $\widehat{p}_i$ & $n \widehat{p}_i$ & $\frac{(\nu_i - n \widehat{p}_i)^2}{n \widehat{p}_i}$ \\
    \hline
    30        & 0.250           & 25.000            & 1.000                                                 \\
    40        & 0.500           & 50.000            & 2.000                                                 \\
    30        & 0.250           & 25.000            & 1.000                                                 \\
    \hline
    $n = 100$ & 1.000           &                   & $X^2 = 4$                                             \\
    \hline
\end{tabular}

Степени свободы $3-1-1=1$, критическая область $(3.84146, \infty)$.


\section*{Вариант 2}

\subsection*{Задача 1}

Две выборки из нормального распределения. В реализации первой выборки: объём $n_1 = 10$, исправленная выборочная дисперсия $s_1^2 = 3$.
В реализации второй выборки: объём $n=15$, исправленная выборочная дисперсия $s_2^2 = 2$. Проверить гипотезу о равенстве дисперсий
при уровне значимости $\alpha = 0.05$.

Решение:

Статистика $f = \frac{(10-1) \cdot 3}{(15-1) \cdot 2} = \frac{27}{28}$, критическая область $(0, 0.68) \cup (3.7979, \infty)$. Гипотеза
принимается.

\subsection*{Задача 2}

Проверить гипотезу о вероятностях при уровне значимости $\alpha = 0.1$.

\begin{tabular}{|c|c|c|c|c|}
    \hline
    Количества  & 15  & 15  & 50  & 20                                     \\
    \hline
    Вероятности & 0.2 & 0.1 & 0.4 & \textcolor{red}{0.2} $\rightarrow$ 0.3 \\
    \hline
\end{tabular}

Решение:

\begin{tabular}{|c|c|c|c|}
    \hline
    $\nu_i$   & $\widehat{p}_i$ & $n \widehat{p}_i$ & $\frac{(\nu_i - n \widehat{p}_i)^2}{n \widehat{p}_i}$ \\
    \hline
    15        & 0.200           & 20.000            & 1.250                                                 \\
    15        & 0.100           & 10.000            & 2.500                                                 \\
    50        & 0.400           & 40.000            & 2.500                                                 \\
    20        & 0.300           & 30.000            & 3.333                                                 \\
    \hline
    $n = 100$ & 1.000           &                   & $X^2 = 9.583$                                         \\
    \hline
\end{tabular}

Степени свободы $4-1=3$, критическая область $(7.81473, \infty)$.

\subsection*{Задача 3}

Получены ли реализации выборок $(0, 1, 1, 0, 2, 1, 2, 1)$ и $(0, 2, 1, 0, 1, 0, 0, 2)$ из одного распределения при уровне значимости $\alpha = 0.01$?

Решение:

\begin{tabular}{|c|c|c|c|}
    \hline
    $x$       & 0 & 1                           & 2                           \\
    \hline
    $F_1^*$   & 0 & $\frac{2}{8} = \frac{1}{4}$ & $\frac{6}{8} = \frac{3}{4}$ \\
    \hline
    $F_2^*$   & 0 & $\frac{4}{8} = \frac{2}{4}$ & $\frac{6}{8} = \frac{3}{4}$ \\
    \hline
    $F^* - F$ & 0 & $\frac{1}{4}$               & 0                           \\
    \hline
\end{tabular}

Статистика $D = \sqrt{\frac{8 \cdot 8}{8 + 8}} \cdot 0.25 = 2 \cdot 0.25 = 0.5$, критическая область $(1.62762, \infty)$.


\subsection*{Задача 4}

Снеговики.

\begin{tabular}{|c|c|c|}
    \hline
                & С ведёрком & Без ведёрка \\
    \hline
    С морковкой & 20         & 25          \\
    \hline
    С картошкой & 15         & 40          \\
    \hline
\end{tabular}

Влияет ли наличие ведёрка на форму носа при уровне значимости $\alpha=0.1$.

Решение:

Количество $n=100$.

\begin{tabular}{|c|c|c|}
    \hline
    Вероятности & 0.35            & 0.65            \\
    \hline
    0.45        & 20 15.750 1.147 & 25 29.250 0.618 \\
    \hline
    0.55        & 15 19.250 0.938 & 40 35.750 0.505 \\
    \hline
\end{tabular}

Статистика $X^2 = 3.20790$, cтепени свободы $(2-1) \cdot (2-1) = 1$, порог $\approx 2.70554$, гипотеза о независимости отклоняется.


\end{document}