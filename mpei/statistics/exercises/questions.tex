\documentclass[a4paper,12pt]{article}
\usepackage[T1]{fontenc}
\usepackage[utf8]{inputenc}
\usepackage[english,russian]{babel}
\usepackage[margin=2cm]{geometry}

\newcommand{\expectation}[1]{\mathtt{M} \left[ #1 \right]}
\newcommand{\variance}[1]{\mathtt{D} \left[ #1 \right]}

\begin{document}

\title{Вопросы и задачи к защитам лабораторных работ}
\author{Тигетов Давид Георгиевич}
\maketitle

\section*{Лабораторная работа 2}

\subsection*{Теория}

\begin{enumerate}
    \item Сформулируйте определения выборки, вариационного ряда, эмпирической функции распределения.
    \item Объясните принцип вычисления частот, относительных частот и построения гистрограмм.
    \item Приведите выражения начальных и центральных выборочных моментов.
\end{enumerate}

\subsection*{Практика}

Задача 1 обязательная, из задач 2 -- 5 одна задача по выбору.

\begin{enumerate}
    \item В результате эксперимента получена реализация выборки $x_1 = 0.5$, $x_2 = -1.5$, $x_3 = -0.2$, $x_4 = 1.1$, $x_5 = 0.6$.
          \begin{enumerate}
              \item Постройте реализацию вариационного ряда.
              \item Нарисуйте график реализации эмпирической функции распределения.
              \item Вычислите реализации выборочного среднего и исправленной выборочной дисперсии.
          \end{enumerate}

    \item $\left( \xi_1, \dots, \xi_n \right)$ --- выборка из нормального распределения $\mathcal{N}\left(m, \sigma^2 \right)$, где $m$ --- известная
          величина, а $\sigma^2$ --- неизвестная величина. Является ли статистика
          \[
              \varphi(\xi_1, \dots, \xi_n) = \frac{1}{n} \sum_{k=1}^n \left( \xi_k - m \right)^2 + \left( \frac{1}{n} \sum_{k=1}^n \xi_k \right)^2.
          \]
          несмещенной и состоятельной оценкой второго начального момента $m_2$ распределения $\mathcal{N}\left(m, \sigma^2 \right)$?

    \item $\left( \xi_1, \xi_2, \xi_3 \right)$ --- выборка, $\expectation{\xi_k} = m$, $\variance{\xi_k} = \sigma^2$. Является ли статистика:
          \[
              \varphi(\xi_1, \xi_2, \xi_3) = \frac{\left( \xi_1 - \xi_2 \right)^2 + \left( \xi_2 - \xi_3 \right)^2 + \left( \xi_3 - \xi_1 \right)^2}{6}
          \]
          несмещенной оценкой $\sigma^2$?

    \item $\left( \xi_1, \dots, \xi_n \right)$ --- выборка из распределения Пуассона $\mathcal{P}(\lambda)$. Является ли статистика:
          \[
              \varphi(\xi_1, \dots, \xi_n) = \frac{1}{2 n} \sum_{k=1}^n \xi_k^2.
          \]
          несмещенной и состоятельной оценкой параметра $\lambda$. Примечание: если $\xi \sim \mathcal{P}(\lambda)$, то $\expectation{\xi} = \lambda$
          и $\variance{\xi} = \lambda$.

    \item $\left( \xi_1, \dots, \xi_n \right)$ --- выборка из равномерного распределения $\mathcal{R}\left[a, b \right]$, где $a$ --- известная величина,
          а $b$ --- неизвестная. Является ли статистика:
          \[
              \varphi \left(\xi_1, \dots, \xi_n \right) = \frac{2}{n} \sum_{k=1}^n \xi_k - a
          \]
          несмещённой и состоятельной оценкой $b$?
\end{enumerate}

\end{document}