\documentclass[a4paper,12pt]{article}
\usepackage[T1]{fontenc}
\usepackage[utf8]{inputenc}
\usepackage[english,russian]{babel}
\usepackage[margin=2cm]{geometry}
\usepackage{array}

\newcommand{\expectation}[1]{\mathtt{M} \left[ #1 \right]}
\newcommand{\variance}[1]{\mathtt{D} \left[ #1 \right]}
\newcommand{\probability}[1]{P \left\{ #1 \right\}}

\newcolumntype{P}[1]{>{\centering\arraybackslash}p{#1}}

\begin{document}

\title{Математическая статистика: вопросы и задачи к защитам лабораторных работ}
\author{Тигетов Давид Георгиевич}
\maketitle

\section*{Лабораторная работа 2}

\subsection*{Теория}

\begin{enumerate}
      \item Сформулируйте определения выборки, вариационного ряда, эмпирической функции распределения.
      \item Объясните принцип вычисления частот, относительных частот и построения гистрограмм.
      \item Приведите выражения начальных и центральных выборочных моментов.
\end{enumerate}

\subsection*{Практика}

Задача 1 обязательная, из задач 2 -- 5 одна задача по выбору.

\begin{enumerate}
      \item В результате эксперимента получена реализация выборки $x_1 = 0.5$, $x_2 = -1.5$, $x_3 = -0.2$, $x_4 = 1.1$, $x_5 = 0.6$.
            \begin{enumerate}
                  \item Постройте реализацию вариационного ряда.
                  \item Нарисуйте график реализации эмпирической функции распределения.
                  \item Вычислите реализации выборочного среднего и исправленной выборочной дисперсии.
            \end{enumerate}

      \item $\left( \xi_1, \dots, \xi_n \right)$ --- выборка из нормального распределения $\mathcal{N}\left(m, \sigma^2 \right)$, где $m$ --- известная
            величина, а $\sigma^2$ --- неизвестная величина. Является ли статистика
            \[
                  \varphi(\xi_1, \dots, \xi_n) = \frac{1}{n} \sum_{k=1}^n \left( \xi_k - m \right)^2 + \left( \frac{1}{n} \sum_{k=1}^n \xi_k \right)^2.
            \]
            несмещенной и состоятельной оценкой второго начального момента $m_2$ распределения $\mathcal{N}\left(m, \sigma^2 \right)$?

      \item $\left( \xi_1, \xi_2, \xi_3 \right)$ --- выборка, $\expectation{\xi_k} = m$, $\variance{\xi_k} = \sigma^2$. Является ли статистика:
            \[
                  \varphi(\xi_1, \xi_2, \xi_3) = \frac{\left( \xi_1 - \xi_2 \right)^2 + \left( \xi_2 - \xi_3 \right)^2 + \left( \xi_3 - \xi_1 \right)^2}{6}
            \]
            несмещенной оценкой $\sigma^2$?

      \item $\left( \xi_1, \dots, \xi_n \right)$ --- выборка из распределения Пуассона $\mathcal{P}(\lambda)$. Является ли статистика:
            \[
                  \varphi(\xi_1, \dots, \xi_n) = \frac{1}{2 n} \sum_{k=1}^n \xi_k^2.
            \]
            несмещенной и состоятельной оценкой параметра $\lambda$. Примечание: если $\xi \sim \mathcal{P}(\lambda)$, то $\expectation{\xi} = \lambda$
            и $\variance{\xi} = \lambda$.

      \item $\left( \xi_1, \dots, \xi_n \right)$ --- выборка из равномерного распределения $\mathcal{R}\left[a, b \right]$, где $a$ --- известная величина,
            а $b$ --- неизвестная. Является ли статистика:
            \[
                  \varphi \left(\xi_1, \dots, \xi_n \right) = \frac{2}{n} \sum_{k=1}^n \xi_k - a
            \]
            несмещённой и состоятельной оценкой $b$?
\end{enumerate}

\section*{Лабораторная работа 3}

\subsection*{Теория}

Вопрос 1 обязательный, из вопросов 2 -- 4 один по выбору.
\begin{enumerate}
      \item Сформулируйте задачу точечного оценивания.
      \item Опишите метод моментов построения точечных оценок.
      \item Опишите метод максимального правдоподобия построения точечных оценок.
      \item Опишите метод порядковых статистик построения точечных оценок.
\end{enumerate}

\subsection*{Практика}

\begin{enumerate}
      \item Случайные величины $\xi_1 \sim \mathcal{N} \left( m, \sigma_1^2 \right)$ и $\xi_2 \sim \mathcal{N} \left( m, \sigma_2^2 \right)$,
            $m$ --- неизвестный параметр, $\sigma_1$ и $\sigma_2$ --- известные величины. Постройте оценку $m$ методом максимального правдоподобия.

      \item $\left( \xi_1, \dots, \xi_n \right)$ --- выборка из распределения Пуассона $\mathcal{P}(\lambda)$. Постройте оценку
            параметра $\lambda$ методом максимального правдоподобия.

      \item $\left( \xi_1, \dots, \xi_n \right)$ --- выборка из равномерного распределения $\mathcal{R} \left[ 0, \theta \right]$.
            Оцените вероятность $\probability{\xi > 1}$, если $\xi \sim \mathcal{R} \left[ 0, \theta \right]$, используя метод
            порядковых статистик.

      \item $\left( \xi_1, \dots, \xi_n \right)$ --- выборка из равномерного распределения $\mathcal{R} \left[ a, b \right]$.
            Постройте оценку $\expectation{\xi^2}$, если $\xi \sim \mathcal{R} \left[ a, b \right]$.

      \item Случайная величина $\xi$ имеет биномиальное распределение $Bi \left(n, p \right)$, $n$ --- известно, найдите оценку
            параметра $p$ методом максимального правдоподобия.

\end{enumerate}

\section*{Лабораторная работа 4}

\subsection*{Теория}

\begin{enumerate}
      \item Сформулируйте определения доверительного интервала, нижней и верхней доверительной границы.
      \item Опишите метод построения доверительных интервалов с помощью центральной статистики.
\end{enumerate}
В следующих вопросах достаточно выписать центральную статистику, указать её распределение, выписать неравенство с центральной статистикой с указанием
порядков квантилей и преобразовать неравенство в интервал.
\begin{enumerate}
      \item Постройте доверительный интервал для математического ожидания по выборке из нормального распределения с известной дисперсией.
      \item Постройте доверительный интервал для математического ожидания по выборке из нормального распределения с неизвестной дисперсией.
      \item Постройте доверительный интервал для дисперсии по выборке из нормального распределения с известным математическим ожиданием.
      \item Постройте доверительный интервал для дисперсии по выборке из нормального распределения с неизвестным математическим ожиданием.
      \item Постройте приближённый доверительный интервал для вероятности события по выборке из распределения Бернулли.
\end{enumerate}
Укажите как изменяются границы построенных интервалов в случаях:
\begin{enumerate}
      \item увеличения уровня доверия,
      \item увеличения объёма выборки.
\end{enumerate}

\subsection*{Практика}

\begin{enumerate}
      \item \cite[157]{Efimov} Измерены ёмкости $n = 16$ конденсаторов с известным среднеквадратическим отклонением ёмкости $\sigma = 4$ мкФ, среднее
            значение ёмкости конденсаторов составило $m = 20$ мкФ. Вычислите нижнюю доверительную границу ёмкости одного конденсатора.

      \item Ошибки прибора имеет известное математическое ожидание равное нулю, а в результате измерений получены следующие значения ошибки:
            -1, -0.2, 0.8, -0.4. Укажите доверительный интервал для дисперсии и среднеквадратического отклонения ошибки с уровнем доверия 0.95.

      \item \cite[160]{Efimov} Вычислите доверительный интервал с уровнем доверия 0.9 для среднего содержания углерода, если в результате
            обработки измерений содержания углерода среди $n = 25$ образцов получены значения выборочного среднего $m = 18$ граммов и исправленной
            выборочной дисперсии $s^2 = 16$ граммов$^2$.

      \item Считая количество машин проезжающих по автостраде потоком Пуассона, вычислите нижнюю доверительную границу с уровнем доверия 0.95
            для параметра потока $\lambda$, если за $T = 5$ минут проехало $m = 150$ машин.
\end{enumerate}

\section*{Лабораторная работа 5}

\subsection*{Теория}

Основной вопрос:

\begin{itemize}
      \item формулировка задачи проверки гипотез и общий принцип её решения.
\end{itemize}

Для следующих задач (одной или нескольких, по вариантам)
\begin{enumerate}
      \item критерий хи-квадрат проверки простой гипотезы о вероятностях,
      \item критерий хи-квадрат проверки сложной гипотезы о вероятностях,
      \item проверка гипотезы о независимости признаков,
      \item проверка гипотезы об однородности,
      \item критерий согласия Колмогорова,
      \item критерий Колмогорова -- Смирнова,
      \item критерий Фишера,
      \item критерий Стьюдента
\end{enumerate}
укажите
\begin{enumerate}
      \item cовокупность наблюдений,
      \item основную и альтернативную гипотезы,
      \item статистику критерия, её распределение при основной гипотезе, её значения при альтернативной гипотезе,
      \item критическую область.
\end{enumerate}

\subsection*{Практика}

\begin{enumerate}
      \item \cite[306]{Efimov} Во время эпидемии гриппа изучалась эффективность прививок. Получены следующие результаты:

            \begin{center}
                  \begin{tabular}{|P{3cm}|P{3cm}|P{3cm}|}
                        \hline
                                          & Заболели & Не заболели \\
                        \hline
                        Без \par прививки & 34       & 111         \\
                        \hline
                        С \par прививкой  & 4        & 192         \\
                        \hline
                  \end{tabular}
            \end{center}

            При каких уровнях значимости гипотеза об эффективности прививки будет отклоняться?

      \item \cite[285]{Efimov} Объём сбыта в пяти районах:

            \begin{center}
                  \begin{tabular}{|P{1.5cm}|P{1cm}|P{1cm}|P{1cm}|P{1cm}|P{1cm}|}
                        \hline
                        Район & 1   & 2   & 3  & 4  & 5   \\
                        \hline
                        Объём & 110 & 130 & 70 & 90 & 100 \\
                        \hline
                  \end{tabular}
            \end{center}

            При каких уровнях значимости принимается гипотеза о равномерности объёма сбыта?

      \item \cite[3.17]{Ivchenko} Среди 2020 семей, имеющих двух детей, 527 семей, в которых два мальчика, и 476 --- две девочки, а в остальных
            1017 семьях дети разного пола. При каком наибольшем уровне значимости критерий согласия Колмогорова принимает гипотезу о том, что
            количество мальчиков в семьях имеет биномиальное распределение $Bi \left(2020, 0.65\right)$.

      \item \cite[246]{Efimov}

            При $n = 600$ подбрасываниях игральной кости шестёрка выпала $m = 75$ раз.
            \begin{enumerate}
                  \item Можно ли утверждать, что вероятность выпадения шестёрки равна $\frac{1}{6}$ при уровне значимости $\alpha = 0.05$?
                  \item Можно ли утверждать, что вероятность выпадения шестёрки меньше $\frac{1}{6}$ при уровне значимости $\alpha = 0.01$?
            \end{enumerate}

      \item По измерениям от двух приборов вычислены исправленные выборочные дисперсии: $\widetilde{\mu}_2^{(1)} = 1.3$ и $\widetilde{\mu}_2^{(2)} = 1.4$.
            Считая измерения реализациями независимых выборок из нормального распределения определите наибольший уровень значимости, при котором принимается
            гипотеза об одинаковой точности измерений приборов.

      \item \cite[281]{Efimov} При 50 подбрасываниях монеты герб появился 20 раз.
            При каких уровнях значимости монету можно считать симметричной?

\end{enumerate}

\begin{thebibliography}{2}
      \bibitem{Efimov} Сборник задач по математике, часть 4 // Под редакцией Ефимова А.В. и Поспелова А.С. - М.: Издательство Физико-математической
      литературы, 2003. - 432 с.
      \bibitem{Ivchenko} Сборник задач по математической статистике // Г.И. Ивченко, Ю.И. Медведев, А.В. Чистяков - М.: Высшая школа, 1989. - 255 с.
\end{thebibliography}

\end{document}