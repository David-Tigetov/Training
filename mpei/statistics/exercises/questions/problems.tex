\documentclass[a4paper,12pt]{article}
\usepackage[T1]{fontenc}
\usepackage[utf8]{inputenc}
\usepackage[english,russian]{babel}
\usepackage[margin=2cm]{geometry}
\usepackage{array}
\usepackage{amsmath}
\usepackage{xcolor}

\newcommand{\expectation}[1]{\mathtt{M} \left[ #1 \right]}
\newcommand{\variance}[1]{\mathtt{D} \left[ #1 \right]}
\newcommand{\probability}[1]{P \left\{ #1 \right\}}
\newcommand{\modulus}[1]{\left | #1 \right |}
\newcommand{\fpd}[1]{\frac{\partial}{\partial #1}}

\newcolumntype{P}[1]{>{\centering\arraybackslash}p{#1}}

\begin{document}

\title{Математическая статистика: вопросы и задачи к защитам лабораторных работ}
\author{Тигетов Давид Георгиевич}
\maketitle

\section*{Лабораторная работа 2}

\subsection*{Теория}

\begin{enumerate}
      \item Сформулируйте определения выборки, вариационного ряда, эмпирической функции распределения.
      \item Объясните принцип вычисления частот, относительных частот и построения гистрограмм.
      \item Приведите выражения начальных и центральных выборочных моментов.
\end{enumerate}

\subsection*{Практика}

Задача 1 обязательная, из задач 2 -- 5 одна задача по выбору.

\begin{enumerate}
      \item В результате эксперимента получена реализация выборки $x_1 = 0.5$, $x_2 = -1.5$, $x_3 = -0.2$, $x_4 = 1.1$, $x_5 = 0.6$.
            \begin{enumerate}
                  \item Постройте реализацию вариационного ряда.
                  \item Нарисуйте график реализации эмпирической функции распределения.
                  \item Вычислите реализации выборочного среднего и исправленной выборочной дисперсии.
            \end{enumerate}

      \item $\left( \xi_1, \dots, \xi_n \right)$ --- выборка из нормального распределения $\mathcal{N}\left(m, \sigma^2 \right)$, где $m$ --- известная
            величина, а $\sigma^2$ --- неизвестная величина. Является ли статистика
            \[
                  \varphi(\xi_1, \dots, \xi_n) = \frac{1}{n} \sum_{k=1}^n \left( \xi_k - m \right)^2 + \left( \frac{1}{n} \sum_{k=1}^n \xi_k \right)^2.
            \]
            несмещенной и состоятельной оценкой второго начального момента $m_2$ распределения $\mathcal{N}\left(m, \sigma^2 \right)$?

      \item $\left( \xi_1, \xi_2, \xi_3 \right)$ --- выборка, $\expectation{\xi_k} = m$, $\variance{\xi_k} = \sigma^2$. Является ли статистика:
            \[
                  \varphi(\xi_1, \xi_2, \xi_3) = \frac{\left( \xi_1 - \xi_2 \right)^2 + \left( \xi_2 - \xi_3 \right)^2 + \left( \xi_3 - \xi_1 \right)^2}{6}
            \]
            несмещенной оценкой $\sigma^2$?

      \item $\left( \xi_1, \dots, \xi_n \right)$ --- выборка из распределения Пуассона $\mathcal{P}(\lambda)$. Является ли статистика:
            \[
                  \varphi(\xi_1, \dots, \xi_n) = \frac{1}{2 n} \sum_{k=1}^n \xi_k^2.
            \]
            несмещенной и состоятельной оценкой параметра $\lambda$. Примечание: если $\xi \sim \mathcal{P}(\lambda)$, то $\expectation{\xi} = \lambda$
            и $\variance{\xi} = \lambda$.

      \item $\left( \xi_1, \dots, \xi_n \right)$ --- выборка из равномерного распределения $\mathcal{R}\left[a, b \right]$, где $a$ --- известная величина,
            а $b$ --- неизвестная. Является ли статистика:
            \[
                  \varphi \left(\xi_1, \dots, \xi_n \right) = \frac{2}{n} \sum_{k=1}^n \xi_k - a
            \]
            несмещённой и состоятельной оценкой $b$?
\end{enumerate}

\section*{Лабораторная работа 3}

\subsection*{Теория}

Вопрос 1 обязательный, из вопросов 2 -- 4 один по выбору.
\begin{enumerate}
      \item Сформулируйте задачу точечного оценивания.
      \item Опишите метод моментов построения точечных оценок.
      \item Опишите метод максимального правдоподобия построения точечных оценок.
      \item Опишите метод порядковых статистик построения точечных оценок.
\end{enumerate}

\subsection*{Практика}

\begin{enumerate}
      \item Случайные величины $\xi_1 \sim \mathcal{N} \left( m, \sigma_1^2 \right)$ и $\xi_2 \sim \mathcal{N} \left( m, \sigma_2^2 \right)$,
            $m$ --- неизвестный параметр, $\sigma_1$ и $\sigma_2$ --- известные величины. Постройте оценку $m$ методом максимального правдоподобия.

      \item $\left( \xi_1, \dots, \xi_n \right)$ --- выборка из распределения Пуассона $\mathcal{P}(\lambda)$. Постройте оценку
            параметра $\lambda$ методом максимального правдоподобия.

      \item $\left( \xi_1, \dots, \xi_n \right)$ --- выборка из равномерного распределения $\mathcal{R} \left[ 0, \theta \right]$.
            Оцените вероятность $\probability{\xi > 1}$, если $\xi \sim \mathcal{R} \left[ 0, \theta \right]$, используя метод
            порядковых статистик.

      \item $\left( \xi_1, \dots, \xi_n \right)$ --- выборка из равномерного распределения $\mathcal{R} \left[ a, b \right]$.
            Постройте оценку $\expectation{\xi^2}$, если $\xi \sim \mathcal{R} \left[ a, b \right]$.

      \item Случайная величина $\xi$ имеет биномиальное распределение $Bi \left(n, p \right)$, $n$ --- известно, найдите оценку
            параметра $p$ методом максимального правдоподобия.

\end{enumerate}

\section*{Лабораторная работа 4}

\subsection*{Теория}

\begin{enumerate}
      \item Сформулируйте определения доверительного интервала, нижней и верхней доверительной границы.
      \item Опишите метод построения доверительных интервалов с помощью центральной статистики.
\end{enumerate}
В следующих вопросах достаточно выписать центральную статистику, указать её распределение, выписать неравенство с центральной статистикой с указанием
порядков квантилей и преобразовать неравенство в интервал.
\begin{enumerate}
      \item Постройте доверительный интервал для математического ожидания по выборке из нормального распределения с известной дисперсией.
      \item Постройте доверительный интервал для математического ожидания по выборке из нормального распределения с неизвестной дисперсией.
      \item Постройте доверительный интервал для дисперсии по выборке из нормального распределения с известным математическим ожиданием.
      \item Постройте доверительный интервал для дисперсии по выборке из нормального распределения с неизвестным математическим ожиданием.
      \item Постройте приближённый доверительный интервал для вероятности события по выборке из распределения Бернулли.
\end{enumerate}
Укажите как изменяются границы построенных интервалов в случаях:
\begin{enumerate}
      \item увеличения уровня доверия,
      \item увеличения объёма выборки.
\end{enumerate}

\subsection*{Практика}

\begin{enumerate}
      \item \cite[157]{Efimov} Измерены ёмкости $n = 16$ конденсаторов с известным среднеквадратическим отклонением ёмкости $\sigma = 4$ мкФ, среднее
            значение ёмкости конденсаторов составило $m = 20$ мкФ. Вычислите нижнюю доверительную границу ёмкости одного конденсатора.

      \item Ошибки прибора имеет известное математическое ожидание равное нулю, а в результате измерений получены следующие значения ошибки:
            -1, -0.2, 0.8, -0.4. Укажите доверительный интервал для дисперсии и среднеквадратического отклонения ошибки с уровнем доверия 0.95.

      \item \cite[160]{Efimov} Вычислите доверительный интервал с уровнем доверия 0.9 для среднего содержания углерода, если в результате
            обработки измерений содержания углерода среди $n = 25$ образцов получены значения выборочного среднего $m = 18$ граммов и исправленной
            выборочной дисперсии $s^2 = 16$ граммов$^2$.

      \item Считая количество машин проезжающих по автостраде потоком Пуассона, вычислите нижнюю доверительную границу с уровнем доверия 0.95
            для параметра потока $\lambda$, если за $T = 5$ минут проехало $m = 150$ машин.
\end{enumerate}

\section*{Лабораторная работа 5}

\subsection*{Теория}

Основной вопрос:

\begin{itemize}
      \item формулировка задачи проверки гипотез и общий принцип её решения.
\end{itemize}

Для следующих задач (одной или нескольких, по вариантам)
\begin{enumerate}
      \item критерий хи-квадрат проверки простой гипотезы о вероятностях,
      \item критерий хи-квадрат проверки сложной гипотезы о вероятностях,
      \item проверка гипотезы о независимости признаков,
      \item проверка гипотезы об однородности,
      \item критерий согласия Колмогорова,
      \item критерий Колмогорова -- Смирнова,
      \item критерий Фишера,
      \item критерий Стьюдента
\end{enumerate}
укажите
\begin{enumerate}
      \item cовокупность наблюдений,
      \item основную и альтернативную гипотезы,
      \item статистику критерия, её распределение при основной гипотезе, её значения при альтернативной гипотезе,
      \item критическую область.
\end{enumerate}

\subsection*{Практика}

\begin{enumerate}
      \item \cite[306]{Efimov} Во время эпидемии гриппа изучалась эффективность прививок. Получены следующие результаты:

            \begin{center}
                  \begin{tabular}{|P{3cm}|P{3cm}|P{3cm}|}
                        \hline
                                          & Заболели & Не заболели \\
                        \hline
                        Без \par прививки & 34       & 111         \\
                        \hline
                        С \par прививкой  & 4        & 192         \\
                        \hline
                  \end{tabular}
            \end{center}

            Проверить гипотезу о независимости заболеваемости и привики для уровня значимости $\alpha = 0.05$.

            Решение: \par
                  Оценки вероятностей $\widehat{p}_A = (0.11143695, 0.88856305)$, $\widehat{p}_B = (0.11143695, 0.88856305)$

                  \begin{tabular}{|c|c|c|c|}
                        \hline
                        $\nu_{ij}$ & $\widehat{p}_{ij}$ & $n \widehat{p}_{ij}$ & $\frac{(\nu_i - n \widehat{p}_{ij})^2}{n \widehat{p}_{ij}}$ \\
                        \hline
                        34         & 0.047              & 16.158               & 19.700                                                      \\
                        111        & 0.064              & 21.842               & 363.948                                                     \\
                        4          & 0.378              & 128.842              & 120.966                                                     \\
                        192        & 0.511              & 174.158              & 1.828                                                       \\
                        \hline
                        $n = 341$  & 1.000              &                      & $X^2 =  506.441$                                            \\
                        \hline
                  \end{tabular}

                  Степени свободы $(2-1) \cdot (2-1) = 1$, порог $\approx 3.84146$, гипотеза о независимости отклоняется.

      \item \cite[285]{Efimov} Объём сбыта в пяти районах:

            \begin{center}
                  \begin{tabular}{|P{1.5cm}|P{1cm}|P{1cm}|P{1cm}|P{1cm}|P{1cm}|}
                        \hline
                        Район & 1   & 2   & 3  & 4  & 5   \\
                        \hline
                        Объём & 110 & 130 & 70 & 90 & 100 \\
                        \hline
                  \end{tabular}
            \end{center}

            При каких уровнях значимости принимается гипотеза о равномерности объёма сбыта?

            Решение: \par
                  \begin{tabular}{|c|c|c|c|}
                        \hline
                        $\nu_i$   & $\widehat{p}_i$ & $n \widehat{p}_i$ & $\frac{(\nu_i - n \widehat{p}_i)^2}{n \widehat{p}_i}$ \\
                        \hline
                        110       & 0.200           & 100.000           & 1.000                                                 \\
                        130       & 0.200           & 100.000           & 9.000                                                 \\
                        70        & 0.200           & 100.000           & 9.000                                                 \\
                        90        & 0.200           & 100.000           & 1.000                                                 \\
                        100       & 0.200           & 100.000           & 0.000                                                 \\
                        \hline
                        $n = 500$ & 1.000           &                   & $X^2 = 20$                                            \\
                        \hline
                  \end{tabular}

                  Степень свободы $5 - 1 = 4$, принимается при уровнях значимости более 0.0005.

      \item \cite[3.17]{Ivchenko} Среди 2020 семей, имеющих двух детей, 527 семей, в которых два мальчика, и 476 --- две девочки, а в остальных
            1017 семьях дети разного пола. Проверьте гипотезу о том, что количество мальчиков в семьях имеет биномиальное распределение,
            по критерию согласия Колмогорова при уровне значимости $\alpha = 0.1$.

            Решение: \par
                  Оценка вероятности $p$ рождения мальчика по методу максимального правдоподобия:
                  \begin{gather*}
                        \Pi(\nu_0, \nu_1, \nu_2) = \frac{2020!}{\nu_0! \nu_1! \nu_2!} (1-p)^{2 \nu_0} (2p(1-p))^{\nu_1} p^{2\nu_2} , \\
                        \ln \Pi(\nu_0, \nu_1, \nu_2) = \ln 2020! - \ln \left( \nu_0! \nu_1! \nu_2! \right) + 2 \nu_0 \ln (1-p) + \nu_1 \ln (2p - 2 p^2) + 2 \nu_2 \ln p , \\
                        \fpd{p} \ln \Pi(\nu_0, \nu_1, \nu_2) = - 2 \nu_0 \frac{1}{1-\widehat{p}} + \nu_1 \frac{2 - 4\widehat{p}}{2\widehat{p}(1-\widehat{p})} + 2 \nu_2 \frac{1}{\widehat{p}} = 0, \\
                        - 2 \nu_0 \widehat{p} + \nu_1 (1 - 2\widehat{p}) + 2 \nu_2 (1-\widehat{p}) = 0, \\
                        \nu_1 + 2 \nu_2 = (2 \nu_0 + 2 \nu_1 + 2 \nu_2) \widehat{p}, \\
                        \widehat{p} = \frac{ \frac{1}{2} \nu_1 + \nu_2}{\nu_0 + \nu_1 + \nu_2} = \frac{ \frac{1}{2} 1017 + 527}{476 + 1017 + 527} = \frac{1035.5}{2020} \approx 0.51262.
                  \end{gather*}

                  \begin{tabular}{|c|c|c|c|}
                        \hline
                        x                                 & 0          & 1          & 2 \\
                        \hline
                        Эмпирическая, $F_n^*(x)$          & 0.23564356 & 0.73910891 & 1 \\
                        \hline
                        Теоретическая, $F(x;\widehat{p})$ & 0.23753926 & 0.73722074 & 1 \\
                        \hline
                        Разность                          & -0.0018957 & 0.00188818 & 0 \\
                        \hline
                  \end{tabular}

                  Статистика $D = \sqrt{n} sup \modulus{F_n^*(x) - F(x;\widehat{p})} \approx 0.085201$, порог $h_\alpha \approx 1.22385$.
                  Гипотеза принимается.

      \item \cite[246]{Efimov}

            При $n = 600$ подбрасываниях игральной кости шестёрка выпала $m = 75$ раз. Можно ли утверждать, что вероятность выпадения шестёрки равна
            $\frac{1}{6}$ при уровне значимости $\alpha = 0.05$?

            Решение: \par
                  \begin{tabular}{|c|c|c|c|}
                        \hline
                        $\nu_i$   & $\widehat{p}_i$ & $n \widehat{p}_i$ & $\frac{(\nu_i - n \widehat{p}_i)^2}{n \widehat{p}_i}$ \\
                        \hline
                        75        & 0.167           & 100.000           & 6.250                                                 \\
                        525       & 0.833           & 500.000           & 1.250                                                 \\
                        600       & 1.000           &                   & 7.500                                                 \\
                        \hline
                        $n = 600$ & 1.000           &                   & $X^2 = 7.5$                                           \\
                        \hline
                  \end{tabular}

                  Степень свободы $2 - 1 = 1$, порог $\probability{\chi_1^2 \ge 7.5} \approx 3.84$.

      \item По $n_1 = 15$ измерениям одного прибора и $n_2 = 20$ измерениям второго прибора вычислены исправленные выборочные дисперсии:
            $\widetilde{\mu}_2^{(1)} = 1.3$ и $\widetilde{\mu}_2^{(2)} = 1.4$. Считая измерения реализациями независимых выборок из нормального
            распределения проверьте гипотезу одинаковой точности измерений приборов против альтернативной гипотезы, утверждающей что ошибка второго
            прибора больше, для уровня значимости $\alpha = 0.1$.

            Решение: \par
                  Статистика $f = \frac{\widetilde{\mu}_2^{(1)}}{\widetilde{\mu}_2^{(2)}} \approx 0.92857$, правосторонняя критическая область
                  $\Gamma_\alpha = (1.87847, \infty)$.

      \item \cite[281]{Efimov} При 50 подбрасываниях монеты герб появился 20 раз. При каких уровнях значимости монету можно считать симметричной?
            Решение: \par
                  \begin{tabular}{|c|c|c|c|}
                        \hline
                        $\nu_i$  & $\widehat{p}_i$ & $n \widehat{p}_i$ & $\frac{(\nu_i - n \widehat{p}_i)^2}{n \widehat{p}_i}$ \\
                        \hline
                        20       & 0.500           & 25.000            & 1.000                                                 \\
                        30       & 0.500           & 25.000            & 1.000                                                 \\
                        \hline
                        $n = 50$ & 1.000           &                   & $X^2 = 2.0$                                           \\
                        \hline
                  \end{tabular}

                  Степень свободы $2 - 1 = 1$, гипотеза о симметричности принимается при уровнях значимости менее
                  $\probability{\chi_1^2 \ge 2.0} \approx 0.15730$.

\end{enumerate}

\section*{Лабораторная работа 6}

\subsection*{Теория}

Однофакторный дисперсионный анализ:
\begin{enumerate}
      \item постановка задачи: наблюдения, основная и альтернативная гипотезы,
      \item основное дисперсионное соотношение, внутригрупповая и межгрупповая дисперсии,
      \item статистика критерия: её распределение при основной гипотезе, её значения при альтернативной гипотезе,
      \item гритическая область: порог по уровню значимости.
\end{enumerate}

\subsection*{Практика}

Для полученных выборок проверьте гипотезу о равенстве математических ожиданий при уровне доверия $\alpha = 0.1$.

Правая часть таблицы после двойной линии является решением.

\begin{enumerate}
      \item
            \begin{tabular}{|P{0.8cm}|P{0.8cm}|P{0.8cm}|P{0.8cm}|P{1cm}||P{2cm}|P{1.5cm}|P{0.5cm}|P{2cm}|P{2.5cm}|}
                  \hline
                     &    &   &   &                & $\sum_{i=1}^{n_j} \left( \xi_i^{(j)} \right)^2$  & $\sum_{i=1}^{n_j} \xi_i^{(j)}$ & $n_j$ & $\left( \sum_{i=1}^{n_j} \xi_i^{(j)} \right)^2$ & $\frac{1}{n_j} \left( \sum_{i=1}^{n_j} \xi_i^{(j)} \right)^2$ \\
                  \hline
                  8  & 5  & 3 & 9 & -              & 179                                              & 25                             & 4     & 625                                             & 156.25                                                        \\
                  \hline
                  10 & 18 & 6 & 5 & 8              & 549                                              & 47                             & 5     & 2209                                            & 441.8                                                         \\
                  \hline
                  \hline
                     &    &   &   & $\sum_{j=1}^k$ & 728                                              & 72                             & 9     &                                                 & 598.05                                                        \\
                  \hline
                     &    &   &   &                & $n \widehat{\mu}_2 = 728 - \frac{72^2}{9} = 152$ & $\frac{72^2}{9} = 576$         &       &                                                 & $n \widetilde{s}^2 = 598.05 - \frac{72^2}{9} = 22.05$         \\
                  \hline
            \end{tabular}
            $n \widehat{s}^2(\xi) = n \widehat{\mu}_2 - n \widetilde{s}^2 = 152 - 22.05 = 129.95$,
            $T = \frac{n \widetilde{s}^2}{2 - 1} \frac{9 - 2}{n \widehat{s}^2} = \frac{22.05}{2-1} \frac{9 - 2}{129.95} \approx 1.188$,
            $h_\alpha = Q_{F(2-1,7)}(1 - \alpha) \approx 3.589$,
            $T < h_\alpha$ : гипотеза принимается.

      \item
            \begin{tabular}{|P{0.8cm}|P{0.8cm}|P{0.8cm}|P{0.8cm}|P{1cm}||P{2cm}|P{1.5cm}|P{0.5cm}|P{2cm}|P{2.5cm}|}
                  \hline
                     &    &    &   &                & $\sum_{i=1}^{n_j} \left( \xi_i^{(j)} \right)^2$ & $\sum_{i=1}^{n_j} \xi_i^{(j)}$ & $n_j$ & $\left( \sum_{i=1}^{n_j} \xi_i^{(j)} \right)^2$ & $\frac{1}{n_j} \left( \sum_{i=1}^{n_j} \xi_i^{(j)} \right)^2$ \\
                  \hline
                  12 & 13 & 10 & 1 & 14             & 610                                             & 50                             & 5     & 2500                                            & 500                                                           \\
                  \hline
                  -1 & 1  & 6  & - & -              & 38                                              & 6                              & 3     & 36                                              & 12                                                            \\
                  \hline
                  \hline
                     &    &    &   & $\sum_{j=1}^k$ & 648                                             & 56                             & 8     &                                                 & 512                                                           \\
                  \hline
                     &    &    &   &                & $648 - \frac{56^2}{8} = 256$                    & $\frac{56^2}{8} = 392$         &       &                                                 & $512 - \frac{56^2}{8} = 120$                                  \\
                  \hline
            \end{tabular}
            $n \widehat{s}^2(\xi) = 256 - 120 = 136$,
            $T = \frac{120}{2-1} \frac{8 - 2}{136} \approx 5.294$,
            $h_\alpha = Q_{F(2-1,8-2)}(1 - \alpha) \approx 3.776$,
            $T > h_\alpha$ : гипотеза отклоняется.

      \item
            \begin{tabular}{|P{0.8cm}|P{0.8cm}|P{0.8cm}|P{0.8cm}|P{1cm}||P{2cm}|P{1.5cm}|P{0.5cm}|P{2cm}|P{2.5cm}|}
                  \hline
                     &   &   &    &                & $\sum_{i=1}^{n_j} \left( \xi_i^{(j)} \right)^2$ & $\sum_{i=1}^{n_j} \xi_i^{(j)}$ & $n_j$ & $\left( \sum_{i=1}^{n_j} \xi_i^{(j)} \right)^2$ & $\frac{1}{n_j} \left( \sum_{i=1}^{n_j} \xi_i^{(j)} \right)^2$ \\
                  \hline
                  13 & 8 & 8 & 5  & -              & 322                                             & 34                             & 4     & 1156                                            & 289                                                           \\
                  \hline
                  2  & 6 & 0 & 13 & -              & 209                                             & 21                             & 4     & 441                                             & 110.25                                                        \\
                  \hline
                  \hline
                     &   &   &    & $\sum_{j=1}^k$ & 531                                             & 55                             & 8     &                                                 & 399.25                                                        \\
                  \hline
                     &   &   &    &                & $531 - \frac{55^2}{8} = 152.875$                & $\frac{55^2}{8} = 378.125$     &       &                                                 & $399.25 - \frac{55^2}{8} = 21.125$                            \\
                  \hline
            \end{tabular}
            $n \widehat{s}^2(\xi) = 152.875 - 21.125 = 131.75$,
            $T = \frac{21.125}{2-1} \frac{8 - 2}{131.75} \approx 0.962$,
            $h_\alpha = Q_{F(2-1,8-2)}(1 - \alpha) \approx 3.776$,
            $T < h_\alpha$ : гипотеза принимается.

      \item
            \begin{tabular}{|P{0.8cm}|P{0.8cm}|P{0.8cm}|P{0.8cm}|P{1cm}||P{2cm}|P{1.5cm}|P{0.5cm}|P{2cm}|P{2.5cm}|}
                  \hline
                     &    &    &   &                & $\sum_{i=1}^{n_j} \left( \xi_i^{(j)} \right)^2$ & $\sum_{i=1}^{n_j} \xi_i^{(j)}$ & $n_j$ & $\left( \sum_{i=1}^{n_j} \xi_i^{(j)} \right)^2$ & $\frac{1}{n_j} \left( \sum_{i=1}^{n_j} \xi_i^{(j)} \right)^2$ \\
                  \hline
                  11 & 15 & 10 & - & -              & 446                                             & 36                             & 3     & 1296                                            & 432                                                           \\
                  \hline
                  9  & 8  & 4  & 8 & -              & 225                                             & 29                             & 4     & 841                                             & 210.25                                                        \\
                  \hline
                  \hline
                     &    &    &   & $\sum_{j=1}^k$ & 671                                             & 65                             & 7     &                                                 & 642.25                                                        \\
                  \hline
                     &    &    &   &                & $671 - \frac{65^2}{7} = 67.429$                 & $\frac{65^2}{7} = 603.57$      &       &                                                 & $642.25 - \frac{65^2}{7} = 38.679$                            \\
                  \hline
            \end{tabular}
            $n \widehat{s}^2(\xi) = 67.429 - 38.679 = 28.75$,
            $T = \frac{38.679}{2-1} \frac{7 - 2}{28.75} \approx 6.727$,
            $h_\alpha = Q_{F(2-1,7-2)}(1 - \alpha) \approx 4.06$,
            $T > h_\alpha$ : гипотеза отклоняется.

      \item
            \begin{tabular}{|P{0.8cm}|P{0.8cm}|P{0.8cm}|P{0.8cm}|P{1cm}||P{2cm}|P{1.5cm}|P{0.5cm}|P{2cm}|P{2.5cm}|}
                  \hline
                     &   &   &   &                & $\sum_{i=1}^{n_j} \left( \xi_i^{(j)} \right)^2$ & $\sum_{i=1}^{n_j} \xi_i^{(j)}$ & $n_j$ & $\left( \sum_{i=1}^{n_j} \xi_i^{(j)} \right)^2$ & $\frac{1}{n_j} \left( \sum_{i=1}^{n_j} \xi_i^{(j)} \right)^2$ \\
                  \hline
                  16 & 9 & 7 & - & -              & 386                                             & 32                             & 3     & 1024                                            & 341.(3)                                                       \\
                  \hline
                  -2 & 6 & 3 & - & -              & 49                                              & 7                              & 3     & 49                                              & 16.(3)                                                        \\
                  \hline
                  \hline
                     &   &   &   & $\sum_{j=1}^k$ & 435                                             & 39                             & 6     &                                                 & 357.(6)                                                       \\
                  \hline
                     &   &   &   &                & $435 - \frac{39^2}{6} = 181.5$                  & $\frac{39^2}{6} = 253.5$       &       &                                                 & $357.(6) - \frac{39^2}{6} = 104.167$                          \\
                  \hline
            \end{tabular}
            $n \widehat{s}^2(\xi) = 181.5 - 104.167 = 77.333$,
            $T = \frac{104.167}{2-1} \frac{6 - 2}{77.333} \approx 5.388$,
            $h_\alpha = Q_{F(2-1,6-2)}(1 - \alpha) \approx 4.545$,
            $T > h_\alpha$ : гипотеза отклоняется.
\end{enumerate}

\section*{Лабораторная работа 7}

\subsection*{Теория}

Вопросы по двум вариантам.

В группах, где задача 1, 2 или 5 (с фиксированным количеством величин), можно дать вопросы из второго варианта (с нефиксированным количеством величин),
и наоборот, где задача 3 или 4, можно дать вопросы из первого варианта.

\begin{enumerate}
      \item Первый вариант --- задача различения гипотез при фиксированном количестве величин $n$:
            \begin{enumerate}
                  \item постановка задачи,
                  \item критерий проверки, вероятности ошибок первого и второго рода,
                  \item минимаксный, байесовский, наиболее мощный критерии,
                  \item критерий отношения вероятностей,
                  \item минимаксный, байесовский, наиболее мощный критерии как критерии отношения вероятностей.
            \end{enumerate}
      \item Второй вариант --- задача различения гипотез при нефиксированном количестве величин $n$:
            \begin{enumerate}
                  \item постановка задачи,
                  \item принцип проверки последовательными критериями,
                  \item последовательный критерий отношения вероятностей,
                  \item количества величин до остановки, их средние значения.
            \end{enumerate}
\end{enumerate}

\subsection*{Практика}

\begin{enumerate}
      \item Задана выборка $\left( \xi_1, \dots, \xi_n \right)$ из нормального распределения $\mathcal{N}(0, \sigma^2)$, где $\sigma$ --- неизвестный
            параметр. Рассматриваются две гипотезы:
            \begin{gather*}
                  H_0: \sigma = \sigma_0 , \\
                  H_1: \sigma = \sigma_1 ,
            \end{gather*}
            где $\sigma_0 = 1.5$ и $\sigma_1 = 1$. Постройте критерий Неймана--Пирсона для $n = 100$ и вероятности ошибки первого рода $\alpha = 0.05$.
            Вычислите вероятность ошибки второго рода.

      \item Задана выборка $\left( \xi_1, \dots, \xi_n \right)$ из распределения Лапласа с плотностью $p_\xi(x | \sigma)$:
            \[
                  p_\xi(x | \sigma)
                  = \frac{\sigma}{2} e^{- \sigma \modulus{x}} .
            \]
            Параметр $\sigma$ неизвестен. Рассматриваются гипотезы:
            \begin{gather*}
                  H_0: \sigma = \sigma_0 , \\
                  H_1: \sigma = \sigma_1 ,
            \end{gather*}
            где $\sigma_0 = 2$, $\sigma_1 = 1$. Постройте критерий Неймана--Пирсона для вероятности ошибки первого рода $\alpha = 0.01$ и вероятности
            ошибки второго рода $\beta = 0.05$.

      \item Задана выборка $\left( \xi_1, \dots, \xi_n \right)$ из геометрического распределения $Geom(p)$ с распределением:
            \begin{gather*}
                  \probability{\xi_i = k} = (1-p)^{k-1} p, \\
                  k = 1, 2, \dots \\
                  \expectation{\xi_i} = \frac{1}{p} , \variance{\xi_i} = \frac{1-p}{p^2} .
            \end{gather*}
            Вероятность $p$ неизвестна. Рассматриваются гипотезы:
            \begin{gather*}
                  H_0: p = p_0 , \\
                  H_1: p = p_1 ,
            \end{gather*}
            где $p_0 = 0.6$ и $p_1 = 0.5$. Постройте последовательный критерий отношения вероятностей для вероятностей ошибок $\alpha = 0.03$
            и $\beta = 0.01$. Определите средние количества наблюдений до остановки критерия.


      \item Задана выборка $\left( \xi_1, \dots, \xi_n \right)$ бинарных величин с распределением:
            \begin{gather*}
                  \probability{\xi_i = k} = p^k (1-p)^{1-k}, \\
                  k = 0,1.
            \end{gather*}
            Вероятность $p$ неизвестна. Рассматриваются гипотезы:
            \begin{gather*}
                  H_0: p = p_0 , \\
                  H_1: p = p_1 ,
            \end{gather*}
            где $p_0 = 0.5$ и $p_1 = 0.3$. Постройте последовательный критерий отношения вероятностей с вероятностями ошибок $\alpha = 0.02$ и
            $\beta = 0.05$. Определите средние количества наблюдений до остановки критерия.

      \item Задана выборка $\left( \xi_1, \dots, \xi_n \right)$ из распределения с плотностью вероятности $p_\xi(x | \sigma)$:
            \[
                  p_\xi(x) = \left \{
                  \begin{array}{ll}
                        0,                                                         & x < 0   \\
                        \sqrt{\frac{2}{\pi \sigma^2}} e^{-\frac{x^2}{2 \sigma^2}}, & x \ge 0
                  \end{array}
                  \right .
            \]
            Параметр $\sigma$ неизвестен. Рассматриваются гипотезы:
            \begin{gather*}
                  H_0: \sigma = \sigma_0 , \\
                  H_1: \sigma = \sigma_1 ,
            \end{gather*}
            где $\sigma_0 = 3$, $\sigma_1 = 2$. Постройте минимаксный критерий при $n = 100$. Определите вероятности ошибок первого и второго рода
            полученного критерия.
\end{enumerate}


\begin{thebibliography}{2}
      \bibitem{Efimov} Сборник задач по математике, часть 4 // Под редакцией Ефимова А.В. и Поспелова А.С. - М.: Издательство Физико-математической
      литературы, 2003. - 432 с.
      \bibitem{Ivchenko} Сборник задач по математической статистике // Г.И. Ивченко, Ю.И. Медведев, А.В. Чистяков - М.: Высшая школа, 1989. - 255 с.
\end{thebibliography}

\end{document}