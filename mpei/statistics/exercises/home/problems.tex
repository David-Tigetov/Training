\documentclass[a4paper,12pt]{article}
\usepackage[T1]{fontenc}
\usepackage[utf8]{inputenc}
\usepackage[english,russian]{babel}
\usepackage[margin=2cm]{geometry}
\usepackage{amsmath}
\usepackage{array}
\usepackage{xcolor}

\newcommand{\probability}[1]{P \left\{ #1 \right\}}
\newcommand{\modulus}[1]{\left | #1 \right |}
\newcommand{\expectation}[1]{\mathtt{M} \left[ #1 \right]}
\newcommand{\cexpectation}[2]{\mathtt{M} \left[ #1 | #2 \right]}
\newcommand{\variance}[1]{\mathtt{D} \left[ #1 \right]}
\newcommand{\cvariance}[2]{\mathtt{D} \left[ #1 | #2 \right]}
\newcommand{\norm}[1]{\left \| #1 \right \|}
\newcommand{\pr}[2]{#1_\mathcal{#2}}
\newcommand{\pro}[2]{#1_{\mathcal{#2}^\perp}}
\newcommand{\element}[1]{\left \{ #1 \right \}}

\newcolumntype{P}[1]{>{\centering\arraybackslash}p{#1}}

\newif\ifsolutions
\solutionstrue

\begin{document}

\title{Математическая статистика: домашние задания}
\author{Тигетов Давид Георгиевич}
\maketitle

\section*{Домашнее задание 1}

\subsection*{Задача 1 \cite[108]{Efimov}}

Докажите, что выборочное среднее, вычисленное по выборке из распределения Пуассона $\mathcal{P}(\lambda)$, будет несмещённой и состоятельной оценкой
параметра $\lambda$.

\subsection*{Задача 2 \cite[103]{Efimov}}

Пусть $m_1$, $m_2$ и $s_1^2$, $s_2^2$ --- несмещенные и состоятельные оценки параметров $\mu$ и $\sigma^2$, построенные для двух выборок
объёмами $n_1$ и $n_2$. Покажите, что статистики:
\begin{gather*}
    m = \frac{n_1 m_1 + n_2 m_2}{n_1 + n_2} , \\
    s^2 = \frac{(n_1 - 1) s_1^2 + (n_2 - 1) s_2^2}{n_1 + n_2 - 2} .
\end{gather*}
будут несмещёнными и состоятельными оценками $\mu$ и $\sigma^2$.

\subsection*{Задача 3 \cite[107]{Efimov}}

Пусть $\widetilde{\theta}$ --- несмещенная оценка параметра $\theta$, $\variance{\widetilde{\theta}} < \infty$. Докажите, что
$\widetilde{\theta}^2$ является смещенной оценкой $\theta^2$, и вычислите смещение.

\subsection*{Задача 4 \cite[110]{Efimov}}

Пусть $\widetilde{\theta}$ --- состоятельная оценка параметра $\theta$, a $\psi(x)$ --- непрерывная функция в области изменения $\widetilde{\theta}$.
Докажите, что $\psi(\widetilde{\theta})$ является состоятельной оценкой $\psi(\theta)$.

\section*{Домашнее задание 2}

\subsection*{Задача 1 \cite[124]{Efimov}}

Случайные величины $\xi_1$, \dots, $\xi_m$ независимы и имеют биномиальные распределения $Bi(n_1, p)$, \dots, $Bi(n_m, p)$, все $n_k$ известны.
Найдите МП-оценку параметра $p$. Покажите, что МП-оценка является несмещённой и вычислите её дисперсию.

\subsection*{Задача 2}

$\left( \xi_1, \dots, \xi_n \right)$ --- выборка из нормального распределения $\mathcal{N} \left( m, \sigma^2 \right)$. Найдите МП-оценку двумерного
параметра $\left( m, \sigma^2 \right)$.

\subsection*{Задача 3 \cite[126]{Efimov}}

$\left( \xi_1, \dots, \xi_n \right)$ --- выборка из распределения с плотностью:
\[
    p_\xi(x|\theta)
    =
    \left\{
    \begin{array}{ll}
        \theta x & x \in \left[ 0, \sqrt{\frac{2}{\theta}}\right]    \\
        0        & x \notin \left[ 0, \sqrt{\frac{2}{\theta}}\right]
    \end{array}
    \right.
\]
Найти МП-оценку математического ожидания этого распределения.

\subsection*{Задача 4}

$\left( \xi_1, \dots, \xi_n \right)$ --- выборка из распределения Рэлея $\mathcal{R}(\theta)$. Постройте оценки параметра $\theta$ методами моментов
и порядковых статистик. Сравните асимптотические дисперсии оценок.

\section*{Домашнее задание 3}

\subsection*{Задача 1 \cite[164]{Efimov}}

По двум выборкам из нормального распределения $\mathcal{N} \left( \mu, \sigma^2 \right)$ объёмами $n_1$ и $n_2$ вычислены выборочные средние
$m_1$, $m_2$, исправленные выборочные дисперсии $s_1^2$, $s_2^2$ и статистики:
\begin{gather*}
    m = \frac{n_1 m_1 + n_2 m_2}{n_1 + n_2} , \\
    s^2 = \frac{(n_1 - 1) s_1^2 + (n_2 - 1) s_2^2}{n_1 + n_2 - 2} .
\end{gather*}

Покажите, что при известной дисперсии $\sigma^2$ доверительный интервал для $\mu$ уровня доверия $\alpha$ имеет вид:
\[
    m - u(\alpha) \frac{\sigma}{\sqrt{n_1 + n_2}}
    < \mu
    < m + u(\alpha) \frac{\sigma}{\sqrt{n_1 + n_2}}
\]
и укажите $u(\alpha)$, а при неизвестной дисперсии доверительный интервал для $\mu$ уровня доверия $\alpha$ имеет вид:
\[
    m - t(\alpha) \frac{s}{\sqrt{n_1 + n_2}}
    < \mu
    < m + t(\alpha) \frac{s}{\sqrt{n_1 + n_2}}
\]
и укажите $t(\alpha)$.

\subsection*{Задача 2 \cite[176, 177]{Efimov}}

По двум выборкам из нормальных распределений $\mathcal{N} \left( \mu_1, \sigma_1^2 \right)$ и $\mathcal{N} \left( \mu_2, \sigma_2^2 \right)$ объёмами
$n_1$ и $n_2$ вычислены выборочные средние $m_1$, $m_2$, исправленные выборочные дисперсии $s_1^2$, $s_2^2$ и статистика:
\[
    s^2 = \frac{(n_1 - 1) s_1^2 + (n_2 - 1) s_2^2}{n_1 + n_2 - 2} .
\]

Покажите, что если дисперсии $\sigma_1^2$ и $\sigma_2^2$ известны, то доверительный интервал
для разности $\mu_1 - \mu_2$ с уровнем доверия $\alpha$ имеет вид:
\[
    (m_1 - m_2) - u(\alpha) \sqrt{\frac{\sigma_1^2}{n_1} + \frac{\sigma_2^2}{n_2}}
    < \mu_1 - \mu_2
    < (m_1 - m_2) + u(\alpha) \sqrt{\frac{\sigma_1^2}{n_1} + \frac{\sigma_2^2}{n_2}} ,
\]
и укажите $u(\alpha)$. Если дисперсии неизвестны, но равны $\sigma_1^2 = \sigma_2^2$, то доверительный интервал для разности $\mu_1 - \mu_2$
с уровнем доверия $\alpha$ имеет вид:
\[
    (m_1 - m_2) - t(\alpha) s \sqrt{\frac{1}{n_1} + \frac{1}{n_2}}
    < \mu_1 - \mu_2
    < (m_1 - m_2) + t(\alpha) s \sqrt{\frac{1}{n_1} + \frac{1}{n_2}} ,
\]
и укажите $t(\alpha)$.

\subsection*{Задача 3 \cite[187]{Efimov}}

Из урны, содержащей черные и белые шары в неизвестной пропорции, случайным образом извлекается 100 шаров с возвращением. Найти доверительный интервал
доли чёрных шаров с уровнем доверия $\alpha=0.95$, если среди вынутых оказалось 30 чёрных шаров. Сколько извлечений нужно сделать, чтобы относительная
частота отличалась от доли чёрных шаров не более чем на 0.01 с вероятностью 0.99?

\subsection*{Задача 4 \cite[191]{Efimov}}

На каждой из 36 АТС города в период с двух до трех часов было зафиксировано в среднем 2 вызова. Считая, что число вызовов для каждой АТС имеет
распределение Пуассона с одним и тем же параметром $\lambda$, приближённо найти нижнюю и верхнюю \textbf{доверительные границы} с уровнем
доверия $\alpha = 0.9$.

\section*{Домашнее задание 4}

\subsection*{Задача 1 \cite[291]{Efimov}}

При испытании радиоэлектронной аппаратуры фиксировалось число отказов. Результаты испытаний:

\begin{center}
    \begin{tabular}{|P{3cm}|P{1cm}|P{1cm}|P{1cm}|P{1cm}|P{1cm}|P{1cm}|P{1cm}|P{1cm}|P{1cm}|}
        \hline
        Число \par отказов:  & 0  & 1  & 2  & 3  & 4  & 5 & 6 & 7 & 8 \\
        \hline
        Число \par испытаний & 28 & 49 & 57 & 25 & 16 & 4 & 4 & 3 & 1 \\
        \hline
    \end{tabular}
\end{center}

С помощью критерия $\chi^2$ определите уровни значимости, при которых принимаются гипотезы о том, что число отказов имеет:
\begin{enumerate}
    \item распределение Пуассона,
    \item биномиальное распределение.
\end{enumerate}
Какое из двух распределений числа отказов лучше согласуется с данными?

Примечание: объединять наблюдения при $n p_i^0 < 5$.

\ifsolutions Решение: \par
    \begin{enumerate}
        \item оценка $\widehat{\lambda} = \frac{375}{187} \approx 2.00535$, вероятности $\widehat{p}_i = \frac{\widehat{\lambda}^i}{i!} e^{- \widehat{\lambda}}$

              \begin{tabular}{|c|c|c|c|}
                  \hline
                  $\nu_i$   & $\widehat{p}_i$ & $n \widehat{p}_i$      & $\frac{(\nu_i - n \widehat{p}_i)^2}{n \widehat{p}_i}$ \\
                  \hline
                  28        & 0.135           & 25.173                 & 0.318                                                 \\
                  49        & 0.270           & 50.480                 & 0.043                                                 \\
                  57        & 0.271           & 50.615                 & 0.805                                                 \\
                  25        & 0.181           & 33.834                 & 2.306                                                 \\
                  16        & 0.091           & 16.962                 & 0.055                                                 \\
                  4         & 0.036           & \textcolor{red}{6.803} & 1.155                                                 \\
                  4         & 0.012           & \textcolor{red}{2.274} & 1.311                                                 \\
                  3         & 0.003           & \textcolor{red}{0.651} & 8.468                                                 \\
                  1         & 0.001           & \textcolor{red}{0.163} & 4.288                                                 \\
                  \hline
                  $n = 187$ & 1.000           &                        & $X^2 = 18.749$                                        \\
                  \hline
              \end{tabular}
              $\Rightarrow$
              \begin{tabular}{|c|c|c|c|c|}
                  \hline
                  $\nu_i$   & $\widehat{p}_i$ & $n \widehat{p}_i$ & $\frac{(\nu_i - n \widehat{p}_i)}{n \widehat{p}_i}$ \\
                  \hline
                  28        & 0.135           & 25.173            & 0.318                                               \\
                  49        & 0.270           & 50.480            & 0.043                                               \\
                  57        & 0.271           & 50.615            & 0.805                                               \\
                  25        & 0.181           & 33.834            & 2.306                                               \\
                  16        & 0.091           & 16.962            & 0.055                                               \\
                  12        & 0.053           & 9.891             & 0.450                                               \\
                  \hline
                  $n = 187$ & 1.000           &                   & $X^2 = 3.977$                                       \\
                  \hline
              \end{tabular}

              Степени свободы $6-1-1=4$, максимальный уровень значимости $\probability{\chi_4^2 \ge 3.977} \approx 0.40915$.

        \item оценка $\widehat{p} = \frac{375}{187 \cdot 8} = \frac{375}{1496} \approx 0.25067$, вероятности $\widehat{p}_i = C_8^i \widehat{p}^i (1-\widehat{p})^{8-i}$.

              \begin{tabular}{|c|c|c|c|c|}
                  \hline
                  $\nu_i$   & $\widehat{p}_i$ & $n \widehat{p}_i$      & $\frac{(\nu_i - n \widehat{p}_i)}{n \widehat{p}_i}$ \\
                  \hline
                  28        & 0.099           & 18.588                 & 4.766                                               \\
                  49        & 0.266           & 49.745                 & 0.011                                               \\
                  57        & 0.311           & 58.243                 & 0.027                                               \\
                  25        & 0.208           & 38.967                 & 5.006                                               \\
                  16        & 0.087           & 16.294                 & 0.005                                               \\
                  4         & 0.023           & \textcolor{red}{4.361} & 0.030                                               \\
                  4         & 0.004           & \textcolor{red}{0.729} & 14.666                                              \\
                  3         & 0.000           & \textcolor{red}{0.070} & 123.174                                             \\
                  1         & 0.000           & \textcolor{red}{0.003} & 341.056                                             \\
                  \hline
                  $n = 187$ & 1.000           &                        & $X^2 = 488.740$                                     \\
                  \hline
              \end{tabular}
              $\Rightarrow$
              \begin{tabular}{|c|c|c|c|c|}
                  \hline
                  $\nu_i$   & $\widehat{p}_i$ & $n \widehat{p}_i$ & $\frac{(\nu_i - n \widehat{p}_i)}{n \widehat{p}_i}$ \\
                  \hline
                  28        & 0.099           & 18.588            & 4.766                                               \\
                  49        & 0.266           & 49.745            & 0.011                                               \\
                  57        & 0.311           & 58.243            & 0.027                                               \\
                  25        & 0.208           & 38.967            & 5.006                                               \\
                  16        & 0.087           & 16.294            & 0.005                                               \\
                  12        & 0.028           & 5.163             & 9.055                                               \\
                  \hline
                  $n = 187$ & 1.000           &                   & $X^2 = 18.870$                                      \\
                  \hline
              \end{tabular}

              Степени свободы 6-1-1 = 4, максимальный уровень значимости $\probability{\chi_4^2 \ge 18.870} \approx 0.00083$.

        \item Распределение Пуассона лучше согласуется с данными, чем биномиальное.
    \end{enumerate}
\fi

\subsection*{Задача 2}

По результатам КМ-1 студенты четвёртого курса получили следующие количества оценок:

\begin{center}
    \begin{tabular}{|P{2cm}|P{1cm}|P{1cm}|P{1cm}|P{1cm}|}
        \hline
        Группа  & 2  & 3 & 4  & 5  \\
        \hline
        A-05-21 & 13 & 2 & 6  & 3  \\
        \hline
        A-13-21 & 0  & 1 & 13 & 11 \\
        \hline
        A-14-21 & 1  & 2 & 7  & 3  \\
        \hline
        A-16-21 & 5  & 2 & 4  & 4  \\
        \hline
        A-18-21 & 10 & 3 & 6  & 4  \\
        \hline
    \end{tabular}
\end{center}

При каком наименьшем уровне значимости отклоняется гипотеза об однородности успеваемости в группах?

\ifsolutions Решение: \par
    $n = 100$, оценки вероятностей $\widehat{p} \approx (0.29, 0.1, 0.36, 0.25)$. В элементе таблицы: наблюдаемое количество $\nu_{ij}$,
    ожидаемое количество $n_i \widehat{p}_j$, отклонение $\frac{(\nu_{ij} - n_i \widehat{p}_j)^2}{n_i \widehat{p}_j}$

    \begin{tabular}{|c|c|c|c|c|c|c|c|c|c|c|c|}
        \hline
        13     6.960     5.242 & 2     2.400     0.067 & 6     8.640     0.807  & 3     6.000     1.500  \\
        0     7.250     7.250  & 1     2.500     0.900 & 13     9.000     1.778 & 11     6.250     3.610 \\
        1     3.770     2.035  & 2     1.300     0.377 & 7     4.680     1.150  & 3     3.250     0.019  \\
        5     4.350     0.097  & 2     1.500     0.167 & 4     5.400     0.363  & 4     3.750     0.017  \\
        10     6.670     1.663 & 3     2.300     0.213 & 6     8.280     0.628  & 4     5.750     0.533  \\
        \hline
    \end{tabular}

    статистика $X^2 = 28.41362$, уровень значимости 0.00481.
\fi

\subsection*{Задача 3 \cite[230]{Efimov}}

Давление в камере контролируется двумя манометрами. Считается, что показания манометров независимы и имеют нормальное распределение с неизвестными
математическими ожиданиями и дисперсиями. По результатам $n = 10$ замерам каждого из манометров вычислены: выборочные средние
$\widehat{m}_1^{(1)} = 15.3$, $\widehat{m}_1^{(2)} = 16.1$ и исправленные выборочные дисперсии $\widetilde{\mu}_{2}^{(1)} = 0.2$,
$\widetilde{\mu}_{2}^{(2)} = 0.15$. Для уровня значимости $\alpha = 0.1$ проверьте гипотезы:
\begin{enumerate}
    \item о равенстве дисперсий показаний с двусторонней критической областью,
    \item о равенстве математических ожиданий показаний с правосторонней критической областью.
\end{enumerate}
\ifsolutions Решение: \par
    \begin{enumerate}
        \item Статистика 1.(3), критическая область $(-\infty, 0.31457) \cup (3.17889, \infty)$, гипотеза принимается.
        \item Статистика -4.27618, критическая область $(1.32773, \infty)$, гипотеза отклоняется.
    \end{enumerate}
\fi

\subsection*{Задача 4}

Получена реализации выборки: 0.768, 1.31, 2.11, 1.29, 0.868, 1.59, 1.49, 3.24, 7.17, 2.1, 0.0434, 1.82, 0.278, 2.1, 1.06, 7.95, 1.69, 0.809, 1.63, 1.15.
Используя критерий согласия Колмогорова определите из какого распределения получена выборка из равномерного или показательного.

Получена ещё одна реализация выборки: 3.72, 0.49, 1.59, 1.21, 0.811, 2.63, 2.42, 0.265, 2.24, 0.596, 0.725, 3.5, 1.38, 0.0537, 3.86.
Проверьте, что распределения двух выборок совпадают, по критерию Колмогорова--Смирнова.

\ifsolutions Решение: \par
    $F(x)$ --- функция распределения, $F^*(x)$ и $G^*(x)$ --- эмпирические функции распределения.

    \begin{enumerate}
        \item Тест Колмогорова для $E(2.023)$: $\sup \modulus{F^*(x) - F(x)} \approx 0.21585$ при $x = 0.768$, статистика
              $\sqrt{20} \sup \modulus{F^*(x) - F(x)} \approx 0.96530$, уровень значимости по предельному распределению 0.309067,
              по точному распределению 0.268179

              \begin{tabular}{|c|c|c|c|c|c|}
                  \hline
                  $x$   & $F(x)$  & $F^*(x)$ & $F^*(x) - F(x)$           & $F^*(x+0)$ & $F^*(x+0) - F(x)$ \\
                  \hline
                  0.043 & 0.02122 & 0.00000  & -0.02122                  & 0.05000    & 0.02878           \\
                  0.278 & 0.12838 & 0.05000  & -0.07838                  & 0.10000    & -0.02838          \\
                  0.768 & 0.31585 & 0.10000  & \textcolor{red}{-0.21585} & 0.15000    & -0.16585          \\
                  0.809 & 0.32957 & 0.15000  & -0.17957                  & 0.20000    & -0.12957          \\
                  0.868 & 0.34884 & 0.20000  & -0.14884                  & 0.25000    & -0.09884          \\
                  1.060 & 0.40779 & 0.25000  & -0.15779                  & 0.30000    & -0.10779          \\
                  1.150 & 0.43355 & 0.30000  & -0.13355                  & 0.35000    & -0.08355          \\
                  1.290 & 0.47142 & 0.35000  & -0.12142                  & 0.40000    & -0.07142          \\
                  1.310 & 0.47662 & 0.40000  & -0.07662                  & 0.45000    & -0.02662          \\
                  1.490 & 0.52117 & 0.45000  & -0.07117                  & 0.50000    & -0.02117          \\
                  1.590 & 0.54426 & 0.50000  & -0.04426                  & 0.55000    & 0.00574           \\
                  1.630 & 0.55318 & 0.55000  & -0.00318                  & 0.60000    & 0.04682           \\
                  1.690 & 0.56624 & 0.60000  & 0.03376                   & 0.65000    & 0.08376           \\
                  1.820 & 0.59323 & 0.65000  & 0.05677                   & 0.70000    & 0.10677           \\
                  2.100 & 0.64580 & 0.70000  & 0.05420                   & 0.80000    & 0.15420           \\
                  2.110 & 0.64755 & 0.80000  & 0.15245                   & 0.85000    & 0.20245           \\
                  3.240 & 0.79837 & 0.85000  & 0.05163                   & 0.90000    & 0.10163           \\
                  7.170 & 0.97109 & 0.90000  & -0.07109                  & 0.95000    & -0.02109          \\
                  7.950 & 0.98034 & 0.95000  & -0.03034                  & 1.00000    & 0.01966           \\
                  \hline
              \end{tabular}

        \item Тест Колмогорова для $R[0.043, 7.950]$: $\sup \modulus{F^*(x) - F(x)} \approx 0.58862$ при $x = 2.11$, статистика
              $\sqrt{20} \sup \modulus{F^*(x) - F(x)} \approx 2.63240$, уровень значимости по предельному распределению $1.9147 \cdot 10^{-6}$,
              по точному распределению $3.97 \cdot 10^{-7}$.

              \begin{tabular}{|c|c|c|c|c|c|}
                  \hline
                  $x$   & $F(x)$  & $F^*(x)$ & $F(x) - F^*(x)$ & $F^*(x+0)$ & $F(x) - F^*(x+0)$        \\
                  \hline
                  0.043 & 0.00000 & 0.00000  & 0.00000         & 0.05000    & 0.05000                  \\
                  0.278 & 0.02967 & 0.05000  & 0.02033         & 0.10000    & 0.07033                  \\
                  0.768 & 0.09164 & 0.10000  & 0.00836         & 0.15000    & 0.05836                  \\
                  0.809 & 0.09683 & 0.15000  & 0.05317         & 0.20000    & 0.10317                  \\
                  0.868 & 0.10429 & 0.20000  & 0.09571         & 0.25000    & 0.14571                  \\
                  1.060 & 0.12858 & 0.25000  & 0.12142         & 0.30000    & 0.17142                  \\
                  1.150 & 0.13996 & 0.30000  & 0.16004         & 0.35000    & 0.21004                  \\
                  1.290 & 0.15767 & 0.35000  & 0.19233         & 0.40000    & 0.24233                  \\
                  1.310 & 0.16020 & 0.40000  & 0.23980         & 0.45000    & 0.28980                  \\
                  1.490 & 0.18296 & 0.45000  & 0.26704         & 0.50000    & 0.31704                  \\
                  1.590 & 0.19561 & 0.50000  & 0.30439         & 0.55000    & 0.35439                  \\
                  1.630 & 0.20067 & 0.55000  & 0.34933         & 0.60000    & 0.39933                  \\
                  1.690 & 0.20826 & 0.60000  & 0.39174         & 0.65000    & 0.44174                  \\
                  1.820 & 0.22470 & 0.65000  & 0.42530         & 0.70000    & 0.47530                  \\
                  2.100 & 0.26011 & 0.70000  & 0.43989         & 0.80000    & 0.53989                  \\
                  2.110 & 0.26138 & 0.80000  & 0.53862         & 0.85000    & \textcolor{red}{0.58862} \\
                  3.240 & 0.40430 & 0.85000  & 0.44570         & 0.90000    & 0.49570                  \\
                  7.170 & 0.90135 & 0.90000  & -0.00135        & 0.95000    & 0.04865                  \\
                  7.950 & 1.00000 & 0.95000  & -0.05000        & 1.00000    & 0.00000                  \\
                  \hline
              \end{tabular}

        \item Тест Колмогорова--Смирнова: $\sup \modulus{F^*(x) - G^*(x)} \approx 0.25$ при $x \in (2.11, 2.24]$, статистика
              $\sqrt{\frac{20 \cdot 15}{20 + 15}} \sup \modulus{F^*(x) - G^*(x)} \approx 0.73193$, уровень значимости по предельному распределению 0.65764,
              по точному распределению 0.587028.

              \begin{tabular}{|c|c|c|c|c|c|}
                  \hline
                  $x$   & $F^*(x)$ & $G^*(x)$ & $F^*(x) - G^*(x)$        \\
                  \hline
                  0.043 & 0.00000  & 0.00000  & 0.00000                  \\
                  0.054 & 0.05000  & 0.00000  & 0.05000                  \\
                  0.265 & 0.05000  & 0.06667  & -0.01667                 \\
                  0.278 & 0.05000  & 0.13333  & -0.08333                 \\
                  0.490 & 0.10000  & 0.13333  & -0.03333                 \\
                  0.596 & 0.10000  & 0.20000  & -0.10000                 \\
                  0.725 & 0.10000  & 0.26667  & -0.16667                 \\
                  0.768 & 0.10000  & 0.33333  & -0.23333                 \\
                  0.809 & 0.15000  & 0.33333  & -0.18333                 \\
                  0.811 & 0.20000  & 0.33333  & -0.13333                 \\
                  0.868 & 0.20000  & 0.40000  & -0.20000                 \\
                  1.060 & 0.25000  & 0.40000  & -0.15000                 \\
                  1.150 & 0.30000  & 0.40000  & -0.10000                 \\
                  1.210 & 0.35000  & 0.40000  & -0.05000                 \\
                  1.290 & 0.35000  & 0.46667  & -0.11667                 \\
                  1.310 & 0.40000  & 0.46667  & -0.06667                 \\
                  1.380 & 0.45000  & 0.46667  & -0.01667                 \\
                  1.490 & 0.45000  & 0.53333  & -0.08333                 \\
                  1.590 & 0.50000  & 0.53333  & -0.03333                 \\
                  1.630 & 0.55000  & 0.60000  & -0.05000                 \\
                  1.690 & 0.60000  & 0.60000  & 0.00000                  \\
                  1.820 & 0.65000  & 0.60000  & 0.05000                  \\
                  2.100 & 0.70000  & 0.60000  & 0.10000                  \\
                  2.110 & 0.80000  & 0.60000  & 0.20000                  \\
                  2.240 & 0.85000  & 0.60000  & \textcolor{red}{0.25000} \\
                  2.420 & 0.85000  & 0.66667  & 0.18333                  \\
                  2.630 & 0.85000  & 0.73333  & 0.11667                  \\
                  3.240 & 0.85000  & 0.80000  & 0.05000                  \\
                  3.500 & 0.90000  & 0.80000  & 0.10000                  \\
                  3.720 & 0.90000  & 0.86667  & 0.03333                  \\
                  3.860 & 0.90000  & 0.93333  & -0.03333                 \\
                  7.170 & 0.90000  & 1.00000  & -0.10000                 \\
                  7.950 & 0.95000  & 1.00000  & -0.05000                 \\
                  \hline
              \end{tabular}
    \end{enumerate}
\fi

\subsection*{Задача 5 \cite[206, 207, 210]{Efimov}}

$\left( \xi_1, \dots, \xi_n \right)$ --- выборка из нормального распределения $\mathcal{N} \left( m, \sigma^2 \right)$, где $m$ --- неизвестное,
а $\sigma = 2$. Для проверки гипотез в отношении $m$ используется критерий со статистикой:
\[
    \widehat{m}_1 = \frac{1}{n} \sum_{i=1}^n \xi_i .
\]
\begin{enumerate}
    \item Постройте критерий проверки гипотезы основной гипотезы $H_0: m \le m_0$ против альтернативной $H_1: m > m_0$. Выполните проверку гипотезы
          для $m_0 = 40$, $n = 64$ и уровне значимости $\alpha = 0.01$, если в результате эксперимента получено значение выборочного среднего $40.2$.
    \item Постройте критерий проверки основной гипотезы $H_0: m = m_0$ против альтернативной гипотезы $H_1: m = m_1$, где $m_1 < m_0$. Какова ошибка
          второго рода $\beta$ критерия при $m_0 = 40$, $m_1 = 39.2$, $n=36$ и уровне значимости $\alpha = 0.1$?
    \item Постройте критерий проверки гипотез $H_0: m = m_0 = 40$, $H_1: m = m_1 = 41$, $H_2: m = m_2 = 41.5$, у которого вероятность ошибки первого
          рода $\alpha \le 0.05$ и вероятность ошибки третьего рода $\gamma \le 0.03$ при $n = 100$. Какова величина ошибки второго рода $\beta$
          такого критерия?
    \item Пусть $n=36$ и критическая область $\Gamma = (40.2, \infty)$. Постройте график функции мощности на отрезке $m \in [39, 41]$.
\end{enumerate}

\ifsolutions Решение: \par
    Cтатистика $\widehat{m}_1 \sim \mathcal{N} (m, \frac{\sigma}{\sqrt{n}})$.

    \begin{enumerate}
        \item Правосторонняя критическая область $\Gamma_\alpha = (h_\alpha, + \infty)$, где:
              \begin{gather*}
                  \probability{\widehat{m}_1 \ge h_\alpha | m_0} \le \alpha , \\
                  %
                  \frac{h_\alpha - m_0}{\frac{\sigma}{\sqrt{n}}} \ge \Phi^{-1} (1 - \alpha) \approx 2.32635, \\
                  %
                  h_\alpha
                  \ge m_0 + \frac{\sigma}{\sqrt{n}} \Phi^{-1} (1 - \alpha)
                  \approx 40.77545.
              \end{gather*}
              $40.2 < h_\alpha$ или $\frac{40.2 - 40}{\frac{\sigma}{\sqrt{64}}} = 0.8$, принимается $H_0$.

        \item Левосторонняя критическая область $\Gamma_\alpha = (- \infty, h_\alpha)$, где        \begin{gather*}
                  \probability{\widehat{m}_1 \le h_\alpha | m_0} \le \alpha , \\
                  %
                  \frac{h_\alpha - m_0}{\frac{\sigma}{\sqrt{n}}} \le \Phi^{-1} (\alpha) , \\
                  %
                  h_\alpha
                  \le m_0 + \frac{\sigma}{\sqrt{n}} \Phi^{-1} (\alpha)
                  \approx 39.57282 .
              \end{gather*}
              Ошибка второго рода:
              \begin{gather*}
                  \probability{\widehat{m}_1 \ge h_\alpha | m_1}
                  = 1 - \Phi \left( \frac{h_\alpha - m_1}{\frac{\sigma}{\sqrt{n}}} \right)
                  \approx 0.13169
              \end{gather*}

        \item Области принятия гипотез $\Pi_0 = (- \infty, h_0 )$, $\Pi_1 = [h_0, h_1]$, $\Pi_2 = (h_1, \infty )$. Пороги из вероятностей ошибок:
              \begin{gather*}
                  \alpha = \probability{\widehat{m}_1 \ge h_0 | m_0} = 1 - \Phi \left( \frac{h_0 - m_0}{\frac{\sigma}{\sqrt{n}}} \right) = 0.05, \\
                  h_0 = m_0 + \frac{\sigma}{\sqrt{n}} \Phi^{-1} (1 - 0.05) \approx 40.32897
              \end{gather*}
              \begin{gather*}
                  \gamma = \probability{\widehat{m}_1 \le h_1 | m_2} = \Phi \left( \frac{h_1 - m_2}{\frac{\sigma}{\sqrt{n}}} \right) = 0.03, \\
                  h_1 = m_2 + \frac{\sigma}{\sqrt{n}} \Phi^{-1} (0.03) \approx 41.12384 ,
              \end{gather*}
              тогда
              \begin{multline*}
                  \beta
                  = \probability{\widehat{m}_1 < h_0 | m_1} + \probability{\widehat{m}_1 > h_1 | m_1} = \\
                  %
                  = \Phi \left( \frac{h_0 - m_1}{\frac{\sigma}{\sqrt{n}}} \right) + 1 - \Phi \left( \frac{h_1 - m_1}{\frac{\sigma}{\sqrt{n}}} \right)
                  \approx 0.26829
              \end{multline*}

        \item Мощность:
              \[
                  W(m)
                  = \probability{\widehat{m}_1 > 40.2 | m}
                  = 1 - \Phi \left( \frac{40.2 - m}{\frac{\sigma}{\sqrt{n}}} \right)
              \]
              \begin{tabular}{|c|c|c|c|c|c|c|c|}
                  \hline
                  39.00 & 0.000159 & 39.50 & 0.017864 & 40.00 & 0.274253 & 40.50 & 0.815940 \\
                  39.05 & 0.000280 & 39.55 & 0.025588 & 40.05 & 0.326355 & 40.55 & 0.853141 \\
                  39.10 & 0.000483 & 39.60 & 0.035930 & 40.10 & 0.382089 & 40.60 & 0.884930 \\
                  39.15 & 0.000816 & 39.65 & 0.049471 & 40.15 & 0.440382 & 40.65 & 0.911492 \\
                  39.20 & 0.001350 & 39.70 & 0.066807 & 40.20 & 0.500000 & 40.70 & 0.933193 \\
                  39.25 & 0.002186 & 39.75 & 0.088508 & 40.25 & 0.559618 & 40.75 & 0.950529 \\
                  39.30 & 0.003467 & 39.80 & 0.115070 & 40.30 & 0.617911 & 40.80 & 0.964070 \\
                  39.35 & 0.005386 & 39.85 & 0.146859 & 40.35 & 0.673645 & 40.85 & 0.974412 \\
                  39.40 & 0.008198 & 39.90 & 0.184060 & 40.40 & 0.725747 & 40.90 & 0.982136 \\
                  39.45 & 0.012224 & 39.95 & 0.226627 & 40.45 & 0.773373 & 40.95 & 0.987776 \\
                  \hline
                        &          &       &          &       &          & 41.00 & 0.991802 \\
                  \hline
              \end{tabular}
    \end{enumerate}
\fi

\section*{Домашнее задание 5}

\subsection*{Задача 1 \cite[278]{Efimov}}

Время химической реакции при различном содержании катализатора распределилось следующим образом:

\begin{tabular}{|c|c|c|c|c|c|c|c|c|c|c|c|c|}
    \hline
         & 1   & 2   & 3   & 4   & 5   & 6   & 7   & 8   & 9   & 10  & 11  & 12  \\
    \hline
    5\%  & 5.9 & 6.0 & 7.0 & 6.5 & 5.5 & 7.0 & 8.1 & 7.5 & 6.2 & 6.4 & 7.1 & 6.9 \\
    \hline
    10\% & 4.0 & 5.1 & 6.2 & 5.3 & 4.5 & 4.4 & 5.3 & 5.4 & 5.6 & 5.2 & -   & -   \\
    \hline
    15\% & 8.2 & 6.8 & 8.0 & 7.5 & 7.0 & 7.2 & 7.9 & 8.1 & 8.5 & 7.8 & 8.1 &     \\
    \hline
\end{tabular}

Проверить гипотезу о равенстве времен реакции при различном содержании катализатора для уровня значимости $\alpha = 0.1$.

\ifsolutions
    Решение:

    \begin{tabular}{|P{1cm}||P{3.6cm}|P{2cm}|P{0.5cm}|P{2.5cm}|P{3.6cm}|}
        \hline
                       & $\sum_{i=1}^{n_j} \left( \xi_i^{(j)} \right)^2$ & $\sum_{i=1}^{n_j} \xi_i^{(j)}$  & $n_j$ & $\left( \sum_{i=1}^{n_j} \xi_i^{(j)} \right)^2$ & $\frac{1}{n_j} \left( \sum_{i=1}^{n_j} \xi_i^{(j)} \right)^2$ \\
        \hline
                       & 540.59                                          & 80.1                            & 12    & 6416.01                                         & 534.668                                                       \\
        \hline
                       & 263.8                                           & 51                              & 10    & 2601                                            & 260.1                                                         \\
        \hline
                       & 661.29                                          & 85.1                            & 11    & 7242.01                                         & 658.365                                                       \\
        \hline
        \hline
        $\sum_{j=1}^k$ & 1465.68                                         & 216.2                           & 33    &                                                 & 1453.13                                                       \\
        \hline
                       & $1465.68 - \frac{216.2^2}{33} = 49.242$         & $\frac{216.2^2}{33} = 1416.438$ &       &                                                 & $1453.13 - \frac{216.2^2}{33} = 36.694$                       \\
        \hline
    \end{tabular}

    $n \widehat{s}^2(\xi) = 49.242 - 39.694 = 12.548$,
    $T = \frac{36.694}{3-1} \frac{33 - 3}{12.648} \approx 43.865$,
    $h_\alpha = Q_{F(3-1,33-3)}(1 - \alpha) \approx 2.489$,
    $T > h_\alpha$ : гипотеза отклоняется.

    Ответ: гипотеза о равенстве отклоняется.
\fi

\subsection*{Задача 2}

В выборке $\left( \xi_1, \dots, \xi_n \right)$ случайные величины бинарные:
\[
    \xi_i
    = \left \{
    \begin{array}{ll}
        1, & \text{с вероятностью } p   \\
        0, & \text{с вероятностью } 1-p
    \end{array}
    \right .
\]
В отношении неизвестной вероятности $p$ выдвигаются основная гипотеза $H_0: p = p_0$ и альтернативная $H_1: p = p_1$. Постройте минимаксный критерий
проверки гипотезы $H_0$ против $H_1$ для $p_0 = 0.4$, $p_1 = 0.7$ и объёма выборки $n = 100$. Вычислите вероятности ошибок первого и второго рода
полученного критерия.

\subsection*{Задача 3}

Задана выборка $\left( \xi_1, \dots, \xi_n \right)$ из распределения Пуассона $\mathcal{P}(\lambda)$ с неизвестным параметром $\lambda$. Выдвигаются
две гипотезы: основная $H_0: \lambda = \lambda_0$ и альтернативная $H_1: \lambda = \lambda_1$, где $\lambda_0 = 2$ и $\lambda_1 = 2.2$. Постройте
критерии Неймана--Пирсона в следующих случаях:
\begin{enumerate}
    \item задан объём выборки $n = 50$ и вероятность ошибки первого рода $\alpha = 0.05$; определите вероятность ошибки второго рода.
    \item заданы вероятности ошибок: первого рода $\alpha = 0.01$ и второго рода $\beta = 0.05$.
\end{enumerate}

\subsection*{Задача 4}

Случайные величины $\xi_i$ являются промежутками времени между автомобилями, проезжающими по автостраде. Считается, что величины $\xi_i$ являются
независимыми в совокупности и имееют одинаковую плотность вероятности $p_\xi(x)$, которая неизвестна. Рассматриваются две гипотезы:
\begin{gather*}
    H_0:
    p_\xi(x) = \left \{
    \begin{array}{ll}
        0,                                       & x < 0   \\
        \sqrt{\frac{2}{\pi}} e^{-\frac{x^2}{2}}, & x \ge 0
    \end{array}
    \right . , \\
    %
    H_1:
    p_\xi(x) = \left \{
    \begin{array}{ll}
        0,      & x < 0   \\
        e^{-x}, & x \ge 0
    \end{array}
    \right . .
\end{gather*}
Постройте последовательный критерий отношения вероятностей проверки гипотезы $H_0$ против $H_1$ для вероятностей ошибок $\alpha = 0.02$ и $\beta = 0.03$.
Определите средние количества наблюдений до остановки критерия.

\section*{Домашнее задание 6}

\subsection*{Задача 1}

Грани правильного тетраэдра обозначены цифрами 1, 2, 3, 4. Тетраэдр подбрасывают два раза и записывают $\nu_1$ --- цифру, выпавшую при первом броске,
и $\nu_2$ --- цифру, выпавшую при втором броске. Вычисляют величины $\eta = \min \{ \nu_1, \nu_2 \}$ и $\xi = \max \{ \nu_1, \nu_2 \}$, причём
величина $\eta$ не является наблюдаемой, зато наблюдаемой является величина $\xi$.
\begin{enumerate}
    \item Для оценивания величины $\eta$ по величине $\xi$ вычислите условное математическое ожидание $\widehat{\eta} = \cexpectation{\eta}{\xi}$ величины $\eta$ относительно $\xi$.
    \item Для определения точности оценки $\widehat{\eta}$ вычислите условную дисперсию $\cvariance{\eta}{\xi}$.
    \item Для определения общего "качества"{} оценки вычислите коэффициент детерминации:
          \[
              R^2 = 1 - \frac{\expectation{\cvariance{\eta}{\xi}}}{\variance{\eta}}.
          \]
\end{enumerate}

\subsection*{Задача 2 \cite[391]{Efimov}}

По набору данных

\begin{center}
    \begin{tabular}{|c|c|c|c|c|c|c|c|c|c|c|}
        \hline
        $x$    & 1   & 2   & 3   & 4   & 5   & 6   & 7   & 8   & 9   & 10  \\
        \hline
        $\eta$ & 0.5 & 0.9 & 0.4 & 2.2 & 0.6 & 1.1 & 0.8 & 2.4 & 1.2 & 4.3 \\
        \hline
    \end{tabular}
\end{center}

\noindent для модели регрессии вида:
\begin{gather*}
    \eta = 1 \cdot \widetilde{\theta}_1 + x \cdot \widetilde{\theta}_2 + x^2 \cdot \widetilde{\theta}_3 + \varepsilon , \\
    \varepsilon \sim \mathcal{N} (0, K), \\
    K = \sigma^2 \cdot diag \{ 1, 2, 3, 4, 5, 6, 7, 8, 9, 10 \},
\end{gather*}
где $diag$ обозначает диагональную матрицу:
\begin{enumerate}
    \item вычислите МНК--оценку $\widehat{\theta}$ неизвестного параметра
          $\widetilde{\theta} = (\widetilde{\theta}_1, \widetilde{\theta}_2, \widetilde{\theta}_3)$,
    \item вычислите коэффициент детерминации $R^2$ и скорректированный коэффициент детерминации $R_{adj}^2$,
    \item постройте доверительные интервалы для всех компонент $\widehat{\theta}_k$ с уровнем доверия $P_g = 0.95$,
    \item проверьте гипотезы о равенстве нулю компонент $\widetilde{\theta}_k$ по отдельности для уровня значимости $\alpha = 0.1$,
    \item проверьте гипотезу о равенстве нулю всех компонент $\widetilde{\theta}_k$ одновременно для уровня значимости $\alpha = 0.05$,
    \item вычислите оценку остаточной дисперсии $\sigma^2$,
    \item постройте доверительный интервал для остаточной дисперсии $\sigma^2$ с уровнем доверия $P_g = 0.9$.
\end{enumerate}

\ifsolutions
    Решение:

    \begin{enumerate}
        \item Нормальная система:
              \[
                  \begin{pmatrix}
                      2.92897 & 10  & 55   \\
                      10      & 55  & 385  \\
                      55      & 385 & 3025
                  \end{pmatrix}
                  \begin{pmatrix}
                      0.65848  \\
                      -0.06106 \\
                      0.02902
                  \end{pmatrix}
                  =
                  \begin{pmatrix}
                      2.91429  \\
                      14.40000 \\
                      14.40000
                  \end{pmatrix}
              \]
              Ковариационная матрица:
              \[
                  \sigma^2
                  \begin{pmatrix}
                      2.33117  & -1.16559 & 0.10596  \\
                      -1.16559 & 0.74946  & -0.07419 \\
                      0.10596  & -0.07419 & 0.00785
                  \end{pmatrix}
              \]

        \item
              \begin{gather*}
                  \pr{\eta}{L} = \begin{pmatrix} 0.62644 \\ 0.65245 \\ 0.73650 \\ 0.87860 \\ 1.07873 \\ 1.33692 \\ 1.65314 \\ 2.02741 \\ 2.45972 \\ 2.95008 \end{pmatrix},
                  \pro{\eta}{L} = \begin{pmatrix} -0.12644 \\ 0.24755 \\ -0.33650 \\ 1.32140 \\ -0.47873 \\ -0.23692 \\ -0.85314 \\ 0.37259 \\ -1.25972 \\ 1.34992 \end{pmatrix} , \\
                  \norm{\pr{\eta}{L}}_W = \frac{1.989}{\sigma} ,
                  \norm{\pr{\eta}{L}}_W^2 = \frac{3.95645}{\sigma^2} ,
                  \norm{\pro{\eta}{L}}_W = \frac{1.02761}{\sigma} ,
                  \norm{\pro{\eta}{L}}_W^2 = \frac{1.05597}{\sigma^2} \\
                  %
                  \widehat{c} = 0.99499,
                  U \widehat{c} = \begin{pmatrix} -0.49499 \\ -0.09499 \\ -0.59499 \\ 1.20501 \\ -0.39499 \\ 0.10501 \\ -0.19499 \\ 1.40501 \\ 0.20501 \\ 3.30501 \end{pmatrix},
                  \norm{U \widehat{c}}_W = \frac{1.45353}{\sigma},
                  \norm{U \widehat{c}}_W^2 = \frac{2.11275}{\sigma^2}, \\
                  %
                  R^2 = 0.5, R_{adj}^2 = 0.35 .
              \end{gather*}
        \item
              \begin{gather*}
                  T^{-1}(0.025) = -2.365, \\
                  \widetilde{\theta}_1 \in (0.65848 - 1.40226, 0.65848 + 1.40226) = (-0.74377, 2.06074) , \\
                  \widetilde{\theta}_2 \in (-0.06106 - 0.79509, -0.06106 + 0.79509) = (-0.85614, 0.73403) , \\
                  \widetilde{\theta}_3 \in (0.02902 - 0.08136, 0.02902 + 0.08136) = (-0.05233, 0.11038) .
              \end{gather*}

        \item
              Содержится ли $t_k$ в $\Gamma_\alpha$:
              \begin{gather*}
                  t_k = \frac{\widehat{\theta}_k}{\sqrt{\element{G^{-1}}_{kk} \frac{\norm{\pro{\eta}{L}}_W^2}{n-m}}} , \\
                  t_1 = 1.11040, t_2 = -0.18159, t_3 = 0.84353 , \\
                  \Gamma_\alpha = (-\infty, -1.895) \cup (1.895, \infty) .
              \end{gather*}
              Все гипотезы о равенстве нулю принимаются.

              Содержится ли ноль в интервале:
              \begin{gather*}
                  T^{-1}(0.05) = -1.895, \\
                  (0.65848 - 1.12351, 0.65848 + 1.12351) = (-0.46503, 1.78199) , \\
                  (-0.06106 - 0.63704, -0.06106 + 0.63704) = (-0.69810, 0.57598) , \\
                  (0.02902 - 0.06518, 0.02902 + 0.06518) = (-0.03616, 0.09421) .
              \end{gather*}
              Все гипотезы о равенстве нулю принимаются.

        \item $F = 8.74238, \alpha^* = 0.00913$.

        \item $\widehat{\sigma}^2 = 0.38840 , \widehat{\sigma}^2 = 0.15085$.

        \item $\sigma^2 \in \left( \frac{1.05597}{14.067140}, \frac{1.05597}{2.16735} \right ) = (0.07507, 0.48722)$.
    \end{enumerate}
\fi

\subsection*{Задача 3}

Для набора данных

\begin{center}
    \begin{tabular}{|c|c|c|c|c|c|c|c|c|c|c|}
        \hline
        $x$ & 0.00  & 0.25                  & 0.50 & 0.75 & 1.00 & 1.25 & 1.50  & 1.75 & 2.00 & 2.25 \\
        $y$ & -0.00 & 0.01                  & 0.08 & 0.19 & 0.30 & 0.42 & 0.50  & 0.55 & 0.59 & 0.60 \\
        \hline
        \hline
        $x$ & 2.50  & 2.75                  & 3.00 & 3.25 & 3.50 & 3.75 & 4.00  & 4.25 & 4.50 & 4.75 \\
        $y$ & 0.62  & 0.55                  & 0.54 & 0.48 & 0.45 & 0.40 & 0.35  & 0.29 & 0.26 & 0.21 \\
        \hline
        \hline
        $x$ & 5.00  & 5.25                  & 5.50 & 5.75 & 6.00 & 6.25 & 6.50  & 6.75 & 7.00 & 7.25 \\
        $y$ & 0.19  & 0.16                  & 0.13 & 0.12 & 0.09 & 0.08 & 0.08  & 0.06 & 0.04 & 0.01 \\
        \hline
        \hline
        $x$ & 7.50  & 7.75                  & 8.00 & 8.25 & 8.50 & 8.75 & 9.00  & 9.25 & 9.50 & 9.75 \\
        $y$ & 0.02  & 0.02                  & 0.02 & 0.01 & 0.02 & 0.01 & -0.00 & 0.01 & 0.00 & 0.02 \\
        \hline
        \hline
        $x$ & 10.00 & \multicolumn{9}{l|}{}                                                          \\
        $y$ & 0.01  & \multicolumn{9}{l|}{}                                                          \\
        \hline
    \end{tabular}
\end{center}

\begin{enumerate}
    \item подберите "хорошую"{} модель регрессии для зависимости $y$ от $x$,
    \item для выбранной модели вычислите оценку параметра и оценку остаточной дисперсии,
\end{enumerate}
считая, что ошибки измерений $y$ одинаковы.

\ifsolutions
    Решение:

    Зависимость $y = x^{2.7} e^{-1.2x} + \varepsilon$.
\fi

\section*{Домашнее задание 7}

\subsection*{Задача 1}

Случайная величина $\eta = \xi^ \varphi$, где величина $\xi \sim \mathcal{R} \left[ 2, 5 \right]$ и величина $\varphi \sim Bi(10, 0.4)$. Постройте
метод статистических испытаний для оценки значения функции распределения $F_\eta(y)$ величины $\eta$ в заданной точке $y_0$ с отклонением менее
$\delta = 10^{-3}$ с вероятностью более $P_\delta = 0.98$.

\ifsolutions
    Решение:

    $(\xi_1, \dots, \xi_n)$ --- выборка из $Bi(10, 0.4)$, $(\varphi_1, \dots, \varphi_n)$ --- выборка из $\mathcal{R} \left[ 2, 5 \right]$,
    оценка:
    \begin{gather*}
        \eta_k = \xi_k \ln \varphi_k , \\
        \varepsilon_k = \left \{
        \begin{array}{ll}
            1, \eta_k < y_0 \\
            0, \eta_k \ge y_0
        \end{array}
        \right . , \\
        F_\eta(y_0) \approx \frac{1}{n} \sum_{k=1}^n \varepsilon_k .
    \end{gather*}
    Требуемый объём выборок:
    \begin{gather*}
        n \ge \widehat{n}_\delta
        = \left( K^{-1} ( P_\delta ) \right)^2 \frac{1}{\delta^2}
        \approx 1.5174^2 \frac{1}{10^{-6}}
        \approx 2.302503 \cdot 10^6
        = 2 302 503 .
    \end{gather*}

\fi

\subsection*{Задача 2}

Случайная величина $\eta = e^{- 2 \modulus{\xi}} \arctg \varphi$, где величина $\xi \sim E(5)$ и $\varphi \sim \mathcal{N}(0,1)$. Постройте
метод статистических испытаний для оценки $\expectation{\eta}$ с отклонением менее $\delta = 10^{-2}$ с вероятностью более $P_\delta = 0.99$.

\ifsolutions
    Решение:

    $(\xi_1, \dots, \xi_n)$ --- выборка из $E(5)$, $(\varphi_1, \dots, \varphi_n)$ --- выборка из $\mathcal{N} (0, 1)$, оценка:
    \begin{gather*}
        \eta_k = e^{- 2 \modulus{\xi_k}} \arctg \varphi_k , \\
        \expectation{\eta} \approx \frac{1}{n} \sum_{k=1}^n \eta_k .
    \end{gather*}
    Требуемый объём выборок:
    \[
        n
        \ge \overline{n}_\delta
        = \left( \Phi^{-1} \left( \frac{1 + P_\delta}{2} \right) \right)^2 \frac{\overline{D}}{\delta^2} ,
    \]
    где
    \begin{gather*}
        - \frac{\pi}{2} \le e^{-2 \modulus{\xi_k}} \arctg \varphi \le \frac{\pi}{2}, \\
        %
        \variance{\eta_k}
        \le \left( \frac{\frac{\pi}{2} - \left(-\frac{\pi}{2}\right)}{2} \right)^2
        = \left( \frac{\pi}{2} \right)^2
        \approx 2.4675
        = \overline{D} ,
    \end{gather*}
    тогда
    \[
        \overline{n}
        = \left( \Phi^{-1} \left( 0.995 \right) \right)^2 \frac{2.4675}{10^{-4}}
        \approx 2.5758^2 \cdot 2.4675 \cdot 10^4
        \approx 16.3712 \cdot 10^4
        = 163 712 .
    \]
\fi

\subsection*{Задача 3}

Постройте метод статистических испытаний для вычисления интеграла
\[
    J = \int \limits_{-2}^{1} e^{\modulus{\cos x}} dx
\]
с точностью менее $\delta = 0.05$ с вероятностью $P_\delta = 0.98$.

\ifsolutions
    Решение:

    \[
        x \in [-2, 1]: 1 \le e^{\modulus{\cos x}} \le e .
    \]
    \begin{multline*}
        J
        = (e-1) \int \limits_{-2}^{1} \frac{e^{\modulus{\cos x}} - 1}{e-1} dx + 1 \cdot (1 - (-2))
        = (e-1) \int \limits_{0}^{1} \frac{e^{\modulus{\cos (-2 + 3 y)}} - 1}{e-1} 3 dy + 3 = \\
        %
        = 3 (e-1) \widetilde{J} + 3 ,
    \end{multline*}
    где
    \[
        \widetilde{J}
        = \int \limits_{0}^{1} \frac{e^{\modulus{\cos (-2 + 3 y)}} - 1}{e-1} dy
        = \int \limits_{0}^{1} \widetilde{f}(y) dy .
    \]
    $(\xi_1, \dots, \xi_n)$ и $(\eta_1, \dots, \eta_n)$ --- независимые выборки из $\mathcal{R} [0, 1]$:
    \begin{gather*}
        \varepsilon_k = \left \{
        \begin{array}{ll}
            1, \eta_k < \widetilde{f}(\xi_k) \\
            0, \eta_k \ge \widetilde{f}(\xi_k)
        \end{array}
        \right . , \\
        %
        \widetilde{J} \approx \frac{1}{n} \sum_{k=1}^n \varepsilon_k .
    \end{gather*}

    Требуемый объём выборок:
    \begin{multline*}
        n
        \ge \overline{n}_\delta
        = \left( \Phi^{-1} \left( \frac{1 + P_\delta}{2} \right) \right)^2 \frac{\overline{D}}{\left( \frac{\delta}{(e-1)(1-(-2))} \right)^2}
        = \left( \Phi^{-1} ( 0.99 ) \right)^2 \frac{\frac{1}{4}}{\left( \frac{0.05}{(e-1) 3} \right)^2} \approx \\
        \approx 2.3263^2 \left( 1.7182 \cdot 3 \right)^2 \frac{0.5^2}{0.05^2}
        \approx 143.79\cdot 10^2
        = 14 379 .
    \end{multline*}
\fi

\section*{Домашнее задание 8}

\subsection*{Задача 1}

Случайные величины $\xi_1, \dots, \xi_n$ независимы в совокупности. Для указанных распределений величин:
\begin{enumerate}
    \item $\xi_i \sim \mathcal{N}(m, \sigma^2)$, где $m$ --- параметр, $\sigma$ --- известное значение,
    \item $\xi_i \sim \mathcal{N}(m, \sigma^2)$, где $m$ --- известное значение, $\sigma$ --- параметр,
    \item $\xi_i \sim E(\theta)$, где $\theta$ --- параметр,
    \item $\xi_i \sim Po(\lambda)$, где $\lambda$ --- параметр,
    \item $\xi_i \sim Bi(n_i, p)$, где $n_i$ --- известные (различные) числа, $p$ --- параметр,
\end{enumerate}
необходимо:
\begin{enumerate}
    \item найти достаточные статистики,
    \item найти $R$--эффективные оценки, вычислить их смещения и дисперсии,
    \item вычислить информацию Фишера,
    \item вычислить нижнюю границу дисперсии Крамера--Рао,
    \item сравнить дисперсии найденных $R$-эффективных оценок и нижних границ Крамера--Рао.
\end{enumerate}

\begin{thebibliography}{1}
    \bibitem{Efimov} Сборник задач по математике, часть 4 // Под редакцией Ефимова А.В. и Поспелова А.С. - М.: Издательство Физико-математической
    литературы, 2003. - 432 с.
\end{thebibliography}
\end{document}