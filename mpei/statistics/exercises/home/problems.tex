\documentclass[a4paper,12pt]{article}
\usepackage[T1]{fontenc}
\usepackage[utf8]{inputenc}
\usepackage[english,russian]{babel}
\usepackage[margin=2cm]{geometry}
\usepackage{amsmath}
\usepackage{array}
\usepackage{xcolor}

\newcommand{\probability}[1]{P \left\{ #1 \right\}}
\newcommand{\variance}[1]{\mathtt{D} \left[ #1 \right]}

\newcolumntype{P}[1]{>{\centering\arraybackslash}p{#1}}

\newif\ifsolutions
\solutionstrue

\begin{document}

\title{Математическая статистика: домашние задания}
\author{Тигетов Давид Георгиевич}
\maketitle

\section*{Домашнее задание 1}

\subsection*{Задача 1 \cite[108]{Efimov}}

Докажите, что выборочное среднее, вычисленное по выборке из распределения Пуассона $\mathcal{P}(\lambda)$, будет несмещённой и состоятельной оценкой
параметра $\lambda$.

\subsection*{Задача 2 \cite[103]{Efimov}}

Пусть $m_1$, $m_2$ и $s_1^2$, $s_2^2$ --- несмещенные и состоятельные оценки параметров $\mu$ и $\sigma^2$, построенные для двух выборок
объёмами $n_1$ и $n_2$. Покажите, что статистики:
\begin{gather*}
    m = \frac{n_1 m_1 + n_2 m_2}{n_1 + n_2} , \\
    s^2 = \frac{(n_1 - 1) s_1^2 + (n_2 - 1) s_2^2}{n_1 + n_2 - 2} .
\end{gather*}
будут несмещёнными и состоятельными оценками $\mu$ и $\sigma^2$.

\subsection*{Задача 3 \cite[107]{Efimov}}

Пусть $\widetilde{\theta}$ --- несмещенная оценка параметра $\theta$, $\variance{\widetilde{\theta}} < \infty$. Докажите, что
$\widetilde{\theta}^2$ является смещенной оценкой $\theta^2$, и вычислите смещение.

\subsection*{Задача 4 \cite[110]{Efimov}}

Пусть $\widetilde{\theta}$ --- состоятельная оценка параметра $\theta$, a $\psi(x)$ --- непрерывная функция в области изменения $\widetilde{\theta}$.
Докажите, что $\psi(\widetilde{\theta})$ является состоятельной оценкой $\psi(\theta)$.

\section*{Домашнее задание 2}

\subsection*{Задача 1 \cite[124]{Efimov}}

Случайные величины $\xi_1$, \dots, $\xi_m$ независимы и имеют биномиальные распределения $Bi(n_1, p)$, \dots, $Bi(n_m, p)$, все $n_k$ известны.
Найдите МП-оценку параметра $p$. Покажите, что МП-оценка является несмещённой и вычислите её дисперсию.

\subsection*{Задача 2}

$\left( \xi_1, \dots, \xi_n \right)$ --- выборка из нормального распределения $\mathcal{N} \left( m, \sigma^2 \right)$. Найдите МП-оценку двумерного
параметра $\left( m, \sigma^2 \right)$.

\subsection*{Задача 3 \cite[126]{Efimov}}

$\left( \xi_1, \dots, \xi_n \right)$ --- выборка из распределения с плотностью:
\[
    p_\xi(x|\theta)
    =
    \left\{
    \begin{array}{ll}
        \theta x & x \in \left[ 0, \sqrt{\frac{2}{\theta}}\right]    \\
        0        & x \notin \left[ 0, \sqrt{\frac{2}{\theta}}\right]
    \end{array}
    \right.
\]
Найти МП-оценку математического ожидания этого распределения.

\subsection*{Задача 4}

$\left( \xi_1, \dots, \xi_n \right)$ --- выборка из распределения Рэлея $\mathcal{R}(\theta)$. Постройте оценки параметра $\theta$ методами моментов
и порядковых статистик. Сравните асимптотические дисперсии оценок.

\section*{Домашнее задание 3}

\subsection*{Задача 1 \cite[164]{Efimov}}

По двум выборкам из нормального распределения $\mathcal{N} \left( \mu, \sigma^2 \right)$ объёмами $n_1$ и $n_2$ вычислены выборочные средние
$m_1$, $m_2$, исправленные выборочные дисперсии $s_1^2$, $s_2^2$ и статистики:
\begin{gather*}
    m = \frac{n_1 m_1 + n_2 m_2}{n_1 + n_2} , \\
    s^2 = \frac{(n_1 - 1) s_1^2 + (n_2 - 1) s_2^2}{n_1 + n_2 - 2} .
\end{gather*}

Покажите, что при известной дисперсии $\sigma^2$ доверительный интервал для $\mu$ уровня доверия $\alpha$ имеет вид:
\[
    m - u(\alpha) \frac{\sigma}{\sqrt{n_1 + n_2}}
    < \mu
    < m + u(\alpha) \frac{\sigma}{\sqrt{n_1 + n_2}}
\]
и укажите $u(\alpha)$, а при неизвестной дисперсии доверительный интервал для $\mu$ уровня доверия $\alpha$ имеет вид:
\[
    m - t(\alpha) \frac{s}{\sqrt{n_1 + n_2}}
    < \mu
    < m + t(\alpha) \frac{s}{\sqrt{n_1 + n_2}}
\]
и укажите $t(\alpha)$.

\subsection*{Задача 2 \cite[176, 177]{Efimov}}

По двум выборкам из нормальных распределений $\mathcal{N} \left( \mu_1, \sigma_1^2 \right)$ и $\mathcal{N} \left( \mu_2, \sigma_2^2 \right)$ объёмами
$n_1$ и $n_2$ вычислены выборочные средние $m_1$, $m_2$, исправленные выборочные дисперсии $s_1^2$, $s_2^2$ и статистика:
\[
    s^2 = \frac{(n_1 - 1) s_1^2 + (n_2 - 1) s_2^2}{n_1 + n_2 - 2} .
\]

Покажите, что если дисперсии $\sigma_1^2$ и $\sigma_2^2$ известны, то доверительный интервал
для разности $\mu_1 - \mu_2$ с уровнем доверия $\alpha$ имеет вид:
\[
    (m_1 - m_2) - u(\alpha) \sqrt{\frac{\sigma_1^2}{n_1} + \frac{\sigma_2^2}{n_2}}
    < \mu_1 - \mu_2
    < (m_1 - m_2) + u(\alpha) \sqrt{\frac{\sigma_1^2}{n_1} + \frac{\sigma_2^2}{n_2}} ,
\]
и укажите $u(\alpha)$. Если дисперсии неизвестны, но равны $\sigma_1^2 = \sigma_2^2$, то доверительный интервал для разности $\mu_1 - \mu_2$
с уровнем доверия $\alpha$ имеет вид:
\[
    (m_1 - m_2) - t(\alpha) s \sqrt{\frac{1}{n_1} + \frac{1}{n_2}}
    < \mu_1 - \mu_2
    < (m_1 - m_2) + t(\alpha) s \sqrt{\frac{1}{n_1} + \frac{1}{n_2}} ,
\]
и укажите $t(\alpha)$.

\subsection*{Задача 3 \cite[187]{Efimov}}

Из урны, содержащей черные и белые шары в неизвестной пропорции, случайным образом извлекается 100 шаров с возвращением. Найти доверительный интервал
доли чёрных шаров с уровнем доверия $\alpha=0.95$, если среди вынутых оказалось 30 чёрных шаров. Сколько извлечений нужно сделать, чтобы относительная
частота отличалась от доли чёрных шаров не более чем на 0.01 с вероятностью 0.99?

\subsection*{Задача 4 \cite[191]{Efimov}}

На каждой из 36 АТС города в период с двух до трех часов было зафиксировано в среднем 2 вызова. Считая, что число вызовов для каждой АТС имеет
распределение Пуассона с одним и тем же параметром $\lambda$, приближённо найти нижнюю и верхнюю \textbf{доверительные границы} с уровнем
доверия $\alpha = 0.9$.

\section*{Домашнее задание 4}

\subsection*{Задача 1 \cite[291]{Efimov}}

При испытании радиоэлектронной аппаратуры фиксировалось число отказов. Результаты испытаний:

\begin{center}
    \begin{tabular}{|P{3cm}|P{1cm}|P{1cm}|P{1cm}|P{1cm}|P{1cm}|P{1cm}|P{1cm}|P{1cm}|P{1cm}|}
        \hline
        Число \par отказов:  & 0  & 1  & 2  & 3  & 4  & 5 & 6 & 7 & 8 \\
        \hline
        Число \par испытаний & 28 & 49 & 57 & 25 & 16 & 4 & 4 & 3 & 1 \\
        \hline
    \end{tabular}
\end{center}

С помощью критерия $\chi^2$ определите уровни значимости, при которых принимаются гипотезы о том, что число отказов имеет:
\begin{enumerate}
    \item распределение Пуассона,
    \item биномиальное распределение.
\end{enumerate}
Какое из двух распределений числа отказов лучше согласуется с данными?

\ifsolutions Решение: \par
    \begin{enumerate}
        \item оценка $\widehat{\lambda} = 2.00535$, вероятности $\widehat{p}_i = \frac{\widehat{\lambda}^i}{i!} e^{- \widehat{\lambda}}$

              \begin{tabular}{|c|c|c|c|}
                  \hline
                  $\nu_i$   & $\widehat{p}_i$ & $n \widehat{p}_i$      & $\frac{(\nu_i - n \widehat{p}_i)^2}{n \widehat{p}_i}$ \\
                  \hline
                  28        & 0.135           & 25.173                 & 0.318                                                 \\
                  49        & 0.270           & 50.480                 & 0.043                                                 \\
                  57        & 0.271           & 50.615                 & 0.805                                                 \\
                  25        & 0.181           & 33.834                 & 2.306                                                 \\
                  16        & 0.091           & 16.962                 & 0.055                                                 \\
                  4         & 0.036           & \textcolor{red}{6.803} & 1.155                                                 \\
                  4         & 0.012           & \textcolor{red}{2.274} & 1.311                                                 \\
                  3         & 0.003           & \textcolor{red}{0.651} & 8.468                                                 \\
                  1         & 0.001           & \textcolor{red}{0.163} & 4.288                                                 \\
                  \hline
                  $n = 187$ & 1.000           &                        & $X^2 = 18.749$                                        \\
                  \hline
              \end{tabular}
              $\Rightarrow$
              \begin{tabular}{|c|c|c|c|c|}
                  \hline
                  $\nu_i$   & $\widehat{p}_i$ & $n \widehat{p}_i$ & $\frac{(\nu_i - n \widehat{p}_i)}{n \widehat{p}_i}$ \\
                  \hline
                  28        & 0.135           & 25.173            & 0.318                                               \\
                  49        & 0.270           & 50.480            & 0.043                                               \\
                  57        & 0.271           & 50.615            & 0.805                                               \\
                  25        & 0.181           & 33.834            & 2.306                                               \\
                  16        & 0.091           & 16.962            & 0.055                                               \\
                  12        & 0.053           & 9.891             & 0.450                                               \\
                  \hline
                  $n = 187$ & 1.000           &                   & $X^2 = 3.977$                                       \\
                  \hline
              \end{tabular}

              Степени свободы $6-1-1=4$, максимальный уровень значимости $\probability{\chi_4^2 \ge 3.977} \approx 0.40915$.

        \item оценка $\widehat{p} = 0.22282$, вероятности $\widehat{p}_i = C_8^i \widehat{p}^i (1-\widehat{p})^{8-i}$.

              \begin{tabular}{|c|c|c|c|c|}
                  \hline
                  $\nu_i$   & $\widehat{p}_i$ & $n \widehat{p}_i$      & $\frac{(\nu_i - n \widehat{p}_i)}{n \widehat{p}_i}$ \\
                  \hline
                  28        & 0.103           & 19.344                 & 3.873                                               \\
                  49        & 0.267           & 49.914                 & 0.017                                               \\
                  57        & 0.306           & 57.241                 & 0.001                                               \\
                  25        & 0.205           & 38.292                 & 4.614                                               \\
                  16        & 0.088           & 16.467                 & 0.013                                               \\
                  4         & 0.025           & \textcolor{red}{4.721} & 0.110                                               \\
                  4         & 0.005           & \textcolor{red}{0.902} & 10.634                                              \\
                  3         & 0.001           & \textcolor{red}{0.111} & 75.286                                              \\
                  1         & 0.000           & \textcolor{red}{0.008} & 123.847                                             \\
                  \hline
                  $n = 187$ & 1.000           &                        & $X^2 = 218.394$                                     \\
                  \hline
              \end{tabular}
              $\Rightarrow$
              \begin{tabular}{|c|c|c|c|c|}
                  \hline
                  $\nu_i$   & $\widehat{p}_i$ & $n \widehat{p}_i$ & $\frac{(\nu_i - n \widehat{p}_i)}{n \widehat{p}_i}$ \\
                  \hline
                  28        & 0.103           & 19.344            & 3.873                                               \\
                  49        & 0.267           & 49.914            & 0.017                                               \\
                  57        & 0.306           & 57.241            & 0.001                                               \\
                  25        & 0.205           & 38.292            & 4.614                                               \\
                  16        & 0.088           & 16.467            & 0.013                                               \\
                  12        & 0.031           & 5.742             & 6.819                                               \\
                  \hline
                  $n = 187$ & 1.000           &                   & $X^2 = 15.337$                                      \\
                  \hline
              \end{tabular}

              Степени свободы 6-1-1 = 4, максимальный уровень значимости $\probability{\chi_4^2 \ge 15.337} \approx 0.00405$.

              \item Распределение Пуассона лучше согласуется с данными, чем биномиальное.
    \end{enumerate}
\fi

\subsection*{Задача 2}

По результатам КМ-1 студенты четвёртого курса получили следующие количества оценок:

\begin{center}
    \begin{tabular}{|P{2cm}|P{1cm}|P{1cm}|P{1cm}|P{1cm}|}
        \hline
        Группа  & 2  & 3 & 4  & 5  \\
        \hline
        A-05-21 & 13 & 2 & 6  & 3  \\
        \hline
        A-13-21 & 0  & 1 & 13 & 11 \\
        \hline
        A-14-21 & 1  & 2 & 7  & 3  \\
        \hline
        A-16-21 & 5  & 2 & 4  & 4  \\
        \hline
        A-18-21 & 10 & 3 & 6  & 4  \\
        \hline
    \end{tabular}
\end{center}

При каком наименьшем уровне значимости отклоняется гипотеза об однородности успеваемости в группах?

\ifsolutions Решение: \par
    $n = 100$, оценки $\widehat{\theta} \approx (0.29, 0.1, 0.36, 0.25)$, статистика $X^2 = 3277124.138$, уровень значимости 0.
\fi

\subsection*{Задача 3 \cite[230]{Efimov}}

Давление в камере контролируется двумя манометрами. Считается, что показания манометров независимы и имеют нормальное распределение с неизвестными
математическими ожиданиями и дисперсиями. По результатам $n = 10$ замерам каждого из манометров вычислены: выборочные средние
$\widehat{m}_1^{(1)} = 15.3$, $\widehat{m}_1^{(2)} = 16.1$ и исправленные выборочные дисперсии $\widetilde{\mu}_{2}^{(1)} = 0.2$,
$\widetilde{\mu}_{2}^{(2)} = 0.15$. Для уровня значимости $\alpha = 0.1$ проверьте гипотезы:
\begin{enumerate}
    \item о равенстве дисперсий показаний с двусторонней критической областью,
    \item о равенстве математических ожиданий показаний с правосторонней критической областью.
\end{enumerate}
\ifsolutions Решение: \par
    \begin{enumerate}
        \item Статистика 1.(3), критическая область $(-\infty, 0.31457) \cup (3.17889, \infty)$, гипотеза принимается.
        \item Статистика -4.27618, критическая область $(-\infty, -1.72913) \cup (1.72913, \infty)$, гипотеза отклоняется.
    \end{enumerate}
\fi

\subsection*{Задача 4}

Получена реализации выборки: 0.768, 1.31, 2.11, 1.29, 0.868, 1.59, 1.49, 3.24, 7.17, 2.1, 0.0434, 1.82, 0.278, 2.1, 1.06, 7.95, 1.69, 0.809, 1.63, 1.15.
Используя критерий согласия Колмогорова определите из какого распределения получена выборка из равномерного или показательного.

Получена ещё одна реализация выборки: 3.72, 0.49, 1.59, 1.21, 0.811, 2.63, 2.42, 0.265, 2.24, 0.596, 0.725, 3.5, 1.38, 0.0537, 3.86.
Проверьте, что распределения двух выборок совпадают, по критерию Колмогорова--Смирнова.

\ifsolutions Решение: \par
    \begin{enumerate}
        \item Тест Колмогорова для $E(2.023)$: статистика 0.21585 при аргументе 0.768, уровень значимости 0.26818.
        \item Тест Колмогорова для $R[0.043, 7.950]$: статистика 0.58862 при аргументе 2.11, уровень значимости 0.
        \item Тест Колмогорова--Смирнова: статистика 0.25 при аргументе 2.11, уровень значимости 0.58703.
    \end{enumerate}
\fi

\subsection*{Задача 5 \cite[206, 207, 210]{Efimov}}

$\left( \xi_1, \dots, \xi_n \right)$ --- выборка из нормального распределения $\mathcal{N} \left( m, \sigma^2 \right)$, где $m$ --- неизвестное,
а $\sigma = 2$. Для проверки гипотез в отношении $m$ используется критерий со статистикой:
\[
    \widehat{m}_1 = \frac{1}{n} \sum_{i=1}^n \xi_i .
\]
\begin{enumerate}
    \item Постройте критерий проверки гипотезы основной гипотезы $H_0: m \le m_0$ против альтернативной $H_1: m > m_0$. Выполните проверку гипотезы
          для $m_0 = 40$, $n = 64$ и уровне значимости $\alpha = 0.01$, в результате эксперимента получено значение выборочного среднего $40.2$.
    \item Постройте критерий проверки основной гипотезы $H_0: m = m_0$ против альтернативной гипотезы $H_1: m = m_1$, где $m_1 < m_0$. Какова ошибка
          второго рода $\beta$ критерия при $m_0 = 40$, $m_1 = 39.2$, $n=36$ и уровне значимости $\alpha = 0.1$?
    \item Постройте критерий проверки гипотез $H_0: m = m_0 = 40$, $H_1: m = m_1 = 41$, $H_2: m = m_2 = 41.5$, у которого вероятность ошибки первого
          рода $\alpha \le 0.05$ и вероятность ошибки третьего рода $\gamma \le 0.03$ при $n = 100$. Какова величина ошибки второго рода $\beta$
          такого критерия?
    \item Пусть $n=36$ и критическая область $\Gamma = (40.2, \infty)$. Постройте график функции мощности на отрезке $m \in [39, 41]$.
\end{enumerate}

\ifsolutions Решение: \par
    Cтатистика $\widehat{m}_1 \sim \mathcal{N} (m, \frac{\sigma}{\sqrt{n}})$.

    \begin{enumerate}
        \item Правосторонняя критическая область $\Gamma_\alpha = (h_\alpha, + \infty)$, где:
              \begin{gather*}
                  \probability{\widehat{m}_1 \ge h_\alpha | m_0} \le \alpha , \\
                  %
                  \frac{h_\alpha - m_0}{\frac{\sigma}{\sqrt{n}}} \le \Phi^{-1} (1 - \alpha) , \\
                  %
                  h_\alpha
                  \le m_0 + \frac{\sigma}{\sqrt{n}} \Phi^{-1} (1 - \alpha)
                  \approx 40.54828 .
              \end{gather*}
              $40.2 < h_\alpha$, принимается $H_0$.

        \item Левосторонняя критическая область $\Gamma_\alpha = (- \infty, h_\alpha)$, где        \begin{gather*}
                  \probability{\widehat{m}_1 \le h_\alpha | m_0} \le \alpha , \\
                  %
                  \frac{h_\alpha - m_0}{\frac{\sigma}{\sqrt{n}}} \le \Phi^{-1} (\alpha) , \\
                  %
                  h_\alpha
                  \le m_0 + \frac{\sigma}{\sqrt{n}} \Phi^{-1} (1 - \alpha)
                  \approx 39.57282 .
              \end{gather*}
              Ошибка второго рода:
              \begin{gather*}
                  \probability{\widehat{m}_1 \ge h_\alpha | m_1}
                  = 1 - \Phi \left( \frac{h_\alpha - m_1}{\frac{\sigma}{\sqrt{n}}} \right)
                  \approx 0.13169
              \end{gather*}

        \item Области принятия гипотез $\Pi_0 = (- \infty, h_0 )$, $\Pi_1 = [h_0, h_1]$, $\Pi_2 = (h_1, \infty )$. Пороги из вероятностей ошибок:
              \begin{gather*}
                  \alpha = \probability{\widehat{m}_1 \ge h_0 | m_0} = 1 - \Phi \left( \frac{h_0 - m_0}{\frac{\sigma}{\sqrt{n}}} \right) = 0.05, \\
                  h_0 = m_0 + \frac{\sigma}{\sqrt{n}} \Phi^{-1} (1 - 0.05) \approx 40.32897
              \end{gather*}
              \begin{gather*}
                  \gamma = \probability{\widehat{m}_2 \le h_1 | m_2} = \Phi \left( \frac{h_1 - m_2}{\frac{\sigma}{\sqrt{n}}} \right) = 0.03, \\
                  h_1 = m_2 + \frac{\sigma}{\sqrt{n}} \Phi^{-1} (0.03) \approx 41.12384 ,
              \end{gather*}
              тогда
              \begin{multline*}
                  \beta
                  = \probability{\widehat{m}_1 < h_0 | m_1} + \probability{\widehat{m}_1 > h_1 | m_1} = \\
                  %
                  = \Phi \left( \frac{h_0 - m_1}{\frac{\sigma}{\sqrt{n}}} \right) + 1 - \Phi \left( \frac{h_1 - m_1}{\frac{\sigma}{\sqrt{n}}} \right)
                  \approx 0.26829
              \end{multline*}

        \item Мощность:
              \[
                  W(m)
                  = \probability{\widehat{m}_1 > 40.2 | m}
                  = 1 - \Phi \left( \frac{40.2 - m}{\frac{\sigma}{\sqrt{n}}} \right)
              \]
              \begin{tabular}{|c|c|c|c|c|c|c|c|}
                  \hline
                  39.00 & 0.000000 & 39.50 & 0.000233 & 40.00 & 0.158655 & 40.50 & 0.933193 \\
                  39.05 & 0.000000 & 39.55 & 0.000577 & 40.05 & 0.226627 & 40.55 & 0.959941 \\
                  39.10 & 0.000000 & 39.60 & 0.001350 & 40.10 & 0.308538 & 40.60 & 0.977250 \\
                  39.15 & 0.000000 & 39.65 & 0.002980 & 40.15 & 0.401294 & 40.65 & 0.987776 \\
                  39.20 & 0.000000 & 39.70 & 0.006210 & 40.20 & 0.500000 & 40.70 & 0.993790 \\
                  39.25 & 0.000001 & 39.75 & 0.012224 & 40.25 & 0.598706 & 40.75 & 0.997020 \\
                  39.30 & 0.000003 & 39.80 & 0.022750 & 40.30 & 0.691462 & 40.80 & 0.998650 \\
                  39.35 & 0.000011 & 39.85 & 0.040059 & 40.35 & 0.773373 & 40.85 & 0.999423 \\
                  39.40 & 0.000032 & 39.90 & 0.066807 & 40.40 & 0.841345 & 40.90 & 0.999767 \\
                  39.45 & 0.000088 & 39.95 & 0.105650 & 40.45 & 0.894350 & 40.95 & 0.999912 \\
                  \hline
                        &          &       &          &       &          & 41.00 & 0.999968 \\
                  \hline
              \end{tabular}
    \end{enumerate}
\fi

\section*{Домашнее задание 5}

\subsection*{Задача 1 \cite[278]{Efimov}}

Время химической реакции при различном содержании катализатора распределилось следующим образом:

\begin{tabular}{|c|c|c|c|c|c|c|c|c|c|c|c|c|}
    \hline
         & 1   & 2   & 3   & 4   & 5   & 6   & 7   & 8   & 9   & 10  & 11  & 12  \\
    \hline
    5\%  & 5.9 & 6.0 & 7.0 & 6.5 & 5.5 & 7.0 & 8.1 & 7.5 & 6.2 & 6.4 & 7.1 & 6.9 \\
    \hline
    10\% & 4.0 & 5.1 & 6.2 & 5.3 & 4.5 & 4.4 & 5.3 & 5.4 & 5.6 & 5.2 & -   & -   \\
    \hline
    15\% & 8.2 & 6.8 & 8.0 & 7.5 & 7.0 & 7.2 & 7.9 & 8.1 & 8.5 & 7.8 & 8.1 &     \\
    \hline
\end{tabular}

Проверить гипотезу о равенстве времен реакции при различном содержании катализатора для уровня значимости $\alpha = 0.1$.

\ifsolutions
    Решение:

    \begin{tabular}{|P{1cm}||P{3.6cm}|P{2cm}|P{0.5cm}|P{2.5cm}|P{3.6cm}|}
        \hline
                       & $\sum_{i=1}^{n_j} \left( \xi_i^{(j)} \right)^2$ & $\sum_{i=1}^{n_j} \xi_i^{(j)}$  & $n_j$ & $\left( \sum_{i=1}^{n_j} \xi_i^{(j)} \right)^2$ & $\frac{1}{n_j} \left( \sum_{i=1}^{n_j} \xi_i^{(j)} \right)^2$ \\
        \hline
                       & 540.59                                          & 80.1                            & 12    & 6416.01                                         & 534.668                                                       \\
        \hline
                       & 263.8                                           & 51                              & 10    & 2601                                            & 260.1                                                         \\
        \hline
                       & 661.29                                          & 85.1                            & 11    & 7242.01                                         & 658.365                                                       \\
        \hline
        \hline
        $\sum_{j=1}^k$ & 1465.68                                         & 216.2                           & 33    &                                                 & 1453.13                                                       \\
        \hline
                       & $1465.68 - \frac{216.2^2}{33} = 49.242$         & $\frac{216.2^2}{33} = 1416.438$ &       &                                                 & $1453.13 - \frac{216.2^2}{33} = 36.694$                       \\
        \hline
    \end{tabular}

    $n \widehat{s}^2(\xi) = 49.242 - 39.694 = 12.548$,
    $T = \frac{36.694}{3-1} \frac{33 - 3}{12.648} \approx 43.865$,
    $h_\alpha = Q_{F(3-1,33-3)}(1 - \alpha) \approx 2.489$,
    $T > h_\alpha$ : гипотеза отклоняется.

    Ответ: гипотеза о равенстве отклоняется.
\fi

\subsection*{Задача 2}

В выборке $\left( \xi_1, \dots, \xi_n \right)$ случайные величины бинарные:
\[
    \xi_i
    = \left \{
    \begin{array}{ll}
        1, & \text{с вероятностью } p   \\
        0, & \text{с вероятностью } 1-p
    \end{array}
    \right .
\]
В отношении неизвестной вероятности $p$ выдвигаются основная гипотеза $H_0: p = p_0$ и альтернативная $H_1: p = p_1$. Постройте минимаксный критерий
проверки гипотезы $H_0$ против $H_1$ для $p_0 = 0.4$, $p_1 = 0.7$ и объёма выборки $n = 100$. Вычислите вероятности ошибок первого и второго рода
полученного критерия.

\subsection*{Задача 3}

Задана выборка $\left( \xi_1, \dots, \xi_n \right)$ из распределения Пуассона $\mathcal{P}(\lambda)$ с неизвестным параметром $\lambda$. Выдвигаются
две гипотезы: основная $H_0: \lambda = \lambda_0$ и альтернативная $H_1: \lambda = \lambda_1$, где $\lambda_0 = 2$ и $\lambda_1 = 2.2$. Постройте
критерии Неймана--Пирсона в следующих случаях:
\begin{enumerate}
    \item задан объём выборки $n = 50$ и вероятность ошибки первого рода $\alpha = 0.05$; определите вероятность ошибки второго рода.
    \item заданы вероятности ошибок: первого рода $\alpha = 0.01$ и второго рода $\beta = 0.05$.
\end{enumerate}

\subsection*{Задача 4}

Случайные величины $\xi_i$ являются промежутками времени между автомобилями, проезжающими по автостраде. Считается, что величины $\xi_i$ являются
независимыми в совокупности и имееют одинаковую плотность вероятности $p_\xi(x)$, которая неизвестна. Рассматриваются две гипотезы:
\begin{gather*}
    H_0:
    p_\xi(x) = \left \{
    \begin{array}{ll}
        0,                                       & x < 0   \\
        \sqrt{\frac{2}{\pi}} e^{-\frac{x^2}{2}}, & x \ge 0
    \end{array}
    \right . , \\
    %
    H_1:
    p_\xi(x) = \left \{
    \begin{array}{ll}
        0,      & x < 0   \\
        e^{-x}, & x \ge 0
    \end{array}
    \right . .
\end{gather*}
Постройте последовательный критерий отношения вероятностей проверки гипотезы $H_0$ против $H_1$ для вероятностей ошибок $\alpha = 0.02$ и $\beta = 0.03$.
Определите средние количества наблюдений до остановки критерия.

\begin{thebibliography}{1}
    \bibitem{Efimov} Сборник задач по математике, часть 4 // Под редакцией Ефимова А.В. и Поспелова А.С. - М.: Издательство Физико-математической
    литературы, 2003. - 432 с.
\end{thebibliography}
\end{document}