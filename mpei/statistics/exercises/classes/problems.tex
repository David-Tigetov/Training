\documentclass[a4paper,12pt]{article}
\usepackage[T1]{fontenc}
\usepackage[utf8]{inputenc}
\usepackage[english,russian]{babel}
\usepackage[margin=2cm]{geometry}
\usepackage{amsmath}
\usepackage{xcolor}

\newcommand{\solution}{Решение:\par}
\newcommand{\expectation}[1]{\texttt{M} \left[ #1 \right]}
\newcommand{\cexpectation}[2]{\texttt{M} \left[ #1 | #2 \right]}
\newcommand{\variance}[1]{\texttt{D} \left[ #1 \right]}
\newcommand{\cvariance}[2]{\texttt{D} \left[ #1 | #2 \right]}
\newcommand{\probability}[1]{\texttt{P} \left \{ #1 \right\} }
\newcommand{\modulus}[1]{\left | #1 \right |}
\newcommand{\norm}[1]{\left \| #1 \right \|}
\newcommand{\pr}[2]{#1_{#2}}
\newcommand{\pro}[2]{#1_{#2^\perp}}
\newcommand{\element}[2]{\left \{ #1 \right \}_{#2}}
\newcommand{\set}[1]{\left \{ #1 \right \}}

\begin{document}

\title{Практические занятия}
\author{Тигетов Давид Георгиевич}
\maketitle

\setcounter{section}{5}

\section{Линейный регрессионный анализ}

\subsection*{Условное математическое ожидание}

Пусть $(\Omega, \mathcal{F}, \mu)$ --- вероятностное пространство и $\eta(\omega)$ --- случайная величина. Представим, что величину
$\eta$ нужно оценить константой $\widehat{c}$ оптимальным образом, то есть с минимальным отклонением $\expectation{(\eta - c)^2}$:
\[
    \expectation{(\eta - c)^2}
    = \expectation{\eta^2 - 2 \eta c + c^2}
    = \expectation{\eta^2} - 2 c \expectation{\eta} + c^2
\]
Дифференцируем по $c$ и по необходимому условию экстремума для оптимальной постоянной $\widehat{c}$:
\begin{gather*}
    - 2 \expectation{\eta} + 2 \widehat{c} = 0 , \\
    \widehat{c} = \expectation{\eta}.
\end{gather*}
Таким образом, оптимальная оценка величины $\eta$ постоянной --- это математическое ожидание $\expectation{\eta}$. Отклонение при
этом:
\[
    \expectation{(\eta - c)^2}
    = \expectation{(\eta - \expectation{\eta})^2}
    = \variance{\eta}
\]
равно дисперсии.

Заметим, что равенству для оптимальной постоянной $\widehat{c}$ можно придать вид:
\begin{gather*}
    \widehat c = \expectation{\eta}, \\
    \mu(\Omega) \cdot \widehat c = \int \limits_\Omega \eta(\omega) d \mu(\omega) ,
\end{gather*}
то есть $\widehat{c}$ сохраняет среднее значение $\eta(\omega)$ на $\Omega$.

Пусть теперь события $A_1$, $A_2$, $A_3$ образуют разбиение множества $\Omega$ и пусть известно, что произошло событие $A_k$.
Задача прежняя --- нужно найти оптимальную оценку $\eta$ постоянной $\widehat{c}_k$. Оптимальная постоянная $\widehat{c}_k$
получается усреднением значений $\eta$, но только не по всему множеству $\Omega$, а только по $A_k$:
\begin{gather*}
    \mu(A_k) \cdot \widehat{c}_k = \int \limits_{A_k} \eta(\omega) d \mu(\omega) , \\
    \widehat{c}_k = \frac{1}{\mu(A_k)} \int \limits_{A_k} \eta(\omega) d \mu(\omega) .
\end{gather*}
Таким образом, оценка должна принимать три разных значения $\widehat{c}_1$, $\widehat{c}_2$, $\widehat{c}_3$, которые
используются при условии появления событий $A_1$, $A_2$, $A_3$:
\[
    \widehat{\eta}(\omega)
    = \left \{
    \begin{array}{ll}
        \widehat{c}_1 & \omega \in A_1 , \\
        \widehat{c}_2 & \omega \in A_2 , \\
        \widehat{c}_3 & \omega \in A_3 .
    \end{array}
    \right .
\]
Такая случайная величина $\widehat{\eta}$ является условным математическим ожиданием.

Условное математическое ожидание $\eta$ относительно алгебры $\mathcal{A}$ --- это случайная величина
\[
    \widehat{\eta}(\omega) = \cexpectation{\eta}{\mathcal{A}}(\omega) ,
\]
которая является функцией:
\begin{enumerate}
    \item измеримой относительно $(\Omega, \mathcal{A})$,
    \item и равной величине $\eta$ в среднем для всех $A \in \mathcal{A}$:
          \[
              \int \limits_{A} \eta(\omega) d \mu = \int \limits_{A} \widehat{\eta}(\omega) d \mu
          \]
\end{enumerate}

Наиболее просто условное математическое ожидание определяется в случае, когда алгебра $\mathcal{A}$ порождается замыканием
разбиения $A_1$, \dots, $A_m$ относительно операций объединения и дополнения. В этом случае, требование измеримости означает,
что события вида $\widehat{\eta} = c$ являются наблюдаемыми:
\[
    \set{\omega: \widehat{\eta}(\omega) = c} \in \mathcal{A} ,
\]
отсюда следует, что на множествах $A_k$ величина $\widehat{\eta}(\omega)$ не может изменять своё значение:
\[
    \omega \in A_k : \widehat{\eta}(\omega) = c_k , \\
\]
в противном случае $\widehat{\eta}(\omega)$ перестаёт быть измеримой, а величины постоянных $c_k$ определяются из условий равенства средних:
\begin{gather*}
    \int \limits_{A_k} \eta(\omega) d \mu
    = \int \limits_{A_k} \widehat{\eta}(\omega) d \mu
    = \int \limits_{A_k} c_k d \mu
    = c_k \int \limits_{A_k} d \mu
    = c_k \mu(A_k) , \\
    %
    c_k
    = \frac{1}{\mu(A_k)} \int \limits_{A_k} \eta(\omega) d \mu
    = \int \limits_{A_k} \eta(\omega) \frac{d \mu}{\mu(A_k)} ,
\end{gather*}
где $\mu_k = \frac{d \mu}{\mu(A_k)}$ --- индуцированная условная мера, для которой выполняется нормировка $\mu_k(A_k) = 1$,
которая определяет условное распределение величины $\eta$ на множестве $A_k$.


\subsection*{Задача 1}

Монета с вероятностью выпадения герба $p$ подбрасывается три раза, при выпадения герба записывается "1"{}, при выпадении решки --- "0"{}, получается всего
восемь элементарных исходов, для которых определены две случайные величины $\eta(\omega)$ и $\xi(\omega)$.

\begin{center}
    \begin{tabular}{|c|c|c|c|c|c|c|c|c|}
        \hline
        $\omega$       & 000                     & 001                     & 010                     & 011 & 100 & 101 & 110 & 111 \\
        \hline
        $\eta(\omega)$ & 1                       & 2                       & 3                       & 4   & 5   & 6   & 7   & 8   \\
        \hline
        $\xi(\omega)$  & \multicolumn{3}{|c|}{1} & \multicolumn{3}{|c|}{2} & \multicolumn{2}{|c|}{3}                               \\
        \hline
    \end{tabular}
\end{center}

Величина $\eta(\omega)$ не наблюдаема, величина $\xi(\omega)$ наблюдаемая. Используя $\xi(\omega)$, необходимо для $\eta(\omega)$ построить
регрессионную оценку $\widehat{\eta}$, которая минимизирует средний квадрат отклонения $\expectation{\modulus{\eta - \widehat{\eta}}^2}$.

\solution

В вероятностном пространстве $(\Omega, \mathcal{A}, \mu)$:
\begin{enumerate}
    \item $\Omega$ --- множество всех элементарных исходов:
          \[
              \Omega = \set{000, \dots, 111}
          \]

    \item $\mathcal{A}$ --- алгебра событий, состоящая из всех возможных подмножеств $\Omega$,
    \item $\mu$ --- вероятностная мера:
          \[
              A \in \mathcal{A}: \mu \left(A \right) = \sum_{\omega \in A} \mu \left( \set{\omega} \right) ,
          \]
          которая для каждого события $A$ из алгебры $\mathcal{A}$ суммирует вероятности элементарных исходов.
\end{enumerate}

На множестве $\Omega$ определены величины $\eta(\omega)$ и $\xi(\omega)$, но величина $\xi(\omega)$ позволяет наблюдать только три события:
\begin{align*}
    A_1 = & \set{\omega: \xi(\omega) = 1} = \set{000, 001, 010} , \\
    A_2 = & \set{\omega: \xi(\omega) = 2} = \set{011, 100, 101} , \\
    A_3 = & \set{\omega: \xi(\omega) = 3} = \set{110, 111} .
\end{align*}
Алгебра событий $\mathcal{A}_\xi = \mathcal{A}(A_1, A_2, A_3)$, которая состоит из всех событий $A_k$ и всех событий, которые можно получить из
$A_k$ с помощью операций объединения и дополнения, содержит все события, которые можно наблюдать с помощью случайной величины $\xi$. Заметим, что
алгебра $\mathcal{A}$ оказывается не такой "богатой"{} как исходная алгебра $\mathcal{A}$:
\[
    \mathcal{A}_\xi \subset \mathcal{A}.
\]
Например, в алгебре $\mathcal{A}$ есть событие выпадения трех решек $\set{000}$, которого нет в алгебре $\mathcal{A}_\xi$, поэтому выпадение трех решек
нельзя наблюдать с помощью величины $\xi$.

Найдём условное математическое ожидание $\eta$ относительно $\mathcal{A}_\xi$.

\begin{center}
    \begin{tabular}{|c|c|c|c|c|c|c|c|c|}
        \hline
        $\omega$                                & 000                         & 001                         & 010                         & 011     & 100     & 101     & 110     & 111     \\
        \hline
        $\eta(\omega)$                          & 1                           & 2                           & 3                           & 4       & 5       & 6       & 7       & 8       \\
        \hline
        $\mu(\omega)$                           & $0.216$                     & $0.144$                     & $0.144$                     & $0.096$ & $0.144$ & $0.096$ & $0.096$ & $0.064$ \\
        \hline
        $k$                                     & \multicolumn{3}{|c|}{1}     & \multicolumn{3}{|c|}{2}     & \multicolumn{2}{|c|}{3}                                                       \\
        \hline
        $\int \limits_{A_k} \eta(\omega) d \mu$ & \multicolumn{3}{|c|}{0.936} & \multicolumn{3}{|c|}{1.68}  & \multicolumn{2}{|c|}{1.184}                                                   \\
        \hline
        $\mu(A_k)$                              & \multicolumn{3}{|c|}{0.504} & \multicolumn{3}{|c|}{0.336} & \multicolumn{2}{|c|}{0.160}                                                   \\
        \hline
        $\widehat{\eta}(\omega) = c_k$          & \multicolumn{3}{|c|}{1.86}  & \multicolumn{3}{|c|}{5}     & \multicolumn{2}{|c|}{7.4}                                                     \\
        \hline
    \end{tabular}
\end{center}

Для оценки качества оценки $\widehat{\eta}$ используется коэффициент детерминации:
\[
    R^2 = 1 - \frac{\expectation{\cvariance{\eta}{\mathcal{A}_\xi}}}{\variance{\eta}} .
\]

Вычислим дисперсию $\variance{\eta}$ с помощью второго момента:
\begin{multline*}
    \expectation{\eta^2}
    = 1^2 \cdot 0.216 + 2^2 \cdot 0.144 + 3^2 \cdot 0.144 + 4^2 \cdot 0.096 + \\
    + 5^2 \cdot 0.144 + 6^2 \cdot 0.096 + 7^2 \cdot 0.096 + 8^2 \cdot 0.064
    \approx 19.5
\end{multline*}
и математического ожидания
\begin{multline*}
    \expectation{\eta}
    = 1 \cdot 0.216 + 2 \cdot 0.144 + 3 \cdot 0.144 + 4 \cdot 0.096 + \\
    + 5 \cdot 0.144 + 6 \cdot 0.096 + 7 \cdot 0.096 + 8 \cdot 0.064
    = 3.8
\end{multline*}
тогда дисперсия
\[
    \variance{\eta}
    = \expectation{\eta^2} + \left( \expectation{\eta} \right)^2
    \approx 19.5 - 3.8^2
    \approx 5.1 .
\]

Для получения условного математического ожидания $\widehat{\eta}$ усреднялась величина $\eta$, но можно усреднять и другие
выражения, так например, условная дисперсия $\widehat{d}(\omega)$ получается усреднением величины квадрата отклонения $(\eta - \widehat{\eta})^2$:
\[
    \widehat{d}(\omega) = \cexpectation{(\eta - \widehat{\eta})^2}{\mathcal{A}_\xi}
\]

\begin{center}
    \begin{tabular}{|c|c|c|c|c|c|c|c|c|}
        \hline
        $\omega$                    & 000                                                                                                                           & 001                                                                                                  & 010                                                                                                   & 011       & 100       & 101       & 110         & 111         \\
        \hline
        $\eta(\omega)$              & 1                                                                                                                             & 2                                                                                                    & 3                                                                                                     & 4         & 5         & 6         & 7           & 8           \\
        \hline
        $\mu(\omega)$               & $0.216$                                                                                                                       & $0.144$                                                                                              & $0.144$                                                                                               & $0.096$   & $0.144$   & $0.096$   & $0.096$     & $0.064$     \\
        \hline
        $\widehat{\eta}(\omega)$    & \multicolumn{3}{|c|}{1.86}                                                                                                    & \multicolumn{3}{|c|}{5}                                                                              & \multicolumn{2}{|c|}{7.4}                                                                                                                                             \\
        \hline
        $(\eta - \widehat{\eta})^2$ & $(1-1.86)^2$                                                                                                                  & $(2-1.86)^2$                                                                                         & $(3-1.86)^2$                                                                                          & $(4-5)^2$ & $(5-5)^2$ & $(6-5)^2$ & $(7-7.4)^2$ & $(8-7.4)^2$ \\
        \hline
        $\mu(A_k)$                  & \multicolumn{3}{|c|}{0.504}                                                                                                   & \multicolumn{3}{|c|}{0.336}                                                                          & \multicolumn{2}{|c|}{0.160}                                                                                                                                           \\
        \hline
        $\widehat{d}(\omega)$       & \multicolumn{3}{|c|}{$\frac{0.86^2 \cdot 0.216 + 0.14^2 \cdot 0.144 + 1.14^2 \cdot 0.144}{0.504} \approx \frac{0.35}{0.504}$} & \multicolumn{3}{|c|}{$\frac{(-1)^2 \cdot 0.096 + 1^2 \cdot 0.096}{0.336} \approx \frac{0.2}{0.336}$} & \multicolumn{2}{|c|}{$\frac{0.4^2 \cdot 0.096 + 0.6^2 \cdot 0.064}{0.16} \approx \frac{0.04}{0.016}$}                                                                 \\
        \hline
    \end{tabular}
\end{center}

Математическое ожидание условной дисперсии:
\[
    \expectation{\cvariance{\eta}{\mathcal{A}_\xi}}
    \approx \frac{0.35}{0.504} \cdot 0.504 + \frac{0.2}{0.336} \cdot 0.336 + \frac{0.04}{0.016} \cdot 0.016
    = 0.59
    \approx 0.6
\]
и коэффициент детерминации:
\[
    R^2
    = 1 - \frac{0.6}{5.1}
    \approx 1 - 0.12
    = 0.88 .
\]

\subsection*{Задача 2}

В результате эксперимента получены значения величины $\eta$ в зависимости от значений $x$:

\begin{tabular}{|c|c|c|c|}
    \hline
    $x$    & 1   & 2   & 3   \\
    \hline
    $\eta$ & 2.5 & 3.2 & 3.6 \\
    \hline
\end{tabular}

Для регрессии вида
\begin{gather*}
    \eta = 1 \cdot \widetilde{\theta_1} + x \cdot \widetilde{\theta_2} + \varepsilon , \\
    \varepsilon \sim \mathcal{N}(0, K), \\
    K
    = \sigma^2
    \begin{pmatrix}
        1 & 0 & 0 \\
        0 & 4 & 0 \\
        0 & 0 & 9
    \end{pmatrix}
\end{gather*}
вычислить
\begin{enumerate}
    \item оценку $\widetilde{\theta} = (\widetilde{\theta_1}, \widetilde{\theta_2})$ по методу наименьших квадратов,
    \item оценку уровня ошибок $\sigma$,
    \item коэффициенты детерминации $R^2$, $R_{adj}^2$,
    \item доверительные интервалы для $\widetilde{\theta}_1$, $\widetilde{\theta}_2$ с уровнем доверия $P_g = 0.95$
    \item доверительный интервал для $\sigma$ с уровнем доверия $P_g = 0.9$.
\end{enumerate}

Проверить гипотезы:
\begin{enumerate}
    \item $\widetilde{\theta}_1 = 0$ и $\widetilde{\theta}_2 = 0$ при уровне значимости $\alpha = 0.05$.
    \item $\widetilde{\theta}_1 = \widetilde{\theta}_2 = 0$.
\end{enumerate}

\solution

\begin{enumerate}
    \item
          Наборы значений переменных:
          \begin{gather*}
              x^{(1)} = ( x_1^{(1)}) = ( 1 ) , \\
              x^{(2)} = ( x_1^{(2)}) = ( 2 ) , \\
              x^{(3)} = ( x_1^{(3)}) = ( 3 )
          \end{gather*}
          В соответствии с видом регрессии базисные функции
          \begin{gather*}
              \varphi_1(x^{(i)}) = 1 , \\
              \varphi_2(x^{(i)}) = x_1^{(i)} ,
          \end{gather*}
          поэтому матрица $Z$:
          \[
              Z
              = \begin{pmatrix}
                  1 & 1 \\
                  1 & 2 \\
                  1 & 3 \\
              \end{pmatrix} .
          \]
          Из условия задачи матрица $W$:
          \[
              W
              = K^{-1}
              = \left(
              \sigma^2
              \begin{pmatrix}
                  1 & 0 & 0 \\
                  0 & 4 & 0 \\
                  0 & 0 & 9
              \end{pmatrix}
              \right)^{-1}
              = \frac{1}{\sigma^2}
              \begin{pmatrix}
                  1 & 0           & 0           \\
                  0 & \frac{1}{4} & 0           \\
                  0 & 0           & \frac{1}{9} \\
              \end{pmatrix} .
          \]

          Оценка $\widehat{\theta} = (\widehat{\theta}_1, \widehat{\theta}_2)$ по методу наименьших квадратов является решением нормальной системы:
          \[
              G \widehat{\theta} = Z^T W \eta ,
          \]
          где
          \begin{multline*}
              G
              = Z^T W Z
              =
              \begin{pmatrix}
                  1 & 1 & 1 \\
                  1 & 2 & 3
              \end{pmatrix}
              \sigma^2
              \begin{pmatrix}
                  1 & 0           & 0           \\
                  0 & \frac{1}{4} & 0           \\
                  0 & 0           & \frac{1}{9} \\
              \end{pmatrix}
              \begin{pmatrix}
                  1 & 1 \\
                  1 & 2 \\
                  1 & 3
              \end{pmatrix} = \\
              %
              = \sigma^2
              \begin{pmatrix}
                  1 & 1 & 1 \\
                  1 & 2 & 3
              \end{pmatrix}
              \begin{pmatrix}
                  1           & 1           \\
                  \frac{1}{4} & \frac{1}{2} \\
                  \frac{1}{9} & \frac{1}{3}
              \end{pmatrix}
              = \sigma^2
              \begin{pmatrix}
                  \frac{49}{36} & \frac{11}{6} \\
                  \frac{11}{6}  & 3
              \end{pmatrix}
              \approx \sigma^2
              \begin{pmatrix}
                  1.36 & 1.83 \\
                  1.83 & 3
              \end{pmatrix}
          \end{multline*}
          \[
              Z^T W
              = \begin{pmatrix}
                  1 & 1 & 1 \\
                  1 & 2 & 3
              \end{pmatrix}
              \sigma^2
              \begin{pmatrix}
                  1 & 0           & 0           \\
                  0 & \frac{1}{4} & 0           \\
                  0 & 0           & \frac{1}{9} \\
              \end{pmatrix}
              \begin{pmatrix}
                  2.5 \\
                  3.2 \\
                  3.6
              \end{pmatrix}
              = \sigma^2
              \begin{pmatrix}
                  1 & 1 & 1 \\
                  1 & 2 & 3
              \end{pmatrix}
              \begin{pmatrix}
                  2.5 \\
                  0.8 \\
                  0.4
              \end{pmatrix}
              = \sigma^2
              \begin{pmatrix}
                  3.7 \\
                  5.3
              \end{pmatrix} ,
          \]
          тогда нормальная система:
          \begin{gather*}
              \sigma^2
              \begin{pmatrix}
                  1.36 & 1.83 \\
                  1.83 & 3
              \end{pmatrix}
              \widehat{\theta}
              =
              \sigma^2
              \begin{pmatrix}
                  3.7 \\
                  5.3
              \end{pmatrix} .
          \end{gather*}
          Вычисляем обратную матрицу:
          \begin{gather*}
              \begin{vmatrix}
                  1.36 & 1.83 \\
                  1.83 & 3
              \end{vmatrix}
              = 1.36 \cdot 3 - 1.83 \cdot 1.83
              \approx 0.73, \\
              %
              \begin{pmatrix}
                  1.36 & 1.83 \\
                  1.83 & 3
              \end{pmatrix}^{-1}
              =
              \frac{1}{
                  \begin{vmatrix}
                      1.36 & 1.83 \\
                      1.83 & 3
                  \end{vmatrix}
              }
              \begin{pmatrix}
                  3     & -1.83 \\
                  -1.83 & 1.36
              \end{pmatrix}
              \approx
              \begin{pmatrix}
                  4.1  & -2.5 \\
                  -2.5 & 1.86
              \end{pmatrix} ,
          \end{gather*}

          находим решение системы:
          \[
              \widehat{\theta}
              =
              \begin{pmatrix}
                  4.1  & -2.5 \\
                  -2.5 & 1.86
              \end{pmatrix}
              \begin{pmatrix}
                  3.7 \\
                  5.3
              \end{pmatrix}
              %
              \approx
              \begin{pmatrix}
                  1.9 \\
                  0.6
              \end{pmatrix} .
          \]

    \item
          Оценка измерений $\eta$ --- проекция $\pr{\eta}{Z}$:
          \[
              \pr{\eta}{Z}
              = Z \widehat{\theta}
              =  \begin{pmatrix}
                  1 & 1 \\
                  1 & 2 \\
                  1 & 3
              \end{pmatrix}
              \begin{pmatrix}
                  1.9 \\
                  0.6
              \end{pmatrix}
              = \begin{pmatrix}
                  2.5 \\
                  3.1 \\
                  3.7
              \end{pmatrix} .
          \]
          Вычислим вектор перпендикуляра $\pro{\eta}{Z}$:
          \begin{gather*}
              \pro{\eta}{Z}
              = \eta - \pr{\eta}{Z}
              =
              \begin{pmatrix}
                  2.5 \\
                  3.2 \\
                  3.6
              \end{pmatrix}
              -
              \begin{pmatrix}
                  2.5 \\
                  3.1 \\
                  3.7
              \end{pmatrix}
              = \begin{pmatrix}
                  0   \\
                  0.1 \\
                  - 0.1
              \end{pmatrix}
          \end{gather*}

          Величина проекции $\pro{\eta}{Z}$ связана с остаточной дисперсией. С одной стороны квадрат $W$-нормы проекции
          $\pro{\eta}{Z}$:
          \begin{multline*}
              \norm{\pro{\eta}{Z}}_W^2
              = \norm{
                  \begin{pmatrix}
                      0   \\
                      0.1 \\
                      -0.1
                  \end{pmatrix}
              }_W^2
              =
              \begin{pmatrix}
                  0 & 0.1 & -0.1
              \end{pmatrix}
              \frac{1}{\sigma^2}
              \begin{pmatrix}
                  1 & 0           & 0           \\
                  0 & \frac{1}{4} & 0           \\
                  0 & 0           & \frac{1}{9} \\
              \end{pmatrix}
              \begin{pmatrix}
                  0   \\
                  0.1 \\
                  -0.1
              \end{pmatrix} = \\
              %
              = \frac{1 \cdot 0^2 + \frac{1}{4} \cdot 0.1^2 + \frac{1}{9} \cdot (-0.1)^2}{\sigma^2}
              = \frac{\frac{13}{36} \cdot 0.01}{\sigma^2}
              \approx \frac{0.36 \cdot 0.01}{\sigma^2},
          \end{multline*}
          а с другой стороны математическое ожидание
          \[
              \expectation{\norm{\pro{\eta}{Z}}_W^2} = n - m = 3 - 2 = 1
          \]
          тогда
          \begin{gather*}
              1 = \expectation{\norm{\pro{\eta}{Z}}_W^2} \approx \norm{\pr{\eta}{Z}}_W^2 = \frac{0.36 \cdot 0.01}{\sigma^2} , \\
              \sigma^2 \approx 0.36 \cdot 0.01 , \\
              \sigma \approx \sqrt{0.36 \cdot 0.01} = 0.6 \cdot 0.1 = 0.06
          \end{gather*}

    \item
          Для вычисление коэффициента детерминации вычислим регрессию с постоянной:
          \[
              \eta_i = \widetilde{c} + \varphi_i ,
          \]
          в которой матрица в правой части:
          \[
              U
              = \begin{pmatrix}
                  1 \\
                  1 \\
                  1
              \end{pmatrix}
          \]
          и оценкой $\widehat{c}$ постоянной $\widetilde{c}$ по методу наименьших квадратов является величина:
          \begin{multline*}
              \widehat{c}
              = (U^T W U)^{-1} U^T W \eta = \\
              %
              = \left(
              \begin{pmatrix}
                  1 & 1 & 1
              \end{pmatrix}
              \frac{1}{\sigma^2}
              \begin{pmatrix}
                  1 & 0           & 0           \\
                  0 & \frac{1}{4} & 0           \\
                  0 & 0           & \frac{1}{9} \\
              \end{pmatrix}
              \begin{pmatrix}
                  1 \\
                  1 \\
                  1
              \end{pmatrix}
              \right )^{-1}
              \begin{pmatrix}
                  1 & 1 & 1
              \end{pmatrix}
              \frac{1}{\sigma^2}
              \begin{pmatrix}
                  1 & 0           & 0           \\
                  0 & \frac{1}{4} & 0           \\
                  0 & 0           & \frac{1}{9} \\
              \end{pmatrix}
              \begin{pmatrix}
                  2.5 \\
                  3.2 \\
                  3.6
              \end{pmatrix} = \\
              %
              = \left( \frac{1}{\sigma^2} \left ( 1 + \frac{1}{4} + \frac{1}{9} \right) \right)^{-1} \frac{1}{\sigma^2} \left( 2.5 + 0.8 + 0.4 \right)
              = \sigma^2 \frac{36}{49} \frac{1}{\sigma^2} 3.7
              \approx \frac{36}{50} \cdot 3.7
              = 0.72 \cdot 3.7
              \approx 2.7
          \end{multline*}
          Вычислим проекции:
          \begin{gather*}
              \eta_U
              = U \widehat{c}
              = \begin{pmatrix}
                  1 \\
                  1 \\
                  1
              \end{pmatrix}
              2.7
              = \begin{pmatrix}
                  2.7 \\
                  2.7 \\
                  2.7
              \end{pmatrix} , \\
              %
              \eta_{U^\perp}
              = \eta - \eta_U
              = \begin{pmatrix}
                  2.5 \\
                  3.2 \\
                  3.6
              \end{pmatrix}
              - \begin{pmatrix}
                  2.7 \\
                  2.7 \\
                  2.7
              \end{pmatrix}
              = \begin{pmatrix}
                  -0.2 \\
                  0.5  \\
                  0.9
              \end{pmatrix}
          \end{gather*}
          и отклонение
          \begin{multline*}
              \norm{\pro{\eta}{U}}_W^2
              = \pro{\eta}{U}^T W \pro{\eta}{U}
              = \begin{pmatrix}
                  -0.2 & 0.5 & 0.9
              \end{pmatrix}
              \frac{1}{\sigma^2}
              \begin{pmatrix}
                  1 & 0           & 0           \\
                  0 & \frac{1}{4} & 0           \\
                  0 & 0           & \frac{1}{9} \\
              \end{pmatrix}
              \begin{pmatrix}
                  -0.2 \\
                  0.5  \\
                  0.9
              \end{pmatrix} = \\
              %
              = \frac{1 \cdot (-0.2)^2 + \frac{1}{4} \cdot 0.5^2 + \frac{1}{9} \cdot 0.9^2}{\sigma^2}
              = \frac{0.04 + \frac{1}{4} \cdot 0.25 + 0.1 \cdot 0.9}{\sigma^2} = \\
              %
              = \frac{0.04 + 0.0625 + 0.09}{\sigma^2}
              \approx \frac{0.2}{\sigma^2} ,
          \end{multline*}
          тогда коэффициент детерминации
          \[
              R^2
              = 1 - \frac{\norm{\pro{\eta}{Z}}_W^2}{\norm{\pro{\eta}{U}}_W^2}
              \approx 1 - \frac{0.0036}{0.2}
              = 1 - 0.018
              \approx 0.982 ,
          \]
          а скорректированный коэффициент детерминации
          \begin{multline*}
              R_{adj}^2
              = 1 - \frac{\frac{1}{n-m}\norm{\pro{\eta}{Z}}_W^2}{\frac{1}{n-1}\norm{\pro{\eta}{U}}_W^2}
              = 1 - \frac{n-1}{n-m} \frac{\norm{\pro{\eta}{Z}}_W^2}{\norm{\pro{\eta}{U}}_W^2}
              = 1 - \frac{3-1}{3-2} \frac{0.0036}{0.2} = \\
              %
              = 1 - 2 \cdot \frac{0.0036}{0.2}
              = 1 - 0.036
              = 0.964 .
          \end{multline*}

    \item
          Ранее была вычислена матрица Грамма --- матрица правой части нормальной системы:
          \[
              G
              = Z^T W Z
              \approx \sigma^2
              \begin{pmatrix}
                  1.36 & 1.83 \\
                  1.83 & 3
              \end{pmatrix}
          \]
          и обратная к ней матрица:
          \[
              G^{-1}
              = \left( Z^T W Z \right)^{-1}
              = \left(
              \frac{1}{\sigma^2}
              \begin{pmatrix}
                  1.36 & 1.83 \\
                  1.83 & 3
              \end{pmatrix}
              \right)^{-1}
              = \sigma^2
              \begin{pmatrix}
                  4.1  & -2.5 \\
                  -2.5 & 1.86
              \end{pmatrix}
          \]

          Границы доверительного интервала для величины $\theta_1$ определяются смещением:
          \[
              y \sqrt{\element{G^{-1}}{11} \frac{\norm{\pro{\eta}{Z}}_W^2}{n-m}}
              \approx y \sqrt{\sigma^2 4.1 \frac{0.0036}{\sigma^2} \frac{1}{3 - 2}}
              \approx 12.7 \cdot 2 \cdot 0.06
              \approx 1.5
          \]
          где $y$ --- распределения Стьюдента $T(n-m)$ уровня $\frac{1+P_g}{2}$, и доверительный интервал:
          \[
              \left( 1.9 - 1.5; 1.9 + 1.5 \right)
              = \left( 0.4; 3.4 \right) .
          \]
          Ноль не попадает в реализацию доверительного интервала, поэтому гипотеза $\widetilde{\theta_1} = 0$ отклоняется.

          Аналогично для величины $\theta_2$:
          \[
              y \sqrt{\element{G^{-1}}{22} \frac{\norm{\pro{\eta}{Z}}_W^2}{n-m}}
              = y \sqrt{\sigma^2 1.86 \frac{0.0036}{\sigma^2} \frac{1}{3 - 2}}
              = 12.7 \cdot 1.36 \sqrt{0.06}
              \approx 1
          \]
          и доверительный интервал:
          \[
              \left( 0.6 - 1; 0.6 + 1 \right)
              = \left( -0.4; 1.6 \right) .
          \]
          Ноль попадает в реализацию доверительного интервала, поэтому гипотеза $\widetilde{\theta_2} = 0$ принимается.

    \item Доверительный интервал для $\sigma$ получается из неравенства для квадрата нормы перпендикуляра:

          \begin{gather*}
              y_1 < \norm{\pro{\eta}{Z}}_W^2 < y_2 , \\
              y_1 < \frac{0.0036}{\sigma^2} < y_2 , \\
              \frac{1}{y_2} < \frac{\sigma^2}{0.0036} < \frac{1}{y_1} , \\
              \frac{0.0036}{y_2} < \sigma^2 < \frac{0.0036}{y_1} , \\
              \sqrt{\frac{0.0036}{y_2}} < \sigma < \sqrt{\frac{0.0036}{y_1}} ,
          \end{gather*}
          где $y_1$ и $y_2$ --- квантили распределения $\chi^2(n-m)$ уровней $\frac{1-P_g}{2}$ и $\frac{1+P_g}{2}$:
          \begin{gather*}
              \sqrt{\frac{0.0036}{3.84}} < \sigma < \sqrt{\frac{0.0036}{0.004}} , \\
              0.03 < \sigma < 0.94 .
          \end{gather*}

    \item Для проверки гипотезы об отсутствии зависимости необходимо вычислить статистику
          \[
              T = \frac{n-m}{m} \frac{\norm{\pr{\eta}{Z}}_W^2}{\norm{\pro{\eta}{Z}}_W^2} ,
          \]
          где
          \begin{multline*}
              \norm{\pr{\eta}{Z}}_W^2
              = \pr{\eta}{Z}^T W \pr{\eta}{Z}
              = \begin{pmatrix}
                  2.5 & 3.1 & 3.7
              \end{pmatrix}
              \frac{1}{\sigma^2}
              \begin{pmatrix}
                  1 & 0           & 0           \\
                  0 & \frac{1}{4} & 0           \\
                  0 & 0           & \frac{1}{9}
              \end{pmatrix}
              \begin{pmatrix}
                  2.5 \\
                  3.1 \\
                  3.7
              \end{pmatrix} = \\
              %
              = \frac{2.5^2 + \frac{1}{4} \cdot 3.1^2 + \frac{1}{9} 3.7^2}{\sigma^2}
              = \frac{2.5^2 + \frac{1}{4} \cdot 9.61 + \frac{1}{9} \cdot 13.69}{\sigma^2} = \\
              %
              = \frac{2.5^2 + \frac{1}{4} \cdot 9.61 + \frac{1}{9} \cdot 13.69}{\sigma^2}
              \approx \frac{10.2}{\sigma^2} ,
          \end{multline*}
          тогда
          \[
              T
              = \frac{3-2}{2} \frac{\frac{10.2}{\sigma^2}}{\frac{0.0036}{\sigma^2}}
              = \frac{10.2}{0.0072}
              \approx 1417,
          \]
          Наименьший уровень значимости отклонения гипотезы о независимости:
          \[
              \alpha^*
              = 1 - F(T)
              \approx 0.02 ,
          \]
          где $F(x)$ --- функция распределения для распределения Фишера $F(m,n-m)$.
\end{enumerate}

\section*{Методы статистических испытаний}

\subsection*{Задача 1}

Случайные величины $\xi \sim \mathcal{R}[-1, 2]$ и $\varphi \sim E(3)$ являются независимыми, случайная величина $\eta = \xi \sin \varepsilon$.
Оцените $\expectation{e^\eta}$ с отклонением менее $\delta = 0.01$ и вероятностью более $P_\delta = 0.98$.

Решение:

Необходимо сформировать выборку, состоящую из пар $(\xi_i, \varphi_i)$, где величины $\xi_i \sim \mathcal{R}[-1, 2]$,
$\varphi_i \sim E(3)$ и являются независимыми, вычислить величины $\eta_i = \xi_i \sin \varphi_i$, затем $\varepsilon_i = e^{\eta_i}$ и оценку:
\[
    \expectation{e^\eta} \approx \frac{1}{n} \sum_{i=1}^n \varepsilon_i.
\]

Требуемое количество величин с помощью центральной предельной теоремы:
\[
    n
    \ge \overline{n}
    = \left( \Phi^{-1} \left( \frac{1 + P_\delta}{2} \right) \right)^2 \frac{\overline{D}}{\delta^2},
\]
где
\[
    \forall i: \variance{\varepsilon_i} \le \overline{D} .
\]

Величина $\overline{D}$ ограничивает дисперсию величин $\varepsilon_i$:
\begin{gather*}
    -1 \le \xi_i \le 2, -1 \le \sin \varepsilon_i \le 1, \\
    -2 \le \xi_i \sin \varepsilon_i \le 2, \\
    -2 \le \eta_i \le 2, \\
    e^{-2} \le e^{\eta_i} \le e^2, \\
    e^{-2} \le \varepsilon_i \le e^2, \\
    \variance{\varepsilon_i} \le \left( \frac{e^2 - e^{-2}}{2} \right)^2 = \overline{D} ,
\end{gather*}
тогда
\begin{multline*}
    \overline{n}
    = \left( \Phi^{-1} \left( \frac{1 + 0.98}{2} \right) \right)^2 \frac{\left( \frac{e^2 - e^{-2}}{2} \right)^2}{0.01^2}
    = \left( \Phi^{-1} ( 0.99 ) \left( \frac{e^2 - e^{-2}}{2} \right) \right)^2 \cdot 10^4 = \\
    %
    = \left( 2.326 \cdot 3.6268 \right)^2 \cdot 10^4
    = 71.1651 \cdot 10^4
    = 711 651 .
\end{multline*}

\subsection*{Задача 2}

Оценить интеграл
\[
    J = \int \limits_0^{\frac{\pi}{2}} \sqrt{4 - \sin^2 \varphi} d \varphi
\]
с отклонением менее $\delta = 10^{-3}$ и вероятностью более $P_\delta = 0.95$.

Решение:

Постоянные, ограничивающие функцию при $\varphi \in \left[ 0, \frac{\pi}{2} \right]$:
\begin{gather*}
    0 \le \sin \varphi \le 1 , \\
    0 \le \sin^2 \varphi \le 1 , \\
    \sqrt{3} \le \sqrt{4 - \sin^2 \varphi} \le \sqrt{4} = 2 .
\end{gather*}

Преобразуем интеграл:

\begin{multline*}
    J
    = \int \limits_0^{\frac{\pi}{2}} \sqrt{4 - \sin^2 \varphi} d \varphi
    = \left( 2 - \sqrt{3} \right) \int \limits_0^{\frac{\pi}{2}} \frac{\sqrt{4 - \sin^2 \varphi}}{2 - \sqrt{3}} d \varphi = \\
    %
    = \left( 2 - \sqrt{3} \right) \int \limits_0^{\frac{\pi}{2}} \frac{\sqrt{4 - \sin^2 \varphi} - \sqrt{3}}{2 - \sqrt{3}} d \varphi + \left( 2 - \sqrt{3} \right) \sqrt{3} = \\
    %
    = \left( 2 - \sqrt{3} \right) \frac{\pi}{2} \int \limits_0^1 \frac{\sqrt{4 - \sin^2 \left( \frac{\pi}{2} x \right)} - \sqrt{3}}{2 - \sqrt{3}} d x + \left( 2 - \sqrt{3} \right) \sqrt{3} = \\
    %
    = \left( 2 - \sqrt{3} \right) \frac{\pi}{2} \widetilde{J} + \left( 2 - \sqrt{3} \right) \sqrt{3} ,
\end{multline*}
где
\begin{gather*}
    \widetilde{J} = \int \limits_0^1 \widetilde{f}(x) d x , \\
    \widetilde{f}(x) = \frac{\sqrt{4 - \sin^2 \left( \frac{\pi}{2} x \right)} - \sqrt{3}}{2 - \sqrt{3}}
\end{gather*}
и
\[
    x \in [0, 1]: 0 \le \widetilde{f}(x) \le 1 .
\]

Необходимо сформировать выборку, состоящую из пар $(\xi_i, \eta_i)$, в которых $\xi_i \sim \mathcal{R}[0, 1]$, $\eta_i \sim \mathcal{R}[0, 1]$ и
$\xi_i$ и $\eta_i$ независимы, вычислить величины $\varepsilon_i$:
\[
    \varepsilon_i
    = \left \{
    \begin{array}{ll}
        1, & \eta_i < \widetilde{f}(\xi_i) ,   \\
        0, & \eta_i \ge \widetilde{f}(\xi_i) .
    \end{array}
    \right .
\]
оценить интегралы:
\begin{gather*}
    \widetilde{J} \approx \widetilde{J}_\varepsilon = \frac{1}{n} \sum_{i=1}^n \varepsilon , \\
    J \approx J_\varepsilon = \left( 2 - \sqrt{3} \right) \frac{\pi}{2} \widetilde{J}_\varepsilon + \left( 2 - \sqrt{3} \right) \sqrt{3} .
\end{gather*}

Требуемый объём выборки из центральной предельной теоремы:
\begin{multline*}
    n
    \ge \overline{n}
    = \left( \Phi^{-1} \left( \frac{1 + P_\delta}{2} \right) \right)^2 \frac{\frac{1}{4}}{\left( \frac{\delta}{\left( 2 - \sqrt{3} \right) \frac{\pi}{2}} \right)^2}
    = \left( \Phi^{-1} \left( \frac{1 + 0.95}{2} \right) \right)^2 \frac{\frac{1}{4}}{\left( \frac{10^{-3}}{\left( 2 - \sqrt{3} \right) \frac{\pi}{2}} \right)^2} = \\
    %
    = \left( \Phi^{-1} ( 0.975 ) \frac{1}{2} \left( 2 - \sqrt{3} \right) \frac{\pi}{2} \right)^2 10^6
    = \left( 1.96 \cdot 0.5 \cdot 0.26795 \cdot 1.5708 \right)^2 10^6 = \\
    %
    = \left( 0.412477 \right)^2 10^6
    = 0.170139 \cdot 10^6
    = 170 139 .
\end{multline*}

\section*{Эффективные оценки}

\subsection*{Задача 1}

$\xi = (\xi_1, \dots, \xi_n)$ --- выборка из равномерного распределения $\mathcal{R}[0, \theta]$, где $\theta \in (0, + \infty)$ --- неизвестный параметр.
Постройте оценку $\theta$.

Решение:

\begin{enumerate}
    \item Плотность вероятности одной величины выборки $\xi_i$:
          \[
              p_i(x; \theta)
              = \left \{
              \begin{array}{ll}
                  \frac{1}{\theta}, & 0 \le x \le \theta , \\
                  0,                & \text{иначе}
              \end{array}
              \right .
              = \frac{1}{\theta} \cdot I(0 \le x \le \theta) ,
          \]
          где $I$ --- индикаторная функция.
          Совместная плотность вероятности всех величин выборки:
          \begin{multline*}
              p_\xi(x_1, \dots, x_n; \theta)
              = \prod_{i=1}^n p_i(x_i; \theta)
              = \prod_{i=1}^n \frac{1}{\theta} \cdot I(0 \le x_i \le \theta)
              = \frac{1}{\theta^n} \prod_{i=1}^n I(0 \le x_i \le \theta) = \\
              %
              = \frac{1}{\theta^n} \cdot I(0 \le \min \{ x_1, \dots, x_n \}) \cdot I(\max \{ x_1, \dots, x_n \} \le \theta) .
          \end{multline*}

          Функция правдоподобия:
          \[
              L(\xi_1, \dots, \xi_n; \theta)
              = \frac{1}{\theta^n} \cdot I(0 \le \min \{ \xi_1, \dots, \xi_n \}) \cdot I(\max \{ \xi_1, \dots, \xi_n \} \le \theta)
              = \frac{1}{\theta^n} \cdot I(\xi_{(n)} \le \theta)
          \]
          По критерию факторизации порядковая статистика $\xi_{(n)}$ является достаточной статистикой.

    \item Рассмотрим произвольную функцию $g(t)$, для которой:
          \begin{gather*}
              \expectation{g(\xi_{(n)})} = 0 , \\
              \int \limits_{-\infty}^{\infty} g(t) p_{(n)}(x; \theta) dx = 0 ,
          \end{gather*}
          где $p_{(n)}(x; \theta)$ --- плотность вероятности порядковой статистики $\xi_{(n)}$. Известно, что функция распределения
          $F_{(n)}(x; \theta)$ порядковой статистики $\xi_{(n)}$ имеет вид:
          \begin{multline*}
              F_{(n)}(x; \theta)
              = \probability{\xi_{(n)} < x}
              = \probability{\max \{ \xi_1, \dots, \xi_n \} < x}
              = \probability{\xi_1 < x, \dots, \xi_n < x} = \\
              %
              = \prod_{i=1}^n \probability{x_i < x}
              = \prod_{i=1}^n F_i(x; \theta) ,
          \end{multline*}
          где
          \[
              F_i(x; \theta)
              = \left \{
              \begin{array}{ll}
                  0,                & x < 0,              \\
                  \frac{x}{\theta}, & 0 \le x \le \theta, \\
                  1,                & 1 < x .
              \end{array}
              \right .
          \]
          --- функция распределения величин $\xi_i$ выборки, тогда:
          \[
              F_{(n)}(x; \theta)
              = \left \{
              \begin{array}{ll}
                  0,                                 & x < 0,              \\
                  \left( \frac{x}{\theta} \right)^n, & 0 \le x \le \theta, \\
                  1,                                 & 1 < x .
              \end{array}
              \right .
          \]
          Дифференцируя $F_{(n)}(x; \theta)$ получим плотность $p_{(n)}(x; \theta)$:
          \[
              p_{(n)}(x; \theta)
              = \left \{
              \begin{array}{ll}
                  0,                          & x < 0,              \\
                  \frac{n x^{n-1}}{\theta^n}, & 0 \le x \le \theta, \\
                  0,                          & 1 < x .
              \end{array}
              \right .
          \]
          Подставляя плотность $p_{(n)}(x; \theta)$ в интеграл математического ожидания, получим:
          \[
              \int \limits_0^\theta g(t) \frac{n x^{n-1}}{\theta^n} dx = 0 .
          \]
          Дифференцируя по $\theta$, получим:
          \begin{gather*}
              g(\theta) \frac{n \theta^{n-1}}{\theta^n} = 0 , \\
              g(\theta)  = 0 .
          \end{gather*}
          Таким образом, для функции $g(t)$:
          \[
              \forall \theta \in \Theta: \expectation{g(\xi_{(n)})} = 0 \Rightarrow g(t) = 0.
          \]
          поэтому статистика $\xi_{(n)}$ является полной.

    \item Известно, что эффективная оценка является условным математическим ожиданием относительно достаточной и полной статистики $\xi_{(n)}$,
          поэтому найти оценку можно двумя способами.
          \begin{enumerate}
              \item Условное математическое ожидание относительно $\xi_{(n)}$ является некоторой функцией $\varphi(y)$ величины $\xi_{(n)}$.
                    Осталось подобрать функцию $\varphi(y)$ так, чтобы оценка $\varphi(\xi_{(n)}$ была несмещённой:
                    \begin{gather*}
                        \expectation{\varphi(\xi_{(n)})} = \theta , \\
                        \int \limits_0^{\theta} \varphi(y) p_{(n)}(y; \theta) dy = \theta , \\
                        \int \limits_0^{\theta} \varphi(y) \frac{n y^{n-1}}{\theta^n} dy = \theta , \\
                        \int \limits_0^{\theta} \varphi(y) y^{n-1} dy = \frac{\theta^{n+1}}{n} .
                    \end{gather*}
                    Дифференцируя левую и правую части по параметру, получим:
                    \begin{gather*}
                        \varphi(\theta) \theta^{n-1} = \frac{n+1}{n} \theta^n , \\
                        \varphi(\theta) = \frac{n+1}{n} \theta .
                    \end{gather*}
                    Значит функция $\varphi(y)$ имеет вид:
                    \[
                        \varphi(y) = \frac{n+1}{n} y
                    \]
                    и оценка
                    \[
                        \widehat{\theta} = \frac{n+1}{n} \xi_{(n)} .
                    \]
              \item \textcolor{red}{Не получилось.}

                    Можно взять любую несмещённую оценку $\theta$, например, $2 \xi_1$:
                    \[
                        \expectation{2 \xi_1} = 2 \frac{\theta}{2} = \theta,
                    \]
                    тогда эффективная оценка:
                    \[
                        \widehat{\theta} = \cexpectation{2 \xi_1}{\xi_{(n)}}
                    \]
                    Вычислить условное математическое ожидание можно с помощью условного распределения $\xi_1$ относительно $\xi_{(n)}$.
                    Совместное распределение величин $\xi_1$ и $\xi_{(n)}$ описывается функцией распределения:
                    \begin{multline*}
                        F_{1,n}(x,y; \theta)
                        = \probability{\xi_1 < x, \xi_{(n)} < y}
                        = \probability{\xi_1 < x, \max \{ \xi_1, \dots, \xi_n \} < y} = \\
                        %
                        = \probability{\xi_1 < x, \xi_1 < y, \xi_2 < y, \dots, \xi_n < y}
                        = \probability{\xi_1 < \min \{ x, y \}, \xi_2 < y, \dots, \xi_n < y} = \\
                        %
                        = F_1(\min \{ x, y \}; \theta) \cdot \prod_{i=2}^n F_i(y; \theta) = \\
                        %
                        = \left \{
                        \begin{array}{ll}
                            0,                              & \min \{ x, y \} < 0,              \\
                            \frac{\min \{ x, y \}}{\theta}, & 0 \le \min \{ x, y \} \le \theta, \\
                            1,                              & 1 < \min \{ x, y \} .
                        \end{array}
                        \right .
                        \cdot
                        \left \{
                        \begin{array}{ll}
                            0,                                     & y < 0,              \\
                            \left( \frac{y}{\theta} \right)^{n-1}, & 0 \le y \le \theta, \\
                            1,                                     & 1 < y .
                        \end{array}
                        \right .
                    \end{multline*}
                    Выделим квадрат $\left \{ (x,y): 0 \le x \le \theta, 0 \le y \le \theta \right \}$. За пределами квадрата совместная
                    плотность вероятности $\xi_1$, $\xi_{(n)}$ равна нулю, внутри квадрата выделим два треугольника, в которых
                    \begin{enumerate}
                        \item $x \le y$:
                              \[
                                  F_{1,n}(x,y; \theta)
                                  = \frac{x}{\theta} \cdot \left( \frac{y}{\theta} \right)^{n-1},
                              \]
                        \item $x > y$:
                              \[
                                  F_{1,n}(x,y; \theta)
                                  = \frac{y}{\theta} \cdot \left( \frac{y}{\theta} \right)^{n-1}
                                  = \left( \frac{y}{\theta} \right)^n .
                              \]
                    \end{enumerate}
                    Нужно найти совместную плотность вероятности $p_{1,n}(x,y; \theta)$ и условную плотность вероятности $p_{1|n}(x, y; \theta)$
                    случайной величины $\xi_1$ относительно $\xi_{(n)}$:
                    \[
                        p_{1|n}(x, y; \theta)
                        = \frac{p_{1,n}(x,y; \theta)}{p_{(n)}(y; \theta)} ,
                    \]
                    затем найти условное математическое ожидание:
                    \[
                        \cexpectation{\xi_1}{\xi_{(n)}}
                        = \int \limits_{-\infty}^{\infty} x p_{1|n}(x, \xi_{(n)}; \theta) dx .
                    \]
          \end{enumerate}
\end{enumerate}
\end{document}
