\documentclass[a4paper,12pt]{article}
\usepackage[T1]{fontenc}
\usepackage[utf8]{inputenc}
\usepackage[english,russian]{babel}
\usepackage[left=1cm,right=1cm,top=1cm,bottom=2cm]{geometry}
\usepackage{enumitem}
\usepackage{ulem}
\usepackage{amsmath}

\newif\iftutor
\tutortrue

\newcommand{\probability}[1]{P \left\{ #1 \right\}}
\newcommand{\variance}[1]{\mathtt{D} \left[ #1 \right]}
\newcommand{\modulus}[1]{\left | #1 \right |}
\newcommand{\sample}[2]{\left( #1_1, \dots, #1_{#2} \right)}

\newcommand{\theme}[1]{\hfil \textbf{#1} \hfil}

\newcounter{enumt}
\setcounter{enumt}{0}
\newenvironment{task}
{
    \noindent\begin{minipage}[t][6cm]{17.1cm}
        \begin{center}
            \textbf{Билет \refstepcounter{enumt}\arabic{enumt}}
        \end{center}
}
{\end{minipage}}

\begin{document}

\setlist{nosep}
\setlist[enumerate,1]{label=\arabic{enumi}.,ref=\arabic*}
\setlist[enumerate,2]{label*=\arabic{enumii}.,ref=\arabic{enumi}.\arabic{enumii}}

\title{Экзамен \\ по математической статистике}
\author{Тигетов Давид Георгиевич}
\maketitle

\section{Расписание}

\subsection{А-05-21}

\begin{center}
    \begin{tabular}{|c|p{10cm}|}
        \hline
        Время                  & Действия                                                                                                             \\
        \hline
        9:05 -- 9:15           & Подготовка, свободный проход в аудиторию.                                                                            \\
        \hline
        9:15 -- 9:20           & Проверка листов, выдача билетов.                                                                                     \\
        \hline
        \textbf{9:20 -- 10:20} & Учащиеся пишут экзамен.                                                                                              \\
        \hline
        10:20 -- 10:25         & Учащиеся сдают работы и покидают аудиторию.                                                                          \\
        \hline
        10:25 -- 11:25         & Преподаватели проверяют работы, приглашают учащихся для ответов на дополнительные вопросы, выставляют оценки в БАРС. \\
        \hline
    \end{tabular}
\end{center}


\subsection{А-13-21}

\begin{center}
    \begin{tabular}{|c|p{10cm}|}
        \hline
        Время                   & Действия                                                                                                             \\
        \hline
        10:50 -- 11:00          & Подготовка, свободный проход в аудиторию.                                                                            \\
        \hline
        11:00 -- 11:05          & Проверка листов, выдача билетов.                                                                                     \\
        \hline
        \textbf{11:05 -- 12:05} & Учащиеся пишут экзамен.                                                                                              \\
        \hline
        12:05 -- 12:10          & Учащиеся сдают работы и покидают аудиторию.                                                                          \\
        \hline
        12:10 -- 13:10          & Преподаватели проверяют работы, приглашают учащихся для ответов на дополнительные вопросы, выставляют оценки в БАРС. \\
        \hline
    \end{tabular}
\end{center}

\subsection{А-18-21}

\begin{center}
    \begin{tabular}{|c|p{10cm}|}
        \hline
        Время                   & Действия                                                                                                             \\
        \hline
        13:30 -- 13:40          & Подготовка, свободный проход в аудиторию.                                                                            \\
        \hline
        13:40 -- 13:45          & Проверка листов, выдача билетов.                                                                                     \\
        \hline
        \textbf{13:45 -- 14:45} & Учащиеся пишут экзамен.                                                                                              \\
        \hline
        14:45 -- 14:50          & Учащиеся сдают работы и покидают аудиторию.                                                                          \\
        \hline
        14:50 -- 15:50          & Преподаватели проверяют работы, приглашают учащихся для ответов на дополнительные вопросы, выставляют оценки в БАРС. \\
        \hline
    \end{tabular}
\end{center}

\subsection{А-14,16-21}

\begin{center}
    \begin{tabular}{|c|p{10cm}|}
        \hline
        Время                   & Действия                                                                                                             \\
        \hline
        13:30 -- 13:40          & Подготовка, свободный проход в аудиторию.                                                                            \\
        \hline
        13:40 -- 13:45          & Проверка листов, выдача билетов.                                                                                     \\
        \hline
        \textbf{13:45 -- 14:45} & Учащиеся пишут экзамен.                                                                                              \\
        \hline
        14:45 -- 14:50          & Учащиеся сдают работы и покидают аудиторию.                                                                          \\
        \hline
        14:50 -- 15:50          & Преподаватели проверяют работы, приглашают учащихся для ответов на дополнительные вопросы, выставляют оценки в БАРС. \\
        \hline
    \end{tabular}
\end{center}

\section{Билеты}

\subsection{Состав}

В каждом билете два основных вопроса и одна задача.

\begin{center}
    \begin{tabular}{|c|p{2cm}|c|l|p{9cm}|}
        \hline
        № & Задание                           & Баллы    & Время    & Критерий \\
        \hline
        1 & Основной вопрос с доказательством & 0 -- 2   & 25 минут &
        \begin{itemize}
            \item 0 -- нет ответа или ответ не соответствует вопросу,
            \item 0.5 -- определения и формулировки без доказательств,
            \item 1 -- доказательства со значительными ошибками,
            \item 1.5 -- ответ неполный или доказательства с незначительными ошибками,
            \item 2 -- ответ полный, без ошибок.
        \end{itemize}
        \\
        \hline
        2 & Основной вопрос                   & 0 -- 1.5 & 20 минут &
        \begin{itemize}
            \item 0 -- нет ответа или ответ не соответствует вопросу,
            \item 0.5 -- ответ неполный или содержит значительные ошибки,
            \item 1 -- ответ неполный или содержит незначительные ошибки,
            \item 1.5 -- ответ полный, без ошибок.
        \end{itemize}
        \\
        \hline
        3 & Задача                            & 0 -- 1.5 & 15 минут &
        \begin{itemize}
            \item 0 -- нет решения,
            \item 0.5 -- решение неполное или содержит значительные ошибки (неверный метод),
            \item 1 -- решение неполное или содержит незначительные ошибки (арифметические),
            \item 1.5 -- решение полное, не содержит ошибок.
        \end{itemize}
        \\
        \hline
    \end{tabular}
\end{center}

При сумме баллов за ответ 2.5, 3.5 и 4.5 или по предложению преподавателя могут задаваться дополнительные вопросы: при правильных ответах сумма баллов будет
увеличена, при неправильных ответах --- уменьшена.

\pagebreak

\subsection{Основные вопросы}

\theme{Определения.}
\begin{enumerate}
    \item \label{def} Основные определения математической статистики: выборка, объем выборки, реализация выборки, вариационный ряд, порядковые статистики, эмпирическая функция распределения и её свойства (теоремы о сходимости).
\end{enumerate}

\theme{Точечное оценивание.}
\begin{enumerate}[resume]
    \item \label{est:pro} Задача точечного оценивания неизвестных величин: параметров, вероятностей и моментов. Статистики и оценки, свойства оценок: несмещенность и состоятельность. Сравнение несмещенных оценок на основе дисперсий. Оптимальная оценка. Обобщение критерия сравнения оценок на основе дисперсий с использованием среднеквадратического отклонения, функции потерь и функции риска.
    \item \label{est:lt} Состоятельность оценок и предельные теоремы: теорема Бернулли (без доказательства), теорема Хинчина (без доказательства), неравенство Маркова, неравенство Чебышева, закон больших чисел в форме Чебышёва и в форме Маркова. Утверждение о состоятельности оценки с убывающей дисперсией.
    \item \label{est:pt-cdf} Задача точечного оценивания вероятности события, построение оценки и свойства оценки. Задача точечного оценивания значений функции распределения, построение оценки и свойства оценки.
    \item \label{est:ev} Задача точечного оценивания математического ожидания и дисперсии. Выборочное среднее, выборочная дисперсия и исправленная выборочная дисперсии. Несмещенность и состоятельность выборочного среднего, выборочной дисперсии и исправленной выборочной дисперсии (без вывода формулы дисперсии выборочной дисперсии). Точечное оценивание старших моментов: выборочные моменты и их свойства несмещенности и состоятельности.
    \item \label{est:hom} \sout{Точечное оценивание старших моментов: выборочные моменты, их свойства несмещенности и состоятельности.}
    \item \label{est:lin} Постановка задачи построения точечной линейной оценки среднего при разноточных измерениях, метод построения линейной оценки с минимальной дисперсией и свойства коэффициентов.
\end{enumerate}

\theme{Методы точечного оценивания.}
\begin{enumerate}[resume]
    \item \label{estm:mom} Метод моментов построения точечных оценок, свойства моментных оценок.
    \item \label{estm:ml} Метод максимального правдоподобия построения точечных оценок. Свойства МП-оценок: состоятельность и асимптотическая нормальность.
    \item \label{estm:vs} Метод порядковых статистик: построение оценок и оценка квантилей. Функции распределения порядковых статистик, функции плотности вероятности порядковых статистик (без доказательства). Теорема Крамера об асимптотической нормальности порядковых статистик и свойства оценок по методу порядковых статистик.
\end{enumerate}


\theme{Интервальное оценивание.}
\begin{enumerate}[resume]
    \item \label{int:cs} Доверительный интервал, верхняя и нижняя доверительные границы. Центральная статистика и общий метод построения доверительных интервалов с помощью центральной статистики. Метод построения центральной статистики.
    \item \label{int:exp-kn} Построение наикратчайшего доверительного интервала для математического ожидания по выборке из нормального распределения с известной дисперсией.
    \item \label{int:var-kn} Распределение хи-квадрат и построение доверительных интервалов для дисперсии и среднеквадратического отклонения по выборке из нормального распределения с известным математическим ожиданием.
    \item \label{int:var-unkn} Теорема Фишера о выборочном среднем и исправленной выборочной дисперсии. Построение доверительных интервалов для дисперсии и среднеквадратического отклонения по выборке из нормального распределения с неизвестным математическим ожиданием.
    \item \label{int:exp-unkn} Теорема Фишера о выборочном среднем и исправленной выборочной дисперсии. Построение доверительного интервала для математического ожидания по выборке из нормального распределения с неизвестной дисперсией.
    \item \label{int:asy} Построение доверительных интервалов с использованием асимптотической нормальности. Построение доверительного интервала для вероятности события, решение неравенства: точное, приближенное с заменой вероятности оценкой и приближенное с оценкой дисперсии сверху.
    \item \label{int:cor} Построение доверительного интервала для коэффициента корреляции двумерного нормального распределения с неизвестными математическими ожиданиями и дисперсиями.
\end{enumerate}

\theme{Проверка гипотез.}
\begin{enumerate}[resume]
    \item \label{hyp:bas} Основные определения в задачах проверки статистических гипотез: статистическая гипотеза (простая и сложная), основная и альтернативная гипотезы (альтернативные распределения), статистический критерий и статистика критерия, критическая область и общий принцип проверки гипотез, вероятности ошибок первого и второго родов, функция мощности критерия (функции мощности как характеристика критерия и вид функции мощности «хорошего» критерия), свойства несмещенности и состоятельности критерия.
    \item \label{hyp:chi-pr} Постановка задачи проверки простой гипотезы о вероятностях и критерий хи-квадрат. Утверждение о неограниченности по вероятности статистики критерия хи-квадрат, если верна альтернативная гипотеза (без доказательства). Теорема Пирсона об асимптотическом распределение статистики критерия хи-квадрат, если основная гипотеза верна (без доказательства). Состоятельность критерия хи-квадрат (без доказательства). Нецентральное распределение хи-квадрат и асимптотическое распределение статистики критерия хи-квадрат (без доказательства). Условие применимости критерия хи-квадрат на практике.
    \item \label{hyp:chi-smp} Постановка задачи проверки простой гипотезы о вероятностях. Применение критерия хи-квадрат к задаче проверке гипотезы о распределении полностью известном.
    \item \label{hyp:chi-pr-par} Постановка задачи проверки сложной гипотезы о вероятностях и критерий хи-квадрат. Теорема Фишера об асимптотическом распределении минимальной по параметру статистики, если основная гипотеза верна (без доказательства). Теорема Фишера об асимптотическом распределении статистики с МП-оценкой параметра, если основная гипотеза верна (без доказательства). Применение критерия хи-квадрат к задаче проверки гипотезы о распределении с неизвестным параметром.
    \item \label{hyp:chi-ind} Постановка задачи проверки гипотезы о независимости признаков и применение критерия хи-квадрат.
    \item \label{hyp:con} Постановка задачи проверки гипотезы об однородности и критерий проверки: статистика критерия и критическая область.
    \item \label{hyp:K} Критерий согласия Колмогорова: постановка задачи, основная и альтернативная гипотезы, статистика критерия, значения статистики критерия, если верна альтернативная гипотеза, распределение статистики критерия точное и приближённое, если верна основная гипотеза, теорема Колмогорова (без доказательства), критическая область, выбор критической области по уровню значимости.
    \item \label{hyp:KS} Критерий согласия Колмогорова--Смирнова: постановка задачи, основная и альтернативная гипотезы, статистика критерия, значения статистики критерия, если верна альтернативная гипотеза, распределение статистики, если верна основная гипотеза, теорема Смирнова (без доказательства), критическая область, выбор критической области по уровню значимости.
    \item \label{hyp:F} Критерий Фишера: постановка задачи, основная и альтернативная гипотезы, статистика критерия, значения статистики критерия, если верна альтернативная гипотеза, распределение статистики, если верна основная гипотеза, выбор критической области по уровню значимости.
    \item \label{hyp:S} Критерий Стьюдента: постановка задачи, основная и альтернативная гипотезы, статистика критерия, значения статистики критерия, если верна альтернативная гипотеза, распределение статистики, если верна основная гипотеза, выбор критической области по уровню значимости.
    \item \label{hyp:fa} Однофакторный дисперсионный анализ: постановка задачи, основное дисперсионное соотношение, межгрупповая и внутригрупповая дисперсии, распределение внутригрупповой дисперсии, значения межгрупповой дисперсии, если верна альтернативная гипотеза, распределение межгрупповой дисперсии, если основная гипотеза верна, статистика критерия, выбор критической области по уровню значимости.
    \item \label{hyp:par} Задача проверки параметрических гипотез, статистический критерий и функции вероятностей ошибок первого и второго рода. Функция мощности, равномерно наиболее мощный критерий.
    \item \label{hyp:dis-fn-pr} Постановка задачи различения двух простых гипотез, вероятности ошибок первого и второго рода. Критерий отношения вероятностей, свойства вероятностей ошибок критериев отношения вероятностей, утверждение о линейных комбинациях вероятностей ошибок критериев отношения вероятностей.
    \item \label{hyp:dis-fn-cr} Постановка задачи различения двух простых гипотез, вероятности ошибок первого и второго рода, функция мощности, наиболее мощный критерий, минимаксный критерий и байесовский критерий, критерии отношения вероятностей, теорема о построении минимаксного, байесовского и наиболее мощного критерия как соответствующих критериев отношения вероятностей.
    \item \label{hyp:dis-inf} Постановка задачи различения двух простых гипотез, последовательные критериии, применение последовательных критериев в задаче различения двух простых гипотез, вероятности ошибок первого и второго родов и случайная величина количества шагов до остановки.
    \item \label{hyp:dis-pr} Постановка задачи различения двух простых гипотез, последовательный критерий отношения вероятностей. Утверждение о границах и вероятностях ошибок последовательного критерия отношения вероятностей. «Приближенный» последовательный критерий отношения вероятностей для заданных вероятностей ошибок, утверждение о вероятностях ошибок «приближенного» последовательного критерия отношения вероятностей.
    \item \label{hyp:dis-cnt} Постановка задачи различения двух простых гипотез, последовательный критерий отношения вероятностей. Утверждение о вероятности остановки последовательного критерия отношения вероятностей. Тождество Вальда и приближенный метод расчета математических ожиданий случайной величины количества шагов до остановки «приближенного» последовательного критерия отношения вероятностей.
\end{enumerate}

\theme{Регрессия.}
\begin{enumerate}[resume]
    \item \label{reg:tp} Теоретическая и практическая задачи регрессии.
    \item \label{reg:lin:est} Постановка задачи линейной регрессии. Решение задачи оптимизации на линейном подпространстве, нормальная система уравнений, свойства матрицы Грамма, оценка по методу наименьших квадратов.
    \item \label{reg:lin:pro} Постановка задачи линейной регрессии. Свойства оценки по методу наименьших квадратов.
    \item \label{reg:lin:dis} Постановка задачи линейной регрессии. Операторы проецирования. Вектор возмущения и его проекции, квадраты норм вектора возмущения и проекций, их выражения и математические ожидания.
    \item \label{reg:lin:det} Постановка задачи линейной регрессии. Коэффициент детерминации и скорректированный коэффициент детерминации.
    \item \label{reg:nor:dist} Постановка задачи нормальной линейной регрессии. Распределения оценки по методу наименьших квадратов и квадратов норм вектора возмущений и его проекций. Независимость оценки и проекций вектора возмущений.
    \item \label{reg:nor:conf} Постановка задачи нормальной линейной регрессии. Доверительные интервалы для неизвестных компонент параметра, доверительная область для параметра, проверка гипотезы об отсутствии зависимости (о нулевом параметре).
    \item \label{reg:nor:var} Постановка задачи нормальной линейной регрессии с остаточной дисперсией. Точечная оценка и доверительный интервал для остаточной дисперсии.
\end{enumerate}

\theme{Методы статистических испытаний.}
\begin{enumerate}[resume]
    \item \label{mc:cdf} Принцип методов статистических испытаний. Оценка значений функций распределения.
    \item \label{mc:mom} Принцип методов статистических испытаний. Оценка моментов случайных величин.
    \item \label{mc:int} Принцип методов статистических испытаний. Оценка интегралов.
\end{enumerate}

\theme{Эффективные оценки.}
\begin{enumerate}[resume]
    \item \label{eff:comp} Сравнение оценок на основе среднеквадратического отклонения. Смещение оценок и классы оценок по смещению. Эффективная оценки в заданном классе оценок. Существование оценки с заданным смещением и несмещенной оценки. Утверждение о единственности оценки эффективной в заданном классе.
    \item \label{eff:fact} Достаточные статистики. Замечание о сохранении количества информации Фишера. Теорема Неймана-Фишера (критерий факторизации). Следствие об оценке максимального правдоподобия. Следствие об R-эффективной оценке. Следствие о подчинённости достаточных статистик.
    \item \label{eff:BKR} Достаточные статистики. Теорема Блекуэлла, Колмогорова, Рао. Последовательное «улучшение» оценок.
    \item \label{eff:est} Достаточные статистики, полные статистики. Утверждение об оценках, являющихся функциями полной статистики. Утверждение об эффективной оценке. План построения эффективных оценок.
    \item \label{eff:CR} Функция правдоподобия, функция вклада и информации Фишера. Условия регулярности. Теорема о неравенстве Крамера--Рао. R-эффективная оценка, замечание об R-эффективной и эффективной оценке в классе.
    \item \label{eff:exp} Условия регулярности. R-эффективные оценки. Теорема об экспоненциальных семействах распределений и R-эффективных оценках.
    \item \label{eff:F} Функция правдоподобия, функция вклада, информация Фишера. Вычисление информации Фишера с помощью второй производной. Аддитивность информации Фишера. Информация Фишера в случае выборки и характер убывания нижних границ дисперсий оценок в случае выборки.
\end{enumerate}

\subsection{Дополнительные вопросы}

\theme{Определения.}
\begin{enumerate}
    \item Записать определение выборки, объема выборки, реализации выборки.
    \item Записать определение вариационного ряда и порядковой статистики.
    \item Записать определение эмпирической функции распределения.
\end{enumerate}

\theme{Точечное оценивание.}
\begin{enumerate}[resume]
    \item Записать определения: оценки, свойств несмещенности и состоятельности оценки.
    \item Сформулировать теорему Бернулли и теорему Хинчина.
    \item Сформулировать неравенство Маркова, Чебышева.
    \item Сформулировать закон больших чисел в форме Чебышева, в форме Маркова.
    \item Сформулировать утверждение о состоятельности асимптотически несмещенной оценки с убывающей дисперсией.
    \item Записать оценку вероятности события, показать её несмещенность и состоятельность.
    \item Записать оценку значения функции распределения, показать её несмещенность и состоятельность.
    \item Записать определение выборочного среднего, выборочной дисперсии и исправленной выборочной дисперсии.
\end{enumerate}

\theme{Методы точечного оценивания.}
\begin{enumerate}[resume]
    \item По выборке из показательного распределения построить оценку параметра по методу моментов.
    \item По выборке из распределения Пуассона построить оценку параметра методом максимального правдоподобия.
    \item По выборке из нормального распределения построить оценку математического ожидания методом порядковых статистик.
\end{enumerate}

\theme{Интервальное оценивание.}
\begin{enumerate}[resume]
    \item По выборке из нормального распределения с неизвестным математическим ожиданием и известной дисперсией построить доверительную границу
          или доверительный интервал для математического ожидания.
    \item По выборке из нормального распределения с известным математическим ожиданием и неизвестной дисперсией построить доверительную границу
          или доверительный интервал для дисперсии.
    \item По выборке из нормального распределения с неизвестным математическим ожиданием и неизвестной дисперсией построить доверительную границу
          или доверительный интервал для математического ожидания.
    \item По выборке из нормального распределения с неизвестным математическим ожиданием и неизвестной дисперсией построить доверительную границу
          или доверительный интервал для дисперсии.
\end{enumerate}

\theme{Проверка гипотез.}
\begin{enumerate}[resume]
    \item Написать статистику критерия хи-квадрат в задаче проверки простой гипотезы о вероятностях. Указать распределение статистики при условии, что основная гипотеза верна.
    \item Написать статистику критерия хи-квадрат в задаче проверки сложной гипотезы о вероятностях. Указать метод оценки параметров и распределение статистики при условии, что основная гипотеза верна.
    \item Написать статистику критерия в задаче проверки гипотезы о независимости признаков. Указать распределение статистики при условии, что основная гипотеза верна.
    \item Написать статистику критерия в задаче проверки гипотезы об однородности. Указать распределение статистики при условии, что основная гипотеза верна.
    \item Написать статистику критерия согласия Колмогорова. Указать распределение статистики при условии, что основная гипотеза верна.
    \item Написать статистику критерия согласия Колмогорова--Смирнова. Указать распределение статистики при условии, что основная гипотеза верна.
    \item Написать статистику критерия Фишера. Указать распределение статистики при условии, что основная гипотеза верна.
    \item Написать статистику критерия Стьюдента. Указать распределение статистики при условии, что основная гипотеза верна.
    \item Сформулирвоать задачу различения двух простых гипотез, записать определения вероятностей ошибок первого и второго рода и равномерно наиболее мощного критерия.
    \item Сформулирвоать задачу различения двух простых гипотез, записать определения вероятностей ошибок первого и второго рода и минимаксного критерия.
    \item Сформулирвоать задачу различения двух простых гипотез, записать определения вероятностей ошибок первого и второго рода и байесовского критерия.
    \item Записать определение критерия отношения вероятностей и объяснить метод проверки гипотез на основе критерия отношения вероятностей.
    \item Записать определение последовательного критерия отношения вероятностей и объяснить метод проверки гипотез на основе критерия отношения вероятностей.
\end{enumerate}

\theme{Регрессия.}
\begin{enumerate}[resume]
    \item Сформулировать задачу линейной регрессии, записать нормальную систему уравнений метода наименьших квадратов.
    \item Сформулировать задачу линейной регрессии, перечислить свойства оценки по методу наименьших квадратов.
    \item Сформулировать задачу линейной регрессии, записать коэффициент детерминации и скорректированный коэффициент детерминации.
    \item Сформулировать задачу нормальной линейной регрессии, записать доверительный интервал для компонента оценки.
    \item Сформулировать задачу нормальной линейной регрессии, записать критерий проверки гипотезы об отсутствии зависимости.
    \item Сформулировать задачу нормальной линейной регрессии с остаточной дисперсией, записать оценку и доверительный интервал для остаточной дисперсии.
\end{enumerate}

\theme{Методы статистических испытаний.}
\begin{enumerate}[resume]
    \item Сформирулировать принцип методов статистических испытаний.
\end{enumerate}

\theme{Эффективные оценки.}
\begin{enumerate}[resume]
    \item Сформулировать определение достаточной статистики и критерий факторизации.
    \item Сформулировать условия регулярности и теорему о неравенстве Крамера-Рао.
    \item Сформулировать условия регулярности и теорему об экспоненциальных семействах распределений.
    \item Записать определения функции правдоподобия, функции вклада и информации Фишера.
\end{enumerate}

\iftutor
    \subsection{Разбиение}
    Категории:
    \begin{enumerate}
        \item сложные (доказательство):
              \ref{int:var-unkn},
              \ref{hyp:fa},
              \ref{hyp:dis-fn-cr},
              \ref{hyp:dis-pr},
              \ref{hyp:dis-cnt},
              \ref{reg:lin:est},
              \ref{reg:lin:dis},
              \ref{reg:nor:dist},
              \ref{reg:nor:conf},
              \ref{eff:comp},
              \ref{eff:fact},
              \ref{eff:BKR},
              \ref{eff:est},
              \ref{eff:CR},
              \ref{eff:exp},
              \ref{eff:F},

        \item средние:
              \ref{est:lt},
              \ref{est:lin},
              \ref{estm:vs},
              \ref{int:exp-kn},
              \ref{int:exp-unkn},
              \ref{int:asy},
              \ref{int:cor},
              \ref{hyp:chi-pr},
              \ref{hyp:chi-pr-par},
              \ref{hyp:dis-fn-pr},
              \ref{reg:lin:pro},
              \ref{reg:lin:det},
              \ref{reg:nor:var},
              \ref{mc:cdf},
              \ref{mc:mom},
              \ref{mc:int},

        \item простые:
              \sout{\ref{def}},
              \ref{est:pro},
              \ref{est:pt-cdf},
              \ref{est:ev},
              \sout{\ref{est:hom}},
              \ref{estm:mom},
              \ref{estm:ml},
              \ref{int:cs},
              \sout{\ref{int:var-kn}},
              \ref{hyp:bas},
              \ref{hyp:chi-pr-par},
              \ref{hyp:chi-ind},
              \ref{hyp:con},
              \ref{hyp:K},
              \ref{hyp:KS},
              \ref{hyp:F},
              \ref{hyp:S},
              \sout{\ref{hyp:par}},
              \ref{hyp:dis-inf},
              \sout{\ref{reg:tp}}.
    \end{enumerate}

    \begin{center}
        \begin{tabular}{|c|c|c|c|c|c|c|c|c|c|c|c|c|c|c|c|c|c|c|}
            \hline
            Билет  & 1             & 2                 & 3              & 4                & 5             & 6                  & 7            & 8            & 9                    & 10                & 11                & 12                  & 13                   & 14                 & 15           & 16                 \\
            \hline
            \hline
            Вопрос & \ref{eff:BKR} & \ref{reg:lin:dis} & \ref{eff:comp} & \ref{hyp:dis-pr} & \ref{eff:exp} & \ref{int:var-unkn} & \ref{hyp:fa} & \ref{eff:CR} & \ref{reg:nor:conf}   & \ref{eff:est}     & \ref{reg:lin:est} & \ref{hyp:dis-fn-cr} & \ref{eff:fact}       & \ref{reg:nor:dist} & \ref{eff:F}  & \ref{hyp:dis-cnt}  \\
            \hline
            Вопрос & \ref{hyp:KS}  & \ref{estm:mom}    & \ref{hyp:S}    & \ref{int:cor}    & \ref{hyp:con} & \ref{reg:lin:pro}  & \ref{est:ev} & \ref{mc:mom} & \ref{hyp:chi-pr-par} & \ref{reg:nor:var} & \ref{est:pro}     & \ref{int:cs}        & \ref{hyp:chi-pr-par} & \ref{estm:vs}      & \ref{est:lt} & \ref{int:exp-unkn} \\
            \hline
            Задача &               &                   &                &                  &               &                    &              &              &                      &                   &                   &                     &                      &                    &              &                    \\
            \hline
        \end{tabular}

        \begin{tabular}{|c|c|c|c|c|c|c|c|c|c|c|c|c|c|c|c|c|c|c|}
            \hline
            Билет  & 17           & 18            & 19                  & 20            & 21                  & 22                 & 23                 & 24                 & 25               & 26               & 27                & 28                & 29                & 30            & 31                & 32               \\
            \hline
            \hline
            Вопрос & \ref{eff:CR} & \ref{hyp:fa}  & \ref{hyp:dis-fn-cr} & \ref{eff:est} & \ref{reg:lin:est}   & \ref{reg:nor:conf} & \ref{int:var-unkn} & \ref{reg:nor:dist} & \ref{eff:exp}    & \ref{hyp:dis-pr} & \ref{eff:comp}    & \ref{reg:lin:dis} & \ref{eff:F}       & \ref{eff:BKR} & \ref{hyp:dis-cnt} & \ref{eff:fact}   \\
            \hline
            Вопрос & \ref{mc:int} & \ref{estm:ml} & \ref{int:exp-kn}    & \ref{est:ev}  & \ref{hyp:dis-fn-pr} & \ref{hyp:F}        & \ref{mc:cdf}       & \ref{est:lin}      & \ref{est:pt-cdf} & \ref{int:asy}    & \ref{hyp:dis-inf} & \ref{hyp:bas}     & \ref{hyp:chi-ind} & \ref{hyp:K}   & \ref{reg:lin:det} & \ref{hyp:chi-pr} \\
            \hline
            Задача &              &               &                     &               &                     &                    &                    &                    &                  &                  &                   &                   &                   &               &                   &                  \\
            \hline
        \end{tabular}
    \end{center}

    \pagebreak

    \fbox{
        \begin{task}
            \ref{eff:BKR}. Достаточные статистики. Теорема Блекуэлла, Колмогорова, Рао. Последовательное «улучшение» оценок.

            \ref{hyp:KS}. Критерий согласия Колмогорова--Смирнова: постановка задачи, основная и альтернативная гипотезы, статистика критерия, значения статистики критерия, если верна альтернативная гипотеза, распределение статистики, если верна основная гипотеза, теорема Смирнова (без доказательства), критическая область, выбор критической области по уровню значимости.
        \end{task}
    }

    \fbox{
        \begin{task}
            \ref{reg:lin:dis}. Постановка задачи линейной регрессии. Операторы проецирования. Вектор возмущения и его проекции, квадраты норм вектора возмущения и проекций, их выражения и математические ожидания.

            \ref{estm:mom}. Метод моментов построения точечных оценок, свойства моментных оценок.
        \end{task}
    }

    \fbox{
        \begin{task}
            \ref{eff:comp}. Сравнение оценок на основе среднеквадратического отклонения. Смещение оценок и классы оценок по смещению. Эффективная оценки в заданном классе оценок. Существование оценки с заданным смещением и несмещенной оценки. Утверждение о единственности оценки эффективной в заданном классе.

            \ref{hyp:S}. Критерий Стьюдента: постановка задачи, основная и альтернативная гипотезы, статистика критерия, значения статистики критерия, если верна альтернативная гипотеза, распределение статистики, если верна основная гипотеза, выбор критической области по уровню значимости.
        \end{task}
    }

    \fbox{
        \begin{task}
            \ref{hyp:dis-pr}. Постановка задачи различения двух простых гипотез, последовательный критерий отношения вероятностей. Утверждение о границах и вероятностях ошибок последовательного критерия отношения вероятностей. «Приближенный» последовательный критерий отношения вероятностей для заданных вероятностей ошибок, утверждение о вероятностях ошибок «приближенного» последовательного критерия отношения вероятностей.

            \ref{int:cor}. Построение доверительного интервала для коэффициента корреляции двумерного нормального распределения с неизвестными математическими ожиданиями и дисперсиями.
        \end{task}
    }

    \fbox{
        \begin{task}
            \ref{eff:exp}. Условия регулярности. R-эффективные оценки. Теорема об экспоненциальных семействах распределений и R-эффективных оценках.

            \ref{hyp:con}. Постановка задачи проверки гипотезы об однородности и критерий проверки: статистика критерия и критическая область.
        \end{task}
    }

    \fbox{
        \begin{task}
            \ref{int:var-unkn}. Теорема Фишера о выборочном среднем и исправленной выборочной дисперсии. Построение доверительных интервалов для дисперсии и среднеквадратического отклонения по выборке из нормального распределения с неизвестным математическим ожиданием.

            \ref{reg:lin:pro}. Постановка задачи линейной регрессии. Свойства оценки по методу наименьших квадратов.
        \end{task}
    }

    \fbox{
        \begin{task}
            \ref{hyp:fa}. Однофакторный дисперсионный анализ: постановка задачи, основное дисперсионное соотношение, межгрупповая и внутригрупповая дисперсии, распределение внутригрупповой дисперсии, значения межгрупповой дисперсии, если верна альтернативная гипотеза, распределение межгрупповой дисперсии, если основная гипотеза верна, статистика критерия, выбор критической области по уровню значимости.

            \ref{est:ev}. Задача точечного оценивания математического ожидания и дисперсии. Выборочное среднее, выборочная дисперсия и исправленная выборочная дисперсии. Несмещенность и состоятельность выборочного среднего, выборочной дисперсии и исправленной выборочной дисперсии (без вывода формулы дисперсии выборочной дисперсии). Точечное оценивание старших моментов: выборочные моменты и их свойства несмещенности и состоятельности.
        \end{task}
    }

    \fbox{
        \begin{task}
            \ref{eff:CR}. Функция правдоподобия, функция вклада и информации Фишера. Условия регулярности. Теорема о неравенстве Крамера--Рао. R-эффективная оценка, замечание об R-эффективной и эффективной оценке в классе.

            \ref{mc:mom}. Принцип методов статистических испытаний. Оценка моментов случайных величин.
        \end{task}
    }

    \fbox{
        \begin{task}
            \ref{reg:nor:conf}. Постановка задачи нормальной линейной регрессии. Доверительные интервалы для неизвестных компонент параметра, доверительная область для параметра, проверка гипотезы об отсутствии зависимости (о нулевом параметре).

            \ref{hyp:chi-pr-par}. Постановка задачи проверки сложной гипотезы о вероятностях и критерий хи-квадрат. Теорема Фишера об асимптотическом распределении минимальной по параметру статистики, если основная гипотеза верна (без доказательства). Теорема Фишера об асимптотическом распределении статистики с МП-оценкой параметра, если основная гипотеза верна (без доказательства). Применение критерия хи-квадрат к задаче проверки гипотезы о распределении с неизвестным параметром.
        \end{task}
    }

    \fbox{
        \begin{task}
            \ref{eff:est}. Достаточные статистики, полные статистики. Утверждение об оценках, являющихся функциями полной статистики. Утверждение об эффективной оценке. План построения эффективных оценок.

            \ref{reg:nor:var}. Постановка задачи нормальной линейной регрессии с остаточной дисперсией. Точечная оценка и доверительный интервал для остаточной дисперсии.
        \end{task}
    }

    \fbox{
        \begin{task}
            \ref{reg:lin:est}. Постановка задачи линейной регрессии. Решение задачи оптимизации на линейном подпространстве, нормальная система уравнений, свойства матрицы Грамма, оценка по методу наименьших квадратов.

            \ref{est:pro}. Задача точечного оценивания неизвестных величин: параметров, вероятностей и моментов. Статистики и оценки, свойства оценок: несмещенность и состоятельность. Сравнение несмещенных оценок на основе дисперсий. Оптимальная оценка. Обобщение критерия сравнения оценок на основе дисперсий с использованием среднеквадратического отклонения, функции потерь и функции риска.
        \end{task}
    }

    \fbox{
        \begin{task}
            \ref{hyp:dis-fn-cr}. Постановка задачи различения двух простых гипотез, вероятности ошибок первого и второго рода, функция мощности, наиболее мощный критерий, минимаксный критерий и байесовский критерий, критерии отношения вероятностей, теорема о построении минимаксного, байесовского и наиболее мощного критерия как соответствующих критериев отношения вероятностей.

            \ref{int:cs}. Доверительный интервал, верхняя и нижняя доверительные границы. Центральная статистика и общий метод построения доверительных интервалов с помощью центральной статистики. Метод построения центральной статистики.
        \end{task}
    }

    \fbox{
        \begin{task}
            \ref{eff:fact}. Достаточные статистики. Замечание о сохранении количества информации Фишера. Теорема Неймана-Фишера (критерий факторизации). Следствие об оценке максимального правдоподобия. Следствие об R-эффективной оценке. Следствие о подчинённости достаточных статистик.

            \ref{hyp:chi-pr-par}. Постановка задачи проверки сложной гипотезы о вероятностях и критерий хи-квадрат. Теорема Фишера об асимптотическом распределении минимальной по параметру статистики, если основная гипотеза верна (без доказательства). Теорема Фишера об асимптотическом распределении статистики с МП-оценкой параметра, если основная гипотеза верна (без доказательства). Применение критерия хи-квадрат к задаче проверки гипотезы о распределении с неизвестным параметром.
        \end{task}
    }

    \fbox{
        \begin{task}
            \ref{reg:nor:dist}. Постановка задачи нормальной линейной регрессии. Распределения оценки по методу наименьших квадратов и квадратов норм вектора возмущений и его проекций. Независимость оценки и проекций вектора возмущений.

            \ref{estm:vs}. Метод порядковых статистик: построение оценок и оценка квантилей. Функции распределения порядковых статистик, функции плотности вероятности порядковых статистик (без доказательства). Теорема Крамера об асимптотической нормальности порядковых статистик и свойства оценок по методу порядковых статистик.
        \end{task}
    }

    \fbox{
        \begin{task}
            \ref{eff:F}. Функция правдоподобия, функция вклада, информация Фишера. Вычисление информации Фишера с помощью второй производной. Аддитивность информации Фишера. Информация Фишера в случае выборки и характер убывания нижних границ дисперсий оценок в случае выборки.

            \ref{est:lt}. Состоятельность оценок и предельные теоремы: теорема Бернулли (без доказательства), теорема Хинчина (без доказательства), неравенство Маркова, неравенство Чебышева, закон больших чисел в форме Чебышёва и в форме Маркова. Утверждение о состоятельности оценки с убывающей дисперсией.
        \end{task}
    }

    \fbox{
        \begin{task}
            \ref{hyp:dis-cnt}. Постановка задачи различения двух простых гипотез, последовательный критерий отношения вероятностей. Утверждение о вероятности остановки последовательного критерия отношения вероятностей. Тождество Вальда и приближенный метод расчета математических ожиданий случайной величины количества шагов до остановки «приближенного» последовательного критерия отношения вероятностей.

            \ref{int:exp-unkn}. Теорема Фишера о выборочном среднем и исправленной выборочной дисперсии. Построение доверительного интервала для математического ожидания по выборке из нормального распределения с неизвестной дисперсией.
        \end{task}
    }

    \fbox{
        \begin{task}
            \ref{eff:CR}. Функция правдоподобия, функция вклада и информации Фишера. Условия регулярности. Теорема о неравенстве Крамера--Рао. R-эффективная оценка, замечание об R-эффективной и эффективной оценке в классе.

            \ref{mc:int}. Принцип методов статистических испытаний. Оценка интегралов.
        \end{task}
    }

    \fbox{
        \begin{task}
            \ref{hyp:fa}. Однофакторный дисперсионный анализ: постановка задачи, основное дисперсионное соотношение, межгрупповая и внутригрупповая дисперсии, распределение внутригрупповой дисперсии, значения межгрупповой дисперсии, если верна альтернативная гипотеза, распределение межгрупповой дисперсии, если основная гипотеза верна, статистика критерия, выбор критической области по уровню значимости.

            \ref{estm:ml}. Метод максимального правдоподобия построения точечных оценок. Свойства МП-оценок: состоятельность и асимптотическая нормальность.
        \end{task}
    }

    \fbox{
        \begin{task}
            \ref{hyp:dis-fn-cr}. Постановка задачи различения двух простых гипотез, вероятности ошибок первого и второго рода, функция мощности, наиболее мощный критерий, минимаксный критерий и байесовский критерий, критерии отношения вероятностей, теорема о построении минимаксного, байесовского и наиболее мощного критерия как соответствующих критериев отношения вероятностей.

            \ref{int:exp-kn}. Построение наикратчайшего доверительного интервала для математического ожидания по выборке из нормального распределения с известной дисперсией.
        \end{task}
    }

    \fbox{
        \begin{task}
            \ref{eff:est}. Достаточные статистики, полные статистики. Утверждение об оценках, являющихся функциями полной статистики. Утверждение об эффективной оценке. План построения эффективных оценок.

            \ref{est:ev}. Задача точечного оценивания математического ожидания и дисперсии. Выборочное среднее, выборочная дисперсия и исправленная выборочная дисперсии. Несмещенность и состоятельность выборочного среднего, выборочной дисперсии и исправленной выборочной дисперсии (без вывода формулы дисперсии выборочной дисперсии). Точечное оценивание старших моментов: выборочные моменты и их свойства несмещенности и состоятельности.
        \end{task}
    }

    \fbox{
        \begin{task}
            \ref{reg:lin:est}. Постановка задачи линейной регрессии. Решение задачи оптимизации на линейном подпространстве, нормальная система уравнений, свойства матрицы Грамма, оценка по методу наименьших квадратов.

            \ref{hyp:dis-fn-pr}. Постановка задачи различения двух простых гипотез, вероятности ошибок первого и второго рода. Критерий отношения вероятностей, свойства вероятностей ошибок критериев отношения вероятностей, утверждение о линейных комбинациях вероятностей ошибок критериев отношения вероятностей.
        \end{task}
    }

    \fbox{
        \begin{task}
            \ref{reg:nor:conf}. Постановка задачи нормальной линейной регрессии. Доверительные интервалы для неизвестных компонент параметра, доверительная область для параметра, проверка гипотезы об отсутствии зависимости (о нулевом параметре).

            \ref{hyp:F}. Критерий Фишера: постановка задачи, основная и альтернативная гипотезы, статистика критерия, значения статистики критерия, если верна альтернативная гипотеза, распределение статистики, если верна основная гипотеза, выбор критической области по уровню значимости.
        \end{task}
    }

    \fbox{
        \begin{task}
            \ref{int:var-unkn}. Теорема Фишера о выборочном среднем и исправленной выборочной дисперсии. Построение доверительных интервалов для дисперсии и среднеквадратического отклонения по выборке из нормального распределения с неизвестным математическим ожиданием.

            \ref{mc:cdf}. Принцип методов статистических испытаний. Оценка значений функций распределения.
        \end{task}
    }

    \fbox{
        \begin{task}
            \ref{reg:nor:dist}. Постановка задачи нормальной линейной регрессии. Распределения оценки по методу наименьших квадратов и квадратов норм вектора возмущений и его проекций. Независимость оценки и проекций вектора возмущений.

            \ref{est:lin}. Постановка задачи построения точечной линейной оценки среднего при разноточных измерениях, метод построения линейной оценки с минимальной дисперсией и свойства коэффициентов.
        \end{task}
    }

    \fbox{
        \begin{task}
            \ref{eff:exp}. Условия регулярности. R-эффективные оценки. Теорема об экспоненциальных семействах распределений и R-эффективных оценках.

            \ref{est:pt-cdf}. Задача точечного оценивания вероятности события, построение оценки и свойства оценки. Задача точечного оценивания значений функции распределения, построение оценки и свойства оценки.
        \end{task}
    }

    \fbox{
        \begin{task}
            \ref{hyp:dis-pr}. Постановка задачи различения двух простых гипотез, последовательный критерий отношения вероятностей. Утверждение о границах и вероятностях ошибок последовательного критерия отношения вероятностей. «Приближенный» последовательный критерий отношения вероятностей для заданных вероятностей ошибок, утверждение о вероятностях ошибок «приближенного» последовательного критерия отношения вероятностей.

            \ref{int:asy}. Построение доверительных интервалов с использованием асимптотической нормальности. Построение доверительного интервала для вероятности события, решение неравенства: точное, приближенное с заменой вероятности оценкой и приближенное с оценкой дисперсии сверху.
        \end{task}
    }

    \fbox{
        \begin{task}
            \ref{eff:comp}. Сравнение оценок на основе среднеквадратического отклонения. Смещение оценок и классы оценок по смещению. Эффективная оценки в заданном классе оценок. Существование оценки с заданным смещением и несмещенной оценки. Утверждение о единственности оценки эффективной в заданном классе.

            \ref{hyp:dis-inf}. Постановка задачи различения двух простых гипотез, последовательные критериии, применение последовательных критериев в задаче различения двух простых гипотез, вероятности ошибок первого и второго родов и случайная величина количества шагов до остановки.
        \end{task}
    }

    \fbox{
        \begin{task}
            \ref{reg:lin:dis}. Постановка задачи линейной регрессии. Операторы проецирования. Вектор возмущения и его проекции, квадраты норм вектора возмущения и проекций, их выражения и математические ожидания.

            \ref{hyp:bas}. Основные определения в задачах проверки статистических гипотез: статистическая гипотеза (простая и сложная), основная и альтернативная гипотезы (альтернативные распределения), статистический критерий и статистика критерия, критическая область и общий принцип проверки гипотез, вероятности ошибок первого и второго родов, функция мощности критерия (функции мощности как характеристика критерия и вид функции мощности «хорошего» критерия), свойства несмещенности и состоятельности критерия.
        \end{task}
    }

    \fbox{
        \begin{task}
            \ref{eff:F}. Функция правдоподобия, функция вклада, информация Фишера. Вычисление информации Фишера с помощью второй производной. Аддитивность информации Фишера. Информация Фишера в случае выборки и характер убывания нижних границ дисперсий оценок в случае выборки.

            \ref{hyp:chi-ind}. Постановка задачи проверки гипотезы о независимости признаков и применение критерия хи-квадрат.
        \end{task}
    }

    \fbox{
        \begin{task}
            \ref{eff:fact}. Достаточные статистики. Замечание о сохранении количества информации Фишера. Теорема Неймана-Фишера (критерий факторизации). Следствие об оценке максимального правдоподобия. Следствие об R-эффективной оценке. Следствие о подчинённости достаточных статистик.

            \ref{hyp:K}. Критерий согласия Колмогорова: постановка задачи, основная и альтернативная гипотезы, статистика критерия, значения статистики критерия, если верна альтернативная гипотеза, распределение статистики критерия точное и приближённое, если верна основная гипотеза, теорема Колмогорова (без доказательства), критическая область, выбор критической области по уровню значимости.
        \end{task}
    }

    \fbox{
        \begin{task}
            \ref{hyp:dis-cnt}. Постановка задачи различения двух простых гипотез, последовательный критерий отношения вероятностей. Утверждение о вероятности остановки последовательного критерия отношения вероятностей. Тождество Вальда и приближенный метод расчета математических ожиданий случайной величины количества шагов до остановки «приближенного» последовательного критерия отношения вероятностей.

            \ref{reg:lin:det}. Постановка задачи линейной регрессии. Коэффициент детерминации и скорректированный коэффициент детерминации.
        \end{task}
    }

    \fbox{
        \begin{task}
            \ref{eff:BKR}. Достаточные статистики. Теорема Блекуэлла, Колмогорова, Рао. Последовательное «улучшение» оценок.

            \ref{hyp:chi-pr}. Постановка задачи проверки простой гипотезы о вероятностях и критерий хи-квадрат. Утверждение о неограниченности по вероятности статистики критерия хи-квадрат, если верна альтернативная гипотеза (без доказательства). Теорема Пирсона об асимптотическом распределение статистики критерия хи-квадрат, если основная гипотеза верна (без доказательства). Состоятельность критерия хи-квадрат (без доказательства). Нецентральное распределение хи-квадрат и асимптотическое распределение статистики критерия хи-квадрат (без доказательства). Условие применимости критерия хи-квадрат на практике.
        \end{task}
    }

    \section{Задачи}

    \theme{Свойства оценок}

    \begin{enumerate}[topsep=1cm, parsep=1cm]
        \item $\sample{\xi}{2}$ --- выборка из равномерного распределения $R [0, a]$, где $a$ --- неизвестный параметр. Является ли статистика
              $\xi_1 + \xi_2$ несмещённой оценкой $a$? Какие статистики являются несмещенными оценками $a^2$: $(\xi_1 + \xi_2)^2$,
              $\frac{5}{6}(\xi_1 + \xi_2)^2$, $\frac{6}{7}(\xi_1 + \xi_2)^2$, $\frac{7}{8}(\xi_1 + \xi_2)^2$?

        \item $\xi_{(1)} \le \xi_{(2)} \le \xi_{(3)}$ --- вариационный ряд, построенный по выборке из равномерного распределения $R[0, a]$, где
              $a$ --- неизвестный параметр. Какие статистики являются несмещенными оценками $a$: $2 \xi_{(1)}$, $3 \xi_{(1)}$, $4 \xi_{(1)}$, $5 \xi_{(1)}$?

        \item Величина $\xi$ имеет биномиальное распределение $Bi(n,p)$, величина $\nu = \frac{\xi}{n}$. Какие оценки являются несмещенными:
              $1 - \nu$ для $1-p$, $\left( 1 - \nu \right)^2$ для $(1-p)^2$, $\nu^2 - \frac{\nu(1-\nu)}{n-1}$ для $p^2$, $\nu^2 - \frac{\nu(1-\nu)}{n}$ для $p^2$,
              $n \nu (1-\nu) \frac{n}{n-1}$ для $\variance{\xi} = np(1-p)$.
    \end{enumerate}

    \theme{Методы точечного оценивания.}

    \begin{enumerate}[resume, topsep=1cm, parsep=1cm]
        \item Имеется 10 измерений интервалов времени между отказами одной машины: 8.3, 3.5, 9.6, 4.5, 4.2, 7.3, 6.3, 5.2, 6.8, 3.5.
              Предполагая, что наблюдения подчиняются показательному распределению с параметром $\theta$, постройте оценку для среднего времени между
              отказами $\frac{1}{\theta}$. Является ли оценка несмещенной и состоятельной?

        \item Величины 9, 8, 10, 4, 7 являются количествами звонков, поступивших на АТС за 1ый, 2ой, 3ий, 4ый и 5ый час наблюдения соответственно.
              Считая, что случайная величина количества звонков за один час имеет распределение Пуассона с неизвестным параметром $\lambda$, постройте
              оценку параметра $\lambda$ методом максимального правдоподобия.

        \item На выходе детектора регистрируются $n = 400$ импульсов с аплитудами $\xi_i = a + \varepsilon_i$, где $a$ --- неизвестная амлитуда,
              и $\varepsilon_i$ --- помехи, являющиеся независимыми в совокупности случайными величинами, имеющими нормальное распределение
              $\mathcal{N} (0, \sigma^2)$, где $\sigma^2 = 7$. Вычислено значение статистики $Z$:
              \[
                  Z = \frac{1}{n} \sum_{i=1}^n \xi_i^2.
              \]
              Постройте оценку для амплитуды $a$.

        \item На вход устройства $n = 100$ раз подаются импульсы с аплитудами $\xi_i = a + \varepsilon_i$, где $a > 10$ --- неизвестная амлитуда,
              и $\varepsilon_i$ --- помехи, являющиеся независимыми в совокупности случайными величинами, имеющими равномерное распределение
              $R[-10, 10]$. На выходе устройства получено значение:
              \[
                  Z = \frac{1}{n} \sum_{i=1}^n \modulus{\xi_i}.
              \]
              Постройте оценку для амплитуды $a$.

        \item $\sample{\xi}{n}$ --- выборка из распределения Лапласа с плотностью вероятности:
              \[
                  p(x; a)
                  = \frac{1}{2} e^{- \modulus{x-a}}.
              \]
              Постройте оценку параметра $a$ методом порядковых статистик.

        \item $\sample{\xi}{n}$ --- выборка из распределения Лапласа с неизвестным параметром $a$:
              \[
                  p(x;a)
                  = \frac{a}{2} e^{-a \modulus{x}}.
              \]
              Постройте оценку параметра $a$ методом максимального правдоподобия.

        \item $\sample{\xi}{n}$ --- выборка из распределения Эрланга 2-го порядка с неизвестным параметром $\theta$, плотность
              вероятности которого имеет вид:
              \[
                  p(x; \theta)
                  = \left \{
                  \begin{array}{ll}
                      0,                                        & x < 0    \\
                      \frac{x}{\theta^2} e^{-\frac{x}{\theta}}, & 0 \le x,
                  \end{array}
                  \right .
              \]
              Постройте оценку для параметра $\theta$ по методу моментов.

        \item $\sample{\xi}{n}$ --- выборка из геометрического распределения $Geom(p)$:
              \[
                  \probability{\xi = k} = p (1-p)^{k-1}.
              \]
              Постройте оценку неизвестного параметра $p$ методом максимального правдоподобия.

        \item $\sample{\xi}{n}$ --- выборка из распределения с плотностью вероятности:
              \[
                  p(x; a, b)
                  = \left \{
                  \begin{array}{ll}
                      0,                                                    & x < a,    \\
                      \frac{2}{b \sqrt{2 \pi}} e^{- \frac{(x-a)^2}{2 b^2}}, & a \le x .
                  \end{array}
                  \right .
              \]
              Cчитая параметр $a$ известным, постройте оценку параметра $b$, определите является ли построенная оценка несмещённой и состоятельной.

        \item $\sample{\xi}{n}$ --- выборка из распределения с плотностью вероятности:
              \[
                  p(x; a)
                  = \left \{
                  \begin{array}{ll}
                      \frac{1}{a} \left( 1 - \modulus{\frac{x}{a}} \right) & \modulus{x} \le a , \\
                      0,                                                   & \modulus{x} > a .
                  \end{array}
                  \right .
              \]
              $\sample{\eta}{n}$ --- выборка, в которой $\eta_k = \xi_k^2$.
              По выборке $\sample{\eta}{n}$ постройте оценку параметра $a$ по методу моментов и методу порядковых статистик.
              Для оценки по методу моментов проверьте несмещенность. Используя оценку по методу моментов, постройте доверительный интервал
              для $a$ с уровнем доверия $P_g = 0.95$, используя оценку по методу порядковых статистик, постройте доверительный интервал
              для $a$ с уровнем доверия $P_g = 0.98$, при $n = 60$.

        \item $\sample{\xi}{n}$ --- выборка из распределения с плотностью вероятности:
              \[
                  p(x; a)
                  = \left \{
                  \begin{array}{ll}
                      \frac{x+a}{2 a^2} & \modulus{x} \le a , \\
                      0,                & \modulus{x} > a .
                  \end{array}
                  \right .
              \]
              $\sample{\eta}{n}$ --- выборка, в которой $\eta_k = \modulus{\xi}$. По выборке $\sample{\eta}{n}$ постройте оценку параметра $a$ методами
              моментов, максимального правдоподобия, порядковых статистик.

        \item $\sample{\xi}{n}$ --- выборка из распределения с плотностью вероятности:
              \[
                  p(x; a)
                  = \left \{
                  \begin{array}{ll}
                      \frac{1}{a} \modulus{x} & \modulus{x} \le a , \\
                      0,                      & \modulus{x} > a .
                  \end{array}
                  \right .
              \]
              Постройте оценку параметра $a$, проверьте несмещённость и состоятельность оценки.

        \item $\sample{\xi}{n}$ --- выборка из распределения с плотностью вероятности:
              \[
                  p(x; a)
                  = \left \{
                  \begin{array}{ll}
                      \frac{2}{a} x & x \in [0, a],     \\
                      0,            & x \notin [0, a] .
                  \end{array}
                  \right .
              \]
              Постройте оценку параметра $a$, проверьте несмещённость и состоятельность оценки.

        \item $\sample{\xi}{n}$ --- выборка из распределения с плотностью вероятности:
              \[
                  p(x; a, b)
                  = \left \{
                  \begin{array}{ll}
                      2 \frac{x-a}{b-a} & x \in [a, b],     \\
                      0,                & x \notin [a, b] .
                  \end{array}
                  \right .
              \]
              Постройте оценки параметров $a$ и $b$, проверьте несмещённость и состоятельность оценок.

        \item $\sample{\xi}{n}$ --- выборка из распределения с плотностью вероятности:
              \[
                  p(x; a, b)
                  = \left \{
                  \begin{array}{ll}
                      2 \frac{x-a}{b^2} & x \in [a, a + b],     \\
                      0,                & x \notin [a, a + b] .
                  \end{array}
                  \right .
              \]
              Постройте оценки параметров $a$ и $b$, проверьте несмещённость и состоятельность оценок.

        \item $\sample{\xi}{n}$ --- выборка из равномерного распределения $R[0, a]$. Проверьте несмещенность и состоятельность оценки:
              $2 \overline{\xi}$, $\overline{\xi} + \frac{\xi_{(n)}}{2}$, $(n+1) \xi_{(1)}$, $\xi_{(1)} + \xi_{(n)}$, $\frac{n+1}{n} \xi_{(n)}$,
              где $\overline{\xi}$ --- выборочное среднее, $\xi_{(1)}$ и $\xi_{(n)}$ --- порядковые статистики.

        \item $\sample{\xi}{n}$ --- выборка из распределения с вероятностями:
              \begin{align*}
                  \probability{\xi_i = -1} & = 1 - a - a^2 - 4b , \\
                  \probability{\xi_i = 2}  & =  5b ,              \\
                  \probability{\xi_i = 5}  & = a + a^2 - b .
              \end{align*}
              Считая параметр $a$ неизвестным, постройте оценку параметра $b$.

        \item $\sample{\xi}{n}$ --- выборка из равномерного распределения $R[0,a]$, $\sample{\eta}{n}$ --- выборка из равномерного распределения $R[a, 2a]$.
              Используя только суммы $\zeta_i = \xi_i + \eta_i$, постройте оценку параметра $a$.

        \item $\sample{\xi}{n}$ --- выборка из распределения Релея с плотностью вероятности:
              \[
                  p(x; \sigma)
                  = \left \{
                  \begin{array}{ll}
                      0,                                              & x < 0     \\
                      \frac{x}{\sigma^2} e^{-\frac{x^2}{2 \sigma^2}}, & 0 \le x .
                  \end{array}
                  \right .
              \]
              Постройте оценки параметра $\sigma^2$ методом моментов и методом максимального правдоподобия. Проверьте несмещенность и состоятельность
              оценок. (Для распределения Релея математическое ожидание $\sigma \sqrt{\frac{\pi}{2}}$ и дисперсия $\frac{4-\pi}{2} \sigma^2$).

        \item $\sample{\xi}{n}$ --- выборка из геометрического распределения $Geom(p)$:
              \begin{gather*}
                  \probability{\xi = k} = (1-p)^{k-1} p , \\
                  k = 1, 2, ...
              \end{gather*}
              Постройте оценку параметра $p$ методом моментов и методом максимального правдоподобия. Определите несмещенность и состоятельность оценок.

        \item $\sample{\xi}{n}$ --- выборка из распределения с плотностью вероятности:
              \[
                  p(x; a, b)
                  = \left \{
                  \begin{array}{ll}
                      0,                              & x < a   \\
                      \frac{1}{b} e^{-\frac{x-a}{b}}, & a \le x
                  \end{array}
                  \right .
              \]
              Постройте оценки параметров $a$ и $b$. Определите несмещённость и состоятельность оценок.

        \item $\sample{\xi}{n}$ --- выборка из отрицательного биномиального распределения $NB(2,p)$, где $p$ --- неизвестный параметр:
              \begin{gather*}
                  \probability{\xi_i = k} = C_{k+1}^k (1-p)^k p^2 , \\
                  k \in \{0, 1, 2, \dots \} .
              \end{gather*}
              Постройте оценку параметра $p$ методом моментов и методом максимального правдоподобия. Определите несмещённость и состоятельность оценок.
    \end{enumerate}

    \theme{Интервальное оценивание.}

    \begin{enumerate}[topsep=1cm, parsep=1cm, resume]
        \item На выходе детектора регистрируются $n = 400$ импульсов с аплитудами $\xi_i = a + \varepsilon_i$, где $a$ --- неизвестная амлитуда,
              и $\varepsilon_i$ --- помехи, являющиеся независимыми в совокупности случайными величинами, имеющими нормальное распределение
              $\mathcal{N} (0, \sigma^2)$, где $\sigma^2 = 7$. Вычислено значение статистики $Z$:
              \[
                  Z = \frac{1}{n} \sum_{i=1}^n \xi_i^2.
              \]
              Постройте доверительный интервал для амплитуды $a$ с уровнем доверия $P_g = 0.95$.

        \item На вход устройства $n = 100$ раз подаются импульсы с аплитудами $\xi_i = a + \varepsilon_i$, где $a > 10$ --- неизвестная амлитуда,
              и $\varepsilon_i$ --- помехи, являющиеся независимыми в совокупности случайными величинами, имеющими равномерное распределение
              $R[-10, 10]$. На выходе устройства получено значение:
              \[
                  Z = \frac{1}{n} \sum_{i=1}^n \modulus{\xi_i}.
              \]
              Постройте доверительный интервал для амплитуды $a$ с уровнем доверия $P_g = 0.98$.

        \item Длительность промежутков времени между отказами имеет показательное распределение $E(\theta)$, измерено $n=20$ промежутков времени между
              отказами, вычислены выборочное среднее $\overline{m} = 11.5$ и исправленная выборочная дисперсия $s^2 = 108$. Постройте приближённый доверительный
              интервал для параметра $\theta$ с уровнем доверия $P_g = 0.95$.

        \item $\sample{\xi}{n}$ --- выборка из нормального распределения $\mathcal{N}(m, \sigma^2)$, для которой $\sum_{i=1}^n \xi = 104$ и
              $\sum_{i=1}^n \xi^2 = 793$. Постройте доверительный интервал для $\sigma$ с уровнем доверия $P_g = 0.99$ для $n = 15$ в двух случаях:
              величина $m = 5$ и величина $m$ неизвестна.

              \item $\sample{\xi}{n}$ --- выборка из распределения Пуассона $Po(\lambda)$, для которой $m = \sum_{i=1}^n \xi = 12.5$ и 
              $s^2 = \frac{1}{n-1} \sum_{i=1}^n (\xi - m)^2 = 112$, $n = 20$. Постройте приближённый доверительный интервал для $\lambda$ с уровнем доверия
              $P_g = 0.98$.

        \item $\sample{\xi}{n}$ --- выборка из распределения Лапласа с плотностью вероятности:
              \[
                  p(x; a)
                  = \frac{1}{2} e^{- \modulus{x-a}}.
              \]
              Постройте доверительный интервал для параметра $a$ с уровнем доверия $P_g = 0.95$ для $n = 30$.

        \item Величина $a_1$ измерялась три раза, результаты измерений --- величины  $\xi_1$, $\xi_2$, $\xi_3$ --- независимы и имеют ошибку измерения
              с нормальным распределением $\mathcal{N}(0,\sigma^2)$. Величина $a_2$ измерялась пять раз, результаты измерений --- величины  $\eta_1$, $\eta_2$,
              $\eta_3$, $\eta_4$, $\eta_5$ --- независимы и имеют ошибку измерения с нормальным распределением $\mathcal{N}(0,\sigma^2)$.
              Постройте доверительный интервал для разности $a_1 - a_2$ с уровнем доверия $P_g = 0.95$.

        \item Величины $\xi_1$ и $\xi_2$ --- два независимых измерения неизвестной величины $a$, сделанные двумя разными приборами. Ошибка первого прибора
              имеет нормальное распределение $\mathcal{N}(0,\sigma_1^2)$, ошибка второго прибора имеет нормальное распределение $\mathcal{N}(0,\sigma_2^2)$.
              Постройте доверительный интервал для величины $a$ с уровнем доверия $P_g = 0.98$.

        \item $\sample{\xi}{n}$ --- выборка из распределения с плотностью вероятности:
              \[
                  p(x; a, b)
                  = \left \{
                  \begin{array}{ll}
                      0,                                                    & x < a,    \\
                      \frac{2}{b \sqrt{2 \pi}} e^{- \frac{(x-a)^2}{2 b^2}}, & a \le x .
                  \end{array}
                  \right .
              \]
              Считая параметр $b$ известным, постройте доверительный интервал для $a$ с уровнем доверия $P_g = 0.95$ для $n=60$.

        \item $\sample{\xi}{n}$ --- выборка из распределения с плотностью вероятности:
              \[
                  p(x; a)
                  = \left \{
                  \begin{array}{ll}
                      \frac{x+a}{2 a^2} & \modulus{x} \le a , \\
                      0,                & \modulus{x} > a .
                  \end{array}
                  \right .
              \]
              $\sample{\eta}{n}$ --- выборка, в которой $\eta_k = \modulus{\xi}$. Постройте доверительный интервал для параметра $a$ с уровнем доверия
              $P_g = 0.98$ для $n = 60$.

        \item $\sample{\xi}{n}$ --- выборка из распределения с плотностью вероятности:
              \[
                  p(x; a)
                  = \left \{
                  \begin{array}{ll}
                      \frac{1}{a} \modulus{x} & \modulus{x} \le a , \\
                      0,                      & \modulus{x} > a .
                  \end{array}
                  \right .
              \]
              Постройте доверительный интервал для параметра $a$ с уровнем доверия $P_g = 0.95$ для $n = 40$.

        \item $\sample{\xi}{n}$ --- выборка из распределения с плотностью вероятности:
              \[
                  p(x; a)
                  = \left \{
                  \begin{array}{ll}
                      \frac{2}{a} x & x \in [0, a],     \\
                      0,            & x \notin [0, a] .
                  \end{array}
                  \right .
              \]
              Постройте доверительный интервал для параметра $a$ с уровнем доверия $P_g = 0.95$ для $n = 50$.

        \item $\sample{\xi}{n}$ --- выборка из распределения с плотностью вероятности:
              \[
                  p(x; a, b)
                  = \left \{
                  \begin{array}{ll}
                      2 \frac{x-a}{b-a} & x \in [a, b],     \\
                      0,                & x \notin [a, b] .
                  \end{array}
                  \right .
              \]
              Постройте доверительные интервалы для параметров $a$ и $b$ с уровнем доверия $P_g = 0.98$ для $n = 70$.

        \item $\sample{\xi}{n}$ --- выборка из распределения с плотностью вероятности:
              \[
                  p(x; a, b)
                  = \left \{
                  \begin{array}{ll}
                      2 \frac{x-a}{b^2} & x \in [a, a + b],     \\
                      0,                & x \notin [a, a + b] .
                  \end{array}
                  \right .
              \]
              Считая величину $b$ известной, постройте доверительный интервал для параметра $a$ с уровнем доверия $P_g = 0.95$ для $n = 40$.

        \item $\sample{\xi}{n}$ --- выборка из распределения с вероятностями:
              \begin{align*}
                  \probability{\xi_i = -1} & = 1 - a - a^2 - 4b , \\
                  \probability{\xi_i = 2}  & =  5b ,              \\
                  \probability{\xi_i = 5}  & = a + a^2 - b .
              \end{align*}
              Считая параметр $a$ неизвестным, постройте приближённый доверительный интервал для параметра $b$ с уровнем доверия $P_g = 0.98$.

        \item $\sample{\xi}{n}$ --- выборка из равномерного распределения $R[0,a]$, $\sample{\eta}{n}$ --- выборка из равномерного распределения $R[a, 2a]$.
              Используя только суммы $\zeta_i = \xi_i + \eta_i$, постройте приближённый доверительный интервал для параметра $a$
              с уровнем доверия $P_g = 0.99$.

        \item $\sample{\xi}{n}$ --- выборка из равномерного распределения $R[0,a]$, $\sample{\eta}{n}$ --- выборка из равномерного распределения $R[a, 2a]$.
              Используя только суммы $\zeta_i = \xi_i + \eta_i$, постройте оценку параметра $a$.

        \item $\sample{\xi}{n}$ --- выборка из распределения Релея с плотностью вероятности:
              \[
                  p(x; \sigma)
                  = \left \{
                  \begin{array}{ll}
                      0,                                              & x < 0    \\
                      \frac{x}{\sigma^2} e^{-\frac{x^2}{2 \sigma^2}}, & 0 \le x,
                  \end{array}
                  \right .
              \]
              Постройте доверительный интервал для параметра $\sigma^2$ с уровнем доверия $P_g = 0.95$ для $n=50$. (Распределение Релея имеет
              случайная величина $\sqrt{\alpha_1^2 + \alpha_2^2}$, где $\alpha_1$ и $\alpha_2$ --- независимые случайные величины, имеющие
              нормальное распределение $\mathcal{N}(0, \sigma^2)$).

        \item $\sample{\xi}{n}$ --- выборка из геометрического распределения $Geom(p)$:
              \begin{gather*}
                  \probability{\xi = k} = (1-p)^{k-1} p , \\
                  k = 1, 2, ...
              \end{gather*}
              Постройте приближенный доверительный интервал для параметра $p$ с уровнем доверия $P_g = 0.95$ для $n = 50$.(Для геометрического
              распределения математическое ожидание $p$ и дисперсия $\frac{1-p}{p^2}$).

        \item $\sample{\xi}{n}$ --- выборка из распределения с плотностью вероятности:
              \[
                  p(x; a, b)
                  = \left \{
                  \begin{array}{ll}
                      0,                              & x < a   \\
                      \frac{1}{b} e^{-\frac{x-a}{b}}, & a \le x
                  \end{array}
                  \right .
              \]
              Считая величину $b$ известной, постройте приближённый доверительный интервал для $a$ с уровнем доверия $P_g = 0.98$ для $n = 40$.

        \item $\sample{\xi}{n}$ --- выборка из отрицательного биномиального распределения $NB(2,p)$, где $p$ --- неизвестный параметр:
              \begin{gather*}
                  \probability{\xi_i = k} = C_{k+1}^k (1-p)^k p^2 , \\
                  k \in \{0, 1, 2, \dots \} .
              \end{gather*}
              Используя оценку параметра постройте приближённый доверительный интервал с уровнем доверия $P_g = 0.95$ для $n=50$.
    \end{enumerate}

    \theme{Проверка гипотез (хи-квадрат).}

    \begin{enumerate}[topsep=1cm, parsep=1cm, resume]
        \item При снятии показаний измерительного прибора десятые доли деления шкалы прибора оценивались на глаз наблюдателем.
              В таблице приведены количества цифр, записанных наблюдателем в качестве десятых долей при 200 измерениях:

              \begin{center}
                  \begin{tabular}{|c|c|c|c|c|c|c|c|c|c|c|c|}
                      \hline
                      Цифры  & 0  & 1  & 2  & 3  & 4  & 5  & 6  & 7  & 8  & 9  \\
                      \hline
                      Кол-во & 35 & 16 & 15 & 17 & 17 & 19 & 11 & 16 & 30 & 24 \\
                      \hline
                  \end{tabular}
              \end{center}

              Проверьте согласуются ли эти данные с гипотезой о том, что каждая цифра при этом могла появиться с равной вероятностью при уровене значимости $\alpha = 0.01$.

        \item Игральная кость брошена 480 раз. Результаты бросаний:
              \begin{center}
                  \begin{tabular}{|c|c|c|c|c|c|c|}
                      \hline
                      Цифра  & 1  & 2  & 3  & 4  & 5  & 6  \\
                      \hline
                      Кол-во & 70 & 75 & 65 & 90 & 85 & 95 \\
                      \hline
                  \end{tabular}
              \end{center}
              Проверить гипотезу о симметричности кости при уровене значимости $\alpha = 0.05$.

        \item В социологичесом опросе среди мужчин за --- 85, против --- 35, среди женщин за --- 35, против --- 25. Можно ли считать, что
              получение ответа "за"{} либо "против"{} не зависит от пола респондента при уровне значимости значимости $\alpha = 0.02$.

        \item Среди студентов одного курса проведены зачеты по физике и иностранному языку, полученные количества студентов сведены в таблицу:

              \begin{center}
                  \begin{tabular}{|c|c|c|}
                      \hline
                                   & Физика зачёт & Физика незачёт \\
                      \hline
                      Язык зачёт   & 50           & 20             \\
                      \hline
                      Язык незачёт & 20           & 10             \\
                      \hline
                  \end{tabular}
              \end{center}
              Проверьте гипотезу о независимости успеваемости по физики и иностранному языку при уровене значимости $\alpha = 0.01$.

        \item Стрелок сделал 100 выстрелов в каждом из двух тиров, в первом тире число попаданий составило 75, во втором --- 85.
              Проверьте гипотезу об одинаковых вероятностях попадания в двух тирах при уровне значимости $\alpha = 0.02$ .

        \item За время работы первой смены было изготовлено 90 деталей, среди которых 15 деталей оказались бракованными, за время работы второй смены было
              изготовлено 110 деталей, среди которых 10 деталей оказались бракованными. Проверьте гипотезу о равенстве вероятностей изготовления бракованной
              детали в двух сменах при уровне значимости $\alpha = 0.05$.

        \item Произведено $n = 400$ изделий, из которых изделий 1го сорта 141, 2го --- 120, 3го --- 99 и бракованных изделий 40. Проверьте гипотезу
              о равенстве вероятностей производства изделий 1го, 2го и 3го сортов при уровне значимости $\alpha = 0.05$.
    \end{enumerate}

    \theme{Различение двух простых гипотез.}

    \begin{enumerate}[topsep=1cm, parsep=1cm, resume]
        \item $\left( \xi_1, \dots, \xi_n \right)$ --- выборка из распределения Пуассона с неизвестным параметром $\lambda$, $n=100$.
              Постройте критерий Неймана--Пирсона с вероятностью ошибки первого рода $\alpha = 0.02$ для различения двух гипотез:
              \begin{gather*}
                  H_0: \lambda = \lambda_0 = 3 , \\
                  H_1: \lambda = \lambda_1 = 2.
              \end{gather*}
              Вычислите приближенно вероятность ошибки второго рода построенного критерия.

        \item $\left\{ \xi_k \right\}_{k=1}^\infty$ --- последовательность независимых случайных величин имеющих распределения Эрланга 2-го порядка с плотностью:
              \[
                  p(x; \theta)
                  = \left \{
                  \begin{array}{ll}
                      0,                                        & x < 0    \\
                      \frac{x}{\theta^2} e^{-\frac{x}{\theta}}, & 0 \le x,
                  \end{array}
                  \right .
              \]
              Постройте последовательный критерий отношения вероятностей для различения двух гипотез:
              \begin{gather*}
                  H_0: \theta = \theta_0 = 2 , \\
                  H_1: \theta = \theta_1 = 1.
              \end{gather*}
              Вычислите приближённо средние количества наблюдений до остановки построенного критерия (математическое ожидание распределения Эрланга 2-го
              порядка равно $2 \theta$).

        \item $\sample{\xi}{n}$ --- выборка из распределения с плотностью вероятности:
              \begin{gather*}
                  p(x; \theta) =
                  \left \{
                  \begin{array}{ll}
                      0,                        & x < 0,    \\
                      \theta^2 x e^{-\theta x}, & 0 \le x .
                  \end{array}
                  \right .
              \end{gather*}
              Постройте минимаксный критерий для различения гипотез:
              \begin{gather*}
                  H_0: \theta = \theta_0 = 0.2 , \\
                  H_1: \theta = \theta_1 = 0.25.
              \end{gather*}
              Определите объём выборки $n$ так, чтобы вероятности ошибок были равны 0.2. Оцените функцию мощности критерия и постройте её график.
    \end{enumerate}

    \theme{Регрессия.}

    \theme{Методы статистических испытаний.}

    \begin{enumerate}[topsep=1cm, parsep=1cm, resume]
        \item С помощью метода статистических испытаний оцените интеграл:
              \[
                  J = \int \limits_{-1}^1 e^{-x^2} dx
              \]
              с точностью менее $\delta = 10^{-2}$, вероятностью более $P_\delta = 0.98$.
    \end{enumerate}

    \theme{Эффективные оценки.}
    \begin{enumerate}[topsep=1cm, parsep=1cm, resume]
        \item $\sample{\xi}{n}$ --- выборка из распределения Пуассона $Po(\lambda)$ с параметром $\lambda$. Найдите достаточные для параметра статистики
              и $R$--эффективные оценки, определите их смещения и дисперсии, вычислите информацию Фишера и нижнюю границу дисперсии Крамера--Рао.

        \item $\sample{\xi}{n}$ --- выборка из показательного распределения $E(\theta)$ с параметром $\theta$. Найдите достаточные для параметра статистики
              и $R$--эффективные оценки, определите их смещения и дисперсии, вычислите информацию Фишера и нижнюю границу дисперсии Крамера--Рао.

        \item $\sample{\xi}{n}$ --- выборка из нормального распределения $\mathcal{N} (m, \sigma^2)$ с параметром $m$, и известным значением $\sigma^2$.
              Найдите достаточные для параметра статистики и $R$--эффективные оценки, определите их смещения и дисперсии, вычислите информацию Фишера
              и нижнюю границу дисперсии Крамера--Рао.

        \item $\sample{\xi}{n}$ --- выборка из нормального распределения $\mathcal{N} (m, \sigma^2)$ с известным значением $m$, и параметром $\sigma^2$.
              Найдите достаточные для параметра статистики и $R$--эффективные оценки, определите их смещения и дисперсии, вычислите информацию Фишера
              и нижнюю границу дисперсии Крамера--Рао.

        \item В совокупности $\sample{\xi}{n}$ величины независимы, каждая величина $\xi_i$ имеет биномиальное распределение $Bi(n_i,p)$, $n_i$
              известны и возможно различны, $p$ --- неизвестный параметр. Найдите достаточные для параметра статистики и $R$--эффективные оценки,
              определите их смещения и дисперсии, вычислите информацию Фишера и нижнюю границу дисперсии Крамера--Рао.

        \item В паре $(\xi_1, \xi_2)$ величины независимы, $\xi_1$ имеет нормальное распределение $\mathcal{N}(a, \sigma_1^2)$, $\xi_2$ имеет
              нормальное распределение $\mathcal{N}(a, \sigma_2^2)$. Найдите достаточные для параметра статистики и $R$--эффективные оценки,
              определите их смещения и дисперсии, вычислите информацию Фишера и нижнюю границу дисперсии Крамера--Рао.

        \item $\sample{\xi}{n}$ --- выборка из распределения Эрланга 2-го порядка с плотностью:
              \[
                  p(x; \theta)
                  = \left \{
                  \begin{array}{ll}
                      0,                                        & x < 0    \\
                      \frac{x}{\theta^2} e^{-\frac{x}{\theta}}, & 0 \le x,
                  \end{array}
                  \right .
              \]
              Найдите достаточные для параметра статистики и $R$--эффективные оценки, определите их смещения и дисперсии, вычислите информацию Фишера
              и нижнюю границу дисперсии Крамера--Рао.

        \item $\sample{\xi}{n}$ --- выборка из распределения Релея с плотностью вероятности:
              \[
                  p(x; \sigma)
                  = \left \{
                  \begin{array}{ll}
                      0,                                              & x < 0    \\
                      \frac{x}{\sigma^2} e^{-\frac{x^2}{2 \sigma^2}}, & 0 \le x,
                  \end{array}
                  \right .
              \]
              Найдите достаточные для параметра статистики и $R$--эффективные оценки, определите их смещения и дисперсии, вычислите информацию Фишера
              и нижнюю границу дисперсии Крамера--Рао.

              \item $\sample{\xi}{n}$ --- выборка из распределения с плотностью вероятности:
              \[
                  p(x; a, b)
                  = \left \{
                  \begin{array}{ll}
                      0,                              & x < a   \\
                      \frac{1}{b} e^{-\frac{x-a}{b}}, & a \le x
                  \end{array}
                  \right .
              \]
              Постройте достаточную статистику для двумерного параметра $(a,b)$.

              \item $\sample{\xi}{n}$ --- выборка из геометрического распределения $Geom(p)$:
              \begin{gather*}
                  \probability{\xi = k} = (1-p)^{k-1} p , \\
                  k = 1, 2, ...
              \end{gather*}
              Найдите достаточные для параметра $p$ статистики и $R$--эффективные оценки, определите их смещения и дисперсии, вычислите информацию Фишера
              и нижнюю границу дисперсии Крамера--Рао.

        \item $\sample{\xi}{n}$ --- выборка из отрицательного биномиального распределения $NB(2,p)$, где $p$ --- неизвестный параметр:
              \begin{gather*}
                  \probability{\xi_i = k} = C_{k+1}^k (1-p)^k p^2 , \\
                  k \in \{0, 1, 2, \dots \} .
              \end{gather*}
              Найдите достаточные для параметра статистики и $R$--эффективные оценки.
    \end{enumerate}
\fi

\end{document}