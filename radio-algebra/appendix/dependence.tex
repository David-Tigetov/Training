\section{О линейной зависимости числовых векторов}

Векторы $x$ и $y$ являются линейно зависимыми, если:
\[
    x = c y,
\]
для некоторого числа $c$.

Пример
\[
    \begin{pmatrix}
        6 \\
        8 \\
        - 12
    \end{pmatrix}
    = 2
    \begin{pmatrix}
        3 \\
        4 \\
        -6
    \end{pmatrix}
\]

\textcolor{red}{рисунок с лучами и масштабированием}

В случае комплексных векторов вместо числовых лучей появляются комплексные плоскости. Умножение на комплексное число можно представить как изменение модуля и
поворот, поэтому в случае векторов допускается не только масштабирование компонент, но и одновременный поворот всех компонент на один угол.

\textcolor{red}{рисунок с плоскостями масштабированием и поворотом}

Пример
\[
    \begin{pmatrix}
        -1 + 3i \\
        2 - 1i  \\
        3 + 1i
    \end{pmatrix}
    =
    (1 + 2i)
    \begin{pmatrix}
        1 + i \\
        -i    \\
        1 - i
    \end{pmatrix}
    .
\]