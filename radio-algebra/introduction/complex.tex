\chapter{Комплексное представление сигналов}


\section{Комплексные числа}

Комплексные числа $z = a + i b$ состоят из двух частей: действительной
\[
    a = \real{z},
\]
и мнимой
\[
    b = \image{z},
\]
где $a$ и $b$ --- действительные числа.

Комплексные числа можно представлять векторами на комплексной плоскости.

Если комплексное число не нулевое, то можно получить тригонометрическую форму:
\begin{gather*}
    z
    = a + i b
    = \sqrt{a^2 + b^2} \left ( \frac{a}{\sqrt{a^2 + b^2}} + i \frac{b}{\sqrt{a^2 + b^2}} \right )
    = A \left ( \cos \varphi + i \sin \varphi \right )
    = A \cos \varphi + i A \sin \varphi , \\
    %
    A = \sqrt{a^2 + b^2} , \\
    \cos \varphi = \frac{a}{\sqrt{a^2 + b^2}} , \\
    \sin \varphi = \frac{b}{\sqrt{a^2 + b^2}} ,
\end{gather*}
где $A$ и $\varphi$ --- модуль и аргумент комплексного числа.


\section{Комплексная экспонента}

Для вещественных чисел $a$ определена функция $e^a$, а её продолжением на комплексную плоскость является функция:
\[
    e^{a + i b}
    = e^a \cdot e^{i b}
    = e^a \cdot \left ( \cos b + i \sin b \right ).
\]


\section{Комплексное представление сигналов}

Пусть функция $u(t)$ описывает напряжение сигнала:
\[
    u(t) = A(t) \cos \varphi(t) ,
\]
где $A(t)$ --- амплитуда, $\varphi(t)$ --- фаза.

Функция $u(t)$ соответствует действительной части комплексно-значной функции $v(t)$ вещественного переменного $t$:
\begin{gather*}
    u(t) = \real{v(t)} , \\
    %
    v(t)
    = A(t) \cos \varphi(t) + i A \sin \varphi(t)
    = A(t) \left ( \cos \varphi(t) + i \sin \varphi(t) \right )
    = A(t) e^{i \varphi(t)} .
\end{gather*}


\section{Сложение сигналов}

Пусть имеются сигналы, которые описываются функциями $u_1(t)$ и $u_2(t)$:
\begin{gather*}
    u_1(t) = A_1(t) \cos \varphi_1(t) , \\
    u_2(t) = A_2(t) \cos \varphi_2(t) ,
\end{gather*}
которым соответствую комплексные представления:
\begin{gather*}
    v_1(t) = A_1(t) e^{i \varphi_1(t)} , \\
    v_2(t) = A_2(t) e^{i \varphi_2(t)} .
\end{gather*}
тогда в результате сложения получается функция $u(t)$:
\[
    u(t)
    = u_1(t) + u_2(t)
    = \real{v_1(t)} + \real{v_2(t)}
    = \real{v_1(t) + v_2(t)} ,
\]
которой соответствует комплексное представление:
\[
    v(t) = v_1(t) + v_2(t).
\]


\section{Преобразование}

Пусть сигнал с функцией $u(t)$ и комплексным представлением $v(t)$:
\begin{gather*}
    u(t) = A(t) \cos \varphi(t) , \\
    v(t) = A(t) e^{i \varphi(t)}
\end{gather*}
преобразуется в некотором устройстве, и в результате преобразования изменяются амплитуда или фаза:
\[
    \widetilde{u}(t) = B(t) \cdot A(t) \cos ( \varphi(t) + \theta(t) ) ,
\]
тогда комплексное представление $\widetilde{v}(t)$ сигнала $\widetilde{u}(t)$ имеет вид:
\[
    \widetilde{v}(t)
    = A(t) \cdot B(t) e^{i (\varphi(t) + \theta(t))}
    = A(t) e^{i \varphi(t)} \cdot B(t) e^{i \theta(t))} = v(t) \cdot s(t),
\]
где функция $s(t)$ является комплексным представлением преобразования:
\[
    s(t) = B(t) e^{i \theta(t)} .
\]


\section{Комплексная огибающая}

Наиболее часто в радиолокации встречаются узкополосные сигналы, представляющие собой суперпозицию колебаний с частотами из узкой полосы частот, вокруг несущей
частоты $\omega$ (частота $\omega$ обычно составляет несколько мегагерц, поскольку антенны излучают сигналы высоких частот):
\[
    u(t) = A(t) \cos ( \omega t + \theta(t) ),
\]
где функции $A(t)$ и $\varphi(t)$ представляют модуляцию сигнала. Такому сигналу соответствует комплексное представление:
\[
    v(t)
    = A(t) e^{i ( \omega t + \theta(t) )}
    = A(t) e^{i \theta(t)} \cdot e^{i \omega t} .
\]
Первый множитель
\[
    v_s(t) = A(t) e^{i \theta(t)}
\]
является комплексной огибающей.


\section{Квадратурный детектор}

Сигнал $u(t)$ можно представить в виде суммы:
\[
    u(t)
    = A(t) \cos ( \omega t + \varphi(t) )
    = A(t) \cos \varphi(t) \cos \omega t - A(t) \sin \varphi(t) \sin \omega t .
\]
в которой множители
\begin{gather*}
    I(t) = A(t) \cos \varphi(t) , \\
    Q(t) = - A(t) \sin \varphi(t) ,
\end{gather*}
называются синфазной и квадратурной составляющими соответственно.

Составляющие $I(t)$ и $Q(t)$ определяют действительную и мнимую части комплексной огибающей $v(t)$:
\begin{gather*}
    I(t) = \real{v(t)} , \\
    Q(t) = - \image{v(t)} .
\end{gather*}

Устройство, которое выделяет составляющие $I(t)$ и $Q(t)$, называется квадратурным детектором.