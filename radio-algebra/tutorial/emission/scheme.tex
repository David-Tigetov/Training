\chapter{Диаграммообразующая схема}

Пусть $a_\alpha$ и $b_\alpha$ --- векторы огибающих сигналов сечения входа схемы и $a$ и $b$ --- векторы огибащих сигналов сечения выхода схемы. У схемы два входа ---
$a_\alpha$ и $b$ и два выхода --- $a$ и $b_\alpha$, которые связаны со входами:
\begin{gather}
    a        = S_{\beta \alpha} a_\alpha + S_{\beta \beta} b
    \label{emission:power:upper_output}, \\
    b_\alpha = S_{\alpha \alpha} a_\alpha + S_{\alpha \beta} b
    \label{emission:power:lower_output}
\end{gather}
Причем $a$ и $b$ ---  огибающие сигналов в сечении входа антенны, для которых справедливо равенство:
\begin{equation}
    \label{emission:power:antenna:reflections}
    b = S_\beta a ,
\end{equation}
где $S_\beta$ --- матрица рассеяния антенны.

Согласно равенствам \eqref{emission:power:upper_output} и \eqref{emission:power:antenna:reflections} вектор огибающих входа антенны $a$:
\begin{align}
    a & = S_{\beta \alpha} a_\alpha + S_{\beta \beta} b , \notag \\
    a & = S_{\beta \alpha} a_\alpha + S_{\beta \beta} S_\beta a , \notag \\
    a - S_{\beta \beta} S_\beta a & = S_{\beta \alpha} a_\alpha , \notag \\
    ( I - S_{\beta \beta} S_\beta ) a & = S_{\beta \alpha} a_\alpha , \notag \\
    a & = ( I - S_{\beta \beta} S_\beta )^{-1} S_{\beta \alpha} a_\alpha , \label{emission:power:antenna:input}
\end{align}
тогда линейная часть диаграммы направленности антенны с диаграммообразующей схемой:
\[
    \widetilde{F}_s(w, a_\alpha)
    = F_a(w) a(a_\alpha)
    = F_a(w) ( I - S_{\beta \beta} S_\beta )^{-1} S_{\beta \alpha} a_\alpha
    = F_s(w) a_\alpha,
\]
где
\[
    F_s(w) = F_a(\vec{w}) ( I - S_{\beta \beta} S_\beta )^{-1} S_{\beta \alpha} .
\]

Из равенств \eqref{emission:power:upper_output}, \eqref{emission:power:antenna:reflections} и \eqref{emission:power:antenna:input} вектор огибающих выхода
диаграммообразующей схемы:
\begin{align*}
    b_\alpha & = S_{\alpha \alpha} a_\alpha + S_{\alpha \beta} b , \\
    b_\alpha & = S_{\alpha \alpha} a_\alpha + S_{\alpha \beta} S a , \\
    b_\alpha & = S_{\alpha \alpha} a_\alpha + S_{\alpha \beta} S ( I - S_{\beta \beta} S )^{-1} S_{\beta \alpha} a_\alpha , \\
    b_\alpha & = ( S_{\alpha \alpha} + S_{\alpha \beta} S ( I - S_{\beta \beta} S )^{-1} S_{\beta \alpha} ) a_\alpha , \\
\end{align*}
откуда матрица рассеяния для диаграммообразующей схемы:
\begin{equation}
    \label{emission:power:scheme:reflections}
    S_\alpha = S_{\alpha \alpha} + S_{\alpha \beta} S ( I - S_{\beta \beta} S )^{-1} S_{\beta \alpha} .
\end{equation}


\section{Рассеяние мощности с диаграммообразующей схемой}

Для диаграммообразующей схемы справедливо неравенство, аналогичное неравенству \eqref{emission:emitter:power:inequality}:
\begin{align}
    a_\alpha^* Q_\alpha a_\alpha & \le a_\alpha^* a_\alpha , \label{emission:power:scheme:inequality}
\end{align}
где $Q_\alpha$ --- матрица, составленная из элементов, характеризующих степень ортогональности лучей диаграммообразующей схемы:
\[
    Q_{\alpha,jk} = \frac{1}{4 \pi} \iint \limits_{4 \pi} g_{j}^*(\vec{w}) g_{k}(\vec{w}) d \Omega
\]
Введем величины нормы:
\[
    h_j = \frac{1}{4 \pi} \iint \limits_{4 \pi} g_{j}^*(\vec{w}) g_{j}(\vec{w}) d \Omega ,
\]
и преобразуем элементы матрицы $Q_\alpha$ к виду:
\[
    Q_{\alpha,jk}
    =
    \sqrt{h_j}
    \cdot
    \frac{\frac{1}{4 \pi} \iint \limits_{4 \pi} g_j^*(\vec{w}) g_k(\vec{w}) d \Omega}{\sqrt{h_j} \sqrt{h_k}}
    \cdot
    \sqrt{h_k} ,
\]
тогда
\[
    Q_\alpha = \sqrt{H} R \sqrt{H} ,
\]
где
\[
    \sqrt{H}
    = \begin{pmatrix}
          \sqrt{h_1} & 0          & 0          & \dots  & 0          \\
          0          & \sqrt{h_2} & 0          & \dots  & 0          \\
          0          & 0          & \sqrt{h_3} & \dots  & 0          \\
          \vdots     & \vdots     & \vdots     & \ddots & \vdots     \\
          0          & 0          & 0          & \dots  & \sqrt{h_n} \\
    \end{pmatrix} ,
    \;
    %
    R_{jk} = \frac{\frac{1}{4 \pi} \iint \limits_{4 \pi} g_j^*(\vec{w}) g_k(\vec{w}) d \Omega}{\sqrt{h_j} \sqrt{h_k}}
\]

Таким образом, неравенство \eqref{emission:power:scheme:inequality} имеет вид:
\[
    a_\alpha^* \sqrt{H} R \sqrt{H} a_\alpha \le a_\alpha^* a_\alpha ,
\]
Пусть $x = \sqrt{H} a_\alpha$, тогда:
\begin{align*}
    x^* R x & \le x^* (\sqrt{H}^{-1})^* (\sqrt{H}^{-1}) x , \\
    x^* R x & \le x^* H^{-1} x , \\
\end{align*}
Пусть $r_{max}$ --- наибольшее собственное значение матрицы $R$ и $x_{max}$ --- соответствующей этому числу собственный вектор, а $h_{min}$ --- наименьшее из
значений $h_k$, тогда:
\begin{align*}
    x_{max}^* R x_{max} & \le x_{max}^* H^{-1} x_{max} , \\
    x_{max}^* r_{max} x_{max} & \le \sum_{k=1}^n \frac{1}{h_k} x_{max,k}^* x_{max,k} , \\
    r_{max} \norm{x_{max}}^2 & \le \sum_{k=1}^n \frac{1}{h_k} \modulus{x_{max,k}}^2 , \\
    r_{max} \norm{x_{max}}^2 & \le \sum_{k=1}^n \frac{1}{h_{min}} \modulus{x_{max,k}}^2 , \\
    r_{max} \norm{x_{max}}^2 & \le \frac{1}{h_{min}} \sum_{k=1}^n \modulus{x_{max,k}}^2 , \\
    r_{max} \norm{x_{max}}^2 & \le \frac{1}{h_{min}} \norm{x_{max}}^2 , \\
    r_{max} & \le \frac{1}{h_{min}} , \\
    h_{min} & \le \frac{1}{r_{max}} .
\end{align*}


\section{Двухлучевая диаграммообразующая схема}

Пусть в диаграммообразующей схеме два входа, тогда матрица $R$ имеет вид:
\[
    R
    = \begin{pmatrix}
          1        & R_{12} \\
          R_{12}^* & 1
    \end{pmatrix} ,
\]
где
\[
    R_{12}
    =
    \frac
    {\frac{1}{4 \pi} \iint \limits_{4 \pi} g_1^*(\vec{w}) g_2(\vec{w}) d \Omega}
    {\sqrt{\frac{1}{4 \pi} \iint \limits_{4 \pi} g_1^*(\vec{w}) g_1(\vec{w}) d \Omega} \cdot \sqrt{\frac{1}{4 \pi} \iint \limits_{4 \pi} g_2^*(\vec{w}) g_2(\vec{w}) d \Omega}}
    = \frac
    {\iint \limits_{4 \pi} g_1^*(\vec{w}) g_2(\vec{w}) d \Omega}
    {\sqrt{\iint \limits_{4 \pi} \norm{g_1}^2 d \Omega} \cdot \sqrt{\frac{1}{4 \pi} \iint \limits_{4 \pi} \norm{g_2}^2 d \Omega}}
\]
и наибольшее собственное значение имеет вид:
\begin{multline*}
    r_{max}
    = \frac{\tr(R) + \sqrt{\tr^2(R) - 4 \det(R)}}{2} = \\
    %
    = \frac{2 + \sqrt{2^2 - 4 (1 - \modulus{R_{12}}^2)}}{2}
    = \frac{2 + \sqrt{4 - 4 + 4 \modulus{R_{12}}^2)}}{2} = \\
    %
    = \frac{2 + 2 \modulus{R_{12}}}{2}
    = 1 + \modulus{R_{12}} .
\end{multline*}
Таким образом,
\[
    h_{min} \le \frac{1}{1 + \modulus{R_{12}}} .
\]
