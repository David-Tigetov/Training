\chapter{Плоская волна}

Будем использовать упрощённую модель плоской волны, фронт распространения которой является прямой.

\textcolor{red}{Рисунок волн.}

Зафиксируем декартову систему координат. В точке соответствующей началу отсчёта колебания имеют фазу:
\[
    \varphi_0(t) = \varphi_0 + \omega t.
\]
Точка начала отсчёта называется фазовым центром.

Наша цель заключается в расчёте фаз во всех других точках. Если через начало координат провести прямую параллельно фронту распространения волны, то во всех точках
этой прямой фаза будет такая же.

А что делать с остальными точками? Рассмотрим волновой вектор $\vec{w}$, который направлен в сторону распространения волны перпендикулярно фронту и имеет длину
\[
    \modulus{\vec{w}}
    = \frac{\omega}{v}
    = \frac{\omega \cdot T}{v \cdot T}
    = \frac{2 \pi}{\lambda} ,
\]
где $\omega$ --- угловая скорость колебаний, $v$ --- линейная скорость распространения волны, $T$ --- период колебаний, $\lambda$ --- длина волны (расстояние,
которое проходит волна за один период).

Проведём прямую через начало координат в направлении волнового вектора и рассмотрим изменение фазы колебаний в точках прямой. Если продвинуться на расстояние
$l$ по прямой в направлении волнового вектора $\vec{w}$ до точки $A$, то фаза изменится на $2 \pi \frac{l}{\lambda}$, а если продвинуться в обратную сторону
до точки $A^\prime$, то фаза изменится на $- 2 \pi \frac{l}{\lambda}$. Таким образом, если $l$ --- расстояние со знаком от точки прямой до начала отсчёта
(положительное направление отсчёта в направлении волнового вектора), то измение фазы $\Delta \varphi(l)$ будет равно:
\[
    \Delta \varphi(l)
    = 2 \pi \frac{l}{\lambda}
    = \frac{2 \pi} {\lambda} l
    = \modulus{\vec{w}} l .
\]
Таким образом, $\modulus{\vec{w}}$ является линейный коэффициентом изменения фазы. Если через точку $A$ провести прямую параллельную фронту распространения волны,
то все точки на этой прямой будут иметь такое же изменение фазы.

Теперь понятно как вычислить изменение фазы для любой точки $B$: необходимо через точку $B$ провести прямую параллельную фронту распространения волны, найти точку
пересечения с прямой проведённой через начало координат в направлении волнового вектора и вычислить расстояние $l$ от начала отсчёта до точки пересечения. Если
точка $B$ имеет радиус-вектор $\vec{r}$, то расстояние $l$ является проекцией вектора $\vec{r}$ на направление волнового вектора $\vec{w}$:
\[
    l = \scalarproduct{\vec{r}}{\frac{\vec{w}}{\modulus{\vec{w}}}}
\]
изменение фазы:
\[
    \Delta \varphi ( \vec{r} )
    = \modulus{\vec{w}} l
    = \modulus{\vec{w}} \scalarproduct{\vec{r}}{\frac{\vec{w}}{\modulus{\vec{w}}}}
    = \scalarproduct{\vec{r}}{\modulus{\vec{w}}  \frac{\vec{w}}{\modulus{\vec{w}}}}
    = \scalarproduct{\vec{r}}{\vec{w}}
\]
и фаза в точке $B$:
\[
    \varphi(t, \vec{r})
    = \varphi_0(t) + \Delta \varphi ( \vec{r} )
    = \varphi_0 + \omega t + \scalarproduct{\vec{r}}{\vec{w}}
    = \varphi_0 + \scalarproduct{\vec{r}}{\vec{w}} + \omega t .
\]
