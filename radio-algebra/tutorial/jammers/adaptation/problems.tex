\chapter{Адаптация}


\section{Приёмники}

Состояние приёмников описывается вектором комплексных огибающих:
\[
    Y = X + z \cdot U
\]

Нужно разработать преобразование $\mathcal{F}(Y)$, которое выделяет огибающую $z$, причём преобразование $\mathcal{F}(Y)$ должно быть простым, более точно линейным
по компонентам $Y$:
\[
    \mathcal{F}(Y)
    = w_1^H y_1 + w_2^H y_2 + \dots + w_n^H y_n
    = W^H Y ,
\]
с весовым вектором
\[
    W =
    \begin{pmatrix}
        w_1   \\
        \dots \\
        w_n
    \end{pmatrix} .
\]
Таким образом,
\[
    \mathcal{F}(Y)
    = W^H \left ( X + z U\right )
    = W^H X + z \cdot W^H U
\]
Желательно чтобы
\begin{gather*}
    \modulus{W^H X} \ll \modulus{W^H U} , \\
    \modulus{W^H X}^2 \ll \modulus{W^H U}^2
\end{gather*}
Поскольку слева стоит случайная величина, то будем ориентироваться на её среднее значение:
\[
    \expectation{\modulus{W^H X}^2} \ll \modulus{W^H U}^2 .
\]
Более точно, будем стараться выбирать вектор $W$ так, чтобы наибольшим было отношение $\rho$:
\[
    \rho ( W ) = \frac{\modulus{W^H U}^2}{\expectation{\modulus{W^H X}^2}} .
\]
Необходимо найти вектор $W_{max}$, при котором отношение $\rho(W)$ достигает наибольшего значения.
\[
    \rho(W_{max}) = \max \limits_W \rho(W) .
\]
Отношение $\rho$ показывает отношение мощности полезного сигнала $\modulus{W^H U}^2$ к средней мощности шумов $\modulus{W^H X}^2$:
\[
    \expectation{\modulus{W^H X}^2}
    = \expectation{W^H X X^H W}
    = W^H \expectation{X X^H} W
    = W^H \variance{X} W .
\]
Обозначим
\[
    R = \variance{X}
\]
Таким образом,
\[
    \rho ( W )
    = \frac{\modulus{W^H U}^2}{W^H \variance{X} W}
    = \frac{W^H U U^H W}{W^H R W} .
\]

Будем считать, что ковариационная матрица $R$ является положительно определённой:
\[
    R > 0 ,
\]
тогда существует квадратный корень $R^\frac{1}{2}$:
\[
    R = R^\frac{H}{2} \cdot R^\frac{1}{2}
\]
и отношение
\[
    \rho ( W )
    = \frac{W^H U U^H W}{W^H R^\frac{H}{2} \cdot R^\frac{1}{2} W}
    = \frac{W^H U U^H W}{\left ( R^\frac{1}{2} W \right )^H \cdot R^\frac{1}{2} W}
\]
Пусть
\[
    Z = R^\frac{1}{2} W ,
\]
тогда
\[
    \rho ( W ) = \frac{W^H U U^H W}{Z^H Z} .
\]
Нужно чтобы в числителе тоже был вектор $Z$. Заметим, что матрица $R^\frac{1}{2}$ невырождена, поэтому существует обратная:
\begin{gather*}
    R^{-\frac{1}{2}} R^\frac{1}{2} = I_n , \\
    R^{\frac{H}{2}} R^{-\frac{H}{2}} = I_n .
\end{gather*}
Добавим эти произведения в числитель
\[
    \rho ( W )
    = \frac{W^H R^\frac{H}{2} R^{-\frac{H}{2}} U U^H R^{-\frac{1}{2}} R^\frac{1}{2} W}{Z^H Z}
    = \frac{\left (  R^\frac{1}{2} W \right )^H R^{-\frac{H}{2}} U U^H R^{-\frac{1}{2}} R^\frac{1}{2} W}{Z^H Z}
    = \frac{Z^H R^{-\frac{H}{2}} U U^H R^{-\frac{1}{2}} Z}{Z^H Z} .
\]
Получили отношение Релея:
\begin{gather*}
    \rho ( W ) = \frac{Z^H A Z}{Z^H Z} , \\
    A = R^{-\frac{H}{2}} U U^H R^{-\frac{1}{2}} = R^{-\frac{H}{2}} U \left ( R^{-\frac{H}{2}} U \right )^H.
\end{gather*}
Наибольшее значение отношение Релея равно наибольшему собственному значению матрицы $A$. Матрица $A$ имеет ранг 1, поэтому у матрицы $A$ есть только одно
отличное от нуля собственное значение $\lambda_{max}$ и, как не трудно заметить, этому собственному значению соответствует вектор $R^{-\frac{H}{2}} U$,
действительно:
\[
    A \left ( R^{-\frac{H}{2}} U \right )
    = R^{-\frac{H}{2}} U \left ( R^{-\frac{H}{2}} U \right )^H R^{-\frac{H}{2}} U
    = R^{-\frac{H}{2}} U \norm{R^{-\frac{H}{2}} U}^2
    = \norm{R^{-\frac{H}{2}} U}^2 \cdot R^{-\frac{H}{2}} U .
\]
Отсюда же следует, что
\[
    \max \limits_W \rho(W) = \lambda_{max} = \norm{R^{-\frac{H}{2}} U}^2 .
\]
Таким образом, вектор $Z_{max}$:
\[
    Z_{max} = R^{-\frac{H}{2}} U ,
\]
а исходный вектор $W_{max}$:
\begin{align*}
    R^{\frac{1}{2}} W_{max} & = R^{-\frac{H}{2}} U , \\
    W_{max} & = R^{-\frac{1}{2}} R^{-\frac{H}{2}} U , \\
    W_{max} & = \left ( R^{\frac{H}{2}} R^{\frac{1}{2}} \right )^{-1} U , \\
    W_{max} & = R^{-1} U .
\end{align*}


\section{Отсутствие источников излучения}

Состояние шумов
\[
    X = E .
\]
Ковариационная матрица
\[
    R = \variance{X} = \sigma_0 I_n .
\]
Оптимальный весовой вектор $W_{max}$:
\[
    W_{max}
    = R^{-1} U
    = \frac{1}{\sigma_0^2} U .
\]

