\chapter{Антенная решётка}


\section{Один приёмник}

Если приёмник поместить в среду с электромагнитными колебаниями, то приёмник будет выделять из общего электромагнитного фона только колебания из узкой полосы вокруг
несущей частоты $\omega$ и на выходе приёмника будет наблюдаться сигнал с комплексным представлением:
\[
    v_1(t) = A(t) e^{i \theta(t)} \cdot e^{i \omega t} ,
\]
где $A(t) e^{i \varphi(t)}$ --- комплексная огибающая.

Если рядом с первым приёмником поместить второй приёмник, то на выходе второго приёмника тоже будет наблюдаться сигнал с комплексным представлением $v_2(t)$.
Какова функция $v_2(t)$? Оказывается функция $v_2(t)$ не является произвольной и связана с функцией сигнала первого приёмника $v_1(t)$ и эта взаимосвязь
функций определяется характером распространения волны на несущей частоте $\omega$.


\section{Два приёмника}

Пусть имеется плоская волна с волновым вектором $\vec{w}$ в некоторой декартовой системе координат, и в этой системе местоположение первого приёмника определяется
радиус-вектором $\vec{r}_1$, а второго --- радиус-вектором $\vec{r}_2$, тогда фазы колебаний в первом и втором приёмниках $\varphi_1(t)$ и $\varphi_2(t)$:
\begin{align*}
    \varphi_1(t) & = \varphi_0(t) + \scalarproduct{\vec{r}_1}{\vec{w}} , \\
    \varphi_2(t) & = \varphi_0(t) + \scalarproduct{\vec{r}_2}{\vec{w}} .
\end{align*}

В целях упростить выражения для фаз поместим фазовый центр в первый приёмник и направим ось абсцисс системы координат в направлении второго приёмника. В этом случае,
$\vec{r}_1$ = 0, поэтому:
\begin{align*}
    \varphi_1(t) & = \varphi_0(t) , \\
    \varphi_2(t) & = \varphi_0(t) + \scalarproduct{\vec{r}_2}{\vec{w}} = \varphi_1(t) + \scalarproduct{\vec{r}_2}{\vec{w}} .
\end{align*}

Пусть угол между осью ординат и волновым вектором равен $\alpha$, а длина $\modulus{\vec{r}_2} = d$, тогда
\begin{gather*}
    \scalarproduct{\vec{r}_2}{\vec{w}}
    = \modulus{\vec{r}_2} \modulus{\vec{w}} \cos \left ( \frac{\pi}{2} - \alpha \right )
    = d \frac{2 \pi}{\lambda} \sin \alpha
    = 2 \pi \frac{d}{\lambda} \sin \alpha ,
\end{gather*}
поэтому фаза колебаний второго приёмника:
\begin{gather*}
    \varphi_2(t) = \varphi_1(t) + \Delta \varphi , \\
    \Delta \varphi = 2 \pi \frac{d}{\lambda} \sin \alpha .
\end{gather*}

Если на выходе первого приёмника имеется сигнал $u_1(t)$:
\[
    u_1(t)
    = A \cos \left ( \varphi_1(t) \right )
    = A \cos \left ( \varphi_0 + \omega t \right )
\]
с комплексным представлением:
\[
    v_1(t)
    = A e^{i \varphi_0} \cdot e^{i \omega t} ,
\]
то на выходе второго приёмника будет сигнал $u_2(t)$:
\[
    u_2(t)
    = A \cos \left ( \varphi_1(t) + \Delta \varphi \right )
    = A \cos \left ( \varphi_0 + \omega t + \Delta \varphi \right )
\]
с комплексным представлением:
\[
    v_2(t)
    = A e^{i \varphi_0 + \Delta \varphi } \cdot e^{i \omega t} .
\]
Таким образом, комплексные огибающие $v_1$ и $v_2$ первого и второго приёмников
\begin{align*}
    s_1 & = A e^{i \varphi_0} , \\
    s_2 & = A e^{i \varphi_0 + \Delta \varphi} = A e^{i \varphi_0} \cdot e^{i \Delta \varphi} = s_1 \cdot e^{i \Delta \varphi}
\end{align*}


\section{Одномерная решётка}

Продолжим помещать приёмники на оси абсцисс через равные расстояния $d$ (расстояние между первым и вторым приёмниками) и получим эквидистантную антенную решётку.
Пусть приёмник с номером $k$ (нумерация в положительном направлении оси абсцисс) имеет радиус-вектор $\vec{r}_k$, тогда фаза $\varphi_k(t)$ у приёмника
с номером $k$:
\[
    \varphi_k(t) = \varphi_1(t) + \scalarproduct{\vec{r}_k}{\vec{w}},
\]
где длина вектора $\modulus{\vec{r}_k} = (k-1) d$, поэтому скалярное произведение
\[
    \scalarproduct{\vec{r}_k}{\vec{w}}
    = \modulus{\vec{r}_k} \modulus{\vec{w}} \cos \left ( \frac{\pi}{2} - \alpha \right )
    = (k-1) d \frac{2 \pi}{\lambda} \sin \alpha
    = (k-1) 2 \pi \frac{d}{\lambda} \sin \alpha
    = (k-1) \Delta \varphi,
\]
откуда на выходе $k$-го приёмника будет сигнал $u_k(t)$:
\[
    u_k(t)
    = A \cos \left ( \varphi_1(t) + (k-1) \Delta \varphi \right )
    = A \cos \left ( \varphi_0 + \omega t + (k-1) \Delta \varphi \right )
\]
с комплексным представлением:
\[
    v_k(t)
    = A e^{i \varphi_0 + (k-1) \Delta \varphi } \cdot e^{i \omega t} .
\]
и комплексной огибающей:
\[
    s_k
    = A e^{i \varphi_0 + (k-1) \Delta \varphi }
    = A e^{i \varphi_0 } \cdot e^{i (k-1) \Delta \varphi}
    = s_1 \cdot e^{i (k-1) \Delta \varphi} .
\]


\section{Двумерная решётка}

Нужно провести плоскость через волновой вектор и радиус-вектор приёмника.

Радиус-вектор приёмника заменить суммой двух векторов вдоль направления оси абсцисс и ординат, скалярное произведение с радиус-вектором приёмника будет состоять
из двух слагаемых, определяющих изменение фазы по оси абсцисс и оси ординат.

