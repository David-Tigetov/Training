\chapter{Обнаружение и пеленгация источников}


\section{\textcolor{red}{Задачи}}

Рассматривается антенная решётка, образованная некоторым количеством приёмников, состояния которых описываются комплексными огибающими.

При отсутствии источников излучения комплексные огибающие приёмников определяются внутренними шумами, которые носят случайный и хаотичный характер.

При появлении источника излучения возникает упорядоченность в наборе комплексных огибающих приёмников. Воздействие источника излучения на каждый приёмник
в отдельности является случайным, но воздействие источника излучения на совокупность приёмников имеет вполне регулярный характер, который
определяется местоположением источника излучения.

Конечно, собственные шумы приёмников не исчезают при появлении источника излучения, поэтому комплексные огибающие приёмников определяются
суперпозицией собственных шумов и воздействием источника излучения, а при наличии нескольких источников излучения, суммарным воздействием всех источников излучения.

Если дополнительно совокупность приёмников выполняет приём отраженного сигнала, то к комплексным огибающим приёмников дополнительно примешивается и
воздействие отраженного сигнала, которое так же имеет случайный, но направленный характер.

Анализируя комплексные огибающие всех приёмников, необходимо решить следующие задачи:
\begin{itemize}
    \item обнаружения - определить наличие источников излучения или оценить их количество,
    \item пеленгации - определить направления на источники излучения,
    \item адаптации - уменьшить воздействие источников излучения.
\end{itemize}


\section{Приёмники}

Рассмотрим антенную решётку, состоящую из $n$ приемников, расположенных на одной прямой через равные расстояния (эквидистантная решётка). Состояние каждого
приёмника в фиксированный момент времени определяется комплексной огибающей $x_k$, а состояние всей антенной решётки определяется вектором $X$:
\[
    X =
    \begin{pmatrix}
        x_1   \\
        \dots \\
        x_n
    \end{pmatrix}
    .
\]


\section{Отсутствие источников излучения}

Если источники излучения отсутствуют, то комплексные огибающие приёмников определяются их внутренними шумами:
\[
    X = E,
\]
где
\[
    E =
    \begin{pmatrix}
        e_1   \\
        \dots \\
        e_n
    \end{pmatrix}
\]
--- случайный вектор, компоненты которого $e_k$ --- комплексные случайные величины:
\begin{gather*}
    \expectation{
        \begin{pmatrix}
            \real{e_k} \\ \image{e_k}
        \end{pmatrix}
    } =
    \begin{pmatrix}
        0 \\
        0
    \end{pmatrix} , \\
    %
    \variance{
        \begin{pmatrix}
            \real{e_k} \\ \image{e_k}
        \end{pmatrix}
    } =
    \begin{pmatrix}
        \frac{1}{2} \sigma_0^2 & 0                      \\
        0                      & \frac{1}{2} \sigma_0^2
    \end{pmatrix} ,
\end{gather*}
где $\sigma_0^2$ --- мощность собственных шумов. Отсюда математическое ожидание
\[
    \expectation{e_k}
    = \expectation{\real{e_k} + i \image{e_k}}
    = \expectation{\real{e_k}} + i \expectation{\image{e_k}}
    = 0 + i \cdot 0
    = 0
\]
и дисперсия
\begin{multline*}
    \variance{e_k}
    = \expectation{\left ( e_k - \expectation{e_k} \right ) \left ( e_k - \expectation{e_k} \right )^H}
    = \expectation{e_k e_k^H} = \\
    %
    = \expectation{\left ( \real{e_k} + i \image{e_k} \right ) \left ( \real{e_k} - i \image{e_k} \right )}
    = \expectation{\left ( \real{e_k} \right )^2 + \left ( \image{e_k} \right )^2} = \\
    %
    = \expectation{\left ( \real{e_k} \right )^2} + \expectation{\left ( \image{e_k} \right )^2}
    = \frac{1}{2} \sigma_0^2 + \frac{1}{2} \sigma_0^2
    = \sigma_0^2 .
\end{multline*}
Таким образом, для вектора $E$ математическое ожидание:
\[
    \expectation{E}
    = \begin{pmatrix}
          0 \\
          \dots \\
          0
    \end{pmatrix} .
\]
Будем считать, что величины $e_k$ некоррелированы, тогда ковариационная матрица
\[
    \variance{E}
    = \expectation{\left ( E - \expectation{E} \right ) \left ( E - \expectation{E} \right )^H}
    = \expectation{E E^H}
    = \begin{pmatrix}
          \sigma_0^2 & 0          & \dots  & 0          \\
          0          & \sigma_0^2 & \dots  & 0          \\
          \vdots     & \vdots     & \ddots & \vdots     \\
          0          & 0          & \dots  & \sigma_0^2
    \end{pmatrix}
    = \sigma_0^2 I_n ,
\]
где $I_n$ --- единичная матрица.

Вектор $X$ имеет такие же характеристики, что и вектор $E$:
\begin{gather*}
    \expectation{X} = 0, \\
    \variance{X} = \sigma_0^2 I_n ,
\end{gather*}
при этом спектр ковариационной матрицы состоит из одного значения:
\[
    \spectrum{\variance{X}} = \set{\sigma_0^2}.
\]


\section{Один источник излучения}

\subsection{Состояние приёмников}

Пусть с некоторого направления производится излучение сигнала одним источником излучения, и на приёмники падает плоская волна. Пусть $\alpha$
угол между осью ординат и волновым вектором, $\lambda$ --- длина волны сигнала источника излучения и $d$ --- шаг расстановки приёмников, тогда расстояние между
приёмником с номером $k$ и первым приёмником равно $(k-1) d$, а смещение фазы в $k$-ом приёмнике:
\begin{gather}
    \Delta \varphi_k
    = (k-1) \cdot \Delta \varphi, \notag \\
    %
    \Delta \varphi
    = 2 \pi \frac{d}{\lambda} \sin \alpha
    \label{jammers:single:phase_shift}
\end{gather}
где $\Delta \varphi$ --- изменение фазы между соседними приёмниками.

Пусть в первом приёмнике комплексная огибающая сигнала источника излучения равна $s_1$ (в некоторый фиксированный момент времени), тогда из-за смещения фазы в
$k$-ом приёмнике на величину $(k-1) \Delta \varphi$, комплексная огибающая сигнала источника излучения будет равна $s_1 e^{i (k-1) \Delta \varphi}$. Если все
множители $e^{i (k-1) \Delta \varphi}$ собрать в вектор $\breve{X}_1$:
\begin{equation}
    \label{jammers:single:direction}
    \breve{X}_1
    =
    \begin{pmatrix}
        1                      \\
        e^{i \Delta \varphi}   \\
        e^{i 2 \Delta \varphi} \\
        \dots                  \\
        e^{i (n-1) \Delta \varphi}
    \end{pmatrix} ,
\end{equation}
тогда вектор комплексных огибающих сигнала источника излучения для всех приёмников будет иметь простой вид --- $s_1 \breve{X}_1$.

Поскольку на принимаемый сигнал источника излучения накладываются собственные шумы приёмников, то вектор комплексных огибающих приёмников будет иметь вид суммы:
\begin{equation}
    \label{jammers:single:state}
    X = E + s_1 \breve{X}_1 .
\end{equation}
В произведении $s_1 \breve{X}_1$ можно выделить случайную составляющую в виде комплексной огибающей $s_1$ и регулярную составляющую в виде неслучайного вектора
$\breve{X}_1$, который определяется направлением на источник излучения. Таким образом, при наличии источника излучения в состоянии $X$:
\begin{enumerate}
    \item возникает смещение на случайную величину $s_1$,
    \item появляется регулярная структура, задаваемая вектором $\breve{X}_1$, которая определяется местоположением источника излучения.
\end{enumerate}
Комплексная огибающая $s_1$ является комплексной случайной величиной, для которой:
\begin{gather*}
    \expectation{
        \begin{pmatrix}
            \real{s_1} \\ \image{s_1}
        \end{pmatrix}
    } =
    \begin{pmatrix}
        0 \\
        0
    \end{pmatrix} , \\
    %
    \variance{
        \begin{pmatrix}
            \real{s_1} \\ \image{s_1}
        \end{pmatrix}
    } =
    \begin{pmatrix}
        \frac{1}{2} \sigma_1^2 & 0                      \\
        0                      & \frac{1}{2} \sigma_1^2
    \end{pmatrix} .
\end{gather*}
Дополнительно считается, что $s_1$ некоррелированна с величинами комплексных огибающих собственных шумов $e_k$:
\begin{gather*}
    \covariance{s_1}{e_l} = 0 , \\
    l = \overline{1, n}.
\end{gather*}

Характеристики состояния $X$ извеняются при наличии источника излучения и, в соответствии с представлением \eqref{jammers:single:state}, математическое ожидание
$X$:
\[
    \expectation{X}
    = \expectation{E} + \expectation{s_k} \breve{X}_k
    = 0 + 0 \cdot \breve{X}_k
    = 0.
\]
и ковариационная матрица $X$:
\begin{multline*}
    \variance{X}
    = \expectation{\left ( X - \expectation{X} \right ) \left ( X - \expectation{X} \right )^H}
    = \expectation{X X^H} = \\
    %
    = \expectation{\left ( E + s_1 \breve{X}_1 \right ) \left ( E + s_1 \breve{X}_1 \right )^H} = \\
    %
    = \expectation{E E^H + E \left ( s_1 \breve{X}_1 \right ) + \left ( s_1 \breve{X}_1 \right ) E^H + \left ( s_1 \breve{X}_1 \right ) \left ( s_1 \breve{X}_1 \right )^H} = \\
    %
    = \expectation{E E^H} + \expectation{E s_1} \breve{X}_1 + \expectation{s_1 E^H} \breve{X}_1 + \expectation{s_1 \overline{s}_1} \breve{X}_1 \breve{X}_1^H = \\
    %
    = \sigma_0^2 I_n + 0 + 0 + \expectation{\modulus{s_1}^2} \breve{X}_1 \breve{X}_1^H
    = \sigma_0^2 I_n + \sigma_1^2 \breve{X}_1 \breve{X}_1^H
    .
\end{multline*}

\subsection{Обнаружение и пеленгация}

Заметим, что при появлении источника излучения в ковариационной матрице $\variance{X}$ появилось слагаемое $\sigma_1^2 \breve{X}_1 \breve{X}_1^H$:
\[
    \sigma_0^2 I_n \rightarrow \sigma_0^2 I_n + \sigma_1^2 \breve{X}_1 \breve{X}_1^H ,
\]
которое изменило спектр матрицы и набор собственных векторов. Одним из собственных векторов является вектор направления $\breve{X}_1$:
\begin{multline*}
    \variance{X} \breve{X}_1
    = \left ( \sigma_0^2 I_n + \sigma_1^2 \breve{X}_1 \breve{X}_1^H \right ) \breve{X}_1
    = \sigma_0^2 I_n \breve{X}_1 + \sigma_1^2 \breve{X}_1 \breve{X}_1^H \breve{X}_1 = \\
    %
    = \sigma_0^2 \breve{X}_1 + \sigma_1^2 \left ( \breve{X}_1^H \breve{X}_1 \right ) \breve{X}_1
    = \left ( \sigma_0^2 + \sigma_1^2 \breve{X}_1^H \breve{X}_1 \right ) \breve{X}_1 ,
\end{multline*}
где в соответствии с определением \eqref{jammers:single:direction} вектора направления $\breve{X}_1$:
\begin{equation}
    \label{jammers:single:direction_self_product}
    \breve{X}_1^H \breve{X}_1
    = \sum_{k=0}^{n-1} \overline{e^{i k \Delta \varphi}} \cdot e^{i k \Delta \varphi}
    = \sum_{k=0}^{n-1} \modulus{e^{i k \Delta \varphi}}
    = \sum_{k=0}^{n-1} 1
    = n ,
\end{equation}
поэтому
\[
    \variance{X} \breve{X}_1
    = \left ( \sigma_0^2 + \sigma_1^2 \breve{X}_1^H \breve{X}_1 \right ) \breve{X}_1
    = \left ( \sigma_0^2 + \sigma_1^2 n \right ) \breve{X}_1 .
\]
Таким образом, вектор $\breve{X}_1$ соответствует собственному значению $\sigma_0^2 + \sigma_1^2 n$.

Другим собственным значением является $\sigma_0^2$, поскольку для любого вектора $Y \perp \breve{X}_1$, то есть $\breve{X}_1^H Y = 0$:
\[
    \variance{X} Y
    = \left ( \sigma_0^2 I_n + \sigma_1^2 \breve{X}_1 \breve{X}_1^H \right ) Y
    = \sigma_0^2 I_n Y + \sigma_1^2 \breve{X}_1 \breve{X}_1^H Y
    = \sigma_0^2 I_n Y + \sigma_1^2 \breve{X}_1 \cdot 0
    = \sigma_0^2 Y .
\]
Таким образом, при наличии источника излучения спектр ковариационной матрицы $\variance{X}$:
\[
    \spectrum{\variance{X}} = \set{\sigma_0^2, \sigma_0^2 + \sigma_1^2 n}
\]
\textcolor{red}{Почему нет других собственных значений?}

Если выполняется условие
\begin{equation}
    \label{jammers:single:powers_relation}
    \sigma_1^2 n \gg \sigma_0^2 ,
\end{equation}
то можно сформировать правило обнаружения:
\begin{enumerate}
    \item вычислить наибольшее собственное значение ковариационной матрицы $\variance{X}$:
    \[
        \sigma_0^2 + \sigma_1^2 n = \max \spectrum{\variance{X}}
    \]
    \item сравнить величины $\sigma_0^2 + \sigma_1^2 n$ и $\sigma_0^2$, если первая величина существенно больше второй, то принять решение о наличии источника
    излучения, в противном случае считать, что источник излучения отсутствует.
\end{enumerate}

Условие \eqref{jammers:single:powers_relation}, при котором появляется возможность формирования процедуры обнаружения, выполняется, если, например, мощность
сигнала источника излучения $\sigma_1^2$ существенно больше мощности собственных шумов $\sigma_0^2$:
\[
    \sigma_1^2 \gg \sigma_0^2 ,
\]
но даже если это условие не выполняется, то есть сигнал источника излучения имеет малую мощность, то можно набрать достаточно большое количество приёмников $n$
для выполнения условия \eqref{jammers:single:powers_relation}.

Для решения задачи пеленгации необходимо найти собственный вектор, соответствующий собственному значению $\sigma_0^2 + \sigma_1^2 n$, таким собственным вектором
является вектор $c \cdot \breve{X}_1$ ($c \in \mathbb{C}$). Используя этот вектор, точнее его вторую компоненту $c \cdot e^{i \Delta \varphi}$, можно найти смещение фазы
$\Delta \varphi$ и вычислить угол $\alpha$ между нормалью решётки и направлением на источник излучения из равенства \eqref{jammers:single:phase_shift}:
\begin{gather}
    \Delta \varphi = \arg \left ( c \cdot e^{i \Delta \varphi} \right ), \notag \\
    \sin \alpha = \frac{2 \pi}{\Delta \varphi} \cdot \frac{\lambda}{d} \label{jammers:single:angle}.
\end{gather}


\section{Несколько источников излучения}

При наличии $m$ источников излучения, где $1 \le m < n$, будем считать, что направления действия источников различны: углы $\alpha_k$ между нормалью решётки и направлениями
на источники излучения являются различными, отсюда различными являются и смещения фаз $\Delta \varphi_k$.

\subsection{Состояние приёмников}

У каждого $k$-го источника свои угол $\alpha_k$, определяемый направлением на источник излучения, сдвиг фазы $\Delta \varphi_k$, комплексная огибающая $s_k$ и
вектор направления $\breve{X}_k$:
\[
    \breve{X}_k =
    \begin{pmatrix}
        1                        \\
        e^{i \Delta \varphi_k}   \\
        e^{i 2 \Delta \varphi_k} \\
        \dots                    \\
        e^{i (n-1) \Delta \varphi_k}
    \end{pmatrix}
\]
При одновременном приёме сигналов от всех источников в каждом $l$-ом приёмнике комплексные огибающие складываются и на сумму накладывается собственный шум приёмника:
\[
    e_l + s_1 e^{i l \Delta \varphi_1} + s_2 e^{i l \Delta \varphi_2} + \dots + s_m e^{i l \Delta \varphi_m}
\]
Общее состояние всех приёмников задается суммой:
\[
    X = E + s_1 \breve{X}_1 + \dots + s_m \breve{X}_m ,
\]
в которой комплексные огибающие $s_k$ являются случайными величинами:
\begin{gather*}
    \expectation{
        \begin{pmatrix}
            \real{s_k} \\ \image{s_k}
        \end{pmatrix}
    } =
    \begin{pmatrix}
        0 \\
        0
    \end{pmatrix} , \\
    %
    \variance{
        \begin{pmatrix}
            \real{s_k} \\ \image{s_k}
        \end{pmatrix}
    } =
    \begin{pmatrix}
        \frac{1}{2} \sigma_1^k & 0                      \\
        0                      & \frac{1}{2} \sigma_1^k
    \end{pmatrix} ,
\end{gather*}
и величины $s_1$, \dots, $s_m$ считаются некоррелированными:
\begin{gather*}
    \covariance{s_k}{s_j} = 0 , \\
    \covariance{s_k}{e_l} = 0 , \\
    k,j = \overline{1,m}, k \neq j, \\
    l = \overline{1, n} .
\end{gather*}

Математическое ожидание вектора состояния приёмников $X$:
\[
    \expectation{X}
    = \expectation{E} + \sum_{k=1}^m \expectation{s_k} \breve{X}_k
    = 0 + \sum_{k=1}^m 0 \cdot \breve{X}_k
    = 0.
\]

Ковариационная матрица $X$:
\begin{multline*}
    \variance{X}
    = \expectation{\left ( X - \expectation{X} \right ) \left ( X - \expectation{X} \right )^H}
    = \expectation{X X^H} = \\
    %
    = \expectation{\left ( E + \sum_{k=1}^m s_k \breve{X}_k \right ) \left ( E + \sum_{k=1}^m s_k \breve{X}_k \right )^H} = \\
    %
    = \expectation{E E^H + E \left ( \sum_{k=1}^m s_k \breve{X}_k \right ) + \left ( \sum_{k=1}^m s_k \breve{X}_k \right ) E^H + \left ( \sum_{k=1}^m s_k \breve{X}_k \right ) \left ( \sum_{k=1}^m s_k \breve{X}_k \right )^H} = \\
    %
    = \expectation{E E^H} + \sum_{k=1}^m \expectation{E s_k} \breve{X}_k + \sum_{k=1}^m \expectation{s_k E^H} \breve{X}_k + \sum_{k=1}^m \expectation{s_k \overline{s}_k} \breve{X}_k \breve{X}_k^H = \\
    %
    = \sigma_0^2 I_n + \sum_{k=1}^m \expectation{\modulus{s_k}^2} \breve{X}_k \breve{X}_k^H
    = \sigma_0^2 I_n + \sum_{k=1}^m \sigma_k^2 \breve{X}_k \breve{X}_k^H
\end{multline*}
или в матричной форме
\[
    \variance{X} = \sigma_0^2 I_n + \breve{X} S \breve{X}^H , \\
\]
где
\begin{gather*}
    S =
    \begin{pmatrix}
        \sigma_1^2 & 0          & \dots  & 0          \\
        0          & \sigma_2^2 & \dots  & 0          \\
        \vdots     & \vdots     & \ddots & \vdots     \\
        0          & 0          & \dots  & \sigma_m^2
    \end{pmatrix}, \\
    %
    \breve{X} =
    \begin{pmatrix}
        \breve{X}_1 & \dots & \breve{X}_m
    \end{pmatrix} .
\end{gather*}

\subsection{Обнаружение}

Как и в случае одного источника излучения ковариационная матрица $\variance{X}$ вектора состояния $X$ изменяется аналогичным образом:
\[
    \sigma_0^2 I_n \rightarrow \sigma_0^2 I_n + \breve{X} S \breve{X}^H
\]
при этом также изменяется и спектр ковариационной матрицы $\variance{X}$.

Заметим, что векторы $\breve{X}_1$, \dots, $\breve{X}_m$ являются линейно независимыми, поскольку все сдвиги фаз $\Delta \varphi_k$ различны, поэтому ранг матрицы
$\breve{X} S \breve{X}^H$ равен количеству векторов $m$:
\[
    \rank{\breve{X} S \breve{X}^H} = m
\]
и в спектре матрицы $\breve{X} S \breve{X}^H$ есть $m$ ненулевых собственных значений $\lambda_1$, \dots, $\lambda_m$ и нулевое значение, поскольку
$m < n$:
\begin{gather*}
    \spectrum{\breve{X} S \breve{X}^H} = \set{0, \lambda_1, \dots, \lambda_m}, \\
    \lambda_k \neq 0, \\
    \lambda_1 \le \dots \le \lambda_m .
\end{gather*}
Поскольку матрица $\breve{X} S \breve{X}^H$ является неотрицательно определённой:
\[
    \breve{X} S \breve{X}^H \ge 0 ,
\]
то собственные значения также неотрицательны:
\[
    \lambda_k \ge 0,
\]
а поскольку собственные значения отличны от нуля, то:
\[
    0 < \lambda_1 \le \dots \le \lambda_m
\]
\textcolor{red}{и при этом различны}:
\[
    0 < \lambda_1 < \dots < \lambda_m .
\]

Добавление матрицы $\sigma_0^2 I_n$ приводит к смещению спектра на величину $\sigma_0^2$:
\[
    \spectrum{\variance{X}}
    = \spectrum{\sigma_0^2 I_n + \breve{X} S \breve{X}^H}
    = \set{\sigma_0^2, \sigma_0^2 + \lambda_1, \dots, \sigma_0^2 + \lambda_m}
\]

\textcolor{red}{Как связаны величины $\lambda_1$, \dots, $\lambda_m$ и $\sigma_1^2$, \dots, $\sigma_m^2$?}

Поскольку $\lambda_1 > 0$, то можно сформулировать правило обнаружения источников излучения и определения их количества:
\begin{enumerate}
    \item вычислить наибольшее собственное значение $\lambda_{max}$ ковариационной матрицы $\variance{X}$,
    \item сравнить $\lambda_{max}$ с $\sigma_0^2$, если $\lambda_{max} > \sigma_0^2$, то принять решение о наличии источников излучения, в противном случае
    считать, что источники излучения осутствуют,
    \item при наличии источников излучения последовательно находить собственные значения $\sigma_0^2 + \lambda_k$ до тех пор пока не будет получено собственное
    значение $\sigma_0^2$ для определения количества источников излучения.
\end{enumerate}

\subsection{Пеленгация}

Предварительно рассмотрим простой случай, в котором векторы направлений $\breve{X}_1$, \dots, $\breve{X}_m$ являются почти ортогональными:
\begin{gather*}
    X_k^H X_k = n, \\
    X_k^H X_j = \delta_{kj}, \\
    \modulus{\delta_{kj}} < \delta \ll n, \\
    k \neq j .
\end{gather*}
На практике векторы направлений $\breve{X}_k$ почти ортогональны, если направления на источники излучения в достаточной мере разнесены по углу, то есть являются
различимыми. Угловая мера различимости зависит от количества приёмников $n$.

Векторы направлений $\breve{X}_j$ являются близкими к собственным векторам ковариационной матрицы $\variance{X}$:
\begin{multline*}
    \variance{X} \breve{X}_j
    = \left ( \sigma_0^2 I_n + \breve{X} S \breve{X}^H \right ) \breve{X_j}
    = \left ( \sigma_0^2 I_n + \sum_{k=1}^m \sigma_k^2 \breve{X}_k \breve{X}_k^H \right ) \breve{X}_j
    = \sigma_0^2 \breve{X}_j + \sum_{k=1}^m \sigma_k^2 \breve{X}_k \breve{X}_k^H \breve{X}_j = \\
    %
    = \sigma_0^2 \breve{X}_j + \left ( \sigma_j^2 \breve{X}_j^H \breve{X}_j \right ) \breve{X}_j + \sum_{k \neq j} \left ( \sigma_k^2 \breve{X}_k^H \breve{X}_j \right ) \breve{X}_k = \\
    %
    = \left ( \sigma_0^2 + \sigma_j^2 \breve{X}_j^H \breve{X}_j \right ) \breve{X}_j + \sum_{k \neq j} \left ( \sigma_k^2 \breve{X}_k^H \breve{X}_j \right ) \breve{X}_k
    = \left ( \sigma_0^2 + \sigma_j^2 n \right ) \breve{X}_j + \sum_{k \neq j} \sigma_k^2 \delta_{kj} \breve{X}_k ,
\end{multline*}
тогда и квадратичная форма:
\begin{multline*}
    X_j^H \variance{X} \breve{X}_j
    = X_j^H \left ( \left ( \sigma_0^2 + \sigma_j^2 n \right ) \breve{X}_j + \sum_{k \neq j} \sigma_k^2 \delta_{kj} \breve{X}_k \right ) = \\
    %
    = \left ( \sigma_0^2 + \sigma_j^2 n \right ) X_j^H \breve{X}_j + \sum_{k \neq j} \sigma_k^2 \delta_{kj} X_j^H \breve{X}_k
    = \left ( \sigma_0^2 + \sigma_j^2 n \right ) n + \sum_{k \neq j} \sigma_k^2 \delta_{kj} \overline{\delta_{kj}} = \\
    %
    = \sigma_0^2 n + \sigma_j^2 n^2 + \sum_{k \neq j} \sigma_k^2 \modulus{\delta_{kj}}^2
    < \sigma_0^2 n + \sigma_j^2 n^2 + (m-1) \modulus{\delta}^2 .
\end{multline*}
В некотором приближении можно считать, что
\begin{gather*}
    X_j^H \variance{X} \breve{X}_j \approx \sigma_0^2 n + \sigma_j^2 n^2 , \\
    j = \overline{1,m} ,
\end{gather*}
и оказывается, что значения в правой части являются близкими к локальным максимумам квадратичной формы с
ковариационной матрицей $\variance{X}$. Если взять произвольный вектор направления $\breve{Y}(\Delta \varphi)$ как функцию смещения фазы $\Delta \varphi$:
\[
    \breve{Y}(\Delta \varphi) =
    \begin{pmatrix}
        1                      \\
        e^{i \Delta \varphi}   \\
        e^{i 2 \Delta \varphi} \\
        \dots                  \\
        e^{i (n-1) \Delta \varphi}
    \end{pmatrix} ,
\]
то квадратичная форма
\begin{multline}
    \label{jammers:multiple:quadric}
    \breve{Y}^H \variance{X} \breve{Y}
    = \breve{Y}^H \left ( \sigma_0^2 I_n + \breve{X} S \breve{X}^H \right ) \breve{Y}
    = \breve{Y}^H \left ( \sigma_0^2 I_n + \sum_{k=1}^m \sigma_k^2 \breve{X}_k \breve{X}_k^H \right ) \breve{Y} = \\
    %
    = \sigma_0^2 \breve{Y}^H \breve{Y} + \sum_{k=1}^m \sigma_k^2 \breve{Y}^H \breve{X}_k \breve{X}_k^H \breve{Y}
    = \sigma_0^2 n + \sum_{k=1}^m \sigma_k^2 \breve{Y}^H \breve{X}_k \left ( \breve{Y}^H \breve{X}_k \right )^H = \\
    %
    = \sigma_0^2 n + \sum_{k=1}^m \sigma_k^2 \modulus{\breve{Y}^H \breve{X}_k}^2 ,
\end{multline}
где произведения
\begin{multline*}
    \breve{Y}^H \breve{X}_k
    =
    \begin{pmatrix}
        1                       &
        e^{- i \Delta \varphi}   &
        e^{- i 2 \Delta \varphi} &
        \dots                   &
        e^{- i (n-1) \Delta \varphi}
    \end{pmatrix}
    \begin{pmatrix}
        1                        \\
        e^{i \Delta \varphi_k}   \\
        e^{i 2 \Delta \varphi_k} \\
        \dots                    \\
        e^{i (n-1) \Delta \varphi_k}
    \end{pmatrix}
    = \\
    %
    = 1 + e^{i (\Delta \varphi_k - \Delta \varphi )} + e^{i 2 (\Delta \varphi_k - \Delta \varphi )} + \dots + e^{i (n-1) (\Delta \varphi_k - \Delta \varphi )} .
\end{multline*}
Представление о полученной сумме и её модуле можно получить, рассматривая сумму чисел вида $e^{i l (\Delta \varphi_k - \Delta \varphi )}$ на комплексной
плоскости.

\textcolor{red}{Приложение на Matlab.}

Модуль $\modulus{Y^H X_k}$ как функция смещения фазы $\Delta \varphi$, определяющей вектор $Y$, представляет собой осцилирующую "убывающую"{} функцию с пиком
в точке $\Delta \varphi = \Delta \varphi_k$, соответствующей вектору направлению $\breve{X}_k$. Сумма в квадратичной форме \eqref{jammers:multiple:quadric}
представляет собой суперпозицию функций-модулей $\modulus{Y^H X_k}$, которая в случае достаточного разделения векторов направлений $\breve{X}_k$ имеет локальные
максимумы в окрестности векторов направлений $\breve{X}_k$ и достигает значений, примерно равных $\sigma_0^2 n + \sigma_k^2 n^2$.

\textcolor{red}{Рисунок с разделёнными локальными максимумами.}

Если некоторые векторы направлений $\breve{X}_k$ оказываются "близкими"{}, то возникает один локальный максимум квадратичной формы в окрестности
"близких"{} векторов направлений.

\textcolor{red}{Рисунок с объединённым локальным максимумом.}

Таким образом, нахождение смещений фазы $\Delta \varphi$ и векторов направлений $Y$, при которых наблюдается локальный максимум квадратичной формы
$Y^H \variance{X} Y$, позволяет находить направления "близкие"{} к направлениям на источники излучений $\breve{X}_k$, причём ошибка в определении направлений
становится тем меньше, чем больше количество приёмников.

\section{Альтернативная пеленгация}

Существует несколько альтернативных методов определения направлений на источники излучения, связанных с поиском экстремумов функций от направления, один из которых
связан с поиском максимумов квадратичной формы с обратной матрицей $\variance{X}^{-1}$:
\[
    Y^H \variance{X}^{-1} Y \rightarrow \max
\]
Собственные векторы обратной матрицы $\variance{X}^{-1}$ совпадают с собственными векторами ковариационной матрицы $\variance{X}$, а собственные значения
обратной матрицы $\variance{X}^{-1}$ являются обратными собственным значениям ковариационной матрицы $\variance{X}$. Если в направлении $\breve{X}_j$
наблюдались приближенные значения:
\[
    X_j^H \variance{X} \breve{X}_j \approx \sigma_0^2 n + \sigma_j^2 n^2 , \\
\]
то в том же направлении будут наблюдаться приближенно обратные значения \textcolor{red}{(это нужно проверить)}:
\[
    X_j^H \variance{X}^{-1} \breve{X}_j \approx \frac{1}{\sigma_0^2 n + \sigma_j^2 n^2} ,
\]
