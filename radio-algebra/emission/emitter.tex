\chapter{Излучатель}


\section{Модель излучателя}

Рассматривается излучатель с двумя каналами поляризации и двумя входами, соответствующим этим двум каналам. Возникающие физические явления описываются только
линейными выражениями, а нелинейные эффекты не учитываются.

Пусть $a$ --- вектор огибающих входов излучателя:
\[
    a
    = \begin{pmatrix}
          a_1 \\
          a_2
    \end{pmatrix} .
\]

В результате отражений и связи между каналами, наводятся отраженные сигналы с огибающими $b$:
\begin{gather*}
    b
    = \begin{pmatrix}
          b_1 \\
          b_2
    \end{pmatrix}
    = \begin{pmatrix}
          s_{11} a_1 + s_{12} a_2 \\
          s_{21} a_1 + s_{22} a_2
    \end{pmatrix}
    = S a, \\
    %
    S
    = \begin{pmatrix}
          s_{11} & s_{12} \\
          s_{21} & s_{22}
    \end{pmatrix},
\end{gather*}
где $S$ --- матрица рассеяния. Неотраженные части сигналов формируют электромагнитные волны, электрические колебания которых характеризуются огибающими $e$:
\begin{gather*}
    e
    = \begin{pmatrix}
          e_\theta \\
          e_\varphi
    \end{pmatrix}
    = \begin{pmatrix}
          t_{1, \theta} a_1 + t_{2, \theta} a_2 \\
          t_{1, \varphi} a_1 + t_{2, \varphi} a_2
    \end{pmatrix}
    = T a, \\
    %
    T
    = \begin{pmatrix}
          t_{1, \theta}  & t_{2, \theta}  \\
          t_{1, \varphi} & t_{2, \varphi}
    \end{pmatrix} .
\end{gather*}
где $T$ --- матрица коэффициентов передачи со входов в волноводы.


\section{Диаграмма направленности}

Из волноводов происходит излучение электромагнитной волны в пространство, излучение является неоднородным и его характеристики зависят от рассматриваемого направления.
Если с излучателем связана сферическая система координат (с началом отсчёта в конце волноводов), то направление можно задать с помощью волнового вектора $w$,
длина которого:
\[
    \modulus{w} = \frac{2 \pi}{\lambda},
\]
где $\lambda$ --- длина волны на несущей частоте. В выбранном направлении $w$ на расстоянии $R$ напряженность электрического поля $E$ является функцией
направления $w$, расстояния $R$ и огибающих волн $e$ в волноводах излучателя, которые зависят от огибающих $a$ сигналов на входах:
\[
    E = E(w, R, a).
\]
В дальней зоне при больших расстояниях $R$ напряженность $E$ можно приближенно представить линейной частью по огибающим сигналов $a$:
\begin{equation} \label{emission:emitter:diagram:tension}
    E(w,R)
    \approx F(w) a \cdot \frac{e^{i \modulus{w} R}}{R} ,
\end{equation}
где множитель $e^{i \modulus{w} R}$ определяет смещение по фазе на расстоянии $R$ от начала отсчёта, множитель $\frac{1}{R}$ показывает затухание амплитуды
вектора напряженности $E$, и $F(w)$ --- диаграмма направленности излучателя:
\[
    F(w)
    = \begin{pmatrix}
          f_{1, \theta}(w)  & f_{2, \theta}(w)  \\
          f_{1, \varphi}(w) & f_{2, \varphi}(w)
    \end{pmatrix} ,
\]
в которой столбцы задают парциальные диаграммы направленности по каналам поляризации:
\begin{equation}
    f_1(w)
    = \begin{pmatrix}
          f_{1,\theta}(w) \\
          f_{1,\varphi}(w)
    \end{pmatrix}
    , \;
    f_2(w)
    = \begin{pmatrix}
          f_{2,\theta}(w) \\
          f_{2,\varphi}(w)
    \end{pmatrix}
    \label{emission:emitter:diagram:partial}
    .
\end{equation}


\section{Энергетическое ограничение}

Для излучателя выполняется закон сохранения энергии --- суммарная мощность входных сигналов $P_{inp}$ совпадает с суммой мощности отраженных сигналов $P_{ref}$,
мощности излучения $P_{rad}$ и мощности $P_{dis}$ дисспативных потерь:
\begin{equation}
    \label{emission:emitter:power:equality}
    P_{ref} + P_{rad} + P_{dis} = P_{inp} ,
\end{equation}
где
\begin{align}
    P_{inp} & = \norm{a}^2 = a^* a , \label{emission:emitter:power:input}\\
    P_{ref} & = \norm{b}^2 = \norm{S a} = a^* S^* S a \notag,
\end{align}
и выражение для мощности излучения $P_{rad}$ имеет вид:
\begin{gather*}
    P_{rad}
    = \iiint \limits_S \norm{E(w)}^2 ds
    = \iint \limits_{4 \pi} \norm{F(w) a}^2 d \Omega , \\
    %
    \norm{F(w) a}^2
    = a^* F^*(w) F(w) a .
\end{gather*}
Таким образом, мощность излучения $P_{rad}$ можно представить в виде квадратичной формы:
\begin{equation}
    \label{emission:emitter:power:radiated}
    P_{rad}
    = a^* Q a ,
\end{equation}
где элементами матрицы $Q$ являются интегралы вида:
\[
    Q_{jk} = \iint \limits_{4 \pi} f_j^*(w) f_k(w) d \Omega .
\]

Таким образом, равенство \eqref{emission:emitter:power:equality} можно представить в виде:
\begin{gather*}
    a^* S^* S a + a^* Q a + P_{dis} = a^* a , \\
    a^* Q a = a^* a - a^* S^* S a - P_{dis}, \\
    a^* Q a = a^* ( I - S^* S ) a - P_{dis} .
\end{gather*}
Откуда
\begin{gather*} \label{emission:emitter:power:inequality}
    a^* Q a \le a^* ( I - S^* S ) a \le a^* a.
\end{gather*}


\section{Коэффициент усиления}

Реализованный коэффициент усиления $G(w)$ определяет относительную величину плотности потока $\Pi(\theta, \varphi)$ мощности в дальней зоне в
направлении $w$:
\[
    G(w)
    = \frac{4 \pi R^2}{P_{inp}} \cdot \Pi(w)
    = \frac{4 \pi R^2}{P_{inp}} \cdot \frac{1}{Z_0} \norm{E(w)}^2
    = \frac{4 \pi R^2}{P_{inp}} \cdot \frac{1}{Z_0} \frac{\norm{F(w) a}^2}{R^2}
    = \frac{4 \pi}{Z_0} \cdot \frac{\norm{F(w) a}^2}{P_{inp}} ,
\]
где $Z_0 = 120 \pi$ --- волновое сопротивление свободного пространства.

Пусть направление $w$ фиксировано, тогда коэффициент усиления пропорционален отношению:
\[
    \rho(a)
    = \frac{\norm{F(w) a}^2}{P_{inp}}
    = \frac{a^* F^*(w) F(w) a}{a^* a}
\]
и возникает вопрос, каким образом нужно сформировать огибающие входных сигналов $a$ чтобы отношение $\rho(a)$ и коэффициент усиления в заданном направлении $\vec{w}$
оказался наибольшим?

Отношение $\rho(a)$ является отношением Релея: числитель --- квадратичная форма с эрмитовой матрией $F^*(w) F(w)$, знаменатель --- квадрат нормы $a$.
Наибольшее значение $G_{max}$:
\[
    G_{max} = \max \limits_{a} G
\]
достигается в направлении $a_{max}$, совпадающим с направлением собственных векторов, соответствующих наибольшему собственному значению $\lambda_{max}$.
Вектор $a_{max}$ и число $\lambda_{max}$ удовлетворяют уравнению:
\[
    F^*(\vec{w}) F(\vec{w}) a_{max} = \lambda_{max} a_{max}
\]
Домножим левую и правую части уравнения на $F(\vec{w})$ слева:
\[
    F(\vec{w}) F^*(\vec{w}) F(\vec{w}) a_{max} = \lambda_{max} F(\vec{w}) a_{max} .
\]
Объединяя два уравнения получим систему:
\begin{gather}
    \left \{
    \begin{array}{c}
        F^*(\vec{w}) F(\vec{w}) a_{max} = \lambda_{max} a_{max} \\
        F(\vec{w}) F^*(\vec{w}) p_{max} = \lambda_{max} p_{max}
    \end{array}
    \right .
    \label{emission:emitter:gain:system}, \\
    p_{max} = F(\vec{w}) a_{max} \notag.
\end{gather}

Пусть
\[
    B
    = F(\vec{w}) F^*(\vec{w})
    = \begin{pmatrix}
          f_\theta f_\theta^*  & f_{\theta} f_\varphi^* \\
          f_\varphi f_\theta^* & f_\varphi f_\varphi^*
    \end{pmatrix} ,
\]
где $f_\theta$ и $f_\varphi$ --- строки матрицы $F(\vec{w})$:
\begin{gather*}
    f_\theta
    = \begin{pmatrix}
          \vec{f}_{1,\theta}(\vec{w}) & \vec{f}_{2,\theta}(\vec{w})
    \end{pmatrix}, \\
    %
    f_\varphi
    = \begin{pmatrix}
          \vec{f}_{1,\varphi}(\vec{w}) & \vec{f}_{n,\varphi}(\vec{w})
    \end{pmatrix} .
\end{gather*}

У матрицы $B$ два собственных значения $\lambda_{min}$ и $\lambda_{max}$, которые являются корнями характеристического уравнения:
\begin{multline*}
    \begin{vmatrix}
        f_\theta f_\theta^* - \lambda & f_{\theta} f_\varphi^*          \\
        f_\varphi f_\theta^*          & f_\varphi f_\varphi^* - \lambda
    \end{vmatrix}
    = (f_\theta f_\theta^* - \lambda) (f_\varphi f_\varphi^* - \lambda) - f_\varphi f_\theta^* f_{\theta} f_\varphi^* = \\
    %
    = \lambda^2 - ( f_\theta f_\theta^* + f_\varphi f_\varphi^* ) \lambda + f_\theta f_\theta^* f_\varphi f_\varphi^* - f_\varphi f_\theta^* f_{\theta} f_\varphi^* = \\
    %
    = \lambda^2 - \tr(B) \lambda + \det(B) ,
\end{multline*}
где $\tr(B)$ и $\det(B)$ --- след и определитель матрицы $B$. Корни характеристического уравнения:
\begin{align}
    \lambda_{min} & = \frac{\tr(B) - \sqrt{\tr^2(B) - 4 \det(B)}}{2} \label{emission:emitter:gain:minimum_eigenvalue} , \\
    \lambda_{max} & = \frac{\tr(B) + \sqrt{\tr^2(B) - 4 \det(B)}}{2} \label{emission:emitter:gain:maximum_eigenvalue}.
\end{align}
Вектор $p_{max}$ находим как решение второго уравнения системы \eqref{emission:emitter:gain:system}:
\begin{gather*}
    B p_{max} = \lambda_{max} p_{max} , \\
    ( B - \lambda I ) p_{max} = 0 .
\end{gather*}
Можно выбрать решение, для которого $\norm{p_{max}} = 1$. Далее вектор $a_{max}$ находим из первого уравнения системы \eqref{emission:emitter:gain:system}:
\begin{align*}
    F^*(\vec{w}) F(\vec{w}) a_{max}                         & = \lambda_{max} a_{max} , \\
    \frac{1}{\lambda_{max}} F^*(\vec{w}) F(\vec{w}) a_{max} & = a_{max} , \\
    \frac{1}{\lambda_{max}} F^*(\vec{w}) p_{max}            & = a_{max} .
\end{align*}
Наибольший коэффициент усиления $G_{max}$:
\begin{multline*}
    G_{max}
    = \frac{a_{max}^* F^*(\vec{w}) F(\vec{w}) a_{max}}{a_{max}^* a_{max}} = \\
    %
    = \frac{p_{max}^* F(\vec{w}) \left ( \frac{1}{\lambda_{max}} \right )^* F^*(\vec{w}) F(\vec{w}) \frac{1}{\lambda_{max}} F^*(\vec{w}) p_{max}}{ p_{max}^* F(\vec{w}) \left ( \frac{1}{\lambda_{max}} \right )^* \frac{1}{\lambda_{max}} F^*(\vec{w}) p_{max}} = \\
    %
    = \frac{\modulus{\frac{1}{\lambda_{max}}}^2 p_{max}^* F(\vec{w}) F^*(\vec{w}) F(\vec{w}) F^*(\vec{w}) p_{max}}{ \modulus{\frac{1}{\lambda_{max}}}^2 p_{max}^* F(\vec{w}) F^*(\vec{w}) p_{max}}
    = \frac{\norm{F(\vec{w}) F^*(\vec{w}) p_{max}}^2}{ \norm{F^*(\vec{w}) p_{max}}^2} .
\end{multline*}


\section{Коэффициент полезного действия}

\subsection{Для излучателя}

Коэффициент полезного действия показывает долю излучённой мощности:
\[
    \eta
    = \frac{P_{rad}}{P_{inp}}
    = \frac{a^* Q a}{a^* a}
\]
в силу равенств \eqref{emission:emitter:power:input} и \eqref{emission:emitter:power:radiated}. Коэффициент полезного действия $\eta$ является отношением Релея,
поэтому его значения ограничены наименьшим $\eta_{min}$ и наибольшим $\eta_{max}$ собственными значениями матрицы $Q$:
\[
    \eta_{min} \le \eta \le \eta_{max} .
\]
Поскольку $Q$ матрица порядка 2, то величины $\eta_{min}$ и $\eta_{max}$ можно найти согласно равенствам \eqref{emission:emitter:gain:minimum_eigenvalue} и
\eqref{emission:emitter:gain:maximum_eigenvalue} (с матрицей $Q$ вместо матрицы $B$).

\subsection{Для поляризации}

Представим элементы матрицы $Q$ в нормированном виде:
\[
    Q_{jk}
    =
    \sqrt{Q_{jj}}
    \cdot
    \frac{\iint \limits_{4 \pi} f_j^*(\vec{w}) f_k(\vec{w}) d \Omega}{\sqrt{Q_{jj}} \sqrt{Q_{kk}}}
    \cdot
    \sqrt{Q_{kk}} ,
\]
тогда
\[
    Q = \sqrt{D} R \sqrt{D} ,
\]
где
\[
    \sqrt{D}
    = \begin{pmatrix}
          \sqrt{Q_{11}} & 0          \\
          0          & \sqrt{Q_{22}} \\
    \end{pmatrix} ,
    \;
    %
    R_{jk} = \frac{Q_{jk}}{\sqrt{Q_{jj}} \sqrt{Q_{kk}}} .
\]

Согласно неравенству \eqref{emission:emitter:power:inequality}:
\begin{align*}
    a^* Q a & \le a^* a, \\
    a* \sqrt{D} R \sqrt{D} a & \le a^* a .
\end{align*}
Пусть $x = \sqrt{D} a$, тогда:
\begin{align*}
    x^* R x & \le x^* (\sqrt{D}^{-1})^* (\sqrt{D}^{-1}) x , \\
    x^* R x & \le x^* D^{-1} x .
\end{align*}
Пусть $r_{max}$ --- наибольшее собственное значение матрицы $R$ и $x_{max}$ --- соответствующий этому числу собственный вектор, а $Q_{min} = \min \{ Q_{11}, Q_{22} \}$,
тогда:
\begin{align*}
    x_{max}^* R x_{max} & \le x_{max}^* D^{-1} x_{max} , \\
    x_{max}^* r_{max} x_{max} & \le \sum_{k=1}^n \frac{1}{Q_{kk}} x_{max,k}^* x_{max,k} , \\
    r_{max} \norm{x_{max}}^2 & \le \sum_{k=1}^n \frac{1}{Q_{kk}} \modulus{x_{max,k}}^2 , \\
    r_{max} \norm{x_{max}}^2 & \le \sum_{k=1}^n \frac{1}{Q_{min}} \modulus{x_{max,k}}^2 , \\
    r_{max} \norm{x_{max}}^2 & \le \frac{1}{Q_{min}} \sum_{k=1}^n \modulus{x_{max,k}}^2 , \\
    r_{max} \norm{x_{max}}^2 & \le \frac{1}{Q_{min}} \norm{x_{max}}^2 , \\
    r_{max} & \le \frac{1}{Q_{min}} , \\
    Q_{min} & \le \frac{1}{r_{max}} .
\end{align*}

Матрица $R$ имеет вид:
\[
    R
    = \begin{pmatrix}
          1        & R_{12} \\
          R_{12}^* & 1
    \end{pmatrix} ,
\]
и наибольшее собственное значение матрицы $R$ согласно равенству \eqref{emission:emitter:gain:maximum_eigenvalue} имеет вид:
\begin{multline*}
    r_{max}
    = \frac{\tr(R) + \sqrt{\tr^2(R) - 4 \det(R)}}{2} = \\
    %
    = \frac{2 + \sqrt{2^2 - 4 (1 - \modulus{R_{12}}^2)}}{2}
    = \frac{2 + \sqrt{4 - 4 + 4 \modulus{R_{12}}^2)}}{2} = \\
    %
    = \frac{2 + 2 \modulus{R_{12}}}{2}
    = 1 + \modulus{R_{12}} .
\end{multline*}
Таким образом,
\[
    \min \{ Q_{11}, Q_{22} \} \le \frac{1}{1 + \modulus{R_{12}}} .
\]

Пусть в излучателе используется только первый канал поляризации, тогда:
\[
    a
    = \begin{pmatrix}
          \alpha \\
          0
    \end{pmatrix}
\]
и коэффициент полезного действия:
\[
    \eta_1
    = \frac{a^* Q a}{a^* a}
    = \frac{\alpha^* Q_{11} \alpha}{\alpha^* \alpha}
    = Q_{11}.
\]
Аналогично при работе только второго канала поляризации коэффициент полезного действия:
\[
    \eta_2 = Q_{22} .
\]
Таким образом, величина $Q_{jj}$ совпадает с коэффициентом полезного действия канала поляризации, и:
\[
    \min \{ \eta_1, \eta_2 \} \le \frac{1}{1 + \modulus{R_{12}}} .
\]