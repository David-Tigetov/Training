\chapter{Антенна}

\section{Модель антенны}

Рассматривается декартова система координат, в которой имеются $n$ излучателей, образующих антенну. Заданы местоположения излучателей $r_k$, и вектор огибающих
сигналов на входах излучателей:
\[
    a
    = \begin{pmatrix}
          a_1 \\\
          \dots \\\
          a_n
    \end{pmatrix} .
\]
Можно считать, что у каждого излучателя только один канал поляризации, в случае двух и более каналов необходимо в точку $r_k$ поместить ещё один или более
излучателей с другими поляризациями.

Напряженность электрического поля $E(w,R)$, создаваемого антенной, в дальней зоне будет приближённо равна:
\[
    E(w,R,a) = \widetilde{F}_a(w,a) \cdot \frac{e^{i \modulus{w} R}}{R} ,
\]
где $\widetilde{F}(w,a)$ --- диаграмма направленности антенны при огибающих $a$:
\[
    \widetilde{F}_a(w,a) = \sum_{k=1}^n f_k(w) a_k e^{i \scalarproduct{\vec{r}_k}{\vec{w}}} ,
\]
где $f_k(w)$ --- парциальная диаграмма направленности $k$-го излучателя в составе антенны:
\[
    f_k(w) =
    \begin{pmatrix}
        f_{k,\theta}(w) \\
        f_{k,\varphi}(w)
    \end{pmatrix}
    ,
\]
и множитель $e^{i \scalarproduct{\vec{r}_k}{\vec{w}}}$ соответствует смещению фазы при приведении напряжённостей поля излучателей к общему началу отсчёта.

Если парциальные диаграммы $f(w)$ и смещения фаз собрать в матрицу $f(w)$:
\begin{gather*}
    F_a(\vec{w}) =
    \begin{pmatrix}
        \vec{f}_{1,\theta}(\vec{w}) e^{i \scalarproduct{\vec{r}_1}{\vec{w}}}  & \dots & \vec{f}_{n,\theta}(\vec{w}) e^{i \scalarproduct{\vec{r}_n}{\vec{w}}}  \\
        \vec{f}_{1,\varphi}(\vec{w}) e^{i \scalarproduct{\vec{r}_1}{\vec{w}}} & \dots & \vec{f}_{n,\varphi}(\vec{w}) e^{i \scalarproduct{\vec{r}_n}{\vec{w}}}
    \end{pmatrix} ,
\end{gather*}
тогда диаграмма направленности $\widetilde{F}_a(w,a)$ будет иметь вид:
\[
    \widetilde{F}_a(w,a) = F_a(w) a.
\]
Таким образом, напряженность имеет вид:
\[
    E = F_a(w) a \cdot \frac{e^{i \modulus{w} R}}{R} .
\]