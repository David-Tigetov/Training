\chapter{Классические вероятности}

\section*{Введение}

В теории вероятностей решение задач выполняется в рамках вероятностного пространства $\left ( \Omega, \Sigma, P \right )$. В первом занятии разбирались вопросы формирования
множества всех элементарных исходов $\Omega$, событий из алгебры $\Sigma$ и операций над этими событиями. В этом занятии мы перейдем к одной из основных задач теории вероятностей
--- вычислению вероятностей с использованием заданной вероятностей меры $P$.

Вероятностная мера $P$ может быть задана различными способами. В частности, если множество элементарных исходов $\Omega$ имеет конечное число элементов $\modulus{\Omega} < \infty$,
то всем элементарным исходам $\omega_i \in \Omega$ ($i=1,\dots,\modulus{\Omega}$) можно приписать равные вероятности:
\begin{equation}
    \probability{\event{\omega_i}} = \frac{1}{\modulus{\Omega}}.
\end{equation}
Такое определение вероятностей меры называется \textbf{классическим определением вероятности}.

Теперь представим, что есть некоторое событие $A \in \Sigma$ из алгебры наблюдаемых событий. Как вычислить вероятность $\probability{A}$? Пусть событие $A$ состоит из $k$
элементарных исходов:
\begin{equation}
    A = \set{ \omega_{i_1}, \omega_{i_2}, \dots, \omega_{i_k}}.
\end{equation}
Мы знаем, что вероятностная мера $P$ обладает свойством аддитивности --- если два события $B$ и $C$ несовместны, то вероятности суммы событий есть сумма вероятностей:
\begin{equation}
    B \cap C = \emptyset : \; \probability{B + C} = \probability{B} + \probability{C}.
\end{equation}

Представим событие $A$ в виде объединения:
\begin{equation}
    A = \underbrace{\set{\omega_{i_1}}}_B \cup \underbrace{\set{\omega_{i_2}, \dots, \omega_{i_k}}}_C
\end{equation}
и воспользуемся свойством аддитивности меры $P$:
\begin{equation}
    \probability{A} = \probability{\set{\omega_{i_1}}} + \probability{\set{\omega_{i_2}, \dots, \omega_{i_k}}} .
\end{equation}
Поступая далее аналогичным образом, в итоге получим:
\begin{equation}
    \probability{A} = \probability{\set{\omega_{i_1}}} + \probability{\set{\omega_{i_2}}} + \dots + \probability{\set{\omega_{i_k}}} .
\end{equation}
Поскольку все элементарные исходы $\omega_{i_j}$ имеют одинаковые вероятности, то:
\begin{equation}
    \probability{A}
    = \underbrace{\frac{1}{\modulus{\Omega}} + \frac{1}{\modulus{\Omega}} + \dots + \frac{1}{\modulus{\Omega}}}_k
    = \frac{\modulus{A}}{\modulus{\Omega}} .
\end{equation}

Таким образом, вероятность определяется количествами элементарных исходов в событии $A$ и во всём множестве элементарных исходов $\Omega$.

\section*{Задача 18.66}

В магазин поступило 30 новых цветных телевизоров, среди которых 5 имеют скрытые дефекты. Наудачу отбирается один телевизор для проверки. Какова вероятность, что он не имеет
скрытых дефектов?

\subsection*{Решение:}

Пусть событие $A = \event{\text{выбранный телевизор не имеет дефектов}}$. Количество телевизоров, не имеющих скрытые дефекты, $30-5=25$, значит
\begin{equation}
    \modulus{A} = 25 .
\end{equation}
Всего телевизоров 30, значит
\begin{equation}
    \modulus{\Omega} = 30 .
\end{equation}
Таким образом, вероятность события $A$:
\begin{equation}
    \probability{A}
    = \frac{\modulus{A}}{\modulus{\Omega}}
    = \frac{25}{30}
    = \frac{5}{6}
    .
\end{equation}

\subsection*{Ответ:}
$\frac{5}{6}$.

\section*{Задача [18.70]}

Подбрасываются два кубика. Фиксируются количества очков, выпавших на первом и втором кубиках. Найти вероятности событий
\begin{enumerate}
    \item $A = \event{\text{сумма очков нечётна}}$,
    \item $B = \event{\text{произведение очков чётно}}$.
\end{enumerate}

\subsection*{Решение:}

\begin{enumerate}
    \item
    Множество элементарных исходов $\Omega = \set{(i, j): i, j \in \left \{ 1, 2, 3, 4, 5, 6\right \}}$ содержит 36 элементарных исходов:
    \begin{equation}
        \modulus{\Omega} = 6 \cdot 6 = 36.
    \end{equation}

    \item
    Для подсчёта количества элементарных исходов в событии $A$ составим таблицу суммы:
    \begin{center}
        \begin{tabular}{c|c|c|c|c|c|c|}
            $i$, $j$ & 1 & 2 & 3 & 4  & 5  & 6  \\
            \hline
            1        & 2 & 3 & 4 & 5  & 6  & 7  \\
            \hline
            2        & 3 & 4 & 5 & 6  & 7  & 8  \\
            \hline
            3        & 4 & 5 & 6 & 7  & 8  & 9  \\
            \hline
            4        & 5 & 6 & 7 & 8  & 9  & 10 \\
            \hline
            5        & 6 & 7 & 8 & 9  & 10 & 11 \\
            \hline
            6        & 7 & 8 & 9 & 10 & 11 & 12 \\
            \hline
        \end{tabular}
    \end{center}
    Теперь подсчитаем количество клеток с нечетными числами (по диагоналям):
    \begin{equation}
        \modulus{A} = 2 + 4 + 6 + 4 + 2 = 18 .
    \end{equation}
    Таким образом, вероятность
    \begin{equation}
        \probability{A} = \frac{\modulus{A}}{\modulus{\Omega}} = \frac{18}{36} = \frac{1}{2} .
    \end{equation}

    При подсчёте количества элементарных исходов в событии $A$ можно было рассуждать и следующим образом: для получения нечётной суммы достаточно чтобы
    \begin{itemize}
        \item на первом кубике выпало нечётное число, а на втором --- чётное,
        \item на первом кубике выпало чётное число, а на втором --- нечётное.
    \end{itemize}
    Поскольку нечётных очков $\set{1, 3, 5}$ всего 3, и четных $\set{2, 4, 6}$ тоже 3, то
    \begin{equation}
        \modulus{A} = 3 \cdot 3 + 3 \cdot 3 = 18.
    \end{equation}

    \item
    Для подсчёта количества элементарных исходов в событии $B$ составим таблицу произведения:
    \begin{center}
        \begin{tabular}{c|c|c|c|c|c|c|}
            $i$, $j$ & 1 & 2  & 3  & 4  & 5  & 6  \\
            \hline
            1        & 1 & 2  & 3  & 4  & 5  & 6  \\
            \hline
            2        & 2 & 4  & 6  & 8  & 10 & 12 \\
            \hline
            3        & 3 & 6  & 9  & 12 & 15 & 18 \\
            \hline
            4        & 4 & 8  & 12 & 16 & 20 & 24 \\
            \hline
            5        & 5 & 10 & 15 & 20 & 25 & 30 \\
            \hline
            6        & 6 & 12 & 18 & 24 & 30 & 36 \\
            \hline
        \end{tabular}
    \end{center}
    Подсчёт по строкам количества клеток с чётным числом приводит результату:
    \begin{equation}
        \modulus{B} = 3 + 6 + 3 + 6 + 3 + 6 = 27 .
    \end{equation}
    Таким образом, вероятность
    \begin{equation}
        \probability{B} = \frac{\modulus{B}}{\modulus{\Omega}} = \frac{27}{36} = \frac{3}{4} .
    \end{equation}

    Количество элементарных исходов в $B$ можно было подсчитать и следующим образом: для получения чётного произведения достаточно чтобы
    \begin{itemize}
        \item на первом кубике выпало нечётное число, а на втором --- чётное,
        \item на первом кубике выпало чётное число, а на втором --- любое.
    \end{itemize}
    Таким образом,
    \begin{equation}
        \modulus{B} = 3 \cdot 3 + 3 \cdot 6 = 9 + 18 = 27 .
    \end{equation}
\end{enumerate}

\section*{Ответ:}

$\probability{A} = \frac{1}{2}$, $\probability{B} = \frac{3}{4}$.

\section*{Задача 18.73}

Первого сентября на первом курсе одного из факультетов запланировано по расписанию три лекции по разным предметам. Всего на первом курсе изучается десять предметов. Студент,
не успевший ознакомится с расписанием, пытается его угадать. Какова вероятность успеха в данном эксперименте, если считать, что любое расписание из трех предметов равновозможно?

\subsection*{Решение:}

Подсчитаем количество возможных вариантов расписания (элементарных исходов). Количество вариантов выбора предметов:
\begin{itemize}
    \item для первой лекции --- 10 (всего предметов 10),
    \item для второй лекции --- 9 (поскольку предметы разные, и один предмет уже нельзя выбрать),
    \item для третьей лекции --- 8 (поскольку предметы разные и уже два предмета выбрать нельзя).
\end{itemize}
Таким образом, мощность множества $\Omega$ элементарных исходов:
\begin{equation}
    \modulus{\Omega} = A_{10}^3 = 10 \cdot 9 \cdot 8 = 720 .
\end{equation}

Угадать расписание значит угадать все три предмета, то есть количество исходов в этом событии лишь одно, поэтому вероятность угадать расписание $P_s$:
\begin{equation}
    P_s = \frac{1}{\modulus{\Omega}} = \frac{1}{720} .
\end{equation}

\subsection*{Ответ:}
$\frac{1}{720}$.

\section*{Задача 18.80}

Среди кандидатов в студенческий совет факультета 3 первокурсника, 5 второкурсников и 7 третьекурсников. Из этого состава наудачу выбирают 5 человек на предстоящую конференцию.
Найти вероятности следующих событий:
\begin{enumerate}
    \item $A = \event{\text{будут выбраны одни третьекурсники}}$,
    \item $B = \event{\text{все первокурсники попадут на конференцию}}$,
    \item $C = \event{\text{не будет выбрано ни одного второкурсника}}$.
\end{enumerate}

\subsection*{Решение:}
\begin{enumerate}
    \item
    Подсчитаем количество всех возможных вариантов выбора. Общее количество студентов $3+5+7=15$, из них нужно отобрать $5$, поэтому всего вариантов:
    \begin{equation}
        \modulus{\Omega} = C_{15}^5 .
    \end{equation}

    \item
    Событие $A$ содержит варианты, в которых выбирают 5 третьекурсников. Количество вариантов равняется числу сочетаний:
    \begin{equation}
        \modulus{A} = C_7^5 .
    \end{equation}
    Вероятность события $A$:
    \begin{multline}
        \probability{A}
        = \frac{\modulus{A}}{\modulus{\Omega}}
        = \frac{C_7^5}{C_{15}^5}
        = \frac{\frac{7!}{5!2!}}{\frac{15!}{5!10!}}
        = \frac{\frac{7!}{2!}}{\frac{15!}{10!}} = \\
        %
        = \frac{7 \cdot 6 \cdot 5 \cdot 4 \cdot 3}{15 \cdot 14 \cdot 13 \cdot 12 \cdot 11}
        = \frac{7 \cdot 6 \cdot 4}{14 \cdot 13 \cdot 12 \cdot 11}
        = \frac{6 \cdot 2}{13 \cdot 12 \cdot 11}
        = \frac{1}{13 \cdot 11}
        = \frac{1}{130 + 13}
        = \frac{1}{143} .
    \end{multline}

    \item
    Событие $B$ содержит варианты, в которых выбирают 3 первокурсника, это можно сделать единственным образом, а остальных 2 студентов можно выбрать произвольным образом
    из второкурсников и третьекурсников, их всего $5+7 = 12$, поэтому количество вариантов:
    \begin{equation}
        \modulus{B}
        = C_3^3 \cdot C_{12}^2
        = 1 \cdot C_{12}^2
    \end{equation}
    Вероятность события $B$:
    \begin{multline}
        \probability{B}
        = \frac{\modulus{B}}{\modulus{\Omega}}
        = \frac{C_{12}^2}{C_{15}^5}
        = \frac{\frac{12!}{2!10!}}{\frac{15!}{5!10!}}
        = \frac{12! \cdot 5!}{2! \cdot 15!} = \\
        %
        = \frac{5 \cdot 4 \cdot 3}{15 \cdot 14 \cdot 13}
        = \frac{4}{14 \cdot 13}
        = \frac{2}{7 \cdot 13}
        = \frac{2}{70 + 21}
        = \frac{2}{91} .
    \end{multline}

    \item
    Событие $C$ содержит варианты, в которых не выбирают второкурсников, значит нужно набрать 5 студентов из первокурсников и третьекурсников, их всего $3 + 7 = 10$, поэтому
    количество вариантов:
    \begin{equation}
        \modulus{C} = C_{10}^5 .
    \end{equation}
    Вероятность события $C$:
    \begin{multline}
        \probability{C}
        = \frac{\modulus{C}}{\modulus{\Omega}}
        = \frac{C_{10}^5}{C_{15}^5}
        = \frac{\frac{10!}{5!5!}}{\frac{15!}{5!10!}}
        = \frac{\frac{10!}{5!}}{\frac{15!}{10!}} = \\
        %
        = \frac{10 \cdot 9 \cdot 8 \cdot 7 \cdot 6}{15 \cdot 14 \cdot 13 \cdot 12 \cdot 11}
        = \frac{5 \cdot 2 \cdot 3 \cdot 3 \cdot 8 \cdot 7 \cdot 6}{15 \cdot 14 \cdot 13 \cdot 12 \cdot 11}
        = \frac{2 \cdot 3 \cdot 8 \cdot 7 \cdot 6}{14 \cdot 13 \cdot 12 \cdot 11}
        = \frac{3 \cdot 8 \cdot 6}{13 \cdot 12 \cdot 11} = \\
        %
        = \frac{3 \cdot 2 \cdot 4 \cdot 6}{13 \cdot 12 \cdot 11}
        = \frac{3 \cdot 4}{13 \cdot 11}
        = \frac{12}{13 \cdot 11}
        = \frac{12}{130 + 13}
        = \frac{12}{143} .
    \end{multline}
\end{enumerate}

\subsection*{Ответ:}
\begin{enumerate}
    \item $\probability{A} = \frac{1}{143}$,
    \item $\probability{B} = \frac{2}{91}$,
    \item $\probability{C} = \frac{12}{143}$.
\end{enumerate}

\section*{Задача 18.88}

Цифры 1, 2, \dots, 9 записываются в случайном порядке. Найти вероятности событий:
\begin{enumerate}
    \item $B = \event{\text{цифры 1 и 2 будут стоять рядом и в порядке возрастания}}$,
    \item $C = \event{\text{цифры 3, 6 и 9 будут стоять рядом}}$.
\end{enumerate}

\subsection*{Решение:}
\begin{enumerate}
    \item
    Подсчитаем количество возможных вариантов записи цифр (элементарных исходов): первую позицию может занимать любая из 9 цифр, вторую позицию --- любая из оставшихся 8 цифр,
    третью позицию --- любая из оставшихся 7 цифр, и так далее, для последней позиции останется 1 цифра. Таким образом, всего вариантов записи цифр:
    \begin{equation}
        \modulus{\Omega} = 9 \cdot 8 \cdot 7 \cdot \dots \cdot 1 = 9!
    \end{equation}

    \item
    Считаем количество вариантов записи цифр в событии $B$.
    \begin{equation}
        \begin{array}{ccccccccccr}
            1        & 2        & \diamond & \diamond & \diamond & \diamond & \diamond & \diamond & \diamond & | 7! \\
            \diamond & 1        & 2        & \diamond & \diamond & \diamond & \diamond & \diamond & \diamond & | 7! \\
            \diamond & \diamond & 1        & 2        & \diamond & \diamond & \diamond & \diamond & \diamond & | 7! \\
            \dots    &          &          &          &          &          &          &          &          &      \\
            \diamond & \diamond & \diamond & \diamond & \diamond & \diamond & \diamond & 1        & 2        & | 7! \\
        \end{array}
    \end{equation}

    Всего 8 вариантов расположения цифр 1 и 2, и на каждый вариант приходится еще $7!$ вариантов записи остальных цифр. Итого,
    \begin{equation}
        \modulus{B} = 8 \cdot 7! = 8!
    \end{equation}
    Вероятность события $B$
    \begin{equation}
        \probability{B} = \frac{\modulus{B}}{\modulus{\Omega}} = \frac{8!}{9!} = \frac{1}{9} .
    \end{equation}

    \item
    Считаем количество вариантов записи цифр в событии $C$.
    \begin{equation}
        \begin{array}{lccccccccccr}
            & 3        & 6        & 9        & \diamond & \diamond & \diamond & \diamond & \diamond & \diamond & | 6! \\
            3! | & \dots    &          &          &          &          &          &          &          &          &      \\
            & 9        & 6        & 3        & \diamond & \diamond & \diamond & \diamond & \diamond & \diamond & | 6! \\
            & \diamond & 3        & 6        & 9        & \diamond & \diamond & \diamond & \diamond & \diamond & | 6! \\
            3! | & \dots    &          &          &          &          &          &          &          &          &      \\
            & \diamond & 9        & 6        & 3        & \diamond & \diamond & \diamond & \diamond & \diamond & | 6! \\
            & \dots    &          &          &          &          &          &          &          &          &      \\
            & \diamond & \diamond & \diamond & \diamond & \diamond & \diamond & 3        & 6        & 9        & | 6! \\
            3! | & \dots    &          &          &          &          &          &          &          &          &      \\
            & \diamond & \diamond & \diamond & \diamond & \diamond & \diamond & 9        & 6        & 3        & | 6! \\
        \end{array}
    \end{equation}
    Группа цифр $\left \{ 3, 6, 9 \right \}$ может занимать 7 различных местоположений, на каждое местоположение приходится $3!$ вариантов расположения цифр $\left \{ 3, 6, 9 \right \}$ внутри группы,
    и на каждый такой вариант еще $6!$ вариантов расположения остальных шести цифр. Таким образом,
    \begin{equation}
        \modulus{C} = 7 \cdot 3! \cdot 6! = 3! \cdot 7!
    \end{equation}
    Вероятность события $C$:
    \begin{equation}
        \probability{C}
        = \frac{\modulus{C}}{\modulus{\Omega}}
        = \frac{3! \cdot 7!}{9!}
        = \frac{3!}{9 \cdot 8}
        = \frac{3 \cdot 2}{9 \cdot 8}
        = \frac{1}{3 \cdot 4}
        = \frac{1}{12} .
    \end{equation}
\end{enumerate}

\subsection*{Ответ:}
\begin{enumerate}
    \item $\probability{B} = \frac{1}{9}$,
    \item $\probability{C} = \frac{1}{12}$.
\end{enumerate}

\section*{Задача 18.100}

Бросается 10 одинаковых игральных кубиков. Вычислить вероятности событий:
\begin{enumerate}
    \item $A = \event{\text{ни на одном кубике не выпадет 6}}$,
    \item $B = \event{\text{хотя бы на одном кубике выпадет 6}}$,
    \item $C = \event{\text{ровно на 3 кубиках выпадет 6}}$.
\end{enumerate}

\subsection*{Решение:}
\begin{enumerate}
    \item
    Каждый элементарный исход будет представлять в виде вектора из 10 компонент: $i$-ая компонента обозначает число очков на кубике с номером $i$.
    Общее количество всех таких векторов:
    \begin{equation}
        \modulus{\Omega} = \underbrace{6 \cdot 6 \cdot \dots 6}_{10} = 6^{10}.
    \end{equation}

    \item
    Событие $A$ образовано элементарными исходами (векторами), в которых нет 6, то есть количество допустимых очков для каждого кубика всего 5,
    поэтому количество элеметарных исходов в событии $A$:
    \begin{equation}
        \modulus{A} = \underbrace{5 \cdot 5 \cdot \dots 5}_{10} = 5^{10}.
    \end{equation}
    Вероятность события $A$:
    \begin{equation}
        \probability{A}
        = \frac{\modulus{A}}{\modulus{\Omega}}
        = \frac{5^{10}}{6^{10}}
        = \left ( \frac{5}{6} \right )^{10} .
    \end{equation}

    \item
    Для вычисления вероятности события $B$ заметим, что событие $B$, в котором выпадает хотя бы одна 6, является противоположным событию $A$,
    в котором не выпадает ни одной 6:
    \begin{equation}
        B = \overline{A} .
    \end{equation}
    В силу известного свойства о вероятности противоположного события:
    \begin{equation}
        \probability{B} = \probability{\overline{A}} = 1 - \probability{A} = 1 - \left ( \frac{5}{6} \right )^{10} .
    \end{equation}

    \item
    Событие $C$ образовано элементарными исходами (векторами), в которых 3 любые компоненты равны 6, а остальные компоненты ---
    не равны 6. Значит нужно выбрать 3 компоненты из 10, которые будут равны 6, --- это можно сделать количеством способов, равным $C_{10}^3$,
    в остальных 7 компонентах нельзя выбирать число очков равное 6, значит для каждой осталось только 5 вариантов:
    \begin{equation}
        \modulus{C} = C_{10}^3 \cdot 5^7 .
    \end{equation}
    Вероятность события $C$:
    \begin{multline}
        \probability{C}
        = \frac{\modulus{C}}{\modulus{\Omega}}
        = \frac{C_{10}^3 \cdot 5^7}{6^{10}}
        = \frac{\frac{10!}{3!7!} \cdot 5^7}{6^{10}} = \\
        %
        = \frac{\frac{10 \cdot 9 \cdot 8}{3 \cdot 2}  \cdot 5^7}{6^{10}}
        = \frac{10 \cdot 9 \cdot 8}{3 \cdot 2 \cdot 6^3} \cdot \left ( \frac{5}{6} \right )^7
        = \frac{10 \cdot 3 \cdot 4}{6 \cdot 6 \cdot 6} \cdot \left ( \frac{5}{6} \right )^7
        = \frac{2 \cdot 5 \cdot 3 \cdot 2 \cdot 2}{2 \cdot 3 \cdot 2 \cdot 3 \cdot 2 \cdot 3} \cdot \left ( \frac{5}{6} \right )^7 = \\
        %
        = \frac{5}{3 \cdot 3} \cdot \left ( \frac{5}{6} \right )^7
        = \frac{5}{9} \cdot \left ( \frac{5}{6} \right )^7 .
    \end{multline}
\end{enumerate}

\subsection*{Ответ:}
\begin{enumerate}
    \item $\probability{A} = \left ( \frac{5}{6} \right )^{10}$,
    \item $\probability{B} = 1 - \left ( \frac{5}{6} \right )^{10}$,
    \item $\probability{C} = \frac{5}{9} \cdot \left ( \frac{5}{6} \right )^7$.
\end{enumerate}

\section*{Задачи для самостоятельного решения}

Из раздела 18 сборника задач Ефимова и Поспелова.
\begin{enumerate}
    \item На занятии: 69, 76, 92, 103.
    \item Дома: 74, 85, 93, 97, 98, 105, 110, 128.
\end{enumerate}

Из сборника задач типового расчёта Чудесенко: 1, 2, 3, 4.