\documentclass[12pt]{article}

\usepackage[T1]{fontenc}
\usepackage[utf8]{inputenc}
\usepackage[english,russian]{babel}
\usepackage[margin=2cm]{geometry}
\usepackage{amsmath}
\usepackage{amsfonts}
\usepackage{stmaryrd}
\usepackage{tikz}
\usepackage{wasysym}

% команды вывода первой частной производной
\newcommand{\fpd}[1]{\frac{\partial}{\partial #1}}
\newcommand{\fpda}[2]{\frac{\partial #1}{\partial #2}}
\newcommand{\fpdp}[2]{\fpd{#2} \left ( #1 \right )}

\newcommand{\expectation}[1]{\mathtt{M} \left [ #1 \right ]}
\newcommand{\conditionalexpectation}[2]{\expectation{ #1 \left | #2 \right .}}
\newcommand{\variance}[1]{\mathtt{D} \left [ #1 \right ]}
\newcommand{\covariance}[2]{\mathtt{cov} \left ( #1, #2 \right )}

\newcommand{\modulus}[1]{\left | #1 \right |}
\newcommand{\norm}[1]{\left \| {#1} \right \|}

\newcommand{\event}[1]{\left \{ #1 \right \} }
\newcommand{\probability}[1]{P \event{#1}}


\begin{document}

    \section*{Задача 28}
    Случайная величина $\xi$ имеет сдвинутое показательное распределение с параметром $\lambda > 0$ и плотностью распределения $p_\xi(x)$:
    \[
        p_\xi(x)
        = \left \{
        \begin{array}{ll}
            0,                             & x < 2   \\
            \lambda e^{- \lambda (x - 2)}, & x \ge 2
        \end{array}
        \right .
        .
    \]
    Случайная величина $\eta = \varphi(\xi)$, где функция $\varphi(x)$ задана графически:

    \begin{figure}[!ht]
        \centering
        \begin{tikzpicture}
            % оси
            \draw [->] ( -1, 0 ) -- ( 8, 0 ) node [below] at ( 8, 0 ) {$X$};
            \draw [->] ( 0, -5 ) -- ( 0, 2 ) node [left] at ( 0, 2 ) {$Y$};

            % метки на оси X
            \draw ( 1, -0.1 ) -- ( 1, 0.1 ) node [below] at ( 1, 0 ) {$1$};
            \draw ( 2, -0.1 ) -- ( 2, 0.1 ) node [below] at ( 2, 0 ) {$2$};
            \draw ( 3, -0.1 ) -- ( 3, 0.1 ) node [below] at ( 3, 0 ) {$3$};
            \draw ( 4, -0.1 ) -- ( 4, 0.1 ) node [below] at ( 4, 0 ) {$4$};
            \draw ( 5, -0.1 ) -- ( 5, 0.1 ) node [below] at ( 5, 0 ) {$5$};
            \draw ( 6, -0.1 ) -- ( 6, 0.1 ) node [below] at ( 6, 0 ) {$6$};
            \draw ( 7, -0.1 ) -- ( 7, 0.1 ) node [below] at ( 7, 0 ) {$7$};

            % метки на оси Y
            \draw ( -0.1, -5 ) -- ( 0.1, -5 ) node [left] at ( 0, -5 ) {$-5$};
            \draw ( -0.1, -4 ) -- ( 0.1, -4 ) node [left] at ( 0, -4 ) {$-4$};
            \draw ( -0.1, -3 ) -- ( 0.1, -3 ) node [left] at ( 0, -3 ) {$-3$};
            \draw ( -0.1, -2 ) -- ( 0.1, -2 ) node [left] at ( 0, -2 ) {$-2$};
            \draw ( -0.1, -1 ) -- ( 0.1, -1 ) node [left] at ( 0, -1 ) {$-1$};
            \draw ( -0.1, 1 ) -- ( 0.1, 1 ) node [left] at ( 0, 1 ) {$1$};

            % график функции
            \draw [ultra thick] ( 2, 1 ) -- ( 4, -1 ) -- ( 6, -1 ) -- ( 8, -5 );
            \node [right] at ( 2, 1 ) {$-(x - 3)$};
            \node [right] at ( 7, -2 ) {$-2(x - 6) - 1$};
        \end{tikzpicture}
        \caption{График функции преобразования $\varphi(x)$.}
    \end{figure}

    Найти функцию распределения $F_\eta(y)$ и плотность распределения $p_\eta(y)$ случайной величины $\eta$.

    \subsection*{Решение:}
    Найдем функцию распределения $F_\xi(x)$ случайной величины $\xi$:
    \[
        F_\xi(x)
        = \int \limits_{-\infty}^x p_x(t) dt
        = \left \{
        \begin{array}{ll}
            0,                         & x < 2   \\
            1 - e^{- \lambda (x - 2)}, & x \ge 2
        \end{array}
        \right .
        .
    \]

    Теперь нужно выразить функцию распределения $F_\eta(y)$ через функцию распределения $F_\xi(x)$. По определению функции распределения:
    \[
        F_\eta(y)
        = \probability{\eta < y}
        = \probability{\varphi(\xi) < y}.
    \]
    Нужно неравенство $\eta < y$, оно же $\varphi(\xi) < y$, преобразовать к условию для $\xi$, которое принимает различный вид и изменяется при переходе через уровень $y=-1$, поскольку
    в этой точке меняется сама функция преобразования $\varphi(x)$. Условие для $\xi$ легко представить себе, если провести на графике функции $\varphi(x)$ уровень $y$ (рисунок
    \ref{28:transformation}).

    \begin{figure}[!ht]
        \centering
        \begin{tikzpicture}
            % оси
            \draw [->] ( -1, 0 ) -- ( 8, 0 ) node [below] at ( 8, 0 ) {$X$};
            \draw [->] ( 0, -5 ) -- ( 0, 2 ) node [left] at ( 0, 2 ) {$Y$};

            % метки на оси X
            \draw ( 1, -0.1 ) -- ( 1, 0.1 ) node [below] at ( 1, 0 ) {$1$};
            \draw ( 2, -0.1 ) -- ( 2, 0.1 ) node [below] at ( 2, 0 ) {$2$};
            \draw ( 3, -0.1 ) -- ( 3, 0.1 ) node [below] at ( 3, 0 ) {$3$};
            \draw ( 4, -0.1 ) -- ( 4, 0.1 ) node [below] at ( 4, 0 ) {$4$};
            \draw ( 5, -0.1 ) -- ( 5, 0.1 ) node [below] at ( 5, 0 ) {$5$};
            \draw ( 6, -0.1 ) -- ( 6, 0.1 ) node [below] at ( 6, 0 ) {$6$};
            \draw ( 7, -0.1 ) -- ( 7, 0.1 ) node [below] at ( 7, 0 ) {$7$};

            % метки на оси Y
            \draw ( -0.1, -5 ) -- ( 0.1, -5 ) node [left] at ( 0, -5 ) {$-5$};
            \draw ( -0.1, -4 ) -- ( 0.1, -4 ) node [left] at ( 0, -4 ) {$-4$};
            \draw ( -0.1, -3 ) -- ( 0.1, -3 ) node [left] at ( 0, -3 ) {$-3$};
            \draw ( -0.1, -2 ) -- ( 0.1, -2 ) node [left] at ( 0, -2 ) {$-2$};
            \draw ( -0.1, -1 ) -- ( 0.1, -1 ) node [left] at ( 0, -1 ) {$-1$};
            \draw ( -0.1, 1 ) -- ( 0.1, 1 ) node [left] at ( 0, 1 ) {$1$};

            % график функции
            \draw [ultra thick] ( 2, 1 ) -- ( 4, -1 ) -- ( 6, -1 ) -- ( 8, -5 );

            % уроверь y
            \draw [dashed] ( -1, -0.6 ) -- ( 8, -0.6 ) node [below] ( 6, -0.6 ) {уровень $y$};
            \draw [dashed] ( 3.6, -0.6 ) -- ( 3.6, 0 );

            % неравенства
            \draw ( 0, -0.6 ) to [out=-70,in=90] ( 0.5, -5) node [right] at ( 0.5, -3 ) {$\eta < y$};
            \draw ( 3.6, 0 ) to [out=20,in=180] ( 8, 0.5 ) node [above] at ( 7, 0.5) {$-y + 3 < \xi$};
        \end{tikzpicture}
        \caption{Преобразование неравенства $\eta < y$.}
        \label{28:transformation}
    \end{figure}

    \begin{enumerate}
        \item если $y \le -1$:
        \[
            \varphi(\xi) < y
            \Leftrightarrow
            -2 ( \xi - 6 ) - 1 < y
            \Leftrightarrow
            - \frac{y+1}{2} + 6 < \xi
        \]
        \item если $-1 < y \le 1 $:
        \[
            \varphi(\xi) < y
            \Leftrightarrow
            - ( \xi - 3 ) < y
            \Leftrightarrow
            - y + 3 < \xi .
        \]
        \item если $ 1 < y $ (из графика):
        \[
            \varphi(\xi) < y
            \Leftrightarrow
            \xi \ge 2 .
        \]
    \end{enumerate}
    Таким образом,
    \begin{multline*}
        F_\eta(y)
        =
        \left \{
        \begin{array}{ll}
            \probability{- \frac{y+1}{2} + 6 < \xi}, & y \le -1     \\
            \probability{- y + 3 < \xi},             & -1 < y \le 1 \\
            \probability{2 \le \xi},                 & 1 < y
        \end{array}
        \right .
        =
        \left \{
        \begin{array}{ll}
            1 - F_\xi \left ( - \frac{y+1}{2} + 6 \right ), & y \le -1     \\
            1 - F_\xi \left ( - y + 3 \right ),             & -1 < y \le 1 \\
            1,                                              & 1 < y
        \end{array}
        \right .
        = \\
        %
        =
        \left \{
        \begin{array}{ll}
            1 - 1 + e^{ -\lambda \left ( - \frac{y+1}{2} + 6 - 2 \right )}, & y \le -1     \\
            1 - 1 + e^{ -\lambda \left ( - y + 3 - 2 \right )},             & -1 < y \le 1 \\
            1,                                                              & 1 < y
        \end{array}
        \right .
        =
        \left \{
        \begin{array}{ll}
            e^{ -\lambda \left ( - \frac{y+1}{2} + 4 \right )}, & y \le -1     \\
            e^{ -\lambda \left ( - y + 1 \right )},             & -1 < y \le 1 \\
            1,                                                  & 1 < y
        \end{array}
        \right . .
    \end{multline*}

    График функции $F_\eta(y)$ представлен на рисунке \ref{28:eta_cumulative_distribution_plot}.

    \begin{figure}
        \centering
        \begin{tikzpicture}[yscale=5]
            % оси
            \draw [->] ( -5, 0 ) -- ( 3, 0 ) node [below] at ( 3, 0 ) {$Y$};
            \draw [->] ( 0, -0.1 ) -- ( 0, 1.2 );

            % y <= -1
            \draw [domain=-5:-1] plot(\x, {exp(-0.5*(-0.5*(\x+1)+4))});
            % -1 < y <= 1
            \draw [domain=-1:1] plot(\x, {exp(-0.5*(-\x+1))});
            % 1 < y
            \draw ( 1, 1 ) -- ( 3, 1 );

            % метки на оси Y
            \draw ( -4, -0.02 ) -- ( -4, 0.02 ) node [below] at ( -4, 0 ) {$-4$};
            \draw ( -3, -0.02 ) -- ( -3, 0.02 ) node [below] at ( -3, 0 ) {$-3$};
            \draw ( -2, -0.02 ) -- ( -2, 0.02 ) node [below] at ( -2, 0 ) {$-2$};
            \draw ( -1, -0.02 ) -- ( -1, 0.02 ) node [below] at ( -1, 0 ) {$-1$};
            \draw ( 1, -0.02 ) -- ( 1, 0.02 ) node [below] at ( 1, 0 ) {$1$};
            \draw ( 2, -0.02 ) -- ( 2, 0.02 ) node [below] at ( 2, 0 ) {$2$};

            % метки на оси Z
            \draw ( -0.02, 1 ) -- ( 0.02, 1 ) node [left] at ( 0, 1 ) {$1$};
        \end{tikzpicture}
        \caption{График функции распределения $F_\eta(y)$.}
        \label{28:eta_cumulative_distribution_plot}
    \end{figure}

    Плотность распределения $p_\eta(y)$ --- функция,  которая удовлетворяет равенству:
    \begin{equation}
        \label{28:density}
        F_\eta(y)
        = \int \limits_{- \infty}^y p_\eta(t) dt .
    \end{equation}
    В некоторых случаях, функцию $p_\eta(y)$ удаётся найти дифференцированием левой и правой частей равенства \eqref{28:density}, но в данном случае в точке $-1$ функция
    $F_\eta(y)$ не является дифференцируемой и имеет разрыв величиной:
    \[
        \lim \limits_{y \rightarrow -1 + 0} F_\eta(y) - F_\eta(-1)
        = e^{-\lambda \left (- (-1) + 1 \right )} - e^{-\lambda \left ( - \frac{-1 + 1}{2} + 4 \right )}
        = e^{-2 \lambda} - e^{- 4 \lambda}
        .
    \]
    Тем не менее, используя интегральное свойство дельта-функции ($\varepsilon > 0$):
    \[
        \int \limits_{-\varepsilon}^{\varepsilon} \delta(x) dx = 1 ,
    \]
    можно получить выражение для плотности распределения $p_\eta(y)$:
    \[
        p_\eta(y)
        =
        \left \{
        \begin{array}{ll}
            \frac{\lambda}{2} e^{ -\lambda \left ( - \frac{y+1}{2} + 4 \right )}, & y < -1       \\
            \lambda e^{ -\lambda \left ( - y + 1 \right )},                       & -1 < y \le 1 \\
            0,                                                                    & 1 < y

        \end{array}
        \right .
        + \left ( e^{-2 \lambda} - e^{- 4 \lambda} \right ) \delta(y+1)
        .
    \]

    \subsection*{Ответ:}
    \begin{enumerate}
        \item
        $
        F_\eta(y)
        =
        \left \{
        \begin{array}{ll}
            e^{ -\lambda \left ( - \frac{y+1}{2} + 4 \right )}, & y \le -1     \\
            e^{ -\lambda \left ( - y + 1 \right )},             & -1 < y \le 1 \\
            1,                                                  & 1 < y
        \end{array}
        \right . ,
        $
        \item
        $
        p_\eta(y)
        =
        \left \{
        \begin{array}{ll}
            \frac{\lambda}{2} e^{ -\lambda \left ( - \frac{y+1}{2} + 4 \right )}, & y < -1       \\
            \lambda e^{ -\lambda \left ( - y + 1 \right )},                       & -1 < y \le 1 \\
            0,                                                                    & 1 < y

        \end{array}
        \right .
        + \left ( e^{-2 \lambda} - e^{- 4 \lambda} \right ) \delta(y+1)
        .
        $
    \end{enumerate}

\end{document}