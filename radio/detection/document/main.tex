\documentclass[a4paper,12pt]{article}
\usepackage[T1]{fontenc}
\usepackage[utf8]{inputenc}
\usepackage[english,russian]{babel}
\usepackage[margin=2cm]{geometry}
\usepackage{amsmath}
\usepackage{amssymb}
\usepackage{stmaryrd}
\usepackage{amsfonts}
\usepackage{tikz}
\usepackage{subcaption}

\newcommand{\expectation}[1]{\mathtt{M} \left [ #1 \right ]}
\newcommand{\variance}[1]{\mathtt{D} \left [ #1 \right ]}
\newcommand{\covariance}[2]{\mathtt{cov} \left ( #1 , #2 \right )}
\newcommand{\probability}[1]{P \left \{ #1 \right \}}

\newcommand{\modulus}[1]{\left | #1 \right |}

\newcommand{\fd}[1]{ \frac{d}{d #1}}
\newcommand{\fda}[2]{ \frac{d #1}{d #2}}
\newcommand{\md}[2]{ \frac{d^{#1}}{d #2^{#1}}}

\begin{document}

    \title{Корреляционный детектор}
    \author{Тигетов Давид}
    \maketitle

    \tableofcontents


    \section{Обнаружение сигнала}

    \textbf{Разделы 3.3 -- 3.7 книги \cite{SCHIRMAN}.}

    Принимается сигнал $y(t)$, который всегда содержит шумы $n(t)$ и возможно содержит отраженный от объекта сигнал $x(y; \alpha; \beta)$,
    $\alpha$ --- набор фиксируемых параметров (амплитуда, временная задержка, доплеровское смещение отраженного сигнала), $\beta$ --- неизвестные (случайные)
    параметры. В отношении сигнала $y(t)$ выдвигаются две гипотезы:
    \begin{itemize}
        \item $H_0: y(t) = n(t)$ --- нет отраженного сигнала,
        \item $H_1: y(t) = n(t) + x(t; \alpha; \beta)$ --- есть отраженный сигнал.
    \end{itemize}

    Вводится дискретизация по времени с некоторым шагом $\Delta t$, вычисляемым в соответствии с теоремой Котельникова по максимальной частоте спектра принимаемого
    сигнала, --- получаются времена $t_i$ и принимаемый сигнал $y(t)$ заменяется вектором значений $Y = \left ( y(t_1), y(t_2), \dots, y(t_n) \right )^T$.

    Гипотезы о наличии и отсутствии отраженного сигнала переходят в гипотезы о распределении $p_Y(Y)$ вектора значений $Y$:
    \begin{itemize}
        \item $H_0: p_Y(Y) = p_n(Y)$,
        \item $H_1: p_Y(Y) = p_{sn}(Y; \alpha | \beta)$,
    \end{itemize}
    где $p_n(Y)$ --- распределение в случае шума, $p_{sn}(Y; \alpha | \beta)$ --- распределение в случае смеси шума и сигнала (считается условным распределением
    относительно $\beta$). Возникает необходимость построения разбиения $\left ( \Gamma_0, \Gamma_1 \right )$ в множестве всех реализаций вектора $Y$.

    Если наборы параметров $\alpha$ и $\beta$ фиксированы, то задача преобразуется в задачу о различении двух простых гипотез и решается введением критерия
    качества --- минимаксного, байесовского или равномерно наиболее мощного (Неймана-Пирсона) --- каждый из которых использует отношение правдоподобия
    $l(Y; \alpha | \beta)$:
    \[
        l(Y; \alpha | \beta) = \frac{p_{sn}(Y; \alpha | \beta)}{p_n(Y)} .
    \]

    При фиксированных $\alpha$ и $\beta$ отраженный сигнал $x(t; \alpha; \beta)$ является детерминированной функцией и порождает вектор значений
    $x_i = x(t_i; \alpha; \beta)$, которые смещают распределение шума, так что:
    \[
        p_{sn}(y_1, \dots, y_n; \alpha | \beta) = p_n ( y_1 - x_1, \dots, y_n - x_n ) .
    \]

    Если $n(t)$ --- белый гауссовский шум, тогда отношение правдоподобия имеет вид:
    \begin{multline*}
        l(y_1, \dots, y_n; \alpha | \beta)
        = \frac
        {
            e^{-\frac{(y_1 - x_1)^2 \Delta t}{N_0}} \cdot e^{-\frac{(y_2 - x_2)^2 \Delta t}{N_0}} \cdot ... \cdot e^{-\frac{(y_n - x_n)^2 \Delta t}{N_0}}
        }
        {
            e^{-\frac{y_1^2 \Delta t}{N_0}} \cdot e^{-\frac{y_2^2 \Delta t}{N_0}} \cdot ... \cdot e^{-\frac{y_n^2 \Delta t}{N_0}}
        }
        = \\
        %
        = e^{-\frac{1}{N_0} \sum_{i=1}^n x_i^2 \Delta t} \cdot e^{-\frac{2}{N_0} \sum_{i=1}^n y_i x_i \Delta t}
    \end{multline*}
    где $\frac{N_0}{2}$ --- мощность (дисперсия) шума.

    Если шаг дискретизации $\Delta t$ стремится к нулю, то суммы в показателях стремятся к интегралам:
    \begin{gather*}
        \sum_{i=1}^n x_i^2 \Delta t \rightarrow \int \limits_{-\infty}^{\infty} x^2(t; \alpha; \beta) dt = E(\alpha; \beta) , \\
        \sum_{i=1}^n y_i x_i \Delta t \rightarrow \int \limits_{-\infty}^{\infty} y(t) x(t; \alpha; \beta) dt = z(y(t); \alpha | \beta) .
    \end{gather*}
    где $E(\alpha | \beta)$ --- энергия сигнала $x(t; \alpha | \beta)$ и $z(y(t); \alpha | \beta)$ --- корреляционный интеграл.

    Таким образом, отношение правдоподобия имеет вид:
    \[
        l(y(t); \alpha | \beta) = e^{-\frac{E(\alpha | \beta)}{N_0}} \cdot e^{\frac{2 z(y(t); \alpha | \beta)}{N_0}}
    \]

    Формирование единого критерия предполагается делать путем усреднением с учётом вероятностного распределения $p_\beta(b)$ случайного параметра $\beta$.
    Отношение праводоподобия усредняется по множеству $B$ значений $\beta$:
    \[
        l(y(t); \alpha) = \int \limits_B l(y(t); \alpha | b) p_\beta(b) d b .
    \]
    В случае гауссовского белого шума:
    \[
        l(y(t); \alpha) = \int \limits_B e^{-\frac{E(\alpha | b)}{N_0}} \cdot e^{\frac{2 z(y(t); \alpha | b)}{N_0}} p_\beta(b) d b .
    \]


    \section{Отраженный сигнал}

    \textbf{Раздел 3.8 книги \cite{SCHIRMAN}.}

    Пусть отраженный сигнал состоит из ненулевого и конечного количества периодов $L$ гармонических колебаний:
    \[
        x(t | \beta_1, \beta_2)
        =
        \left \{
        \begin{array}{ll}
            \beta_1 \cos ( \omega t - \beta_2 ) , & t \in \left [ 0, \frac{2 \pi}{\omega} L \right ]    \\
            0,                                    & t \notin \left [ 0, \frac{2 \pi}{\omega} L \right ]
        \end{array}
        \right .
    \]
    где $\beta_1$ и $\beta_2$ --- независимые случайные величины, $\beta_1$ --- случайная амплитуда, имеющая распределение Релея, и $\beta_2$ --- случайная начальная фаза,
    равномерно распределенная на отрезке $[0; 2 \pi]$.

    Отраженный сигнал можно представить в виде суммы:
    \begin{gather*}
        x(t | \beta_1, \beta_2 ) = \beta_1 \cos ( \omega t ) \cos \beta_2 + \beta_1 \sin ( \omega t ) \sin \beta_2 , \\
        x(t | \beta_1, \beta_2 ) = \beta_1 \left ( x_1(t) \cos \beta_2 + x_2(t) \sin \beta_2 \right ), \\
        x_1(t) = \cos ( \omega t ) , \\
        x_2(t) = \sin ( \omega t ) .
    \end{gather*}

    \subsection{Отношение правдоподобия}

    Энергия сигнала:
    \begin{multline*}
        E(\beta_1, \beta_2)
        = \int \limits_{-\infty}^{\infty} x^2(t | \beta ) dt
        = \int \limits_{0}^{\frac{2 \pi}{\omega} L} \beta_1^2 \left ( x_1(t) \cos \beta_2 + x_2(t) \sin \beta_2 \right )^2 dt = \\
        %
        = \beta_1^2 \int \limits_{0}^{\frac{2 \pi}{\omega} L} \left ( x_1(t) \cos \beta_2 + x_2(t) \sin \beta_2 \right )^2 dt = \\
        %
        = \beta_1^2 \left (
        \cos^2 \beta_2 \int \limits_{0}^{\frac{2 \pi}{\omega} L} x_1^2(t) dt
        + 2 \cos \beta_2 \sin \beta_2 \int \limits_{0}^{\frac{2 \pi}{\omega} L} x_1(t) x_2(t) dt
        + \sin^2 \beta_2 \int \limits_{0}^{\frac{2 \pi}{\omega} L} x_2^2(t) dt
        \right ) = \\
        %
        = \beta_1^2 \left (
        \cos^2 \beta_2 \int \limits_{0}^{\frac{2 \pi}{\omega} L} \cos^2 (\omega t) dt
        + 2 \cos \beta_2 \sin \beta_2 \int \limits_{0}^{\frac{2 \pi}{\omega} L} \cos (\omega t) \sin (\omega t) dt
        + \sin^2 \beta_2 \int \limits_{0}^{\frac{2 \pi}{\omega} L} \sin^2 (\omega t) dt
        \right ) = \\
        %
        = \beta_1^2 \left (
        \cos^2 \beta_2 \int \limits_{0}^{\frac{2 \pi}{\omega} L} \frac{1 + \cos 2 \omega t }{2} dt
        + 2 \cos \beta_2 \sin \beta_2 \int \limits_{0}^{\frac{2 \pi}{\omega} L} \frac{1}{2} \sin (2 \omega t) dt
        + \sin^2 \beta_2 \int \limits_{0}^{\frac{2 \pi}{\omega} L} \frac{1 - \cos 2 \omega t}{2} dt
        \right ) = \\
        %
        = \beta_1^2 \left ( \cos^2 \beta_2 \frac{1}{2} \frac{2 \pi}{\omega} L + \sin^2 \beta_2 \frac{1}{2} \frac{2 \pi}{\omega} L \right )
        = \beta_1^2 \frac{\pi}{\omega} L .
    \end{multline*}
    Таким образом, энергия зависит только от случайной амлитуды $\beta_1$, и не зависит от случайной фазы $\beta_2$:
    \[
        E(\beta_1, \beta_2) = \beta_1^2 E,
    \]
    где
    \[
        E
        = \int \limits_{0}^{\frac{2 \pi}{\omega} L} \cos^2 \left ( \omega t - \beta_2 \right ) dt
        = \int \limits_{0}^{\frac{2 \pi}{\omega} L} \cos^2 \left ( \omega t \right ) dt
        = \int \limits_{0}^{\frac{2 \pi}{\omega} L} \sin^2 \left ( \omega t \right ) dt
        = \frac{\pi}{\omega} L
        .
    \]

    Корреляционный интеграл:
    \begin{multline*}
        z(y(t) | \beta)
        = \int \limits_{-\infty}^{\infty} y(t) x(t | \beta_1, \beta_2) dt
        = \int \limits_0^{\frac{2 \pi}{\omega} L} y(t) \beta_1 \left ( x_1(t) \cos \beta_2 + x_2(t) \sin \beta_2 \right ) dt = \\
        %
        = \beta_1 \left ( \int \limits_0^{\frac{2 \pi}{\omega} L} y(t) x_1(t) \cos \beta_2 dt + \int \limits_0^{\frac{2 \pi}{\omega} L} y(t) x_2(t) \sin \beta_2 dt \right ) = \\
        %
        = \beta_1 \left ( \int \limits_0^{\frac{2 \pi}{\omega} L} y(t) x_1(t) dt \cdot \cos \beta_2 + \int \limits_0^{\frac{2 \pi}{\omega} L} y(t) x_2(t) dt \cdot \sin \beta_2 \right ) = \\
        %
        = \beta_1 \left ( z_1 \cos \beta_2 + z_2 \sin \beta_2 \right ) = \\
        %
        = \beta_1 \sqrt{z_1^2 + z_2^2} \left ( \frac{z_1}{\sqrt{z_1^2 + z_2^2}} \cos \beta_2 + \frac{z_2}{\sqrt{z_1^2 + z_2^2}} \sin \beta_2 \right )
        = \beta_1 Z \left ( \cos \theta \cos \beta_2 + \sin \theta \sin \beta_2 \right ) = \\
        %
        = \beta_1 Z \cos ( \theta - \beta_2 ) ,
    \end{multline*}
    где
    \begin{align*}
        z_1         & = \int \limits_0^{\frac{2 \pi}{\omega} L} y(t) x_1(t) dt = \int \limits_0^{\frac{2 \pi}{\omega} L} y(t) \cos ( \omega t) dt , \\
        z_2         & = \int \limits_0^{\frac{2 \pi}{\omega} L} y(t) x_2(t) dt = \int \limits_0^{\frac{2 \pi}{\omega} L} y(t) \sin ( \omega t) dt , \\
        Z           & = \sqrt{z_1^2 + z_2^2} , \\
        \cos \theta & = \frac{z_1}{\sqrt{z_1^2 + z_2^2}} , \\
        \sin \theta & = \frac{z_2}{\sqrt{z_1^2 + z_2^2}} .
    \end{align*}

    Отношение правдоподобия
    \[
        l(y(t) | \beta_1, \beta_2 ) = e^{-\frac{\beta_1^2 E}{N_0}} \cdot e^{\frac{2 \beta_1 Z}{N_0} \cos ( \theta - \beta_2)} ,
    \]
    и среднее значение отношения правдоподобия (с усреднением по амплитуде $\beta_1$, имеющей распределение Релея, и начальной фазе $\beta$, имеющей равномерное
    распределение на отрезке $[0, 2 \pi]$):
    \begin{multline*}
        l(y(t))
        = \int \limits_0^{\infty} \left ( \int \limits_0^{2 \pi} l(y(t) | b_1, b_2) \frac{1}{2 \pi} d b_2 \right ) \frac{b_1}{B_1^2} e^{- \frac{b_1^2}{2 B_1^2}} d b_1 = \\
        %
        = \int \limits_0^{\infty} \left ( \int \limits_0^{2 \pi} e^{-\frac{b_1^2 E}{N_0}} \cdot e^{\frac{2 b_1 Z}{N_0} \cos ( \theta - b_2)} \frac{1}{2 \pi} d b_2 \right ) \frac{b_1}{B_1^2} e^{- \frac{b_1^2}{2 B_1^2}} d b_1 = \\
        %
        = \int \limits_0^{\infty} e^{-\frac{b_1^2 E}{N_0}} \left ( \frac{1}{2 \pi} \int \limits_0^{2 \pi} e^{\frac{2 b_1 Z}{N_0} \cos ( \theta - b_2)} d b_2 \right ) \frac{b_1}{B_1^2} e^{- \frac{b_1^2}{2 B_1^2}} d b_1 = \\
        %
        = \int \limits_0^{\infty} e^{-\frac{b_1^2 E}{N_0}} I_0 \left ( \frac{2 b_1 Z}{N_0} \right ) \frac{b_1}{B_1^2} e^{- \frac{b_1^2}{2 B_1^2}} d b_1 .
    \end{multline*}
    где $I_0(\cdot)$ --- модифицированная функция Бесселя первого рода нулевого порядка. Уже отсюда видно, что среднее отношение правдоподобия $l(y(t))$ является монотонной
    функцией модульного значения корреляционного интеграла $Z$. Впрочем, можно вычислить интеграл, предварительно преобразовав его к виду:
    \[
        l(y(t))
        = \int \limits_0^{\infty} \frac{b_1}{B_1^2} I_0 \left ( \frac{2 b_1 Z}{N_0} \right ) e^{- \frac{b_1^2}{2 B_1^2} -\frac{b_1^2 E}{N_0}} d b_1
        = \int \limits_0^{\infty} \frac{b_1}{B_1} I_0 \left ( \frac{2 Z B_1}{N_0} \frac{b_1}{B_1} \right ) e^{- \frac{b_1^2}{B_1^2} \left ( \frac{1}{2} + \frac{B_1^2 E}{N_0} \right )} \frac{d b_1}{B_1}.
    \]
    Далее, используя табличный интеграл:
    \[
        \int \limits_0^{\infty} x I_0(\mu x) e^{-\nu x^2} dx = \frac{1}{2 \nu} e^{\frac{\mu^2}{4 \nu}},
    \]
    получим:
    \begin{multline*}
        l(y(t))
        = \frac{1}{2 \left ( \frac{1}{2} + \frac{B_1^2 E}{N_0} \right )} e^{\frac{\left ( \frac{2 Z B_1}{N_0} \right )^2}{4 \left ( \frac{1}{2} + \frac{B_1^2 E}{N_0} \right )}}
        = \frac{1}{1 + \frac{B_1^2 E}{\frac{N_0}{2}}} e^{\frac{B_1^2 \left ( \frac{Z}{\frac{N_0}{2}} \right )^2}{2 \left ( 1 + \frac{B_1^2 E}{\frac{N_0}{2}} \right )}} = \\
        %
        = \frac{\frac{N_0}{2}}{\frac{N_0}{2} + B_1^2 E} e^{\frac{1}{\frac{N_0}{2}} \frac{B_1^2 Z^2}{2 \left ( \frac{N_0}{2} + B_1^2 E \right )}}
        = \frac{\frac{N_0}{2}}{\frac{N_0}{2} + E} e^{\frac{1}{\frac{N_0}{2}} \frac{B_1^2 Z^2}{2 \left ( \frac{N_0}{2} + E \right )}}
        .
    \end{multline*}

    Отношение правдоподобия $l(y(t))$ является монотонно убывающей функцией энергии $E$ и монотонно возрастающей функцией модульного значения корреляционного интеграла $Z$,
    что позволяет перейти к эквивалентному условию:
    \[
        \left \{ l(y(t)) > l_0 \right \} \Leftrightarrow \left \{ Z > Z_0 \right \} ,
    \]
    и далее рассматривать только величину $Z$.

    \subsection{Шум}

    Если принятый сигнал содержит только белый шум --- $y(t) = n(t)$, то компоненты корреляционного интеграла имеют нулевые математические ожидания:
    \[
        \expectation{z_{1,2}}
        = \expectation{\int \limits_{0}^{\frac{2 \pi}{\omega}L} n(t) x_{1,2}(t) dt}
        = \int \limits_{0}^{\frac{2 \pi}{\omega}L} \expectation{n(t)} x_{1,2}(t) dt
        = \int \limits_{0}^{\frac{2 \pi}{\omega}L} 0 \cdot x_{1,2}(t) dt
        = 0 ,
    \]
    и дисперсии:
    \begin{equation}
        \label{equation:signal_model:noise:variance}
        \variance{z_{1,2}}
        = \expectation{z_{1,2}^2} - \left ( \expectation{z_{1,2}} \right )^{2}
        = \expectation{z_{1,2}^2},
    \end{equation}
    \begin{multline}
        \label{equation:signal_model:noise:squared_moment}
        \expectation{z_{1,2}^2}
        = \expectation{\int \limits_{0}^{\frac{2 \pi}{\omega}L} n(t) x_{1,2}(t) dt \cdot \int \limits_{0}^{\frac{2 \pi}{\omega}L} n(s) x_{1,2}(s) ds}
        = \expectation{\int \limits_{0}^{\frac{2 \pi}{\omega}L} \int \limits_{0}^{\frac{2 \pi}{\omega}L} n(t) n(s) x_{1,2}(t) x_{1,2}(s) ds dt} = \\
%
        = \int \limits_{0}^{\frac{2 \pi}{\omega}L} \int \limits_{0}^{\frac{2 \pi}{\omega}L} \expectation{n(t) n(s)} x_{1,2}(t) x_{1,2}(s) ds dt
        = \int \limits_{0}^{\frac{2 \pi}{\omega}L} \int \limits_{0}^{\frac{2 \pi}{\omega}L} \frac{N_0}{2} \delta (t-s) x_{1,2}(t) x_{1,2}(s) ds dt = \\
        = \frac{N_0}{2} \int \limits_{0}^{\frac{2 \pi}{\omega}L} x_{1,2}(t) x_{1,2}(t) dt
        = \frac{N_0}{2} \int \limits_{0}^{\frac{2 \pi}{\omega}L} x_{1,2}^2(t) dt .
    \end{multline}
    Таким образом, дисперсии компонент $z_1$ и $z_2$ имеют вид:
    \begin{gather*}
        \variance{z_1}
        = \frac{N_0}{2} \int \limits_{0}^{\frac{2 \pi}{\omega}L} x_1^2(t) dt
        = \frac{N_0}{2} \int \limits_{0}^{\frac{2 \pi}{\omega}L} \cos^2(\omega t) dt
        = \frac{N_0}{2} \cdot E, \\
        %
        \variance{z_2}
        = \frac{N_0}{2} \int \limits_{0}^{\frac{2 \pi}{\omega}L} x_2^2(t) dt
        = \frac{N_0}{2} \int \limits_{0}^{\frac{2 \pi}{\omega}L} \sin^2(\omega t) dt
        = \frac{N_0}{2} \cdot E .
    \end{gather*}
    и равны между собой:
    \[
        \variance{z_1} = \variance{z_2} = \nu_0^2 .
    \]
    Ковариация компонент:
    \begin{multline*}
        \covariance{z_1}{z_2}
        = \covariance{\int \limits_{0}^{\frac{2 \pi}{\omega}L} n(t) x_1(t) dt}{\int \limits_{0}^{\frac{2 \pi}{\omega}L} n(s) x_2(s) ds} = \\
        %
        = \expectation{\int \limits_{0}^{\frac{2 \pi}{\omega}L} n(t) x_1(t) dt \cdot \int \limits_{0}^{\frac{2 \pi}{\omega}L} n(s) x_2(s) ds}
        = \expectation{\int \limits_{0}^{\frac{2 \pi}{\omega}L} \int \limits_{0}^{\frac{2 \pi}{\omega}L} n(t) x_1(t) n(s) x_2(s) ds dt} = \\
        %
        = \int \limits_{0}^{\frac{2 \pi}{\omega}L} \int \limits_{0}^{\frac{2 \pi}{\omega}L} \expectation{n(t) n(s)} x_1(t) x_2(s) ds dt
        = \int \limits_{0}^{\frac{2 \pi}{\omega}L} \int \limits_{0}^{\frac{2 \pi}{\omega}L} \frac{N_0}{2} \delta(t-s) x_1(t) x_2(s) ds dt = \\
        %
        = \frac{N_0}{2} \int \limits_{0}^{\frac{2 \pi}{\omega}L} x_1(t) x_2(t) dt
        = \frac{N_0}{2} \int \limits_{0}^{\frac{2 \pi}{\omega}L} \cos( \omega t ) \sin( \omega t ) dt
        = \frac{N_0}{2} \int \limits_{0}^{\frac{2 \pi}{\omega}L} \frac{1}{2} \sin( 2 \omega t ) dt
        = 0 .
    \end{multline*}
    Поскольку $n(t)$ --- гауссовский шум, то компоненты $z_1$ и $z_2$ считаются гауссовскими, а поскольку из ковариация равна нулю, то компоненты $z_1$ и $z_2$ являются
    независимыми.

    В этом случае получается, что модульное значение $Z = \sqrt{z_1^2 + z_2^2}$ имеет распределение Релея:
    \begin{equation}
        \label{noise:correlation_integral_pdf}
        p_{n,Z}(z) = \frac{z}{\nu_0^2} e^{- \frac{z^2}{2 \nu_0^2}} .
    \end{equation}

    \subsection{Шум и сигнал}

    Пусть теперь в принятом сигнале содержится сигнал $x(t)$:
    \[
        y(t) = n(t) + x(t | \beta_1, \beta_2) ,
    \]
    тогда компоненты корреляционного интеграла
    \begin{multline*}
        z_{1,2}
        = \int \limits_{0}^{\frac{2 \pi}{\omega}L} y(t) x_{1,2}(t) dt
        = \int \limits_{0}^{\frac{2 \pi}{\omega}L} ( n(t) + x(t;\beta) ) x_{1,2}(t) dt = \\
%
        = \int \limits_{0}^{\frac{2 \pi}{\omega}L} n(t) x_{1,2}(t) dt + \int \limits_{0}^{\frac{2 \pi}{\omega}L} x(t;\beta) x_{1,2}(t) dt = \\
%
        = \int \limits_{0}^{\frac{2 \pi}{\omega}L} n(t) x_{1,2}(t) dt + \int \limits_{0}^{\frac{2 \pi}{\omega}L} \left ( x_1(t) \cos \beta + x_2(t) \sin \beta \right ) x_{1,2}(t) dt = \\
        = \int \limits_{0}^{\frac{2 \pi}{\omega}L} n(t) x_{1,2}(t) dt + \cos \beta \int \limits_{0}^{\frac{2 \pi}{\omega}L} x_1(t) x_{1,2}(t) dt + \sin \beta \int \limits_{0}^{\frac{2 \pi}{\omega}L} x_2(t) x_{1,2}(t) dt
    \end{multline*}
    имеют математическое ожидание:
    \begin{multline*}
        \expectation{z_{1,2}}
        = \expectation{\int \limits_{0}^{\frac{2 \pi}{\omega}L} n(t) x_{1,2}(t) dt}
        + \expectation{\cos \beta \int \limits_{0}^{\frac{2 \pi}{\omega}L} x_1(t) x_{1,2}(t) dt}
        + \expectation{\sin \beta \int \limits_{0}^{\frac{2 \pi}{\omega}L} x_2(t) x_{1,2}(t) dt} = \\
        %
        = \int \limits_{0}^{\frac{2 \pi}{\omega}L} \expectation{n(t)} x_{1,2}(t) dt
        + \cos \beta \int \limits_{0}^{\frac{2 \pi}{\omega}L} x_1(t) x_{1,2}(t) dt
        + \sin \beta \int \limits_{0}^{\frac{2 \pi}{\omega}L} x_2(t) x_{1,2}(t) dt = \\
        %
        = \cos \beta \int \limits_{0}^{\frac{2 \pi}{\omega}L} x_1(t) x_{1,2}(t) dt
        + \sin \beta \int \limits_{0}^{\frac{2 \pi}{\omega}L} x_2(t) x_{1,2}(t) dt .
    \end{multline*}
    Поскольку
    \[
        \int \limits_{0}^{\frac{2 \pi}{\omega}L} x_1(t) x_2(t) dt
        = \int \limits_{0}^{\frac{2 \pi}{\omega}L} \sin( \omega t ) \cos( \omega t ) dt
        = \int \limits_{0}^{\frac{2 \pi}{\omega}L} \frac{1}{2} \sin( 2 \omega t ) dt
        = 0,
    \]
    то
    \begin{gather*}
        \expectation{z_1}
        = \cos \beta \int \limits_{0}^{\frac{2 \pi}{\omega}L} x_1^2(t) dt + \sin \beta \int \limits_{0}^{\frac{2 \pi}{\omega}L} x_1(t) x_2(t) dt
        = \cos \beta \int \limits_{0}^{\frac{2 \pi}{\omega}L} \cos^2 ( \omega t ) dt
        = \cos \beta \cdot E, \\
        %
        \expectation{z_2}
        = \cos \beta \int \limits_{0}^{\frac{2 \pi}{\omega}L} x_1(t) x_2(t) dt + \sin \beta \int \limits_{0}^{\frac{2 \pi}{\omega}L} x_2^2(t) dt
        = \sin \beta \int \limits_{0}^{\frac{2 \pi}{\omega}L} \sin^2 ( \omega t ) dt
        = \sin \beta \cdot E .
    \end{gather*}

    Дисперсия компонент $z_1$ и $z_2$ остаётся прежней (как в равенствах \eqref{equation:signal_model:noise:variance} и \eqref{equation:signal_model:noise:squared_moment}):
    \[
        \variance{z_{1,2}} = \frac{N_0}{2} \int \limits_{0}^{\frac{2 \pi}{\omega}L} x_{1,2}^2(t) dt = \nu_0^2,
    \]
    поскольку часть
    \[
        \cos \beta \int \limits_{0}^{\frac{2 \pi}{\omega}L} x_1(t) x_{1,2}(t) dt + \sin \beta \int \limits_{0}^{\frac{2 \pi}{\omega}L} x_2(t) x_{1,2}(t) dt = const
    \]
    не является случайной.

    В силу смещения математических ожиданий компонент $z_1$ и $z_2$ модульное значение корреляционного интеграла $Z$ имеет обобщенное распределение Релея:
    \begin{equation}
        \label{noise_and_signal:correlation_integral_pdf}
        p_{sn,Z}(z ; E) = \frac{z}{\nu_0^2} e^{- \frac{z^2 + E^2}{2 \nu_0^2}} I_0 \left ( \frac{z E}{\nu_0^2} \right ) .
    \end{equation}

    \subsection{Кривые обнаружения}

    Критерий обнаружения связан с построением разбиения $\left \{ \Gamma_0, \Gamma_1 \right \}$ множества значений статистики $Z$:
    \[
        \Gamma_0 = \left \{ z: z < z_0 \right \} , \Gamma_1 = \left \{ z: z \ge z_0 \right \} .
    \]

    Характеристиками процедуры обнаружения являются две вероятности: $F$ --- ложного обнаружения отраженного сигнала и $D$ --- правильного обнаружения отраженного сигнала,
    \begin{align*}
        F(z_0) = & \int \limits_{z_0}^{\infty} p_{n,Z}(z) dz , \\
        D(z_0; E) = & \int \limits_{z_0}^{\infty} p_{sn,Z}(z ; E) dz ,
    \end{align*}
    где $p_{n,Z}(z)$ --- условная плотность вероятности в случае, когда принятый сигнал содержит только шум, и $p_{sn,Z}(z | E)$ --- условная плотность вероятности
    статистики $Z$ в случае, когда принятый сигнал является смесью шума и сигнала, которая зависит от энергии сигнала $E$.

    По заданной величине $\widetilde{F}$ фиксируют уровень $z_0$:
    \[
        F(z_0) = \widetilde{F},
    \]
    тогда вероятность правильного обнаружения $D(z_0 ; E)$ становится функцией только энергии $E$, и график этой функции называется кривой обнаружения.


    \section{Показатели при квадратичном суммировании}

    \textbf{Приложение 4 (к разделу 3.19)} из книги \cite{SCHIRMAN}.

    В случае смеси шума и отраженного сигнала модульное значение корреляционного интеграла $Z$ имеет обобщенное распределение Релея:
    \[
        p_{sn,Z}(z ; E)
        = \left \{
        \begin{array}{ll}
            0,                                                                                             & z < 0   \\
            \frac{z}{\nu_0^2} I_0 \left ( \frac{z E}{\nu_0^2} \right ) e^{- \frac{E^2 + z^2}{2 \nu_0^2}} , & z \ge 0
        \end{array}
        \right .
    \]
    которое можно представить в виде:
    \[
        p_{sn,Z}(z ; E)
        = \left \{
        \begin{array}{ll}
            0,                                                                                                                                                                            & z < 0   \\
            \frac{z}{\nu_0} \frac{1}{\nu_0} I_0 \left ( \frac{z}{\nu_0} \frac{E}{\nu_0} \right ) e^{- \frac{\left ( \frac{E}{\nu_0} \right )^2 + \left ( \frac{z}{\nu_0} \right)^2}{2}} , & z \ge 0
        \end{array}
        \right .
        .
    \]
    Для сокращения дальнейшего вывода перейдем к величине $U = \frac{Z}{\nu_0}$ представляющей значение $Z$ нормированное на с.к.о. $\nu_0$ компонент корреляционного интеграла в случае шума:
    \begin{multline*}
        p_{sn,U}(u ; E)
        = \left . p_{sn,Z}(z; E) \fda{Z}{U} \right |_{z = u \nu_0}
        = p_{sn,Z}(u \nu_0; E) \nu_0 = \\
        %
        = \left \{
        \begin{array}{ll}
            0,                                                                                                                        & u < 0   \\
            u \frac{1}{\nu_0} I_0 \left ( u \frac{E}{\nu_0} \right ) e^{- \frac{\left ( \frac{E}{\nu_0} \right )^2 + u^2}{2}} \nu_0 , & u \ge 0
        \end{array}
        \right .
        = \left \{
        \begin{array}{ll}
            0,                                                                                                  & u < 0   \\
            u I_0 \left ( u \frac{E}{\nu_0} \right ) e^{- \frac{\left ( \frac{E}{\nu_0} \right )^2 + u^2}{2}} , & u \ge 0
        \end{array}
        \right .
        .
    \end{multline*}
    Введем величину:
    \[
        q
        = \frac{E}{\nu_0}
        = \frac{E}{\sqrt{\frac{N_0}{2} E}}
        = \sqrt{\frac{E}{\frac{N_0}{2}}},
    \]
    тогда
    \[
        p_{sn,U}(u)
        = \left \{
        \begin{array}{ll}
            0,                                    & u < 0   \\
            u I_0(qu) e^{- \frac{q^2 + u^2}{2}} , & u \ge 0
        \end{array}
        \right .
    \]

    Квадрат $U$ имеет плотность:
    \[
        p_{U^2}(y)
        = p_U(u) \left . \frac{dU}{dU^2} \right |_{u = \sqrt{y}}
        = \left \{
        \begin{array}{ll}
            0 ,                                                                    & y < 0   \\
            \sqrt{y} I_0(q \sqrt{y}) e^{- \frac{q^2 + y}{2}} \frac{1}{2 \sqrt{y}}, & y \ge 0
        \end{array}
        \right .
        = \left \{
        \begin{array}{ll}
            0 ,                                                  & y < 0   \\
            \frac{1}{2} I_0(q \sqrt{y}) e^{- \frac{q^2 + y}{2}}, & y \ge 0
        \end{array}
        \right .
        .
    \]
    Характеристическая функция величины $U^2$:
    \begin{multline*}
        \theta_{U^2}(s)
        = \int \limits_{- \infty}^{\infty} e^{j s y} p_{U^2}(y) dy
        = \int \limits_{0}^{\infty} e^{j s y} \frac{1}{2} I_0(q \sqrt{y}) e^{- \frac{q^2 + y}{2}} dy
        = \int \limits_{0}^{\infty} e^{j s u^2} \frac{1}{2} I_0(q u) e^{- \frac{q^2 + u^2}{2}} 2 u du = \\
        %
        = e^{- \frac{q^2}{2}} \int \limits_{0}^{\infty} e^{- \frac{u^2}{2} ( 1 - 2 j s)} I_0(q u) u du
        = e^{- \frac{q^2}{2}} \int \limits_{0}^{\infty} e^{- \frac{u^2}{2} ( 1 - 2 j s)} \frac{1}{2 \pi} \int \limits_0^{2 \pi} e^{q u \cos \alpha} d \alpha u du = \\
        %
        = e^{- \frac{q^2}{2}} \frac{1}{2 \pi} \int \limits_{0}^{\infty} \int \limits_0^{2 \pi} u e^{- \frac{u^2}{2} ( 1 - 2 j s) + q u \cos \alpha} d \alpha du .
    \end{multline*}
    Замена переменных $x = u \cos \alpha$ и $y = u \sin \alpha$. Якобиан замены $\modulus{\frac{\partial (x,y)}{\partial (u,\alpha)}} = u$ значит якобиан обратной замены
    $\modulus{\frac{\partial (u,\alpha)}{\partial (x,y)}} = \frac{1}{u} = \frac{1}{\sqrt{x^2 + y^2}}$. Характеристическая функция:
    \begin{multline*}
        \theta_{U^2}(s)
        = e^{- \frac{q^2}{2}} \frac{1}{2 \pi} \int \limits_{-\infty}^{\infty} \int \limits_{-\infty}^{\infty} \sqrt{x^2 + y^2} e^{- \frac{x^2 + y^2}{2} ( 1 - 2 j s) + q x} \frac{1}{\sqrt{x^2 + y^2}}dx dy = \\
        %
        = e^{- \frac{q^2}{2}} \frac{1}{2 \pi} \int \limits_{-\infty}^{\infty} \int \limits_{-\infty}^{\infty} e^{- \frac{x^2 + y^2}{2} ( 1 - 2 j s) + q x} dx dy = \\
        = e^{- \frac{q^2}{2}} \frac{1}{2 \pi} \int \limits_{-\infty}^{\infty} \int \limits_{-\infty}^{\infty} e^{- \frac{x^2}{2} ( 1 - 2 j s) + q x - \frac{y^2}{2} ( 1 - 2 j s)} dx dy = \\
        %
        = e^{- \frac{q^2}{2}} \frac{1}{\sqrt{2 \pi}} \int \limits_{-\infty}^{\infty} e^{- \frac{x^2}{2} ( 1 - 2 j s) + q x} dx \cdot \frac{1}{\sqrt{2 \pi}} \int \limits_{-\infty}^{\infty} e^{- \frac{y^2}{2} ( 1 - 2 j s)} dx dy = \\
        %
        = e^{- \frac{q^2}{2}} \frac{1}{\sqrt{2 \pi}} \int \limits_{-\infty}^{\infty} e^{- \frac{x^2}{2 \frac{1}{1 - 2 j s}}  + q x} dx \cdot \frac{1}{\sqrt{2 \pi}} \int \limits_{-\infty}^{\infty} e^{- \frac{y^2}{2 \frac{1}{1 - 2 j s}}} dx dy = \\
        %
        = e^{- \frac{q^2}{2}} \frac{1}{\sqrt{2 \pi}} \int \limits_{-\infty}^{\infty} e^{- \frac{x^2 - 2 \frac{q}{1 - 2 j s} x}{2 \frac{1}{ 1 - 2 j s}}} dx \cdot \frac{\frac{1}{\sqrt{1 - 2 j s}}}{\sqrt{2 \pi} \frac{1}{\sqrt{1 - 2 j s}}} \int \limits_{-\infty}^{\infty} e^{- \frac{y^2}{2 \frac{1}{\sqrt{1 - 2 j s}}}} dy = \\
        %
        = e^{- \frac{q^2}{2}} \frac{1}{\sqrt{2 \pi}} \int \limits_{-\infty}^{\infty} e^{- \frac{\left ( x - \frac{q}{1 - 2 j s} \right )^2 + \frac{q^2}{(1 - 2 j s)^2}}{2 \frac{1}{1 - 2 j s}}} dx \cdot \frac{1}{\sqrt{1 - 2 j s}} = \\
        %
        = e^{- \frac{q^2}{2} + \frac{q^2}{2 (1 - 2 j s)}} \frac{1}{\sqrt{1 - 2 j s}} \cdot \frac{1}{\sqrt{1 - 2 j s}}
        = e^{- \frac{q^2}{2}} \frac{e^{\frac{q^2}{2 (1 - 2 j s)}}}{1 - 2 j s} .
    \end{multline*}

    Введем обозначение $a(s) = \frac{1}{1 - 2 j s}$ и $b = \frac{q^2}{2}$, тогда:
    \[
        \theta_{U^2}(s) = e^{-b} a(s) e^{a(s) b} .
    \]
    Рассмотрим сумму квадратов независимых величин:
    \[
        U_{\Sigma} = U_1^2 + U_2^2 + \dots + U_M^2 .
    \]

    Характеристическая функция $U_{\Sigma}$ является произведением характеристических функций $\theta_{U^2}(s)$:
    \begin{multline*}
        \theta_{U_{\Sigma}}(s)
        = \theta_{U_1}(s) \cdot \theta_{U_2}(s) \cdot ... \cdot \theta_{U_M}(s)
        = e^{- M b} a(s)^M e^{a(s) M b} = \\
        %
        = e^{- B} a(s)^M e^{a(s) B}
        = e^{- B} a(s) \md{M-1}{B} e^{a(s) B}
    \end{multline*}
    где $B = M b$.

    Плотность вероятности $U_\Sigma$ определяется по характеристической функции обратным преобразованием:
    \begin{multline*}
        p_{sn,U_\Sigma}(u)
        = \frac{1}{2 \pi} \int \limits_{-\infty}^{\infty} \theta_\Sigma(s) e^{-j s u} ds
        = \frac{1}{2 \pi} \int \limits_{-\infty}^{\infty} e^{- B} a(s) \md{M-1}{B} e^{a(s) B} e^{-j s u} ds = \\
        %
        = \frac{1}{2 \pi} \int \limits_{-\infty}^{\infty} e^{- B} \frac{1}{1 - 2 j s} \md{M-1}{B} e^{\frac{1}{1 - 2 j s} B} e^{-j s u} ds = \\
        = e^{- B} \md{M-1}{B} \left ( \frac{1}{2 \pi} \int \limits_{-\infty}^{\infty} \frac{e^{\frac{B}{1 - 2 j s}}}{1 - 2 j s}  e^{-j s u} ds \right )
        = e^{- B} \md{M-1}{B} \psi(B, u),
    \end{multline*}
    где
    \[
        \psi(B, u) = \frac{1}{2 \pi} \int \limits_{-\infty}^{\infty} \frac{e^{\frac{B}{1 - 2 j s}}}{1 - 2 j s}  e^{-j s u} ds
    \]

    Вычислять интеграл не будем, воспользуемся следующим замечанием: если в сумме $U_\Sigma$ только одно слагаемое, то $U_\Sigma = U_1^2$ и равны плотности:
    \begin{gather*}
        p_{sn,U_\Sigma}(u) = p_{U^2}(u) , \\
        e^{- b} \psi(b, u) = \frac{1}{2} I_0 \left ( q \sqrt{u} \right ) e^{- \frac{q^2 + u}{2}} , \\
        e^{- \frac{q^2}{2}} \psi(b, u) = \frac{1}{2} I_0 \left ( q \sqrt{u} \right ) e^{- \frac{q^2}{2}} e^{- \frac{u}{2}} , \\
        \psi(b, u) = \frac{1}{2} I_0 \left ( q \sqrt{u} \right ) e^{- \frac{u}{2}} , \\
        \psi(b, u) = \frac{1}{2} I_0 \left ( \sqrt{2 b} \sqrt{u} \right ) e^{- \frac{u}{2}} , \\
        \psi(b, u) = \frac{1}{2} I_0 \left ( \sqrt{2 b u} \right ) e^{- \frac{u}{2}} .
    \end{gather*}
    Отсюда следует:
    \[
        \psi(B, u) = \frac{1}{2} I_0 \left ( \sqrt{2 B u} \right ) e^{- \frac{u}{2}}
    \]
    Подставляем полученное равенство в выражение для плотности:
    \begin{multline*}
        p_{sn,U_\Sigma}(u)
        = e^{- B} \md{M-1}{B} \psi(B, u)
        = e^{- B} \md{M-1}{B} \frac{1}{2} I_0 \left ( \sqrt{2 B u} \right ) e^{- \frac{u}{2}} = \\
        %
        = \frac{1}{2} e^{- \left ( B + \frac{u}{2} \right )} \md{M-1}{B} I_0 \left ( \sqrt{2 B u} \right ) .
    \end{multline*}

    Для производных справедливо равенство:
    \[
        \md{M-1}{B} I_0 \left ( \sqrt{2 B u} \right ) = \left ( \frac{u}{2B} \right )^\frac{M-1}{2} I_{M-1} \left ( \sqrt {2 B u} \right ) ,
    \]
    которое доказывается по индукции. При $M=1$ оно является тождеством:
    \[
        I_0 \left ( \sqrt{2 B u} \right ) = I_0 \left ( \sqrt {2 B u} \right ) ,
    \]
    а дифферинцируя левые и правые части получим:
    \[
        \fd{B} \left( \md{M-1}{B} I_0 \left ( \sqrt{2 B u} \right ) \right) = \fd{B} \left( \left ( \frac{u}{2B} \right )^\frac{M-1}{2} I_{M-1} \left ( \sqrt {2 B u} \right ) \right)
    \]
    \begin{multline*}
        \md{M}{B} I_0 \left ( \sqrt{2 B u} \right ) = \\
        %
        = \frac{M-1}{2} \left ( \frac{u}{2 B} \right )^{\frac{M-1}{2} - 1} \left ( - \frac{u}{2 B^2} \right ) I_{M-1} \left ( \sqrt {2 B u} \right ) + \left( \frac{u}{2B} \right)^\frac{M-1}{2} \fd{B} I_{M-1} \left( \sqrt {2 B u} \right) = \\
        %
        = \left ( \frac{u}{2 B} \right )^\frac{M}{2} \left ( \frac{M-1}{2} \left ( \frac{u}{2 B} \right )^{-\frac{1}{2}} \left ( \frac{u}{2 B} \right )^{-1} \left ( - \frac{u}{2 B^2} \right ) I_{M-1} \left ( \sqrt {2 B u} \right ) + \fd{B} I_{M-1} \left( \sqrt {2 B u} \right ) \right ) = \\
        %
        = \left ( \frac{u}{2 B} \right )^\frac{M}{2} \left ( - \frac{M-1}{2} \left ( \frac{u}{2 B} \right )^{-\frac{1}{2}} \frac{1}{B} I_{M-1} \left ( \sqrt {2 B u} \right ) + \fd{B} I_{M-1} \left( \sqrt {2 B u} \right ) \right ) = \\
        %
        = \left ( \frac{u}{2 B} \right )^\frac{M}{2} \left ( - (M-1) \frac{\sqrt{2 B}}{2 \sqrt{u} B} I_{M-1} \left ( \sqrt {2 B u} \right ) + \fd{B} I_{M-1} \left( \sqrt {2 B u} \right ) \right ) = \\
        %
        = \left ( \frac{u}{2 B} \right )^\frac{M}{2} \left ( - (M-1) \frac{1}{\sqrt{2 u B}} I_{M-1} \left ( \sqrt {2 B u} \right ) + \fd{B} I_{M-1} \left( \sqrt {2 B u} \right ) \right ) = \\
        %
        = \left ( \frac{u}{2 B} \right )^\frac{M}{2} I_M \left ( \sqrt{2 B u}\right ) ,
    \end{multline*}
    в силу рекуррентного соотношения для функций Бесселя:
    \[
        I_M(x) = \fd{x} I_{M-1}(x) - \frac{M-1}{x} I_{M-1}(x) .
    \]

    Таким образом, плотность вероятности суммы $U_\Sigma$:
    \[
        p_{sn,U_\Sigma}(u) = \frac{1}{2} e^{- \left ( B + \frac{u}{2} \right )} \left ( \frac{u}{2 B} \right )^\frac{M-1}{2} I_{M-1} \left ( \sqrt{2 B u}\right ) ,
    \]
    где $B = M \frac{q^2}{2}$. Такое условное распределение сумма $U_\Sigma$ имеет в случае наличия отраженного сигнала.

    Если отраженного сигнала нет, то $q = 0$, обобщенное распределение Релея величин $U_i$ становится обычным распределением Релея, и $B = 0$. В этом случае, плотность
    вероятности $p_{n,U_\Sigma}(u)$ величины $U_\Sigma$ найдем как предел при $B \rightarrow 0$ плотности $p_{sn,U_\Sigma}(u)$. При вычислении предела используется
    приближенное равенство:
    \[
        \lim_{x \rightarrow 0} I_{M-1}(x) \approx \frac{1}{(M-1)!} \left ( \frac{x}{2} \right )^{M-1} .
    \]
    Таким образом,
    \begin{multline*}
        p_{n,U_\Sigma}(u)
        = \lim_{B \rightarrow 0} \frac{1}{2} e^{- \left ( B + \frac{u}{2} \right )} \left ( \frac{u}{2 B} \right )^\frac{M-1}{2} \frac{1}{(M-1)!} \left ( \frac{\sqrt{2 B u}}{2} \right )^{M-1} = \\
        %
        = \frac{1}{2} e^{- \frac{u}{2}} \lim_{B \rightarrow 0} \frac{u^\frac{M-1}{2}}{\left ( 2 B \right )^\frac{M-1}{2}} \frac{1}{(M-1)!} \frac{\left ( 2 B \right )^\frac{M-1}{2} u^\frac{M-1}{2}}{2^{M-1}} = \\
        %
        = \frac{1}{2} e^{- \frac{u}{2}} \lim_{B \rightarrow 0} \frac{1}{\left ( 2 B \right )^\frac{M-1}{2}} \frac{u^\frac{M-1}{2} u^\frac{M-1}{2}}{(M-1)! 2^{M-1}} \left ( 2 B \right )^\frac{M-1}{2}
        = \frac{1}{2} e^{- \frac{u}{2}} \frac{u^{M-1}}{(M-1)! 2^{M-1}}
        = \frac{u^{M-1}}{(M-1)! 2^{M}} e^{- \frac{u}{2}} .
    \end{multline*}

    Пусть $u_t$ --- пороговое значение принятия решения о наличии сигнала, тогда вероятности правильного обнаружения $D(M, u_t)$ и ложного обнаружения $F(M, u_t)$ имеют выражения:
    \begin{gather}
        D_M(u_t)
        = \int \limits_{u_t}^{\infty} p_{sn,U_\Sigma}(u) du
        = \int \limits_{u_t}^{\infty} \frac{1}{2} e^{- \left ( B + \frac{u}{2} \right )} \left ( \frac{u}{2 B} \right )^\frac{M-1}{2} I_{M-1} \left ( \sqrt{2 B u}\right ) du ,
        \label{quadric_pack:true_detection_probability} \\
        %
        F_M(u_t)
        = \int \limits_{u_t}^{\infty} p_{n,U_\Sigma}(u) du
        = \int \limits_{u_t}^{\infty} \frac{u^{M-1}}{(M-1)! 2^{M}} e^{- \frac{u}{2}} du
        \label{quadric_pack:false_detection_probability}
    \end{gather}
    Выполним замену переменных $t = \sqrt{\frac{u}{2}}$ ($2 t^2 = u$):
    \begin{multline*}
        D_M(u_t)
        = \int \limits_{\sqrt{\frac{u_t}{2}}}^{\infty} \frac{1}{2} e^{- \left ( B + t^2 \right )} \left ( \frac{1}{B} \right )^\frac{M-1}{2} t^{M-1} I_{M-1} \left ( 2 \sqrt{B} t \right ) 4 t dt = \\
        %
        = 2 B^{-\frac{M-1}{2}} e^{-B} \int \limits_{\sqrt{\frac{u_t}{2}}}^{\infty} t^M e^{- t^2} I_{M-1} \left ( 2 \sqrt{B} t \right ) dt,
    \end{multline*}
    \begin{gather*}
        F_M(u_t)
        = \int \limits_{u_t}^{\infty} \frac{1}{2} \frac{t^{2(M-1)}}{(M-1)!} e^{- t^2} 4 t dt
        = 2 \int \limits_{u_t}^{\infty} \frac{t^{2M-1}}{(M-1)!} e^{- t^2} dt
    \end{gather*}
    Полученные интегралы можно свести к неполной функции Торонто:
    \[
        T_\alpha (m, n, r) = 2 r^{n-m+1} e^{-r^2} \int \limits_0^{\alpha} t^{m-n} e^{-t^2} I_n(2rt) dt
    \]
    С использованием неполной функции Торонто:
    \begin{multline*}
        D_M(u_t)
        = 1 - T_{\sqrt{\frac{u_t}{2}}} \left ( 2M - 1, M - 1, \sqrt{B} \right ) = \\
        %
        = 1 - 2 \sqrt{B}^{M-1-(2M-1)+1} e^{-\sqrt{B}^2} \int \limits_0^{\sqrt{\frac{u_t}{2}}} t^{2M-1-(M-1)} e^{-t^2} I_{M-1}(2 \sqrt{B} t) dt = \\
        %
        = 1 - 2 \sqrt{B}^{-M+1} e^{- B} \int \limits_0^{\sqrt{\frac{u_t}{2}}} t^{M} e^{-t^2} I_{M-1}(2 \sqrt{B} t) dt
        = 1 - 2 B^{- \frac{M-1}{2}} e^{-B} \int \limits_0^{\sqrt{\frac{u_t}{2}}} t^{M} e^{-t^2} I_{M-1}(2 \sqrt{B} t) dt ,
    \end{multline*}
    \begin{multline*}
        F_M(u_t)
        = 1 - T_{\sqrt{\frac{u_t}{2}}} \left ( 3M, M - 1, 0 \right ) = \\
        %
        = 1 - 2 \int \limits_0^{\sqrt{\frac{u_t}{2}}} t^{3M-(M-1)} e^{-t^2} I_{M-1}(0) dt
        = 1 - 2 \int \limits_0^{\sqrt{\frac{u_t}{2}}} t^{2M-1} e^{-t^2} \frac{1}{(M-1)!} dt .
    \end{multline*}

    При больших значениях $D$ её величину можно вычислять приближенно. Рассмотрим величину $\eta = \sqrt{U_\Sigma} - h$, где величину $h$ определим позже.
    Плотность вероятности $\eta$:
    \begin{multline*}
        p_{sn,\eta}(y)
        = p_{sn,U_\Sigma}((y+h)^2) \fd{y}(y+h)^2 = \\
        %
        = \frac{1}{2} e^{- \left ( B + \frac{(y+h)^2}{2} \right )} \left ( \frac{(y+h)^2}{2 B} \right )^\frac{M-1}{2} I_{M-1} \left ( \sqrt{2 B (y+h)^2} \right ) 2 (y+h) = \\
        %
        = e^{- \left ( B + \frac{(y+h)^2}{2} \right )} \left ( \frac{y+h}{\sqrt{2 B}} \right )^{M-1} (y+h) \frac{\sqrt{2 B}}{\sqrt{2 B}} I_{M-1} \left ( \sqrt{2 B} (y+h) \right ) = \\
        %
        = e^{- \left ( B + \frac{(y+h)^2}{2} \right )} \left ( \frac{y+h}{\sqrt{2 B}} \right )^M \sqrt{2 B} I_{M-1} \left ( \sqrt{2 B} (y+h) \right ) = \\
        %
        = e^{- \left ( B + \frac{(y+h)^2}{2} \right )} \sqrt{2 B} \left ( 1 + \frac{y + h - \sqrt{2 B}}{\sqrt{2 B}} \right )^M I_{M-1} \left ( \sqrt{2 B} (y+h) \right ) .
    \end{multline*}
    С учётом приближений
    \begin{gather*}
        I_{M-1}(x) \approx \frac{1}{\sqrt{2 \pi x}} e^x \; ( x \gg 1 ) , \\
        ( 1 + x )^M \approx e^{x M} \; ( x \ll 1 ) .
    \end{gather*}
    получим
    \begin{multline*}
        p_{sn,\eta}(y)
        = e^{- \left ( B + \frac{(y+h)^2}{2} \right )} \sqrt{2 B} e^{\frac{y + h - \sqrt{2 B}}{\sqrt{2 B}} M} \frac{1}{\sqrt{2 \pi \sqrt{2 B} (y + h)}} e^{\sqrt{2 B} (y + h)} = \\
        %
        = \sqrt{\frac{2 B}{2 \pi \sqrt{2 B} (y + h)}} e^{- B - \frac{(y+h)^2}{2} + \frac{y + h - \sqrt{2 B}}{\sqrt{2 B}} M + \sqrt{2 B} (y + h)} = \\
        %
        = \sqrt{\frac{\sqrt{2 B}}{2 \pi (y + h)}} e^{- B - \frac{y^2}{2} - y h - \frac{h^2}{2} + \frac{y + h - \sqrt{2 B}}{\sqrt{2 B}} M + \sqrt{2 B} (y + h)} .
    \end{multline*}
    Выберем величину $h$ так чтобы коэффициент при $y$ в показателе был равен нулю:
    \[
        -h + \frac{M}{\sqrt{2 B}} + \sqrt{2 B} = 0 ,
    \]
    откуда
    \[
        h
        = \frac{M}{\sqrt{2 B}} + \sqrt{2 B}
        = \frac{M}{\sqrt{2 M \frac{q^2}{2}}} + \sqrt{2 M \frac{q^2}{2}}
        = \frac{\sqrt{M}}{q} + \sqrt{M} q
        = \sqrt{M} \left ( \frac{1}{q} + q \right ) .
    \]
    При таком выборе $h$ в показателе нет $y$:
    \[
        p_{sn,\eta}(y)
        = \sqrt{\frac{\sqrt{2 B}}{2 \pi (y + h)}} e^{- B - \frac{y^2}{2} - \frac{h^2}{2} + \frac{h - \sqrt{2 B}}{\sqrt{2 B}} M + \sqrt{2 B} h} .
    \]
    Преобразовывая показатель с учётом значения $h$, получим:
    \begin{multline*}
        - B - \frac{h^2}{2} + \frac{h - \sqrt{2 B}}{\sqrt{2 B}} M + \sqrt{2 B} h = \\
        %
        = - B - \frac{\left ( \frac{M}{\sqrt{2 B}} + \sqrt{2 B} \right )^2}{2} + \frac{\frac{M}{\sqrt{2 B}} + \sqrt{2 B} - \sqrt{2 B}}{\sqrt{2 B}} M + \sqrt{2 B} \left ( \frac{M}{\sqrt{2 B}} + \sqrt{2 B} \right ) = \\
        %
        = - B - \frac{\frac{M^2}{2 B} + 2 \frac{M}{\sqrt{2 B}} \sqrt{2 B} + 2 B}{2} + \frac{M^2}{2 B} + M + 2 B
        = - B - \frac{M^2}{4 B} - M - B + M + 2 B = \\
        %
        = - \frac{M^2}{4 B}
        = - \frac{M^2}{4 M \frac{q^2}{2}}
        = - \frac{M}{2 q^2} .
    \end{multline*}
    Таким образом, плотность имеет вид:
    \[
        p_{sn,\eta}(y)
        = \sqrt{\frac{\sqrt{2 B}}{2 \pi (y + h)}} e^{- \frac{y^2}{2} - \frac{M}{2 q^2}} .
    \]
    Приближение плотности не слишком точное.


    \section{Двухэтапное обнаружение}
    Пусть имеется зона поиска, состоящая из $K$ направлений. В каждом направлении независимо с вероятностью $P_t$ может находится объект. Обзор зоны поиска осуществляется
    гармоническими сигналами с энергетической характеристикой $q = \sqrt{\frac{E}{\frac{N_0}{2}}}$. Рассматриваются двухэтапные методы обнаружения: на первом этапе во всех
    $K$ направлениях используются $M_1$ импульсов, в тех направлениях где с порогом $u_1$ произошло обнаружение, производится второй этап состоящий из ещё $M_2$ импульсов
    и если на втором этапе опять происходит обнаружение с порогом $u_2$, то оно считается обнаружением двухэтапного метода. Заданы условные вероятности $D^*$ ---
    правильного обнаружения и $F^*$ --- ложного обнаружения.

    Необходимо подобрать количества $M_1$ и $M_2$ импульсов, а также пороги $u_1$ и $u_2$ так, чтобы:
    \begin{enumerate}
        \item условные вероятности обнаружений были не хуже заданных ($D^*$ и $F^*$),
        \item количества сигналов не превосходят разумные пределы: $M_1 \le \overline{M}_1$ и $M_2 \le \overline{M}_2$,
        \item среднее количество импульсов на все $K$ направлений было наименьшим.
    \end{enumerate}

    Условные вероятности правильного $D_M(u)$ и ложного $F_M(u)$ обнаружений определяются равенствами \eqref{quadric_pack:true_detection_probability} и
    \eqref{quadric_pack:false_detection_probability}. Условные вероятности правильного и ложного обнаружений в двухэтапном методе при фиксированных количествах
    $M_1$ и $M_2$ в силу независимости событий обнаружений описываются функциями:
    \begin{equation}
        \label{equation:two_stage_detection:probabilities}
        \begin{array}{c}
            D(u_1, u_2) = D_{M_1}(u_1) \cdot D_{M_2}(u_2) , \\
            F(u_1, u_2) = F_{M_1}(u_1) \cdot F_{M_2}(u_2) .
        \end{array}
    \end{equation}

    Пусть $\nu$ --- случайная величина количества импульсов, которая требуется двухэтапному методу для просмотра всей зоны поиска. Поскольку обнаружение в каждом направлении
    является независимым событием, то $\nu$ имеет распределение Бернулли:
    \[
        \expectation{\nu}
        = K \cdot \left ( M_1 + \left ( P_t D_{M_1}(u_1) + \left ( 1 - P_t \right ) F_{M_1}(u_1) \right ) M_2 \right ) .
    \]
    Обычно, вероятность нахождения объекта в направлении $P_t$ оказывается малой, поэтому $P_t D_{M_1}(u_1)$ мало, и большая часть выполнения второго этапа обусловлена
    ложными обнаружениями на первом этапе --- слагаемое с $F_{M_1}(u_1)$.

    При поиске наименьшего значения $\expectation{\nu}$ будем перебирать возможные значения для $M_1$ и $M_2$, начиная с $M_1 = 1$ и $M_2 = 1$. При фиксированных $M_1$ и
    $M_2$ будем минимизировать вероятность обнаружения на первом этапе:
    \[
        P_t D_{M_1}(u_1) + \left ( 1 - P_t \right ) F_{M_1}(u_1) \rightarrow min ,
    \]
    Поскольку вероятности $D_{M_1}(u_1)$ и $F_{M_1}(u_1)$ одновременно монотонно убывают с ростом порога $u_1$, то необходимо найти наибольшее значение $u_1$:
    \[
        u_1 \rightarrow max,
    \]
    при котором обеспечивается качество процедуры обнаружения:
    \begin{equation}
        \label{equation:two_stage_detection:probabilities_conditions}
        \left \{
        \begin{array}{l}
            D_{M_1}(u_1) \cdot D_{M_2}(u_2) \ge D^* \\
            F_{M_1}(u_1) \cdot F_{M_2}(u_2) \le F^*
        \end{array}
        \right .
        .
    \end{equation}
    Вполне возможно, что при выбранном количестве $M_1$ второй этап вовсе не потребуется (формально можно считать, что на втором этапе не используются сигналы $M_2 = 0$
    и всегда принимается решение об обнаружении), если существует величина порога $u_1$, при которой:
    \[
        \left \{
        \begin{array}{l}
            D_{M_1}(u_1) \ge D^* \\
            F_{M_1}(u_1) \le F^*
        \end{array}
        \right .
        .
    \]
    Пусть величина $u_{1,D}$ является решением уравнения:
    \[
        D_{M_1}(u_1) \ge D^* .
    \]
    Если при этом же значении $F_{M_1}(u_{1,D}) \le F^*$, тогда $u_{1,D}$ является оптимальным порогом первого этапа обнаружения, а второй не требуется:
    \begin{gather*}
        u_1^* = u_{1,D} , \\
        u_2^* = 0 .
    \end{gather*}
    В противном случае при $F_{M_1}(u_{1,D}) > F^*$ требуется второй этап обнаружения, подбор количества $M_2$ и величины порога $u_2$, при которых выполняются условия
    \eqref{equation:two_stage_detection:probabilities_conditions}.

    На плоскости порогов $(u_1, u_2)$ кривые:
    \begin{gather*}
        D (u_1, u_2) = D^* , \\
        F (u_1, u_2) = F^*
    \end{gather*}
    могут иметь различный вид в зависимости от $q$ и количеств $M_1$, $M_2$ (рисунки \ref{figure:two_state_detection:probability_curves}). В случаях, представленных
    на рисунках \ref{figure:two_state_detection:probability_curves}.б) и \ref{figure:two_state_detection:probability_curves}.в) имеется решение --- точка
    $\left (u_1^*, u_2^* \right )$, одновременно удовлетворяющая двум условиям:
    \begin{gather*}
        D (u_1^*, u_2^*) = D^* , \\
        F (u_1^*, u_2^*) = F^* .
    \end{gather*}

    \begin{figure}[h]
        \centering
        \begin{subfigure}{0.3\textwidth}
            \begin{tikzpicture}[scale=6]
                % оси
                \draw [->] ( -0.1, 0 ) -- ( 1, 0 ) node [below] at ( 1, 0 ) {$u_1$};
                \draw [->] ( 0, -0.1 ) -- ( 0, 1 ) node [left] at ( 0, 1 ) {$u_2$};
                \node [below left ] ( 0, 0 ) {$0$};

                % кривая D
                \draw ( 0.4, 0 ) to [out=90,in=0] ( 0, 0.45 ) node [below] at ( 0.4, 0 ) {$u_{1,D}$};
                \draw ( 0.35, 0.05 ) -- ( 0.2, 0.1 ) node [left] at ( 0.2, 0.1 ) {$1$};
                % кривая F
                \draw [dashed] ( 0.75, 0 ) to [out=115,in=-10] ( 0, 0.7 );
                \draw ( 0.79, 0.05 ) -- ( 0.94, 0.1 ) node [right] at ( 0.94, 0.1 ) {$2$};
            \end{tikzpicture}
            \caption{Нет пересечения кривых и решения}
        \end{subfigure}

        \hfill

        \begin{subfigure}{0.3\textwidth}
            \begin{tikzpicture}[scale=6]
                % оси
                \draw [->] ( -0.1, 0 ) -- ( 1, 0 ) node [below] at ( 1, 0 ) {$u_1$};
                \draw [->] ( 0, -0.1 ) -- ( 0, 1 ) node [left] at ( 0, 1 ) {$u_2$};
                \node [below left ] ( 0, 0 ) {$0$};

                % координаты точки пересечения
                \def \fu { 0.3 };
                \def \su { 0.4 };

                % кривая D
                \draw ( 0.6, 0 ) to [out=100,in=-30] ( \fu, \su ) to [out=150,in=-20] ( 0, 0.55 ) node [below] at ( 0.6, 0 ) {$u_{1,D}$};
                \draw ( 0.55, 0.05 ) -- ( 0.4, 0.1 ) node [left] at ( 0.4, 0.1 ) {$1$};
                % кривая F
                \draw [dashed] ( 0.8, 0 ) to [out=100,in=-20] ( \fu, \su ) to [out=160,in=0] ( 0, 0.43 );
                \draw ( 0.82, 0.05 ) -- ( 0.97, 0.1 ) node [right] at ( 0.97, 0.1 ) {$2$};

                % область
                \path [fill=gray] ( 0, 0.55 ) to [out=-20,in=150] ( \fu, \su ) to [out=160,in=0] ( 0, 0.43 );

                % искомая точка
                \draw [fill] ( \fu, \su ) circle [radius=0.015];
                \draw ( \fu, \su ) -- ( 0.5, 0.5 ) node [right] at ( 0.5, 0.5 ) {$(u_1^*, u_2^*)$};
            \end{tikzpicture}
            \caption{Есть пересечение кривых вида I и решение}
        \end{subfigure}

        \hfill

        \begin{subfigure}{0.3\textwidth}
            \begin{tikzpicture}[scale=6]
                % оси
                \draw [->] ( -0.1, 0 ) -- ( 1, 0 ) node [below] at ( 1, 0 ) {$u_1$};
                \draw [->] ( 0, -0.1 ) -- ( 0, 1 ) node [left] at ( 0, 1 ) {$u_2$};
                \node [below left ] ( 0, 0 ) {$0$};

                % координаты первой точки пересечения
                \def \fu { 0.55 };
                \def \su { 0.4 };

                % промежуточная точка
                \def \fv { 0.5 };
                \def \sv { 0.55 };

                % координаты второй точки пересечения
                \def \ft { 0.3 };
                \def \st { 0.6 };

                % кривая D
                \draw ( 0.6, 0 ) to [out=95,in=-80] ( \fu, \su ) to [out=100,in=-30] ( \fv, \sv ) to [out=150,in=-10] ( \ft, \st ) to [out=170,in=0] ( 0, 0.62 ) node [below] at ( 0.6, 0 ) {$u_{1,D}$};
                \draw ( 0.55, 0.05 ) -- ( 0.4, 0.1 ) node [left] at ( 0.4, 0.1 ) {$1$};
                % кривая F
                \draw [dashed] ( 0.8, 0 ) to [out=110,in=-50] ( \fu, \su ) to [out=130,in=-30] ( \ft, \st ) to [out=150,in=-10] ( 0, 0.75 );
                \draw ( 0.82, 0.05 ) -- ( 0.97, 0.1 ) node [right] at ( 0.97, 0.1 ) {$2$};

                % область
                \path [fill=gray] ( \ft, \st ) to [out=-10,in=150] ( \fv, \sv ) to [out=-30,in=100] ( \fu, \su ) to [out=130,in=-30] ( \ft, \st );

                % искомая точка
                \draw [fill] ( \fu, \su ) circle [radius=0.015];
                \draw ( \fu, \su ) -- ( 0.7, 0.5 ) node [right] at ( 0.7, 0.55 ) {$(u_1^*, u_2^*)$};
            \end{tikzpicture}
            \caption{Есть пересечение кривых вида II и решение}
        \end{subfigure}

        \caption{Графики кривых условных вероятностей: 1 --- $D (u_1, u_2) = D^*$, 2 --- $F (u_1, u_2) = F^*$}
        \label{figure:two_state_detection:probability_curves}
    \end{figure}

    Поиск решения $\left (u_1^*, u_2^* \right )$ выполняется итерационным методом путём построения последовательности приближений:
    \begin{align*}
        & \left ( u_1^{(0)}   , u_2^{(0)} \right ) = \left ( u_{1,D}     , 0 \right ), \\
        & \dots \\
        & \left ( u_1^{(k)}   , u_2^{(k+1)} \right ), \\
        & \left ( u_1^{(k+1)} , u_2^{(k+1)} \right ), \\
        & \dots
    \end{align*}
    в которой попеременно уточняются компоненты --- величина $u_2^{(k+1)}$  вычисляется из уравнения
    \[
        F (u_1^{(k)}, u_2^{(k+1)}) = F^* ,
    \]
    а величина $u_1^{(k+1)}$ из уравнения
    \[
        D (u_1^{(k+1)}, u_2^{(k+1)}) = D^* .
    \]
    Графическое представление итерационного процесса поиска решения $\left ( u_1^*, u_2^* \right )$ представлено на рисунках \ref{figure:two_state_detection:iterations}.
    Останов итерационного процесса производится тогда, когда очередное изменение одной из компонент признаётся несущественным --- меньшим некоторой малой величины $\varepsilon$:
    \[
        \modulus{u_1^{(k+1)} - u_1^{(k)}} < \varepsilon \; \text{или} \; \modulus{u_2^{(k+1)} - u_2^{(k)}} < \varepsilon .
    \]


    Заметим, что в случае а), в котором нет решения, итерационный процесс выходит за уровень $u_{2,D}$, который представляет наибольший возможный порог второго обнаружителя,
    определяемый равенством:
    \[
        D_{M_1} ( u_{2,D} ) = D^* .
    \]
    При выходе за уровень $u_{2,D}$ итерационный процесс останавливается, решения нет.

    \begin{figure}[h]
        \centering
        \begin{subfigure}{0.3\textwidth}
            \begin{tikzpicture}[scale=6]
                % оси
                \draw [->] ( -0.1, 0 ) -- ( 1, 0 ) node [below] at ( 1, 0 ) {$u_1$};
                \draw [->] ( 0, -0.1 ) -- ( 0, 1 ) node [left] at ( 0, 1 ) {$u_2$};
                \node [below left ] ( 0, 0 ) {$0$};

                % кривая D
                \draw ( 0.55, 0 ) to [out=90,in=0] ( 0, 0.45 ) node [below] at ( 0.55, 0 ) {$\left ( u_{1,D}, 0 \right )$};
                % кривая F
                \draw [dashed] ( 0.75, 0 ) to [out=115,in=-10] ( 0, 0.7 );

                % уровень u_2
                \draw [dotted] ( 0, 0.45 ) -- ( 1, 0.45 ) node [left] at ( 0, 0.45 ) {$u_{2,D}$};

                % итерации
                \draw [->] ( 0.55, 0 ) -- ( 0.55, 0.35 ) node [right] ( 0.55, 0.35 ) {$\left ( u_1^{(0)}, u_2^{(1)} \right )$};
                \draw [->] ( 0.55, 0.35 ) -- ( 0.35, 0.35 ) node [below left] ( 0.47, 0.27 ) {$\left ( u_1^{(1)}, u_2^{(1)} \right )$};
                \draw [->] ( 0.35, 0.35 ) -- ( 0.35, 0.55 ) node [above right] ( 0.35, 0.55 ) {$\left ( u_1^{(1)}, u_2^{(2)} \right )$};;
            \end{tikzpicture}
            \caption{Нет пересечения кривых и решения}
        \end{subfigure}

        \hfill

        \begin{subfigure}{0.3\textwidth}
            \begin{tikzpicture}[scale=6]
                % оси
                \draw [->] ( -0.1, 0 ) -- ( 1, 0 ) node [below] at ( 1, 0 ) {$u_1$};
                \draw [->] ( 0, -0.1 ) -- ( 0, 1 ) node [left] at ( 0, 1 ) {$u_2$};
                \node [below left ] ( 0, 0 ) {$0$};

                % координаты точки пересечения
                \def \fu { 0.3 };
                \def \su { 0.4 };

                % кривая D
                \draw ( 0.6, 0 ) to [out=100,in=-30] ( \fu, \su ) to [out=150,in=-20] ( 0, 0.55 ) node [below] at ( 0.6, 0 ) {$\left ( u_{1,D}, 0 \right )$};
                % кривая F
                \draw [dashed] ( 0.8, 0 ) to [out=100,in=-20] ( \fu, \su ) to [out=160,in=0] ( 0, 0.43 );

                % область
                \path [fill=gray] ( 0, 0.55 ) to [out=-20,in=150] ( \fu, \su ) to [out=160,in=0] ( 0, 0.43 );

                % искомая точка
                \draw [fill] ( \fu, \su ) circle [radius=0.015];
                \draw ( \fu, \su ) -- ( 0.5, 0.5 ) node [right] at ( 0.5, 0.5 ) {$(u_1^*, u_2^*)$};

                % итерации
                \draw [->] ( 0.6, 0 ) -- ( 0.6, 0.27 ) node [above right] ( 0.6, 0.27 ) {$\left ( u_1^{(0)}, u_2^{(1)} \right )$};
                \draw [->] ( 0.6, 0.27 ) -- ( 0.47, 0.27 ) node [below left] ( 0.47, 0.27 ) {$\left ( u_1^{(1)}, u_2^{(1)} \right )$};
                \draw [->] ( 0.47, 0.27 ) -- ( 0.47, 0.34 );
                \draw [->] ( 0.47, 0.34 ) -- ( 0.39, 0.34 );
            \end{tikzpicture}
            \caption{Есть пересечение кривых вида I и решение}
        \end{subfigure}

        \hfill

        \begin{subfigure}{0.3\textwidth}
            \begin{tikzpicture}[scale=6]
                % оси
                \draw [->] ( -0.1, 0 ) -- ( 1, 0 ) node [below] at ( 1, 0 ) {$u_1$};
                \draw [->] ( 0, -0.1 ) -- ( 0, 1 ) node [left] at ( 0, 1 ) {$u_2$};
                \node [below left ] ( 0, 0 ) {$0$};

                % координаты первой точки пересечения
                \def \fu { 0.55 };
                \def \su { 0.4 };

                % промежуточная точка
                \def \fv { 0.5 };
                \def \sv { 0.55 };

                % координаты второй точки пересечения
                \def \ft { 0.3 };
                \def \st { 0.6 };

                % кривая D
                \draw ( 0.6, 0 ) to [out=95,in=-80] ( \fu, \su ) to [out=100,in=-30] ( \fv, \sv ) to [out=150,in=-10] ( \ft, \st ) to [out=170,in=0] ( 0, 0.62 ) node [below] at ( 0.6, 0 ) {$\left ( u_{1,D}, 0 \right )$};
                % кривая F
                \draw [dashed] ( 0.8, 0 ) to [out=110,in=-50] ( \fu, \su ) to [out=130,in=-30] ( \ft, \st ) to [out=150,in=-10] ( 0, 0.75 );

                % область
                \path [fill=gray] ( \ft, \st ) to [out=-10,in=150] ( \fv, \sv ) to [out=-30,in=100] ( \fu, \su ) to [out=130,in=-30] ( \ft, \st );

                % искомая точка
                \draw [fill] ( \fu, \su ) circle [radius=0.015];
                \draw ( \fu, \su ) -- ( 0.7, 0.5 ) node [right] at ( 0.7, 0.55 ) {$(u_1^*, u_2^*)$};

                % итерации
                \draw [->] ( 0.6, 0 ) -- ( 0.6, 0.34 ) node [right] ( 0.6, 0.27 ) {$\left ( u_1^{(0)}, u_2^{(1)} \right )$};
                \draw [->] ( 0.6, 0.34 ) -- ( 0.56, 0.34 ) node [left] ( 0.47, 0.27 ) {$\left ( u_1^{(1)}, u_2^{(1)} \right )$};
            \end{tikzpicture}
            \caption{Есть пересечение кривых вида II и решение}
        \end{subfigure}

        \caption{Итерационный поиск решения: 1 --- $D (u_1, u_2) = D^*$, 2 --- $F (u_1, u_2) = F^*$}
        \label{figure:two_state_detection:iterations}
    \end{figure}


    \section{Последовательное обнаружение}

    Пусть имеется возможность последовательно излучать и принимать сигналы $y_i(t)$. По каждому принятому сигналу вычисляется модульное значение корреляционного интеграла
    $Z_i$. Величина $Z_i$ имеет распределение Релея $p_{n,Z}(z)$ \eqref{noise:correlation_integral_pdf}, если принятый сигнал содержит только шум $y_i(t) = n(t)$, и
    обобщенное распределение Релея $p_{sn,Z}(z)$ \eqref{noise_and_signal:correlation_integral_pdf}, если принятый сигнал содержит смесь шума и отраженного сигнала
    $y_i(t) = n(t) + x(t|\beta_1, \beta_2)$:
    \begin{gather*}
        p_{n,Z}(z)
        = \frac{z}{\nu_0^2} e^{- \frac{z^2}{2 \nu_0^2}}
        = \frac{1}{\nu_0} \frac{z}{\nu_0} e^{- \frac{\left ( \frac{z}{\nu_0} \right )^2}{2}} , \\
        %
        p_{sn,Z}(z)
        = \frac{z}{\nu_0^2} e^{- \frac{z^2 + E^2}{2 \nu_0^2}} I_0 \left ( \frac{z E}{\nu_0^2} \right )
        = \frac{1}{\nu_0} \frac{z}{\nu_0} e^{- \frac{\left ( \frac{z}{\nu_0} \right)^2 + \left ( \frac{E}{\nu_0} \right)^2}{2}} I_0 \left ( \frac{z}{\nu_0} \frac{E}{\nu_0} \right ).
    \end{gather*}
    Нормированные величины $U_i = \frac{Z_i}{\nu_0}$ имеют соответственно плотности вероятности:
    \begin{gather*}
        p_{n,U}(u) = u e^{- \frac{u^2}{2}} , \\
        p_{sn,U}(u) = u e^{- \frac{u^2 + q^2}{2}} I_0 \left ( u q \right ) .
    \end{gather*}

    В последовательной процедуре различения гипотез на основе независимых величин $U_i$ отношение правдоподобия $L(U_1, \dots, U_k)$ имеет вид:
    \[
        L (U_1, \dots, U_k)
        = \frac{p_{sn,U}(U_1) \cdot ... \cdot p_{sn,U}(U_k)}{p_{n,U}(U_1) \cdot ... \cdot p_{n,U}(U_k)}
        = \prod_{i=1}^k \frac{p_{sn,U}(U_i)}{p_{n,U}(U_i)} .
    \]

    Двойное неравенство последовательного критерия с порогами $A$ и $B$:
    \begin{gather*}
        A < L (U_1, \dots, U_k) < B , \\
        A < \prod_{i=1}^k \frac{p_{sn,U}(U_i)}{p_{n,U}(U_i)} < B , \\
        \ln A < \sum_{i=1}^k \ln \left ( \frac{p_{sn,U}(U_i)}{p_{n,U}(U_i)} \right ) < \ln B ,
    \end{gather*}
    где величина одного слагаемого
    \begin{multline*}
        \zeta
        = \ln \left ( \frac{p_{sn,U}(\xi)}{p_{n,U}(\xi)} \right )
        = \ln \left ( \frac{\xi e^{- \frac{\xi^2 + q^2}{2}} I_0 \left ( \xi q \right )}{\xi e^{- \frac{\xi^2}{2}}} \right )
        = \ln \left ( e^{- \frac{\xi^2 + q^2}{2} + \frac{\xi^2}{2}} I_0 \left ( \xi q \right ) \right ) = \\
        %
        = \ln \left ( e^{- \frac{q^2}{2}} I_0 \left ( \xi q \right ) \right )
        = - \frac{q^2}{2} + \ln \left ( I_0 \left ( \xi q \right ) \right ) .
    \end{multline*}

    Среднее значение одного слагаемого:
    \[
        \expectation{\zeta \left | H_j \right .} = - \frac{q^2}{2} + \expectation{\ln I_0 \left ( \xi q \right )} .
    \]
    При $H_0$ величина $\xi$ распределена с плотностью $p_{n,U}(u)$, при $H_1$ --- с плотностью $p_{sn,U}(u)$.

    Пусть $\alpha$ и $\beta$ обозначают вероятности ошибок первого и второго родов. Они связаны с условными вероятностями правильного $D^*$ и ложного $F^*$ обнаружений
    равенствами:
    \begin{gather*}
        \alpha = F^* , \\
        \beta = 1 - D^* .
    \end{gather*}
    Пороги $A$ и $B$ определяются через вероятности ошибок:
    \[
        A \approx \frac{\beta}{1 - \alpha} , \; B \approx \frac{1 - \beta}{\alpha} .
    \]

    Пусть $\nu$ обозначает количество величин $U_i$ до принятия одной из гипотез, тогда в соответствии с тождеством Вальда о математическом ожидании суммы:
    \begin{gather*}
        \expectation{\nu \left | H_0 \right .}
        = \frac{\ln A \cdot \left ( 1 - \alpha \right ) + \ln B \cdot \alpha}{\expectation{\zeta \left | H_0 \right .}} , \\
        %
        \expectation{\nu \left | H_1 \right .}
        = \frac{\ln A \cdot \beta + \ln B \cdot \left ( 1 - \beta \right )}{\expectation{\zeta \left | H_1 \right .}} .
    \end{gather*}


    \section{Приложение 1. Обобщенное распределение Релея}

    Пусть $\xi_1 \sim \mathcal{N} \left ( \mu_1, \sigma^2 \right )$ и $\xi_2 \sim \mathcal{N} \left ( \mu_2, \sigma^2 \right )$ независимые случайные величины.
    Рассмотрим распределение величины $\eta = \sqrt{\xi_1^2 + \xi_2^2}$:
    \[
        F_{\eta}(y)
        = \probability{\eta < y}
        = \probability{\sqrt{\xi_1^2 + \xi_2^2} < y}
        = \left \{
        \begin{array}{ll}
            0,                                     & y < 0   \\
            \probability{\xi_1^2 + \xi_2^2 < y^2}, & 0 \ge y
        \end{array}
        \right .
    \]
    Вычислим вероятность:
    \begin{multline*}
        \probability{\xi_1^2 + \xi_2^2 < y^2}
        = \iint \limits_{x_1^2 + x_2^2 < y^2} p_{\xi_1, \xi_2} (x_1, x_2) dx_1 dx_2 = \\
        %
        = \iint \limits_{x_1^2 + x_2^2 < y^2} \frac{1}{\sqrt{2 \pi} \sigma} e^{- \frac{(x_1 - \mu_1)^2}{2 \sigma^2}} \frac{1}{\sqrt{2 \pi} \sigma} e^{- \frac{(x_2 - \mu_2)^2}{2 \sigma^2}} dx_1 dx_2 = \\
        %
        = \frac{1}{2 \pi \sigma^2} \iint \limits_{x_1^2 + x_2^2 < y^2} e^{- \frac{(x_1 - \mu_1)^2}{2 \sigma^2} - \frac{(x_2 - \mu_2)^2}{2 \sigma^2}} dx_1 dx_2
        = \frac{1}{2 \pi \sigma^2} \iint \limits_{x_1^2 + x_2^2 < y^2} e^{- \frac{x_1^2 - 2 x_1 \mu_1 + \mu_1^2 + x_2^2 - 2 x_2 \mu_2 + \mu_2^2}{2 \sigma^2}} dx_1 dx_2 = \\
        %
        = \frac{1}{2 \pi \sigma^2} \iint \limits_{x_1^2 + x_2^2 < y^2} e^{- \frac{x_1^2 + x_2^2 - 2 x_1 \mu_1 - 2 x_2 \mu_2 + \mu_1^2 + \mu_2^2}{2 \sigma^2}} dx_1 dx_2 = \\
        %
        = \frac{1}{2 \pi \sigma^2} e^{- \frac{\mu_1^2 + \mu_2^2}{2 \sigma^2}} \iint \limits_{x_1^2 + x_2^2 < y^2} e^{- \frac{x_1^2 + x_2^2}{2 \sigma^2}} e^{\frac{x_1 \mu_1 + x_2 \mu_2}{\sigma^2}} dx_1 dx_2 = \\
        %
        = \frac{1}{2 \pi \sigma^2} e^{- \frac{\mu_1^2 + \mu_2^2}{2 \sigma^2}} \int \limits_0^y \int \limits_0^{2 \pi} \rho e^{- \frac{\rho^2}{2 \sigma^2}} e^{\frac{\rho \cos \varphi \mu_1 + \rho \sin \varphi \mu_2}{\sigma^2}} d \varphi d \rho = \\
        %
        = \frac{1}{2 \pi \sigma^2} e^{- \frac{\mu_1^2 + \mu_2^2}{2 \sigma^2}} \int \limits_0^y \rho e^{- \frac{\rho^2}{2 \sigma^2}} \int \limits_0^{2 \pi} e^{\frac{\rho \sqrt{\mu_1^2 + \mu_2^2} \left ( \cos \varphi \frac{\mu_1}{\sqrt{\mu_1^2 + \mu_2^2}} + \sin \varphi \frac{\mu_2}{\mu_1^2 + \mu_2^2} \right ) }{\sigma^2}} d \varphi d \rho = \\
        %
        = \frac{1}{2 \pi \sigma^2} e^{- \frac{\mu^2}{2 \sigma^2}} \int \limits_0^y \rho e^{- \frac{\rho^2}{2 \sigma^2}} \int \limits_0^{2 \pi} e^{\frac{\rho \mu \left ( \cos \varphi \frac{\mu_1}{\mu} + \sin \varphi \frac{\mu_2}{\mu} \right ) }{\sigma^2}} d \varphi d \rho = \\
        %
        = \frac{1}{2 \pi \sigma^2} e^{- \frac{\mu^2}{2 \sigma^2}} \int \limits_0^y \rho e^{- \frac{\rho^2}{2 \sigma^2}} \int \limits_0^{2 \pi} e^{\frac{\rho \mu}{\sigma^2} \cos ( \varphi - \alpha )} d \varphi d \rho = \\
        %
        = \frac{1}{\sigma^2} e^{- \frac{\mu^2}{2 \sigma^2}} \int \limits_0^y \rho e^{- \frac{\rho^2}{2 \sigma^2}} \frac{1}{2 \pi} \int \limits_0^{2 \pi} e^{\frac{\rho \mu}{\sigma^2} \cos ( \varphi - \alpha )} d \varphi d \rho = \\
        %
        = \frac{1}{\sigma^2} e^{- \frac{\mu^2}{2 \sigma^2}} \int \limits_0^y \rho e^{- \frac{\rho^2}{2 \sigma^2}} I_0 \left ( \frac{\rho \mu}{\sigma^2} \right ) d \rho ,
    \end{multline*}
    где $\mu = \sqrt{\mu_1^2 + \mu_2^2}$ и $\alpha$ --- угол, такой что $\cos \alpha = \frac{\mu_1}{\mu}$ и $\sin \alpha = \frac{\mu_2}{\mu}$.

    Таким образом, функция распределения величины $\eta$:
    \[
        F_{\eta}(y)
        = \left \{
        \begin{array}{ll}
            0,                                                                                                                                                             & y < 0   \\
            \frac{1}{\sigma^2} e^{- \frac{\mu^2}{2 \sigma^2}} \int \limits_0^y \rho e^{- \frac{\rho^2}{2 \sigma^2}} I_0 \left ( \frac{\rho \mu}{\sigma^2} \right ) d \rho, & 0 \ge y
        \end{array}
        \right .
        ,
    \]
    а плотность вероятности:
    \[
        p_{\eta}(y)
        = \left \{
        \begin{array}{ll}
            0,                                                                                                                             & y < 0   \\
            \frac{1}{\sigma^2} e^{- \frac{\mu^2}{2 \sigma^2}} y e^{- \frac{y^2}{2 \sigma^2}} I_0 \left ( \frac{y \mu}{\sigma^2} \right ) , & 0 \ge y
        \end{array}
        \right .
        %
        = \left \{
        \begin{array}{ll}
            0,                                                                                                    & y < 0   \\
            \frac{y}{\sigma^2} e^{- \frac{y^2 + \mu^2}{2 \sigma^2}} I_0 \left ( \frac{y \mu}{\sigma^2} \right ) , & 0 \ge y
        \end{array}
        \right .
        .
    \]


    \section{Приложение 2. Модифицированные функции Бесселя}

    Модифицированное уравнение Бесселя:
    \[
        z^2 \md{2}{z} \omega + z \fd{z} \omega - ( z^2 + \nu^2 ) \omega = 0 .
    \]
    Решением уравнения является модифицированная функция Бесселя первого рода:
    \[
        I_\nu(z) = \sum_{k=0}^{\infty} \frac{\left ( \frac{z}{2} \right )^{2k+\nu}}{k! \Gamma(k+\nu+1)} .
    \]
    Одно из интегральных представлений ($n \in \mathbb{Z}, Re(z) > 0$):
    \[
        I_\nu(z) = \frac{1}{\pi} \int \limits_0^{\pi} e^{z \cos t} \cos (nt) dt .
    \]

    \begin{thebibliography}{4}
	\bibitem{Gantmacher} Гантмахер Ф. Р., Теория матриц / Изд. "Наука"{}, Москва, 1966.

	\bibitem{Massey} Massey J. L. Shift-Register Synthesis and BCH Decoding / IEEE Transactions on Information Theory,
		vol. IT-15, No. 1, January 1969.

	\bibitem{Montgomery} Montgomery P. A block Lanczos algorithm for finding dependencies over GF(2)

	\bibitem{Thome} Thome E. Fast Computation of Linear Generators for Matrix Sequences and Application to the Block Weidemann Algorithm /
		International Conference on Symbolic and Algebraic Computation, 2001.

	\bibitem{Wiedemann} Wiedemann D. H., Solving Sparse Linear Equations Over Finite Fields / Transactions of Information Theory,
		vol. IT-32, No. 1, January, 1986.

	\bibitem{Zamarashkin} Замарашкин Н. Л. Алгоритмы для разреженных систем линейных уравнений в $GF(2)$ / Москва: Издательство Московского
		Университета, 2013.
\end{thebibliography}


\end{document}