\documentclass[12pt]{article}

\usepackage[T1]{fontenc}
\usepackage[utf8]{inputenc}
\usepackage[english,russian]{babel}
\usepackage[margin=2cm]{geometry}
\usepackage{amsmath}
\usepackage{amsfonts}

% команды вывода первой частной производной
\newcommand{\fpd}[1]{\frac{\partial}{\partial #1}}
\newcommand{\fpda}[2]{\frac{\partial #1}{\partial #2}}
\newcommand{\fpdp}[2]{\fpd{#2} \left ( #1 \right )}

\newcommand{\expectation}[1]{\mathtt{M} \left [ #1 \right ]}
\newcommand{\conditionalexpectation}[2]{\expectation{ #1 \left | #2 \right .}}
\newcommand{\variance}[1]{\mathtt{D} \left [ #1 \right ]}
\newcommand{\covariance}[2]{\mathtt{cov} \left ( #1, #2 \right )}

\newcommand{\modulus}[1]{\left | #1 \right |}
\newcommand{\norm}[1]{\left \| {#1} \right \|}

\newcommand{\event}[1]{\left \{ #1 \right \} }
\newcommand{\probability}[1]{P \event{#1}}


\begin{document}

    \title{Домашнее задание №9}
    \author{Тигетов Давид Георгиевич}
    \date{}
    \maketitle

    \section*{Задача 9.4}
    $X_t = Z \cos \left ( t - \Phi \right )$, $T = \left ( - \infty, + \infty \right )$, где случайные величины $Z$ и $\Phi$ независимые, $\expectation{Z} = a$, $\variance{Z} = b^2$,
    $\Phi$ равномерно распределена на отрезке $\left [ 0, 2 \pi \right ]$. Вычислить $\mu(t)$, $\sigma^2(t)$, $r(s,t)$.

    \subsection*{Решение:}
    Математическое ожидание:
    \[
        \mu(t)
        = \expectation{Z \cos \left ( t - \Phi \right )}
        = \expectation{Z} \expectation{\cos \left ( t - \Phi \right )}
        = 0,
    \]
    поскольку
    \[
        \expectation{\cos \left ( t - \Phi \right )}
        = \int \limits_0^{2 \pi} \cos \left ( t - \varphi \right ) \frac{1}{2 \pi} d \varphi
        = \frac{1}{2 \pi} \left . \left ( - \sin \left ( t - \varphi \right ) \right ) \right |_0^{2 \pi}
        = \frac{1}{2 \pi} \left ( \sin t - \sin \left ( t - 2 \pi \right ) \right )
        = 0 .
    \]
    Дисперсия
    \[
        \sigma^2(t)
        = \variance{Z \cos \left ( t - \Phi \right )}
        = \expectation{Z^2 \cos^2 \left ( t - \Phi \right )}
        = \expectation{Z^2} \expectation{\cos^2 \left ( t - \Phi \right )},
    \]
    где
    \[
        \expectation{Z^2}
        = \variance{Z} + \left ( \expectation{Z} \right )^2
        = b^2 + a^2 ,
    \]
    и
    \begin{multline*}
        %
        \expectation{\cos^2 \left ( t - \Phi \right )}
        = \int \limits_0^{2 \pi} \cos^2 \left ( t - \varphi \right ) \frac{1}{2 \pi} d \varphi
        = \frac{1}{2 \pi} \int \limits_0^{2 \pi} \frac{1 + \cos 2 \left ( t - \varphi \right )}{2} d \varphi = \\
        %
        = \frac{1}{4 \pi} \left ( \int \limits_0^{2 \pi} d \varphi + \int \limits_0^{2 \pi} \cos 2 \left ( t - \varphi \right ) d \varphi \right )
        = \frac{1}{4 \pi} \left ( 2 \pi - \left . \frac{\sin 2 \left ( t - \varphi \right )}{2} \right |_0^{2 \pi} \right ) = \\
        %
        = \frac{1}{4 \pi} \left ( 2 \pi - \frac{\sin 2 t - \sin 2 \left ( t - 2 \pi \right )}{2} \right )
        = \frac{1}{4 \pi} 2 \pi
        = \frac{1}{2}.
    \end{multline*}
    Таким образом,
    \[
        \sigma^2(t) = \frac{b^2 + a^2}{2} .
    \]

    Ковариационная функция
    \begin{multline*}
        R(s,t)
        = \covariance{X_s}{X_t}
        = \expectation{X_s X_t}
        = \expectation{Z \cos \left ( s - \Phi \right ) Z \cos \left ( t - \Phi \right )} = \\
        %
        = \expectation{Z^2 \cos \left ( s - \Phi \right ) \cos \left ( t - \Phi \right )}
        = \expectation{Z^2} \expectation{\cos \left ( s - \Phi \right ) \cos \left ( t - \Phi \right )},
    \end{multline*}
    где
    \begin{multline*}
        \expectation{\cos \left ( s - \Phi \right ) \cos \left ( t - \Phi \right )}
        = \expectation{\frac{1}{2} \left ( \cos \left ( s - t \right ) + \cos \left ( s + t - 2 \Phi \right ) \right )} = \\
        %
        = \frac{1}{2} \left ( \cos \left ( s - t \right ) + \expectation{\cos \left ( s + t - 2 \Phi \right )} \right )
        = \frac{1}{2} \cos \left ( s - t \right ),
    \end{multline*}
    поскольку
    \[
        \expectation{\cos \left ( s + t - 2 \Phi \right )}
        = \int \limits_0^{2 \pi} \cos \left ( s + t - 2 \Phi \right ) \frac{1}{2 \pi} d \varphi
        = 0 .
    \]
    Таким образом,
    \begin{gather*}
        R(s,t) = \left ( b^2 + a^2 \right ) \frac{1}{2} \cos \left ( s - t \right ), \\
        %
        r(s,t)
        = \frac{R(s,t)}{\sigma(s) \cdot \sigma(t)}
        = \frac{\left ( b^2 + a^2 \right ) \frac{1}{2} \cos \left ( s - t \right )}{\frac{b^2 + a^2}{2}}
        = \cos \left ( s - t \right ).
    \end{gather*}

    \subsection*{Ответ:}
    \begin{enumerate}
        \item $\mu(t) = 0$,
        \item $\sigma^2(t) = \frac{b^2 + a^2}{2}$,
        \item $r(s,t) = \cos \left ( s - t \right )$.
    \end{enumerate}

    \section*{Задача 9.8}
    Доказать, используя результат задачи 9.7, что математическое ожидание числа пересечений траекторией процесса $G_t$ из примера 1 горизонтального уровня $a>0$ на $\left [ 0, b \right ]$
    равно $\frac{\lambda b}{\pi} e^{-\frac{a^2}{2}}$.

    \subsection*{Решение:}
    Пусть $\nu$ --- количество пересечений процессом $G_t$ горизонтального уровня $a$, тогда:
    \[
        \nu = I_{x \ge a}(Z) \cdot \nu_{\Phi},
    \]
    где $I_{x \ge a}(Z)$ --- индикаторная случайная величина, показывающая, что при $Z < a$ количество пересечений нулевое, а при $Z \ge a$ определяется величиной $\nu_{\Phi}$ ---
    количеством пересечений процессом $\cos \left ( \lambda t - \Phi \right )$ уровня $\frac{a}{Z}$ на отрезке $\left [ 0, b \right ]$.

    Поскольку величины $Z$ и $\Phi$ независимы, то и функции от них $I_{x \ge a}(Z)$ и $\nu_{\Phi}$ также независимы, поэтому:
    \[
        \expectation{\nu} = \expectation{I_{x \ge a}(Z)} \cdot \expectation{\nu_{\Phi}} ,
    \]
    где
    \begin{gather*}
        I_{x \ge a}(Z)
        = \left \{
        \begin{array}{ll}
            0, & \text{с вероятностью } \probability{Z<a}     \\
            1, & \text{с вероятностью } \probability{Z \ge a}
        \end{array}
        \right . , \\
        %
        \expectation{I_{x \ge a}(Z)}
        = \probability{Z \ge a}
        = \int \limits_a^\infty x e^{- \frac{x^2}{2}} dx
        = \left . - e^{- \frac{x^2}{2}} \right |_a^\infty
        = e^{- \frac{a^2}{2}}.
    \end{gather*}

    Если выполнить масштабирование времени $\tau = \lambda t$, то можно заметить, что количество пересечений процессом
    $\cos \left ( \lambda t - \Phi \right )$ уровня $\frac{a}{Z}$ на отрезке $\left [ 0, b \right ]$ такое как и количество пересечений процессом $\cos \left ( \tau - \Phi \right )$
    уровня $\frac{a}{Z}$ на отрезке $\left [ 0, \lambda b \right ]$, поэтому, используя результат задачи 9.7, получим:
    \[
        \nu_{\Phi} = \frac{\lambda b}{\pi}.
    \]

    В итоге,
    \[
        \expectation{\nu}
        = e^{- \frac{a^2}{2}} \cdot \frac{\lambda b}{\pi}.
    \]
\end{document}