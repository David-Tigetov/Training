\documentclass[12pt]{article}

\usepackage[T1]{fontenc}
\usepackage[utf8]{inputenc}
\usepackage[english,russian]{babel}
\usepackage[margin=2cm]{geometry}
\usepackage{amsmath}
\usepackage{amsfonts}
\usepackage{tikz}

% команды вывода первой частной производной
\newcommand{\fpd}[1]{\frac{\partial}{\partial #1}}
\newcommand{\fpda}[2]{\frac{\partial #1}{\partial #2}}
\newcommand{\fpdp}[2]{\fpd{#2} \left ( #1 \right )}

\newcommand{\expectation}[1]{\mathtt{M} \left [ #1 \right ]}
\newcommand{\conditionalexpectation}[2]{\expectation{ #1 \left | #2 \right .}}
\newcommand{\variance}[1]{\mathtt{D} \left [ #1 \right ]}
\newcommand{\covariance}[2]{\mathtt{cov} \left ( #1, #2 \right )}

\newcommand{\modulus}[1]{\left | #1 \right |}
\newcommand{\norm}[1]{\left \| {#1} \right \|}

\newcommand{\event}[1]{\left \{ #1 \right \} }
\newcommand{\probability}[1]{P \event{#1}}


\begin{document}

    \title{Домашнее задание №12}
    \author{Тигетов Давид Георгиевич}
    \date{}
    \maketitle

    \section*{Задача 12.1}
    а) Используя марковское свойство, вывести соотношение Колмогорова-Чепмена (формула (3)).
    б) Доказать равенство (4).

    \subsection*{Решение:}
    Согласно формуле умножения:
    \begin{multline*}
        \probability{X(0) = i, X(t) = k, X(t+h) = j} = \\
        %
        = \probability{X(0) = i, X(t) = k} \cdot \conditionalprobability{X(t+h) = j}{X(0) = i, X(t) = k} = \\
        %
        = \probability{X(0) = i} \cdot \conditionalprobability{X(t) = k}{X(0) = i} \cdot \conditionalprobability{X(t+h) = j}{X(0) = i, X(t) = k}.
    \end{multline*}
    Используем марковское свойство в последнем множителе в правой части:
    \[
        \probability{X(0) = i, X(t) = k, X(t+h) = j} = \probability{X(0) = i} \cdot \conditionalprobability{X(t) = k}{X(0) = i} \cdot \conditionalprobability{X(t+h) = j}{X(t) = k} , \\
    \]
    Суммируем по всем состояниям в момент $t$, используем свойство совместного распределения в левой части:
    \begin{multline*}
        \sum_k \probability{X(0) = i, X(t) = k, X(t+h) = j} = \\
        \shoveright{= \sum_k \probability{X(0) = i} \cdot \conditionalprobability{X(t) = k}{X(0) = i} \cdot \conditionalprobability{X(t+h) = j}{X(t) = k},} \\
        %
        \shoveleft{\probability{X(0) = i, X(t+h) = j} =} \\
        = \probability{X(0) = i} \sum_k \conditionalprobability{X(t) = k}{X(0) = i} \cdot \conditionalprobability{X(t+h) = j}{X(t) = k}.
    \end{multline*}
    Делим на $\probability{X(0) = i}$, используем определение условной вероятности:
    \begin{gather*}
        \frac{\probability{X(0) = i, X(t+h) = j}}{\probability{X(0) = i}} = \sum_k \conditionalprobability{X(t) = k}{X(0) = i} \cdot \conditionalprobability{X(t+h) = j}{X(t) = k}, \\
        \conditionalprobability{X(t+h) = j}{X(0) = i} = \sum_k \conditionalprobability{X(t) = k}{X(0) = i} \cdot \conditionalprobability{X(t+h) = j}{X(t) = k}, \\
        p_{ij}(0, t+h) = \sum_k p_{ik}(0, t) \cdot p_{kj}(t, t+h) .
    \end{gather*}
    Используем однородность
    \[
        p_{ij}(t+h) = \sum_k p_{ik}(t) \cdot p_{kj}(h) .
    \]

    Суммируем по всем возможным состояния в момент времени 0 с учётом вероятностей:
    \begin{gather*}
        \sum_i p_{ij}(t+h) p_i(0) = \sum_i \left ( \sum_k p_{ik}(t) \cdot p_{kj}(h) \right ) p_i(0) , \\
        \sum_i p_{ij}(t+h) p_i(0) = \sum_k \left ( \sum_i p_i(0) p_{ik}(t) \right ) \cdot p_{kj}(h).
    \end{gather*}
    Используем формулу полной вероятности:
    \[
        p_j(t+h) = \sum_k p_k(t) \cdot p_{kj}(h) .
    \]
    Переименовываем индекс:
    \[
        p_j(t+h) = \sum_i p_i(t) \cdot p_{ij}(h) .
    \]

    \section*{Задача 12.5}
    Для процесса рождения и гибели с иммиграцией $X(t)$ из примера 3:
    \begin{enumerate}
        \item получить дифференциальное уравнение для тренда $\mu(t) = \expectation{X(t)}$;
        \item решить это уравнение и построить графики функций $\mu(t)$ для $\lambda = \mu$, $\lambda > \mu$ и $\lambda < \mu$.
    \end{enumerate}

    \subsection*{Решение:}
    Для функций $p_j(t) = \probability{X(t)=j}$ справедливо уравнение:
    \[
        p_j^{\prime}(t) = \sum_{i \neq j} q_{ij} p_i(t) - q_j p_j(t),
    \]
    где
    \begin{gather*}
        q_{ij}
        = \left \{
        \begin{array}{ll}
            i \lambda + \alpha, & j = i+1           \\
            i \mu,              & j = i-1           \\
            0,                  & \modulus{j-i} > 1
        \end{array}
        \right . , \\
        q_{j} = j \lambda + \alpha + j \mu ,
    \end{gather*}
    поэтому
    \[
        p_0^{\prime}(t) = \mu p_1(t) - \alpha p_0(t),
    \]
    а для $j \ge 1$:
    \begin{multline*}
        p_j^{\prime}(t)
        = ( (j-1) \lambda + \alpha ) p_{j-1}(t) + (j+1) \mu p_{j+1}(t) - ( j \lambda + \alpha + j \mu ) p_j(t) = \\
        %
        = \lambda (j-1) p_{j-1}(t) + \alpha p_{j-1}(t) + \mu (j+1) p_{j+1}(t) - ( \lambda + \mu ) j p_j(t) - \alpha p_j(t) .
    \end{multline*}
    Умножаем на $j$ левую и правую части и суммируем по всем состояниям $j$ (при $j=0$ уравнение для $p_0^\prime(t)$ можно не учитывать):
    \begin{multline*}
        \sum_{j=1}^\infty j p_j^{\prime}(t) = \\
        %
        = \lambda \sum_{j=1}^\infty j (j-1) p_{j-1}(t) + \alpha \sum_{j=1}^\infty j p_{j-1}(t) + \mu \sum_{j=1}^\infty j (j+1) p_{j+1}(t) - ( \lambda + \mu ) \sum_{j=1}^\infty j^2 p_j(t) - \alpha \sum_{j=1}^\infty j p_j(t).
    \end{multline*}
    Рассмотрим суммы:
    \begin{align*}
        \sum_{j=1}^\infty j p_j^{\prime}(t) &
        = \left ( \sum_{j=1}^\infty j p_j(t) \right )^\prime
        = \mu^\prime(t), \\
        %
        \sum_{j=1}^\infty j (j-1) p_{j-1}(t) &
        = \sum_{j=0}^\infty (j+1) j p_j(t)
        = \sum_{j=0}^\infty j^2 p_j(t) + \sum_{j=0}^\infty j p_j(t)
        = \mu_2(t) + \mu(t) , \\
        %
        \sum_{j=1}^\infty j p_{j-1}(t) &
        = \sum_{j=0}^\infty (j+1) p_j(t)
        = \sum_{j=0}^\infty j p_j(t) + \sum_{j=0}^\infty p_j(t)
        = \mu(t) + 1 , \\
        %
        \sum_{j=1}^\infty j (j+1) p_{j+1}(t) &
        = \sum_{j=2}^\infty (j-1) j p_j(t)
        = \sum_{j=2}^\infty j^2 p_j(t) - \sum_{j=2}^\infty j p_j(t) = \\ &
        = \sum_{j=2}^\infty j^2 p_j(t) + 1^2 \cdot p_1(t) - 1 \cdot p_1(t) - \sum_{j=2}^\infty j p_j(t) = \\ &
        = \sum_{j=1}^\infty j^2 p_j(t) - \sum_{j=1}^\infty j p_j(t)
        = \mu_2(t) - \mu(t) , \\
        %
        \sum_{j=1}^\infty j^2 p_j(t) &
        = \mu_2(t), \\
        %
        \sum_{j=1}^\infty j p_j(t) &
        = \mu(t)
    \end{align*}
    Таким образом,
    \begin{multline*}
        \mu^{\prime}(t)
        = \lambda ( \mu_2(t) + \mu(t) ) + \alpha ( \mu(t) + 1 ) + \mu ( \mu_2(t) - \mu(t) ) - ( \lambda + \mu ) \mu_2(t) - \alpha \mu(t) = \\
        %
        = ( \lambda + \mu - \lambda - \mu ) \mu_2(t) + ( \lambda + \alpha - \mu - \alpha ) \mu(t) + \alpha
        = ( \lambda - \mu ) \mu(t) + \alpha .
    \end{multline*}
    Уравнение для $\mu(t)$:
    \[
        \mu^{\prime}(t) = ( \lambda - \mu ) \mu(t) + \alpha .
    \]
    Если $\lambda \neq \mu$, то общее решение:
    \[
        \mu(t) = c(t) \cdot e^{(\lambda - \mu) t} .
    \]
    находится методом вариации постоянной:
    \begin{gather*}
        c^{\prime}(t) e^{(\lambda - \mu) t} + c(t) (\lambda - \mu) e^{(\lambda - \mu) t} = ( \lambda - \mu ) c(t) e^{(\lambda - \mu) t} + \alpha , \\
        c^{\prime}(t) e^{(\lambda - \mu) t} = \alpha , \\
        c^{\prime}(t) = \alpha e^{- (\lambda - \mu) t} , \\
        c(t) = - \frac{\alpha}{\lambda - \mu} e^{- (\lambda - \mu) t} + c ,
    \end{gather*}
    и $c$ - произвольная постоянная:
    \[
        \mu(t) = c \cdot e^{(\lambda - \mu) t} - \frac{\alpha}{\lambda - \mu}.
    \]
    Если $\lambda = \mu$, то уравнение преобразуется к виду:
    \[
        \mu^{\prime}(t) = \alpha ,
    \]
    и общее решение:
    \[
        \mu(t) = \alpha t + c,
    \]
    где $c$ --- произвольная постоянная.

    Постоянная $c$ определяется начальным условием --- значением $\mu(0)$.

    \begin{figure}[!ht]
        \centering
        \begin{tikzpicture}
            \draw [->] ( -0.5, 0 ) -- ( 6, 0 ) node [below] at ( 6, 0 ) {$t$};
            \draw [->] ( 0, -0.5 ) -- ( 0, 5 ) node [left] at ( 0, 5 ) {$N$};

            \draw [dashed, domain=0:3] plot ( \x, {exp(0.5*\x)} ) node [right] at ( 3, 4 ) {$\lambda > \mu$};
            \draw [domain=0:6] plot ( \x, {0.2*\x + 1} )  node [above] at ( 5, 2 ) {$\lambda = \mu$};
            \draw [dotted, domain=0:6] plot ( \x, {exp(-0.7*\x)} ) node [above] at ( 5, 0 ) {$\lambda < \mu$};
        \end{tikzpicture}
        \caption{Графики функции $\mu(t)$ при различных $\lambda$ и $\mu$.}
    \end{figure}
\end{document}