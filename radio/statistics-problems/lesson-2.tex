\documentclass[12pt]{article}

\usepackage[T1]{fontenc}
\usepackage[utf8]{inputenc}
\usepackage[english,russian]{babel}
\usepackage[margin=2cm]{geometry}
\usepackage{amsmath}

% команды вывода первой частной производной
\newcommand{\fpd}[1]{\frac{\partial}{\partial #1}}
\newcommand{\fpda}[2]{\frac{\partial #1}{\partial #2}}
\newcommand{\fpdp}[2]{\fpd{#2} \left ( #1 \right )}

\newcommand{\expectation}[1]{\mathtt{M} \left [ #1 \right ]}
\newcommand{\conditionalexpectation}[2]{\expectation{ #1 \left | #2 \right .}}
\newcommand{\variance}[1]{\mathtt{D} \left [ #1 \right ]}
\newcommand{\covariance}[2]{\mathtt{cov} \left ( #1, #2 \right )}

\newcommand{\modulus}[1]{\left | #1 \right |}
\newcommand{\norm}[1]{\left \| {#1} \right \|}

\newcommand{\event}[1]{\left \{ #1 \right \} }
\newcommand{\probability}[1]{P \event{#1}}


\begin{document}

    \title{Домашнее задание №2}
    \author{Тигетов Давид Георгиевич}
    \date{}
    \maketitle

    \section*{Задача 2.1}
    Доказать непосредственно из определения, что $\widehat{\theta} = \min \set{X_1, \dots, X_n}$ служит состоятельной оценкой параметра $\theta$ в модели с функцией распределения
    $F_\theta(x) = 1 - e^{-(x-\theta)}$ при $x > \theta$.

    \subsection*{Решение:}
    Пусть $F_{\widehat{\theta}}(y)$ --- функция распределения оценки $\widehat{\theta}$:
    \begin{multline*}
        F_{\widehat{\theta}}(y)
        = \probability{\widehat{\theta} < y}
        = 1 - \probability{\widehat{\theta} \ge y}
        = 1 - \probability{\min \set{X_1, \dots, X_n} \ge y} = \\
        %
        = 1 - \probability{X_1 \ge y, \dots, X_n \ge y}
        = 1 - \probability{X_1 \ge y} \cdot ... \cdot \probability{X_n \ge y} = \\
        %
        = 1 - \left ( 1 - F_\theta(y) \right )^n .
    \end{multline*}
    Пусть $\varepsilon$ --- произвольное положительное число, вычислим вероятность отклонения:
    \begin{multline*}
        \probability{\modulus{\widehat{\theta} - \theta} \le \varepsilon}
        = \probability{- \varepsilon \le \widehat{\theta} - \theta \le \varepsilon}
        = \probability{\theta - \varepsilon \le \widehat{\theta} \le \theta + \varepsilon}
        = F_{\widehat{\theta}}(\theta + \varepsilon) - F_{\widehat{\theta}}(\theta - \varepsilon) = \\
        %
        = 1 - \left ( 1 - F_\theta(\theta + \varepsilon) \right )^n - \left ( 1 - \left ( 1 - F_\theta(\theta - \varepsilon) \right )^n \right ) = \\
        %
        = 1 - \left ( 1 - F_\theta(\theta + \varepsilon) \right )^n - \left ( 1 - \left ( 1 - 0 \right )^n \right ) = \\
        %
        = 1 - \left ( 1 - F_\theta(\theta + \varepsilon) \right )^n
        = 1 - \left ( 1 - 1 + e^{-(\theta + \varepsilon - \theta)} \right )^n
        = 1 - e^{-n\varepsilon}.
    \end{multline*}
    Отсюда
    \[
        \probability{\modulus{\widehat{\theta} - \theta} > \varepsilon}
        = 1 - \probability{\modulus{\widehat{\theta} - \theta} \le \varepsilon}
        = 1 - 1 + e^{-n\varepsilon}
        = e^{-n\varepsilon}
        \rightarrow 0
        \text{, при } n \rightarrow \infty.
    \]
    Это по определению означает состоятельность оценки $\widehat{\theta}$.

    \section*{Задача 2.5}
    Доказать, что квадратичный риск произвольной оценки $\widehat{\theta}$ равен сумме $\variance{\widehat{\theta}} + b_n^2$, где $b_n = \expectation{\widehat{\theta}} - \theta$ ---
    смещение оценки $\widehat{\theta}$. Используя эту формулу и неравенство Чебышёва:
    \[
        \forall \varepsilon > 0 : \probability{\modulus{\xi} > \varepsilon} \le \frac{\expectation{\xi^2}}{\varepsilon^2},
    \]
    доказать следующую лемму о состоятельности (п. 3 темы 2.1): если дисперсия и смещение оценки стремятся к 0, то оценка состоятельна.

    \subsection*{Решение:}
    По определению квадратичного риска:
    \begin{multline*}
        R(\theta)
        = \expectation{\left ( \widehat{\theta} - \theta \right )^2}
        = \expectation{\left ( \widehat{\theta} - \expectation{\widehat{\theta}} + \expectation{\widehat{\theta}} - \theta \right )^2} = \\
        %
        = \expectation{\left ( \widehat{\theta} - \expectation{\widehat{\theta}} \right )^2 + 2 \left ( \widehat{\theta} - \expectation{\widehat{\theta}} \right ) \left ( \expectation{\widehat{\theta}} - \theta \right ) + \left ( \expectation{\widehat{\theta}} - \theta \right )^2} = \\
        %
        = \expectation{\left ( \widehat{\theta} - \expectation{\widehat{\theta}} \right )^2} + 2 \left ( \expectation{\widehat{\theta}} - \expectation{\widehat{\theta}} \right ) \left ( \expectation{\widehat{\theta}} - \theta \right ) + \left ( \expectation{\widehat{\theta}} - \theta \right )^2
        = \variance{\widehat{\theta}} + b_n^2 .
    \end{multline*}

    Пусть $\varepsilon > 0$ --- произвольное положительное число, тогда применение неравенства Чебышёва к величине $\widehat{\theta} - \theta$ приводит к неравенству:
    \[
        \probability{\modulus{\widehat{\theta} - \theta} > \varepsilon}
        \le \frac{\expectation{\left ( \widehat{\theta} - \theta \right )^2}}{\varepsilon^2}
        = \frac{\variance{\widehat{\theta}} + b_n^2}{\varepsilon^2}
        \rightarrow 0 ,
    \]
    если $\variance{\widehat{\theta}} \rightarrow 0$ и $b_n \rightarrow 0$ при $n \rightarrow \infty$.
\end{document}