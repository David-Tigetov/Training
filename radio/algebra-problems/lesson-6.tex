\documentclass[12pt]{article}

\usepackage[T1]{fontenc}
\usepackage[utf8]{inputenc}
\usepackage[english,russian]{babel}
\usepackage[margin=2cm]{geometry}
\usepackage{amsmath}
\usepackage{amsfonts}
\usepackage{xcolor}
\usepackage{color}

% команды вывода первой частной производной
\newcommand{\fpd}[1]{\frac{\partial}{\partial #1}}
\newcommand{\fpda}[2]{\frac{\partial #1}{\partial #2}}
\newcommand{\fpdp}[2]{\fpd{#2} \left ( #1 \right )}

\newcommand{\expectation}[1]{\mathtt{M} \left [ #1 \right ]}
\newcommand{\conditionalexpectation}[2]{\expectation{ #1 \left | #2 \right .}}
\newcommand{\variance}[1]{\mathtt{D} \left [ #1 \right ]}
\newcommand{\covariance}[2]{\mathtt{cov} \left ( #1, #2 \right )}

\newcommand{\modulus}[1]{\left | #1 \right |}
\newcommand{\norm}[1]{\left \| {#1} \right \|}

\newcommand{\event}[1]{\left \{ #1 \right \} }
\newcommand{\probability}[1]{P \event{#1}}


\begin{document}

    \title{Задача 6}
    \author{Тигетов Давид Георгиевич}
    \date{}
    \maketitle

    \section*{Пункт а}
    Ортогонализовать систему векторов по методу Грама-Шмидта.
    \[
        a =
        \begin{pmatrix}
            -1 \\ 1 \\ -1 \\ 1
        \end{pmatrix},
        b =
        \begin{pmatrix}
            -2 \\ -4 \\ 2 \\ 0
        \end{pmatrix},
        c =
        \begin{pmatrix}
            0 \\ -10 \\ 10 \\ 0
        \end{pmatrix},
        d =
        \begin{pmatrix}
            4 \\ 8 \\ 0 \\ 0
        \end{pmatrix}.
    \]
    \subsection*{Решение:}
    Ортогонализация $b$:
    \[
        \widetilde{b}
        = b - \frac{\scalarproduct{b}{a}}{\scalarproduct{a}{a}} a
        = b - \frac{2-4-2}{4} a
        = b + a
        =
        \begin{pmatrix}
            -3 \\ -3 \\ 1 \\ 1
        \end{pmatrix} ,
    \]

    Ортогонализация $c$:
    \begin{multline*}
        \widetilde{c}
        = c - \frac{\scalarproduct{c}{a}}{\scalarproduct{a}{a}} a - \frac{\scalarproduct{c}{\widetilde{b}}}{\scalarproduct{\widetilde{b}}{\widetilde{b}}} \widetilde{b}
        = c - \frac{-10 - 10}{4} a - \frac{30 + 10}{9 + 9 + 1 + 1} \widetilde{b}
        = c - \frac{-20}{4} a - \frac{40}{20} \widetilde{b}
        = c + 5 a - 2 \widetilde{b} = \\
        %
        =
        \begin{pmatrix}
            0 - 5 + 6   \\
            -10 + 5 + 6 \\
            10 - 5 - 2  \\
            0 + 5 - 2
        \end{pmatrix}
        =
        \begin{pmatrix}
            1 \\
            1 \\
            3 \\
            3
        \end{pmatrix},
    \end{multline*}

    Ортогонализация $d$:
    \begin{multline*}
        \widetilde{d}
        = d - \frac{\scalarproduct{d}{a}}{\scalarproduct{a}{a}} a - \frac{\scalarproduct{d}{\widetilde{b}}}{\scalarproduct{\widetilde{b}}{\widetilde{b}}} \widetilde{b} - \frac{\scalarproduct{d}{\widetilde{c}}}{\scalarproduct{\widetilde{c}}{\widetilde{c}}} \widetilde{c}
        = d - \frac{-4 + 8}{4} a - \frac{-12 - 24}{20} \widetilde{b} - \frac{4 + 8}{20} \widetilde{c}
        = d - a + \frac{36}{20} \widetilde{b} - \frac{12}{20} \widetilde{c} = \\
        %
        =
        \begin{pmatrix}
            4 + 1 - \frac{3 \cdot 12}{20} \cdot 3 - \frac{12}{20} \\
            8 - 1 - \frac{3 \cdot 12}{20} \cdot 3 - \frac{12}{20} \\
            0 + 1 + \frac{3 \cdot 12}{20} - \frac{12}{20} \cdot 3 \\
            0 - 1 + \frac{3 \cdot 12}{20} - \frac{12}{20} \cdot 3
        \end{pmatrix}
        =
        \begin{pmatrix}
            5 - \frac{10 \cdot 12}{20} \\
            7 - \frac{10 \cdot 12}{20} \\
            1                          \\
            -1
        \end{pmatrix}
        =
        \begin{pmatrix}
            5 - 6 \\
            7 - 6 \\
            1     \\
            -1
        \end{pmatrix}
        =
        \begin{pmatrix}
            -1 \\
            1  \\
            1  \\
            -1
        \end{pmatrix}
    \end{multline*}

    \subsection*{Ответ:}
    $
    \begin{pmatrix}
        -1 \\ 1 \\ -1 \\ 1
    \end{pmatrix},
    \begin{pmatrix}
        -3 \\ -3 \\ 1 \\ 1
    \end{pmatrix},
    \begin{pmatrix}
        1 \\ 1 \\ 3 \\ 3
    \end{pmatrix},
    \begin{pmatrix}
        -1 \\ 1 \\ 1 \\ -1
    \end{pmatrix}
    $.

    \section*{Пункт б}
    Найти базис ортогонального дополнения $V^\perp$ подпространства $V = \left< a_1, a_2, a_3 \right>$.
    \[
        a_1 =
        \begin{pmatrix}
            1 \\ 0 \\ -2 \\ -1
        \end{pmatrix},
        a_2 =
        \begin{pmatrix}
            1 \\ 1 \\ 5 \\ 4
        \end{pmatrix},
        a_1 =
        \begin{pmatrix}
            0 \\ 1 \\ 1 \\ 1
        \end{pmatrix}.
    \]

    \subsection*{Решение:}
    Преобразуем матрицу из векторов-столбцов:
    \[
        \begin{pmatrix}
            1  & 1 & 0 \\
            0  & 1 & 1 \\
            -2 & 5 & 1 \\
            -1 & 4 & 1
        \end{pmatrix}
        \rightarrow
        \begin{pmatrix}
            1  & 0 & 0 \\
            0  & 1 & 1 \\
            -2 & 7 & 1 \\
            -1 & 5 & 1
        \end{pmatrix}
        \rightarrow
        \begin{pmatrix}
            1  & 0 & 0 \\
            0  & 0 & 1 \\
            -2 & 6 & 1 \\
            -1 & 4 & 1
        \end{pmatrix}
        = A
    \]
    Теперь нужно найти решение ОСЛУ:
    \begin{gather*}
        x^T A = 0, \\
        x =
        \begin{pmatrix}
            -1 \\
            -1 \\
            -2 \\
            3
        \end{pmatrix}
    \end{gather*}

    \subsection*{Ответ:}
    $\left ( -1, -1, -2, 3 \right )^T$

    \section*{Пункт в}
    Ортогонализовать систему векторов по методу Грама-Шмидта.
    \[
        a =
        \begin{pmatrix}
            1 + i \\
            0     \\
            1 - i
        \end{pmatrix},
        b =
        \begin{pmatrix}
            1 + i \\
            -i    \\
            1 + i
        \end{pmatrix},
        c =
        \begin{pmatrix}
            -2 i \\
            1    \\
            2
        \end{pmatrix}
    \]

    \subsection*{Решение:}
    Ортогонализация $b$:
    \begin{multline*}
        \widetilde{b}
        = b - \frac{\overline{\scalarproduct{b}{a}}}{\scalarproduct{a}{a}} a
        = b - \frac{\overline{(1-i)(1+i) + (1-i)(1-i)}}{(1-i)(1+i) + (1+i)(1-i)} a
        = b - \frac{\overline{1 - i^2 + 1 - 2i + i^2}}{1 - i^2 + 1 - i^2} a = \\
        %
        = b - \frac{\overline{2 - 2 i}}{4} a
        = b - \frac{1+i}{2} a
        = \begin{pmatrix}
              1 + i - \frac{1+i}{2} (1+i) \\
              -i                          \\
              1 + i - \frac{1+i}{2} (1-i)
        \end{pmatrix}
        = \begin{pmatrix}
              1 + i - \frac{1 + 2i + i^2}{2} \\
              -i                             \\
              1 + i - \frac{1 - i^2}{2}
        \end{pmatrix}
        = \\
        %
        = \begin{pmatrix}
              1 + i - i \\
              -i        \\
              1 + i - 1
        \end{pmatrix}
        =
        \begin{pmatrix}
            1  \\
            -i \\
            i
        \end{pmatrix}
        .
    \end{multline*}
    Проверка:
    \[
        \scalarproduct{\widetilde{b}}{a}
        = 1 ( 1 + i ) + (-i) ( 1 - i )
        = 1 + i - i + i^2
        = 0.
    \]

    Ортогонализация $c$:
    \begin{multline*}
        \widetilde{c}
        = c - \frac{\overline{\scalarproduct{c}{a}}}{\scalarproduct{a}{a}} a - \frac{\overline{\scalarproduct{c}{\widetilde{b}}}}{\scalarproduct{\widetilde{b}}{\widetilde{b}}} \widetilde{b}
        = c - \frac{\overline{2i (1+i) + 2 (1-i)}}{4} a - \frac{\overline{2i - i + 2i}}{1 - i^2 - i^2} \widetilde{b} = \\
        %
        = c - \frac{\overline{2i + 2i^2 + 2 - 2i}}{4} a - \frac{\overline{3i}}{3} \widetilde{b}
        = c - \frac{\overline{0}}{4} a + \frac{-3i}{3} \widetilde{b}
        = c + i \widetilde{b}
        =
        \begin{pmatrix}
            -2i + i \\
            1 - i^2 \\
            2 + i^2
        \end{pmatrix}
        =
        \begin{pmatrix}
            -i \\
            2  \\
            1
        \end{pmatrix}
        .
        %
    \end{multline*}
    Проверка:
    \begin{gather*}
        \scalarproduct{\widetilde{c}}{a} = i ( 1 + i ) + 1 - i = i + i^2 + 1 - i = 0 , \\
        \scalarproduct{\widetilde{c}}{\widetilde{b}} = i - 2i + i = 0 .
    \end{gather*}

    \subsection*{Ответ:}
    $
    \begin{pmatrix}
        1 + i \\
        0     \\
        1 - i
    \end{pmatrix},
    \begin{pmatrix}
        1  \\
        -i \\
        i
    \end{pmatrix},
    \begin{pmatrix}
        -i \\
        2  \\
        1
    \end{pmatrix}
    $.

    \section*{Пункт г}
    Выписать ОСЛУ, задающую $V^\perp$, если $V$ задано следующей ОСЛУ:
    \[
        \begin{pmatrix}
            1 & -2 & 1  & 3  \\
            3 & -4 & -1 & 7  \\
            0 & 1  & -2 & -1
        \end{pmatrix}
        \begin{pmatrix}
            x_1 \\
            x_2 \\
            x_3 \\
            x_4
        \end{pmatrix}
        = 0
    \]

    \subsection*{Решение:}
    Найдем базис $V$:
    \[
        \begin{pmatrix}
            1 & -2 & 1  & 3  \\
            3 & -4 & -1 & 7  \\
            0 & 1  & -2 & -1
        \end{pmatrix}
        \rightarrow
        \begin{pmatrix}
            1 & -2 & 1  & 3  \\
            0 & 2  & -4 & -2 \\
            0 & 1  & -2 & -1
        \end{pmatrix}
        \rightarrow
        \begin{pmatrix}
            1 & -2 & 1  & 3  \\
            0 & 1  & -2 & -1 \\
            0 & 0  & 0  & 0
        \end{pmatrix} .
    \]
    Базис $v_1$, $v_2$:
    \[
        v_1 =
        \begin{pmatrix}
            3 \\
            2 \\
            1 \\
            0
        \end{pmatrix},
        v_2 =
        \begin{pmatrix}
            -1 \\
            1 \\
            0 \\
            1
        \end{pmatrix}.
    \]

    \subsection*{Ответ:}
    Матрица ОСЛУ
    $
    \begin{pmatrix}
        3 & 2 & 1 & 0 \\
        -1 & 1 & 0 & 1
    \end{pmatrix}
    $.
\end{document}