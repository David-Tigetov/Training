\documentclass[12pt]{article}

\usepackage[T1]{fontenc}
\usepackage[utf8]{inputenc}
\usepackage[english,russian]{babel}
\usepackage[margin=2cm]{geometry}
\usepackage{amsmath}
\usepackage{amsfonts}
\usepackage{xcolor}
\usepackage{color}

\begin{document}

    \title{Задача 4}
    \author{Тигетов Давид Георгиевич}
    \date{}
    \maketitle

    \section*{Пункт а}
    Задать параметрически пересечение аффинных подпространств, которые содержат точки $P_1$ ($P_2$ соответственно), и чьи присоединенные линейные подпространства порождены векторами
    $a_1$, $b_1$, $c_1$ ($a_2$, $b_2$, $c_2$ соответственно)
    \begin{gather*}
        P_1 =
        \begin{pmatrix}
            1 \\ -1 \\ -1 \\ 1
        \end{pmatrix} ,
        a_1 =
        \begin{pmatrix}
            -1 \\ -2 \\ -1 \\ 1
        \end{pmatrix} ,
        b_1 =
        \begin{pmatrix}
            -1 \\ -1 \\ 1 \\ 0
        \end{pmatrix} ,
        c_1 =
        \begin{pmatrix}
            2 \\ 1 \\ 0 \\ 0
        \end{pmatrix} , \\
        %
        P_2 =
        \begin{pmatrix}
            2 \\ 1 \\ 0 \\ 0
        \end{pmatrix} ,
        a_2 =
        \begin{pmatrix}
            2 \\ 0 \\ 1 \\ 0
        \end{pmatrix} ,
        b_2 =
        \begin{pmatrix}
            1 \\ 0 \\ 1 \\ 0
        \end{pmatrix} ,
        c_2 =
        \begin{pmatrix}
            1 \\ 1 \\ 0 \\ 0
        \end{pmatrix} .
    \end{gather*}

    \subsection*{Решение:}
    Для определения пересечения аффинных подпространства найдем все наборы координат $x_1$, $x_2$, $x_3$ и $y_1$, $y_2$, $y_3$ такие, что:
    \begin{gather*}
        P_1 + a_1 x_1 + b_1 x_2 + c_1 x_3 = P_2 + a_2 y_1 + b_2 y_2 + c_2 y_3 , \\
        %
        a_1 x_1 + b_1 x_2 + c_1 x_3 - a_2 y_1 - b_2 y_2 - c_2 y_3 = P_2 - P_1 , \\
        %
        \begin{pmatrix}
            -1 & -1 & 2 \\
            -2 & -1 & 1 \\
            -1 & 1  & 0 \\
            1  & 0  & 0
        \end{pmatrix}
        \begin{pmatrix}
            x_1 \\ x_2 \\ x_3
        \end{pmatrix}
        -
        \begin{pmatrix}
            2 & 1 & 1 \\
            0 & 0 & 1 \\
            1 & 1 & 0 \\
            0 & 0 & 0
        \end{pmatrix}
        \begin{pmatrix}
            y_1 \\ y_2 \\ y_3
        \end{pmatrix}
        =
        \begin{pmatrix}
            1 \\
            2 \\
            1 \\
            -1
        \end{pmatrix} , \\
        %
        \begin{pmatrix}
            0 & -1 & 2 \\
            0 & -1 & 1 \\
            0 & 1  & 0 \\
            1 & 0  & 0
        \end{pmatrix}
        \begin{pmatrix}
            x_1 \\ x_2 \\ x_3
        \end{pmatrix}
        -
        \begin{pmatrix}
            2 & 1 & 1 \\
            0 & 0 & 1 \\
            1 & 1 & 0 \\
            0 & 0 & 0
        \end{pmatrix}
        \begin{pmatrix}
            y_1 \\ y_2 \\ y_3
        \end{pmatrix}
        =
        \begin{pmatrix}
            0 \\
            0 \\
            0 \\
            -1
        \end{pmatrix} , \\
        %
        \begin{pmatrix}
            0 & 0 & 2 \\
            0 & 0 & 1 \\
            0 & 1 & 0 \\
            1 & 0 & 0
        \end{pmatrix}
        \begin{pmatrix}
            x_1 \\ x_2 \\ x_3
        \end{pmatrix}
        -
        \begin{pmatrix}
            3 & 2 & 1 \\
            1 & 1 & 1 \\
            1 & 1 & 0 \\
            0 & 0 & 0
        \end{pmatrix}
        \begin{pmatrix}
            y_1 \\ y_2 \\ y_3
        \end{pmatrix}
        =
        \begin{pmatrix}
            0 \\
            0 \\
            0 \\
            -1
        \end{pmatrix} ,
    \end{gather*}
    \begin{gather}
        \label{4:a:system}
        \begin{pmatrix}
            0 & 0 & 0 \\
            0 & 0 & 1 \\
            0 & 1 & 0 \\
            1 & 0 & 0
        \end{pmatrix}
        \begin{pmatrix}
            x_1 \\ x_2 \\ x_3
        \end{pmatrix}
        -
        \begin{pmatrix}
            1 & 0 & -1 \\
            1 & 1 & 1  \\
            1 & 1 & 0  \\
            0 & 0 & 0
        \end{pmatrix}
        \begin{pmatrix}
            y_1 \\ y_2 \\ y_3
        \end{pmatrix}
        =
        \begin{pmatrix}
            0 \\
            0 \\
            0 \\
            -1
        \end{pmatrix} .
    \end{gather}

    При нулевой правой части получим координаты векторов, образующих присоединенное подпространство:
    \begin{gather*}
        \begin{array}{c}
            \begin{pmatrix}
                x_1 \\ x_2 \\ x_3
            \end{pmatrix}
            =
            \begin{pmatrix}
                0 \\ 1 \\ 1
            \end{pmatrix}             \\
            \begin{pmatrix}
                y_1 \\ y_2 \\ y_3
            \end{pmatrix}
            =
            \begin{pmatrix}
                0 \\ 1 \\ 0
            \end{pmatrix} \\
        \end{array} ,
        \begin{array}{c}
            \begin{pmatrix}
                x_1 \\ x_2 \\ x_3
            \end{pmatrix}
            =
            \begin{pmatrix}
                0 \\ 1 \\ 2
            \end{pmatrix}             \\
            \begin{pmatrix}
                y_1 \\ y_2 \\ y_3
            \end{pmatrix}
            =
            \begin{pmatrix}
                1 \\ 0 \\ 1
            \end{pmatrix}
        \end{array} ,
    \end{gather*}
    Следовательно, присоединенное пространство образовано векторами:
    \[
        b_1 + c_1 =
        \begin{pmatrix}
            1 \\ 0 \\ 1 \\ 0
        \end{pmatrix}
        = b_2,
        b_1 + 2 c_1 =
        \begin{pmatrix}
            3 \\ 1 \\ 1 \\ 0
        \end{pmatrix}
        = a_2 + c_2 .
    \]
    Кстати, векторы линейно-независимы.

    Решение системы \eqref{4:a:system} даёт вектор аффинного подпространства:
    \[
        \begin{array}{c}
            \begin{pmatrix}
                x_1 \\ x_2 \\ x_3
            \end{pmatrix}
            =
            \begin{pmatrix}
                -1 \\ 0 \\ 0
            \end{pmatrix}             \\
            \begin{pmatrix}
                y_1 \\ y_2 \\ y_3
            \end{pmatrix}
            =
            \begin{pmatrix}
                0 \\ 0 \\ 0
            \end{pmatrix} \\
        \end{array} ,
    \]
    откуда:
    \[
        P_1 - a_1 = P_2 ,
    \]
    то есть $P_2$ --- вектор из пересечения аффинных подпространств.

    \subsection*{Ответ:}
    $P_2 + \lambda_1 b_2 + \lambda_2 \left ( a_2 + c_2 \right )$, где $\lambda_1$ и $\lambda_2$ --- произвольные постоянные.

    \section*{Пункт б}
    Найти матрицу перехода от базиса $e_1$, $e_2$, $e_3$ к базису $\widetilde{e}_1$, $\widetilde{e}_2$, $\widetilde{e}_3$
    \[
        \begin{array}{l}
            e_1 =
            \begin{pmatrix}
                1 \\ 2 \\ 1
            \end{pmatrix},              \\
            \widetilde{e}_1
            =
            \begin{pmatrix}
                3 \\ 5 \\ 8
            \end{pmatrix},
        \end{array}
        \begin{array}{l}
            e_2 =
            \begin{pmatrix}
                2 \\ 3 \\ 3
            \end{pmatrix},              \\
            \widetilde{e}_2
            =
            \begin{pmatrix}
                5 \\ 14 \\ 13
            \end{pmatrix},
        \end{array}
        \begin{array}{l}
            e_3 =
            \begin{pmatrix}
                3 \\ 8 \\ 2
            \end{pmatrix},              \\
            \widetilde{e}_3
            =
            \begin{pmatrix}
                1 \\ 9 \\ 2
            \end{pmatrix}.
        \end{array}
    \]

    \subsection*{Решение:}
    Матрица перехода $P$ от базиса $e_i$ к базису $\widetilde{e}_i$ должна удовлетворять матричному уравнению:
    \begin{gather*}
        \begin{pmatrix}
            1 & 2 & 3 \\
            2 & 3 & 8 \\
            1 & 3 & 2
        \end{pmatrix}
        P
        =
        \begin{pmatrix}
            3 & 5  & 1 \\
            5 & 14 & 9 \\
            8 & 13 & 2
        \end{pmatrix} , \\
        %
        \begin{pmatrix}
            1 & 2  & 3  \\
            0 & -1 & 2  \\
            0 & 1  & -1
        \end{pmatrix}
        P
        =
        \begin{pmatrix}
            3  & 5 & 1 \\
            -1 & 4 & 7 \\
            5  & 8 & 1
        \end{pmatrix} , \\
        %
        \begin{pmatrix}
            1 & 0  & 7 \\
            0 & -1 & 2 \\
            0 & 0  & 1
        \end{pmatrix}
        P
        =
        \begin{pmatrix}
            1  & 13 & 15 \\
            -1 & 4  & 7  \\
            4  & 12 & 8
        \end{pmatrix} , \\
        %
        \begin{pmatrix}
            1 & 0  & 0 \\
            0 & -1 & 0 \\
            0 & 0  & 1
        \end{pmatrix}
        P
        =
        \begin{pmatrix}
            -27 & -71 & -41 \\
            -9  & -20 & -9  \\
            4   & 12  & 8
        \end{pmatrix} , \\
        %
        P
        =
        \begin{pmatrix}
            -27 & -71 & -41 \\
            9   & 20  & 9   \\
            4   & 12  & 8
        \end{pmatrix}
    \end{gather*}
    Проверка:
    \[
        \begin{pmatrix}
            1 & 2 & 3 \\
            2 & 3 & 8 \\
            1 & 3 & 2
        \end{pmatrix}
        P
        =
        \begin{pmatrix}
            -27 + 18 + 12 & -71 + 40 + 36  & -41 + 18 + 24 \\
            -54 + 27 + 32 & -142 + 60 + 96 & -82 + 27 + 64 \\
            -27 + 27 + 8  & -71 + 60 + 24  & -41 + 27 + 16 \\
        \end{pmatrix}
        =
        \begin{pmatrix}
            3 & 5  & 1 \\
            5 & 14 & 9 \\
            8 & 13 & 2
        \end{pmatrix}
    \]
    Ура!

    \subsection*{Ответ:}
    $
    \begin{pmatrix}
        -27 & -71 & -41 \\
        9   & 20  & 9   \\
        4   & 12  & 8
    \end{pmatrix}
    $

    \section*{Пункт в}
    Как изменится матрица перехода от первого базиса ко второму, если:
    \begin{enumerate}
        \item поменять местами два вектора первого базиса,
        \item поменять местами два вектора второго базиса,
        \item записать векторы обоих базисов в обратном порядке.
    \end{enumerate}
    \subsection*{Решение:}
    Пусть $e_1$, \dots, $e_n$ --- первый базис, $\widetilde{e}_1$, \dots, $\widetilde{e}_n$ --- второй базис и $P$ --- матрица перехода:
    \[
        \begin{pmatrix}
            \widetilde{e}_1, \dots, \widetilde{e}_n
        \end{pmatrix}
        =
        \begin{pmatrix}
            e_1, \dots, e_n
        \end{pmatrix}
        \begin{pmatrix}
            p_{11} & p_{12} & \dots  & p_{1n} \\
            p_{21} & p_{22} & \dots  & p_{2n} \\
            \vdots & \vdots & \ddots & \vdots \\
            p_{n1} & p_{n2} & \dots  & p_{nn} \\
        \end{pmatrix}
    \]

    Если меняются местами векторы $e_i$ и $e_j$, то в матрице перехода меняются местами строки $i$ и $j$.

    Если меняются местами векторы $\widetilde{e}_i$ и $\widetilde{e}_j$, то в матрице перехода меняются местами столбцы $i$ и $j$.

    Если векторы обоих базисов записываются в обратном порядке, то матрицу перехода нужно "отразить" относительно побочной диагонали (диагонали проходящей через элементы $p_{n1}$ и $p_{1n}$).

    \subsection*{Ответ:}
    \begin{enumerate}
        \item в матрице поменяются местами строки,
        \item в матрице поменяются местами столбцы,
        \item в матрице элементы "отразятся"{} относительно побочной диагонали.
    \end{enumerate}

    \section*{Пункт г}
    Пусть $\Pi_1$, $\Pi_2$ --- аффинные двумерные подпространства в $\mathbb{K}^n$. Докажите, что существует аффинное подпространство $\Pi$ размерности не более 5, содержащее и $\Pi_1$,
    и $\Pi_2$. Каким может быть расположение $\Pi_1$ и $\Pi_2$, если $n \ge 5$.

    \subsection*{Решение:}
    \begin{enumerate}
        \item
        Пусть $\Pi_1 = p_1 + U_1$ и $u_{11}$, $u_{12}$ --- базис $U_1$, и $\Pi_2 = p_2 + U_2$ и $u_{21}$, $u_{22}$ --- базис $U_2$. Образуем множество $\Pi$:
        \[
            \Pi = p_1 + \left < p_2 - p_1, u_{11}, u_{12}, u_{21}, u_{22} \right >.
        \]
        Множество $\Pi$ по построению является аффинным подпространством (это нужно доказывать?).

        Легко видеть, что
        \begin{gather*}
            \Pi_1 \subseteq \Pi :
            p \in \Pi_1
            \Rightarrow
            p = p_1 + c_1 u_{11} + c_2 u_{22} \in \Pi, \\
            %
            \Pi_2 \subseteq \Pi :
            p \in \Pi_2
            \Rightarrow
            p = p_2 + c_1 u_{21} + c_2 u_{22} = p_1 + (p_2 - p_1) + c_1 u_{21} + c_2 u_{22} \in \Pi .
        \end{gather*}

        Очевидно, что размерность присоединенного подпространства $\left < p_2 - p_1, u_{11}, u_{12}, u_{21}, u_{22} \right >$ никак не может быть больше 5: в наборе из 5 векторов ещё
        оказаться линейно зависимые подсистемы.

        \item Возможные варианты расположения $\Pi_1$ и $\Pi_2$ исчерпываются вариантами:
        \begin{enumerate}
            \item $p_1 - p_2 \notin U_1 + U_2$: $\Pi_1$ и $\Pi_2$ не пересекаются,
            \item $p_1 - p_2 \in U_1 + U_2$: $\Pi_1$ и $\Pi_2$ пересекаются, и размерность присоединенного подпространства $dim\left ( U_1 \cap U_2 \right ) \in \{ 0, 1, 2 \}$, поскольку $k \ge 5$
            допускает $U_1 \cap U_2 = \emptyset$.
        \end{enumerate}
    \end{enumerate}
\end{document}