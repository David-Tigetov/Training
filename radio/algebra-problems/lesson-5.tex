\documentclass[12pt]{article}

\usepackage[T1]{fontenc}
\usepackage[utf8]{inputenc}
\usepackage[english,russian]{babel}
\usepackage[margin=2cm]{geometry}
\usepackage{amsmath}
\usepackage{amsfonts}
\usepackage{xcolor}
\usepackage{color}

% команды вывода первой частной производной
\newcommand{\fpd}[1]{\frac{\partial}{\partial #1}}
\newcommand{\fpda}[2]{\frac{\partial #1}{\partial #2}}
\newcommand{\fpdp}[2]{\fpd{#2} \left ( #1 \right )}

\newcommand{\expectation}[1]{\mathtt{M} \left [ #1 \right ]}
\newcommand{\conditionalexpectation}[2]{\expectation{ #1 \left | #2 \right .}}
\newcommand{\variance}[1]{\mathtt{D} \left [ #1 \right ]}
\newcommand{\covariance}[2]{\mathtt{cov} \left ( #1, #2 \right )}

\newcommand{\modulus}[1]{\left | #1 \right |}
\newcommand{\norm}[1]{\left \| {#1} \right \|}

\newcommand{\event}[1]{\left \{ #1 \right \} }
\newcommand{\probability}[1]{P \event{#1}}


\begin{document}

    \title{Задача 5}
    \author{Тигетов Давид Георгиевич}
    \date{}
    \maketitle

    \section*{Пункт а}
    Составьте какую-нибудь ОСЛУ, для которой множество решений представляется в виде линейной оболочки следующих векторов.
    \[
        a_1 =
        \begin{pmatrix}
            1 \\ -1 \\ 1 \\ -1 \\ 1
        \end{pmatrix},
        a_2 =
        \begin{pmatrix}
            1 \\ 1 \\ 0 \\ 0 \\ 3
        \end{pmatrix},
        a_3 =
        \begin{pmatrix}
            3 \\ 1 \\ 1 \\ -1 \\ 7
        \end{pmatrix}.
    \]

    \subsection*{Решение:}
    Проверим векторы на линейную зависимость:
    \[
        \begin{pmatrix}
            1  & 1 & 3  \\
            -1 & 1 & 1  \\
            1  & 0 & 1  \\
            -1 & 0 & -1 \\
            1  & 3 & 7
        \end{pmatrix}
        \rightarrow
        \begin{pmatrix}
            2  & 1 & 2  \\
            0  & 1 & 0  \\
            1  & 0 & 1  \\
            -1 & 0 & -1 \\
            4  & 3 & 4
        \end{pmatrix}
        \rightarrow
        \begin{pmatrix}
            2  & 1 & 0 \\
            0  & 1 & 0 \\
            1  & 0 & 0 \\
            -1 & 0 & 0 \\
            4  & 3 & 0
        \end{pmatrix}
    \]
    Значит нужно выбрать три линейно независимых вектора, ортогональных двум ненулевым векторам-столбцам, например:
    \[
        \begin{pmatrix}
            -1 & 1  & 2 & 0 & 0 \\
            1  & -1 & 0 & 2 & 0 \\
            -6 & -3 & 0 & 0 & 3
        \end{pmatrix}
        .
    \]

    \subsection*{Ответ:}
    \[
        \begin{pmatrix}
            -1 & 1  & 2 & 0 & 0 \\
            1  & -1 & 0 & 2 & 0 \\
            -6 & -3 & 0 & 0 & 3
        \end{pmatrix}
        .
    \]

    \section*{Пункт б}
    Найдите длины сторон треугольника ABC в пространстве $\mathbb{R}^5$.
    \[
        A =
        \begin{pmatrix}
            2 \\ 4 \\ 2 \\ 4 \\ 2
        \end{pmatrix},
        B =
        \begin{pmatrix}
            6 \\ 4 \\ 4 \\ 4 \\ 6
        \end{pmatrix},
        C =
        \begin{pmatrix}
            5 \\ 7 \\ 5 \\ 7 \\ 2
        \end{pmatrix}.
    \]

    \subsection*{Решение:}
    Сторона $\overline{AB}$:
    \begin{gather*}
        \overline{AB}
        = B - A
        = \begin{pmatrix}
              4 \\ 0 \\ 2 \\ 0 \\ 4
        \end{pmatrix}, \\
        %
        \modulus{\overline{AB}}
        = \sqrt{4^2 + 2^2 + 4^2}
        = \sqrt{16 + 4 + 16}
        = \sqrt{36}
        = 6 .
    \end{gather*}

    Сторона $BC$:
    \begin{gather*}
        \overline{BC}
        = C - B
        = \begin{pmatrix}
              -1 \\ 3 \\ 1 \\ 3 \\ -4
        \end{pmatrix}, \\
        %
        \modulus{\overline{BC}}
        = \sqrt{(-1)^2 + 3^2 + 1^2 + 3^2 + (-4)^2}
        = \sqrt{1 + 9 + 1 + 9 + 16}
        = \sqrt{36}
        = 6 .
    \end{gather*}

    Сторона $AC$:
    \begin{gather*}
        \overline{AC}
        = C - A
        = \begin{pmatrix}
              3 \\ 3 \\ 3 \\ 3 \\ 0
        \end{pmatrix}, \\
        %
        \modulus{\overline{BC}}
        = \sqrt{3^2 + 3^2 + 3^2 + 3^2}
        = \sqrt{4 \cdot 3^2}
        = 2 \cdot 3
        = 6 .
    \end{gather*}

    \subsection*{Ответ:}
    Длины всех сторон равны 6.

    \section*{Пункт в}
    Задать системой линейных неоднородных уравнений аффинное подпространство, которое содержит точку $P$, и чье присоединенное линейное подпространство порождено векторами $a$, $b$,
    $c$, $d$:
    \begin{gather*}
        P =
        \begin{pmatrix}
            4 \\ 3 \\ 1 \\ 3 \\ 1
        \end{pmatrix},
        a =
        \begin{pmatrix}
            1 \\ 2 \\ 1 \\ 2 \\ 1
        \end{pmatrix},
        b =
        \begin{pmatrix}
            2 \\ 1 \\ -1 \\ 1 \\ 0
        \end{pmatrix},
        c =
        \begin{pmatrix}
            2 \\ -1 \\ -2 \\ 0 \\ -1
        \end{pmatrix},
        d =
        \begin{pmatrix}
            -1 \\ -1 \\ 1 \\ 0 \\ 0
        \end{pmatrix}.
    \end{gather*}

    \subsection*{Решение:}
    Попробуем упростить векторы $a$ -- $d$:
    \begin{gather*}
        \begin{pmatrix}
            1 & 2  & 2  & -1 \\
            2 & 1  & -1 & -1 \\
            1 & -1 & -2 & 1  \\
            2 & 1  & 0  & 0  \\
            1 & 0  & -1 & 0
        \end{pmatrix}
        \rightarrow
        \begin{pmatrix}
            3  & 2  & 2  & -1 \\
            1  & 1  & -1 & -1 \\
            -1 & -1 & -2 & 1  \\
            2  & 1  & 0  & 0  \\
            0  & 0  & -1 & 0
        \end{pmatrix}
        \rightarrow
        \begin{pmatrix}
            -1 & 2  & 2  & -1 \\
            -1 & 1  & -1 & -1 \\
            1  & -1 & -2 & 1  \\
            0  & 1  & 0  & 0  \\
            0  & 0  & -1 & 0
        \end{pmatrix}
        \rightarrow
        \begin{pmatrix}
            0 & 1 & 0  & -1 \\
            0 & 0 & -3 & -1 \\
            0 & 0 & 0  & 1  \\
            0 & 1 & 0  & 0  \\
            0 & 0 & -1 & 0
        \end{pmatrix}
        .
    \end{gather*}
    Матрица ОСЛУ имеет вид:
    \[
        A =
        \begin{pmatrix}
            -1 & 0  & -1 & 1 & 0 \\
            0  & -1 & -1 & 0 & 3
        \end{pmatrix}.
    \]
    Для определения правой части вычислим:
    \[
        A \cdot P =
        \begin{pmatrix}
            -4 - 1 + 3 \\
            -3 - 1 + 3
        \end{pmatrix}
        =
        \begin{pmatrix}
            -2 \\
            -1
        \end{pmatrix}
        .
    \]

    \subsection*{Ответ:}
    $
    \left \{
    \begin{array}{ccccccccccc}
        -x_1 &   &     & - & x_3 & + & x_4 &   &       & = & -2 \\
        & - & x_2 & - & x_3 &   &     & + & 3 x_5 & = & -1
    \end{array}
    \right .
    $

    \section*{Пункт г}
    Рассмотрим функцию $f$ от векторов $x, y \in \mathbb{R}^2$, заданную формулой
    \[
        f(x,y) =
        \begin{pmatrix}
            x_1 & x_2
        \end{pmatrix}
        \begin{pmatrix}
            a & b \\
            c & d
        \end{pmatrix}
        \begin{pmatrix}
            y_1 \\
            y_2
        \end{pmatrix}
        ,
        \text{где }
        x =
        \begin{pmatrix}
            x_1 \\
            x_2
        \end{pmatrix},
        y =
        \begin{pmatrix}
            y_1 \\
            y_2
        \end{pmatrix}.
    \]
    Доказать, что эта формула определяет евклидово скалярное произведение в $\mathbb{R}^2$ тогда и только тогда, когда $a > 0$, $ad - b^2 > 0$.

    \subsection*{Решение:}
    \begin{enumerate}
        \item
        Билинейность $f(x,y)$ очевидна и не накладывает никаких условий на $a$, $b$, $c$ и $d$.

        \item
        Равенство $f(x,y) = f(y,x)$ для всех $x$ и $y$ равносильно равенству:
        \begin{gather*}
            x^T
            \begin{pmatrix}
                a & b \\
                c & d
            \end{pmatrix}
            y
            =
            y^T
            \begin{pmatrix}
                a & b \\
                c & d
            \end{pmatrix}
            x , \\
            %
            x^T
            \begin{pmatrix}
                a & b \\
                c & d
            \end{pmatrix}
            y
            =
            x^T
            \begin{pmatrix}
                a & c \\
                b & d
            \end{pmatrix}
            y , \\
            %
            \begin{pmatrix}
                a & b \\
                c & d
            \end{pmatrix}
            =
            \begin{pmatrix}
                a & c \\
                b & d
            \end{pmatrix}, \\
            %
            b = c .
        \end{gather*}

        \item
        Из равенства $f(x,y)=f(y,x)$ и условия $f(x,x) > 0$ для всех $x \neq 0$ следует неравенство:
        \begin{gather*}
            f(x,x) > 0 , \\
            a x_1^2 + b x_1 x_2 + c x_2 x_1 + d x_2^2 > 0, \\
            a x_1^2 + 2 b x_1 x_2 + d x_2^2 > 0 ,
        \end{gather*}
        Возьмем $x_2 \neq 0$, тогда:
        \begin{gather*}
            a \left ( \frac{x_1}{x_2} \right )^2 + 2 b \frac{x_1}{x_2} + d > 0 .
        \end{gather*}
        Ветви параболы направлены вверх, дискриминант отрицательный:
        \begin{gather*}
            a > 0, \\
            b^2 - ad < 0 \Rightarrow ad - b^2 > 0 .
        \end{gather*}

        \item
        Таким образом,
        \[
            f(x,y) = f(y,x) \Leftrightarrow b = c
        \]
        и
        \[
            \left .
            \begin{array}{r}
                f(x,y) = f(y,x) \\
                \forall x \neq 0: f(x,x) > 0
            \end{array}
            \right \}
            \Rightarrow
            \left \{
            \begin{array}{l}
                a > 0, \\
                ad - b^2 > 0
            \end{array}
            \right .
        \]
        Мне кажется, что в постановке задачи не хватает условия для $c$.
    \end{enumerate}
\end{document}