\documentclass[12pt]{article}

\usepackage[T1]{fontenc}
\usepackage[utf8]{inputenc}
\usepackage[english,russian]{babel}
\usepackage[margin=2cm]{geometry}
\usepackage{amsmath}
\usepackage{amsfonts}
\usepackage{xcolor}
\usepackage{color}
\usepackage{amssymb}
\usepackage{lscape}

% команды вывода первой частной производной
\newcommand{\fpd}[1]{\frac{\partial}{\partial #1}}
\newcommand{\fpda}[2]{\frac{\partial #1}{\partial #2}}
\newcommand{\fpdp}[2]{\fpd{#2} \left ( #1 \right )}

\newcommand{\expectation}[1]{\mathtt{M} \left [ #1 \right ]}
\newcommand{\conditionalexpectation}[2]{\expectation{ #1 \left | #2 \right .}}
\newcommand{\variance}[1]{\mathtt{D} \left [ #1 \right ]}
\newcommand{\covariance}[2]{\mathtt{cov} \left ( #1, #2 \right )}

\newcommand{\modulus}[1]{\left | #1 \right |}
\newcommand{\norm}[1]{\left \| {#1} \right \|}

\newcommand{\event}[1]{\left \{ #1 \right \} }
\newcommand{\probability}[1]{P \event{#1}}


\begin{document}

    \title{Задача 13}
    \author{Тигетов Давид Георгиевич}
    \date{}
    \maketitle

    \section*{Пункт а \textcolor{red}{[верно]}}
    Найдите жорданову нормальную форму и жорданов базис для операторов, заданных матрицами:
    \[
        A =
        \begin{pmatrix}
            -2 & -3 & -4 \\
            -1 & 0  & 0  \\
            1  & 1  & 1
        \end{pmatrix},
        B =
        \begin{pmatrix}
            -5 & -9 & 2  & 7   \\
            -3 & -2 & 1  & 3   \\
            9  & 9  & -5 & -12 \\
            -9 & -9 & 3  & 10
        \end{pmatrix}
    \]

    \subsection*{Решение:}
    У матрицы $A$ два собственных значения $\lambda_1 = -1$ и $\lambda_2 = 1$.

    Для собственного значения $\lambda_1$ есть только один собственный вектор, поэтому строим базис в корневом подпространстве $\kernel{\left ( A - \lambda_1 E \right)^2}$, в котором
    выберем вектор
    \begin{gather*}
        v_2 \in \kernel{\left ( A - \lambda_1 E \right)^2}, \\
        v_2 \notin \kernel{\left ( A - \lambda_1 E \right)}.
    \end{gather*}
    Таким вектором является:
    \[
        v_2 = \begin{pmatrix}
                  1 \\ 0 \\ 0
        \end{pmatrix}.
    \]
    Под действием $\left ( A - \lambda_1 E \right )$ вектор $v_1$ переходит во второй базисный вектор $v_1$:
    \[
        v_1
        = \left ( A - \lambda_1 E \right ) v_2
        = \begin{pmatrix}
              -1 \\ -1 \\ 1
        \end{pmatrix}.
    \]

    Для собственного значения $\lambda_2$ есть собственный вектор:
    \[
        v_3 = \begin{pmatrix}
                  -1 \\ 1 \\ 0
        \end{pmatrix} .
    \]

    Матрица перехода к базису $v_1$, $v_2$, $v_3$:
    \[
        C_A =
        \begin{pmatrix}
            -1 & 1 & -1 \\
            -1 & 0 & 1  \\
            1  & 0 & 0
        \end{pmatrix}.
    \]

    Матрица оператора в базисе $v_1$, $v_2$, $v_3$:
    \[
        J_A
        = C_A^{-1} A C_A
        = \begin{pmatrix}
              -1 & 1  & 0 \\
              0  & -1 & 0 \\
              0  & 0  & 1
        \end{pmatrix}.
    \]

    У матрицы $B$ два собственных значения $\lambda_1 = -2$ и $\lambda_2 = 1$.

    Для $\lambda_1$ всего один собственный вектор --- требуется рассмотреть корневое подпространство. Кратность корня равна 2, поэтому достаточно рассмотреть
    $\kernel{\left ( B - \lambda_1 E \right )^2}$. Возьмем вектор
    \begin{gather*}
        v_2 \in \kernel{\left ( B - \lambda_1 E \right)^2}, \\
        v_2 \notin \kernel{\left ( B - \lambda_1 E \right)}, \\
        v_2 = \begin{pmatrix}
                  0 \\ 0 \\ 1 \\ 0
        \end{pmatrix}.
    \end{gather*}
    В качестве второго базисного вектора из $\kernel{\left ( B - \lambda_1 E \right )^2}$ возьмем
    \[
        v_1
        = \left ( B - \lambda_1 E \right ) v_2
        = \begin{pmatrix}
              2 \\ 1 \\ -3 \\ 3
        \end{pmatrix} .
    \]

    Для $\lambda_2$ также один собственный вектор, рассматриваем корневое подпространство $\kernel{\left ( B - \lambda_1 E \right )^2}$:
    \begin{gather*}
        v_4 \in \kernel{\left ( B - \lambda_2 E \right)^2}, \\
        v_4 \notin \kernel{\left ( B - \lambda_2 E \right)}, \\
        v_4 = \begin{pmatrix}
                  -2 \\ 1 \\ 0 \\ 0
        \end{pmatrix}, \\
        %
        v_3
        = \left ( B - \lambda_2 E \right) v_4
        = \begin{pmatrix}
              3 \\ 3 \\ -9 \\ 9
        \end{pmatrix}.
    \end{gather*}

    Матрица перехода к базису $v_1$, $v_2$, $v_3$, $v_4$:
    \[
        C_B
        = \begin{pmatrix}
              2  & 0 & 3  & -2 \\
              1  & 0 & 3  & 1  \\
              -3 & 1 & -9 & 0  \\
              3  & 0 & 9  & 0
        \end{pmatrix}.
    \]
    Матрица оператора в базисе $v_1$, $v_2$, $v_3$, $v_4$:
    \[
        J_B
        = C_B^{-1} B C_B
        = \begin{pmatrix}
              -2 & 1  & 0 & 0 \\
              0  & -2 & 0 & 0 \\
              0  & 0  & 1 & 1 \\
              0  & 0  & 0 & 1
        \end{pmatrix}.
    \]

    \subsection*{Ответ:}
    Для матрицы $A$ жорданова форма:
    \[
        \begin{pmatrix}
            -1 & 1  & 0 \\
            0  & -1 & 0 \\
            0  & 0  & 1
        \end{pmatrix}.
    \]
    в жордановом базисе:
    \begin{gather*}
        \begin{pmatrix}
            -1 \\ -1 \\ 1
        \end{pmatrix},
        \begin{pmatrix}
            1 \\ 0 \\ 0
        \end{pmatrix},
        \begin{pmatrix}
            -1 \\ 1 \\ 0
        \end{pmatrix} .
    \end{gather*}

    Для матрицы $B$ жорданова форма:
    \[
        \begin{pmatrix}
            -2 & 1  & 0 & 0 \\
            0  & -2 & 0 & 0 \\
            0  & 0  & 1 & 1 \\
            0  & 0  & 0 & 1
        \end{pmatrix}.
    \]
    в жордановом базисе:
    \begin{gather*}
        \begin{pmatrix}
            2 \\ 1 \\ -3 \\ 3
        \end{pmatrix} ,
        \begin{pmatrix}
            0 \\ 0 \\ 1 \\ 0
        \end{pmatrix},
        \begin{pmatrix}
            3 \\ 3 \\ -9 \\ 9
        \end{pmatrix},
        \begin{pmatrix}
            -2 \\ 1 \\ 0 \\ 0
        \end{pmatrix}.
    \end{gather*}

    \section*{Пункт б}
    Найдите $A^{1000000}$ и $B^{1000000}$, где
    \[
        A =
        \begin{pmatrix}
            -2 & -3 & -4 \\
            -1 & 0  & 0  \\
            1  & 1  & 1
        \end{pmatrix},
        B =
        \begin{pmatrix}
            -5 & -9 & 2  & 7   \\
            -3 & -2 & 1  & 3   \\
            9  & 9  & -5 & -12 \\
            -9 & -9 & 3  & 10
        \end{pmatrix}.
    \]

    \subsection*{Решение:}
    Используя нормальную жорданову форму из пункта а:
        {
        \color{blue}
        \begin{gather*}
            J_A = C_A^{-1} A C_A , \\
            C_A J_A C_A^{-1} = A , \\
            C_A J_A^N C_A^{-1} = A^N ,
        \end{gather*}
    }
    \[
        J_A^N
        = \begin{pmatrix}
              \begin{pmatrix}
                  -1 & 1  \\
                  0  & -1
              \end{pmatrix}^N & 0 \\
              0 & 1^N
        \end{pmatrix}
        = \begin{pmatrix}
        (-1)
              ^N & (-1)^{\textcolor{blue}{N-1}} \cdot N & 0   \\
              0  & (-1)^N                               & 0   \\
              0  & 0                                    & 1^N
        \end{pmatrix} .
    \]
    Таким образом,
        {
        \color{blue}
        \[
            A^{1000000}
            = C_A
            \begin{pmatrix}
                1 & -1000000 & 0 \\
                0 & 1        & 0 \\
                0 & 0        & 1
            \end{pmatrix}
            C_A^{-1}
            = \begin{pmatrix}
                  1000001  & 1000000  & 2000000  \\
                  1000000  & 1000001  & 2000000  \\
                  -1000000 & -1000000 & -1999999
            \end{pmatrix} .
        \]
    }

    Аналогично
    \[
        B^N = \textcolor{blue}{C_B} J_B^N \textcolor{blue}{C_B^{-1}} ,
    \]
    где
    \[
        J_B^N
        = \begin{pmatrix}
              \begin{pmatrix}
                  -2 & 1  \\
                  0  & -2
              \end{pmatrix}^N & 0 \\
              0 & \begin{pmatrix}
                      1 & 1 \\
                      0 & 1
              \end{pmatrix}^N
        \end{pmatrix}
        =
        \begin{pmatrix}
        (-2)
            ^N & (-2)^{N-1} \cdot N & 0   & 0                                 \\
            0  & (-2)^N             & 0   & 0                                 \\
            0  & 0                  & 1^N & \textcolor{blue}{1^{N-1}} \cdot N \\
            0  & 0                  & 0   & 1^N                               \\
        \end{pmatrix},
    \]
    тогда
    \[
        \color{blue}
        B^N
        = \begin{pmatrix}
              2  & 0 & 3  & -2 \\
              1  & 0 & 3  & 1  \\
              -3 & 1 & -9 & 0  \\
              3  & 0 & 9  & 0
        \end{pmatrix}
        \begin{pmatrix}
        (-2)
            ^N & (-2)^{N-1} \cdot N & 0   & 0                                 \\
            0  & (-2)^N             & 0   & 0                                 \\
            0  & 0                  & 1^N & \textcolor{blue}{1^{N-1}} \cdot N \\
            0  & 0                  & 0   & 1^N                               \\
        \end{pmatrix}
        \begin{pmatrix}
            1             & 2             & 0 & -1            \\
            0             & 0             & 1 & 1             \\
            - \frac{1}{3} & - \frac{2}{3} & 0 & \frac{4}{9}   \\
            0             & 1             & 0 & - \frac{1}{3}
        \end{pmatrix}
        .
    \]
    \begin{landscape}
        \color{blue}
        \begin{multline*}
            B^{1000000}
            = \begin{pmatrix}
                  2 \cdot (-2)^{1000000} - 1 & 4 \cdot (-2)^{1000000} - 4 + 3000000   & 2000000 \cdot (-2)^{99999}                   & -2 \cdot (-2)^{1000000} + 2000000 \cdot (-2)^{99999} + 2 - 1000000 \\
                  (-2)^{1000000} - 1         & 2 \cdot (-2)^{1000000} - 1 + 3000000   & 1000000 \cdot (-2)^{99999}                   & - (-2)^{1000000} + 1000000 \cdot (-2)^{99999} + 1 - 1000000 \\
                  3 - 3 \cdot (-2)^{1000000} & - 6 \cdot (-2)^{1000000} + 6 - 9000000 & (-2)^{1000000} - 3000000 \cdot (-2)^{999999} & 4 \cdot (-2)^{1000000} - 3000000 \cdot (-2)^{999999} - 4 + 3000000 \\
                  3 \cdot (-2)^{1000000} - 3 & 6 \cdot (-2)^{1000000} - 6 + 9000000   & 3000000 \cdot (-2)^{99999}                   & - 3 \cdot (-2)^{1000000} + 3000000 \cdot (-2)^{999999} + 4 - 3000000
            \end{pmatrix}
            = \\
            %
            = \begin{pmatrix}
                  2 \cdot (-2)^{1000000} - 1 & 4 \cdot (-2)^{1000000} + 2999996   & 2000000 \cdot (-2)^{99999}                   & -2 \cdot (-2)^{1000000} + 2000000 \cdot (-2)^{99999} - 999998 \\
                  (-2)^{1000000} - 1         & 2 \cdot (-2)^{1000000} + 2999999   & 1000000 \cdot (-2)^{99999}                   & - (-2)^{1000000} + 1000000 \cdot (-2)^{99999} - 999999           \\
                  3 - 3 \cdot (-2)^{1000000} & - 6 \cdot (-2)^{1000000} - 8999994 & (-2)^{1000000} - 3000000 \cdot (-2)^{999999} & 4 \cdot (-2)^{1000000} - 3000000 \cdot (-2)^{999999} + 2999996 \\
                  3 \cdot (-2)^{1000000} - 3 & 6 \cdot (-2)^{1000000} + 8999994   & 3000000 \cdot (-2)^{99999}                   & - 3 \cdot (-2)^{1000000} + 3000000 \cdot (-2)^{999999} - 2999996
            \end{pmatrix}
        \end{multline*}
        \color{blue}
    \end{landscape}

    \subsection*{Ответ:}
    $
    A^{1000000}
    = \begin{pmatrix}
          1000001  & 1000000  & 2000000  \\
          1000000  & 1000001  & 2000000  \\
          -1000000 & -1000000 & -1999999
    \end{pmatrix}
    $,
    $B^{1000000}  = ...$.

    \section*{Пункт в \textcolor{red}{[верно]}}
    Найдите жорданову форму и жорданов базис для оператора, заданного матрицей
    \[
        A = \begin{pmatrix}
                1 & -1 & 1 & 0  \\
                1 & 1  & 0 & 1  \\
                0 & 0  & 1 & -1 \\
                0 & 0  & 1 & 1
        \end{pmatrix} .
    \]

    \subsection*{Решение:}
    У матрицы $A$ два собственных значения $\lambda_1 = 1 - i$ и $\lambda_2 = 1 + i$.

    Собственное подпространство $\kernel{A - \lambda_1 E}$ имеет размерность 1, поэтому рассмотрим корневое подпространство $\kernel{\left ( A - \lambda_1 E \right )^2}$,
    в котором выберем базисные векторы:
    \begin{gather*}
        v_2 \in \kernel{\left ( A - \lambda_1 E \right)^2}, \\
        v_2 \notin \kernel{\left ( A - \lambda_1 E \right)}, \\
        v_2 = \begin{pmatrix}
                  0 \\ 0 \\ -i \\ 1
        \end{pmatrix}, \\
        v_1
        = \left ( A - \lambda_1 E \right) v_2
        = \begin{pmatrix}
              -i \\ 1 \\ 0 \\ 0
        \end{pmatrix}.
    \end{gather*}

    Аналогично в корневом подпростанстве $\kernel{\left ( A - \lambda_2 E \right )^2}$ выберем базисные векторы
    \begin{gather*}
        v_4 \in \kernel{\left ( A - \lambda_2 E \right)^2}, \\
        v_4 \notin \kernel{\left ( A - \lambda_2 E \right)}, \\
        v_4 = \begin{pmatrix}
                  0 \\ 0 \\ i \\ 1
        \end{pmatrix}, \\
        v_3
        = \left ( A - \lambda_1 E \right) v_2
        = \begin{pmatrix}
              i \\ 1 \\ 0 \\ 0
        \end{pmatrix} .
    \end{gather*}

    Матрица перехода:
    \[
        C
        = \begin{pmatrix}
              -i & 0  & i & 0 \\
              1  & 0  & 1 & 0 \\
              0  & -i & 0 & i \\
              0  & 1  & 0 & 1
        \end{pmatrix}.
    \]

    В базисе $v_1$, $v_2$, $v_3$, $v_4$ матрица оператора имеет вид:
    \[
        J
        = C^{-1} A C
        = \begin{pmatrix}
              1 - i & 1     & 0     & 0     \\
              0     & 1 - i & 0     & 0     \\
              0     & 0     & 1 + i & 1     \\
              0     & 0     & 0     & 1 + i
        \end{pmatrix}.
    \]

    \subsection*{Ответ:}
    Жорданова форма:
    \[
        \begin{pmatrix}
            1 - i & 1     & 0     & 0     \\
            0     & 1 - i & 0     & 0     \\
            0     & 0     & 1 + i & 1     \\
            0     & 0     & 0     & 1 + i
        \end{pmatrix}.
    \]
    в базисе
    \[
        \begin{pmatrix}
            -i \\ 1 \\ 0 \\ 0
        \end{pmatrix}, \\
        \begin{pmatrix}
            0 \\ 0 \\ -i \\ 1
        \end{pmatrix}, \\
        \begin{pmatrix}
            i \\ 1 \\ 0 \\ 0
        \end{pmatrix} ,
        \begin{pmatrix}
            0 \\ 0 \\ i \\ 1
        \end{pmatrix} .
    \]

    \section*{Пункт г}
    Найдите $A^{1000000}$, где
    \[
        A = \begin{pmatrix}
                1 & -1 & 1 & 0  \\
                1 & 1  & 0 & 1  \\
                0 & 0  & 1 & -1 \\
                0 & 0  & 1 & 1
        \end{pmatrix} .
    \]

    \subsection*{Решение:}
    Согласно пункту в
    \begin{multline*}
        A^N
        = \textcolor{blue}{C} J^N \textcolor{blue}{C^{-1}}
        = \textcolor{blue}{C}
        \begin{pmatrix}
            \begin{pmatrix}
                1 - i & 1     \\
                0     & 1 - i
            \end{pmatrix}^N & 0 \\
            0 & \begin{pmatrix}
                    1 + i & 1     \\
                    0     & 1 + i
            \end{pmatrix}^N
        \end{pmatrix}
        \textcolor{blue}{C^{-1}} = \\
        %
        =
        \begin{pmatrix}
            -i & 0  & i & 0 \\
            1  & 0  & 1 & 0 \\
            0  & -i & 0 & i \\
            0  & 1  & 0 & 1
        \end{pmatrix}
        \begin{pmatrix}
        ( 1 - i )
            ^N & ( 1 - i )^{N-1} \cdot N & 0           & 0                   \\
            0  & ( 1 - i )^N             & 0           & 0                   \\
            0  & 0                       & ( 1 + i )^N & ( 1 + i )^N \cdot N \\
            0  & 0                       & 0           & ( 1 + i )^N
        \end{pmatrix}
        \begin{pmatrix}
            \frac{i}{2}   & \frac{1}{2} & 0            & 0           \\
            0             & 0           & \frac{i}{2}  & \frac{1}{2} \\
            - \frac{i}{2} & \frac{1}{2} & 0            & 0           \\
            0             & 0           & -\frac{i}{2} & \frac{1}{2}
        \end{pmatrix} .
    \end{multline*}
    \begin{landscape}
        \[
            A^{1000000}
            =
            \begin{pmatrix}
                \frac{(1-i)^{1000000} + (1+i)^{1000000}}{2}                 & \frac{(1+i)^{1000000} \cdot i - (1-i)^{1000000} \cdot i}{2} & 500000 \cdot (1-i)^{999999} + 500000 \cdot (1+i)^{999999} & - (1-i)^{999999} \cdot 500000i + (1+i)^{999999} \cdot 500000i \\
                \frac{(1-i)^{1000000} \cdot i - (1+i)^{1000000} \cdot i}{2} & \frac{(1-i)^{1000000} + (1+i)^{1000000}}{2} & (1-i)^{999999} \cdot 500000i - (1+i)^{999999} \cdot 500000i & 500000 \cdot (1-i)^{999999} + 500000 \cdot (1+i)^{999999} \\
                0                                                           & 0                                                           & \frac{(1-i)^{1000000} + (1+i)^{1000000}}{2}                 & \frac{(1+i)^{1000000} \cdot i - (1-i)^{1000000} \cdot i}{2}   \\
                0                                                           & 0                                                           & \frac{(1-i)^{1000000} \cdot i - (1+i)^{1000000} \cdot i}{2} & \frac{(1-i)^{1000000} + (1+i)^{1000000}}{2}
            \end{pmatrix}.
        \]
    \end{landscape}

    \subsection*{Ответ:}
    \[
        A^N
        =
        \begin{pmatrix}
            -i & 0  & i & 0 \\
            1  & 0  & 1 & 0 \\
            0  & -i & 0 & i \\
            0  & 1  & 0 & 1
        \end{pmatrix}
        \begin{pmatrix}
        ( 1 - i )
            ^N & ( 1 - i )^{N-1} \cdot N & 0           & 0                   \\
            0  & ( 1 - i )^N             & 0           & 0                   \\
            0  & 0                       & ( 1 + i )^N & ( 1 + i )^N \cdot N \\
            0  & 0                       & 0           & ( 1 + i )^N
        \end{pmatrix}
        \begin{pmatrix}
            \frac{i}{2}   & \frac{1}{2} & 0            & 0           \\
            0             & 0           & \frac{i}{2}  & \frac{1}{2} \\
            - \frac{i}{2} & \frac{1}{2} & 0            & 0           \\
            0             & 0           & -\frac{i}{2} & \frac{1}{2}
        \end{pmatrix}
        .
    \]

\end{document}