% окружение Matlab
\newenvironment{Matlab}{\par \vspace{0.2cm}}{\vspace{0.2cm} \par}
% команда внутри окружения
\newcommand{\Mcommand}[1]{\noindent \texttt{> #1} \par}
% команда в Matlab
\newcommand{\matlab}[1]{\begin{Matlab} \Mcommand{#1} \end{Matlab}}

% определяемое понятие
\newcommand{\definition}[1]{\textit{#1}}

% создаем счетчик для определений и утверждений
\newcounter{clausecounter}[chapter]
% определяем формат вывода счетчика
\renewcommand{\theclausecounter}{\thechapter.\arabic{clausecounter}}

% утверждение
\newenvironment{statement}{\par \refstepcounter{clausecounter} \textbf{Утверждение \theclausecounter.} \itshape \par}{\par}
% доказательство
\newenvironment{proof}{\textbf{Доказательство.} \par}{\par $\blacksquare$}
% пример
\newenvironment{example}{\textbf{Пример.} \par}{\par $\blacksquare$}

% линейное комплексное пространство
\newcommand{\Cspace}[1]{\mathbb{C}^{#1}}

% действительная часть
\newcommand{\real}[1]{\mathtt{Re} \left [ #1 \right ]}
% мнимая часть
\newcommand{\image}[1]{\mathtt{Im} \left [ #1 \right ]}

% нормальное распределение
\newcommand{\normal}[2]{\mathcal{N} \left ( #1, #2 \right )}

% математическое ожидание
\newcommand{\expectation}[1]{\mathtt{M} \left [ #1 \right ]}
% дисперсия
\newcommand{\variance}[1]{\mathtt{D} \left [ #1 \right ]}
% ковариация
\newcommand{\covariance}[2]{\mathtt{cov} \left ( #1, #2 \right )}

% модуль
\newcommand{\modulus}[1]{\left | #1 \right |}
% норма
\newcommand{\norm}[1]{\left \| #1 \right \|}
% скалярное произведение
\newcommand{\scalarproduct}[2]{\left ( #1, #2 \right )}

% ранг
\newcommand{\rank}[1]{rank \left ( #1 \right )}
% спектр
\newcommand{\spectrum}[1]{spec \left ( #1 \right )}
% множество
\newcommand{\set}[1]{\left \{ #1 \right \}}
% линейная оболочка
\newcommand{\linear}[1]{\mathcal{L} \left ( #1 \right )}
% определитель
\newcommand{\determinant}[1]{\left | #1 \right |}

% производная
\newcommand{\derivative}[1]{\frac{d}{d #1}}

% след матрицы
\DeclareMathOperator{\tr}{tr}
% размерность пространства
\DeclareMathOperator{\dimension}{dim}