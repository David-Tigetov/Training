\chapter{Излучение}

\section{Диаграмма направленности}

Рассматривается система $n$ излучателей, местоположения которых в декартовой системе описывается векторами $r_k$.

\textcolor{red}{Рисунок.}

К излучателям подводятся сигналы с огибающими $a_k$, объединёнными в вектор $a$:
\[
    a = \begin{pmatrix}
        a_1    \\
        a_2    \\
        \vdots \\
        a_n
    \end{pmatrix}
    .
\]

Система излучателей формирует электромагнитное поле, которое распространяется в пространстве в радиальном направлении.

Электрическое поле характеризуется вектором электрической напряжённости $E$. Известно, что вектор напряженности $E$ совершает колебания в плоскости,
ортогональной направлению распространения электромагнитной волны. В общем случае вектор $E$ описывает эллипс. Сигналы с огибающими $a$, которые подаются
на входы излучателей, имеют одинаковую частоту $\omega$, вектор напряженности $E$ колеблется с такой же частотой $\omega$, поэтому колебания вектора
напряженности $E$ тоже будем описывать с помощью огибающей.

Описывать колебания вектора напряженности $E$ в декартовой системе неудобно, поскольку плоскость, в которой вектор $E$ совершает колебания, изменяется
при изменении рассматриваемой точки поля. Описывать колебания вектора напряжённости $E$ удобно в сферической системе координат, в которой единичный
вектор радиуса $u_\rho$ направлен в сторону распространения волны, а единичные векторы азимута $u_\varphi$ и угла места $u_\theta$ располагаются
в ортогональной плоскости. При изменении рассматриваемой точки, орты $u_\rho$, $u_\varphi$ и $u_\theta$ сами разворачиваются, так что $u_\rho$
всегда направлен радиально и совпадает с направлением распространения волны, а векторы $u_\varphi$ и $u_\theta$ всегда располагаются в плоскости,
ортогональной $u_\rho$, в которой колеблется вектор напряжённости $E$.

\textcolor{red}{Рисунок.}

Колебания вектора напряжённости $E$ будет описывать через огибающие колебаний двух его проекций: $E_\varphi$ --- на азимутальный орт $u_\varphi$
и $E_\theta$ --- на угломестный орт $u_\theta$ (радиальная компонента вектора напряжённости $E$ равна нулю).

Таким образом, вектор напряжённости $E$ имеет две угловые компоненты:
\[
    E =
    \begin{pmatrix}
        E_\varphi \\
        E_\theta
    \end{pmatrix}
    = E(a, r)
\]
которые зависят от входных сигналов излучателей $a$ и рассматриваемой точки с радиус-вектором $r$.

В дальней зоне, при достаточном удалении от излучателей, напряженность можно приближенно представить в виде:
\begin{equation}
    \label{emission:diagram:tension}
    E(a, r) \approx F(\varphi, \theta) \cdot a \cdot \frac{e^{i \scalarproduct{w}{r}}}{\modulus{r}},
\end{equation}
где $\varphi$ и $\theta$ --- азимут и угол места радиус-вектора $r$ в сферической системе (они задают угловое направление),
$\modulus{r}$ --- длина радиус вектора $r$, $w$ --- волновой вектор. Матрица $F(\varphi, \theta)$ называется диаграммой направленности
системы излучателей. Эта матрица является коэффициентом при $a$ в разложении функции напряжённости $E(a, r)$.

Множитель $e^{i \scalarproduct{w}{r}}$ показывает свдиг фазы, а множитель $\frac{1}{\modulus{r}}$ --- затухание амплитуды
колебаний при удалении от центра координат.

Волновой вектор $w$ --- это вектор, который в каждой точке направлен в сторону распространения волны и имеет величину, определяемую отношением
скорости изменения состояния к скорости распространения состояния. Распространение волны --- это процесс, при котором точки среды изменяют свое состояние
(скалярное либо векторное) и передают его другим точкам среды. Рассмотрим пример: точка совершает колебательные движения с частотой $\omega$
вдоль оси ординат, а текущую координату (состояние) сносит в направлении оси абсцисс с некоторой скоростью $v$ (Как если бы Вы левой рукой водили
ручкой вверх и вниз, а правой рукой тащили бы лист бумаги вправо).

\textcolor{red}{Рисунок}.

Волновой вектор $w$ направлен в сторону оси абсцисс, его величина:
\[
    w
    = \frac{\omega}{v}
    = \frac{\omega T}{v T}
    = \frac{2 \pi}{\lambda} ,
\]
где $T$ --- период колебаний точки, $\lambda$ --- длина волны.

Диаграмма направленности $F(\varphi, \theta)$ --- матрица размера $2 \times n$, которая имеет вид:
\[
    F(\varphi, \theta)
    = \begin{pmatrix}
        f_{1,\varphi}(\varphi, \theta) e^{-i \scalarproduct{w}{r_1}} & \dots & f_{k,\varphi}(\varphi, \theta) e^{-i \scalarproduct{w}{r_k}} & \dots & f_{n,\varphi}(\varphi, \theta) e^{-i \scalarproduct{w}{r_n}} \\
        f_{1,\theta}(\varphi, \theta) e^{-i \scalarproduct{w}{r_1}}  & \dots & f_{k,\theta}(\varphi, \theta) e^{-i \scalarproduct{w}{r_k}}  & \dots & f_{n,\theta}(\varphi, \theta) e^{-i \scalarproduct{w}{r_n}}  \\
    \end{pmatrix} ,
\]
вектор
\[
    f_k(\varphi, \theta)
    = \begin{pmatrix}
        f_{k,\varphi}(\varphi, \theta) \\
        f_{k,\theta}(\varphi, \theta)
    \end{pmatrix}
\]
называется парциальной диаграммой направленности $k$-го излучателя: $f_{k,\varphi}(\varphi, \theta)$ определяет вклад $k$-го излучателя в азимутальную
компоненту напряженности, а $f_{k,\varphi}(\varphi, \theta)$ --- в угломестную.


\section{Коэффициент усиления}

К системе излучателей подводятся сигналы с огибающими $a$, мощность которых $P_{inp}$:
\[
    P_{inp} = \norm{a}_2^2.
\]
Система излучателей преобразует сигналы в излучение, при этом излучение неоднородно в пространстве и мощность излучения может быть различной в
различных направлениях. При этом суммарная мощность излучения не превышает входную мощность $P_{inp}$. В точке, определяемой радиус-вектором,
плотность потока мощности $\Pi(a, r)$:
\[
    \Pi(a, r) = \frac{1}{Z_0} \norm{E(a, r)}_2^2 ,
\]
где $Z_0 = 120 \pi$ --- волновое сопротивление пространства.

\subsection{Вычисление}

Распределение мощности излучения характеризуется реализованным коэффициентом усиления $G(\varphi, \theta)$:
\[
    G(\varphi, \theta)
    = \frac{\Pi(a, r)}{\frac{P_{inp}}{4 \pi \modulus{r}^2}},
\]
где в числителе стоит плотность потока мощности $\Pi(a, r)$ в точке с радиус-вектором $r$, а в знаменателе --- среднее значение мощности,
усреднённое по поверхности сферы радиуса $\modulus{r}$. Преобразуя выражение для реализованного коэффициента усиления, получим:
\begin{multline}
    G(\varphi, \theta)
    = \frac{4 \pi \modulus{r}^2}{P_{inp}} \cdot \Pi(a, r)
    = \frac{4 \pi \modulus{r}^2}{P_{inp}} \cdot \frac{1}{Z_0} \norm{E(a, r))}_2^2
    \approx \frac{4 \pi \modulus{r}^2}{P_{inp}} \cdot \frac{1}{Z_0} \norm{F(\varphi, \theta) a \cdot \frac{e^{i \scalarproduct{w}{r}}}{\modulus{r}}}_2^2 = \\
    %
    = \frac{4 \pi \modulus{r}^2}{P_{inp}} \cdot \frac{1}{Z_0} \norm{F(\varphi, \theta) a}_2^2 \frac{\modulus{e^{i \scalarproduct{w}{r}}}^2}{\modulus{r}^2}
    = \frac{4 \pi \modulus{r}^2}{P_{inp}} \cdot \frac{1}{Z_0} \norm{F(\varphi, \theta) a}_2^2 \frac{1}{\modulus{r}^2}
    = \frac{4 \pi}{Z_0} \cdot \frac{\norm{F(\varphi, \theta) a}^2}{P_{inp}} .
\end{multline}

Если направление $(\varphi, \theta)$ фиксировано, то коэффициент усиления пропорционален отношению:
\begin{equation}
    \label{emission:emitter:gain:rayleigh}
    \rho(a)
    = \frac{\norm{F(\varphi, \theta) a}_2^2}{P_{inp}}
    = \frac{\norm{F(\varphi, \theta) a}_2^2}{\norm{a}_2^2}
    = \frac{a^* F^*(\varphi, \theta) F(\varphi, \theta) a}{a^* a}
\end{equation}
и возникает вопрос, каким образом нужно сформировать огибающие входных сигналов $a$ чтобы отношение $\rho(a)$ и коэффициент усиления в заданном
направлении $(\varphi, \theta)$ оказался наибольшим?

Отношение $\rho(a)$ является отношением Релея: числитель --- квадратичная форма с эрмитовой матрией $F^* F$, знаменатель --- квадрат нормы $a$.
Наибольшее значение $G_{max}$:
\[
    G_{max} = \max \limits_{a} G
\]
достигается в направлении $a_{max}$, совпадающим с направлением собственных векторов, соответствующих наибольшему собственному числу $\lambda_{max}$
матрицы $F^*F$. Вектор $a_{max}$ и число $\lambda_{max}$ удовлетворяют уравнению:
\begin{equation}
    \label{emission:emitter:gain:optimal_input}
    F^* F a_{max} = \lambda_{max} a_{max}
\end{equation}

Первый способ нахождения $a_{max}$ заключается в решении системы~\eqref{emission:emitter:gain:optimal_input}. В некоторых случаях матрица
$F^*F$ оказывается сложной для определения собственных чисел и векторов, поскольку порядок матрицы равен количеству излучателей $n$.

Второй способ определения $a_{max}$ связан с нахождение вектора оптимальной поляризации $p_{max}$. Домножим левую и правую части
уравнения~\eqref{emission:emitter:gain:optimal_input} на $F$ слева:
\[
    F F^* F a_{max} = \lambda_{max} F a_{max} .
\]
В левой и правой частях получился вектор $F a_{max}$, который обозначим $p_{pmax}$:
\[
    p_{max} = F a_{max} .
\]
Таким образом,
\[
    F F^* p_{max} = \lambda_{max} p_{max} .
\]
и вектор $p_{max}$ является собственным вектором матрицы $F F^*$, соответствующий тому же собственному числу $\lambda_{max}$.

Пусть $f_\varphi$ и $f_\theta$ --- строки матрицы $F$, тогда:
\[
    F F^*
    = \begin{pmatrix}
        f_\varphi \\
        f_\theta
    \end{pmatrix}
    \begin{pmatrix}
        f_\varphi^* & f_\theta^*
    \end{pmatrix}
    = \begin{pmatrix}
        f_\varphi f_\varphi^*  & f_\varphi f_\theta^*
        f_{\theta} f_\varphi^* & f_\theta f_\theta^*  \\
    \end{pmatrix} .
\]
У матрицы $F F^*$ два собственных значения $\lambda_{min}$ и $\lambda_{max}$, которые являются корнями характеристического уравнения:
\begin{multline*}
    \begin{vmatrix}
        f_\theta f_\theta^* - \lambda & f_{\theta} f_\varphi^*          \\
        f_\varphi f_\theta^*          & f_\varphi f_\varphi^* - \lambda
    \end{vmatrix}
    = (f_\theta f_\theta^* - \lambda) (f_\varphi f_\varphi^* - \lambda) - f_\varphi f_\theta^* f_{\theta} f_\varphi^* = \\
    %
    = \lambda^2 - ( f_\theta f_\theta^* + f_\varphi f_\varphi^* ) \lambda + f_\theta f_\theta^* f_\varphi f_\varphi^* - f_\varphi f_\theta^* f_{\theta} f_\varphi^* = \\
    %
    = \lambda^2 - \tr(F F^*) \lambda + \det(F F^*) ,
\end{multline*}
где $\tr(F F^*)$ и $\det(F F^*)$ --- след и определитель матрицы $F F^*$. Корни характеристического уравнения:
\begin{align}
    \lambda_{min} & = \frac{\tr(F F^*) - \sqrt{\tr^2(F F^*) - 4 \det(F F^*)}}{2} \label{emission:emitter:gain:minimum_eigenvalue} , \\
    \lambda_{max} & = \frac{\tr(F F^*) + \sqrt{\tr^2(F F^*) - 4 \det(F F^*)}}{2} \label{emission:emitter:gain:maximum_eigenvalue}.
\end{align}
Вектор $p_{max}$ находим как решение однородной системы:
\begin{gather*}
    F F^* p_{max} = \lambda_{max} p_{max} , \\
    ( F F^* - \lambda I ) p_{max} = 0 .
\end{gather*}
Из равенства \eqref{emission:emitter:gain:optimal_input}:
\begin{align*}
    F^* F a_{max}                       & = \lambda_{max} a_{max} , \\
    F^* p_{max}                         & = \lambda_{max} a_{max} , \\
    \frac{1}{\lambda_{max}} F^* p_{max} & = a_{max} .
\end{align*}

\subsection{Пример 1: один канал}

Излучение только одного канала.

Пусть выбрано и зафиксировано направление $(\varphi, \theta)$, и в этом направлении диаграмма направленности $F(\varphi, \theta)$ имеет вид:
\begin{gather*}
    F(\varphi, \theta)
    = \begin{pmatrix}
        A_1 e^{i \alpha_1} & 0 \\
        A_2 e^{i \alpha_2} & 0
    \end{pmatrix}
\end{gather*}
где амплитуды $A_1 \in \mathbb{R}$, $A_2 \in \mathbb{R}$ и смещения фаз $\alpha_1 \in \mathbb{R}$, $\alpha_2 \in \mathbb{R}$.

Необходимо найти оптимальный вектор комплексных огибающих $a_{max}$ сигналов на входах излучателей, при котором получается наибольший коэффициент
усиления~\eqref{emission:emitter:gain:rayleigh}:
\[
    \rho(a)
    = \frac{a^*F^*(\varphi, \theta) F(\varphi, \theta) a}{a^*a}.
\]
Решаем задачу первым способом, матрица системы~\eqref{emission:emitter:gain:optimal_input}:
\[
    F^*(\varphi, \theta) F(\varphi, \theta)
    =
    \begin{pmatrix}
        A_1 e^{-i \alpha_1} & A_2 e^{-i \alpha_2} \\
        0                   & 0
    \end{pmatrix}
    \begin{pmatrix}
        A_1 e^{i \alpha_1} & 0 \\
        A_2 e^{i \alpha_2} & 0
    \end{pmatrix}
    =
    \begin{pmatrix}
        A_1^2 + A_2^2 & 0 \\
        0             & 0
    \end{pmatrix}
\]
Отсюда легко получаются собственные числа:
\begin{gather*}
    \determinant{F^*(\varphi, \theta) F(\varphi, \theta) - \lambda I} = 0 , \\
    %
    \begin{vmatrix}
        A_1^2 + A_2^2 - \lambda & 0         \\
        0                       & - \lambda
    \end{vmatrix}
    = 0 , \\
    %
    \lambda_{max} = A_1^2 + A_2^2 , \\
    \lambda_{min} = 0 .
\end{gather*}
Наибольшее значение отношения Релея:
\[
    \rho_{max} = \lambda_{max} = A_1^2 + A_2^2 .
\]
Оптимальный вектор комплесных огибающих сигналов на входах $a_{max}$:
\begin{gather*}
    ( F^*(\varphi, \theta) F(\varphi, \theta) - \lambda_{max} I ) a_{max} = 0 , \\
    %
    \begin{pmatrix}
        A_1^2 + A_2^2 - (A_1^2 + A_2^2) & 0                 \\
        0                               & - (A_1^2 + A_2^2)
    \end{pmatrix}
    a_{max} = 0 , \\
    %
    \begin{pmatrix}
        0 & 0                 \\
        0 & - (A_1^2 + A_2^2)
    \end{pmatrix}
    a_{max} = 0 , \\
    %
    a_{max}
    = \begin{pmatrix}
        1 \\
        0
    \end{pmatrix} .
\end{gather*}
Оптимальная поляризация $p_{max}$:
\[
    p_{max}
    = F(\varphi, \theta) a_{max}
    = \begin{pmatrix}
        A_1 e^{i \alpha_1} & 0 \\
        A_2 e^{i \alpha_2} & 0
    \end{pmatrix}
    \begin{pmatrix}
        1 \\
        0
    \end{pmatrix}
    = \begin{pmatrix}
        A_1 e^{i \alpha_1} \\
        A_2 e^{i \alpha_2}
    \end{pmatrix} .
\]

\subsection{Пример 2: разная поляризация}

Излучение в разной фазе в двух ортогональных плоскостях по углу места и по азимуту.

Пусть в выбранном направлении $(\varphi, \theta)$ диаграмма направленности имеет вид:
\begin{gather*}
    F(\varphi, \theta) = \begin{pmatrix}
        A_1 e^{i \alpha_1} & 0                  \\
        0                  & A_2 e^{i \alpha_2}
    \end{pmatrix}
\end{gather*}
где амплитуды $A_1 \in \mathbb{R}$, $A_2 \in \mathbb{R}$, смещения фаз $\alpha_1 \in \mathbb{R}$, $\alpha_2 \in \mathbb{R}$ и для определённости
\[
    A_1 > A_2 .
\]

Необходимо найти оптимальный вектор комплексных огибающих $a_{max}$ сигналов на входах излучателей, при котором получается наибольший коэффициент
усиления~\eqref{emission:emitter:gain:rayleigh}:
\[
    \rho(a)
    = \frac{a^* F^*(\varphi, \theta) F(\varphi, \theta) a}{a^* a}.
\]
Решаем задачу первым способом, матрица системы~\eqref{emission:emitter:gain:optimal_input}:
\[
    F^*(\varphi, \theta) F(\varphi, \theta)
    =
    \begin{pmatrix}
        A_1 e^{-i \alpha_1} & 0                   \\
        0                   & A_2 e^{-i \alpha_2}
    \end{pmatrix}
    \begin{pmatrix}
        A_1 e^{i \alpha_1} & 0                  \\
        0                  & A_2 e^{i \alpha_2}
    \end{pmatrix}
    =
    \begin{pmatrix}
        A_1^2 & 0     \\
        0     & A_2^2
    \end{pmatrix}
\]
Собственные числа:
\begin{gather*}
    \determinant{F^*(\varphi, \theta) F(\varphi, \theta) - \lambda I} = 0 , \\
    %
    \begin{vmatrix}
        A_1^2 - \lambda & 0               \\
        0               & A_2^2 - \lambda
    \end{vmatrix}
    = 0 , \\
    %
    \lambda_{max} = A_1^2 , \\
    \lambda_{min} = A_2^2 .
\end{gather*}
Наибольшее значение отношения Релея:
\[
    \rho_{max} = \lambda_{max} = A_1^2 .
\]
Оптимальный вектор комплексных огибающих сигналов на входах излучателя:
\begin{gather*}
    ( F^*(\varphi, \theta) F(\varphi, \theta) - \lambda_{max} I ) a_{max} = 0 , \\
    %
    \begin{pmatrix}
        A_1^2 - A_1^2 & 0             \\
        0             & A_2^2 - A_1^2
    \end{pmatrix}
    a_{max} = 0 , \\
    %
    \begin{pmatrix}
        0 & 0             \\
        0 & A_2^2 - A_1^2
    \end{pmatrix}
    a_{max} = 0 , \\
    %
    a_{max}
    = \begin{pmatrix}
        1 \\
        0
    \end{pmatrix} .
\end{gather*}
Оптимальная поляризация $p_{max}$:
\[
    p_{max}
    = F(\varphi, \theta) a_{max}
    = \begin{pmatrix}
        A_1 e^{i \alpha_1} & 0                  \\
        0                  & A_2 e^{i \alpha_2}
    \end{pmatrix}
    \begin{pmatrix}
        1 \\
        0
    \end{pmatrix}
    = \begin{pmatrix}
        A_1 e^{i \alpha_1} \\
        0
    \end{pmatrix} .
\]

\subsection{Пример 3: одинаковая поляризация}

Два канала производят колебания в разной фазе, но в одном направлении --- азимутальном.

Пусть в выбранном направлении $(\varphi, \theta)$ диаграмма направленности имеет вид:
Пусть парциальные диаграммы направленности каналов излучателя имеют вид:
\[
    F(\varphi, \theta)
    =
    \begin{pmatrix}
        A_1 e^{i \alpha_1} & A_2 e^{i \alpha_2} \\
        0                  & 0
    \end{pmatrix}
\]
где амплитуды $A_1 \in \mathbb{R}$, $A_2 \in \mathbb{R}$, смещения фаз $\alpha_1 \in \mathbb{R}$, $\alpha_2 \in \mathbb{R}$.

В первом способе матрица системы~\eqref{emission:emitter:gain:optimal_input}:
\[
    F^*(\varphi, \theta) F(\varphi, \theta)
    =
    \begin{pmatrix}
        A_1 e^{-i \alpha_1} & 0 \\
        A_2 e^{-i \alpha_2} & 0
    \end{pmatrix}
    \begin{pmatrix}
        A_1 e^{i \alpha_1} & A_2 e^{i \alpha_2} \\
        0                  & 0
    \end{pmatrix}
    =
    \begin{pmatrix}
        A_1^2                            & A_1 e^{-i \alpha_1 + i \alpha_2} \\
        A_2 e^{-i \alpha_2 + i \alpha_1} & A_2^2
    \end{pmatrix}
\]
получается "сложной"{}, поэтому рассмотрим матрицу системы для оптимальной поляризации:
\[
    F(\varphi, \theta) F^*(\varphi, \theta)
    =
    \begin{pmatrix}
        A_1 e^{i \alpha_1} & A_2 e^{i \alpha_2} \\
        0                  & 0
    \end{pmatrix}
    \begin{pmatrix}
        A_1 e^{-i \alpha_1} & 0 \\
        A_2 e^{-i \alpha_2} & 0
    \end{pmatrix}
    =
    \begin{pmatrix}
        A_1^2 + A_2^2 & 0 \\
        0             & 0
    \end{pmatrix} .
\]
Эта матрица "простая"{}, и её собственные числа
\begin{gather*}
    \determinant{F(\varphi, \theta)F^*(\varphi, \theta) - \lambda I} = 0 , \\
    %
    \begin{vmatrix}
        A_1^2 + A_2^2 - \lambda & 0         \\
        0                       & - \lambda
    \end{vmatrix} = 0, \\
    %
    \lambda_{max} = A_1^2 + A_2^2 , \\
    \lambda_{min} = 0 .
\end{gather*}

Наибольшее значение отношения Релея:
\[
    \rho_{max} = \lambda_{max} = A_1^2 + A_2^2.
\]

Наибольшему собственному числу $\lambda_{max}$ соответствует оптимальный вектор поляризации $p_{max}$:
\begin{gather*}
    (F(\varphi, \theta)F^*(\varphi, \theta) - \lambda_{max} I) p_{max} = 0 , \\
    %
    \begin{pmatrix}
        A_1^2 + A_2^2 - \lambda_{max} & 0               \\
        0                             & - \lambda_{max}
    \end{pmatrix}
    p_{max} = 0 , \\
    %
    \begin{pmatrix}
        0 & 0                 \\
        0 & - (A_1^2 + A_2^2)
    \end{pmatrix}
    p_{max} = 0 , \\
    %
    p_{max} = \begin{pmatrix}
        1 \\
        0
    \end{pmatrix} .
\end{gather*}
Оптимальный вектор огибающих сигналов на входах излучателя:
\[
    a_{max}
    = F^*(\varphi, \theta) p_{max}
    = \begin{pmatrix}
        A_1 e^{- i \alpha_1} & 0 \\
        A_2 e^{- i \alpha_2} & 0
    \end{pmatrix}
    \begin{pmatrix}
        1 \\
        0
    \end{pmatrix}
    = \begin{pmatrix}
        A_1 e^{- i \alpha_1} \\
        A_2 e^{- i \alpha_2}
    \end{pmatrix} .
\]
Компоненты вектора $a_{max}$ указывают на необходимость обратных смещений фаз входных сигналов двух каналов излучателя, с тем чтобы колебания излучения складывались в одной фазе.

\subsection{Пример 4: общий случай}

Пусть в выбранном направлении $(\varphi, \theta)$ диаграмма направленности имеет вид:
\[
    F(\varphi, \theta)
    =
    \begin{pmatrix}
        0.1 e^{i \frac{\pi}{6}} & 0.3 e^{i \frac{5 \pi}{4}}  \\
        0.2 e^{i \frac{\pi}{3}} & 0.1 e^{- i \frac{\pi}{10}}
    \end{pmatrix} .
\]
Вычисления оптимальных векторов огибающих и поляризации смотри в файле Matlab \texttt{emission/two/gain.m}.

\subsection{Пример 5: крестовой излучатель}

В декартовой системе координат $X$, $Y$, $Z$ первый излучатель находится в начале координат и имеет парциальную диаграмму направленности:
\[
    f_1(\varphi, \theta) =
    \begin{pmatrix}
        \cos \varphi \cos \theta e^{\frac{\pi}{3}} \\
        \cos \theta
    \end{pmatrix} ,
\]
а в декартовых координатах:
\[
    f_{1,c}(\varphi, \theta) = \cos \varphi \cos \theta e^{\frac{\pi}{3}} \cdot u_{\varphi} + \cos \theta \cdot u_{\theta},
\]
где $u_\varphi$, $u_\theta$ --- образы орт координатных прямых углов $\varphi$ и $\theta$. В системе $C = (X, Y, Z)$ декартовы координаты точки связаны
со сферическими координатами системы $S = (\varphi, \theta, \rho)$ равенствами:
\begin{gather*}
    x = \rho \cos \theta \cos \varphi, \\
    y = \rho \cos \theta \sin \varphi, \\
    z = \rho \sin \theta.
\end{gather*}
Частные производные дают направления орт координатных прямых углов:
\begin{gather*}
    U_{\varphi} =
    \begin{pmatrix}
        x_{\varphi}^{\prime} \\
        y_{\varphi}^{\prime} \\
        z_{\varphi}^{\prime} \\
    \end{pmatrix}
    =
    \begin{pmatrix}
        - \rho \cos \theta \sin \varphi \\
        \rho \cos \theta \cos \varphi   \\
        0
    \end{pmatrix} , \\
    %
    U_{\theta} =
    \begin{pmatrix}
        x_{\theta}^{\prime} \\
        y_{\theta}^{\prime} \\
        z_{\theta}^{\prime} \\
    \end{pmatrix}
    =
    \begin{pmatrix}
        - \rho \sin \theta \cos \varphi \\
        - \rho \sin \theta \sin \varphi \\
        \rho \cos \theta
    \end{pmatrix} .
\end{gather*}
После нормировки получим:
\begin{gather*}
    u_{\varphi}
    = \frac{U_{\varphi}}{\norm{U_{\varphi}}}
    = \begin{pmatrix}
        - \sin \varphi \\
        \cos \varphi   \\
        0
    \end{pmatrix} , \\
    %
    u_{\theta}
    = \frac{U_\theta}{\norm{U_\theta}}
    = \begin{pmatrix}
        - \sin \theta \cos \varphi \\
        - \sin \theta \sin \varphi \\
        \cos \theta
    \end{pmatrix} .
\end{gather*}

Со вторым каналом свяжем декартову систему $\widetilde{C} = (\widetilde{X}$, $\widetilde{Y}$, $\widetilde{Z})$ и соответствующую сферическую систему
$\widetilde{S} = (\widetilde{\varphi}, \widetilde{\theta}, \widetilde{\rho})$. В этих системах парциальная диаграмма направленности второго канала:
\[
    \widetilde{f}_2(\widetilde{\varphi}, \widetilde{\theta}) =
    \begin{pmatrix}
        \cos \widetilde{\theta} e^{-i\frac{\pi}{4}} \\
        \cos \widetilde{\theta}
    \end{pmatrix}
\]
или
\[
    \widetilde{f}_{2,c}(\widetilde{\varphi}, \widetilde{\theta})
    = \cos \widetilde{\theta} e^{-\frac{\pi}{4}} \cdot \widetilde{u}_{\widetilde{\varphi}} + \cos \widetilde{\theta} \cdot \widetilde{u}_{\widetilde{\theta}} ,
\]
где
\begin{gather*}
    \widetilde{u}_{\widetilde{\varphi}}
    = \begin{pmatrix}
        - \sin \widetilde{\varphi} \\
        \cos \widetilde{\varphi}   \\
        0
    \end{pmatrix} , \\
    %
    \widetilde{u}_{\widetilde{\theta}}
    = \begin{pmatrix}
        - \sin \widetilde{\theta} \cos \widetilde{\varphi} \\
        - \sin \widetilde{\theta} \sin \widetilde{\varphi} \\
        \cos \widetilde{\theta}
    \end{pmatrix}.
\end{gather*}
Координаты $\widetilde{x}$, $\widetilde{y}$ и $\widetilde{z}$ системы $\widetilde{C}$ связаны с координатами $x$, $y$, $z$ системы $C$ равенствами:
\begin{gather*}
    x = \widetilde{x} , \\
    y = - \widetilde{z} , \\
    z = \widetilde{y} .
\end{gather*}
Векторы $\widetilde{u}_{\widetilde{\varphi}}$ и $\widetilde{u}_{\widetilde{\theta}}$ в системе $C$ будут иметь координаты:
\begin{gather*}
    u_{\widetilde{\varphi}}
    = \begin{pmatrix}
        - \sin \widetilde{\varphi} \\
        0                          \\
        \cos \widetilde{\varphi}
    \end{pmatrix} , \\
    %
    u_{\widetilde{\theta}}
    = \begin{pmatrix}
        - \sin \widetilde{\theta} \cos \widetilde{\varphi} \\
        - \cos \widetilde{\theta}                          \\
        - \sin \widetilde{\theta} \sin \widetilde{\varphi}
    \end{pmatrix}.
\end{gather*}
И диаграмма направленности второго канала в декартовых координатах системы $C$:
\begin{multline*}
    f_{2,c}(\widetilde{\varphi}, \widetilde{\theta})
    = \cos \widetilde{\theta} e^{-i\frac{\pi}{4}} \cdot \widetilde{u}_{\widetilde{\varphi}} + \cos \widetilde{\theta} \cdot u_{\widetilde{\theta}} = \\
    %
    = \cos \widetilde{\theta}  e^{-i\frac{\pi}{4}}
    \begin{pmatrix}
        - \sin \widetilde{\varphi} \\
        0                          \\
        \cos \widetilde{\varphi}
    \end{pmatrix}
    + \cos \widetilde{\theta}
    \begin{pmatrix}
        - \sin \widetilde{\theta} \cos \widetilde{\varphi} \\
        - \cos \widetilde{\theta}                          \\
        - \sin \widetilde{\theta} \sin \widetilde{\varphi}
    \end{pmatrix} = \\
    %
    = e^{-i\frac{\pi}{4}}
    \begin{pmatrix}
        - \cos \widetilde{\theta}  \sin \widetilde{\varphi} \\
        0                                                   \\
        \cos \widetilde{\theta} \cos \widetilde{\varphi}
    \end{pmatrix}
    + \begin{pmatrix}
        - \cos \widetilde{\theta} \sin \widetilde{\theta} \cos \widetilde{\varphi} \\
        - \cos^2 \widetilde{\theta}                                                \\
        - \cos \widetilde{\theta} \sin \widetilde{\theta} \sin \widetilde{\varphi}
    \end{pmatrix} .
\end{multline*}
Теперь нужно заменить углы $\widetilde{\varphi}$ и $\widetilde{\theta}$ углами $\varphi$ и $\theta$.

Из соответствия координат систем $C$ и $\widetilde{C}$ получим равенства:
\begin{align*}
    \rho \cos \theta \cos \varphi & = \widetilde{\rho} \cos \widetilde{\theta} \cos \widetilde{\varphi} , \\
    \rho \cos \theta \sin \varphi & = - \widetilde{\rho} \sin \widetilde{\theta} ,                        \\
    \rho \sin \theta              & = \widetilde{\rho} \cos \widetilde{\theta} \sin \widetilde{\varphi} .
\end{align*}
Учитывая что $\rho = \widetilde{\rho}$, получим равенства для углов:
\begin{align*}
    \cos \theta \cos \varphi & = \cos \widetilde{\theta} \cos \widetilde{\varphi} , \\
    \cos \theta \sin \varphi & = - \sin \widetilde{\theta} ,                        \\
    \sin \theta              & = \cos \widetilde{\theta} \sin \widetilde{\varphi} .
\end{align*}
Откуда из перемножения левых и правых частей первого и второго равенств:
\[
    \cos^2 \theta \cos \varphi \sin \varphi = - \cos \widetilde{\theta} \sin \widetilde{\theta} \cos \widetilde{\varphi} ,
\]
из перемножения левых и правых частей второго и третьего равенств:
\[
    \cos \theta \sin \theta \sin \varphi = - \cos \widetilde{\theta} \sin \widetilde{\theta} \sin \widetilde{\varphi}, \\
\]
из возведения в квадрат левой и правой частей второго равенства:
\begin{gather*}
    \cos^2 \theta \sin^2 \varphi = \sin^2 \widetilde{\theta} , \\
    1 - \cos^2 \theta \sin^2 \varphi = 1 - \sin^2 \widetilde{\theta} , \\
    1 - \cos^2 \theta \sin^2 \varphi = \cos^2 \widetilde{\theta} .
\end{gather*}
Таким образом, диаграмма направленности второго канала в декартовых координатах:
\[
    f_{2,c}(\varphi, \theta)
    = e^{-i\frac{\pi}{4}}
    \begin{pmatrix}
        - \sin \theta \\
        0             \\
        \cos \theta \cos \varphi
    \end{pmatrix}
    + \begin{pmatrix}
        \cos^2 \theta \cos \varphi \sin \varphi \\
        \cos^2 \theta \sin^2 \varphi - 1        \\
        \cos \theta \sin \theta \sin \varphi
    \end{pmatrix}
\]
и для получения парциальной диаграммы направленности второго канала нужно вычислить проекции (скалярные произведения) на орты $u_\theta$ и $u_\varphi$:
\[
    f_2(\varphi, \theta)
    = \begin{pmatrix}
        \scalarproduct{f_{2,c}(\varphi, \theta)}{u_\varphi} \\
        \scalarproduct{f_{2,c}(\varphi, \theta)}{u_\theta}
    \end{pmatrix} .
\]

Построение диаграмм в файле \texttt{emission/cross/arrows.m}.

Вычисление коэффициента усиления в файле \texttt{emission/cross/gain.m}.

\section{Коэффициент полезного действия}

\subsection{Энергетическое ограничение}

Сигналы, поступающие на входы излучателей, частично отражаются обратно и частично переходят в излучение. Часть излучения, которое формирует излучатель,
затекает обратно через другие излучатели, в результате в каналах наводятся дополнительные отражённые сигналы. В итоге, мощность входных сигналов
$P_{inp}$ частично переходит в мощность излучения $P_{rad}$, частично в мощность отражённых сигналов во входных каналах излучателей $P_{ref}$ и
частично растрачивается на диссипативные потери мощностью $P_{dis}$:
\[
    P_{rad} + P_{ref} + P_{dis} = P_{inp}.
\]

Диссипативные потери сложно проанализировать, поэтому просто исключим их мощность $P_{dis} >0$ и получим неравенство:
\begin{equation}
    \label{emission:efficiency:power_inequality}
    P_{ref} + P_{rad} \le P_{inp} .
\end{equation}

Перенос мощности отраженных сигналов $P_{ref}$ в правую часть даёт неравенства:
\begin{gather*}
    P_{rad} \le P_{inp} - P_{ref}, \\
    P_{rad} \le P_{inp} - P_{ref} \le P_{inp}, \\
    \frac{P_{rad}}{P_{inp}} \le 1 - \frac{P_{ref}}{P_{inp}} \le 1 .
\end{gather*}

\subsection{Коэффициент полезного действия}

Коэффициентом полезного действия (КПД) называется отношение мощности излучения ко входной мощности:
\[
    \eta = \frac{P_{rad}}{P_{inp}}.
\]
Легко видеть, что
\[
    \eta \le 1 - \frac{P_{ref}}{P_{inp}} = \eta_s ,
\]
и величина $\eta_s$ является верхней границей КПД $\eta$.

Входная мощность $P_{inp}$ связана с вектором огибающих входных сигналов $a$, поступающих на входы излучателей:
\[
    P_{inp} = \norm{a}_2^2 .
\]

Пусть $b$ обозначает вектор огибающих отражённых сигналов, будем считать, что огибающие отражённых сигналов линейно связаны с огибающими входных сигналов.
Для излучателя с номером $k$:
\[
    b_k = s_{k1} a_1 + \dots + s_{kk} a_k + \dots + s_{k,n} a_n.
\]
Пусть $S$ --- матрица коэффициентов $s_{ij}$, тогда:
\[
    b = S a ,
\]
где $S$ называется матрицей рассеяния.

Мощность отражённых сигналов:
\[
    \norm{b}_2^2
    = b^* b
    = \left ( S a \right )^* S a
    = a^* S^* S a .
\]

Таким образом, граница КПД $\eta_s$:
\[
    \eta_s(a)
    = 1 - \frac{P_{ref}}{P_{inp}}
    = 1 - \frac{a S^* S a}{a^* a}
\]
и ограничение для КПД можно получить из анализа матрицы рассеяния $S$.

Само значение КПД определяется мощностью излучения $P_{rad}$, которое представляет собой интеграл по сфере $S_r$ радиуса $r$ (позже выяснится,
что величина $r$ значение не имеет) от мощности электрического поля:
\[
    P_{rad}
    = \iint \limits_{S_r} \norm{E(a, r)}_2^2 ds ,
\]
Подставляя выражение \eqref{emission:diagram:tension} для вектора напряжённости $E(a, r)$, получим:
\begin{multline*}
    P_{rad}
    = \iint \limits_{S_r} \norm{F(\varphi, \theta) a \cdot \frac{e^{i \scalarproduct{w}{r}}}{r}}^2 ds
    = \frac{1}{r^2} \iint \limits_{S_r} \norm{F(\varphi, \theta) a}^2 \modulus{\frac{e^{i \scalarproduct{w}{r}}}{r}}^2 ds = \\
    %
    = \frac{1}{r^2} \iint \limits_{S_r} \norm{F(\varphi, \theta) a}^2 \frac{\modulus{e^{i \scalarproduct{w}{r}}}^2}{r^2} ds
    = \frac{1}{r^2} \iint \limits_{S_r} \norm{F(\varphi, \theta) a}^2 ds = \\
    %
    = \frac{1}{r^2} \iint \limits_{S_r} a^* F^*(\varphi, \theta) F(\varphi, \theta) a ds ,
\end{multline*}
где квадратичная форма:
\[
    a^* F^*(\varphi, \theta) F(\varphi, \theta) a
    = \begin{pmatrix}
        a_1^* & \dots & a_n^*
    \end{pmatrix}
    \begin{pmatrix}
        f_1^* f_1 & \dots  & f_1^* f_k & \dots  & f_1^* f_n \\
        \vdots    & \vdots & \vdots    & \vdots & \vdots    \\
        f_k^* f_1 & \dots  & f_k^* f_k & \dots  & f_k^* f_n \\
        \vdots    & \vdots & \vdots    & \vdots & \vdots    \\
        f_n^* f_1 & \dots  & f_n^* f_k & \dots  & f_n^* f_n
    \end{pmatrix}
    \begin{pmatrix}
        a_1    \\
        \vdots \\
        a_n
    \end{pmatrix}
    ,
\]
где $f_k$ --- столбцы диаграммы направленности $F(\varphi, \theta)$. Интеграл от квадратичной формы равен сумме интегралов, которую также можно
представить квадратичной формой, поскольку вектор $a$ является постоянным:
\[
    P_{rad}
    = a^* Q a,
\]
где матрица $Q$ образована элементами
\[
    Q_{ij}
    = \frac{1}{r^2} \iint \limits_{S_r} f_i^*(\varphi, \theta) f_j(\varphi, \theta) ds
\]

Таким образом, КПД $\eta$:
\[
    \eta(a) = \frac{a^* Q a}{a^* a}
\]
является отношением Релея, поэтому его значения ограничены наименьшим $\eta_{min}$ и наибольшим $\eta_{max}$ собственными значениями матрицы $Q$:
\[
    \eta_{min} \le \eta(a) \le \eta_{max} .
\]

\subsection{КПД каналов}

Представим, что вся мощность входных сигналов направлена в излучатель с номером $k$, то есть вектор огибающих
\[
    a
    = \begin{pmatrix}
        0      \\
        \vdots \\
        1      \\
        \vdots \\
        0
    \end{pmatrix} ,
\]
тогда величина КПД излучателя с номером $k$:
\[
    \eta_k
    = \frac{a^* Q a}{a^* a}
    = \frac{Q_{kk}}{1}
    = Q_{kk} .
\]
Таким образом, диагональные элемент матрицы $Q$ показывают КПД излучателей. Внедиагональные элементы представляют собой коэффициенты пересечения диаграмм.

Оказывается, что КПД каналов $\eta_1$ и $\eta_2$ связаны с коэффициентом пересечения диаграмм. Представим элементы матрицы $Q$ в нормированном виде:
\[
    Q_{ij}
    =
    \sqrt{Q_{ii}}
    \cdot
    \frac{\iint \limits_{S_r} f_i^*(\varphi, \theta) f_j(\varphi, \theta) ds}{\sqrt{Q_{ii}} \sqrt{Q_{jj}}}
    \cdot
    \sqrt{Q_{jj}} ,
\]
тогда
\[
    Q = \sqrt{D} R \sqrt{D} ,
\]
где
\[
    \sqrt{D}
    = \begin{pmatrix}
        \sqrt{Q_{11}} & 0             & \dots & 0             \\
        0             & \sqrt{Q_{22}} & \dots & 0             \\
        0             & 0             & \dots & \sqrt{Q_{nn}} \\
    \end{pmatrix} ,
    \;
    %
    R_{ij} = \frac{Q_{ij}}{\sqrt{Q_{ii}} \sqrt{Q_{jj}}} .
\]

Согласно неравенству~\eqref{emission:efficiency:power_inequality}:
\begin{align*}
    P_{rad}                   & \le P_{inp} , \\
    a^* Q a                   & \le a^* a,    \\
    a^* \sqrt{D} R \sqrt{D} a & \le a^* a .
\end{align*}
Пусть $x = \sqrt{D} a$, тогда:
\begin{align*}
    x^* R x & \le x^* (\sqrt{D}^{-1})^* (\sqrt{D}^{-1}) x , \\
    x^* R x & \le x^* D^{-1} x .
\end{align*}
Пусть $r_{max}$ --- наибольшее собственное значение матрицы $R$ и $x_{max}$ --- соответствующий этому числу собственный вектор, а $Q_{min} = \min \{ Q_{11}, Q_{22} \}$,
тогда:
\begin{align*}
    x_{max}^* R x_{max}        & \le x_{max}^* D^{-1} x_{max} ,                             \\
    x_{max}^* r_{max} x_{max}  & \le \sum_{k=1}^n \frac{1}{Q_{kk}} x_{max,k}^* x_{max,k} ,  \\
    r_{max} \norm{x_{max}}_2^2 & \le \sum_{k=1}^n \frac{1}{Q_{kk}} \modulus{x_{max,k}}^2 ,  \\
    r_{max} \norm{x_{max}}_2^2 & \le \sum_{k=1}^n \frac{1}{Q_{min}} \modulus{x_{max,k}}^2 , \\
    r_{max} \norm{x_{max}}_2^2 & \le \frac{1}{Q_{min}} \sum_{k=1}^n \modulus{x_{max,k}}^2 , \\
    r_{max} \norm{x_{max}}_2^2 & \le \frac{1}{Q_{min}} \norm{x_{max}}_2^2 ,                 \\
    r_{max}                    & \le \frac{1}{Q_{min}} ,                                    \\
    Q_{min}                    & \le \frac{1}{r_{max}} .
\end{align*}

Пусть, например, в системе всего два излучателя, тогда матрица $R$ имеет вид:
\begin{gather}
    R
    = \begin{pmatrix}
        1        & R_{12} \\
        R_{12}^* & 1
    \end{pmatrix}
    \notag, \\
    %
    R_{12} = \frac{Q_{12}}{\sqrt{Q_{11}} \sqrt{Q_{22}}}
    \label{emission:emitter:efficiency:diagram_intersection}
\end{gather}
тогда наибольшее собственное значение матрицы $R$ (аналогично равенству~\eqref{emission:emitter:gain:maximum_eigenvalue}):
\begin{multline*}
    r_{max}
    = \frac{\tr(R) + \sqrt{\tr^2(R) - 4 \det(R)}}{2} = \\
    %
    = \frac{2 + \sqrt{2^2 - 4 (1 - \modulus{R_{12}}^2)}}{2}
    = \frac{2 + \sqrt{4 - 4 + 4 \modulus{R_{12}}^2}}{2} = \\
    %
    = \frac{2 + 2 \modulus{R_{12}}}{2}
    = 1 + \modulus{R_{12}} .
\end{multline*}
Таким образом,
\begin{align*}
    \min \{ Q_{11}, Q_{22} \} & \le \frac{1}{1 + \modulus{R_{12}}} , \\
    \min \{ \eta_1, \eta_2 \} & \le \frac{1}{1 + \modulus{R_{12}}} .
\end{align*}

\subsubsection{Вычисление интегралов}

Возникает вопрос о вычислении интегралов вида:
\[
    P = \frac{1}{r^2} \iint \limits_{S_r} p(\varphi, \theta) ds
\]
по сфере $S_r$ радиуса $r$.

Кратко, нужно перейти к угловым координатами $\varphi$ и $\theta$, в которых элемент поверхности $ds$ приближённо представляет собой прямоугольник
со сторонами $r d\varphi$ и $r \cos \theta d\theta$:
\[
    ds = r \cos \theta d\varphi \cdot r d\theta = r^2 d\Omega,
\]
тогда поверхностый интеграл преобразуется к кратному интегралу и затем к повторному интегралу:
\[
    P
    = \iint \limits_{S_r} p(\varphi, \theta) d\Omega
    = \int \limits_0^{2 \pi} \left( \int \limits_{-\frac{\pi}{2}}^{\frac{\pi}{2}} p(\varphi, \theta) \cos \theta d\theta \right) d\varphi .
\]

Далее следует формальный вывод последнего выражения через вычисление исходного поверхностного интеграла в декартовых координатах и перехода к угловым координатам.

Для вычисления интеграла сфера разделяется на верхнюю $z \ge 0$ и нижнюю $z < 0$ полусферы:
\begin{gather*}
    x^2 + y^2 + z^2 = r^2 , \\
    z^2 = r^2 - x^2 - y^2 , \\
    z = \pm \sqrt{r^2 - x^2 - y^2} .
\end{gather*}
Интеграл сводится к сумме двух кратных интегралов в декартовых координатах:
\begin{multline*}
    P
    = \frac{1}{r^2} \iint
    \limits_{
        \begin{array}{c}
            x^2 + y^2 \le r^2, \\
            z = \sqrt{r^2 - x^2 - y^2}
        \end{array}
    } p(\varphi, \theta) \sqrt{(z_x^\prime)^2 + (z_y^\prime)^2 + 1} dxdy + \\
    + \frac{1}{r^2} \iint
    \limits_{
        \begin{array}{c}
            x^2 + y^2 \le r^2, \\
            z = - \sqrt{r^2 - x^2 - y^2}
        \end{array}
    } p(\varphi, \theta) \sqrt{(z_x^\prime)^2 + (z_y^\prime)^2 + 1} dxdy ,
\end{multline*}
где производные под корнем:
\begin{gather*}
    z = \pm \sqrt{r^2 - x^2 - y^2} , \\
    z_x^\prime = \pm \frac{-2 x}{2 \sqrt{r^2 - x^2 - y^2}} , \\
    z_y^\prime = \pm \frac{-2 y}{2 \sqrt{r^2 - x^2 - y^2}} ,
\end{gather*}
и выражения под корнями в двух интегралах преобразуются одинаково:
\begin{multline*}
    (z_x^\prime)
    ^2 + (z_y^\prime)^2 + 1
    = \frac{x^2}{r^2 - x^2 - y^2} + \frac{y^2}{r^2 - x^2 - y^2} + 1 = \\
    %
    = \frac{x^2 + y^2 + r^2 + x^2 + y^2}{r^2 - x^2 - y^2}
    = \frac{r^2}{r^2 - x^2 - y^2}
\end{multline*}
При переходе в кратных интегралах к угловым координатам:
\begin{gather*}
    x = r \cos \theta \cos \varphi , \\
    y = r \cos \theta \sin \varphi ,
\end{gather*}
выражение под корнем преобразуется далее:
\begin{gather*}
    \frac{r^2}{r^2 - x^2 - y^2}
    = \frac{r^2}{r^2 - r^2 \cos^2 \theta \cos^2 \varphi - r^2 \cos^2 \theta \sin^2 \varphi}
    = \frac{r^2}{r^2 - r^2 \cos^2 \theta}
    = \frac{1}{1 - \cos^2 \theta}
    = \frac{1}{\sin^2 \theta} , \\
    %
    \sqrt{\frac{1}{\sin^2 \theta}}
    = \frac{1}{\modulus{\sin \theta}} .
\end{gather*}
Якобиан пребразования для угловых координат
\[
    \frac{\partial (x, y)}{\partial (\theta, \varphi)}
    = \begin{vmatrix}
        - r \sin \theta \cos \varphi & - r \cos \theta \sin \varphi \\
        - r \sin \theta \sin \varphi & r \cos \theta \cos \varphi
    \end{vmatrix}
    = - r^2 \cos \theta \sin \theta .
\]
Таким образом,
\begin{multline*}
    P = \frac{1}{r^2} \int \limits_0^{2 \pi} \int \limits_0^\frac{\pi}{2} p(\varphi, \theta) \frac{1}{\modulus{\sin \theta}} \modulus{r^2 \cos \theta \sin \theta} d\theta d\varphi + \\
    \shoveright{+ \frac{1}{r^2} \int \limits_0^{2 \pi} \int \limits_{-\frac{\pi}{2}}^0 p(\varphi, \theta) \frac{1}{\modulus{\sin \theta}} \modulus{r^2 \cos \theta \sin \theta} d\theta d\varphi = } \\
    %
    = \int \limits_0^{2 \pi} \int \limits_0^\frac{\pi}{2} p(\varphi, \theta) \modulus{\cos \theta} d\theta d\varphi
    + \int \limits_0^{2 \pi} \int \limits_{-\frac{\pi}{2}}^0 p(\varphi, \theta) \modulus{\cos \theta} d\theta d\varphi = \\
    %
    = \int \limits_0^{2 \pi} \int \limits_{-\frac{\pi}{2}}^\frac{\pi}{2} p(\varphi, \theta) \cos \theta d\theta d\varphi .
\end{multline*}

\subsubsection{Пример 2: разная поляризация}

Пусть в направлении $(\varphi, \theta)$ диаграмма направленности излучателя имеет вид:
\begin{gather*}
    F(\varphi, \theta)
    = \begin{pmatrix}
        A_1 e^{i \alpha_1} \cos \theta & 0                              \\
        0                              & A_2 e^{i \alpha_2} \cos \theta
    \end{pmatrix} , \\
    %
    A_1 \ge A_2.
\end{gather*}
Вычислим элементы матрицы $Q$:
\begin{multline*}
    Q_{11}
    = \int \limits_0^{2 \pi} \int \limits_{-\frac{\pi}{2}}^\frac{\pi}{2} f_1^*(\varphi, \theta) f_1(\varphi, \theta) \cos \theta d\theta d\varphi .
    = \int \limits_0^{2 \pi} \int \limits_{-\frac{\pi}{2}}^\frac{\pi}{2} A_1 e^{-i \alpha_1} \cos \theta A_1 e^{i \alpha} \cos \theta \cos \theta d\theta d\varphi = \\
    %
    = \int \limits_0^{2 \pi} \int \limits_{-\frac{\pi}{2}}^\frac{\pi}{2} A_1^2 \cos^3 \theta d\theta d\varphi
    = A_1^2 \int \limits_0^{2 \pi} \int \limits_{-\frac{\pi}{2}}^\frac{\pi}{2} \cos^3 \theta d\theta d\varphi
    = A_1^2 2 \pi \int \limits_{-\frac{\pi}{2}}^\frac{\pi}{2} \cos^3 \theta d\theta = \\
    %
    = A_1^2 2 \pi \int \limits_{-\frac{\pi}{2}}^\frac{\pi}{2} (1 - \sin^2 \theta ) \cos \theta d\theta
    = A_1^2 2 \pi \left( \int \limits_{-\frac{\pi}{2}}^\frac{\pi}{2} \cos \theta d\theta - \int \limits_{-\frac{\pi}{2}}^\frac{\pi}{2} \sin^2 \theta \cos \theta d\theta \right) = \\
    %
    = A_1^2 2 \pi \left( \left. \sin \theta \right|_{-\frac{\pi}{2}}^\frac{\pi}{2} - \left. \frac{\sin^3 \theta}{3} \right|_{-\frac{\pi}{2}}^\frac{\pi}{2} \right)
    = A_1^2 2 \pi \left( 2 - \frac{2}{3} \right)
    = \frac{8}{3} \pi A_1^2 .
\end{multline*}
Аналогично:
\begin{gather*}
    Q_{22}
    = \int \limits_0^{2 \pi} \int \limits_{-\frac{\pi}{2}}^\frac{\pi}{2} A_2^2 \cos^3 \theta d\theta d\varphi
    = \frac{8}{3} \pi A_2^2 , \\
    %
    Q_{12}
    = \int \limits_0^{2 \pi} \int \limits_{-\frac{\pi}{2}}^\frac{\pi}{2} 0 \cdot \cos \theta d\theta d\varphi
    = 0 .
\end{gather*}
Таким образом,
\[
    Q
    = \frac{8}{3} \pi
    \begin{pmatrix}
        A_1^2 & 0     \\
        0     & A_2^2
    \end{pmatrix} .
\]
Поскольку матрица $Q$ является диагональной, то можно сразу определить минимальный $\eta_{min}$ и максимальный $\eta_{max}$ КПД, которые являются минимальным и максимальным собственным
числом матрицы $Q$:
\begin{align*}
    \eta_{min} & = A_2^2 \cdot \frac{8}{3} \pi , \\
    \eta_{max} & = A_1^2 \cdot \frac{8}{3} \pi ,
\end{align*}
поскольку $A_1 > A_2$.

Коэффициент пересечения диаграмм \eqref{emission:emitter:efficiency:diagram_intersection}:
\begin{gather*}
    R_{12}
    = \frac{Q_{12}}{\sqrt{Q_{11}} \sqrt{Q_{22}}}
    = \frac{0}{A_1 \cdot A_2}
    = 0, \\
    %
    \modulus{R_{12}} = 0 .
\end{gather*}
Откуда
\[
    \min \{ \eta_1, \eta_2 \} \le \frac{1}{1 + \modulus{R_{12}}} = 1.
\]

\subsubsection{Пример 3: одинаковая поляризация}

Пусть в направлении $w$ диаграмма направленности излучателя имеет вид:
\[
    F(\varphi, \theta)
    = \begin{pmatrix}
        A_1 e^{i \alpha_1} \cos \theta & A_2 e^{i \alpha_2} \cos \theta \\
        0                              & 0
    \end{pmatrix} .
\]
Вычисляем элементы матрицы $Q$:
\begin{gather*}
    Q_{11}
    = \int \limits_0^{2 \pi} \int \limits_{-\frac{\pi}{2}}^\frac{\pi}{2} A_1^2 \cos^3 \theta d\theta d\varphi
    = \frac{8}{3} \pi A_2^2 , \\
    %
    Q_{12}
    = \int \limits_0^{2 \pi} \int \limits_{-\frac{\pi}{2}}^\frac{\pi}{2} A_1 A_2 e^{i \alpha_1 - i \alpha_2} \cos^3 \theta d\theta d\varphi
    = \frac{8}{3} \pi A_1 A_2 e^{i(\alpha_1 - \alpha_2)} , \\
    %
    Q_{22}
    = \int \limits_0^{2 \pi} \int \limits_{-\frac{\pi}{2}}^\frac{\pi}{2} A_2^2 \cos^3 \theta d\theta d\varphi
    = \frac{8}{3} \pi A_2^2 .
\end{gather*}
Таким образом,
\[
    Q
    = \frac{8}{3} \pi
    \begin{pmatrix}
        A_1^2                              & A_1 A_2 e^{i(\alpha_1 - \alpha_2)} \\
        A_1 A_2 e^{i(\alpha_2 - \alpha_1)} & A_2^2
    \end{pmatrix} .
\]
След и определитель матрицы $Q$:
\begin{gather*}
    \tr(Q) = \frac{8}{3} \pi ( A_1^2 + A_2^2 ) , \\
    %
    \det(Q)
    = \left( \frac{8}{3} \pi \right)^2 \left( A_1^2 A_2^2 - A_1^2 A_2^2 \right)
    = 0 .
\end{gather*}
Корень из дискриминанта:
\[
    \sqrt{\tr^2(Q) - 4 \det(Q)}
    = \frac{8}{3} \pi \sqrt{( A_1^2 + A_2^2 )^2}
    = \frac{8}{3} \pi ( A_1^2 + A_2^2 ),
\]
тогда
\begin{align*}
    \eta_{min} &
    = \frac{A_1^2 + A_2^2 - ( A_1^2 + A_2^2 )}{2} \cdot \frac{8}{3} \pi
    = 0 ,        \\
    %
    \eta_{max} &
    = \frac{A_1^2 + A_2^2 + ( A_1^2 + A_2^2 )}{2} \cdot \frac{8}{3} \pi
    = \frac{2 (A_1^2 + A_2^2)}{2} \cdot \frac{8}{3} \pi
    = (A_1^2 + A_2^2) \cdot \frac{8}{3} \pi .
\end{align*}

Коэффициент пересечения диаграмм \eqref{emission:emitter:efficiency:diagram_intersection}:
\begin{gather*}
    R_{12}
    = \frac{Q_{12}}{\sqrt{Q_{11}} \sqrt{Q_{22}}}
    = \frac{A_1 A_2 e^{i(\alpha_1 - \alpha_2)}}{A_1 \cdot A_2}
    = e^{i(\alpha_1 - \alpha_2)}, \\
    %
    \modulus{R_{12}} = 1 .
\end{gather*}
Откуда
\[
    \min \{ \eta_1, \eta_2 \} \le \frac{1}{1 + \modulus{R_{12}}} = \frac{1}{1+1} = \frac{1}{2}
\]

\subsubsection{Пример 4: общий случай}

Пусть в направлении $w$ диаграмма направленности имеет вид:
\[
    F(\varphi, \theta)
    =
    \begin{pmatrix}
        0.1 e^{i \frac{\pi}{6}} \cos \theta  & 0.3 e^{i \frac{5 \pi}{4}} \cos \theta   \\
        0.2 e^{i \frac{\pi}{3}} \sin \varphi & 0.1 e^{- i \frac{\pi}{10}} \sin \varphi
    \end{pmatrix} .
\]
Вычисления оптимальных векторов огибающих и поляризации смотри в файле Matlab \texttt{emission/two/efficiency.m}.

\subsubsection{Пример 5: крестовой излучатель}

Вычисление КПД в файле \texttt{emission/cross/efficiency.m}.

\section{Диаграммообразующая схема}

Пусть $a_\alpha$ и $b_\alpha$ --- векторы огибающих сигналов сечения входа схемы и $a$ и $b$ --- векторы огибащих сигналов сечения выхода схемы. У схемы два входа ---
$a_\alpha$ и $b$ и два выхода --- $a$ и $b_\alpha$, которые связаны со входами:
\begin{gather}
    a        = S_{\beta \alpha} a_\alpha + S_{\beta \beta} b
    \label{emission:scheme:upper_output}, \\
    b_\alpha = S_{\alpha \alpha} a_\alpha + S_{\alpha \beta} b
    \label{emission:scheme:lower_output}
\end{gather}
Причем $a$ и $b$ --- огибающие сигналов в сечении входа антенны, для которых справедливо равенство:
\begin{equation}
    \label{emission:scheme:reflections}
    b = S_\beta a ,
\end{equation}
где $S_\beta$ --- матрица рассеяния антенны.

Согласно равенствам~\eqref{emission:scheme:upper_output} и~\eqref{emission:scheme:reflections} вектор огибающих входа антенны $a$:
\begin{align}
    a                                 & = S_{\beta \alpha} a_\alpha + S_{\beta \beta} b , \notag                                                 \\
    a                                 & = S_{\beta \alpha} a_\alpha + S_{\beta \beta} S_\beta a , \notag                                         \\
    a - S_{\beta \beta} S_\beta a     & = S_{\beta \alpha} a_\alpha , \notag                                                                     \\
    ( I - S_{\beta \beta} S_\beta ) a & = S_{\beta \alpha} a_\alpha , \notag                                                                     \\
    a                                 & = ( I - S_{\beta \beta} S_\beta )^{-1} S_{\beta \alpha} a_\alpha , \label{emission:scheme:antenna_input}
\end{align}
тогда линейная часть диаграммы направленности антенны с диаграммообразующей схемой:
\[
    \widetilde{F}_s(\varphi, \theta, a_\alpha)
    = F_a(\varphi, \theta) a(a_\alpha)
    = F_a(\varphi, \theta) ( I - S_{\beta \beta} S_\beta )^{-1} S_{\beta \alpha} a_\alpha
    = F_s(\varphi, \theta) a_\alpha,
\]
где
\[
    F_s(\varphi, \theta) = F_a(\vec{w}) ( I - S_{\beta \beta} S_\beta )^{-1} S_{\beta \alpha} .
\]

Из равенств~\eqref{emission:scheme:upper_output},~\eqref{emission:scheme:reflections} и~\eqref{emission:scheme:antenna_input} вектор огибающих выхода
диаграммообразующей схемы:
\begin{align*}
    b_\alpha & = S_{\alpha \alpha} a_\alpha + S_{\alpha \beta} b ,                                                          \\
    b_\alpha & = S_{\alpha \alpha} a_\alpha + S_{\alpha \beta} S a ,                                                        \\
    b_\alpha & = S_{\alpha \alpha} a_\alpha + S_{\alpha \beta} S ( I - S_{\beta \beta} S )^{-1} S_{\beta \alpha} a_\alpha , \\
    b_\alpha & = ( S_{\alpha \alpha} + S_{\alpha \beta} S ( I - S_{\beta \beta} S )^{-1} S_{\beta \alpha} ) a_\alpha ,
\end{align*}
откуда матрица рассеяния для диаграммообразующей схемы:
\[
    S_\alpha = S_{\alpha \alpha} + S_{\alpha \beta} S ( I - S_{\beta \beta} S )^{-1} S_{\beta \alpha} .
\]
