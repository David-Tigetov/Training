\chapter{Излучение сигналов}


\section{Комплексное представление сигналов}

\subsection{Комплексные числа}

Комплексные числа $z = a + i b$ состоят из двух частей: действительной
\[
    a = \real{z},
\]
и мнимой
\[
    b = \image{z},
\]
где $a$ и $b$ --- действительные числа.

Комплексные числа можно представлять векторами на комплексной плоскости.

Если комплексное число не нулевое, то можно получить тригонометрическую форму:
\begin{gather*}
    z
    = a + i b
    = \sqrt{a^2 + b^2} \left ( \frac{a}{\sqrt{a^2 + b^2}} + i \frac{b}{\sqrt{a^2 + b^2}} \right )
    = A \left ( \cos \varphi + i \sin \varphi \right )
    = A \cos \varphi + i A \sin \varphi , \\
    %
    A = \sqrt{a^2 + b^2} , \\
    \cos \varphi = \frac{a}{\sqrt{a^2 + b^2}} , \\
    \sin \varphi = \frac{b}{\sqrt{a^2 + b^2}} ,
\end{gather*}
где $A$ и $\varphi$ --- модуль и аргумент комплексного числа.

\subsection{Комплексная экспонента}

Для вещественных чисел $a$ определена функция $e^a$, а её продолжением на комплексную плоскость является функция:
\[
    e^{a + i b}
    = e^a \cdot e^{i b}
    = e^a \cdot \left ( \cos b + i \sin b \right ).
\]

\subsection{Комплексное представление сигналов}

Пусть функция $u(t)$ описывает напряжение сигнала:
\[
    u(t) = A(t) \cos \varphi(t) ,
\]
где $A(t)$ --- амплитуда, $\varphi(t)$ --- фаза.

Функция $u(t)$ соответствует действительной части комплексно-значной функции $v(t)$ вещественного переменного $t$:
\begin{gather*}
    u(t) = \real{v(t)} , \\
    %
    v(t)
    = A(t) \cos \varphi(t) + i A(t) \sin \varphi(t)
    = A(t) \left ( \cos \varphi(t) + i \sin \varphi(t) \right )
    = A(t) e^{i \varphi(t)} .
\end{gather*}

\subsection{Сложение сигналов}

Пусть имеются сигналы, которые описываются функциями $u_1(t)$ и $u_2(t)$:
\begin{gather*}
    u_1(t) = A_1(t) \cos \varphi_1(t) , \\
    u_2(t) = A_2(t) \cos \varphi_2(t) ,
\end{gather*}
которым соответствую комплексные представления:
\begin{gather*}
    v_1(t) = A_1(t) e^{i \varphi_1(t)} , \\
    v_2(t) = A_2(t) e^{i \varphi_2(t)} .
\end{gather*}
тогда в результате сложения получается функция $u(t)$:
\[
    u(t)
    = u_1(t) + u_2(t)
    = \real{v_1(t)} + \real{v_2(t)}
    = \real{v_1(t) + v_2(t)} ,
\]
которой соответствует комплексное представление:
\[
    v(t) = v_1(t) + v_2(t).
\]

\subsection{Преобразование}

Пусть сигнал с функцией $u(t)$ и комплексным представлением $v(t)$:
\begin{gather*}
    u(t) = A(t) \cos \varphi(t) , \\
    v(t) = A(t) e^{i \varphi(t)}
\end{gather*}
преобразуется в некотором устройстве, и в результате преобразования изменяются амплитуда или фаза:
\[
    \widetilde{u}(t) = B(t) \cdot A(t) \cos ( \varphi(t) + \theta(t) ) ,
\]
тогда комплексное представление $\widetilde{v}(t)$ сигнала $\widetilde{u}(t)$ имеет вид:
\[
    \widetilde{v}(t)
    = A(t) \cdot B(t) e^{i (\varphi(t) + \theta(t))}
    = A(t) e^{i \varphi(t)} \cdot B(t) e^{i \theta(t)} = v(t) \cdot s(t),
\]
где функция $s(t)$ является комплексным представлением преобразования:
\[
    s(t) = B(t) e^{i \theta(t)} .
\]

\subsection{Комплексная огибающая}

Наиболее часто в радиолокации встречаются узкополосные сигналы, представляющие собой суперпозицию колебаний с частотами из узкой полосы частот, вокруг несущей
частоты $\omega$ (частота $\omega$ обычно составляет несколько мегагерц, поскольку антенны излучают сигналы высоких частот):
\[
    u(t) = A(t) \cos ( \omega t + \theta(t) ),
\]
где функции $A(t)$ и $\varphi(t)$ представляют модуляцию сигнала. Такому сигналу соответствует комплексное представление:
\[
    v(t)
    = A(t) e^{i ( \omega t + \theta(t) )}
    = A(t) e^{i \theta(t)} \cdot e^{i \omega t} .
\]
Первый множитель
\[
    v_s(t) = A(t) e^{i \theta(t)}
\]
является комплексной огибающей.

\subsection{Квадратурный детектор}

Сигнал $u(t)$ можно представить в виде суммы:
\[
    u(t)
    = A(t) \cos ( \omega t + \varphi(t) )
    = A(t) \cos \varphi(t) \cos \omega t - A(t) \sin \varphi(t) \sin \omega t .
\]
в которой множители
\begin{gather*}
    I(t) = A(t) \cos \varphi(t) , \\
    Q(t) = - A(t) \sin \varphi(t) ,
\end{gather*}
называются синфазной и квадратурной составляющими соответственно.

Составляющие $I(t)$ и $Q(t)$ определяют действительную и мнимую части комплексной огибающей $v(t)$:
\begin{gather*}
    I(t) = \real{v(t)} , \\
    Q(t) = - \image{v(t)} .
\end{gather*}

Устройство, которое выделяет составляющие $I(t)$ и $Q(t)$, называется квадратурным детектором.


\section{Двухканальный излучатель}

\subsection{Диаграмма направленности}

Рассматривается излучатель с двумя каналами поляризации и двумя входами, соответствующим этим двум каналам. Возникающие физические явления описываются только
линейными выражениями, а нелинейные эффекты не учитываются.

Пусть $a$ --- вектор огибающих входов излучателя:
\[
    a
    = \begin{pmatrix}
          a_1 \\
          a_2
    \end{pmatrix} .
\]

В результате отражений и связи между каналами, наводятся отраженные сигналы с огибающими $b$:
\begin{gather*}
    b
    = \begin{pmatrix}
          b_1 \\
          b_2
    \end{pmatrix}
    = \begin{pmatrix}
          s_{11} a_1 + s_{12} a_2 \\
          s_{21} a_1 + s_{22} a_2
    \end{pmatrix}
    = S a, \\
    %
    S
    = \begin{pmatrix}
          s_{11} & s_{12} \\
          s_{21} & s_{22}
    \end{pmatrix},
\end{gather*}
где $S$ --- матрица рассеяния.

Неотраженные части сигналов поступают в волноводы, из которых происходит излучение электромагнитной волны в пространство, излучение является неоднородным и его характеристики
зависят от рассматриваемого направления. Если с излучателем связана сферическая система координат (с началом отсчёта в конце волноводов), то направление можно задать с
помощью волнового вектора $w$, длина которого:
\[
    \modulus{w} = \frac{2 \pi}{\lambda},
\]
где $\lambda$ --- длина волны на несущей частоте. В выбранном направлении $w$ на расстоянии $R$ напряженность электрического поля $E$ является функцией
направления $w$, расстояния $R$ и огибающих волн $e$ в волноводах излучателя, которые зависят от огибающих $a$ сигналов на входах:
\[
    E = E(w, R, a).
\]
В дальней зоне при больших расстояниях $R$ напряженность $E$ можно приближенно представить линейной частью по огибающим сигналов $a$:
\begin{equation}
    \label{emission:emitter:diagram:tension}
    E(w,R,a)
    \approx F(w) a \cdot \frac{e^{i \modulus{w} R}}{R} ,
\end{equation}
где множитель $e^{i \modulus{w} R}$ определяет смещение по фазе на расстоянии $R$ от начала отсчёта, множитель $\frac{1}{R}$ показывает затухание амплитуды
вектора напряженности $E$, и $F(w)$ --- диаграмма направленности излучателя:
\[
    F(w)
    = \begin{pmatrix}
          f_{1, \theta}(w)  & f_{2, \theta}(w)  \\
          f_{1, \varphi}(w) & f_{2, \varphi}(w)
    \end{pmatrix} ,
\]
в которой столбцы задают парциальные диаграммы направленности по каналам поляризации:
\[
    f_1(w)
    = \begin{pmatrix}
          f_{1,\theta}(w) \\
          f_{1,\varphi}(w)
    \end{pmatrix}
    , \;
    f_2(w)
    = \begin{pmatrix}
          f_{2,\theta}(w) \\
          f_{2,\varphi}(w)
    \end{pmatrix}
    .
\]

\subsection{Коэффициент усиления}

\subsubsection{Вычисление}

Реализованный коэффициент усиления $G(w)$ определяет относительную величину плотности потока $\Pi(w)$ мощности в дальней зоне в
направлении $w$:
\[
    G(w)
    = \frac{4 \pi R^2}{P_{inp}} \cdot \Pi(w)
    = \frac{4 \pi R^2}{P_{inp}} \cdot \frac{1}{Z_0} \norm{E(w)}^2
    = \frac{4 \pi R^2}{P_{inp}} \cdot \frac{1}{Z_0} \frac{\norm{F(w) a}^2}{R^2}
    = \frac{4 \pi}{Z_0} \cdot \frac{\norm{F(w) a}^2}{P_{inp}} ,
\]
где $Z_0 = 120 \pi$ --- волновое сопротивление свободного пространства.

Пусть направление $w$ фиксировано, тогда коэффициент усиления пропорционален отношению:
\begin{equation}
    \label{emission:emitter:gain:rayleigh}
    \rho(a)
    = \frac{\norm{F(w) a}^2}{P_{inp}}
    = \frac{\norm{F(w) a}^2}{\norm{a}^2}
    = \frac{a^* F^*(w) F(w) a}{a^* a}
\end{equation}
и возникает вопрос, каким образом нужно сформировать огибающие входных сигналов $a$ чтобы отношение $\rho(a)$ и коэффициент усиления в заданном направлении $w$
оказался наибольшим?

Отношение $\rho(a)$ является отношением Релея: числитель --- квадратичная форма с эрмитовой матрией $F^*(w) F(w)$, знаменатель --- квадрат нормы $a$.
Наибольшее значение $G_{max}$:
\[
    G_{max} = \max \limits_{a} G
\]
достигается в направлении $a_{max}$, совпадающим с направлением собственных векторов, соответствующих наибольшему собственному числу $\lambda_{max}$ матрицы $F^*(w)F(w)$.
Вектор $a_{max}$ и число $\lambda_{max}$ удовлетворяют уравнению:
\begin{equation}
    \label{emission:emitter:gain:optimal_input}
    F^*(w) F(w) a_{max} = \lambda_{max} a_{max}
\end{equation}
Нахождение $a_{max}$ путём решение системы~\eqref{emission:emitter:gain:optimal_input} является первым способом. В некоторых случаях матрица $F^*(w)F(w)$ оказывается сложной
для определения собственных чисел из векторов, поскольку порядок матрицы равен количеству излучателей.

Второй способ определения $a_{max}$ связан с нахождение вектора оптимальной поляризации $p_{max}$. Домножим левую и правую части
уравнения~\eqref{emission:emitter:gain:optimal_input} на $F(w)$ слева:
\[
    F(w) F^*(w) F(w) a_{max} = \lambda_{max} F(w) a_{max} .
\]
Объединяя два уравнения получим систему:
\begin{gather}
    \left \{
    \begin{array}{c}
        F^*(w) F(w) a_{max} = \lambda_{max} a_{max} \\
        F(w) F^*(w) p_{max} = \lambda_{max} p_{max}
    \end{array}
    \right .
    \label{emission:emitter:gain:system}, \\
    p_{max} = F(w) a_{max} \notag.
\end{gather}

Пусть
\[
    B
    = F(w) F^*(w)
    = \begin{pmatrix}
          f_\theta f_\theta^*  & f_{\theta} f_\varphi^* \\
          f_\varphi f_\theta^* & f_\varphi f_\varphi^*
    \end{pmatrix} ,
\]
где $f_\theta$ и $f_\varphi$ --- строки матрицы $F(w)$:
\begin{gather*}
    f_\theta
    = \begin{pmatrix}
          f_{1,\theta}(w) & f_{2,\theta}(w)
    \end{pmatrix}, \\
%
    f_\varphi
    = \begin{pmatrix}
          f_{1,\varphi}(w) & f_{2,\varphi}(w)
    \end{pmatrix} .
\end{gather*}

У матрицы $B$ два собственных значения $\lambda_{min}$ и $\lambda_{max}$, которые являются корнями характеристического уравнения:
\begin{multline*}
    \begin{vmatrix}
        f_\theta f_\theta^* - \lambda & f_{\theta} f_\varphi^*          \\
        f_\varphi f_\theta^*          & f_\varphi f_\varphi^* - \lambda
    \end{vmatrix}
    = (f_\theta f_\theta^* - \lambda) (f_\varphi f_\varphi^* - \lambda) - f_\varphi f_\theta^* f_{\theta} f_\varphi^* = \\
%
    = \lambda^2 - ( f_\theta f_\theta^* + f_\varphi f_\varphi^* ) \lambda + f_\theta f_\theta^* f_\varphi f_\varphi^* - f_\varphi f_\theta^* f_{\theta} f_\varphi^* = \\
%
    = \lambda^2 - \tr(B) \lambda + \det(B) ,
\end{multline*}
где $\tr(B)$ и $\det(B)$ --- след и определитель матрицы $B$. Корни характеристического уравнения:
\begin{align}
    \lambda_{min} & = \frac{\tr(B) - \sqrt{\tr^2(B) - 4 \det(B)}}{2} \label{emission:emitter:gain:minimum_eigenvalue} , \\
    \lambda_{max} & = \frac{\tr(B) + \sqrt{\tr^2(B) - 4 \det(B)}}{2} \label{emission:emitter:gain:maximum_eigenvalue}.
\end{align}
Вектор $p_{max}$ находим как решение второго уравнения системы~\eqref{emission:emitter:gain:system}:
\begin{gather*}
    B p_{max} = \lambda_{max} p_{max} , \\
    ( B - \lambda I ) p_{max} = 0 .
\end{gather*}
Можно выбрать решение, для которого $\norm{p_{max}} = 1$. Далее вектор $a_{max}$ находим из первого уравнения системы~\eqref{emission:emitter:gain:system}:
\begin{align*}
    F^*(w) F(w) a_{max}                         & = \lambda_{max} a_{max} , \\
    \frac{1}{\lambda_{max}} F^*(w) F(w) a_{max} & = a_{max} , \\
    \frac{1}{\lambda_{max}} F^*(w) p_{max}      & = a_{max} .
\end{align*}
{
    \color{red}
    Пропустить?

    Наибольший коэффициент усиления $G_{max}$:
    \begin{multline*}
        G_{max}
        = \frac{a_{max}^* F^*(w) F(w) a_{max}}{a_{max}^* a_{max}} = \\
%
        = \frac{p_{max}^* F(w) \left ( \frac{1}{\lambda_{max}} \right )^* F^*(w) F(w) \frac{1}{\lambda_{max}} F^*(w) p_{max}}{ p_{max}^* F(w) \left ( \frac{1}{\lambda_{max}} \right )^* \frac{1}{\lambda_{max}} F^*(w) p_{max}} = \\
%
        = \frac{\modulus{\frac{1}{\lambda_{max}}}^2 p_{max}^* F(w) F^*(w) F(w) F^*(w) p_{max}}{ \modulus{\frac{1}{\lambda_{max}}}^2 p_{max}^* F(w) F^*(w) p_{max}}
        = \frac{\norm{F(w) F^*(w) p_{max}}^2}{ \norm{F^*(w) p_{max}}^2} .
    \end{multline*}
}

\subsubsection{Пример 1: один канал}

Излучение только одного канала.

Пусть выбрано и зафиксировано направление $w$, и в этом направлении диаграмма направленности $F(w)$ имеет вид:
\begin{gather*}
    F(w)
    = \begin{pmatrix}
          A_1 e^{i \alpha_1} & 0 \\
          A_2 e^{i \alpha_2} & 0
    \end{pmatrix} , \\
%
    f_1(w) = \begin{pmatrix}
                 A_1 e^{i \alpha_1} \\
                 A_2 e^{i \alpha_2}
    \end{pmatrix}
    , \;
    f_2(w) = \begin{pmatrix}
                 0 \\
                 0
    \end{pmatrix} ,
\end{gather*}
где амплитуды $A_1 \in \mathbb{R}$, $A_2 \in \mathbb{R}$ и смещения фаз $\alpha_1 \in \mathbb{R}$, $\alpha_2 \in \mathbb{R}$.

Необходимо найти оптимальный вектор комплексных огибающих $a_{max}$ сигналов на входах излучателей, при котором получается наибольший коэффициент
усиления~\eqref{emission:emitter:gain:rayleigh}:
\[
    \rho(a)
    = \frac{a^*F^*(w)F(w) a}{a^*a}.
\]
Решаем задачу первым способом, матрица системы~\eqref{emission:emitter:gain:optimal_input}:
\[
    F^*(w) F(w)
    =
    \begin{pmatrix}
        A_1 e^{-i \alpha_1} & A_2 e^{-i \alpha_2} \\
        0                   & 0
    \end{pmatrix}
    \begin{pmatrix}
        A_1 e^{i \alpha_1} & 0 \\
        A_2 e^{i \alpha_2} & 0
    \end{pmatrix}
    =
    \begin{pmatrix}
        A_1^2 + A_2^2 & 0 \\
        0             & 0
    \end{pmatrix}
\]
Отсюда легко получаются собственные числа:
\begin{gather*}
    \determinant{F^*(w) F(w) - \lambda I} = 0 , \\
%
    \begin{vmatrix}
        A_1^2 + A_2^2 - \lambda & 0         \\
        0                       & - \lambda
    \end{vmatrix}
    = 0 , \\
%
    \lambda_{max} = A_1^2 + A_2^2 , \\
    \lambda_{min} = 0 .
\end{gather*}
Наибольшее значение отношения Релея:
\[
    \rho_{max} = \lambda_{max} = A_1^2 + A_2^2 .
\]
Оптимальный вектор комплесных огибающих сигналов на входах $a_{max}$:
\begin{gather*}
    ( F^*(w) F(w) - \lambda_{max} I ) a_{max} = 0 , \\
%
    \begin{pmatrix}
        A_1^2 + A_2^2 - (A_1^2 + A_2^2) & 0                 \\
        0                               & - (A_1^2 + A_2^2)
    \end{pmatrix}
    a_{max} = 0 , \\
%
    \begin{pmatrix}
        0 & 0                 \\
        0 & - (A_1^2 + A_2^2)
    \end{pmatrix}
    a_{max} = 0 , \\
%
    a_{max}
    = \begin{pmatrix}
          1 \\
          0
    \end{pmatrix} .
\end{gather*}
Оптимальная поляризация $p_{max}$:
\[
    p_{max}
    = F(w) a_{max}
    = \begin{pmatrix}
          A_1 e^{i \alpha_1} & 0 \\
          A_2 e^{i \alpha_2} & 0
    \end{pmatrix}
    \begin{pmatrix}
        1 \\
        0
    \end{pmatrix}
    = \begin{pmatrix}
          A_1 e^{i \alpha_1} \\
          A_2 e^{i \alpha_2}
    \end{pmatrix} .
\]

\subsubsection{Пример 2: разная поляризация}

Излучение в разной фазе в двух ортогональных плоскостях по углу места и по азимуту.

Пусть в выбранном направлении $w$ диаграмма направленности имеет вид:
\begin{gather*}
    F(w) = \begin{pmatrix}
               A_1 e^{i \alpha_1} & 0                  \\
               0                  & A_2 e^{i \alpha_2}
    \end{pmatrix} , \\
%
    f_1(w) = \begin{pmatrix}
                 A_1 e^{i \alpha_1} \\
                 0
    \end{pmatrix}
    , \;
    f_2(w) = \begin{pmatrix}
                 0 \\
                 A_2 e^{i \alpha_2}
    \end{pmatrix} ,
\end{gather*}
где амплитуды $A_1 \in \mathbb{R}$, $A_2 \in \mathbb{R}$, смещения фаз $\alpha_1 \in \mathbb{R}$, $\alpha_2 \in \mathbb{R}$ и для определённости
\[
    A_1 > A_2 .
\]

Необходимо найти оптимальный вектор комплексных огибающих $a_{max}$ сигналов на входах излучателей, при котором получается наибольший коэффициент
усиления~\eqref{emission:emitter:gain:rayleigh}:
\[
    \rho(a)
    = \frac{a^* F^*(w) F(w) a}{a^* a}.
\]
Решаем задачу первым способом, матрица системы~\eqref{emission:emitter:gain:optimal_input}:
\[
    F^*(w) F(w)
    =
    \begin{pmatrix}
        A_1 e^{-i \alpha_1} & 0                   \\
        0                   & A_2 e^{-i \alpha_2}
    \end{pmatrix}
    \begin{pmatrix}
        A_1 e^{i \alpha_1} & 0                  \\
        0                  & A_2 e^{i \alpha_2}
    \end{pmatrix}
    =
    \begin{pmatrix}
        A_1^2 & 0     \\
        0     & A_2^2
    \end{pmatrix}
\]
Собственные числа:
\begin{gather*}
    \determinant{F^*(w) F(w) - \lambda I} = 0 , \\
%
    \begin{vmatrix}
        A_1^2 - \lambda & 0               \\
        0               & A_2^2 - \lambda
    \end{vmatrix}
    = 0 , \\
%
    \lambda_{max} = A_1^2 , \\
    \lambda_{min} = A_2^2 .
\end{gather*}
Наибольшее значение отношения Релея:
\[
    \rho_{max} = \lambda_{max} = A_1^2 .
\]
Оптимальный вектор комплексных огибающих сигналов на входах излучателя:
\begin{gather*}
    ( F^*(w) F(w) - \lambda_{max} I ) a_{max} = 0 , \\
%
    \begin{pmatrix}
        A_1^2 - A_1^2 & 0             \\
        0             & A_2^2 - A_1^2
    \end{pmatrix}
    a_{max} = 0 , \\
%
    \begin{pmatrix}
        0 & 0             \\
        0 & A_2^2 - A_1^2
    \end{pmatrix}
    a_{max} = 0 , \\
%
    a_{max}
    = \begin{pmatrix}
          1 \\
          0
    \end{pmatrix} .
\end{gather*}
Оптимальная поляризация $p_{max}$:
\[
    p_{max}
    = F(w) a_{max}
    = \begin{pmatrix}
          A_1 e^{i \alpha_1} & 0                  \\
          0                  & A_2 e^{i \alpha_2}
    \end{pmatrix}
    \begin{pmatrix}
        1 \\
        0
    \end{pmatrix}
    = \begin{pmatrix}
          A_1 e^{i \alpha_1} \\
          0
    \end{pmatrix} .
\]

\subsubsection{Пример 3: одинаковая поляризация}

Два канала производят колебания в разной фазе, но в одной плоскости угла места.

Пусть в выбранном направлении $w$ диаграмма направленности имеет вид:
Пусть парциальные диаграммы направленности каналов излучателя имеют вид:
\[
    F(w)
    =
    \begin{pmatrix}
        A_1 e^{i \alpha_1} & A_2 e^{i \alpha_2} \\
        0                  & 0
    \end{pmatrix} , \\
%
    f_1(w) = \begin{pmatrix}
                 A_1 e^{i \alpha_1} \\
                 0
    \end{pmatrix}
    , \;
    f_2(w) = \begin{pmatrix}
                 A_2 e^{i \alpha_2} \\
                 0
    \end{pmatrix} ,
\]
где амплитуды $A_1 \in \mathbb{R}$, $A_2 \in \mathbb{R}$, смещения фаз $\alpha_1 \in \mathbb{R}$, $\alpha_2 \in \mathbb{R}$.

В первом способе матрица системы~\eqref{emission:emitter:gain:optimal_input}:
\[
    F^*(w) F(w)
    =
    \begin{pmatrix}
        A_1 e^{-i \alpha_1} & 0 \\
        A_2 e^{-i \alpha_2} & 0
    \end{pmatrix}
    \begin{pmatrix}
        A_1 e^{i \alpha_1} & A_2 e^{i \alpha_2} \\
        0                  & 0
    \end{pmatrix}
    =
    \begin{pmatrix}
        A_1^2                            & A_1 e^{-i \alpha_1 + i \alpha_2} \\
        A_2 e^{-i \alpha_2 + i \alpha_1} & A_2^2
    \end{pmatrix}
\]
получается "сложной"{}, поэтому рассмотрим матрицу системы для оптимальной поляризации:
\[
    F(w) F^*(w)
    =
    \begin{pmatrix}
        A_1 e^{i \alpha_1} & A_2 e^{i \alpha_2} \\
        0                  & 0
    \end{pmatrix}
    \begin{pmatrix}
        A_1 e^{-i \alpha_1} & 0 \\
        A_2 e^{-i \alpha_2} & 0
    \end{pmatrix}
    =
    \begin{pmatrix}
        A_1^2 + A_2^2 & 0 \\
        0             & 0
    \end{pmatrix} .
\]
Эта матрица "простая"{}, и её собственные числа
\begin{gather*}
    \determinant{F(w)F^*(w) - \lambda I} = 0 , \\
%
    \begin{vmatrix}
        A_1^2 + A_2^2 - \lambda & 0         \\
        0                       & - \lambda
    \end{vmatrix} = 0, \\
%
    \lambda_{max} = A_1^2 + A_2^2 , \\
    \lambda_{min} = 0 .
\end{gather*}

Наибольшее значение отношения Релея:
\[
    \rho_{max} = \lambda_{max} = A_1^2 + A_2^2.
\]

Наибольшему собственному числу $\lambda_{max}$ соответствует оптимальный вектор поляризации $p_{max}$:
\begin{gather*}
    (F(w)F^*(w) - \lambda_{max} I) p_{max} = 0 , \\
%
    \begin{pmatrix}
        A_1^2 + A_2^2 - \lambda_{max} & 0               \\
        0                             & - \lambda_{max}
    \end{pmatrix}
    p_{max} = 0 , \\
%
    \begin{pmatrix}
        0 & 0                 \\
        0 & - (A_1^2 + A_2^2)
    \end{pmatrix}
    p_{max} = 0 , \\
%
    p_{max} = \begin{pmatrix}
                  1 \\
                  0
    \end{pmatrix} .
\end{gather*}
Оптимальный вектор огибающих сигналов на входах излучателя:
\[
    a_{max}
    = F^*(w) p_{max}
    = \begin{pmatrix}
          A_1 e^{- i \alpha_1} & 0 \\
          A_2 e^{- i \alpha_2} & 0
    \end{pmatrix}
    \begin{pmatrix}
        1 \\
        0
    \end{pmatrix}
    = \begin{pmatrix}
          A_1 e^{- i \alpha_1} \\
          A_2 e^{- i \alpha_2}
    \end{pmatrix} .
\]
Компоненты вектора $a_{max}$ указывают на необходимость обратных смещений фаз входных сигналов двух каналов излучателя, с тем чтобы колебания излучения складывались в одной фазе.

\subsubsection{Пример 4: общий случай}

Пусть в выбранном направлении диаграмма направленности имеет вид:
\[
    F(w)
    =
    \begin{pmatrix}
        0.1 e^{i \frac{\pi}{6}} & 0.3 e^{i \frac{5 \pi}{4}}  \\
        0.2 e^{i \frac{\pi}{3}} & 0.1 e^{- i \frac{\pi}{10}}
    \end{pmatrix} .
\]
Вычисления оптимальных векторов огибающих и поляризации смотри в файле Matlab \texttt{emission/two/gain.m}.

\subsubsection{Пример 5: крестовой излучатель}

В декартовой системе координат $X$, $Y$, $Z$ излучатель первого канала направлен вдоль оси $Z$.

Парциальная диаграмма первого канала в сферических координатах:
\[
    f_1(w) =
    \begin{pmatrix}
        \cos \theta \\
        0
    \end{pmatrix} ,
\]
а в декартовых координатах:
\[
    f_{1,c}(w) = \cos \theta \cdot u_{\theta} + 0 \cdot u_{\varphi} ,
\]
где $u_\theta$, $u_\varphi$ --- образы орт координатных прямых углов $\theta$ и $\varphi$. В системе $C = (X, Y, Z)$ декартовы координаты точки связаны со сферическими
координатами системы $S = (\rho, \theta, \varphi)$ равенствами:
\begin{gather*}
    x = \rho \cos \theta \cos \varphi, \\
    y = \rho \cos \theta \sin \varphi, \\
    z = \rho \sin \theta.
\end{gather*}
Частные производные дают направления орт координатных прямых углов:
\begin{gather*}
    U_{\theta} =
    \begin{pmatrix}
        x_{\theta}^{\prime} \\
        y_{\theta}^{\prime} \\
        z_{\theta}^{\prime} \\
    \end{pmatrix}
    =
    \begin{pmatrix}
        - \rho \sin \theta \cos \varphi \\
        - \rho \sin \theta \sin \varphi \\
        \rho \cos \theta
    \end{pmatrix}, \\
%
    U_{\varphi} =
    \begin{pmatrix}
        x_{\varphi}^{\prime} \\
        y_{\varphi}^{\prime} \\
        z_{\varphi}^{\prime} \\
    \end{pmatrix}
    =
    \begin{pmatrix}
        - \rho \cos \theta \sin \varphi \\
        \rho \cos \theta \cos \varphi   \\
        0
    \end{pmatrix}.
\end{gather*}
После нормировки получим:
\begin{gather*}
    u_{\theta}
    = \frac{U_\theta}{\norm{U_\theta}}
    = \begin{pmatrix}
          - \sin \theta \cos \varphi \\
          - \sin \theta \sin \varphi \\
          \cos \theta
    \end{pmatrix}, \\
%
    u_{\varphi}
    = \frac{U_{\varphi}}{\norm{U_{\varphi}}}
    = \begin{pmatrix}
          - \sin \varphi \\
          \cos \varphi   \\
          0
    \end{pmatrix}.
\end{gather*}

Со вторым каналом свяжем декартову систему $\widetilde{C} = (\widetilde{X}$, $\widetilde{Y}$, $\widetilde{Z})$ и соответствующую сферическую систему
$\widetilde{S} = (\widetilde{\rho}, \widetilde{\theta}, \widetilde{\varphi})$. В этих системах парциальная диаграмма направленности второго канала:
\[
    \widetilde{f}_2(\widetilde{w}) =
    \begin{pmatrix}
        \cos \widetilde{\theta} \\
        0
    \end{pmatrix}
\]
или
\[
    \widetilde{f}_{2,c}(\widetilde{w}) = \cos \widetilde{\theta} \cdot \widetilde{u}_{\widetilde{\theta}} + 0 \cdot \widetilde{u}_{\widetilde{\varphi}} ,
\]
где
\[
    \widetilde{u}_{\widetilde{\theta}}
    = \begin{pmatrix}
          - \sin \widetilde{\theta} \cos \widetilde{\varphi} \\
          - \sin \widetilde{\theta} \sin \widetilde{\varphi} \\
          \cos \widetilde{\theta}
    \end{pmatrix}.
\]
Координаты $\widetilde{x}$, $\widetilde{y}$ и $\widetilde{z}$ системы $\widetilde{C}$ связаны с координатами $x$, $y$, $z$ системы $C$ равенствами:
\begin{gather*}
    x = \widetilde{x} , \\
    y = - \widetilde{z} , \\
    z = \widetilde{y} .
\end{gather*}
Вектор $\widetilde{u}_{\widetilde{\theta}}$ в системе $С$ будет иметь координаты:
\[
    u_{\widetilde{\theta}}
    = \begin{pmatrix}
          - \sin \widetilde{\theta} \cos \widetilde{\varphi} \\
          - \cos \widetilde{\theta}                          \\
          - \sin \widetilde{\theta} \sin \widetilde{\varphi}
    \end{pmatrix}.
\]
И диаграмма направленности второго канала в декартовых координатах системы $C$:
\[
    f_{2,c}(\widetilde{w})
    = \cos \widetilde{\theta} \cdot u_{\widetilde{\theta}}
    = \begin{pmatrix}
          - \cos \widetilde{\theta} \sin \widetilde{\theta} \cos \widetilde{\varphi} \\
          - \cos^2 \widetilde{\theta}                                                \\
          - \cos \widetilde{\theta} \sin \widetilde{\theta} \sin \widetilde{\varphi}
    \end{pmatrix} .
\]
Теперь нужно заменить углы $\widetilde{\theta}$ и $\widetilde{\varphi}$ углами $\theta$ и $\varphi$.

Из соответствия координат систем $C$ и $\widetilde{C}$ получим равенства:
\begin{align*}
    \rho \cos \theta \cos \varphi & = \widetilde{\rho} \cos \widetilde{\theta} \cos \widetilde{\varphi} , \\
    \rho \cos \theta \sin \varphi & = - \widetilde{\rho} \sin \widetilde{\theta} , \\
    \rho \sin \theta              & = \widetilde{\rho} \cos \widetilde{\theta} \sin \widetilde{\varphi} .
\end{align*}
Учитывая что $\rho = \widetilde{\rho}$, получим равенства для углов:
\begin{align*}
    \cos \theta \cos \varphi & = \cos \widetilde{\theta} \cos \widetilde{\varphi} , \\
    \cos \theta \sin \varphi & = - \sin \widetilde{\theta} , \\
    \sin \theta              & = \cos \widetilde{\theta} \sin \widetilde{\varphi} .
\end{align*}
Откуда из перемножения левых и правых частей первого и второго равенств:
\[
    \cos^2 \theta \cos \varphi \sin \varphi = - \cos \widetilde{\theta} \sin \widetilde{\theta} \cos \widetilde{\varphi} ,
\]
из перемножения левых и правых частей второго и третьего равенств:
\[
    \cos \theta \sin \theta \sin \varphi = - \cos \widetilde{\theta} \sin \widetilde{\theta} \sin \widetilde{\varphi}, \\
\]
из возведения в квадрат левой и правой частей второго равенства:
\begin{gather*}
    \cos^2 \theta \sin^2 \varphi = \sin^2 \widetilde{\theta} , \\
    1 - \cos^2 \theta \sin^2 \varphi = 1 - \sin^2 \widetilde{\theta} , \\
    1 - \cos^2 \theta \sin^2 \varphi = \cos^2 \widetilde{\theta} .
\end{gather*}
Таким образом, диаграмма направленности второго канала в декартовых координатах:
\[
    f_{2,c}(w)
    = \begin{pmatrix}
          \cos^2 \theta \cos \varphi \sin \varphi \\
          \cos^2 \theta \sin^2 \varphi - 1        \\
          \cos \theta \sin \theta \sin \varphi
    \end{pmatrix}
\]
и для получения парциальной диаграммы направленности второго канала нужно вычислить проекции (скалярные произведения) на орты $u_\theta$ и $u_\varphi$:
\[
    f_2(w)
    = \begin{pmatrix}
          \scalarproduct{f_{2,c}(w)}{u_\theta} \\
          \scalarproduct{f_{2,c}(w)}{u_\varphi}
    \end{pmatrix} .
\]

Построение диаграмм в Matlab в файле arrows.m.

Вычисление наибольшего коэффициента усиления в Matlab в файле optimum.m.

\subsection{Коэффициент полезного действия}

\subsubsection{Энергетическое ограничение}

Для излучателя выполняется закон сохранения энергии --- суммарная мощность входных сигналов $P_{inp}$ совпадает с суммой мощности отраженных сигналов $P_{ref}$,
мощности излучения $P_{rad}$ и мощности $P_{dis}$ дисспативных потерь:
\[
    P_{ref} + P_{rad} + P_{dis} = P_{inp} .
\]
Исключение положительной мощности диссипативных потерь $P_{dis} >0$ приводит к неравенству:
\begin{equation}
    \label{emission:emitter:efficiency:power_inequality}
    P_{ref} + P_{rad} \le P_{inp} .
\end{equation}

Перенос мощности отраженных сигналов $P_{ref}$ в правую часть даёт неравенства:
\begin{gather*}
    P_{rad} \le P_{inp} - P_{ref}, \\
    P_{rad} \le P_{inp} - P_{ref} \le P_{inp}, \\
    \frac{P_{rad}}{P_{inp}} \le 1 - \frac{P_{ref}}{P_{inp}} \le 1 .
\end{gather*}

\subsubsection{КПД}
Коэффициентом полезного действия называется отношение мощности излучения ко входной мощности:
\[
    \eta
    = \frac{P_{rad}}{P_{inp}}.
\]
Легко видеть, что
\[
    \eta \le 1 - \frac{P_{ref}}{P_{inp}} = \eta_s ,
\]
и величина $\eta_s$ является верхней границей КПД $\eta$. Поскольку
\begin{align*}
    P_{inp} & = \norm{a}^2 = a^* a , \\
    P_{ref} & = \norm{b}^2 = \norm{S a}^2 = a^* S^* S a \notag,
\end{align*}
то
\[
    \eta_s(a) = 1 - \frac{a^* S^* S a}{a^* a},
\]
и ограничение для КПД можно получить из анализа матрицы рассеяния $S$.

Само значение КПД определяется мощностью излучения $P_{rad}$, которое представляет собой интеграл по сфере $S_R$ радиуса $R$ (позже выяснится, что величина $R$ значение не имеет)
от мощности электрического поля:
\[
    P_{rad}
    = \iint \limits_{S_R} \norm{E(w)}^2 ds ,
\]
Подставляя выражение \eqref{emission:emitter:diagram:tension} для вектора напряжённости $E(w)$, получим:
\begin{multline*}
    P_{rad}
    = \iint \limits_{S_R} \norm{F(w) a \cdot \frac{e^{i \modulus{w} R}}{R}}^2 ds
    = \frac{1}{R^2} \iint \limits_{S_R} \norm{F(w) a}^2 \modulus{\frac{e^{i \modulus{w} R}}{R}}^2 ds = \\
    %
    = \frac{1}{R^2} \iint \limits_{S_R} \norm{F(w) a}^2 \frac{\modulus{e^{i \modulus{w} R}}^2}{R^2} ds
    = \frac{1}{R^2} \iint \limits_{S_R} \norm{F(w) a}^2 ds
    = \frac{1}{R^2} \iint \limits_{S_R} a^* F^*(w) F(w) a ds = \\
    %
    = \frac{1}{R^2}
    \iint \limits_{S_R}
    \begin{pmatrix}
        a_1^* & a_2^*
    \end{pmatrix}
    \begin{pmatrix}
        f_1^*(w) f_1(w) & f_1^*(w) f_2(w) \\
        f_2^*(w) f_1(w) & f_2^*(w) f_2(w)
    \end{pmatrix}
    \begin{pmatrix}
        a_1 \\
        a_2
    \end{pmatrix}
    ds = \\
    %
    \shoveleft{= a_1^* \left( \frac{1}{R^2} \iint \limits_{S_R} f_1^*(w) f_1(w) ds \right) a_1 + a_1^* \left( \frac{1}{R^2} \iint \limits_{S_R} f_1^*(w) f_2(w) ds \right) a_2 +} \\
    \shoveright{+ a_2^* \left( \frac{1}{R^2} \iint \limits_{S_R} f_2^*(w) f_1(w) ds \right) a_1 + a_2^* \left( \frac{1}{R^2} \iint \limits_{S_R} f_2^*(w) f_2(w) ds \right) a_2 =} \\
    %
    = \begin{pmatrix}
          a_1^* & a_2^*
    \end{pmatrix}
    \begin{pmatrix}
        Q_{11} & Q_{12} \\
        Q_{21} & Q_{22}
    \end{pmatrix}
    \begin{pmatrix}
        a_1 \\
        a_2
    \end{pmatrix}
\end{multline*}
В итоге мощность излучения $P_{rad}$ можно представить в виде квадратичной формы:
\[
    P_{rad}
    = a^* Q a ,
\]
где элементами матрицы $Q$ являются интегралы вида:
\[
    Q_{jk} = \frac{1}{R^2} \iint \limits_{S_R} f_j^*(w) f_k(w) d s .
\]

Таким образом, КПД $\eta$:
\[
    \eta(a) = \frac{a^* Q a}{a^* a}
\]
является отношением Релея, поэтому его значения ограничены наименьшим $\eta_{min}$ и наибольшим $\eta_{max}$ собственными значениями матрицы $Q$:
\[
    \eta_{min} \le \eta(a) \le \eta_{max} .
\]
Поскольку $Q$ матрица порядка 2, то величины $\eta_{min}$ и $\eta_{max}$ можно найти согласно равенствам~\eqref{emission:emitter:gain:minimum_eigenvalue} и
\eqref{emission:emitter:gain:maximum_eigenvalue} (с матрицей $Q$ вместо матрицы $B$):
\begin{align}
    \eta_{min} & = \frac{\tr(Q) - \sqrt{\tr^2(Q) - 4 \det(Q)}}{2}
    \label{emission:emitter:efficiency:minimum}, \\
    %
    \eta_{max} & = \frac{\tr(Q) + \sqrt{\tr^2(Q) - 4 \det(Q)}}{2}
    \label{emission:emitter:efficiency:maximum} .
\end{align}

Оптимальный вектор огибающих $a_{max}$, при котором достигается наибольший КПД $\eta_{max}$ определяется из уравнения:
\begin{gather*}
    Q a_{max} = \eta_{max} a_{max} , \\
    (Q - \eta_{max} I ) a_{max} = 0.
\end{gather*}

\subsubsection{КПД каналов}

Представим, что вся мощность входных сигналов направлена в первый канал, то есть вектор огибающих
\[
    a
    = \begin{pmatrix}
          1 \\
          0
    \end{pmatrix} ,
\]
тогда величина КПД первого канала:
\[
    \eta_1
    = \frac{a^* Q a}{a^* a}
    = \frac{Q_{11}}{1}
    = Q_{11} .
\]
Аналогично, при:
\[
    a
    = \begin{pmatrix}
          0 \\
          1
    \end{pmatrix} ,
\]
тогда КПД второго канала:
\[
    \eta_2
    = \frac{a^* Q a}{a^* a}
    = \frac{Q_{22}}{1}
    = Q_{22} .
\]
Таким образом, элементы матрицы $Q$ показывают:
\begin{align*}
    \begin{array}{rcl}
        Q_{11} = \frac{1}{R^2} \iint \limits_{S_R} f_1^*(w) f_1(w) d s            & - & \text{КПД 1-го канала,}                  \\
        Q_{12} = Q_{21}^* = \frac{1}{R^2} \iint \limits_{S_R} f_1^*(w) f_2(w) d s & - & \text{коэффициент пересечения диаграмм,} \\
        Q_{22} = \frac{1}{R^2} \iint \limits_{S_R} f_2^*(w) f_2(w) d s            & - & \text{КПД 2-го канала.}
    \end{array}
\end{align*}

Оказывается, что КПД каналов $\eta_1$ и $\eta_2$ связаны с коэффициентом пересечения диаграмм. Представим элементы матрицы $Q$ в нормированном виде:
\[
    Q_{jk}
    =
    \sqrt{Q_{jj}}
    \cdot
    \frac{\iint \limits_{S_R} f_j^*(w) f_k(w) ds}{\sqrt{Q_{jj}} \sqrt{Q_{kk}}}
    \cdot
    \sqrt{Q_{kk}} ,
\]
тогда
\[
    Q = \sqrt{D} R \sqrt{D} ,
\]
где
\[
    \sqrt{D}
    = \begin{pmatrix}
          \sqrt{Q_{11}} & 0             \\
          0             & \sqrt{Q_{22}} \\
    \end{pmatrix} ,
    \;
    %
    R_{jk} = \frac{Q_{jk}}{\sqrt{Q_{jj}} \sqrt{Q_{kk}}} .
\]

Согласно неравенству~\eqref{emission:emitter:efficiency:power_inequality}:
\begin{align*}
    P_{rad}                   & \le P_{inp} , \\
    a^* Q a                   & \le a^* a, \\
    a^* \sqrt{D} R \sqrt{D} a & \le a^* a .
\end{align*}
Пусть $x = \sqrt{D} a$, тогда:
\begin{align*}
    x^* R x & \le x^* (\sqrt{D}^{-1})^* (\sqrt{D}^{-1}) x , \\
    x^* R x & \le x^* D^{-1} x .
\end{align*}
Пусть $r_{max}$ --- наибольшее собственное значение матрицы $R$ и $x_{max}$ --- соответствующий этому числу собственный вектор, а $Q_{min} = \min \{ Q_{11}, Q_{22} \}$,
тогда:
\begin{align*}
    x_{max}^* R x_{max} & \le x_{max}^* D^{-1} x_{max} , \\
    x_{max}^* r_{max} x_{max} & \le \sum_{k=1}^n \frac{1}{Q_{kk}} x_{max,k}^* x_{max,k} , \\
    r_{max} \norm{x_{max}}^2 & \le \sum_{k=1}^n \frac{1}{Q_{kk}} \modulus{x_{max,k}}^2 , \\
    r_{max} \norm{x_{max}}^2 & \le \sum_{k=1}^n \frac{1}{Q_{min}} \modulus{x_{max,k}}^2 , \\
    r_{max} \norm{x_{max}}^2 & \le \frac{1}{Q_{min}} \sum_{k=1}^n \modulus{x_{max,k}}^2 , \\
    r_{max} \norm{x_{max}}^2 & \le \frac{1}{Q_{min}} \norm{x_{max}}^2 , \\
    r_{max} & \le \frac{1}{Q_{min}} , \\
    Q_{min} & \le \frac{1}{r_{max}} .
\end{align*}

Матрица $R$ имеет вид:
\begin{gather}
    R
    = \begin{pmatrix}
          1        & R_{12} \\
          R_{12}^* & 1
    \end{pmatrix}
    \notag, \\
    %
    R_{12} = \frac{Q_{12}}{\sqrt{Q_{11}} \sqrt{Q_{22}}}
    \label{emission:emitter:efficiency:diagram_intersection}
\end{gather}

и наибольшее собственное значение матрицы $R$ согласно равенству~\eqref{emission:emitter:gain:maximum_eigenvalue} имеет вид:
\begin{multline*}
    r_{max}
    = \frac{\tr(R) + \sqrt{\tr^2(R) - 4 \det(R)}}{2} = \\
%
    = \frac{2 + \sqrt{2^2 - 4 (1 - \modulus{R_{12}}^2)}}{2}
    = \frac{2 + \sqrt{4 - 4 + 4 \modulus{R_{12}}^2}}{2} = \\
%
    = \frac{2 + 2 \modulus{R_{12}}}{2}
    = 1 + \modulus{R_{12}} .
\end{multline*}
Таким образом,
\begin{align*}
    \min \{ Q_{11}, Q_{22} \} & \le \frac{1}{1 + \modulus{R_{12}}} , \\
    \min \{ \eta_1, \eta_2 \} & \le \frac{1}{1 + \modulus{R_{12}}} .
\end{align*}

\subsubsection{Вычисление интегралов}

Возникает вопрос о вычислении интегралов вида:
\[
    P = \frac{1}{R^2} \iint \limits_{S_R} p(w) ds
\]
по сфере $S_R$ радиуса $R$.

Кратко, нужно перейти к угловым координатами $\varphi$ и $\theta$, в которых элемент поверхности $ds$ приближённо представляет собой прямоугольник со сторонами $R d\varphi$
и $R \cos \theta d\theta$:
\[
    ds = R d\varphi \cdot R \cos \theta d\theta = R^2 d\Omega,
\]
тогда поверхностый интеграл преобразуется к кратному интегралу и затем к повторному интегралу:
\[
    P
    = \iint \limits_{S_R} p(\varphi, \theta) d\Omega
    = \int \limits_0^{2 \pi} \left( \int \limits_{-\frac{\pi}{2}}^{\frac{\pi}{2}} p(\varphi, \theta) \cos \theta d\theta \right) d\varphi .
\]

Далее следует формальный вывод последнего выражения через вычисление исходного поверхностного интеграла в декартовых координатах и перехода к угловым координатам.

Для вычисления интеграла сфера разделяется на верхнюю $z \ge 0$ и нижнюю $z < 0$ полусферы:
\begin{gather*}
    x^2 + y^2 + z^2 = R^2 , \\
    z^2 = R^2 - x^2 - y^2 , \\
    z = \pm \sqrt{R^2 - x^2 - y^2} .
\end{gather*}
Интеграл сводится к сумме двух кратных интегралов в декартовых координатах:
\begin{multline*}
    P
    = \frac{1}{R^2} \iint
    \limits_{
        \begin{array}{c}
            x^2 + y^2 \le R^2, \\
            z = \sqrt{R^2 - x^2 - y^2}
        \end{array}
    } p(w) \sqrt{(z_x^\prime)^2 + (z_y^\prime)^2 + 1} dxdy + \\
    + \frac{1}{R^2} \iint
    \limits_{
        \begin{array}{c}
            x^2 + y^2 \le R^2, \\
            z = - \sqrt{R^2 - x^2 - y^2}
        \end{array}
    } p(w) \frac{1}{R^2} \sqrt{(z_x^\prime)^2 + (z_y^\prime)^2 + 1} dxdy ,
\end{multline*}
где производные под корнем:
\begin{gather*}
    z = \pm \sqrt{R^2 - x^2 - y^2} , \\
    z_x^\prime = \pm \frac{-2 x}{2 \sqrt{R^2 - x^2 - y^2}} , \\
    z_y^\prime = \pm \frac{-2 y}{2 \sqrt{R^2 - x^2 - y^2}} ,
\end{gather*}
и выражения под корнями в двух интегралах преобразуются одинаково:
\begin{multline*}
(z_x^\prime)
    ^2 + (z_y^\prime)^2 + 1
    = \frac{x^2}{R^2 - x^2 - y^2} + \frac{y^2}{R^2 - x^2 - y^2} + 1 = \\
    %
    = \frac{x^2 + y^2 + R^2 + x^2 + y^2}{R^2 - x^2 - y^2}
    = \frac{R^2}{R^2 - x^2 - y^2}
\end{multline*}
При переходе в кратных интегралах к угловым координатам:
\begin{gather*}
    x = R \cos \theta \cos \varphi , \\
    y = R \cos \theta \sin \varphi ,
\end{gather*}
выражение под корнем преобразуется далее:
\begin{gather*}
    \frac{R^2}{R^2 - x^2 - y^2}
    = \frac{R^2}{R^2 - R^2 \cos^2 \theta \cos^2 \varphi - R^2 \cos^2 \theta \sin^2 \varphi}
    = \frac{R^2}{R^2 - R^2 \cos^2 \theta}
    = \frac{1}{1 - \cos^2 \theta}
    = \frac{1}{\sin^2 \theta} , \\
    %
    \sqrt{\frac{1}{\sin^2 \theta}}
    = \frac{1}{\modulus{\sin \theta}} .
\end{gather*}
Якобиан пребразования для угловых координат
\[
    \frac{\partial (x, y)}{\partial (\theta, \varphi)}
    = \begin{vmatrix}
          - R \sin \theta \cos \varphi & - R \cos \theta \sin \varphi \\
          - R \sin \theta \sin \varphi & R \cos \theta \cos \varphi
    \end{vmatrix}
    = - R^2 \cos \theta \sin \theta .
\]
Таким образом,
\begin{multline*}
    P = \frac{1}{R^2} \int \limits_0^{2 \pi} \int \limits_0^\frac{\pi}{2} p(\varphi, \theta) \frac{1}{\modulus{\sin \theta}} \modulus{R^2 \cos \theta \sin \theta} d\theta d\varphi + \\
    \shoveright{+ \frac{1}{R^2} \int \limits_0^{2 \pi} \int \limits_{-\frac{\pi}{2}}^0 p(\varphi, \theta) \frac{1}{\modulus{\sin \theta}} \modulus{R^2 \cos \theta \sin \theta} d\theta d\varphi = } \\
    %
    = \int \limits_0^{2 \pi} \int \limits_0^\frac{\pi}{2} p(\varphi, \theta) \modulus{\cos \theta} d\theta d\varphi
    + \int \limits_0^{2 \pi} \int \limits_{-\frac{\pi}{2}}^0 p(\varphi, \theta) \modulus{\cos \theta} d\theta d\varphi = \\
    %
    = \int \limits_0^{2 \pi} \int \limits_{-\frac{\pi}{2}}^\frac{\pi}{2} p(\varphi, \theta) \cos \theta d\theta d\varphi .
\end{multline*}

\subsubsection{Пример 2: разная поляризация}

Пусть в направлении $w$ диаграмма направленности излучателя имеет вид:
\begin{gather*}
    F(w)
    = \begin{pmatrix}
          A_1 e^{i \alpha_1} \cos \theta & 0                              \\
          0                              & A_2 e^{i \alpha_2} \cos \theta
    \end{pmatrix} , \\
    %
    A_1 \ge A_2.
\end{gather*}
Вычислим элементы матрицы $Q$:
\begin{multline*}
    Q_{11}
    = \int \limits_0^{2 \pi} \int \limits_{-\frac{\pi}{2}}^\frac{\pi}{2} f_1^*(w) f_1(w) \cos \theta d\theta d\varphi .
    = \int \limits_0^{2 \pi} \int \limits_{-\frac{\pi}{2}}^\frac{\pi}{2} A_1 e^{-i \alpha_1} \cos \theta A_1 e^{i \alpha} \cos \theta \cos \theta d\theta d\varphi = \\
    %
    = \int \limits_0^{2 \pi} \int \limits_{-\frac{\pi}{2}}^\frac{\pi}{2} A_1^2 \cos^3 \theta d\theta d\varphi
    = A_1^2 \int \limits_0^{2 \pi} \int \limits_{-\frac{\pi}{2}}^\frac{\pi}{2} \cos^3 \theta d\theta d\varphi
    = A_1^2 2 \pi \int \limits_{-\frac{\pi}{2}}^\frac{\pi}{2} \cos^3 \theta d\theta = \\
    %
    = A_1^2 2 \pi \int \limits_{-\frac{\pi}{2}}^\frac{\pi}{2} (1 - \sin^2 \theta ) \cos \theta d\theta
    = A_1^2 2 \pi \left( \int \limits_{-\frac{\pi}{2}}^\frac{\pi}{2} \cos \theta d\theta - \int \limits_{-\frac{\pi}{2}}^\frac{\pi}{2} \sin^2 \theta \cos \theta d\theta \right) = \\
    %
    = A_1^2 2 \pi \left( \left. \sin \theta \right|_{-\frac{\pi}{2}}^\frac{\pi}{2} - \left. \frac{\sin^3 \theta}{3} \right|_{-\frac{\pi}{2}}^\frac{\pi}{2} \right)
    = A_1^2 2 \pi \left( 2 - \frac{2}{3} \right)
    = \frac{8}{3} \pi A_1^2 .
\end{multline*}
Аналогично:
\begin{gather*}
    Q_{22}
    = \int \limits_0^{2 \pi} \int \limits_{-\frac{\pi}{2}}^\frac{\pi}{2} A_2^2 \cos^3 \theta d\theta d\varphi
    = \frac{8}{3} \pi A_2^2 , \\
    %
    Q_{12}
    = \int \limits_0^{2 \pi} \int \limits_{-\frac{\pi}{2}}^\frac{\pi}{2} 0 \cdot \cos \theta d\theta d\varphi
    = 0 .
\end{gather*}
Таким образом,
\[
    Q
    = \frac{8}{3} \pi
    \begin{pmatrix}
        A_1^2 & 0     \\
        0     & A_2^2
    \end{pmatrix} .
\]
Минимальный $\eta_{min}$ и максимальный $\eta_{max}$ КПД определяются равенствами \eqref{emission:emitter:efficiency:minimum} и \eqref{emission:emitter:efficiency:maximum}, в которых
\begin{gather*}
    \tr(Q) = \frac{8}{3} \pi ( A_1^2 + A_2^2 ) , \\
    %
    \det(Q) = \left( \frac{8}{3} \pi \right)^2 A_1^2 A_2^2 .
\end{gather*}
Корень из дискриминанта:
\begin{multline*}
    \sqrt{\tr^2(Q) - 4 \det(Q)}
    = \frac{8}{3} \pi \sqrt{( A_1^2 + A_2^2 )^2 - 4 A_1^2 A_2^2}
    = \frac{8}{3} \pi \sqrt{A_1^4 + 2 A_1^2 A_2^2 + A_2^4 - 4 A_1^2 A_2^2} = \\
    %
    = \frac{8}{3} \pi \sqrt{A_1^4 - 2 A_1^2 A_2^2 + A_2^4}
    = \frac{8}{3} \pi \sqrt{( A_1^2 - A_2^2 )^2}
    = \frac{8}{3} \pi ( A_1^2 - A_2^2 ),
\end{multline*}
тогда
\begin{align*}
    \eta_{min} &
    = \frac{A_1^2 + A_2^2 - ( A_1^2 - A_2^2 )}{2} \cdot \frac{8}{3} \pi
    = \frac{2 A_2^2}{2} \cdot \frac{8}{3} \pi
    = A_2^2 \cdot \frac{8}{3} \pi , \\
    %
    \eta_{max} &
    = \frac{A_1^2 + A_2^2 + ( A_1^2 - A_2^2 )}{2} \cdot \frac{8}{3} \pi
    = \frac{2 A_1^2}{2} \cdot \frac{8}{3} \pi
    = A_1^2 \cdot \frac{8}{3} \pi . \\
\end{align*}

Коэффициент пересечения диаграмм \eqref{emission:emitter:efficiency:diagram_intersection}:
\begin{gather*}
    R_{12}
    = \frac{Q_{12}}{\sqrt{Q_{11}} \sqrt{Q_{22}}}
    = \frac{0}{A_1 \cdot A_2}
    = 0, \\
    %
    \modulus{R_{12}} = 0 .
\end{gather*}
Откуда
\[
    \min \{ \eta_1, \eta_2 \} \le \frac{1}{1 + \modulus{R_{12}}} = 1.
\]

\subsubsection{Пример 3: одинаковая поляризация}

Пусть в направлении $w$ диаграмма направленности излучателя имеет вид:
\[
    F(w)
    = \begin{pmatrix}
          A_1 e^{i \alpha_1} \cos \theta & A_2 e^{i \alpha_2} \cos \theta \\
          0                              & 0
    \end{pmatrix} .
\]
Вычисляем элементы матрицы $Q$:
\begin{gather*}
    Q_{11}
    = \int \limits_0^{2 \pi} \int \limits_{-\frac{\pi}{2}}^\frac{\pi}{2} A_1^2 \cos^3 \theta d\theta d\varphi
    = \frac{8}{3} \pi A_2^2 , \\
    %
    Q_{12}
    = \int \limits_0^{2 \pi} \int \limits_{-\frac{\pi}{2}}^\frac{\pi}{2} A_1 A_2 e^{i \alpha_1 - i \alpha_2} \cos^3 \theta d\theta d\varphi
    = \frac{8}{3} \pi A_1 A_2 e^{i(\alpha_1 - \alpha_2)} , \\
    %
    Q_{22}
    = \int \limits_0^{2 \pi} \int \limits_{-\frac{\pi}{2}}^\frac{\pi}{2} A_2^2 \cos^3 \theta d\theta d\varphi
    = \frac{8}{3} \pi A_2^2 .
\end{gather*}
Таким образом,
\[
    Q
    = \frac{8}{3} \pi
    \begin{pmatrix}
        A_1^2                              & A_1 A_2 e^{i(\alpha_1 - \alpha_2)} \\
        A_1 A_2 e^{i(\alpha_2 - \alpha_1)} & A_2^2
    \end{pmatrix} .
\]
След и определитель матрицы $Q$:
\begin{gather*}
    \tr(Q) = \frac{8}{3} \pi ( A_1^2 + A_2^2 ) , \\
    %
    \det(Q)
    = \left( \frac{8}{3} \pi \right)^2 \left( A_1^2 A_2^2 - A_1^2 A_2^2 \right)
    = 0 .
\end{gather*}
Корень из дискриминанта:
\[
    \sqrt{\tr^2(Q) - 4 \det(Q)}
    = \frac{8}{3} \pi \sqrt{( A_1^2 + A_2^2 )^2}
    = \frac{8}{3} \pi ( A_1^2 + A_2^2 ),
\]
тогда
\begin{align*}
    \eta_{min} &
    = \frac{A_1^2 + A_2^2 - ( A_1^2 + A_2^2 )}{2} \cdot \frac{8}{3} \pi
    = 0 , \\
    %
    \eta_{max} &
    = \frac{A_1^2 + A_2^2 + ( A_1^2 + A_2^2 )}{2} \cdot \frac{8}{3} \pi
    = \frac{2 (A_1^2 + A_2^2)}{2} \cdot \frac{8}{3} \pi
    = (A_1^2 + A_2^2) \cdot \frac{8}{3} \pi .
\end{align*}

Коэффициент пересечения диаграмм \eqref{emission:emitter:efficiency:diagram_intersection}:
\begin{gather*}
    R_{12}
    = \frac{Q_{12}}{\sqrt{Q_{11}} \sqrt{Q_{22}}}
    = \frac{A_1 A_2 e^{i(\alpha_1 - \alpha_2)}}{A_1 \cdot A_2}
    = e^{i(\alpha_1 - \alpha_2)}, \\
    %
    \modulus{R_{12}} = 1 .
\end{gather*}
Откуда
\[
    \min \{ \eta_1, \eta_2 \} \le \frac{1}{1 + \modulus{R_{12}}} = \frac{1}{1+1} = \frac{1}{2}
\]

\subsubsection{Пример 4: общий случай}

Пусть в направлении $w$ диаграмма направленности имеет вид:
\[
    F(w)
    =
    \begin{pmatrix}
        0.1 e^{i \frac{\pi}{6}} \cos \theta  & 0.3 e^{i \frac{5 \pi}{4}} \cos \theta   \\
        0.2 e^{i \frac{\pi}{3}} \sin \varphi & 0.1 e^{- i \frac{\pi}{10}} \sin \varphi
    \end{pmatrix} .
\]
Вычисления оптимальных векторов огибающих и поляризации смотри в файле Matlab \texttt{emission/two/efficiency.m}.


\section{Антенна}

\subsection{Модель антенны}

Рассматривается декартова система координат, в которой имеются $n$ излучателей, образующих антенну. Заданы местоположения излучателей $r_k$, и вектор огибающих
сигналов на входах излучателей:
\[
    a
    = \begin{pmatrix}
          a_1   \\\
          \dots \\\
          a_n
    \end{pmatrix} .
\]
Можно считать, что у каждого излучателя только один канал поляризации, в случае двух и более каналов необходимо в точку $r_k$ поместить ещё один или более
излучателей с другими поляризациями.

Напряженность электрического поля $E(w,R)$, создаваемого антенной, в дальней зоне будет приближённо равна:
\[
    E(w,R,a) = \widetilde{F}_a(w,a) \cdot \frac{e^{i \modulus{w} R}}{R} ,
\]
где $\widetilde{F}(w,a)$ --- диаграмма направленности антенны при огибающих $a$:
\[
    \widetilde{F}_a(w,a) = \sum_{k=1}^n f_k(w) a_k e^{i \scalarproduct{\vec{r}_k}{w}} ,
\]
где $f_k(w)$ --- парциальная диаграмма направленности $k$-го излучателя в составе антенны:
\[
    f_k(w) =
    \begin{pmatrix}
        f_{k,\theta}(w) \\
        f_{k,\varphi}(w)
    \end{pmatrix}
    ,
\]
и множитель $e^{i \scalarproduct{\vec{r}_k}{w}}$ соответствует смещению фазы при приведении напряжённостей поля излучателей к общему началу отсчёта.

Если парциальные диаграммы $f(w)$ и смещения фаз собрать в матрицу $f(w)$:
\begin{gather*}
    F_a(w) =
    \begin{pmatrix}
        f_{1,\theta}(w) e^{i \scalarproduct{\vec{r}_1}{w}}  & \dots & f_{n,\theta}(w) e^{i \scalarproduct{\vec{r}_n}{w}}  \\
        f_{1,\varphi}(w) e^{i \scalarproduct{\vec{r}_1}{w}} & \dots & f_{n,\varphi}(w) e^{i \scalarproduct{\vec{r}_n}{w}}
    \end{pmatrix} ,
\end{gather*}
тогда диаграмма направленности $\widetilde{F}_a(w,a)$ будет иметь вид:
\[
    \widetilde{F}_a(w,a) = F_a(w) a.
\]
Таким образом, напряженность имеет вид:
\[
    E = F_a(w) a \cdot \frac{e^{i \modulus{w} R}}{R} .
\]


\section{Диаграммообразующая схема}

Пусть $a_\alpha$ и $b_\alpha$ --- векторы огибающих сигналов сечения входа схемы и $a$ и $b$ --- векторы огибащих сигналов сечения выхода схемы. У схемы два входа ---
$a_\alpha$ и $b$ и два выхода --- $a$ и $b_\alpha$, которые связаны со входами:
\begin{gather}
    a        = S_{\beta \alpha} a_\alpha + S_{\beta \beta} b
    \label{emission:scheme:upper_output}, \\
    b_\alpha = S_{\alpha \alpha} a_\alpha + S_{\alpha \beta} b
    \label{emission:scheme:lower_output}
\end{gather}
Причем $a$ и $b$ --- огибающие сигналов в сечении входа антенны, для которых справедливо равенство:
\begin{equation}
    \label{emission:scheme:reflections}
    b = S_\beta a ,
\end{equation}
где $S_\beta$ --- матрица рассеяния антенны.

Согласно равенствам~\eqref{emission:scheme:upper_output} и~\eqref{emission:scheme:reflections} вектор огибающих входа антенны $a$:
\begin{align}
    a & = S_{\beta \alpha} a_\alpha + S_{\beta \beta} b , \notag \\
    a & = S_{\beta \alpha} a_\alpha + S_{\beta \beta} S_\beta a , \notag \\
    a - S_{\beta \beta} S_\beta a & = S_{\beta \alpha} a_\alpha , \notag \\
    ( I - S_{\beta \beta} S_\beta ) a & = S_{\beta \alpha} a_\alpha , \notag \\
    a & = ( I - S_{\beta \beta} S_\beta )^{-1} S_{\beta \alpha} a_\alpha , \label{emission:scheme:antenna_input}
\end{align}
тогда линейная часть диаграммы направленности антенны с диаграммообразующей схемой:
\[
    \widetilde{F}_s(w, a_\alpha)
    = F_a(w) a(a_\alpha)
    = F_a(w) ( I - S_{\beta \beta} S_\beta )^{-1} S_{\beta \alpha} a_\alpha
    = F_s(w) a_\alpha,
\]
где
\[
    F_s(w) = F_a(\vec{w}) ( I - S_{\beta \beta} S_\beta )^{-1} S_{\beta \alpha} .
\]

Из равенств~\eqref{emission:scheme:upper_output},~\eqref{emission:scheme:reflections} и~\eqref{emission:scheme:antenna_input} вектор огибающих выхода
диаграммообразующей схемы:
\begin{align*}
    b_\alpha & = S_{\alpha \alpha} a_\alpha + S_{\alpha \beta} b , \\
    b_\alpha & = S_{\alpha \alpha} a_\alpha + S_{\alpha \beta} S a , \\
    b_\alpha & = S_{\alpha \alpha} a_\alpha + S_{\alpha \beta} S ( I - S_{\beta \beta} S )^{-1} S_{\beta \alpha} a_\alpha , \\
    b_\alpha & = ( S_{\alpha \alpha} + S_{\alpha \beta} S ( I - S_{\beta \beta} S )^{-1} S_{\beta \alpha} ) a_\alpha ,
\end{align*}
откуда матрица рассеяния для диаграммообразующей схемы:
\[
    S_\alpha = S_{\alpha \alpha} + S_{\alpha \beta} S ( I - S_{\beta \beta} S )^{-1} S_{\beta \alpha} .
\]
