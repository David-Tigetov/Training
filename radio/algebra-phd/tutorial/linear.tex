\chapter{Стационарные линейные системы}


\section{Системы}

Дискретные сигналы будем представлять в виде бесконечных последовательностей $x$. Элемент дискретного сигнала $x$ с номером $k \in \mathbb{Z}$ будет обозначать $x[k]$.
Номера $k$ будет считать моментами дискретного времени.

На основе множества дискретных сигналов построим линейное пространство с поэлементным сложением сигналов и умножением на вещественные числа:
\[
    z = x + h \cdot y \Rightarrow z[k] = x[k] + h \cdot y[k].
\]

Дискретные сигналы будем преобразовывать с помощью операторов $\mathcal{L}$:
\[
    y = \mathcal{L}(x).
\]
Будем считать, что оператор соответствует действию системы, на вход которой последовательно подаются компоненты $x[k]$ сигнала $x$, и с выхода системы записываются компоненты
$y[k]$ сигнала $y$.

Далее рассматриваются только физически реализуемые системы, у которых на компонент $y[k]$ выходного сигнала могут влиять только компоненты $x[m]$, которые были получены системой
до момента $k$ включительно, то есть $m \le k$.

Рассматриваемые системы считаем стационарными: характеристики систем не изменяются во времени. Если входной сигнал $x$ сдвинуть на несколько моментов $s$ вправо или влево
на оси дискретного времени, то выходной сигнал системы также сдвинется на такое же количество моментов $s$:
\begin{align*}
    \widetilde{x}[k+s]              & = x[k] ,                        \\
    \mathcal{L}(\widetilde{x})[k+s] & = \mathcal{L}(\widetilde{x})[k]
\end{align*}

Предполагаем, что системы являются линейными:
\begin{gather*}
    \mathcal{L}(x + h \cdot y)
    = \mathcal{L}(x) + \mathcal{L}(h \cdot y)
    = \mathcal{L}(x) + h \cdot \mathcal{L}(y) , \\
    %
    \mathcal{L}(x + h \cdot y)[k]
    = \mathcal{L}(x)[k] + h \cdot \mathcal{L}(y)[k] .
\end{gather*}

\subsection{Декомпозиция}

Свойство линейности позволяет анализировать действие оператора $\mathcal{L}$ на "сложные"{} сигналы, если "сложный"{} сигнал $x$ разложить на "простые"{} сигналы $e_i$:
\[
    x = h_1 \cdot e_1 + \dots + h_n \cdot e_n ,
\]
тогда
\[
    \mathcal{L}(x) = h_1 \cdot \mathcal{L}(e_1) + \dots + h_n \cdot \mathcal{L}(e_n) .
\]

В качестве элементов $e_i$ можно выбирать единичные импульсы (но это не единственный возможный вариант). Рассмотрим сигнал $\delta$ в виде единичного импульса, помещенного в
момент времени 0:
\[
    \delta[k]
    = \left \{
    \begin{array}{rr}
        1 & \text{, если } k = 0 ,  \\
        0 & \text{, если } k \neq 0
    \end{array}
    \right .
\]
В качестве элементов $e_i$ выберем сдвинутые единичные импульсы:
\[
    e_i[k+i]
    = \delta[k] ,
\]
тогда любой сигнал $x$ можно представить в виде:
\[
    x
    = \sum_{i} x[i] e_i .
\]

\subsection{Импульсная характеристика}

Реакция системы на сигнал $x$ оказывается равной:
\[
    y
    = \mathcal{L}(x)
    = \mathcal{L} \left ( \sum_{i} x[i] e_i \right )
    = \sum_{i} \mathcal{L} \left ( x[i] e_i \right )
    = \sum_{i} x[i] \mathcal{L} \left ( e_i \right )
\]
Значит необходимо выяснить выходные сигналы системы при подаче на вход системы импульсов $e_i$:
\[
    \mathcal{L} \left ( e_i \right )[k+i] = \mathcal{L}(\delta)[k]
\]
и эти выходные сигналы оказываются сдвигами выходного сигнала системы, если на вход системы подаётся единичный импульс $\delta$. Обозначим этот сигнал $h$:
\[
    h = \mathcal{L}(\delta).
\]
Сигнал $h$ называется импульсной характеристикой системы $\mathcal{L}$.

Таким образом, элемент $s$ выходного сигнала системы при входном сигнале $x$ представляется суммой:
\begin{equation}
    ~\label{linear:impulse:convolution}
    y[s]
    = \mathcal{L}(x)[s]
    = \sum_{i} x[i] \mathcal{L} \left ( e_i \right )[s]
    = \sum_{i} x[i] \mathcal{L} \left ( \delta \right )[s-i]
    = \sum_{i} x[i] h[s-i] .
\end{equation}
Выходной сигнал представляет собой свёртку входного сигнала $x$ и импульсной характеристики $h$ системы $\mathcal{L}$.

\subsection{Свёртка}

В силу свойства физической реализуемости системы $\mathcal{L}$:
\[
    k < 0 : h[k] = 0 .
\]
Будем дополнительно считать, что импульсная характеристика является конечной, то есть существует число $n>0$:
\[
    k > n-1 : h[k] = 0 ,
\]
тогда в свёртке \eqref{linear:impulse:convolution} останется конечное число слагаемых, поскольку одновременно должны выполняться равенства:
\begin{gather*}
    0 \le s - i \le n-1 , \\
    -(n - 1) \le i - s \le 0 , \\
    s - (n - 1) \le i \le s .
\end{gather*}
Таким образом,
\[
    y[s]
    = \mathcal{L}(x)[s]
    = \sum_{i=s-(n-1)}^s x[i] h[s-i] .
\]

\textcolor{red}{Рисунок: перевёрнутая импульсная характеристика "едет"{} вдоль сигнала.}

Выпишем компонент $y[n-1]$:
\begin{multline*}
    y[n-1]
    = \mathcal{L}(x)[n-1]
    = \sum_{i=0}^{n-1} x[i] h[s-i] = \\
    %
    = x[0] h[n-1] + x[1] h[n-2] + x[2] h[n-3] + \dots + x[n-2] h[1] + x[n-1] h[0] .
\end{multline*}
Можно представить в виде умножения матрицы и столбца:
\[
    \begin{pmatrix}
        \vdots \\
        y[n-1]
    \end{pmatrix}
    = \begin{pmatrix}
        \vdots & \vdots & \vdots & \vdots & \vdots & \vdots \\
        h[n-1] & h[n-2] & h[n-3] & \dots  & h[1]   & h[0]
    \end{pmatrix}
    \begin{pmatrix}
        x[0]   \\
        x[1]   \\
        x[2]   \\
        \vdots \\
        x[n-2] \\
        x[n-1]
    \end{pmatrix}
\]

Выпишем компонент $y[n-2]$:
\begin{multline*}
    y[n-2]
    = \mathcal{L}(x)[n-2]
    = \sum_{i=-1}^{n-2} x[i] h[s-i] = \\
    %
    = x[-1] h[n-1] + x[0] h[n-2] + x[1] h[n-3] + \dots + x[n-3] h[1] + x[n-2] h[0] .
\end{multline*}
Появился компонент $x[-1]$. Поскольку далее будут использоваться периодические дискретные сигналы, то будем считать, что на отрезке от 0 до $n-1$ помещается натуральное количество
периодов сигнала $x$ (число $n$ всегда можно увеличить до ближайшей границы периода справа) и для всех $s$ выполняется
\[
    x[s] = x[s+n] ,
\]
тогда в силу принятой периодичности:
\[
    x[-1] = x[-1+n] = x[n-1].
\]
И в выражении для $y[n-2]$ можно заменить $x[-1]$:
\[
    y[n-2]
    = x[n-1] h[n-1] + x[0] h[n-2] + x[1] h[n-3] + \dots + x[n-3] h[1] + x[n-2] h[0] ,
\]
тогда в матричной форме:
\[
    \begin{pmatrix}
        \vdots \\
        y[n-2] \\
        y[n-1]
    \end{pmatrix}
    = \begin{pmatrix}
        \vdots & \vdots & \vdots & \vdots & \vdots & \vdots \\
        h[n-2] & h[n-3] & h[n-4] & \dots  & h[0]   & h[n-1] \\
        h[n-1] & h[n-2] & h[n-3] & \dots  & h[1]   & h[0]
    \end{pmatrix}
    \begin{pmatrix}
        x[0]   \\
        x[1]   \\
        x[2]   \\
        \vdots \\
        x[n-2] \\
        x[n-1]
    \end{pmatrix}
\]
Продолжая аналогичным образом, с учётом периодичности сигнала $x$, получим:
\[
    \begin{pmatrix}
        y[0]   \\
        y[1]   \\
        y[2]   \\
        \vdots \\
        y[n-2] \\
        y[n-1]
    \end{pmatrix}
    = \begin{pmatrix}
        h[0]   & h[n-1] & h[n-2] & \dots  & h[2]   & h[1]   \\
        h[1]   & h[0]   & h[n-1] & \dots  & h[3]   & h[2]   \\
        h[2]   & h[1]   & h[0]   & \dots  & h[4]   & h[3]   \\
        \vdots & \vdots & \vdots & \vdots & \vdots & \vdots \\
        h[n-2] & h[n-3] & h[n-4] & \dots  & h[0]   & h[n-1] \\
        h[n-1] & h[n-2] & h[n-3] & \dots  & h[1]   & h[0]
    \end{pmatrix}
    \begin{pmatrix}
        x[0]   \\
        x[1]   \\
        x[2]   \\
        \vdots \\
        x[n-2] \\
        x[n-1] \\
    \end{pmatrix} .
\]

\subsection{Циркулянты}

Далее в силу принятой периодичности сигналов будем рассматривать не сами сигналы (бесконечные последовательности), а векторы длины $n$. Введём обозначения для значений сигналов:
\begin{gather*}
    y_k = y[k] , \\
    h_k = h[k] , \\
    x_k = x[k]
\end{gather*}
и векторов:
\[
    y
    = \begin{pmatrix}
        y_0    \\
        y_1    \\
        \vdots \\
        y_{n-1}
    \end{pmatrix}
    , \;
    h
    = \begin{pmatrix}
        h_0    \\
        h_1    \\
        \vdots \\
        h_{n-1}
    \end{pmatrix}
    , \;
    x
    = \begin{pmatrix}
        x_0    \\
        x_1    \\
        \vdots \\
        x_{n-1}
    \end{pmatrix}
    .
\]
Таким образом,
\begin{equation}
    \label{linear:transformation}
    y = C(h) x ,
\end{equation}
где матрица $C(h)$:
\begin{equation}
    \label{linear:circulant}
    C(h)
    = \begin{pmatrix}
        h_0     & h_{n-1} & h_{n-2} & \dots  & h_2    & h_1     \\
        h_1     & h_0     & h_{n-1} & \dots  & h_3    & h_2     \\
        h_2     & h_1     & h_{0}   & \dots  & h_4    & h_3     \\
        \vdots  & \vdots  & \vdots  & \vdots & \vdots & \vdots  \\
        h_{n-2} & h_{n-3} & h_{n-4} & \dots  & h_0    & h_{n-1} \\
        h_{n-1} & h_{n-2} & h_{n-3} & \dots  & h_1    & h_0
    \end{pmatrix}
    .
\end{equation}
Строки матрицы $C(h)$ образованы циклическим сдвигом первой строки и столбцы образованы циклическим сдвигом первого столбца. Такие матрицы называются циркулянтными матрицами или просто
циркулянтами.

У циркулянтов имеется интересное свойство. Пусть $\delta \in \mathbb{C}$ --- комплексный корень из единицы степени $n$:
\[
    \delta^n = 1 .
\]
Образуем строку из степеней числа $\delta$ и умножим на циркулянт $C(h)$ слева:
\[
    \begin{pmatrix}
        \delta^0 & \delta^1 & \delta^2 & \dots & \delta^{n-1}
    \end{pmatrix}
    \begin{pmatrix}
        h_0     & h_{n-1} & h_{n-2} & \dots  & h_2    & h_1     \\
        h_1     & h_0     & h_{n}   & \dots  & h_3    & h_2     \\
        h_2     & h_1     & h_{0}   & \dots  & h_4    & h_3     \\
        \vdots  & \vdots  & \vdots  & \vdots & \vdots & \vdots  \\
        h_{n-2} & h_{n-3} & h_{n-4} & \dots  & h_0    & h_{n-1} \\
        h_{n-1} & h_{n-2} & h_{n-3} & \dots  & h_1    & h_0
    \end{pmatrix}
\]

В результате умножения строки на первый столбец $C(h)$ получим выражение:
\[
    \begin{pmatrix}
        \delta^0 & \delta^1 & \delta^2 & \dots & \delta^{n-1}
    \end{pmatrix}
    \begin{pmatrix}
        h_0   \\
        h_1   \\
        h_2   \\
        \dots \\
        h_{n-1}
    \end{pmatrix}
    =
    h_0 \delta^0 + h_1 \delta^1 + h_2 \delta^2 + \dots + h_{n-1} \delta^{n-1} = \lambda(\delta) ,
\]
представляющее собой значение полинома $\lambda(z)$ при $z = \delta$:
\[
    \lambda(z) = h_0 z^0 + h_1 z^1 + h_2 z^2 + \dots + h_{n-1} z^{n-1} .
\]
При умножении строки на второй столбец получим:
\begin{multline*}
    \begin{pmatrix}
        \delta^0 & \delta^1 & \delta^2 & \dots & \delta^{n-1}
    \end{pmatrix}
    \begin{pmatrix}
        h_{n-1} \\
        h_0     \\
        h_1     \\
        h_2     \\
        \dots   \\
        h_{n-2}
    \end{pmatrix} = \\
    %
    = h_{n-1} \delta^0 + h_0 \delta^1 + h_1 \delta^2 + \dots + h_{n-2} \delta^{n-1} = \\
    %
    = h_{n-1} 1 + h_0 \delta^1 + h_1 \delta^2 + \dots + h_{n-2} \delta^{n-1} = \\
    %
    = h_{n-1} \delta^n + h_0 \delta^1 + h_1 \delta^2 + \dots + h_{n-2} \delta^{n-1} = \\
    %
    = h_0 \delta^1 + h_1 \delta^2 + \dots + h_{n-2} \delta^{n-1} + h_{n-1} \delta^n = \\
    %
    = \delta \left( h_0 \delta^0 + h_1 \delta^1 + \dots + h_{n-2} \delta^{n-2} + h_{n-1} \delta^{n-1} \right) = \\
    %
    = \delta \lambda(\delta) .
\end{multline*}

Аналогичным образом получаются результаты умножения строки на все остальные столбцы матрицы $C$, в итоге получим:
\begin{multline*}
    \begin{pmatrix}
        \delta^0 & \delta^1 & \delta^2 & \dots & \delta^{n-1}
    \end{pmatrix}
    \begin{pmatrix}
        h_0     & h_{n-1} & h_{n-2} & \dots  & h_3    & h_2    & h_1    \\
        h_1     & h_0     & h_{n-1} & \dots  & h_4    & h_3    & h_2    \\
        h_2     & h_1     & h_0     & \dots  & h_5    & h_4    & h_3    \\
        \vdots  & \vdots  & \vdots  & \ddots & \vdots & \vdots & \vdots \\
        h_{n-1} & h_{n-2} & h_{n-3} & \dots  & h_2    & h_1    & h_0
    \end{pmatrix} = \\
    %
    = \begin{pmatrix}
        \lambda(\delta) & \delta \lambda(\delta) & \delta^2 \lambda(\delta) & \dots & \delta^{n-1} \lambda(\delta)
    \end{pmatrix} = \\
    %
    = \lambda(\delta)
    \begin{pmatrix}
        \delta^0 & \delta^1 & \delta^2 & \dots & \delta^{n-1}
    \end{pmatrix}
    .
\end{multline*}


\section{Дискретное преобразование Фурье}

\subsection{Базис}

Ранее для представления сигналов использовались единичные импульсы $e_i$:
\[
    e_i[k]
    =
    \left\{
    \begin{array}{ll}
        1, & k = i    \\
        0, & k \neq i
    \end{array}
    \right.
    ,
\]
которые образовывали базис, по которому раскладывались сигналы. Это не единственный вариант базиса, есть альтернативные варианты, и этих вариантов достаточно много.

Поскольку теперь рассматриваются периодические сигналы, которые повторяются через каждые $n$ моментов времени, то вполне допустимым было бы ввести базис, образованный периодическими
сигналами. Например, можно определить сигнал $\xi_1$, у которого на протяжении $n$ моментов укладывается один период гармонического колебания:
\[
    x_1[k]
    = \cos \frac{2 \pi}{n} k
\]
Далее будет удобно работать с комплексными величинами, поэтому дополним сигнал мнимой частью:
\[
    z_1[k]
    = \cos \frac{2 \pi}{n} k + i \sin \frac{2 \pi}{n} k
    = e^{i \frac{2 \pi}{n} k} .
\]
и заметим, что все значения сигнала являются степенями одного числа $\varepsilon$:
\begin{gather*}
    z_1[k]
    = \left( e^{i \frac{2 \pi}{n}} \right)^k
    = \varepsilon^k, \\
    %
    \varepsilon = e^{i \frac{2 \pi}{n}} .
\end{gather*}

Соберём первые $n$ значений сигнала $z_1$ в вектор:
\[
    \xi_1
    = \begin{pmatrix}
        \varepsilon^0 \\
        \varepsilon^1 \\
        \varepsilon^2 \\
        \vdots        \\
        \varepsilon^{n-1}
    \end{pmatrix} .
\]

Возьмем теперь сигнал, у которого на протяжении $n$ моментов времени укладывается два периода гармонических колебаний:
\begin{gather*}
    x_2[k]
    = \cos \frac{2 \cdot 2 \pi}{n} k , \\
    %
    z_2[k]
    = \cos \frac{2 \cdot 2 \pi}{n} k + i \sin \frac{2 \cdot 2 \pi}{n} k
    = e^{i \frac{2 \cdot 2 \pi}{n} k}, \\
    %
    z_2[k]
    = \left( e^{i \frac{2 \cdot 2 \pi}{n}} \right)^k
    = \left( \left( e^{i \frac{2 \pi}{n}} \right)^2 \right)^k
    = \left( \varepsilon^2 \right)^k
\end{gather*}
Первые $n$ значений соберем в вектор
\[
    \xi_2
    = \begin{pmatrix}
        \varepsilon^{2 \cdot 0} \\
        \varepsilon^{2 \cdot 1} \\
        \varepsilon^{2 \cdot 2} \\
        \vdots                  \\
        \varepsilon^{2 \cdot (n-1)}
    \end{pmatrix}
\]

Аналогично, можно рассматривать сигнал, у которого на протяжении $n$ моментов времени укладывается $m$ периодов гармонических колебаний:
\begin{gather*}
    x_m[k]
    = \cos \frac{m \cdot 2 \pi}{n} k , \\
    %
    z_m[k]
    = \cos \frac{m \cdot 2 \pi}{n} k + i \sin \frac{m \cdot 2 \pi}{n} k
    = e^{i \frac{m \cdot 2 \pi}{n} k}, \\
    %
    z_m[k]
    = \left( e^{i \frac{m \cdot 2 \pi}{n}} \right)^k
    = \left( \left( e^{i \frac{2 \pi}{n}} \right)^m \right)^k
    = \left( \varepsilon^m \right)^k
\end{gather*}
и сформируем вектор
\[
    \xi_m
    = \begin{pmatrix}
        \varepsilon^{m \cdot 0} \\
        \varepsilon^{m \cdot 1} \\
        \varepsilon^{m \cdot 2} \\
        \vdots                  \\
        \varepsilon^{m \cdot (n-1)}
    \end{pmatrix} .
\]

До каких пор можно увеличивать $m$? Оказывается, что при $m = n$:
\[
    x_n[k]
    = \cos \frac{n \cdot 2 \pi}{n} k
    = \cos 2 \pi k
    = 1
    = \cos 0 \cdot k
    = \cos \frac{0 \cdot 2 \pi}{n} k
    = x_0[k]
    , \\
\]
поэтому вектор
\[
    \xi_n
    = \begin{pmatrix}
        1      \\
        1      \\
        1      \\
        \vdots \\
        1
    \end{pmatrix}
    = \xi_0.
\]
Если же $m > n$, то сигналы повторяются:
\begin{multline*}
    x_m[k]
    = \cos \frac{m \cdot 2 \pi}{n} k
    = \cos \frac{((m-n) + n) \cdot 2 \pi}{n} k = \\
    %
    = \cos \left( \frac{(m-n) \cdot 2 \pi}{n} + 2 \pi  \right) k
    = \cos \frac{(m-n) \cdot 2 \pi}{n} k
    = x_{m-n}[k].
\end{multline*}
Таким образом, мы исчерпали все возможные варианты и получили набор векторов $\xi_m$, образованные степенями $\varepsilon^{m \cdot k}$:
\[
    \xi_0
    = \begin{pmatrix}
        \varepsilon^{0 \cdot 0} \\
        \varepsilon^{0 \cdot 1} \\
        \varepsilon^{0 \cdot 2} \\
        \vdots                  \\
        \varepsilon^{0 \cdot (n-1)}
    \end{pmatrix} ,
    \;
    %
    \xi_1
    = \begin{pmatrix}
        \varepsilon^{1 \cdot 0} \\
        \varepsilon^{1 \cdot 1} \\
        \varepsilon^{1 \cdot 2} \\
        \vdots                  \\
        \varepsilon^{1 \cdot (n-1)}
    \end{pmatrix} ,
    \;
    \xi_2
    = \begin{pmatrix}
        \varepsilon^{2 \cdot 0} \\
        \varepsilon^{2 \cdot 1} \\
        \varepsilon^{2 \cdot 2} \\
        \vdots                  \\
        \varepsilon^{2 \cdot (n-1)}
    \end{pmatrix} ,
    \;
    \dots ,
    \;
    \xi_{n-1}
    = \begin{pmatrix}
        \varepsilon^{(n-1) \cdot 0} \\
        \varepsilon^{(n-1) \cdot 1} \\
        \varepsilon^{(n-1) \cdot 2} \\
        \vdots                      \\
        \varepsilon^{(n-1) \cdot (n-1)}
    \end{pmatrix} .
\]
Векторы $\xi_m$ можно рассматривать как элементы линейного пространства $\mathbb{C}^n$, в котором введём скалярное прозведение:
\[
    \scalarproduct{x}{y} = y^* x .
\]
Вычислим скалярное произведение двух векторов:
\[
    \scalarproduct{\xi_j}{\xi_k}
    = \xi_k^* \xi_j
    = \sum_{l=0}^{n-1} \overline{\varepsilon^{k \cdot l}} \varepsilon^{j \cdot l}
\]
Заметим, что для любой степени $p$
\begin{equation}
    ~\label{linear:Fourier:conjugation}
    \overline{\varepsilon^{p}}
    = \frac{\overline{\varepsilon^{p}} \varepsilon^{p}}{\varepsilon^{p}}
    = \frac{\modulus{\varepsilon^{p}}^2}{\varepsilon^{p}}
    = \frac{\left ( \modulus{\varepsilon}^{p} \right )^2}{\varepsilon^{p}}
    = \frac{\left ( 1^{p} \right )^2}{\varepsilon^{p}}
    = \frac{1}{\varepsilon^{p}}
    = \varepsilon^{- p}
    .
\end{equation}
Таким образом,
\[
    \scalarproduct{\xi_j}{\xi_k}
    = \sum_{l=0}^{n-1} \varepsilon^{- k \cdot l} \varepsilon^{j \cdot l}
    = \sum_{l=0}^{n-1} \varepsilon^{(j - k) \cdot l}
\]
Если $j = k$, то сразу получается норма векторов $\xi_j$, порождаемая скалярным произведением:
\[
    \norm{\xi_j}^2
    = \scalarproduct{\xi_j}{\xi_j}
    = \sum_{l=0}^{n-1} \varepsilon^{( j - j ) \cdot l}
    = \sum_{l=0}^{n-1} \varepsilon^0
    = \sum_{j=0}^{n-1} 1
    = n .
\]
Если $j \neq k$, то в сумме, стоящей справа легко увидеть геометрическую прогрессию со знаменателем $\varepsilon^{j - k}$, её сумма:
\[
    \scalarproduct{\xi_j}{\xi_k}
    = \frac{1 - \left( \varepsilon^{(j-k)} \right)^{n-1+1}}{1 - \varepsilon^{j - k}}
    = \frac{1 - \left( \varepsilon^{(j-k)} \right)^n}{1 - \varepsilon^{j - k}}
    = \frac{1 - \left( \varepsilon^n \right)^{j-k}}{1 - \varepsilon^{j - k}}
    = \frac{1 - 1^{j-k}}{1 - \varepsilon^{j - k}}
    = \frac{1 - 1}{1 - \varepsilon^{j - k}}
    = 0 ,
\]
поскольку
\[
    \varepsilon^n
    = \left( e^{- i \frac{2 \pi}{n}} \right)^n
    = e^{- i \frac{2 \pi}{n} \cdot n}
    = e^{- i 2 \pi}
    = 1.
\]

\subsection{Спектр}

Таким образом, векторы $\xi_j$ является ортогональными, поэтому они являются линейно независимыми. Их количество $n$ совпадает с размерностью пространства $\mathbb{C}^n$,
поэтому векторы $\xi_j$ образуют базис в пространстве $\mathbb{C}^n$ и любой вектор $x \in \mathbb{C}^n$ можно разложить по базису $\xi_j$:
\[
    x = \sum_{j=0}^{n-1} X_j \xi_j.
\]
Координаты $X_k$ легко вычислить, поскольку векторы $\xi_k$ ортогональны, то нужно вычислить скалярное произведение:
\[
    \scalarproduct{x}{\xi_k}
    = \scalarproduct{\sum_{j=0}^{n-1} X_j \xi_j}{\xi_k}
    = \sum_{j=0}^{n-1} X_j \scalarproduct{\xi_j}{\xi_k}
    = X_k \scalarproduct{\xi_k}{\xi_k}
    = X_k \norm{\xi_k}^2
    = X_k n
    .
\]
Вектор $X = \left ( n X_0, n X_1, \dots, n X_{n-1} \right )$ называют спектром сигнала $x$.

\subsection{Прямое преобразование}

Вектор $X$ можно получить умножением на матрицу $F_n$:
\[
    X
    = \begin{pmatrix}
        \scalarproduct{x}{\xi_0}     \\
        \scalarproduct{x}{\xi_1}     \\
        \vdots                       \\
        \scalarproduct{x}{\xi_{n-1}} \\
    \end{pmatrix}
    = \begin{pmatrix}
        \xi_0^* x     \\
        \xi_1^* x     \\
        \vdots        \\
        \xi_{n-1}^* x \\
    \end{pmatrix}
    = F_n x ,
\]
где
\begin{multline*}
    F_n =
    \begin{pmatrix}
        \xi_0^* \\
        \xi_1^* \\
        \dots   \\
        \xi_{n-1}^*
    \end{pmatrix}
    =
    \begin{pmatrix}
        \overline{\varepsilon^{0 \cdot 0}}     & \overline{\varepsilon^{0 \cdot 1}}     & \overline{\varepsilon^{0 \cdot 2}}     & \dots  & \overline{\varepsilon^{0 \cdot (n-1)}}     \\
        \overline{\varepsilon^{1 \cdot 0}}     & \overline{\varepsilon^{1 \cdot 1}}     & \overline{\varepsilon^{1 \cdot 2}}     & \dots  & \overline{\varepsilon^{1 \cdot (n-1)}}     \\
        \vdots                                 & \vdots                                 & \vdots                                 & \ddots & \vdots                                     \\
        \overline{\varepsilon^{(n-1) \cdot 0}} & \overline{\varepsilon^{(n-1) \cdot 1}} & \overline{\varepsilon^{(n-1) \cdot 2}} & \dots  & \overline{\varepsilon^{(n-1) \cdot (n-1)}} \\
    \end{pmatrix} = \\
    %
    = \begin{pmatrix}
        \varepsilon^{- 0 \cdot 0}     & \varepsilon^{- 0 \cdot 1}     & \varepsilon^{- 0 \cdot 2}     & \dots  & \varepsilon^{- 0 \cdot (n-1)}     \\
        \varepsilon^{- 1 \cdot 0}     & \varepsilon^{- 1 \cdot 1}     & \varepsilon^{- 1 \cdot 2}     & \dots  & \varepsilon^{- 1 \cdot (n-1)}     \\
        \vdots                        & \vdots                        & \vdots                        & \ddots & \vdots                            \\
        \varepsilon^{- (n-1) \cdot 0} & \varepsilon^{- (n-1) \cdot 1} & \varepsilon^{- (n-1) \cdot 2} & \dots  & \varepsilon^{- (n-1) \cdot (n-1)} \\
    \end{pmatrix} = \\
    %
    = \begin{pmatrix}
        1      & 1                             & 1                             & \dots  & 1                                 \\
        1      & \varepsilon^{- 1 \cdot 1}     & \varepsilon^{- 2 \cdot 1}     & \dots  & \varepsilon^{- (n-1) \cdot 1}     \\
        \vdots & \vdots                        & \vdots                        & \ddots & \vdots                            \\
        1      & \varepsilon^{- 1 \cdot (n-1)} & \varepsilon^{- 2 \cdot (n-1)} & \dots  & \varepsilon^{- (n-1) \cdot (n-1)}
    \end{pmatrix}
    ,
\end{multline*}
последнее равенство получено в силу сопряжения \eqref{linear:Fourier:conjugation}.

Матрица $F_n$ является матрицей Фурье (матрицей прямого дискретного преобразования Фурье).

\subsection{Обратное преобразование}

В силу ортогональности векторов $\xi_k$ для матрицы $F_n$ правой обратной матрицей будет $\frac{1}{n} F_n^*$:
\[
    \frac{1}{n} F_n^*
    =
    \frac{1}{n}
    \begin{pmatrix}
        \xi_0 & \xi_1 & \dots & \xi_{n-1}
    \end{pmatrix}
    =
    \frac{1}{n}
    \begin{pmatrix}
        \varepsilon^{0 \cdot 0}     & \varepsilon^{1 \cdot 0}     & \dots  & \varepsilon^{(n-1) \cdot 0}     \\
        \varepsilon^{0 \cdot 1}     & \varepsilon^{1 \cdot 1}     & \dots  & \varepsilon^{(n-1) \cdot 1}     \\
        \vdots                      & \vdots                      & \ddots & \vdots                          \\
        \varepsilon^{0 \cdot (n-1)} & \varepsilon^{1 \cdot (n-1)} & \dots  & \varepsilon^{(n-1) \cdot (n-1)}
    \end{pmatrix}
    ,
\]
действительно:
\begin{multline*}
    F_n \frac{1}{n} F_n^*
    =
    \begin{pmatrix}
        \xi_0^* \\
        \xi_1^* \\
        \dots   \\
        \xi_{n-1}^*
    \end{pmatrix}
    \frac{1}{n}
    \begin{pmatrix}
        \xi_0 & \xi_1 & \dots & \xi_{n-1}
    \end{pmatrix} = \\
    %
    = \frac{1}{n}
    \begin{pmatrix}
        \scalarproduct{\xi_0}{\xi_0}     & \scalarproduct{\xi_0}{\xi_1}     & \dots  & \scalarproduct{\xi_0}{\xi_{n-1}}     \\
        \scalarproduct{\xi_1}{\xi_0}     & \scalarproduct{\xi_1}{\xi_1}     & \dots  & \scalarproduct{\xi_1}{\xi_{n-1}}     \\
        \vdots                           & \vdots                           & \ddots & \vdots                               \\
        \scalarproduct{\xi_{n-1}}{\xi_0} & \scalarproduct{\xi_{n-1}}{\xi_1} & \dots  & \scalarproduct{\xi_{n-1}}{\xi_{n-1}}
    \end{pmatrix}
    = \frac{1}{n}
    \begin{pmatrix}
        \norm{\xi_0}^2 & 0              & \dots  & 0              \\
        0              & \norm{\xi_1}^2 & \dots  & 0              \\
        \vdots         & \vdots         & \ddots & \vdots         \\
        0              & 0              & \dots  & \norm{\xi_n}^2
    \end{pmatrix}
    = \\
    %
    = \frac{1}{n}
    \begin{pmatrix}
        n      & 0      & \dots  & 0      \\
        0      & n      & \dots  & 0      \\
        \vdots & \vdots & \ddots & \vdots \\
        0      & 0      & \dots  & n
    \end{pmatrix}
    =
    \begin{pmatrix}
        1      & 0      & \dots  & 0      \\
        0      & 1      & \dots  & 0      \\
        \vdots & \vdots & \ddots & \vdots \\
        0      & 0      & \dots  & 1
    \end{pmatrix}
    .
\end{multline*}

Известно, что правая обратная матрица является и левой обратной.

\subsection{Преобразование свёртки}

Теперь обратим внимание, что строки матрицы прямого преобразования Фурье $F_n$ являются векторами $\xi_k^*$:
\[
    \xi_k^*
    = \begin{pmatrix}
        \varepsilon^{- 0 \cdot k} & \varepsilon^{- 1 \cdot k} & \varepsilon^{- 2 \cdot k} & \dots & \varepsilon^{- (n-1) \cdot k}
    \end{pmatrix}
\]
которые образованы степенями корней из единицы $\varepsilon^{-k}$ (действительно, $\left ( \varepsilon^{-k} \right )^n = \left ( \varepsilon^n \right )^{-k} = 1^{-k} = 1$).

Пусть $h$ --- произвольный вектор и $C(h)$ --- циркулянт \eqref{linear:circulant}, тогда в результате умножения одной строки матрицы Фурье $F_n$:
\[
    \xi_k^* C(h)
    = \lambda ( \varepsilon^{-k} ) \begin{pmatrix}
        \varepsilon^{- 0 \cdot k} & \varepsilon^{- 1 \cdot k} & \varepsilon^{- 2 \cdot k} & \dots & \varepsilon^{- (n-1) \cdot k}
    \end{pmatrix}
    = \lambda ( \varepsilon^{-k} ) \xi_k^*
\]
откуда для произведения
\begin{multline*}
    F_n C(h)
    =
    \begin{pmatrix}
        \lambda ( \varepsilon^{-0} ) \xi_0^* \\
        \lambda ( \varepsilon^{-1} ) \xi_1^* \\
        \lambda ( \varepsilon^{-2} ) \xi_2^* \\
        \dots                                \\
        \lambda ( \varepsilon^{-(n-1)} ) \xi_{n-1}^*
    \end{pmatrix} = \\
    %
    =
    \begin{pmatrix}
        \lambda ( \varepsilon^{-0} ) & 0                            & 0                            & \dots  & 0                                \\
        0                            & \lambda ( \varepsilon^{-1} ) & 0                            & \dots  & 0                                \\
        0                            & 0                            & \lambda ( \varepsilon^{-2} ) & \dots  & 0                                \\
        \vdots                       & \vdots                       & \ddots                       & \ddots & 0                                \\
        0                            & 0                            & 0                            & \dots  & \lambda ( \varepsilon^{-(n-1)} )
    \end{pmatrix}
    \begin{pmatrix}
        \xi_0^* \\
        \xi_1^* \\
        \xi_2^* \\
        \dots   \\
        \xi_{n-1}^*
    \end{pmatrix}
    = \\
    %
    =
    \begin{pmatrix}
        \xi_0^* h & 0         & 0         & \dots  & 0             \\
        0         & \xi_1^* h & 0         & \dots  & 0             \\
        0         & 0         & \xi_2^* h & \dots  & 0             \\
        \vdots    & \vdots    & \ddots    & \ddots & 0             \\
        0         & 0         & 0         & \dots  & \xi_{n-1}^* h
    \end{pmatrix}
    \begin{pmatrix}
        \xi_0^* \\
        \xi_1^* \\
        \xi_2^* \\
        \dots   \\
        \xi_{n-1}^*
    \end{pmatrix}
\end{multline*}
поскольку
\[
    \lambda ( \varepsilon^{-k} )
    = h_0 \varepsilon^{- 0 \cdot k} + h_1 \varepsilon^{- 1 \cdot k} + h_2 \varepsilon^{- 2 \cdot k} + \dots + h_{n-1} \varepsilon^{- (n-1) \cdot k} .
\]

Далее легко видеть, что на диагонали располагаются элементы спектра $H$ вектора $h$:
\[
    H
    =
    \begin{pmatrix}
        \xi_0^* h \\
        \xi_1^* h \\
        \xi_2^* h \\
        \dots     \\
        \xi_{n-1}^* h
    \end{pmatrix}
    =
    \begin{pmatrix}
        \xi_0^* \\
        \xi_1^* \\
        \xi_2^* \\
        \dots   \\
        \xi_{n-1}^*
    \end{pmatrix}
    h
    = F_n h
\]
Таким образом,
\[
    F_n C(h) = diag \left( H \right) F_n.
\]
где $diag \left( H \right)$ обозначает диагональную матрицу с элементами $H_k$ на главной диагонали.

Пусть теперь некоторый сигнал $x$ со спектром $X$ преобразуется в сигнал $y$ линейной системой с импульсной характеристикой $h$ \eqref{linear:transformation}:
\[
    y = C(h) x ,
\]
тогда спектр $Y$ сигнала $y$:
\[
    Y
    = F_n y
    = F_n C(h) x
    = diag \left( H \right) F_n x
    = diag \left( H \right) X,
\]
или
\[
    Y_k = H_k \cdot X_k .
\]
Используя спектр $Y$, с помощью обратного преобразования получим сигнал $y$:
\[
    y = \frac{1}{n} F_n^* Y .
\]

Таким образом, для вычисления выходного сигнала $y$ существует два способа:
\begin{enumerate}
    \item прямое вычисление:
          \[
              y = C(h) x,
          \]
    \item через вычисление спектров:
          \begin{gather*}
              H = F_n h, \\
              X = F_n x, \\
              Y_k = H_k \cdot X_k, \\
              y = \frac{1}{n} F_n^* Y_k .
          \end{gather*}
\end{enumerate}
Второй способ требует больше вычислительных операций, но в силу особой структуры матрицы Фурье для вычисления спектров есть быстрый метод, поэтому в некоторых случаях второй способ
оказывается быстрее первого.


\section{Быстрое преобразование Фурье}

Легко видеть, что прямое и обратное преобразование Фурье связаны с матрицами вида:
\[
    F_n(\delta)
    = \begin{pmatrix}
        1      & 1            & 1               & 1               & \dots  & 1                   \\
        1      & \delta       & \delta^2        & \delta^3        & \dots  & \delta^{n-1}        \\
        1      & \delta^2     & \delta^4        & \delta^6        & \dots  & \delta^{2(n-1)}     \\
        1      & \delta^3     & \delta^6        & \delta^9        & \dots  & \delta^{3(n-1)}     \\
        \vdots & \vdots       & \vdots          & \vdots          & \ddots & \vdots              \\
        1      & \delta^{n-1} & \delta^{2(n-1)} & \delta^{3(n-1)} & \dots  & \delta^{(n-1)(n-1)}
    \end{pmatrix}.
\]
В случае умножения на матрицу $F_n(\delta)$ требуется примерно $(n-1) \cdot (n-1)$ умножений, но в силу особого вида матриц $F_n(\delta)$ количество умножений можно сократить,
если $n$ --- составное число. Можно показать, что если $n = k \cdot m$, то преобразование $F_n(\delta)$ разлагается в суперпозицию:
\begin{enumerate}
    \item $m$ преобразований $F_k(\delta^m)$ порядка $k$,
    \item $m-1$ умножение $D_{m-1}$,
    \item $k$ преобразований $F_m(\delta^k)$ порядка $m$.
\end{enumerate}
Символически суперпозицию можно записать в виде:
\[
    F_n(\delta) = m \cdot F_k(\delta^m) \otimes D_{m-1} \otimes k \cdot F_m(\delta^k)
\]
Пусть $S_*(p)$ обозначает количество умножений в преобразовании $F_p(\delta)$, тогда количество умножений в левой части:
\[
    S_*(n)
    = (n-1) \cdot (n-1)
    = n^2 - 2 n + 1
    = k^2 m^2 - 2 k m + 1
    = k m ( k m - 2 ) + 1,
\]
а в правой части:
\begin{multline*}
    m \cdot S_*(k) + m-1 + k \cdot S_*(m) = \\
    %
    = m \cdot (k-1) \cdot (k-1) + m-1 + k \cdot (m-1) \cdot (m-1) = \\
    %
    = m (k^2 - 2 k + 1) + m-1 + k (m^2 - 2 m + 1) = \\
    %
    = m k^2 - 2 m k + 1 + m-1 + k m^2 - 2 k m + k = \\
    %
    = m k^2 + k m^2 - 4 k m + m + k = \\
    %
    = k m ( k + m - 4 ) + m + k .
\end{multline*}
Уменьшение количества умножений:
\[
    \frac{k m ( k m - 2 ) + 1}{k m ( k + m - 4 ) + m + k}
    \approx \frac{k m ( k m - 2 )}{k m ( k + m - 4 )}
    \approx \frac{k m}{k + m} .
\]

\subsection{6 и 3, с таблицами}

$H$ --- спектр сигнала $h$:
\begin{gather*}
    H = F_6(\delta) h , \\
    %
    \begin{pmatrix}
        H_0 \\
        H_1 \\
        H_2 \\
        H_3 \\
        H_4 \\
        H_5
    \end{pmatrix}
    = \begin{pmatrix}
        1 & 1        & 1           & 1           & 1           & 1           \\
        1 & \delta^1 & \delta^2    & \delta^3    & \delta^4    & \delta^5    \\
        1 & \delta^2 & \delta^4    & \delta^6    & \delta^8    & \delta^{10} \\
        1 & \delta^3 & \delta^6    & \delta^9    & \delta^{12} & \delta^{15} \\
        1 & \delta^4 & \delta^8    & \delta^{12} & \delta^{16} & \delta^{20} \\
        1 & \delta^5 & \delta^{10} & \delta^{15} & \delta^{20} & \delta^{25} \\
    \end{pmatrix}
    \begin{pmatrix}
        h_0 \\
        h_1 \\
        h_2 \\
        h_3 \\
        h_4 \\
        h_5
    \end{pmatrix}
\end{gather*}

Нумеруем строки от $0$, \dots, $5$, группируем строки по остаткам: сперва строки с остатком 0, потом строки с остатком 1, и так далее. Перестановка строк приводит к перестановке
элементов вектора спектра $H$:
\[
    \begin{pmatrix}
        H_0 \\
        H_3 \\
        H_1 \\
        H_4 \\
        H_2 \\
        H_5
    \end{pmatrix}
    =
    \underbrace{
        \begin{pmatrix}
            1 & 1        & 1           & 1           & 1           & 1           \\
            1 & \delta^3 & \delta^6    & \delta^9    & \delta^{12} & \delta^{15} \\
            1 & \delta^1 & \delta^2    & \delta^3    & \delta^4    & \delta^5    \\
            1 & \delta^4 & \delta^8    & \delta^{12} & \delta^{16} & \delta^{20} \\
            1 & \delta^2 & \delta^4    & \delta^6    & \delta^8    & \delta^{10} \\
            1 & \delta^5 & \delta^{10} & \delta^{15} & \delta^{20} & \delta^{25} \\
        \end{pmatrix}
    }_{\widetilde{F}_6}
    \begin{pmatrix}
        h_0 \\
        h_1 \\
        h_2 \\
        h_3 \\
        h_4 \\
        h_5
    \end{pmatrix}
\]
Выделение блоков, сокращение степени, вынесение общих множителей в строках:
\begin{multline*}
    \widetilde{F}_6
    =
    \begin{pmatrix}
        \begin{pmatrix}
            1 & 1        \\
            1 & \delta^3
        \end{pmatrix}
         &
        \begin{pmatrix}
            1        & 1        \\
            \delta^6 & \delta^9
        \end{pmatrix}
         &
        \begin{pmatrix}
            1           & 1           \\
            \delta^{12} & \delta^{15}
        \end{pmatrix}
        \\
        \begin{pmatrix}
            1 & 1        \\
            1 & \delta^3
        \end{pmatrix}
        \begin{pmatrix}
            1 & 0      \\
            0 & \delta
        \end{pmatrix}
         &
        \begin{pmatrix}
            1        & 1        \\
            \delta^6 & \delta^9
        \end{pmatrix}
        \begin{pmatrix}
            \delta^2 & 0        \\
            0        & \delta^3
        \end{pmatrix}
         &
        \begin{pmatrix}
            1           & 1           \\
            \delta^{12} & \delta^{15}
        \end{pmatrix}
        \begin{pmatrix}
            \delta^4 & 0        \\
            0        & \delta^5
        \end{pmatrix}
        \\
        \begin{pmatrix}
            1 & 1        \\
            1 & \delta^3
        \end{pmatrix}
        \begin{pmatrix}
            1 & 0        \\
            0 & \delta^2
        \end{pmatrix}
         &
        \begin{pmatrix}
            1        & 1        \\
            \delta^6 & \delta^9
        \end{pmatrix}
        \begin{pmatrix}
            \delta^4 & 0        \\
            0        & \delta^6
        \end{pmatrix}
         &
        \begin{pmatrix}
            1           & 1           \\
            \delta^{12} & \delta^{15}
        \end{pmatrix}
        \begin{pmatrix}
            \delta^8 & 0           \\
            0        & \delta^{10}
        \end{pmatrix}
    \end{pmatrix} = \\
    %
    =
    \begin{pmatrix}
        \begin{pmatrix}
            1 & 1        \\
            1 & \delta^3
        \end{pmatrix}
         &
        \begin{pmatrix}
            1 & 1        \\
            1 & \delta^3
        \end{pmatrix}
         &
        \begin{pmatrix}
            1 & 1        \\
            1 & \delta^3
        \end{pmatrix}
        \\
        \begin{pmatrix}
            1 & 1        \\
            1 & \delta^3
        \end{pmatrix}
        \begin{pmatrix}
            1 & 0      \\
            0 & \delta
        \end{pmatrix}
         &
        \begin{pmatrix}
            1 & 1        \\
            1 & \delta^3
        \end{pmatrix}
        \begin{pmatrix}
            \delta^2 & 0        \\
            0        & \delta^3
        \end{pmatrix}
         &
        \begin{pmatrix}
            1 & 1        \\
            1 & \delta^3
        \end{pmatrix}
        \begin{pmatrix}
            \delta^4 & 0        \\
            0        & \delta^5
        \end{pmatrix}
        \\
        \begin{pmatrix}
            1 & 1        \\
            1 & \delta^3
        \end{pmatrix}
        \begin{pmatrix}
            1 & 0        \\
            0 & \delta^2
        \end{pmatrix}
         &
        \begin{pmatrix}
            1 & 1        \\
            1 & \delta^3
        \end{pmatrix}
        \begin{pmatrix}
            \delta^4 & 0        \\
            0        & \delta^6
        \end{pmatrix}
         &
        \begin{pmatrix}
            1 & 1        \\
            1 & \delta^3
        \end{pmatrix}
        \begin{pmatrix}
            \delta^8 & 0           \\
            0        & \delta^{10}
        \end{pmatrix}
    \end{pmatrix} = \\
    %
    =
    \begin{pmatrix}
        \begin{pmatrix}
            1 & 1        \\
            1 & \delta^3
        \end{pmatrix}
         &
        0
         &
        0
        \\
        0
         &
        \begin{pmatrix}
            1 & 1        \\
            1 & \delta^3
        \end{pmatrix}
         &
        0
        \\
        0
         &
        0
         &
        \begin{pmatrix}
            1 & 1        \\
            1 & \delta^3
        \end{pmatrix}
    \end{pmatrix}
    \begin{pmatrix}
        1 & 0        & 1        & 0        & 1        & 0           \\
        0 & 1        & 0        & 1        & 0        & 1           \\
        1 & 0        & \delta^2 & 0        & \delta^4 & 0           \\
        0 & \delta   & 0        & \delta^3 & 0        & \delta^5    \\
        1 & 0        & \delta^4 & 0        & \delta^8 & 0           \\
        0 & \delta^2 & 0        & \delta^6 & 0        & \delta^{10} \\
    \end{pmatrix} = \\
    %
    =
    \begin{pmatrix}
        F_2(\delta^3) & 0             & 0             \\
        0             & F_2(\delta^3) & 0             \\
        0             & 0             & F_2(\delta^3)
    \end{pmatrix}
    \begin{pmatrix}
        1 & 0        & 1        & 0        & 1        & 0           \\
        0 & 1        & 0        & 1        & 0        & 1           \\
        1 & 0        & \delta^2 & 0        & \delta^4 & 0           \\
        0 & \delta   & 0        & \delta^3 & 0        & \delta^5    \\
        1 & 0        & \delta^4 & 0        & \delta^8 & 0           \\
        0 & \delta^2 & 0        & \delta^6 & 0        & \delta^{10} \\
    \end{pmatrix} ,
\end{multline*}
где $F_2(\delta^3)$ --- матрица Фурье размера 2:
\[
    F_2(\delta^3)
    = \begin{pmatrix}
        1 & 1        \\
        1 & \delta^3
    \end{pmatrix}
\]
Таким образом,
\[
    \begin{pmatrix}
        H_0 \\
        H_3 \\
        H_1 \\
        H_4 \\
        H_2 \\
        H_5
    \end{pmatrix}
    =
    \begin{pmatrix}
        F_2(\delta^3) & 0             & 0             \\
        0             & F_2(\delta^3) & 0             \\
        0             & 0             & F_2(\delta^3)
    \end{pmatrix}
    \begin{pmatrix}
        1 & 0        & 1        & 0        & 1        & 0           \\
        0 & 1        & 0        & 1        & 0        & 1           \\
        1 & 0        & \delta^2 & 0        & \delta^4 & 0           \\
        0 & \delta   & 0        & \delta^3 & 0        & \delta^5    \\
        1 & 0        & \delta^4 & 0        & \delta^8 & 0           \\
        0 & \delta^2 & 0        & \delta^6 & 0        & \delta^{10} \\
    \end{pmatrix}
    \begin{pmatrix}
        h_0 \\
        h_1 \\
        h_2 \\
        h_3 \\
        h_4 \\
        h_5
    \end{pmatrix}
\]
Пусть вектор справа
\[
    \begin{pmatrix}
        \widetilde{H}_0 \\
        \widetilde{H}_1 \\
        \widetilde{H}_2 \\
        \widetilde{H}_3 \\
        \widetilde{H}_4 \\
        \widetilde{H}_5
    \end{pmatrix}
    =
    \begin{pmatrix}
        1 & 0        & 1        & 0        & 1        & 0           \\
        0 & 1        & 0        & 1        & 0        & 1           \\
        1 & 0        & \delta^2 & 0        & \delta^4 & 0           \\
        0 & \delta   & 0        & \delta^3 & 0        & \delta^5    \\
        1 & 0        & \delta^4 & 0        & \delta^8 & 0           \\
        0 & \delta^2 & 0        & \delta^6 & 0        & \delta^{10} \\
    \end{pmatrix}
    \begin{pmatrix}
        h_0 \\
        h_1 \\
        h_2 \\
        h_3 \\
        h_4 \\
        h_5
    \end{pmatrix}
\]
тогда
\[
    \begin{pmatrix}
        H_0 & H_1 & H_2 \\
        H_3 & H_4 & H_5
    \end{pmatrix}
    =
    F_2(\delta^3)
    \begin{pmatrix}
        \widetilde{H}_0 & \widetilde{H}_2 & \widetilde{H}_4 \\
        \widetilde{H}_1 & \widetilde{H}_3 & \widetilde{H}_5
    \end{pmatrix} .
\]
Вектор слева записан по строкам, а вектор справа по столбцам. Столбцы вектора слева являются преобразованиями Фурье векторов справа.

Нам осталось вычислить вектор $\widetilde{H}$. В его вычислении тоже можно увидеть преобразования Фурье, если переставить строки
\[
    \begin{pmatrix}
        \widetilde{H}_0 \\
        \widetilde{H}_2 \\
        \widetilde{H}_4 \\
        \widetilde{H}_1 \\
        \widetilde{H}_3 \\
        \widetilde{H}_5
    \end{pmatrix}
    =
    \begin{pmatrix}
        1 & 0        & 1        & 0        & 1        & 0           \\
        1 & 0        & \delta^2 & 0        & \delta^4 & 0           \\
        1 & 0        & \delta^4 & 0        & \delta^8 & 0           \\
        0 & 1        & 0        & 1        & 0        & 1           \\
        0 & \delta   & 0        & \delta^3 & 0        & \delta^5    \\
        0 & \delta^2 & 0        & \delta^6 & 0        & \delta^{10} \\
    \end{pmatrix}
    \begin{pmatrix}
        h_0 \\
        h_1 \\
        h_2 \\
        h_3 \\
        h_4 \\
        h_5
    \end{pmatrix}
\]
и переставить столбцы:
\[
    \begin{pmatrix}
        \widetilde{H}_0 \\
        \widetilde{H}_2 \\
        \widetilde{H}_4 \\
        \widetilde{H}_1 \\
        \widetilde{H}_3 \\
        \widetilde{H}_5
    \end{pmatrix}
    =
    \begin{pmatrix}
        1 & 1        & 1        & 0        & 0        & 0           \\
        1 & \delta^2 & \delta^4 & 0        & 0        & 0           \\
        1 & \delta^4 & \delta^8 & 0        & 0        & 0           \\
        0 & 0        & 0        & 1        & 1        & 1           \\
        0 & 0        & 0        & \delta   & \delta^3 & \delta^5    \\
        0 & 0        & 0        & \delta^2 & \delta^6 & \delta^{10} \\
    \end{pmatrix}
    \begin{pmatrix}
        h_0 \\
        h_2 \\
        h_4 \\
        h_1 \\
        h_3 \\
        h_5
    \end{pmatrix} .
\]
Выносим множители из последних двух строк:
\[
    \begin{pmatrix}
        \widetilde{H}_0 \\
        \widetilde{H}_2 \\
        \widetilde{H}_4 \\
        \widetilde{H}_1 \\
        \widetilde{H}_3 \\
        \widetilde{H}_5
    \end{pmatrix}
    =
    \begin{pmatrix}
        1 & 0 & 0 & 0 & 0      & 0        \\
        0 & 1 & 0 & 0 & 0      & 0        \\
        0 & 0 & 1 & 0 & 0      & 0        \\
        0 & 0 & 0 & 1 & 0      & 0        \\
        0 & 0 & 0 & 0 & \delta & 0        \\
        0 & 0 & 0 & 0 & 0      & \delta^2
    \end{pmatrix}
    \begin{pmatrix}
        1 & 1        & 1        & 0 & 0        & 0        \\
        1 & \delta^2 & \delta^4 & 0 & 0        & 0        \\
        1 & \delta^4 & \delta^8 & 0 & 0        & 0        \\
        0 & 0        & 0        & 1 & 1        & 1        \\
        0 & 0        & 0        & 1 & \delta^2 & \delta^4 \\
        0 & 0        & 0        & 1 & \delta^4 & \delta^8 \\
    \end{pmatrix}
    \begin{pmatrix}
        h_0 \\
        h_2 \\
        h_4 \\
        h_1 \\
        h_3 \\
        h_5
    \end{pmatrix} .
\]
Видно, что в матрице справа получились два одинаковых блока, которая являются матрицами Фурье размера 3:
\[
    F_3(\delta^2)
    = \begin{pmatrix}
        1 & 1        & 1        \\
        1 & \delta^2 & \delta^4 \\
        1 & \delta^4 & \delta^8 \\
    \end{pmatrix} .
\]
Таким образом,
\[
    \begin{pmatrix}
        \widetilde{H}_0 & \widetilde{H}_1 \\
        \widetilde{H}_2 & \widetilde{H}_3 \\
        \widetilde{H}_4 & \widetilde{H}_5
    \end{pmatrix}
    =
    \begin{pmatrix}
        1 & 1        \\
        1 & \delta   \\
        1 & \delta^2
    \end{pmatrix}
    \circ
    F_3(\delta^2)
    \begin{pmatrix}
        h_0 & h_1 \\
        h_2 & h_3 \\
        h_4 & h_5
    \end{pmatrix} ,
\]
где $\circ$ обозначает поэлементное умножение. Транспонируем для того, чтобы получилось как в правой части вычисления спектра:
\[
    \begin{pmatrix}
        \widetilde{H}_0 & \widetilde{H}_2 & \widetilde{H}_4 \\
        \widetilde{H}_1 & \widetilde{H}_3 & \widetilde{H}_5
    \end{pmatrix}
    =
    \left (
    \begin{pmatrix}
        1 & 1        \\
        1 & \delta   \\
        1 & \delta^2
    \end{pmatrix}
    \circ
    F_3(\delta^2)
    \begin{pmatrix}
        h_0 & h_1 \\
        h_2 & h_3 \\
        h_4 & h_5
    \end{pmatrix}
    \right )^T ,
\]
и всё вместе:
\[
    \begin{pmatrix}
        H_0 & H_1 & H_2 \\
        H_3 & H_4 & H_5
    \end{pmatrix}
    =
    F_2(\delta^3)
    \left (
    \begin{pmatrix}
        1 & 1        \\
        1 & \delta   \\
        1 & \delta^2
    \end{pmatrix}
    \circ
    F_3(\delta^2)
    \begin{pmatrix}
        h_0 & h_1 \\
        h_2 & h_3 \\
        h_4 & h_5
    \end{pmatrix}
    \right )^T .
\]

Полученное выражение можно обобщить на случай $n=2 \cdot m$:
\[
    \begin{pmatrix}
        H_0 & H_1     & H_2     & \dots & H_{m-1}  \\
        H_m & H_{m+1} & H_{m+2} & \dots & H_{2m-1}
    \end{pmatrix}
    =
    F_2(\delta^{m})
    \left (
    \begin{pmatrix}
        1      & 1        \\
        1      & \delta   \\
        1      & \delta^2 \\
        \vdots & \vdots   \\
        1      & \delta^{m-1}
    \end{pmatrix}
    \circ
    F_m(\delta^2)
    \begin{pmatrix}
        h_0        & h_1      \\
        h_2        & h_3      \\
        h_4        & h_5      \\
        \vdots     & \vdots   \\
        h_{2m - 2} & h_{2m-1}
    \end{pmatrix}
    \right )^T ,
\]
где
\begin{gather*}
    \delta^m
    = \delta^\frac{n}{2}
    = \left ( e^{i \frac{2 \pi}{n}}\right )^\frac{n}{2}
    = e^{i \pi}
    = -1 , \\
    %
    F_2(\delta^m)
    = \begin{pmatrix}
        1 & 1 \\
        1 & \delta^m
    \end{pmatrix}
    = \begin{pmatrix}
        1 & 1 \\
        1 & -1
    \end{pmatrix} ,
\end{gather*}
и преобразование $F_2(\delta^m)$ состоит всего из двух операций: одного сложения и одного вычитания. Таким образом, для преобразование $F_n(\delta)$
порядка $n$ нужно сделать два преобразования $F_m(\delta^2)$ порядка $m$, выполнить $m-1$ умножение, для простоты будем считать $m$ умножений, и 
умножить вектор размера $2 \times m$ на матрицу $F_2(\delta^m)$, выполнив одно сложение и одно вычитание на каждый из $m$ столбцов. Таким образом,
если $S_*(n)$ --- количество умножений в преобразовании $F_n(\delta)$ и $S_\pm(n)$ --- количество сложений-вычитаний в преобразовании $F_n(\delta)$,
то 
\begin{gather*}
    S_*(n)
    = m + 2 S_*(m)
    = \frac{n}{2} + 2 S_* \left( \frac{n}{2} \right) , \\
    %
    S_\pm(n)
    = 2 m + 2 S_\pm(m)
    = n + 2 S_\pm \left( \frac{n}{2} \right)
\end{gather*}
Таким образом, при редукции задачи размера $n$ нужно сделать $\frac{n}{2}$ умножений и $n$ сложений-вычитаний и решить две задачи размера $\frac{n}{2}$.

Пусть теперь $n = 2^L$, тогда $L = \log_2 n$, тогда можно продолжить цепочку равенств:
\begin{gather*}
    %
    S_*(n)
    = \frac{n}{2} + 2 \frac{n}{2 \cdot 2} + 4 \frac{n}{4 \cdot 2} + \dots + 2^{L-1} \frac{n}{2^{L-1} \cdot 2}
    = \frac{n}{2} + \frac{n}{2} + \frac{n}{2} + \dots + \frac{n}{2}
    = \frac{n}{2} \cdot L
    = \frac{1}{2} n \log_2 n , \\
\end{gather*}

Выражение для $S_*(n)$ получается следующим образом: для редукции задачи размера $n$ нужно сделать $\frac{n}{2}$ умножений,
в результате получатся 2 задачи размера $\frac{n}{2}$, каждая из которых при редукции потребует
$\frac{\frac{n}{2}}{2}$ умножений, поэтому для редукции двух задач размера $\frac{n}{2}$ потребуется $2 \frac{n}{2 \cdot 2}$ умножений,
далее образуется уже 4 задачи размера $\frac{n}{4}$, редукция которых требует $4 \frac{n}{4 \cdot 2}$ умножений и так далее пока не достигнем размера
задачи 2, таких задач будет $2^{L-1}$ и в каждой будет $\frac{n}{2^{L-1} \cdot 2} = \frac{2^L}{2^{L-1} \cdot 2} = 1$ умножение. 

Аналогично для $S_\pm(n)$: редукция задачи размера $n$ требует $n$ сложений-вычитаний, получаются 2 задачи, редукция каждой из которых требует
$\frac{n}{2}$ сложений-вычитаний, далее получаются 4 задачи, редукция каждой из которых требует $\frac{n}{4}$ сложений-вычитаний, и так далее пока не будет
достигнут размер задачи 4, таких задач $2^{L-2}$ и каждая потребует для редукции $\frac{n}{2^{L-2}} = 2$ сложения-вычитания и решения двух задач
размера 2, каждая из которых потребует $2$ сложения-вычитания.

\begin{multline*}
    S_\pm(n)
    = n + 2 \frac{n}{2} + 4 \frac{n}{4} + \dots + 2^{L-2} \left ( \frac{n}{2^{L-2}} + 2 \cdot 2 \right ) = \\
    %
    = \underbrace{n + n + n + \dots + n}_{L-1} + 2^{L-2+1+1}
    = \underbrace{n + n + n + \dots + n}_{L-1} + n
    = n \cdot L
    = n \log_2 n .
\end{multline*}

\subsection{6 и 3, факторизация}

Матрица степеней
\[
    F_6
    = \begin{pmatrix}
        1 & 1        & 1           & 1           & 1           & 1           \\
        1 & \delta^1 & \delta^2    & \delta^3    & \delta^4    & \delta^5    \\
        1 & \delta^2 & \delta^4    & \delta^6    & \delta^8    & \delta^{10} \\
        1 & \delta^3 & \delta^6    & \delta^9    & \delta^{12} & \delta^{15} \\
        1 & \delta^4 & \delta^8    & \delta^{12} & \delta^{16} & \delta^{20} \\
        1 & \delta^5 & \delta^{10} & \delta^{15} & \delta^{20} & \delta^{25} \\
    \end{pmatrix}
\]
Перестановка строк ($P_6$ --- обратная перестановка строк):
\[
    F_6
    =
    \underbrace{
        \begin{pmatrix}
            1 & 0 & 0 & 0 & 0 & 0 \\
            0 & 0 & 1 & 0 & 0 & 0 \\
            0 & 0 & 0 & 0 & 1 & 0 \\
            0 & 1 & 0 & 0 & 0 & 0 \\
            0 & 0 & 0 & 1 & 0 & 0 \\
            0 & 0 & 0 & 0 & 0 & 1 \\
        \end{pmatrix}
    }_{P_6}
    \begin{pmatrix}
        1 & 1        & 1           & 1           & 1           & 1           \\
        1 & \delta^3 & \delta^6    & \delta^9    & \delta^{12} & \delta^{15} \\
        1 & \delta^1 & \delta^2    & \delta^3    & \delta^4    & \delta^5    \\
        1 & \delta^4 & \delta^8    & \delta^{12} & \delta^{16} & \delta^{20} \\
        1 & \delta^2 & \delta^4    & \delta^6    & \delta^8    & \delta^{10} \\
        1 & \delta^5 & \delta^{10} & \delta^{15} & \delta^{20} & \delta^{25} \\
    \end{pmatrix}
\]
Выделение блоков:
\[
    F_6
    = P_6
    \begin{pmatrix}
        \begin{pmatrix}
            1 & 1        \\
            1 & \delta^3
        \end{pmatrix}
         &
        \begin{pmatrix}
            1        & 1        \\
            \delta^6 & \delta^9
        \end{pmatrix}
         &
        \begin{pmatrix}
            1           & 1           \\
            \delta^{12} & \delta^{15}
        \end{pmatrix}
        \\
        \begin{pmatrix}
            1 & 1        \\
            1 & \delta^3
        \end{pmatrix}
        \begin{pmatrix}
            1 & 0      \\
            0 & \delta
        \end{pmatrix}
         &
        \begin{pmatrix}
            1        & 1        \\
            \delta^6 & \delta^9
        \end{pmatrix}
        \begin{pmatrix}
            \delta^2 & 0        \\
            0        & \delta^3
        \end{pmatrix}
         &
        \begin{pmatrix}
            1           & 1           \\
            \delta^{12} & \delta^{15}
        \end{pmatrix}
        \begin{pmatrix}
            \delta^4 & 0        \\
            0        & \delta^5
        \end{pmatrix}
        \\
        \begin{pmatrix}
            1 & 1        \\
            1 & \delta^3
        \end{pmatrix}
        \begin{pmatrix}
            1 & 0        \\
            0 & \delta^2
        \end{pmatrix}
         &
        \begin{pmatrix}
            1        & 1        \\
            \delta^6 & \delta^9
        \end{pmatrix}
        \begin{pmatrix}
            \delta^4 & 0        \\
            0        & \delta^6
        \end{pmatrix}
         &
        \begin{pmatrix}
            1           & 1           \\
            \delta^{12} & \delta^{15}
        \end{pmatrix}
        \begin{pmatrix}
            \delta^8 & 0           \\
            0        & \delta^{10}
        \end{pmatrix}
    \end{pmatrix}
\]
Сокращение степени:
\[
    F_6
    = P_6
    \begin{pmatrix}
        \begin{pmatrix}
            1 & 1        \\
            1 & \delta^3
        \end{pmatrix}
         &
        \begin{pmatrix}
            1 & 1        \\
            1 & \delta^3
        \end{pmatrix}
         &
        \begin{pmatrix}
            1 & 1        \\
            1 & \delta^3
        \end{pmatrix}
        \\
        \begin{pmatrix}
            1 & 1        \\
            1 & \delta^3
        \end{pmatrix}
        \begin{pmatrix}
            1 & 0      \\
            0 & \delta
        \end{pmatrix}
         &
        \begin{pmatrix}
            1 & 1        \\
            1 & \delta^3
        \end{pmatrix}
        \begin{pmatrix}
            \delta^2 & 0        \\
            0        & \delta^3
        \end{pmatrix}
         &
        \begin{pmatrix}
            1 & 1        \\
            1 & \delta^3
        \end{pmatrix}
        \begin{pmatrix}
            \delta^4 & 0        \\
            0        & \delta^5
        \end{pmatrix}
        \\
        \begin{pmatrix}
            1 & 1        \\
            1 & \delta^3
        \end{pmatrix}
        \begin{pmatrix}
            1 & 0        \\
            0 & \delta^2
        \end{pmatrix}
         &
        \begin{pmatrix}
            1 & 1        \\
            1 & \delta^3
        \end{pmatrix}
        \begin{pmatrix}
            \delta^4 & 0        \\
            0        & \delta^6
        \end{pmatrix}
         &
        \begin{pmatrix}
            1 & 1        \\
            1 & \delta^3
        \end{pmatrix}
        \begin{pmatrix}
            \delta^8 & 0           \\
            0        & \delta^{10}
        \end{pmatrix}
    \end{pmatrix}
\]
Пусть
\[
    F_2(\delta)
    = \begin{pmatrix}
        1 & 1      \\
        1 & \delta
    \end{pmatrix}
\]
--- матрица Фурье размера 2, тогда:
\[
    F_6
    = P_6
    \begin{pmatrix}
        F_2(\delta^3) & 0             & 0             \\
        0             & F_2(\delta^3) & 0             \\
        0             & 0             & F_2(\delta^3)
    \end{pmatrix}
    \begin{pmatrix}
        1 & 0        & 1        & 0        & 1        & 0           \\
        0 & 1        & 0        & 1        & 0        & 1           \\
        1 & 0        & \delta^2 & 0        & \delta^4 & 0           \\
        0 & \delta   & 0        & \delta^3 & 0        & \delta^5    \\
        1 & 0        & \delta^4 & 0        & \delta^8 & 0           \\
        0 & \delta^2 & 0        & \delta^6 & 0        & \delta^{10}
    \end{pmatrix}
\]
Матрица слева преобразуется перестановкой строк и столбцов к матрице (причём той же матрицей $P_6$!):
\begin{multline*}
    \begin{pmatrix}
        1 & 0        & 1        & 0        & 1        & 0           \\
        0 & 1        & 0        & 1        & 0        & 1           \\
        1 & 0        & \delta^2 & 0        & \delta^4 & 0           \\
        0 & \delta   & 0        & \delta^3 & 0        & \delta^5    \\
        1 & 0        & \delta^4 & 0        & \delta^8 & 0           \\
        0 & \delta^2 & 0        & \delta^6 & 0        & \delta^{10}
    \end{pmatrix}
    = P_6^T
    \begin{pmatrix}
        1 & 1        & 1        & 0        & 0        & 0           \\
        1 & \delta^2 & \delta^4 & 0        & 0        & 0           \\
        1 & \delta^4 & \delta^8 & 0        & 0        & 0           \\
        0 & 0        & 0        & 1        & 1        & 1           \\
        0 & 0        & 0        & \delta   & \delta^3 & \delta^5    \\
        0 & 0        & 0        & \delta^2 & \delta^6 & \delta^{10}
    \end{pmatrix}
    P_6 = \\
    %
    = P_6^T
    \begin{pmatrix}
        1 & 0 & 0 & 0 & 0      & 0        \\
        0 & 1 & 0 & 0 & 0      & 0        \\
        0 & 0 & 1 & 0 & 0      & 0        \\
        0 & 0 & 0 & 1 & 0      & 0        \\
        0 & 0 & 0 & 0 & \delta & 0        \\
        0 & 0 & 0 & 0 & 0      & \delta^2 \\
    \end{pmatrix}
    \begin{pmatrix}
        F_3(\delta^2) & 0             \\
        0             & F_3(\delta^2) \\
    \end{pmatrix}
    P_6
\end{multline*}
Таким образом,
\[
    F_6
    = P_6
    \begin{pmatrix}
        F_2(\delta^3) & 0             & 0             \\
        0             & F_2(\delta^3) & 0             \\
        0             & 0             & F_2(\delta^3)
    \end{pmatrix}
    P_6^T
    \begin{pmatrix}
        I_3 & 0   \\
        0   & D_3
    \end{pmatrix}
    \begin{pmatrix}
        F_3(\delta^2) & 0             \\
        0             & F_3(\delta^2) \\
    \end{pmatrix}
    P_6
\]

\subsection{6 и 2}

Матрица степеней
\[
    F_6
    = \begin{pmatrix}
        1 & 1        & 1           & 1           & 1           & 1           \\
        1 & \delta^1 & \delta^2    & \delta^3    & \delta^4    & \delta^5    \\
        1 & \delta^2 & \delta^4    & \delta^6    & \delta^8    & \delta^{10} \\
        1 & \delta^3 & \delta^6    & \delta^9    & \delta^{12} & \delta^{15} \\
        1 & \delta^4 & \delta^8    & \delta^{12} & \delta^{16} & \delta^{20} \\
        1 & \delta^5 & \delta^{10} & \delta^{15} & \delta^{20} & \delta^{25} \\
    \end{pmatrix}
\]
Перестановка строк ($P_6$ --- обратная перестановка строк):
\[
    F_6
    = P_6
    \begin{pmatrix}
        1 & 1        & 1           & 1           & 1           & 1           \\
        1 & \delta^2 & \delta^4    & \delta^6    & \delta^8    & \delta^{10} \\
        1 & \delta^4 & \delta^8    & \delta^{12} & \delta^{16} & \delta^{20} \\
        1 & \delta^1 & \delta^2    & \delta^3    & \delta^4    & \delta^5    \\
        1 & \delta^3 & \delta^6    & \delta^9    & \delta^{12} & \delta^{15} \\
        1 & \delta^5 & \delta^{10} & \delta^{15} & \delta^{20} & \delta^{25} \\
    \end{pmatrix}
\]
Выделение блоков:
\[
    F_6
    = P_6
    \begin{pmatrix}
        \begin{pmatrix}
            1 & 1        & 1        \\
            1 & \delta^2 & \delta^4 \\
            1 & \delta^4 & \delta^8
        \end{pmatrix}
         &
        \begin{pmatrix}
            1           & 1           & 1           \\
            \delta^6    & \delta^8    & \delta^{10} \\
            \delta^{12} & \delta^{16} & \delta^{20}
        \end{pmatrix}
        \\
        \begin{pmatrix}
            1 & 1        & 1        \\
            1 & \delta^2 & \delta^4 \\
            1 & \delta^4 & \delta^8
        \end{pmatrix}
        \begin{pmatrix}
            1 & 0      & 0        \\
            0 & \delta & 0        \\
            0 & 0      & \delta^2
        \end{pmatrix}
         &
        \begin{pmatrix}
            1           & 1           & 1           \\
            \delta^6    & \delta^8    & \delta^{10} \\
            \delta^{12} & \delta^{16} & \delta^{20}
        \end{pmatrix}
        \begin{pmatrix}
            \delta^3 & 0        & 0        \\
            0        & \delta^4 & 0        \\
            0        & 0        & \delta^5
        \end{pmatrix}
    \end{pmatrix}
\]
Сокращение степени в силу $\delta^6 = 1$:
\[
    F_6
    = P_6
    \begin{pmatrix}
        \begin{pmatrix}
            1 & 1        & 1        \\
            1 & \delta^2 & \delta^4 \\
            1 & \delta^4 & \delta^8
        \end{pmatrix}
         &
        \begin{pmatrix}
            1 & 1        & 1        \\
            1 & \delta^2 & \delta^4 \\
            1 & \delta^4 & \delta^8
        \end{pmatrix}
        \\
        \begin{pmatrix}
            1 & 1        & 1        \\
            1 & \delta^2 & \delta^4 \\
            1 & \delta^4 & \delta^8
        \end{pmatrix}
        \begin{pmatrix}
            1 & 0      & 0        \\
            0 & \delta & 0        \\
            0 & 0      & \delta^2
        \end{pmatrix}
         &
        \begin{pmatrix}
            1 & 1        & 1        \\
            1 & \delta^2 & \delta^4 \\
            1 & \delta^4 & \delta^8
        \end{pmatrix}
        \begin{pmatrix}
            \delta^3 & 0        & 0        \\
            0        & \delta^4 & 0        \\
            0        & 0        & \delta^5
        \end{pmatrix}
    \end{pmatrix}
\]
Матрица
\[
    \begin{pmatrix}
        \delta^3 & 0        & 0        \\
        0        & \delta^4 & 0        \\
        0        & 0        & \delta^5
    \end{pmatrix}
    = \delta^3
    \begin{pmatrix}
        1 & 0        & 0        \\
        0 & \delta^1 & 0        \\
        0 & 0        & \delta^2
    \end{pmatrix}
    = (-1)
    \begin{pmatrix}
        1 & 0      & 0        \\
        0 & \delta & 0        \\
        0 & 0      & \delta^2
    \end{pmatrix}
\]
Таким образом,
\[
    F_6
    = P_6
    \begin{pmatrix}
        \begin{pmatrix}
            1 & 1        & 1        \\
            1 & \delta^2 & \delta^4 \\
            1 & \delta^4 & \delta^8
        \end{pmatrix}
         &
        \begin{pmatrix}
            1 & 1        & 1        \\
            1 & \delta^2 & \delta^4 \\
            1 & \delta^4 & \delta^8
        \end{pmatrix}
        \\
        \begin{pmatrix}
            1 & 1        & 1        \\
            1 & \delta^2 & \delta^4 \\
            1 & \delta^4 & \delta^8
        \end{pmatrix}
        \begin{pmatrix}
            1 & 0      & 0        \\
            0 & \delta & 0        \\
            0 & 0      & \delta^2
        \end{pmatrix}
         &
        -
        \begin{pmatrix}
            1 & 1        & 1        \\
            1 & \delta^2 & \delta^4 \\
            1 & \delta^4 & \delta^8
        \end{pmatrix}
        \begin{pmatrix}
            1 & 0      & 0        \\
            0 & \delta & 0        \\
            0 & 0      & \delta^2
        \end{pmatrix}
    \end{pmatrix}
\]
Пусть
\[
    F_3(\delta^2)
    = \begin{pmatrix}
        1 & 1        & 1        \\
        1 & \delta^2 & \delta^4 \\
        1 & \delta^4 & \delta^8
    \end{pmatrix},
    \;
    D_3
    = \begin{pmatrix}
        1 & 0      & 0        \\
        0 & \delta & 0        \\
        0 & 0      & \delta^2
    \end{pmatrix} ,
\]
где $F_3(\delta)$ --- матрица Фурье размера 3, тогда
\begin{multline*}
    F_6(\delta)
    = P_6
    \begin{pmatrix}
        F_3(\delta^2)     & F_3(\delta^2)      \\
        F_3(\delta^2) D_3 & -F_3(\delta^2) D_3 \\
    \end{pmatrix} = \\
    %
    = P_6
    \begin{pmatrix}
        F_3(\delta^2) & 0             \\
        0             & F_3(\delta^2) \\
    \end{pmatrix}
    \begin{pmatrix}
        I_3 & I_3  \\
        D_3 & -D_3 \\
    \end{pmatrix} = \\
    %
    = P_6
    \begin{pmatrix}
        F_3(\delta^2) & 0             \\
        0             & F_3(\delta^2) \\
    \end{pmatrix}
    \begin{pmatrix}
        I_3 & 0   \\
        0   & D_3 \\
    \end{pmatrix}
    \begin{pmatrix}
        I_3 & I_3  \\
        I_3 & -I_3 \\
    \end{pmatrix}
\end{multline*}

\subsection{4 и 2}

Исходная матрица Фурье:
\[
    F_4(\varepsilon)
    = \begin{pmatrix}
        1 & 1                & 1                & 1                \\
        1 & \varepsilon^{-1} & \varepsilon^{-2} & \varepsilon^{-3} \\
        1 & \varepsilon^{-2} & \varepsilon^{-4} & \varepsilon^{-6} \\
        1 & \varepsilon^{-3} & \varepsilon^{-6} & \varepsilon^{-9}
    \end{pmatrix}
    = \begin{pmatrix}
        1 & 1                & 1                                 & 1                                 \\
        1 & \varepsilon^{-1} & \varepsilon^{-2}                  & \varepsilon^{-3}                  \\
        1 & \varepsilon^{-2} & \varepsilon^{-4}                  & \varepsilon^{-4} \varepsilon^{-2} \\
        1 & \varepsilon^{-3} & \varepsilon^{-4} \varepsilon^{-2} & \varepsilon^{-4} \varepsilon^{-5}
    \end{pmatrix}
\]
Сокращение степени с учётом $\varepsilon^{-4} = 1$:
\[
    F_4(\varepsilon)
    = \begin{pmatrix}
        1 & 1                & 1                & 1                \\
        1 & \varepsilon^{-1} & \varepsilon^{-2} & \varepsilon^{-3} \\
        1 & \varepsilon^{-2} & 1                & \varepsilon^{-2} \\
        1 & \varepsilon^{-3} & \varepsilon^{-2} & \varepsilon^{-5}
    \end{pmatrix}
\]
Перестановка строк, сперва все нечётные строки, затем все чётные:
\[
    F_4(\varepsilon)
    =
    P_4
    \begin{pmatrix}
        1 & 1                & 1                & 1                \\
        1 & \varepsilon^{-2} & 1                & \varepsilon^{-2} \\
        1 & \varepsilon^{-1} & \varepsilon^{-2} & \varepsilon^{-3} \\
        1 & \varepsilon^{-3} & \varepsilon^{-2} & \varepsilon^{-5}
    \end{pmatrix} ,
\]
где $P_4$ --- матрица, выполняющая обратную перестановку строк.

Выделение подматриц:
\[
    F_4(\varepsilon)
    = P_4 \begin{pmatrix}
        \begin{pmatrix}
            1 & 1                \\
            1 & \varepsilon^{-2}
        \end{pmatrix}
         &
        \begin{pmatrix}
            1 & 1                \\
            1 & \varepsilon^{-2}
        \end{pmatrix} \\
        \begin{pmatrix}
            1 & 1                \\
            1 & \varepsilon^{-2}
        \end{pmatrix}
        \begin{pmatrix}
            1 & 0                \\
            0 & \varepsilon^{-1}
        \end{pmatrix}
         &
        \varepsilon^{-2}
        \begin{pmatrix}
            1 & 1                \\
            1 & \varepsilon^{-2}
        \end{pmatrix}
        \begin{pmatrix}
            1 & 0                \\
            0 & \varepsilon^{-1}
        \end{pmatrix}
    \end{pmatrix} ,
\]
где
\[
    \varepsilon^{-2}
    = \left( e^{i \frac{2 \pi}{4}} \right)^2
    = e^{i \frac{2 \pi}{2}}
    = e^{i \pi}
    = -1 ,
\]
поэтому
\[
    F_4(\varepsilon)
    = P_4
    \begin{pmatrix}
        F_2(\varepsilon^2)     & F_2(\varepsilon^2)       \\
        F_2(\varepsilon^2) D_2 & - F_2(\varepsilon^2) D_2
    \end{pmatrix} ,
\]
где $F_2(\delta)$ --- матрица Фурье порядка 2 и $D_2$ --- диагональная матрица:
\[
    F_2(\delta)
    = \begin{pmatrix}
        1 & 1           \\
        1 & \delta^{-1}
    \end{pmatrix} ,
    \;
    D_2
    = \begin{pmatrix}
        1 & 0                \\
        0 & \varepsilon^{-1}
    \end{pmatrix} .
\]
Представим, что нужно умножить матрицу $F_4(\varepsilon)$ на некоторый вектор $h$ (как для вычисления спектра), и посмотрим как будет выглядеть умножение:
\begin{multline*}
    F_4(\varepsilon) h
    = P_4
    \begin{pmatrix}
        F_2(\varepsilon^2)     & F_2(\varepsilon^2)       \\
        F_2(\varepsilon^2) D_2 & - F_2(\varepsilon^2) D_2
    \end{pmatrix}
    \begin{pmatrix}
        h_0 \\
        h_1 \\
        h_2 \\
        h_3
    \end{pmatrix}
    = P_4
    \begin{pmatrix}
        F_2(\varepsilon^2)
        \begin{pmatrix}
            h_0 \\
            h_1
        \end{pmatrix}
        +
        F_2(\varepsilon^2)
        \begin{pmatrix}
            h_2 \\
            h_3
        \end{pmatrix} \\
        F_2(\varepsilon^2) D_2
        \begin{pmatrix}
            h_0 \\
            h_1
        \end{pmatrix}
        - F_2(\varepsilon^2) D_2
        \begin{pmatrix}
            h_2 \\
            h_3
        \end{pmatrix}
    \end{pmatrix}
    = \\
    %
    = P_4
    \begin{pmatrix}
        F_2(\varepsilon^2)
        \left(
        \begin{pmatrix}
            h_0 \\
            h_1
        \end{pmatrix}
        +
        \begin{pmatrix}
            h_2 \\
            h_3
        \end{pmatrix}
        \right) \\
        F_2(\varepsilon^2) D_2
        \left(
        \begin{pmatrix}
            h_0 \\
            h_1
        \end{pmatrix}
        -
        \begin{pmatrix}
            h_2 \\
            h_3
        \end{pmatrix}
        \right)
    \end{pmatrix} = \\
    %
    = P_4
    \begin{pmatrix}
        F_2(\varepsilon^2) \\
        F_2(\varepsilon^2) D_2
    \end{pmatrix}
    \begin{pmatrix}
        I_2 & I_2  \\
        I_2 & -I_2
    \end{pmatrix}
    \begin{pmatrix}
        h_0 \\
        h_1 \\
        h_2 \\
        h_3
    \end{pmatrix}
    = P_4
    \begin{pmatrix}
        F_2(\varepsilon^2) & 0                  \\
        0                  & F_2(\varepsilon^2)
    \end{pmatrix}
    \begin{pmatrix}
        I_2 & 0   \\
        0   & D_2
    \end{pmatrix}
    \begin{pmatrix}
        I_2 & I_2  \\
        I_2 & -I_2
    \end{pmatrix}
    \begin{pmatrix}
        h_0 \\
        h_1 \\
        h_2 \\
        h_3
    \end{pmatrix}
\end{multline*}

\subsection{8 и 2}

Матрица Фурье 8.
\[
    F_8
    = \begin{pmatrix}
        1 & 1                & 1                 & 1                 & 1                 & 1                 & 1                 & 1                 \\
        1 & \varepsilon^{-1} & \varepsilon^{-2}  & \varepsilon^{-3}  & \varepsilon^{-4}  & \varepsilon^{-5}  & \varepsilon^{-6}  & \varepsilon^{-7}  \\
        1 & \varepsilon^{-2} & \varepsilon^{-4}  & \varepsilon^{-6}  & \varepsilon^{-8}  & \varepsilon^{-10} & \varepsilon^{-12} & \varepsilon^{-14} \\
        1 & \varepsilon^{-3} & \varepsilon^{-6}  & \varepsilon^{-9}  & \varepsilon^{-12} & \varepsilon^{-15} & \varepsilon^{-18} & \varepsilon^{-21} \\
        1 & \varepsilon^{-4} & \varepsilon^{-8}  & \varepsilon^{-12} & \varepsilon^{-16} & \varepsilon^{-20} & \varepsilon^{-24} & \varepsilon^{-28} \\
        1 & \varepsilon^{-5} & \varepsilon^{-10} & \varepsilon^{-15} & \varepsilon^{-20} & \varepsilon^{-25} & \varepsilon^{-30} & \varepsilon^{-35} \\
        1 & \varepsilon^{-6} & \varepsilon^{-12} & \varepsilon^{-18} & \varepsilon^{-24} & \varepsilon^{-30} & \varepsilon^{-36} & \varepsilon^{-42} \\
        1 & \varepsilon^{-7} & \varepsilon^{-14} & \varepsilon^{-21} & \varepsilon^{-28} & \varepsilon^{-35} & \varepsilon^{-42} & \varepsilon^{-49}
    \end{pmatrix}
\]

Сокращение степени.
\[
    F_8
    = \begin{pmatrix}
        1 & 1                & 1                 & 1                 & 1                & 1                 & 1                 & 1                 \\
        1 & \varepsilon^{-1} & \varepsilon^{-2}  & \varepsilon^{-3}  & \varepsilon^{-4} & \varepsilon^{-5}  & \varepsilon^{-6}  & \varepsilon^{-7}  \\
        1 & \varepsilon^{-2} & \varepsilon^{-4}  & \varepsilon^{-6}  & 1                & \varepsilon^{-2}  & \varepsilon^{-4}  & \varepsilon^{-6}  \\
        1 & \varepsilon^{-3} & \varepsilon^{-6}  & \varepsilon^{-9}  & \varepsilon^{-4} & \varepsilon^{-7}  & \varepsilon^{-10} & \varepsilon^{-13} \\
        1 & \varepsilon^{-4} & \varepsilon^{-8}  & \varepsilon^{-12} & 1                & \varepsilon^{-4}  & \varepsilon^{-8}  & \varepsilon^{-12} \\
        1 & \varepsilon^{-5} & \varepsilon^{-10} & \varepsilon^{-15} & \varepsilon^{-4} & \varepsilon^{-9}  & \varepsilon^{-14} & \varepsilon^{-19} \\
        1 & \varepsilon^{-6} & \varepsilon^{-12} & \varepsilon^{-18} & 1                & \varepsilon^{-6}  & \varepsilon^{-12} & \varepsilon^{-18} \\
        1 & \varepsilon^{-7} & \varepsilon^{-14} & \varepsilon^{-21} & \varepsilon^{-4} & \varepsilon^{-11} & \varepsilon^{-18} & \varepsilon^{-25}
    \end{pmatrix}
\]

Перестановка строк: сперва нечётные, затем чётные ($P_8$ --- матрица обратной перестановки строк).
\[
    F_8
    = P_8
    \begin{pmatrix}
        1 & 1                & 1                 & 1                 & 1                & 1                 & 1                 & 1                 \\
        1 & \varepsilon^{-2} & \varepsilon^{-4}  & \varepsilon^{-6}  & 1                & \varepsilon^{-2}  & \varepsilon^{-4}  & \varepsilon^{-6}  \\
        1 & \varepsilon^{-4} & \varepsilon^{-8}  & \varepsilon^{-12} & 1                & \varepsilon^{-4}  & \varepsilon^{-8}  & \varepsilon^{-12} \\
        1 & \varepsilon^{-6} & \varepsilon^{-12} & \varepsilon^{-18} & 1                & \varepsilon^{-6}  & \varepsilon^{-12} & \varepsilon^{-18} \\
        1 & \varepsilon^{-1} & \varepsilon^{-2}  & \varepsilon^{-3}  & \varepsilon^{-4} & \varepsilon^{-5}  & \varepsilon^{-6}  & \varepsilon^{-7}  \\
        1 & \varepsilon^{-3} & \varepsilon^{-6}  & \varepsilon^{-9}  & \varepsilon^{-4} & \varepsilon^{-7}  & \varepsilon^{-10} & \varepsilon^{-13} \\
        1 & \varepsilon^{-5} & \varepsilon^{-10} & \varepsilon^{-15} & \varepsilon^{-4} & \varepsilon^{-9}  & \varepsilon^{-14} & \varepsilon^{-19} \\
        1 & \varepsilon^{-7} & \varepsilon^{-14} & \varepsilon^{-21} & \varepsilon^{-4} & \varepsilon^{-11} & \varepsilon^{-18} & \varepsilon^{-25}
    \end{pmatrix}
\]

Выделение подматриц.
\[
    F_8
    = \begin{pmatrix}
        \begin{pmatrix}
            1 & 1                & 1                 & 1                 \\
            1 & \varepsilon^{-2} & \varepsilon^{-4}  & \varepsilon^{-6}  \\
            1 & \varepsilon^{-4} & \varepsilon^{-8}  & \varepsilon^{-12} \\
            1 & \varepsilon^{-6} & \varepsilon^{-12} & \varepsilon^{-18} \\
        \end{pmatrix}
         &
        \begin{pmatrix}
            1 & 1                & 1                 & 1                 \\
            1 & \varepsilon^{-2} & \varepsilon^{-4}  & \varepsilon^{-6}  \\
            1 & \varepsilon^{-4} & \varepsilon^{-8}  & \varepsilon^{-12} \\
            1 & \varepsilon^{-6} & \varepsilon^{-12} & \varepsilon^{-18} \\
        \end{pmatrix} \\
        %
        \begin{pmatrix}
            1 & 1                & 1                 & 1                 \\
            1 & \varepsilon^{-2} & \varepsilon^{-4}  & \varepsilon^{-6}  \\
            1 & \varepsilon^{-4} & \varepsilon^{-8}  & \varepsilon^{-12} \\
            1 & \varepsilon^{-6} & \varepsilon^{-12} & \varepsilon^{-18} \\
        \end{pmatrix}
        \begin{pmatrix}
            1 & 0                & 0                & 0                \\
            0 & \varepsilon^{-1} & 0                & 0                \\
            0 & 0                & \varepsilon^{-2} & 0                \\
            0 & 0                & 0                & \varepsilon^{-3} \\
        \end{pmatrix}
         &
        \varepsilon^{-4}
        \begin{pmatrix}
            1 & 1                & 1                 & 1                 \\
            1 & \varepsilon^{-2} & \varepsilon^{-4}  & \varepsilon^{-6}  \\
            1 & \varepsilon^{-4} & \varepsilon^{-8}  & \varepsilon^{-12} \\
            1 & \varepsilon^{-6} & \varepsilon^{-12} & \varepsilon^{-18} \\
        \end{pmatrix}
        \begin{pmatrix}
            1 & 0                & 0                & 0                \\
            0 & \varepsilon^{-1} & 0                & 0                \\
            0 & 0                & \varepsilon^{-2} & 0                \\
            0 & 0                & 0                & \varepsilon^{-3} \\
        \end{pmatrix}
    \end{pmatrix}
\]
Матрица Фурье порядка 4
\[
    F_4(\varepsilon^2)
    = \begin{pmatrix}
        1 & 1                & 1                 & 1                 \\
        1 & \varepsilon^{-2} & \varepsilon^{-4}  & \varepsilon^{-6}  \\
        1 & \varepsilon^{-4} & \varepsilon^{-8}  & \varepsilon^{-12} \\
        1 & \varepsilon^{-6} & \varepsilon^{-12} & \varepsilon^{-18} \\
    \end{pmatrix}
\]
Сокращение степени:
\[
    F_4(\varepsilon^2)
    = \begin{pmatrix}
        1 & 1                & 1                & 1                 \\
        1 & \varepsilon^{-2} & \varepsilon^{-4} & \varepsilon^{-6}  \\
        1 & \varepsilon^{-4} & 1                & \varepsilon^{-4}  \\
        1 & \varepsilon^{-6} & \varepsilon^{-4} & \varepsilon^{-10} \\
    \end{pmatrix}
\]
Перестановка строк ($P_4$ --- матрица обратной перестановки).
\[
    F_4(\varepsilon^2)
    =
    P_4
    \begin{pmatrix}
        1 & 1                & 1                & 1                 \\
        1 & \varepsilon^{-4} & 1                & \varepsilon^{-4}  \\
        1 & \varepsilon^{-2} & \varepsilon^{-4} & \varepsilon^{-6}  \\
        1 & \varepsilon^{-6} & \varepsilon^{-4} & \varepsilon^{-10} \\
    \end{pmatrix}
\]
Выделение подматриц:
\begin{multline*}
    F_4(\varepsilon^2)
    =
    P_4
    \begin{pmatrix}
        \begin{pmatrix}
            1 & 1                \\
            1 & \varepsilon^{-4}
        \end{pmatrix}
         &
        \begin{pmatrix}
            1 & 1                \\
            1 & \varepsilon^{-4}
        \end{pmatrix}
        \\
        \begin{pmatrix}
            1 & \varepsilon^{-2} \\
            1 & \varepsilon^{-6}
        \end{pmatrix}
         &
        \varepsilon^{-4}
        \begin{pmatrix}
            1 & \varepsilon^{-2} \\
            1 & \varepsilon^{-6} \\
        \end{pmatrix}
    \end{pmatrix} = \\
    %
    = P_4
    \begin{pmatrix}
        \begin{pmatrix}
            1 & 1                \\
            1 & \varepsilon^{-4}
        \end{pmatrix}
         &
        \begin{pmatrix}
            1 & 1                \\
            1 & \varepsilon^{-4}
        \end{pmatrix}
        \\
        \begin{pmatrix}
            1 & 1                \\
            1 & \varepsilon^{-4}
        \end{pmatrix}
        \begin{pmatrix}
            1 & 0                \\
            0 & \varepsilon^{-2}
        \end{pmatrix}
         &
        \varepsilon^{-4}
        \begin{pmatrix}
            1 & 1                \\
            1 & \varepsilon^{-4}
        \end{pmatrix}
        \begin{pmatrix}
            1 & 0                \\
            0 & \varepsilon^{-2}
        \end{pmatrix}
    \end{pmatrix} .
\end{multline*}

\section{Фильтр нижних частот}

Пусть $X$, $H$ и $Y$ --- спектры входного сигнала, импульсной характеристики и выходного сигнала соответственно. Равенство:
\[
    Y_k = H_k \cdot X_k
\]
открывает широкие возможности для модификации спектра входного сигнала $X$: если элемент спектра $X_k$ входного сигнала не нулевой, то путем умножения на
$H_k$ в спектре выходного сигнала элемент $Y_k$ можно сделать любым. Это означает, что если во входном сигнале есть гармоника с частотой $\frac{2 \pi}{n} k$,
то её аплитуду можно увеличить и уменьшить в произвольное число раз (и, конечно, обнулить).

Предположим, что нужно спроектировать фильтр нижних частот, то есть такой фильтр, который пропускает (увеличивает или незначительно уменьшает амплитуды
гармоник с нижними частотами), и не пропускает (значительно уменьшает амплитуды с верхними частотами). Далее нужно отделить "нижние"{} частоты от
"верхних"{}, пусть частота среза (cut frequency) $f_c = \frac{2 \pi}{n} k_c$, все частоты не больше $f_c$ будем считать нижними, а все частоты больше $f_c$
будем считать верхними.

Теперь можно задать спектр $H$ таким образом:
\begin{enumerate}
    \item $H_0 = 1$ --- сохраняем величину постоянной,
    \item $H_1 = \dots = H_{k_c} = 1$ --- сохраняем гармоники от 1 до $k_c$,
    \item $H_{k_c + 1} = \dots = H_{n - k_c - 1} = 0$ --- удаляем промежуточные частоты,
    \item $H_{n - k_c} = \dots = H_{n-1} = 1$ --- сохраняем "симметричные"{} гармоники.
\end{enumerate}
Спектр $H$ равен единице по "краям"{} и нулю в "середине"{} (как на рисунке \ref{figure:finite-filter:low-frequencies:system-spectrum}).

\begin{figure}[h]
    \centering
    \begin{tikzpicture}
        % ось X
        \draw [->] ( -0.1, 0 ) -- ( 7, 0 ) node [below] ( 7, 0 ) {$N$};
        % ось Y
        \draw [->] ( 0, -0.1 ) -- ( 0, 1.5 ) node [left] ( 0, 1.5 ) {$\vec{H}$};

        % осчеты
        \draw [-] ( 0, 0.1 ) -- ( 0, -0.1 ) node [below ] ( 0, -0.1 ) {$0$};
        \draw [-] ( 1, 0.1 ) -- ( 1, -0.1 ) node [below ] ( 1, -0.1 ) {$1$};
        \draw [-] ( 2, 0.1 ) -- ( 2, -0.1 ) node [below ] ( 2, -0.1 ) {$2$};
        \draw [-] ( 3, 0.1 ) -- ( 3, -0.1 ) node [below ] ( 3, -0.1 ) {$3$};
        \draw [-] ( 4, 0.1 ) -- ( 4, -0.1 ) node [below ] ( 4, -0.1 ) {$4$};
        \draw [-] ( 5, 0.1 ) -- ( 5, -0.1 ) node [below ] ( 5, -0.1 ) {$5$};
        \draw [-] ( 6, 0.1 ) -- ( 6, -0.1 ) node [below ] ( 6, -0.1 ) {$6$};

        % спектр
        \draw [fill=black] ( 0, 1 ) circle [radius=0.05];
        \draw [fill=black] ( 1, 1 ) circle [radius=0.05];
        \draw [fill=black] ( 2, 1 ) circle [radius=0.05];
        \draw [fill=black] ( 3, 0 ) circle [radius=0.05];
        \draw [fill=black] ( 4, 0 ) circle [radius=0.05];
        \draw [fill=black] ( 5, 1 ) circle [radius=0.05];
        \draw [fill=black] ( 6, 1 ) circle [radius=0.05];
    \end{tikzpicture}
    \caption{Спектр $H$ для $n=7$ и $k_c=2$.}
    \label{figure:finite-filter:low-frequencies:system-spectrum}
\end{figure}

В результате обратного дискретного преобразования Фурье получится импульсная характеристика $h$:
$$
    h = \frac{1}{n} F_n^* H .
$$

Пример фильтрации:
\matlab{low\_pass}
