\chapter{Стационарные линейные системы}

Дискретные сигналы.

\section{Линейность и стационарность}

Линейные системы --- аналог линейных операторов.

Линейность позволяет анализировать действие на базисные компоненты, из которых состоит сигнал.

Дополнительное свойство стационарности.

Дополнительное свойство реализуемости. Из него следует, что отклик не раньше воздействия.

Значит можно рассматривать полу-бесконечные сигналы.

\section{Декомпозиция}

Существует много вариантов декомпозиции --- разложение по базису.

Разложение по базису требует решения системы линейных уравнений.
Если базис ортогонализован, то для коэффициенты разложения легко вычисляются.

\section{Импульсная декомпозиция}

Единичный импульс.
Реакция линейной системы на единичный импульс --- импульсная характеристика.

\section{Импульсная характеристика}

Только конечная импульсная характеристика.
Периодические сигналы.

\section{Свёртка}

Операция свёртки.

Перевёрнутая импульсная характеристика "едет"{} по сигналу.

Матричная форма свёртки --- циркулянт.


\section{Декомпозиция Фурье}

Базис Фурье --- частный случай базиса степеней первообразного корня из единицы.
Базис Фурье имеет простую физическую интерпретацию.
Линейные системы могут только изменить амлитуду и фазу.
Базис Фурье ортогональный.