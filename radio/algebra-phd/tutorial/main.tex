\documentclass[a4paper,12pt]{book}

\usepackage[T1]{fontenc}
\usepackage[utf8]{inputenc}
\usepackage[english,russian]{babel}
\usepackage[margin=2cm]{geometry}
\usepackage{amsmath}
\usepackage{amssymb}
\usepackage{tikz}
\usepackage{color}
\usepackage{amsfonts}

\newcommand{\scalarproduct}[2]{\left ( #1, #2 \right )}
\newcommand{\modulus}[1]{\left | #1 \right |}

\newcommand{\perpendicular}[2]{\texttt{ort}_{#1} {#2}}
\newcommand{\projection}[2]{\texttt{pr}_{#1} {#2}}

\newcommand{\set}[1]{\left \{ #1 \right \}}

\newcommand{\kernel}{\mathtt{Ker}}
\newcommand{\image}{\mathtt{Im}}

\begin{document}

    \title{Прикладные методы линейной алгебры}
    \author{Тигетов Давид Георгиевич}
    \maketitle

    \tableofcontents

    % план занятий
    \chapter{План занятий}

Первый год.

{
\center
\begin{tabular}{|p{3cm}|p{12cm}|}
    \hline
    Занятие 1  & Унитарные пространства. Сопряжённый оператор. Свойства сопряженного оператора.                                 \\
    \hline
    Занятие 2  & Теорема Шура. Нормальные матрицы. Эрмитовые матрицы.                                                           \\
    \hline
    Занятие 3  & Экстремумы отношения Релея.                                                                                    \\
    \hline
    Занятие 4  & Экстремумы отношения Релея. Вычисление в Matlab.                                                               \\
    \hline
    Занятие 5  & Комплексные огибающие. Двухканальный излучатель. Коэффициент усиления. Пример 2. Пример 3. Пример 4. Пример 5. \\
    \hline
    Занятие 6  & Энергетическое ограничение. Коэффициент полезного действия. Примеры. Антенна.                                  \\
    \hline
    Занятие 7  & Плоская волна, набеги фаз в решётке. Обнаружение и пеленгация одного источника.                                \\
    \hline
    Занятие 8  & Обнаружение и пеленгация нескольких источников. Альтернативная пеленгация.                                     \\
    \hline
    Занятие 9  & Задача адаптации. Оценка ковариационной матрицы. Ортогонализатор.                                              \\
    \hline
    Занятие 10 & Сигналы. Циклическая свёртка. Циркулянты.                                                                      \\
    \hline
    Занятие 11 & Матрица Фурье. Спектр циклической свёртки.                                                                     \\
    \hline
    Занятие 12 & Быстрое преобразование Фурье. Фильтр нижних частот.                                                            \\
    \hline
\end{tabular}
\par
}

Второй год.

{
\center
\begin{tabular}{|p{3cm}|p{12cm}|}
    \hline
    Занятие 1  & Унитарные пространства. Сопряжённый оператор. Свойства сопряженного оператора. Теорема Шура. \\
    \hline
    Занятие 2  & Нормальные матрицы. Эрмитовые матрицы. Экстремумы отношения Релея.                           \\
    \hline
    Занятие 3  & Решение задачи 1 в Matlab.                                                                   \\
    \hline
    Занятие 4  & Комплексные огибающие. Диаграмма направленности. Коэффициент усиления. Примеры 2 -- 5.       \\
    \hline
    Занятие 5  & Энергетическое ограничение. Коэффициент полезного действия. Примеры 2 -- 5.                  \\
    \hline
    Занятие 6  & Решение задачи 2 в Matlab.                                                                   \\
    \hline
    Занятие 7  & Решение задачи 2 в Matlab.                                                                   \\
    \hline
    Занятие 8  & Плоская волна, набеги фаз в решётке. Обнаружение и пеленгация одного источника.              \\
    \hline
    Занятие 9  & Обнаружение и пеленгация нескольких источников. Альтернативная пеленгация.                   \\
    \hline
    Занятие 10  & Задача адаптации. Оценка ковариационной матрицы. Ортогонализатор.                            \\
    \hline
    Занятие 11 & Решение задачи 3 в Matlab.                                                                   \\
    \hline
    Занятие 12 & Сигналы. Циклическая свёртка. Циркулянты.                                                    \\
    \hline
    Занятие 13 & Матрица Фурье. Спектр циклической свёртки.                                                   \\
    \hline
    Занятие 14 & Быстрое преобразование Фурье. Фильтр нижних частот.                                          \\
    \hline
\end{tabular}
\par
}


    % отношение Релея
    \chapter{Сопряженный оператор}


\section{Унитарные пространства}

Пусть $\mathbb{C}$ --- обозначает поле комплексных чисел, и $\mathcal{U}$ --- множество векторов, для которых определена операции сложения $+$ векторов и умножения
векторов на число из поля $\mathbb{C}$, обладающих свойствами для любых $u, v, w \in \mathcal{U}$ и $\alpha, \beta \in \mathbb{C}$:
\begin{enumerate}
    \item $u + ( v + w ) = ( u + v ) + w$,
    \item $\exists 0 \in U: u + 0 = 0 + u = u$,
    \item $\exists (-u) \in U: u + (-u) = (-u) + u = 0$,
    \item $u + v = v + u$,
    \item $(\alpha + \beta) u = \alpha u + \beta u$,
    \item $\alpha ( u + v ) = \alpha u + \alpha v$,
    \item $\alpha (\beta u) = (\alpha \beta) u$,
    \item $u = 1 \cdot u$.
\end{enumerate}

Пусть на множестве всех пар векторов пространства $\mathcal{U}$ определена функция $\scalarproduct{\cdot}{\cdot}$, называемая скалярным произведением, обладающая
следующими свойствами для любых $u, v \in \mathcal{U}$ и числа $\lambda \in \mathbb{C}$:
\begin{enumerate}
    \item $\scalarproduct{u}{u} \ge 0$,
    \item $\scalarproduct{u}{u} = 0 \Leftrightarrow u = 0$,
    \item $\scalarproduct{u + w}{v} = \scalarproduct{u}{v} + \scalarproduct{w}{v}$,
    \item $\scalarproduct{\lambda u}{v} = \lambda \scalarproduct{u}{v}$,
    \item $\scalarproduct{u}{v} = \overline{\scalarproduct{v}{u}}$,
\end{enumerate}
где черта $\overline{\cdot}$ обозначает комплексное сопряжение. Из приведённых свойств скалярного произведения следует:
\begin{gather*}
    \scalarproduct{u}{v + w}
    = \overline{\scalarproduct{v + w}{u}}
    = \overline{\scalarproduct{v}{u} + \scalarproduct{w}{u}}
    = \overline{\scalarproduct{v}{u}} + \overline{\scalarproduct{w}{u}}
    = \scalarproduct{u}{v} + \scalarproduct{u}{w}, \\
    %
    \scalarproduct{u}{\lambda v}
    = \overline{\scalarproduct{\lambda v}{u}}
    = \overline{\lambda  \scalarproduct{v}{u}}
    = \overline{\lambda} \overline{\scalarproduct{v}{u}}
    = \overline{\lambda} \scalarproduct{u}{v} .
\end{gather*}

Если в линейном пространстве $\mathcal{U}$ задано скалярное произведение $\scalarproduct{\cdot}{\cdot}$, то такое пространство называется унитарным.

С помощью скалярного произведения можно определить норму $\norm{\cdot}$ векторов:
\[
    \norm{u} = \sqrt{\scalarproduct{u}{u}}.
\]
Такое определение будет удовлетворять всем свойствам нормы:
\begin{enumerate}
    \item $\norm{u} \ge 0$,
    \item $\norm{u} = 0 \Leftrightarrow u = 0$,
    \item $\norm{\lambda u} = \modulus{\lambda} \norm{u}$,
    \item $\norm{u + v} \le \norm{u} + \norm{v}$.
\end{enumerate}


\section{Операторы}

Пусть $\mathcal{U}$ является унитарным пространством со скалярным произведением $\scalarproduct{\cdot}{\cdot}_\mathcal{U}$ и $\mathcal{V}$ тоже унитарное пространство
со своим скалярным произведением $\scalarproduct{\cdot}{\cdot}_\mathcal{V}$.

Пусть $\mathcal{A} : \mathcal{U} \rightarrow \mathcal{V}$ --- оператор, действующий из пространства $\mathcal{U}$ в пространство $\mathcal{V}$.
Оператор $\mathcal{A}^*$ называется сопряженным к $\mathcal{A}$, если:
\begin{gather}
    \scalarproduct{\mathcal{A} u}{v}_{\mathcal{V}} = \scalarproduct{u}{\mathcal{A}^* v}_{\mathcal{U}}
    \label{rayleigh:operator:scalars_equality}, \\
    \forall u \in \mathcal{U}, \forall v \in \mathcal{V}
    \notag.
\end{gather}

Возникает вопрос об условиях существовании сопряженного оператора $\mathcal{A}^*$. Как будет показано далее, при некоторых условиях сопряженный оператор существует,
более того его можно построить.

Пусть оператор $\mathcal{A}$ является линейным, то есть для любых $u_1, u_2 \in \mathcal{U}$ и любого числа $\lambda \in \mathbb{C}$ выполняются равенства:
\begin{enumerate}
    \item $\mathcal{A}(u_1 + u_2) = \mathcal{A}(u_1) + \mathcal{A}(u_2)$,
    \item $\mathcal{A}(\lambda u_1) = \lambda \mathcal{A}(u_1)$.
\end{enumerate}

Линейный оператор можно задать следующим образом: выбрать базисный набор векторов $e_i$ пространства $\mathcal{U}$ и определить действие оператора на базисные векторы
$\mathcal{A} e_i$, затем используя линейность для всякого элемента $u \in \mathcal{U}$:
\[
    u = c_1 e_1 + c_2 e_2 + \dots,
\]
получим:
\[
    \mathcal{A} u
    = \mathcal{A} ( c_1 e_1 + c_2 e_2 + \dots )
    = c_1 \mathcal{A} e_1 + c_2 \mathcal{A} e_2 + \dots
\]

При построении сопряженного оператора $\mathcal{A}^*$ можно использовать такой же способ его определения.

Пусть пространства $\mathcal{U}$ и $\mathcal{V}$ являются конечномерными и набор векторов $e = \{ e_1, \dots, e_n \}$ является ортонормированными базисом
в пространстве $U$, а набор векторов $f = \{f_1, \dots, f_m \}$ является ортономированным базисом в пространстве $\mathcal{V}$. Пусть
$A = [a_{ij}] \in \Cspace{m \times n}$ является матрицей оператора $\mathcal{A}$ в базисах $e$ и $f$, тогда:
\[
    \mathcal{A} e_j = a_{1j} f_1 + \dots + a_{mj} f_m,
\]
тем самым определено действие оператора $\mathcal{A}$ на базисные векторы $e_j$.

Попробуем определить оператор $\mathcal{B}$ так, чтобы равенство \eqref{rayleigh:operator:scalars_equality} выполнялось для набора векторов $e$ и $f$:
\begin{equation}
    \label{rayleigh:operator:basis_scalars_equality}
    \scalarproduct{\mathcal{A} e_j}{f_i}_\mathcal{V} = \scalarproduct{e_j}{\mathcal{B} f_i}_\mathcal{U} .
\end{equation}
Поскольку $\mathcal{B} f_i \in \mathcal{U}$, то:
\[
    \mathcal{B} f_i = b_{1i} e_1 + \dots + b_{ni} e_n ,
\]
причём коэффициенты $b_{ji}$ следует выбирать таким образом, чтобы выполнялось равенство \eqref{rayleigh:operator:basis_scalars_equality}, которое в силу
ортонормированности наборов $e$ и $f$ приводит к равенству:
\begin{align*}
    \scalarproduct{\mathcal{A} e_j}{f_i}_\mathcal{V} & = \scalarproduct{e_j}{\mathcal{B} f_i}_\mathcal{U} , \\
    \scalarproduct{a_{1j} f_1 + \dots + a_{mj} f_m}{f_i}_\mathcal{V} & = \scalarproduct{e_j}{b_{1i} e_1 + \dots + b_{ni} e_n}_\mathcal{U} , \\
    a_{1j} \scalarproduct{f_1}{f_i}_\mathcal{V} + \dots + a_{mj} \scalarproduct{f_m}{f_i}_\mathcal{V} & = \overline{b_{1i}} \scalarproduct{e_j}{e_1}_\mathcal{U} + \dots + \overline{b_{ni}} \scalarproduct{e_j}{e_n}_\mathcal{U} , \\
    a_{ij} \scalarproduct{f_i}{f_i}_\mathcal{V} & = \overline{b_{ji}} \scalarproduct{e_j}{e_j}_\mathcal{U} , \\
    a_{ij} & = \overline{b_{ji}} , \\
    \overline{a_{ij}} & = b_{ji} .
\end{align*}
Таким образом, определено действие оператора $\mathcal{B}$ на базисные векторы $f_i$:
\begin{equation}
    \label{rayleigh:operator:basis_images}
    \mathcal{B} f_i = \overline{a_{i1}} e_1 + \dots + \overline{a_{in}} e_n .
\end{equation}

Теперь распространим действие оператора $\mathcal{B}$ на все векторы $v \in \mathcal{V}$, сделав оператор $\mathcal{B}$ линейным, пусть
\[
    v = v_1 f_1 + \dots + v_m f_m,
\]
тогда
\[
    \mathcal{B} v = v_1 \mathcal{B} f_1 + \dots + v_m \mathcal{B} f_m .
\]
Проверим, что при таком определении оператор $\mathcal{B}$ оказывается сопряженным к оператору $\mathcal{A}$. Пусть $u \in \mathcal{U}$ --- произвольный вектор
пространства $U$:
\[
    u = u_1 e_1 + \dots + u_n e_n,
\]
тогда
\begin{multline*}
    \scalarproduct{\mathcal{A} u}{v}_\mathcal{V}
    = \scalarproduct{\mathcal{A} (u_1 e_1 + \dots + u_n e_n)}{v}_\mathcal{V}
    = \scalarproduct{u_1 \mathcal{A} e_1 + \dots + u_n \mathcal{A} e_n}{v}_\mathcal{V} = \\
    %
    = \scalarproduct{u_1 \mathcal{A} e_1 + \dots + u_n \mathcal{A} e_n}{v_1 f_1 + \dots + v_m f_m}_\mathcal{V} = \\
    %
    = \sum_{i=1}^n u_i \sum_{j=1}^m \overline{v_j} \scalarproduct{\mathcal{A} e_i}{f_j}_\mathcal{V}
    = \sum_{i=1}^n u_i \sum_{j=1}^m \overline{v_j} \scalarproduct{e_i}{\mathcal{B} f_j}_\mathcal{U} = \\
    %
    = \scalarproduct{u_1 e_i + \dots + u_n e_n}{v_1 \mathcal{B} f_1 + \dots + v_m \mathcal{B} f_m}_\mathcal{U} = \\
    %
    = \scalarproduct{u}{v_1 \mathcal{B} f_1 + \dots + v_m \mathcal{B} f_m}_\mathcal{U}
    = \scalarproduct{u}{\mathcal{B} (v_1 f_1 + \dots + v_m f_m)}_\mathcal{U}
    = \scalarproduct{u}{\mathcal{B} v}_\mathcal{U} .
\end{multline*}
Таким образом, $\mathcal{B}$ является сопряженным к $\mathcal{A}$:
\[
    \mathcal{A}^* = \mathcal{B}.
\]
Кроме того, из равенства \eqref{rayleigh:operator:basis_images} следует, что матрица $B$ оператора $\mathcal{B}$ в базисах $f$ и $e$:
\[
    B =
    \begin{pmatrix}
        \overline{a_{11}} & \overline{a_{21}} & \dots  & \overline{a_{m1}} \\
        \overline{a_{12}} & \overline{a_{22}} & \dots  & \overline{a_{m2}} \\
        \vdots            & \vdots            & \ddots & \vdots            \\
        \overline{a_{1n}} & \overline{a_{2n}} & \dots  & \overline{a_{mn}}
    \end{pmatrix}
    =
    \begin{pmatrix}
        \overline{a_{11}} & \overline{a_{12}} & \dots  & \overline{a_{1n}} \\
        \overline{a_{21}} & \overline{a_{22}} & \dots  & \overline{a_{2n}} \\
        \vdots            & \vdots            & \ddots & \vdots            \\
        \overline{a_{n1}} & \overline{a_{n2}} & \dots  & \overline{a_{nm}}
    \end{pmatrix}^T
    = \overline{A}^T .
\]
То есть матрица $A^*$ сопряженного оператора $\mathcal{A}^*$:
\[
    A^* = \overline{A}^T .
\]


\section{Спектральное разложение}

Пусть $A$ --- матрица оператора $\mathcal{A}$ в ортонормированных базисах пространств $\mathcal{U}$ и $\mathcal{V}$. Согласно теореме Шура любая матрицы $A$
унитарно подобна верхнетреугольной матрице, то есть существует унитарная матрица $U$:
\[
    U^* U = I,
\]
такая что
\[
    U^* A U = R, \\
\]
где $R$ --- верхнетреугольная матрица. Умножая последнее равенство слева на $U$ и справа на $U^*$, получим
\begin{align}
    U^* A U & = R,
    \notag \\
    U U^* A U U^* & = U R U^*,
    \notag \\
    A & = U R U^*
    \label{rayleigh:normal:schur_decomposition}
\end{align}

Оператор $\mathcal{A}$ называется нормальным, если:
\[
    \mathcal{A}^* \mathcal{A} = \mathcal{A} \mathcal{A}^* .
\]
Отсюда следует равенство матриц:
\[
    A^* A = A A^* .
\]
Используя в последнем равенстве разложение \eqref{rayleigh:normal:schur_decomposition}, получим:
\begin{align*}
    \left ( U R U^* \right ) ^* U R U^* & = U R U^* \left ( U R U^* \right )^* , \\
    U R^* U^* U R U^* & = U R U^* U R U^* , \\
    U R^* R U^* & = U R R^* U^* , \\
    U^* U R^* R U^* U & = U^* U R R^* U^* U, \\
    R^* R & = R R^* .
\end{align*}
Поскольку $R$ --- верхнетреугольная матрица, то последнее равенство возможно только в том случае, когда $R$ --- диагональная:
\[
    R = \Lambda,
\]
где $\Lambda$ --- диагональная матрица.

Таким образом, если $\mathcal{A}$ --- нормальный оператор, то матрица $A$ такого оператора диагонализуема:
\begin{equation}
    \label{rayleigh:normal:spectral_decomposition}
    A = U \Lambda U^* .
\end{equation}
Такое разложение называется спектральным, и называется так потому, что, умножая справа на $U$, получим равенство:
\[
    A U = U \Lambda,
\]
откуда следует, что столбцы матрицы $U$ определяют собственные векторы, а элементы матрицы $\Lambda$ являются собственными числами.


\section{Эрмитовый оператор}

Частным случаям нормального оператора $\mathcal{A} : \mathcal{U} \rightarrow \mathcal{U}$ является самосопряженный (эрмитов) оператор:
\[
    \mathcal{A}^* = \mathcal{A}.
\]

Отсюда, матрица $A$ оператора $\mathcal{A}$ является самосопряженной (эрмитовой):
\[
    A^* = A.
\]
Используя спектральное разложение \eqref{rayleigh:normal:spectral_decomposition}, получим:
\begin{align*}
    \left ( U \Lambda U^* \right )^* & = U \Lambda U^* , \\
    U \Lambda^* U^* & = U \Lambda U^* , \\
    U^* U \Lambda^* U^* U & = U^* U \Lambda U^* U, \\
    \Lambda^* & = \Lambda .
\end{align*}
Если $\lambda_i$ --- диагональный элемент матрицы $\Lambda$, тогда:
\[
    \overline{\lambda_i} = \lambda_i
\]
а это возможно тогда и только тогда, когда число $\lambda_i$ --- вещественное:
\[
    \lambda_i \in \mathbb{R}.
\]
Таким образом, у эрмитова оператора все собственные значения вещественные.

Рассмотрим квадратичную форму с эрмитовым оператором:
\[
    \scalarproduct{\mathcal{A} u}{u}
    = \scalarproduct{u}{\mathcal{A}^* u}
    = \scalarproduct{u}{\mathcal{A} u}
    = \overline{\scalarproduct{\mathcal{A} u}{u}}
\]
Отсюда следует, что значения квадратичной формы $\scalarproduct{\mathcal{A} u}{u}$ являются вещественными при всех векторах $u$. Пусть дополнительно оператор $A$
является положительно определенным, то есть для всех векторов $u$:
\[
    \scalarproduct{\mathcal{A} u}{u} > 0
\]
Если $u_i$ --- собственный вектор, соответствующей собственному значению $\lambda_i$, тогда:
\begin{align*}
    \scalarproduct{u_i}{u_i} & > 0 , \\
    \scalarproduct{\lambda_i u_i}{u_i} & > 0 , \\
    \lambda_i \scalarproduct{u_i}{u_i} & > 0 , \\
    \lambda_i \norm{u_i}^2 & > 0 , \\
    \lambda_i > 0 ,
\end{align*}
поскольку собственный вектор $u_i \neq 0$ и $\norm{u_i} > 0$. Таким образом, если $\mathcal{A} > 0$, то все его собственные числа положительны, и для диагональной
матрицы $\Lambda$:
\[
    \Lambda
    = \begin{pmatrix}
          \lambda_1 & 0         & \dots  & 0         \\
          0         & \lambda_2 & \dots  & 0         \\
          \vdots    & \vdots    & \ddots & \vdots    \\
          0         & 0         & \dots  & \lambda_n
    \end{pmatrix}
\]
определен квадратный корень:
\[
    \Lambda^\frac{1}{2}
    = \begin{pmatrix}
          \lambda_1^\frac{1}{2} & 0                     & \dots  & 0                     \\
          0                     & \lambda_2^\frac{1}{2} & \dots  & 0                     \\
          \vdots                & \vdots                & \ddots & \vdots                \\
          0                     & 0                     & \dots  & \lambda_n^\frac{1}{2}
    \end{pmatrix}
    .
\]
Тогда из спектрального разложения матрицы $A$ оператора \eqref{rayleigh:normal:spectral_decomposition}:
\[
    A
    = U \Lambda U^*
    = U \Lambda^\frac{1}{2} \Lambda^\frac{1}{2} U^*
    = \left ( U \Lambda^\frac{1}{2} \right ) \left ( U \Lambda^\frac{1}{2} \right )^*
    = A^\frac{1}{2} \left ( A^\frac{1}{2} \right )^*,
\]
где
\[
    A^\frac{1}{2} = U \Lambda^\frac{1}{2}
\]
квадратный корень из эрмитовой матрицы $A$.


\section{Вещественные квадратичные формы}

Если матрица $A$ определена ($A > 0$, $A \ge 0$, $A \le 0$ или $A < 0$), то при всех $x$ квадратичную форму $x^* A x$ можно сравнивать с нулём, а значит она является
вещественным числом. Отсюда сразу следует, что $A$ является эрмитовой. Действительно:
\begin{gather*}
    x^* A x \in \mathbb{R} , \\
    x^* A x = ( x^* A x )^* , \\
    x^* A x = x^* A^* x .
\end{gather*}

Возьмем в качестве $x$ векторы вида $e_k = (0, \dots, 0, 1, 0, \dots, 0)$ с одной единицей, тогда из равенства квадратичных форм следует, что диагональные
элементы матриц $A$ и $A^*$ вещественны и одинаковы:
\begin{gather*}
    e_k^* A e_k = e_k^* A^* e_k , \\
    a_{kk} = a_{kk}^* .
\end{gather*}
Возьмем в качестве $x$ векторы $e_{kj} = (0, \dots, 0, 1, 0, \dots, 0, 1, 0, \dots, 0)$ с двумя единицами, тогда из равенства квадратичных форм следует,
что внедиагональные элементы сопряжены:
\begin{gather*}
    e_{kj}^* A e_{kj} = e_{kj}^* A^* e_{kj} , \\
    a_{kk} + a_{kj} + a_{jk} + a_{jj} = a_{kk}^* + a_{jk}^* + a_{kj}^* + a_{jj}^* , \\
    a_{kk} + a_{kj} + a_{jk} + a_{jj} = a_{kk} + a_{jk}^* + a_{kj}^* + a_{jj} , \\
    a_{kj} + a_{jk} = a_{jk}^* + a_{kj}^* , \\
    a_{kj} + a_{jk} = a_{kj}^* + a_{jk}^* , \\
    a_{kj} + a_{jk} = ( a_{kj} + a_{jk} )^* , \\
    \image{a_{kj} + a_{jk}} = 0 , \\
    \image{a_{kj}} = - \image{a_{jk}} .
\end{gather*}
Возьмем в качестве $x$ векторы $e_{kj} = (0, \dots, 0, 1, 0, \dots, 0, -1, 0, \dots, 0)$ с двумя единицами, тогда из равенства квадратичных форм следует,
что внедиагональные элементы сопряжены:
\begin{gather*}
    e_{kj}^* A e_{kj} = e_{kj}^* A^* e_{kj} , \\
    a_{kk} + i a_{kj} - i a_{jk} - i^2 a_{jj} = a_{kk}^* + i a_{jk}^* - i a_{kj}^* - i^2 a_{jj}^* , \\
    a_{kk} + i a_{kj} - i a_{jk} + a_{jj} = a_{kk}^* + i a_{jk}^* - i a_{kj}^* + a_{jj}^* , \\
    a_{kk} + i a_{kj} - i a_{jk} + a_{jj} = a_{kk} + i a_{jk}^* - i a_{kj}^* + a_{jj} , \\
    i a_{kj} - i a_{jk} = i a_{jk}^* - i a_{kj}^* , \\
    i a_{kj} - i a_{jk} = ( - i a_{jk} + i a_{kj} )^* , \\
    i a_{kj} - i a_{jk} = ( i a_{kj} - i a_{jk})^* , \\
    \image{i a_{kj} - i a_{jk}} = 0 , \\
    \real{a_{kj} - a_{jk}} = 0 , \\
    \real{a_{kj}} = \real{a_{jk}} .
\end{gather*}
Таким образом,
\[
    A = A^*.
\]


\section{Отношение Релея}

В радиолокации физические колебательные процессы часто описываются с помощью вектора комплексных амплитуд $x \in \Cspace{n}$. Кроме того, выделение линейной части
преобразований приводит к векторам $Fx$. Далее, обычно, интересуются энергией, которая пропорциональна квадратам норм:
\begin{gather*}
    \norm{x}^2 = x^* x , \\
    \norm{F x}^2 = x^* F^* F x ,
\end{gather*}
и сравнением энергий, которое приводит к отношениям вида:
\begin{gather*}
    \rho(x) = \frac{x^* F^* F x}{x^* x}.
\end{gather*}
Легко видеть, что матрица $F^* F$ является эрмитовой:
\[
    ( F^* F )^* = F^* F .
\]

В более общем случае рассматривается отношение:
\[
    \rho(x) = \frac{x^* A x}{x^* B x},
\]
где $A$ и $B$ --- эрмитовы матрицы и $B > 0$.

Для положительно определенной матрицы $B > 0$ существует квадратный корень $B^\frac{1}{2}$:
\begin{gather*}
    B = B^\frac{1}{2} ( B^\frac{1}{2} )^* , \\
    %
    B^\frac{1}{2} = U_B D_B^\frac{1}{2} ,
\end{gather*}
причём
\[
    \det B^\frac{1}{2}
    = \det ( U_B D_B^\frac{1}{2} )
    = \det U_B \cdot \det D_B^\frac{1}{2}
    = 1 \cdot \det D_B^\frac{1}{2}
    > 0 ,
\]
поэтому существует обратная матрица $B^{-\frac{1}{2}}$.

Отношение Релея $\rho(x)$ можно представить в виде:
\begin{gather*}
    \rho(x)
    = \frac{x^* A x}{x^* B^\frac{1}{2} ( B^\frac{1}{2} )^* x}
    = \frac{x^* B^\frac{1}{2} B^{-\frac{1}{2}} A ( B^{-\frac{1}{2}} )^* ( B^\frac{1}{2} )^* x}{x^* B^\frac{1}{2} ( B^\frac{1}{2} )^* x} , \\
    %
    \rho(y)
    = \frac{y^* B^{-\frac{1}{2}} A ( B^{-\frac{1}{2}} )^* y}{y^* y}
    = \frac{y^* C y}{y^* y}, \\
    %
    C = B^{-\frac{1}{2}} A ( B^{-\frac{1}{2}} )^*, \\
    %
    y = ( B^\frac{1}{2} )^* x .
\end{gather*}

Заметим, что матрица $C$ является эрмитовой:
\[
    C^*
    = ( B^{-\frac{1}{2}} A ( B^{-\frac{1}{2}} )^* )^*
    = B^{-\frac{1}{2}} A^* ( B^{-\frac{1}{2}} )^*
    = B^{-\frac{1}{2}} A ( B^{-\frac{1}{2}} )^*
    = C,
\]
поэтому она подобна диагональной матрице:
\[
    C = U_C D_C U_C^* .
\]
Используя это представление матрицы $C$, преобразуем отношение Релея к виду:
\begin{gather*}
    \rho(y)
    = \frac{y^* U_C D_C U_C^* y}{y^* y}
    = \frac{y^* U_C D_C U_C^* y}{y^* U_C U_C^* y} , \\
    %
    \rho(z)
    = \frac{z^* D_C z}{z^* z} , \\
    %
    z = U_C^* y
\end{gather*}

Отношение Релея не зависит от величины вектора $z$, а только от его направления, поскольку для вектора $\alpha z$:
\[
    \rho(\alpha z)
    = \frac{\alpha^* z^* D_C \alpha z}{ \alpha^* z^* \alpha z}
    = \frac{\modulus{\alpha}^2 \cdot z^* D_C z}{ \modulus{\alpha}^2 \cdot z^* z}
    = \frac{z^* D_C z}{z^* z}
    = \rho(z) .
\]
Таким образом, можно ограничится рассмотрением векторов $z$, для которых:
\begin{gather*}
    z^* z = 1 , \\
    %
    \rho(z) = z^* D_C z .
\end{gather*}

Пусть
\begin{gather*}
    D_C
    = \begin{pmatrix}
          \lambda_1 & 0         & \dots  & 0         \\
          0         & \lambda_2 & \dots  & 0         \\
          \vdots    & \vdots    & \ddots & \vdots    \\
          0         & 0         & \dots  & \lambda_n
    \end{pmatrix} , \\
    %
    \lambda_1 \ge \lambda_2 \ge \dots \ge \lambda_n.
\end{gather*}
тогда
\begin{gather*}
    \rho(z)
    = \lambda_1 \modulus{z_1}^2 + \lambda_2 \modulus{z_2}^2 + \dots + \lambda_n \modulus{z_n}^2, \\
    %
    \modulus{z_1}^2 + \modulus{z_2}^2 + \dots + \modulus{z_n}^2 = 1.
\end{gather*}
Из последнего равенства следует, что
\[
    0 \le \modulus{z_i}^2 \le 1 ,
\]
поэтому
\begin{gather*}
    \lambda_n \le \rho(z) \le \lambda_1 .
\end{gather*}

Максимальное значение отношение Релея достигает при векторе $z_{max}$:
\[
    z_{max}
    = \begin{pmatrix}
          1     \\
          0     \\
          \dots \\
          0
    \end{pmatrix} ,
\]
которому соответствует вектор $y_{max}$:
\begin{align*}
    z_{max} & = U_C^* y_{max} , \\
    U_C z_{max} & = y_{max} ,
\end{align*}
которому соответствует вектор $x_{max}$:
\begin{align*}
    \left ( B^\frac{1}{2} \right )^* x_{max} & = y_{max} = U_C z_{max} , \\
    \left ( U_B D_B^\frac{1}{2} \right )^* x_{max} & = U_C z_{max} , \\
    D_B^\frac{1}{2} U_B^* x_{max} & = U_C z_{max} , \\
    U_B^* x_{max} & = D_B^{-\frac{1}{2}} U_C z_{max} , \\
    x_{max} & = U_B D_B^{-\frac{1}{2}} U_C z_{max} .
\end{align*}
Аналогично минимальное значение отношение Релея достигает при векторе $z_{min}$:
\[
    z_{min}
    = \begin{pmatrix}
          0     \\
          \dots \\
          0     \\
          1
    \end{pmatrix} ,
\]
которому соответствует вектор $x_{min}$:
\[
    x_{min} = U_B D_B^{-\frac{1}{2}} U_C z_{min}
\]

    % операции в Matlab
    \chapter{Операции в Matlab}

\section{Вычислительные}

\subsection{Арифметические}

Вектор--строка:
\matlab{x = [1 2 3]}

Вектор--столбец:
\matlab{y = [1; 2; 3]}

Матрица:
\matlab{A = [1 2 3; 4 5 6]}

Сложение:
\begin{Matlab}
    \Mcommand{B = [-2 -5 8; 3 1 -8]}
    \Mcommand{C = A + B}
\end{Matlab}

Умножение вектора на матрицу:
\matlab{z = A * y}

Сопряжение (транспонирование)
\matlab{Ac = A'}

Вычисление собственных чисел и векторов:
\begin{Matlab}
    \Mcommand{A = [-3 2; 3 -5];}
    \Mcommand{[V, D] = eig(A);}
\end{Matlab}

\subsection{Поэлементные}

Поэлементное умножение матриц:
\begin{Matlab}
    \Mcommand{A = [1 2 3; 4 5 6];}
    \Mcommand{B = [-2 -5 8; 3 1 -8];}
    \Mcommand{C = A .* B;}
\end{Matlab}

Поэлементное возведение в квадрат:
\begin{Matlab}
    \Mcommand{x = [1 3 5 7];}
    \Mcommand{y = x.$\hat{}$ 2;}
\end{Matlab}

\section{Графические}

График функции в 2D:
\begin{Matlab}
    \Mcommand{x = -1:0.1:1;}
    \Mcommand{y = x.$\hat{}$ 3;}
    \Mcommand{plot(x,y)}
\end{Matlab}

График кривой в 3D:
\begin{Matlab}
    \Mcommand{a = [ 0:0.1:2*pi 2*pi];}
    \Mcommand{x = cos(2*a);}
    \Mcommand{y = sin(2*a);}
    \Mcommand{z = a;}
\end{Matlab}

    % излучение сигналов
    \chapter{Излучение}

\section{Диаграмма направленности}

Рассматривается система $n$ излучателей, местоположения которых в декартовой системе описывается векторами $r_k$.

\textcolor{red}{Рисунок.}

К излучателям подводятся сигналы с огибающими $a_k$, объединёнными в вектор $a$:
\[
    a = \begin{pmatrix}
        a_1    \\
        a_2    \\
        \vdots \\
        a_n
    \end{pmatrix}
    .
\]

Система излучателей формирует электромагнитное поле, которое распространяется в пространстве в радиальном направлении.

Электрическое поле характеризуется вектором электрической напряжённости $E$. Известно, что вектор напряженности $E$ совершает колебания в плоскости,
ортогональной направлению распространения электромагнитной волны. В общем случае вектор $E$ описывает эллипс. Сигналы с огибающими $a$, которые подаются
на входы излучателей, имеют одинаковую частоту $\omega$, вектор напряженности $E$ колеблется с такой же частотой $\omega$, поэтому колебания вектора
напряженности $E$ тоже будем описывать с помощью огибающей.

Описывать колебания вектора напряженности $E$ в декартовой системе неудобно, поскольку плоскость, в которой вектор $E$ совершает колебания, изменяется
при изменении рассматриваемой точки поля. Описывать колебания вектора напряжённости $E$ удобно в сферической системе координат, в которой единичный
вектор радиуса $u_\rho$ направлен в сторону распространения волны, а единичные векторы азимута $u_\varphi$ и угла места $u_\theta$ располагаются
в ортогональной плоскости. При изменении рассматриваемой точки, орты $u_\rho$, $u_\varphi$ и $u_\theta$ сами разворачиваются, так что $u_\rho$
всегда направлен радиально и совпадает с направлением распространения волны, а векторы $u_\varphi$ и $u_\theta$ всегда располагаются в плоскости,
ортогональной $u_\rho$, в которой колеблется вектор напряжённости $E$.

\textcolor{red}{Рисунок.}

Колебания вектора напряжённости $E$ будет описывать через огибающие колебаний двух его проекций: $E_\varphi$ --- на азимутальный орт $u_\varphi$
и $E_\theta$ --- на угломестный орт $u_\theta$ (радиальная компонента вектора напряжённости $E$ равна нулю).

Таким образом, вектор напряжённости $E$ имеет две угловые компоненты:
\[
    E =
    \begin{pmatrix}
        E_\varphi \\
        E_\theta
    \end{pmatrix}
    = E(a, r)
\]
которые зависят от входных сигналов излучателей $a$ и рассматриваемой точки с радиус-вектором $r$.

В дальней зоне, при достаточном удалении от излучателей, напряженность можно приближенно представить в виде:
\begin{equation}
    \label{emission:diagram:tension}
    E(a, r) \approx F(\varphi, \theta) \cdot a \cdot \frac{e^{i \scalarproduct{w}{r}}}{\modulus{r}},
\end{equation}
где $\varphi$ и $\theta$ --- азимут и угол места радиус-вектора $r$ в сферической системе (они задают угловое направление),
$\modulus{r}$ --- длина радиус вектора $r$, $w$ --- волновой вектор. Матрица $F(\varphi, \theta)$ называется диаграммой направленности
системы излучателей. Эта матрица является коэффициентом при $a$ в разложении функции напряжённости $E(a, r)$.

Множитель $e^{i \scalarproduct{w}{r}}$ показывает свдиг фазы, а множитель $\frac{1}{\modulus{r}}$ --- затухание амплитуды
колебаний при удалении от центра координат.

Волновой вектор $w$ --- это вектор, который в каждой точке направлен в сторону распространения волны и имеет величину, определяемую отношением
скорости изменения состояния к скорости распространения состояния. Распространение волны --- это процесс, при котором точки среды изменяют свое состояние
(скалярное либо векторное) и передают его другим точкам среды. Рассмотрим пример: точка совершает колебательные движения с частотой $\omega$
вдоль оси ординат, а текущую координату (состояние) сносит в направлении оси абсцисс с некоторой скоростью $v$ (Как если бы Вы левой рукой водили
ручкой вверх и вниз, а правой рукой тащили бы лист бумаги вправо).

\textcolor{red}{Рисунок}.

Волновой вектор $w$ направлен в сторону оси абсцисс, его величина:
\[
    w
    = \frac{\omega}{v}
    = \frac{\omega T}{v T}
    = \frac{2 \pi}{\lambda} ,
\]
где $T$ --- период колебаний точки, $\lambda$ --- длина волны.

Диаграмма направленности $F(\varphi, \theta)$ --- матрица размера $2 \times n$, которая имеет вид:
\[
    F(\varphi, \theta)
    = \begin{pmatrix}
        f_{1,\varphi}(\varphi, \theta) e^{-i \scalarproduct{w}{r_1}} & \dots & f_{k,\varphi}(\varphi, \theta) e^{-i \scalarproduct{w}{r_k}} & \dots & f_{n,\varphi}(\varphi, \theta) e^{-i \scalarproduct{w}{r_n}} \\
        f_{1,\theta}(\varphi, \theta) e^{-i \scalarproduct{w}{r_1}}  & \dots & f_{k,\theta}(\varphi, \theta) e^{-i \scalarproduct{w}{r_k}}  & \dots & f_{n,\theta}(\varphi, \theta) e^{-i \scalarproduct{w}{r_n}}  \\
    \end{pmatrix} ,
\]
вектор
\[
    f_k(\varphi, \theta)
    = \begin{pmatrix}
        f_{k,\varphi}(\varphi, \theta) \\
        f_{k,\theta}(\varphi, \theta)
    \end{pmatrix}
\]
называется парциальной диаграммой направленности $k$-го излучателя: $f_{k,\varphi}(\varphi, \theta)$ определяет вклад $k$-го излучателя в азимутальную
компоненту напряженности, а $f_{k,\varphi}(\varphi, \theta)$ --- в угломестную.


\section{Коэффициент усиления}

К системе излучателей подводятся сигналы с огибающими $a$, мощность которых $P_{inp}$:
\[
    P_{inp} = \norm{a}_2^2.
\]
Система излучателей преобразует сигналы в излучение, при этом излучение неоднородно в пространстве и мощность излучения может быть различной в
различных направлениях. При этом суммарная мощность излучения не превышает входную мощность $P_{inp}$. В точке, определяемой радиус-вектором,
плотность потока мощности $\Pi(a, r)$:
\[
    \Pi(a, r) = \frac{1}{Z_0} \norm{E(a, r)}_2^2 ,
\]
где $Z_0 = 120 \pi$ --- волновое сопротивление пространства.

\subsection{Вычисление}

Распределение мощности излучения характеризуется реализованным коэффициентом усиления $G(\varphi, \theta)$:
\[
    G(\varphi, \theta)
    = \frac{\Pi(a, r)}{\frac{P_{inp}}{4 \pi \modulus{r}^2}},
\]
где в числителе стоит плотность потока мощности $\Pi(a, r)$ в точке с радиус-вектором $r$, а в знаменателе --- среднее значение мощности,
усреднённое по поверхности сферы радиуса $\modulus{r}$. Преобразуя выражение для реализованного коэффициента усиления, получим:
\begin{multline}
    G(\varphi, \theta)
    = \frac{4 \pi \modulus{r}^2}{P_{inp}} \cdot \Pi(a, r)
    = \frac{4 \pi \modulus{r}^2}{P_{inp}} \cdot \frac{1}{Z_0} \norm{E(a, r))}_2^2
    \approx \frac{4 \pi \modulus{r}^2}{P_{inp}} \cdot \frac{1}{Z_0} \norm{F(\varphi, \theta) a \cdot \frac{e^{i \scalarproduct{w}{r}}}{\modulus{r}}}_2^2 = \\
    %
    = \frac{4 \pi \modulus{r}^2}{P_{inp}} \cdot \frac{1}{Z_0} \norm{F(\varphi, \theta) a}_2^2 \frac{\modulus{e^{i \scalarproduct{w}{r}}}^2}{\modulus{r}^2}
    = \frac{4 \pi \modulus{r}^2}{P_{inp}} \cdot \frac{1}{Z_0} \norm{F(\varphi, \theta) a}_2^2 \frac{1}{\modulus{r}^2}
    = \frac{4 \pi}{Z_0} \cdot \frac{\norm{F(\varphi, \theta) a}^2}{P_{inp}} .
\end{multline}

Если направление $(\varphi, \theta)$ фиксировано, то коэффициент усиления пропорционален отношению:
\begin{equation}
    \label{emission:emitter:gain:rayleigh}
    \rho(a)
    = \frac{\norm{F(\varphi, \theta) a}_2^2}{P_{inp}}
    = \frac{\norm{F(\varphi, \theta) a}_2^2}{\norm{a}_2^2}
    = \frac{a^* F^*(\varphi, \theta) F(\varphi, \theta) a}{a^* a}
\end{equation}
и возникает вопрос, каким образом нужно сформировать огибающие входных сигналов $a$ чтобы отношение $\rho(a)$ и коэффициент усиления в заданном
направлении $(\varphi, \theta)$ оказался наибольшим?

Отношение $\rho(a)$ является отношением Релея: числитель --- квадратичная форма с эрмитовой матрией $F^* F$, знаменатель --- квадрат нормы $a$.
Наибольшее значение $G_{max}$:
\[
    G_{max} = \max \limits_{a} G
\]
достигается в направлении $a_{max}$, совпадающим с направлением собственных векторов, соответствующих наибольшему собственному числу $\lambda_{max}$
матрицы $F^*F$. Вектор $a_{max}$ и число $\lambda_{max}$ удовлетворяют уравнению:
\begin{equation}
    \label{emission:emitter:gain:optimal_input}
    F^* F a_{max} = \lambda_{max} a_{max}
\end{equation}

Первый способ нахождения $a_{max}$ заключается в решении системы~\eqref{emission:emitter:gain:optimal_input}. В некоторых случаях матрица
$F^*F$ оказывается сложной для определения собственных чисел и векторов, поскольку порядок матрицы равен количеству излучателей $n$.

Второй способ определения $a_{max}$ связан с нахождение вектора оптимальной поляризации $p_{max}$. Домножим левую и правую части
уравнения~\eqref{emission:emitter:gain:optimal_input} на $F$ слева:
\[
    F F^* F a_{max} = \lambda_{max} F a_{max} .
\]
В левой и правой частях получился вектор $F a_{max}$, который обозначим $p_{pmax}$:
\[
    p_{max} = F a_{max} .
\]
Таким образом,
\[
    F F^* p_{max} = \lambda_{max} p_{max} .
\]
и вектор $p_{max}$ является собственным вектором матрицы $F F^*$, соответствующий тому же собственному числу $\lambda_{max}$.

Пусть $f_\varphi$ и $f_\theta$ --- строки матрицы $F$, тогда:
\[
    F F^*
    = \begin{pmatrix}
        f_\varphi \\
        f_\theta
    \end{pmatrix}
    \begin{pmatrix}
        f_\varphi^* & f_\theta^*
    \end{pmatrix}
    = \begin{pmatrix}
        f_\varphi f_\varphi^*  & f_\varphi f_\theta^*
        f_{\theta} f_\varphi^* & f_\theta f_\theta^*  \\
    \end{pmatrix} .
\]
У матрицы $F F^*$ два собственных значения $\lambda_{min}$ и $\lambda_{max}$, которые являются корнями характеристического уравнения:
\begin{multline*}
    \begin{vmatrix}
        f_\theta f_\theta^* - \lambda & f_{\theta} f_\varphi^*          \\
        f_\varphi f_\theta^*          & f_\varphi f_\varphi^* - \lambda
    \end{vmatrix}
    = (f_\theta f_\theta^* - \lambda) (f_\varphi f_\varphi^* - \lambda) - f_\varphi f_\theta^* f_{\theta} f_\varphi^* = \\
    %
    = \lambda^2 - ( f_\theta f_\theta^* + f_\varphi f_\varphi^* ) \lambda + f_\theta f_\theta^* f_\varphi f_\varphi^* - f_\varphi f_\theta^* f_{\theta} f_\varphi^* = \\
    %
    = \lambda^2 - \tr(F F^*) \lambda + \det(F F^*) ,
\end{multline*}
где $\tr(F F^*)$ и $\det(F F^*)$ --- след и определитель матрицы $F F^*$. Корни характеристического уравнения:
\begin{align}
    \lambda_{min} & = \frac{\tr(F F^*) - \sqrt{\tr^2(F F^*) - 4 \det(F F^*)}}{2} \label{emission:emitter:gain:minimum_eigenvalue} , \\
    \lambda_{max} & = \frac{\tr(F F^*) + \sqrt{\tr^2(F F^*) - 4 \det(F F^*)}}{2} \label{emission:emitter:gain:maximum_eigenvalue}.
\end{align}
Вектор $p_{max}$ находим как решение однородной системы:
\begin{gather*}
    F F^* p_{max} = \lambda_{max} p_{max} , \\
    ( F F^* - \lambda I ) p_{max} = 0 .
\end{gather*}
Из равенства \eqref{emission:emitter:gain:optimal_input}:
\begin{align*}
    F^* F a_{max}                       & = \lambda_{max} a_{max} , \\
    F^* p_{max}                         & = \lambda_{max} a_{max} , \\
    \frac{1}{\lambda_{max}} F^* p_{max} & = a_{max} .
\end{align*}

\subsection{Пример 1: один канал}

Излучение только одного канала.

Пусть выбрано и зафиксировано направление $(\varphi, \theta)$, и в этом направлении диаграмма направленности $F(\varphi, \theta)$ имеет вид:
\begin{gather*}
    F(\varphi, \theta)
    = \begin{pmatrix}
        A_1 e^{i \alpha_1} & 0 \\
        A_2 e^{i \alpha_2} & 0
    \end{pmatrix}
\end{gather*}
где амплитуды $A_1 \in \mathbb{R}$, $A_2 \in \mathbb{R}$ и смещения фаз $\alpha_1 \in \mathbb{R}$, $\alpha_2 \in \mathbb{R}$.

Необходимо найти оптимальный вектор комплексных огибающих $a_{max}$ сигналов на входах излучателей, при котором получается наибольший коэффициент
усиления~\eqref{emission:emitter:gain:rayleigh}:
\[
    \rho(a)
    = \frac{a^*F^*(\varphi, \theta) F(\varphi, \theta) a}{a^*a}.
\]
Решаем задачу первым способом, матрица системы~\eqref{emission:emitter:gain:optimal_input}:
\[
    F^*(\varphi, \theta) F(\varphi, \theta)
    =
    \begin{pmatrix}
        A_1 e^{-i \alpha_1} & A_2 e^{-i \alpha_2} \\
        0                   & 0
    \end{pmatrix}
    \begin{pmatrix}
        A_1 e^{i \alpha_1} & 0 \\
        A_2 e^{i \alpha_2} & 0
    \end{pmatrix}
    =
    \begin{pmatrix}
        A_1^2 + A_2^2 & 0 \\
        0             & 0
    \end{pmatrix}
\]
Отсюда легко получаются собственные числа:
\begin{gather*}
    \determinant{F^*(\varphi, \theta) F(\varphi, \theta) - \lambda I} = 0 , \\
    %
    \begin{vmatrix}
        A_1^2 + A_2^2 - \lambda & 0         \\
        0                       & - \lambda
    \end{vmatrix}
    = 0 , \\
    %
    \lambda_{max} = A_1^2 + A_2^2 , \\
    \lambda_{min} = 0 .
\end{gather*}
Наибольшее значение отношения Релея:
\[
    \rho_{max} = \lambda_{max} = A_1^2 + A_2^2 .
\]
Оптимальный вектор комплесных огибающих сигналов на входах $a_{max}$:
\begin{gather*}
    ( F^*(\varphi, \theta) F(\varphi, \theta) - \lambda_{max} I ) a_{max} = 0 , \\
    %
    \begin{pmatrix}
        A_1^2 + A_2^2 - (A_1^2 + A_2^2) & 0                 \\
        0                               & - (A_1^2 + A_2^2)
    \end{pmatrix}
    a_{max} = 0 , \\
    %
    \begin{pmatrix}
        0 & 0                 \\
        0 & - (A_1^2 + A_2^2)
    \end{pmatrix}
    a_{max} = 0 , \\
    %
    a_{max}
    = \begin{pmatrix}
        1 \\
        0
    \end{pmatrix} .
\end{gather*}
Оптимальная поляризация $p_{max}$:
\[
    p_{max}
    = F(\varphi, \theta) a_{max}
    = \begin{pmatrix}
        A_1 e^{i \alpha_1} & 0 \\
        A_2 e^{i \alpha_2} & 0
    \end{pmatrix}
    \begin{pmatrix}
        1 \\
        0
    \end{pmatrix}
    = \begin{pmatrix}
        A_1 e^{i \alpha_1} \\
        A_2 e^{i \alpha_2}
    \end{pmatrix} .
\]

\subsection{Пример 2: разная поляризация}

Излучение в разной фазе в двух ортогональных плоскостях по углу места и по азимуту.

Пусть в выбранном направлении $(\varphi, \theta)$ диаграмма направленности имеет вид:
\begin{gather*}
    F(\varphi, \theta) = \begin{pmatrix}
        A_1 e^{i \alpha_1} & 0                  \\
        0                  & A_2 e^{i \alpha_2}
    \end{pmatrix}
\end{gather*}
где амплитуды $A_1 \in \mathbb{R}$, $A_2 \in \mathbb{R}$, смещения фаз $\alpha_1 \in \mathbb{R}$, $\alpha_2 \in \mathbb{R}$ и для определённости
\[
    A_1 > A_2 .
\]

Необходимо найти оптимальный вектор комплексных огибающих $a_{max}$ сигналов на входах излучателей, при котором получается наибольший коэффициент
усиления~\eqref{emission:emitter:gain:rayleigh}:
\[
    \rho(a)
    = \frac{a^* F^*(\varphi, \theta) F(\varphi, \theta) a}{a^* a}.
\]
Решаем задачу первым способом, матрица системы~\eqref{emission:emitter:gain:optimal_input}:
\[
    F^*(\varphi, \theta) F(\varphi, \theta)
    =
    \begin{pmatrix}
        A_1 e^{-i \alpha_1} & 0                   \\
        0                   & A_2 e^{-i \alpha_2}
    \end{pmatrix}
    \begin{pmatrix}
        A_1 e^{i \alpha_1} & 0                  \\
        0                  & A_2 e^{i \alpha_2}
    \end{pmatrix}
    =
    \begin{pmatrix}
        A_1^2 & 0     \\
        0     & A_2^2
    \end{pmatrix}
\]
Собственные числа:
\begin{gather*}
    \determinant{F^*(\varphi, \theta) F(\varphi, \theta) - \lambda I} = 0 , \\
    %
    \begin{vmatrix}
        A_1^2 - \lambda & 0               \\
        0               & A_2^2 - \lambda
    \end{vmatrix}
    = 0 , \\
    %
    \lambda_{max} = A_1^2 , \\
    \lambda_{min} = A_2^2 .
\end{gather*}
Наибольшее значение отношения Релея:
\[
    \rho_{max} = \lambda_{max} = A_1^2 .
\]
Оптимальный вектор комплексных огибающих сигналов на входах излучателя:
\begin{gather*}
    ( F^*(\varphi, \theta) F(\varphi, \theta) - \lambda_{max} I ) a_{max} = 0 , \\
    %
    \begin{pmatrix}
        A_1^2 - A_1^2 & 0             \\
        0             & A_2^2 - A_1^2
    \end{pmatrix}
    a_{max} = 0 , \\
    %
    \begin{pmatrix}
        0 & 0             \\
        0 & A_2^2 - A_1^2
    \end{pmatrix}
    a_{max} = 0 , \\
    %
    a_{max}
    = \begin{pmatrix}
        1 \\
        0
    \end{pmatrix} .
\end{gather*}
Оптимальная поляризация $p_{max}$:
\[
    p_{max}
    = F(\varphi, \theta) a_{max}
    = \begin{pmatrix}
        A_1 e^{i \alpha_1} & 0                  \\
        0                  & A_2 e^{i \alpha_2}
    \end{pmatrix}
    \begin{pmatrix}
        1 \\
        0
    \end{pmatrix}
    = \begin{pmatrix}
        A_1 e^{i \alpha_1} \\
        0
    \end{pmatrix} .
\]

\subsection{Пример 3: одинаковая поляризация}

Два канала производят колебания в разной фазе, но в одном направлении --- азимутальном.

Пусть в выбранном направлении $(\varphi, \theta)$ диаграмма направленности имеет вид:
Пусть парциальные диаграммы направленности каналов излучателя имеют вид:
\[
    F(\varphi, \theta)
    =
    \begin{pmatrix}
        A_1 e^{i \alpha_1} & A_2 e^{i \alpha_2} \\
        0                  & 0
    \end{pmatrix}
\]
где амплитуды $A_1 \in \mathbb{R}$, $A_2 \in \mathbb{R}$, смещения фаз $\alpha_1 \in \mathbb{R}$, $\alpha_2 \in \mathbb{R}$.

В первом способе матрица системы~\eqref{emission:emitter:gain:optimal_input}:
\[
    F^*(\varphi, \theta) F(\varphi, \theta)
    =
    \begin{pmatrix}
        A_1 e^{-i \alpha_1} & 0 \\
        A_2 e^{-i \alpha_2} & 0
    \end{pmatrix}
    \begin{pmatrix}
        A_1 e^{i \alpha_1} & A_2 e^{i \alpha_2} \\
        0                  & 0
    \end{pmatrix}
    =
    \begin{pmatrix}
        A_1^2                            & A_1 e^{-i \alpha_1 + i \alpha_2} \\
        A_2 e^{-i \alpha_2 + i \alpha_1} & A_2^2
    \end{pmatrix}
\]
получается "сложной"{}, поэтому рассмотрим матрицу системы для оптимальной поляризации:
\[
    F(\varphi, \theta) F^*(\varphi, \theta)
    =
    \begin{pmatrix}
        A_1 e^{i \alpha_1} & A_2 e^{i \alpha_2} \\
        0                  & 0
    \end{pmatrix}
    \begin{pmatrix}
        A_1 e^{-i \alpha_1} & 0 \\
        A_2 e^{-i \alpha_2} & 0
    \end{pmatrix}
    =
    \begin{pmatrix}
        A_1^2 + A_2^2 & 0 \\
        0             & 0
    \end{pmatrix} .
\]
Эта матрица "простая"{}, и её собственные числа
\begin{gather*}
    \determinant{F(\varphi, \theta)F^*(\varphi, \theta) - \lambda I} = 0 , \\
    %
    \begin{vmatrix}
        A_1^2 + A_2^2 - \lambda & 0         \\
        0                       & - \lambda
    \end{vmatrix} = 0, \\
    %
    \lambda_{max} = A_1^2 + A_2^2 , \\
    \lambda_{min} = 0 .
\end{gather*}

Наибольшее значение отношения Релея:
\[
    \rho_{max} = \lambda_{max} = A_1^2 + A_2^2.
\]

Наибольшему собственному числу $\lambda_{max}$ соответствует оптимальный вектор поляризации $p_{max}$:
\begin{gather*}
    (F(\varphi, \theta)F^*(\varphi, \theta) - \lambda_{max} I) p_{max} = 0 , \\
    %
    \begin{pmatrix}
        A_1^2 + A_2^2 - \lambda_{max} & 0               \\
        0                             & - \lambda_{max}
    \end{pmatrix}
    p_{max} = 0 , \\
    %
    \begin{pmatrix}
        0 & 0                 \\
        0 & - (A_1^2 + A_2^2)
    \end{pmatrix}
    p_{max} = 0 , \\
    %
    p_{max} = \begin{pmatrix}
        1 \\
        0
    \end{pmatrix} .
\end{gather*}
Оптимальный вектор огибающих сигналов на входах излучателя:
\[
    a_{max}
    = F^*(\varphi, \theta) p_{max}
    = \begin{pmatrix}
        A_1 e^{- i \alpha_1} & 0 \\
        A_2 e^{- i \alpha_2} & 0
    \end{pmatrix}
    \begin{pmatrix}
        1 \\
        0
    \end{pmatrix}
    = \begin{pmatrix}
        A_1 e^{- i \alpha_1} \\
        A_2 e^{- i \alpha_2}
    \end{pmatrix} .
\]
Компоненты вектора $a_{max}$ указывают на необходимость обратных смещений фаз входных сигналов двух каналов излучателя, с тем чтобы колебания излучения складывались в одной фазе.

\subsection{Пример 4: общий случай}

Пусть в выбранном направлении $(\varphi, \theta)$ диаграмма направленности имеет вид:
\[
    F(\varphi, \theta)
    =
    \begin{pmatrix}
        0.1 e^{i \frac{\pi}{6}} & 0.3 e^{i \frac{5 \pi}{4}}  \\
        0.2 e^{i \frac{\pi}{3}} & 0.1 e^{- i \frac{\pi}{10}}
    \end{pmatrix} .
\]
Вычисления оптимальных векторов огибающих и поляризации смотри в файле Matlab \texttt{emission/two/gain.m}.

\subsection{Пример 5: крестовой излучатель}

В декартовой системе координат $X$, $Y$, $Z$ первый излучатель находится в начале координат и имеет парциальную диаграмму направленности:
\[
    f_1(\varphi, \theta) =
    \begin{pmatrix}
        \cos \varphi \cos \theta e^{\frac{\pi}{3}} \\
        \cos \theta
    \end{pmatrix} ,
\]
а в декартовых координатах:
\[
    f_{1,c}(\varphi, \theta) = \cos \varphi \cos \theta e^{\frac{\pi}{3}} \cdot u_{\varphi} + \cos \theta \cdot u_{\theta},
\]
где $u_\varphi$, $u_\theta$ --- образы орт координатных прямых углов $\varphi$ и $\theta$. В системе $C = (X, Y, Z)$ декартовы координаты точки связаны
со сферическими координатами системы $S = (\varphi, \theta, \rho)$ равенствами:
\begin{gather*}
    x = \rho \cos \theta \cos \varphi, \\
    y = \rho \cos \theta \sin \varphi, \\
    z = \rho \sin \theta.
\end{gather*}
Частные производные дают направления орт координатных прямых углов:
\begin{gather*}
    U_{\varphi} =
    \begin{pmatrix}
        x_{\varphi}^{\prime} \\
        y_{\varphi}^{\prime} \\
        z_{\varphi}^{\prime} \\
    \end{pmatrix}
    =
    \begin{pmatrix}
        - \rho \cos \theta \sin \varphi \\
        \rho \cos \theta \cos \varphi   \\
        0
    \end{pmatrix} , \\
    %
    U_{\theta} =
    \begin{pmatrix}
        x_{\theta}^{\prime} \\
        y_{\theta}^{\prime} \\
        z_{\theta}^{\prime} \\
    \end{pmatrix}
    =
    \begin{pmatrix}
        - \rho \sin \theta \cos \varphi \\
        - \rho \sin \theta \sin \varphi \\
        \rho \cos \theta
    \end{pmatrix} .
\end{gather*}
После нормировки получим:
\begin{gather*}
    u_{\varphi}
    = \frac{U_{\varphi}}{\norm{U_{\varphi}}}
    = \begin{pmatrix}
        - \sin \varphi \\
        \cos \varphi   \\
        0
    \end{pmatrix} , \\
    %
    u_{\theta}
    = \frac{U_\theta}{\norm{U_\theta}}
    = \begin{pmatrix}
        - \sin \theta \cos \varphi \\
        - \sin \theta \sin \varphi \\
        \cos \theta
    \end{pmatrix} .
\end{gather*}

Со вторым каналом свяжем декартову систему $\widetilde{C} = (\widetilde{X}$, $\widetilde{Y}$, $\widetilde{Z})$ и соответствующую сферическую систему
$\widetilde{S} = (\widetilde{\varphi}, \widetilde{\theta}, \widetilde{\rho})$. В этих системах парциальная диаграмма направленности второго канала:
\[
    \widetilde{f}_2(\widetilde{\varphi}, \widetilde{\theta}) =
    \begin{pmatrix}
        \cos \widetilde{\theta} e^{-i\frac{\pi}{4}} \\
        \cos \widetilde{\theta}
    \end{pmatrix}
\]
или
\[
    \widetilde{f}_{2,c}(\widetilde{\varphi}, \widetilde{\theta})
    = \cos \widetilde{\theta} e^{-\frac{\pi}{4}} \cdot \widetilde{u}_{\widetilde{\varphi}} + \cos \widetilde{\theta} \cdot \widetilde{u}_{\widetilde{\theta}} ,
\]
где
\begin{gather*}
    \widetilde{u}_{\widetilde{\varphi}}
    = \begin{pmatrix}
        - \sin \widetilde{\varphi} \\
        \cos \widetilde{\varphi}   \\
        0
    \end{pmatrix} , \\
    %
    \widetilde{u}_{\widetilde{\theta}}
    = \begin{pmatrix}
        - \sin \widetilde{\theta} \cos \widetilde{\varphi} \\
        - \sin \widetilde{\theta} \sin \widetilde{\varphi} \\
        \cos \widetilde{\theta}
    \end{pmatrix}.
\end{gather*}
Координаты $\widetilde{x}$, $\widetilde{y}$ и $\widetilde{z}$ системы $\widetilde{C}$ связаны с координатами $x$, $y$, $z$ системы $C$ равенствами:
\begin{gather*}
    x = \widetilde{x} , \\
    y = - \widetilde{z} , \\
    z = \widetilde{y} .
\end{gather*}
Векторы $\widetilde{u}_{\widetilde{\varphi}}$ и $\widetilde{u}_{\widetilde{\theta}}$ в системе $C$ будут иметь координаты:
\begin{gather*}
    u_{\widetilde{\varphi}}
    = \begin{pmatrix}
        - \sin \widetilde{\varphi} \\
        0                          \\
        \cos \widetilde{\varphi}
    \end{pmatrix} , \\
    %
    u_{\widetilde{\theta}}
    = \begin{pmatrix}
        - \sin \widetilde{\theta} \cos \widetilde{\varphi} \\
        - \cos \widetilde{\theta}                          \\
        - \sin \widetilde{\theta} \sin \widetilde{\varphi}
    \end{pmatrix}.
\end{gather*}
И диаграмма направленности второго канала в декартовых координатах системы $C$:
\begin{multline*}
    f_{2,c}(\widetilde{\varphi}, \widetilde{\theta})
    = \cos \widetilde{\theta} e^{-i\frac{\pi}{4}} \cdot \widetilde{u}_{\widetilde{\varphi}} + \cos \widetilde{\theta} \cdot u_{\widetilde{\theta}} = \\
    %
    = \cos \widetilde{\theta}  e^{-i\frac{\pi}{4}}
    \begin{pmatrix}
        - \sin \widetilde{\varphi} \\
        0                          \\
        \cos \widetilde{\varphi}
    \end{pmatrix}
    + \cos \widetilde{\theta}
    \begin{pmatrix}
        - \sin \widetilde{\theta} \cos \widetilde{\varphi} \\
        - \cos \widetilde{\theta}                          \\
        - \sin \widetilde{\theta} \sin \widetilde{\varphi}
    \end{pmatrix} = \\
    %
    = e^{-i\frac{\pi}{4}}
    \begin{pmatrix}
        - \cos \widetilde{\theta}  \sin \widetilde{\varphi} \\
        0                                                   \\
        \cos \widetilde{\theta} \cos \widetilde{\varphi}
    \end{pmatrix}
    + \begin{pmatrix}
        - \cos \widetilde{\theta} \sin \widetilde{\theta} \cos \widetilde{\varphi} \\
        - \cos^2 \widetilde{\theta}                                                \\
        - \cos \widetilde{\theta} \sin \widetilde{\theta} \sin \widetilde{\varphi}
    \end{pmatrix} .
\end{multline*}
Теперь нужно заменить углы $\widetilde{\varphi}$ и $\widetilde{\theta}$ углами $\varphi$ и $\theta$.

Из соответствия координат систем $C$ и $\widetilde{C}$ получим равенства:
\begin{align*}
    \rho \cos \theta \cos \varphi & = \widetilde{\rho} \cos \widetilde{\theta} \cos \widetilde{\varphi} , \\
    \rho \cos \theta \sin \varphi & = - \widetilde{\rho} \sin \widetilde{\theta} ,                        \\
    \rho \sin \theta              & = \widetilde{\rho} \cos \widetilde{\theta} \sin \widetilde{\varphi} .
\end{align*}
Учитывая что $\rho = \widetilde{\rho}$, получим равенства для углов:
\begin{align*}
    \cos \theta \cos \varphi & = \cos \widetilde{\theta} \cos \widetilde{\varphi} , \\
    \cos \theta \sin \varphi & = - \sin \widetilde{\theta} ,                        \\
    \sin \theta              & = \cos \widetilde{\theta} \sin \widetilde{\varphi} .
\end{align*}
Откуда из перемножения левых и правых частей первого и второго равенств:
\[
    \cos^2 \theta \cos \varphi \sin \varphi = - \cos \widetilde{\theta} \sin \widetilde{\theta} \cos \widetilde{\varphi} ,
\]
из перемножения левых и правых частей второго и третьего равенств:
\[
    \cos \theta \sin \theta \sin \varphi = - \cos \widetilde{\theta} \sin \widetilde{\theta} \sin \widetilde{\varphi}, \\
\]
из возведения в квадрат левой и правой частей второго равенства:
\begin{gather*}
    \cos^2 \theta \sin^2 \varphi = \sin^2 \widetilde{\theta} , \\
    1 - \cos^2 \theta \sin^2 \varphi = 1 - \sin^2 \widetilde{\theta} , \\
    1 - \cos^2 \theta \sin^2 \varphi = \cos^2 \widetilde{\theta} .
\end{gather*}
Таким образом, диаграмма направленности второго канала в декартовых координатах:
\[
    f_{2,c}(\varphi, \theta)
    = e^{-i\frac{\pi}{4}}
    \begin{pmatrix}
        - \sin \theta \\
        0             \\
        \cos \theta \cos \varphi
    \end{pmatrix}
    + \begin{pmatrix}
        \cos^2 \theta \cos \varphi \sin \varphi \\
        \cos^2 \theta \sin^2 \varphi - 1        \\
        \cos \theta \sin \theta \sin \varphi
    \end{pmatrix}
\]
и для получения парциальной диаграммы направленности второго канала нужно вычислить проекции (скалярные произведения) на орты $u_\theta$ и $u_\varphi$:
\[
    f_2(\varphi, \theta)
    = \begin{pmatrix}
        \scalarproduct{f_{2,c}(\varphi, \theta)}{u_\varphi} \\
        \scalarproduct{f_{2,c}(\varphi, \theta)}{u_\theta}
    \end{pmatrix} .
\]

Построение диаграмм в файле \texttt{emission/cross/arrows.m}.

Вычисление коэффициента усиления в файле \texttt{emission/cross/gain.m}.

\section{Коэффициент полезного действия}

\subsection{Энергетическое ограничение}

Сигналы, поступающие на входы излучателей, частично отражаются обратно и частично переходят в излучение. Часть излучения, которое формирует излучатель,
затекает обратно через другие излучатели, в результате в каналах наводятся дополнительные отражённые сигналы. В итоге, мощность входных сигналов
$P_{inp}$ частично переходит в мощность излучения $P_{rad}$, частично в мощность отражённых сигналов во входных каналах излучателей $P_{ref}$ и
частично растрачивается на диссипативные потери мощностью $P_{dis}$:
\[
    P_{rad} + P_{ref} + P_{dis} = P_{inp}.
\]

Диссипативные потери сложно проанализировать, поэтому просто исключим их мощность $P_{dis} >0$ и получим неравенство:
\begin{equation}
    \label{emission:efficiency:power_inequality}
    P_{ref} + P_{rad} \le P_{inp} .
\end{equation}

Перенос мощности отраженных сигналов $P_{ref}$ в правую часть даёт неравенства:
\begin{gather*}
    P_{rad} \le P_{inp} - P_{ref}, \\
    P_{rad} \le P_{inp} - P_{ref} \le P_{inp}, \\
    \frac{P_{rad}}{P_{inp}} \le 1 - \frac{P_{ref}}{P_{inp}} \le 1 .
\end{gather*}

\subsection{Коэффициент полезного действия}

Коэффициентом полезного действия (КПД) называется отношение мощности излучения ко входной мощности:
\[
    \eta = \frac{P_{rad}}{P_{inp}}.
\]
Легко видеть, что
\[
    \eta \le 1 - \frac{P_{ref}}{P_{inp}} = \eta_s ,
\]
и величина $\eta_s$ является верхней границей КПД $\eta$.

Входная мощность $P_{inp}$ связана с вектором огибающих входных сигналов $a$, поступающих на входы излучателей:
\[
    P_{inp} = \norm{a}_2^2 .
\]

Пусть $b$ обозначает вектор огибающих отражённых сигналов, будем считать, что огибающие отражённых сигналов линейно связаны с огибающими входных сигналов.
Для излучателя с номером $k$:
\[
    b_k = s_{k1} a_1 + \dots + s_{kk} a_k + \dots + s_{k,n} a_n.
\]
Пусть $S$ --- матрица коэффициентов $s_{ij}$, тогда:
\[
    b = S a ,
\]
где $S$ называется матрицей рассеяния.

Мощность отражённых сигналов:
\[
    \norm{b}_2^2
    = b^* b
    = \left ( S a \right )^* S a
    = a^* S^* S a .
\]

Таким образом, граница КПД $\eta_s$:
\[
    \eta_s(a)
    = 1 - \frac{P_{ref}}{P_{inp}}
    = 1 - \frac{a S^* S a}{a^* a}
\]
и ограничение для КПД можно получить из анализа матрицы рассеяния $S$.

Само значение КПД определяется мощностью излучения $P_{rad}$, которое представляет собой интеграл по сфере $S_r$ радиуса $r$ (позже выяснится,
что величина $r$ значение не имеет) от мощности электрического поля:
\[
    P_{rad}
    = \iint \limits_{S_r} \norm{E(a, r)}_2^2 ds ,
\]
Подставляя выражение \eqref{emission:diagram:tension} для вектора напряжённости $E(a, r)$, получим:
\begin{multline*}
    P_{rad}
    = \iint \limits_{S_r} \norm{F(\varphi, \theta) a \cdot \frac{e^{i \scalarproduct{w}{r}}}{r}}^2 ds
    = \frac{1}{r^2} \iint \limits_{S_r} \norm{F(\varphi, \theta) a}^2 \modulus{\frac{e^{i \scalarproduct{w}{r}}}{r}}^2 ds = \\
    %
    = \frac{1}{r^2} \iint \limits_{S_r} \norm{F(\varphi, \theta) a}^2 \frac{\modulus{e^{i \scalarproduct{w}{r}}}^2}{r^2} ds
    = \frac{1}{r^2} \iint \limits_{S_r} \norm{F(\varphi, \theta) a}^2 ds = \\
    %
    = \frac{1}{r^2} \iint \limits_{S_r} a^* F^*(\varphi, \theta) F(\varphi, \theta) a ds ,
\end{multline*}
где квадратичная форма:
\[
    a^* F^*(\varphi, \theta) F(\varphi, \theta) a
    = \begin{pmatrix}
        a_1^* & \dots & a_n^*
    \end{pmatrix}
    \begin{pmatrix}
        f_1^* f_1 & \dots  & f_1^* f_k & \dots  & f_1^* f_n \\
        \vdots    & \vdots & \vdots    & \vdots & \vdots    \\
        f_k^* f_1 & \dots  & f_k^* f_k & \dots  & f_k^* f_n \\
        \vdots    & \vdots & \vdots    & \vdots & \vdots    \\
        f_n^* f_1 & \dots  & f_n^* f_k & \dots  & f_n^* f_n
    \end{pmatrix}
    \begin{pmatrix}
        a_1    \\
        \vdots \\
        a_n
    \end{pmatrix}
    ,
\]
где $f_k$ --- столбцы диаграммы направленности $F(\varphi, \theta)$. Интеграл от квадратичной формы равен сумме интегралов, которую также можно
представить квадратичной формой, поскольку вектор $a$ является постоянным:
\[
    P_{rad}
    = a^* Q a,
\]
где матрица $Q$ образована элементами
\[
    Q_{ij}
    = \frac{1}{r^2} \iint \limits_{S_r} f_i^*(\varphi, \theta) f_j(\varphi, \theta) ds
\]

Таким образом, КПД $\eta$:
\[
    \eta(a) = \frac{a^* Q a}{a^* a}
\]
является отношением Релея, поэтому его значения ограничены наименьшим $\eta_{min}$ и наибольшим $\eta_{max}$ собственными значениями матрицы $Q$:
\[
    \eta_{min} \le \eta(a) \le \eta_{max} .
\]

\subsection{КПД каналов}

Представим, что вся мощность входных сигналов направлена в излучатель с номером $k$, то есть вектор огибающих
\[
    a
    = \begin{pmatrix}
        0      \\
        \vdots \\
        1      \\
        \vdots \\
        0
    \end{pmatrix} ,
\]
тогда величина КПД излучателя с номером $k$:
\[
    \eta_k
    = \frac{a^* Q a}{a^* a}
    = \frac{Q_{kk}}{1}
    = Q_{kk} .
\]
Таким образом, диагональные элемент матрицы $Q$ показывают КПД излучателей. Внедиагональные элементы представляют собой коэффициенты пересечения диаграмм.

Оказывается, что КПД каналов $\eta_1$ и $\eta_2$ связаны с коэффициентом пересечения диаграмм. Представим элементы матрицы $Q$ в нормированном виде:
\[
    Q_{ij}
    =
    \sqrt{Q_{ii}}
    \cdot
    \frac{\iint \limits_{S_r} f_i^*(\varphi, \theta) f_j(\varphi, \theta) ds}{\sqrt{Q_{ii}} \sqrt{Q_{jj}}}
    \cdot
    \sqrt{Q_{jj}} ,
\]
тогда
\[
    Q = \sqrt{D} R \sqrt{D} ,
\]
где
\[
    \sqrt{D}
    = \begin{pmatrix}
        \sqrt{Q_{11}} & 0             & \dots & 0             \\
        0             & \sqrt{Q_{22}} & \dots & 0             \\
        0             & 0             & \dots & \sqrt{Q_{nn}} \\
    \end{pmatrix} ,
    \;
    %
    R_{ij} = \frac{Q_{ij}}{\sqrt{Q_{ii}} \sqrt{Q_{jj}}} .
\]

Согласно неравенству~\eqref{emission:efficiency:power_inequality}:
\begin{align*}
    P_{rad}                   & \le P_{inp} , \\
    a^* Q a                   & \le a^* a,    \\
    a^* \sqrt{D} R \sqrt{D} a & \le a^* a .
\end{align*}
Пусть $x = \sqrt{D} a$, тогда:
\begin{align*}
    x^* R x & \le x^* (\sqrt{D}^{-1})^* (\sqrt{D}^{-1}) x , \\
    x^* R x & \le x^* D^{-1} x .
\end{align*}
Пусть $r_{max}$ --- наибольшее собственное значение матрицы $R$ и $x_{max}$ --- соответствующий этому числу собственный вектор, а $Q_{min} = \min \{ Q_{11}, Q_{22} \}$,
тогда:
\begin{align*}
    x_{max}^* R x_{max}        & \le x_{max}^* D^{-1} x_{max} ,                             \\
    x_{max}^* r_{max} x_{max}  & \le \sum_{k=1}^n \frac{1}{Q_{kk}} x_{max,k}^* x_{max,k} ,  \\
    r_{max} \norm{x_{max}}_2^2 & \le \sum_{k=1}^n \frac{1}{Q_{kk}} \modulus{x_{max,k}}^2 ,  \\
    r_{max} \norm{x_{max}}_2^2 & \le \sum_{k=1}^n \frac{1}{Q_{min}} \modulus{x_{max,k}}^2 , \\
    r_{max} \norm{x_{max}}_2^2 & \le \frac{1}{Q_{min}} \sum_{k=1}^n \modulus{x_{max,k}}^2 , \\
    r_{max} \norm{x_{max}}_2^2 & \le \frac{1}{Q_{min}} \norm{x_{max}}_2^2 ,                 \\
    r_{max}                    & \le \frac{1}{Q_{min}} ,                                    \\
    Q_{min}                    & \le \frac{1}{r_{max}} .
\end{align*}

Пусть, например, в системе всего два излучателя, тогда матрица $R$ имеет вид:
\begin{gather}
    R
    = \begin{pmatrix}
        1        & R_{12} \\
        R_{12}^* & 1
    \end{pmatrix}
    \notag, \\
    %
    R_{12} = \frac{Q_{12}}{\sqrt{Q_{11}} \sqrt{Q_{22}}}
    \label{emission:emitter:efficiency:diagram_intersection}
\end{gather}
тогда наибольшее собственное значение матрицы $R$ (аналогично равенству~\eqref{emission:emitter:gain:maximum_eigenvalue}):
\begin{multline*}
    r_{max}
    = \frac{\tr(R) + \sqrt{\tr^2(R) - 4 \det(R)}}{2} = \\
    %
    = \frac{2 + \sqrt{2^2 - 4 (1 - \modulus{R_{12}}^2)}}{2}
    = \frac{2 + \sqrt{4 - 4 + 4 \modulus{R_{12}}^2}}{2} = \\
    %
    = \frac{2 + 2 \modulus{R_{12}}}{2}
    = 1 + \modulus{R_{12}} .
\end{multline*}
Таким образом,
\begin{align*}
    \min \{ Q_{11}, Q_{22} \} & \le \frac{1}{1 + \modulus{R_{12}}} , \\
    \min \{ \eta_1, \eta_2 \} & \le \frac{1}{1 + \modulus{R_{12}}} .
\end{align*}

\subsubsection{Вычисление интегралов}

Возникает вопрос о вычислении интегралов вида:
\[
    P = \frac{1}{r^2} \iint \limits_{S_r} p(\varphi, \theta) ds
\]
по сфере $S_r$ радиуса $r$.

Кратко, нужно перейти к угловым координатами $\varphi$ и $\theta$, в которых элемент поверхности $ds$ приближённо представляет собой прямоугольник
со сторонами $r d\varphi$ и $r \cos \theta d\theta$:
\[
    ds = r \cos \theta d\varphi \cdot r d\theta = r^2 d\Omega,
\]
тогда поверхностый интеграл преобразуется к кратному интегралу и затем к повторному интегралу:
\[
    P
    = \iint \limits_{S_r} p(\varphi, \theta) d\Omega
    = \int \limits_0^{2 \pi} \left( \int \limits_{-\frac{\pi}{2}}^{\frac{\pi}{2}} p(\varphi, \theta) \cos \theta d\theta \right) d\varphi .
\]

Далее следует формальный вывод последнего выражения через вычисление исходного поверхностного интеграла в декартовых координатах и перехода к угловым координатам.

Для вычисления интеграла сфера разделяется на верхнюю $z \ge 0$ и нижнюю $z < 0$ полусферы:
\begin{gather*}
    x^2 + y^2 + z^2 = r^2 , \\
    z^2 = r^2 - x^2 - y^2 , \\
    z = \pm \sqrt{r^2 - x^2 - y^2} .
\end{gather*}
Интеграл сводится к сумме двух кратных интегралов в декартовых координатах:
\begin{multline*}
    P
    = \frac{1}{r^2} \iint
    \limits_{
        \begin{array}{c}
            x^2 + y^2 \le r^2, \\
            z = \sqrt{r^2 - x^2 - y^2}
        \end{array}
    } p(\varphi, \theta) \sqrt{(z_x^\prime)^2 + (z_y^\prime)^2 + 1} dxdy + \\
    + \frac{1}{r^2} \iint
    \limits_{
        \begin{array}{c}
            x^2 + y^2 \le r^2, \\
            z = - \sqrt{r^2 - x^2 - y^2}
        \end{array}
    } p(\varphi, \theta) \sqrt{(z_x^\prime)^2 + (z_y^\prime)^2 + 1} dxdy ,
\end{multline*}
где производные под корнем:
\begin{gather*}
    z = \pm \sqrt{r^2 - x^2 - y^2} , \\
    z_x^\prime = \pm \frac{-2 x}{2 \sqrt{r^2 - x^2 - y^2}} , \\
    z_y^\prime = \pm \frac{-2 y}{2 \sqrt{r^2 - x^2 - y^2}} ,
\end{gather*}
и выражения под корнями в двух интегралах преобразуются одинаково:
\begin{multline*}
    (z_x^\prime)
    ^2 + (z_y^\prime)^2 + 1
    = \frac{x^2}{r^2 - x^2 - y^2} + \frac{y^2}{r^2 - x^2 - y^2} + 1 = \\
    %
    = \frac{x^2 + y^2 + r^2 + x^2 + y^2}{r^2 - x^2 - y^2}
    = \frac{r^2}{r^2 - x^2 - y^2}
\end{multline*}
При переходе в кратных интегралах к угловым координатам:
\begin{gather*}
    x = r \cos \theta \cos \varphi , \\
    y = r \cos \theta \sin \varphi ,
\end{gather*}
выражение под корнем преобразуется далее:
\begin{gather*}
    \frac{r^2}{r^2 - x^2 - y^2}
    = \frac{r^2}{r^2 - r^2 \cos^2 \theta \cos^2 \varphi - r^2 \cos^2 \theta \sin^2 \varphi}
    = \frac{r^2}{r^2 - r^2 \cos^2 \theta}
    = \frac{1}{1 - \cos^2 \theta}
    = \frac{1}{\sin^2 \theta} , \\
    %
    \sqrt{\frac{1}{\sin^2 \theta}}
    = \frac{1}{\modulus{\sin \theta}} .
\end{gather*}
Якобиан пребразования для угловых координат
\[
    \frac{\partial (x, y)}{\partial (\theta, \varphi)}
    = \begin{vmatrix}
        - r \sin \theta \cos \varphi & - r \cos \theta \sin \varphi \\
        - r \sin \theta \sin \varphi & r \cos \theta \cos \varphi
    \end{vmatrix}
    = - r^2 \cos \theta \sin \theta .
\]
Таким образом,
\begin{multline*}
    P = \frac{1}{r^2} \int \limits_0^{2 \pi} \int \limits_0^\frac{\pi}{2} p(\varphi, \theta) \frac{1}{\modulus{\sin \theta}} \modulus{r^2 \cos \theta \sin \theta} d\theta d\varphi + \\
    \shoveright{+ \frac{1}{r^2} \int \limits_0^{2 \pi} \int \limits_{-\frac{\pi}{2}}^0 p(\varphi, \theta) \frac{1}{\modulus{\sin \theta}} \modulus{r^2 \cos \theta \sin \theta} d\theta d\varphi = } \\
    %
    = \int \limits_0^{2 \pi} \int \limits_0^\frac{\pi}{2} p(\varphi, \theta) \modulus{\cos \theta} d\theta d\varphi
    + \int \limits_0^{2 \pi} \int \limits_{-\frac{\pi}{2}}^0 p(\varphi, \theta) \modulus{\cos \theta} d\theta d\varphi = \\
    %
    = \int \limits_0^{2 \pi} \int \limits_{-\frac{\pi}{2}}^\frac{\pi}{2} p(\varphi, \theta) \cos \theta d\theta d\varphi .
\end{multline*}

\subsubsection{Пример 2: разная поляризация}

Пусть в направлении $(\varphi, \theta)$ диаграмма направленности излучателя имеет вид:
\begin{gather*}
    F(\varphi, \theta)
    = \begin{pmatrix}
        A_1 e^{i \alpha_1} \cos \theta & 0                              \\
        0                              & A_2 e^{i \alpha_2} \cos \theta
    \end{pmatrix} , \\
    %
    A_1 \ge A_2.
\end{gather*}
Вычислим элементы матрицы $Q$:
\begin{multline*}
    Q_{11}
    = \int \limits_0^{2 \pi} \int \limits_{-\frac{\pi}{2}}^\frac{\pi}{2} f_1^*(\varphi, \theta) f_1(\varphi, \theta) \cos \theta d\theta d\varphi .
    = \int \limits_0^{2 \pi} \int \limits_{-\frac{\pi}{2}}^\frac{\pi}{2} A_1 e^{-i \alpha_1} \cos \theta A_1 e^{i \alpha} \cos \theta \cos \theta d\theta d\varphi = \\
    %
    = \int \limits_0^{2 \pi} \int \limits_{-\frac{\pi}{2}}^\frac{\pi}{2} A_1^2 \cos^3 \theta d\theta d\varphi
    = A_1^2 \int \limits_0^{2 \pi} \int \limits_{-\frac{\pi}{2}}^\frac{\pi}{2} \cos^3 \theta d\theta d\varphi
    = A_1^2 2 \pi \int \limits_{-\frac{\pi}{2}}^\frac{\pi}{2} \cos^3 \theta d\theta = \\
    %
    = A_1^2 2 \pi \int \limits_{-\frac{\pi}{2}}^\frac{\pi}{2} (1 - \sin^2 \theta ) \cos \theta d\theta
    = A_1^2 2 \pi \left( \int \limits_{-\frac{\pi}{2}}^\frac{\pi}{2} \cos \theta d\theta - \int \limits_{-\frac{\pi}{2}}^\frac{\pi}{2} \sin^2 \theta \cos \theta d\theta \right) = \\
    %
    = A_1^2 2 \pi \left( \left. \sin \theta \right|_{-\frac{\pi}{2}}^\frac{\pi}{2} - \left. \frac{\sin^3 \theta}{3} \right|_{-\frac{\pi}{2}}^\frac{\pi}{2} \right)
    = A_1^2 2 \pi \left( 2 - \frac{2}{3} \right)
    = \frac{8}{3} \pi A_1^2 .
\end{multline*}
Аналогично:
\begin{gather*}
    Q_{22}
    = \int \limits_0^{2 \pi} \int \limits_{-\frac{\pi}{2}}^\frac{\pi}{2} A_2^2 \cos^3 \theta d\theta d\varphi
    = \frac{8}{3} \pi A_2^2 , \\
    %
    Q_{12}
    = \int \limits_0^{2 \pi} \int \limits_{-\frac{\pi}{2}}^\frac{\pi}{2} 0 \cdot \cos \theta d\theta d\varphi
    = 0 .
\end{gather*}
Таким образом,
\[
    Q
    = \frac{8}{3} \pi
    \begin{pmatrix}
        A_1^2 & 0     \\
        0     & A_2^2
    \end{pmatrix} .
\]
Поскольку матрица $Q$ является диагональной, то можно сразу определить минимальный $\eta_{min}$ и максимальный $\eta_{max}$ КПД, которые являются минимальным и максимальным собственным
числом матрицы $Q$:
\begin{align*}
    \eta_{min} & = A_2^2 \cdot \frac{8}{3} \pi , \\
    \eta_{max} & = A_1^2 \cdot \frac{8}{3} \pi ,
\end{align*}
поскольку $A_1 > A_2$.

Коэффициент пересечения диаграмм \eqref{emission:emitter:efficiency:diagram_intersection}:
\begin{gather*}
    R_{12}
    = \frac{Q_{12}}{\sqrt{Q_{11}} \sqrt{Q_{22}}}
    = \frac{0}{A_1 \cdot A_2}
    = 0, \\
    %
    \modulus{R_{12}} = 0 .
\end{gather*}
Откуда
\[
    \min \{ \eta_1, \eta_2 \} \le \frac{1}{1 + \modulus{R_{12}}} = 1.
\]

\subsubsection{Пример 3: одинаковая поляризация}

Пусть в направлении $w$ диаграмма направленности излучателя имеет вид:
\[
    F(\varphi, \theta)
    = \begin{pmatrix}
        A_1 e^{i \alpha_1} \cos \theta & A_2 e^{i \alpha_2} \cos \theta \\
        0                              & 0
    \end{pmatrix} .
\]
Вычисляем элементы матрицы $Q$:
\begin{gather*}
    Q_{11}
    = \int \limits_0^{2 \pi} \int \limits_{-\frac{\pi}{2}}^\frac{\pi}{2} A_1^2 \cos^3 \theta d\theta d\varphi
    = \frac{8}{3} \pi A_2^2 , \\
    %
    Q_{12}
    = \int \limits_0^{2 \pi} \int \limits_{-\frac{\pi}{2}}^\frac{\pi}{2} A_1 A_2 e^{i \alpha_1 - i \alpha_2} \cos^3 \theta d\theta d\varphi
    = \frac{8}{3} \pi A_1 A_2 e^{i(\alpha_1 - \alpha_2)} , \\
    %
    Q_{22}
    = \int \limits_0^{2 \pi} \int \limits_{-\frac{\pi}{2}}^\frac{\pi}{2} A_2^2 \cos^3 \theta d\theta d\varphi
    = \frac{8}{3} \pi A_2^2 .
\end{gather*}
Таким образом,
\[
    Q
    = \frac{8}{3} \pi
    \begin{pmatrix}
        A_1^2                              & A_1 A_2 e^{i(\alpha_1 - \alpha_2)} \\
        A_1 A_2 e^{i(\alpha_2 - \alpha_1)} & A_2^2
    \end{pmatrix} .
\]
След и определитель матрицы $Q$:
\begin{gather*}
    \tr(Q) = \frac{8}{3} \pi ( A_1^2 + A_2^2 ) , \\
    %
    \det(Q)
    = \left( \frac{8}{3} \pi \right)^2 \left( A_1^2 A_2^2 - A_1^2 A_2^2 \right)
    = 0 .
\end{gather*}
Корень из дискриминанта:
\[
    \sqrt{\tr^2(Q) - 4 \det(Q)}
    = \frac{8}{3} \pi \sqrt{( A_1^2 + A_2^2 )^2}
    = \frac{8}{3} \pi ( A_1^2 + A_2^2 ),
\]
тогда
\begin{align*}
    \eta_{min} &
    = \frac{A_1^2 + A_2^2 - ( A_1^2 + A_2^2 )}{2} \cdot \frac{8}{3} \pi
    = 0 ,        \\
    %
    \eta_{max} &
    = \frac{A_1^2 + A_2^2 + ( A_1^2 + A_2^2 )}{2} \cdot \frac{8}{3} \pi
    = \frac{2 (A_1^2 + A_2^2)}{2} \cdot \frac{8}{3} \pi
    = (A_1^2 + A_2^2) \cdot \frac{8}{3} \pi .
\end{align*}

Коэффициент пересечения диаграмм \eqref{emission:emitter:efficiency:diagram_intersection}:
\begin{gather*}
    R_{12}
    = \frac{Q_{12}}{\sqrt{Q_{11}} \sqrt{Q_{22}}}
    = \frac{A_1 A_2 e^{i(\alpha_1 - \alpha_2)}}{A_1 \cdot A_2}
    = e^{i(\alpha_1 - \alpha_2)}, \\
    %
    \modulus{R_{12}} = 1 .
\end{gather*}
Откуда
\[
    \min \{ \eta_1, \eta_2 \} \le \frac{1}{1 + \modulus{R_{12}}} = \frac{1}{1+1} = \frac{1}{2}
\]

\subsubsection{Пример 4: общий случай}

Пусть в направлении $w$ диаграмма направленности имеет вид:
\[
    F(\varphi, \theta)
    =
    \begin{pmatrix}
        0.1 e^{i \frac{\pi}{6}} \cos \theta  & 0.3 e^{i \frac{5 \pi}{4}} \cos \theta   \\
        0.2 e^{i \frac{\pi}{3}} \sin \varphi & 0.1 e^{- i \frac{\pi}{10}} \sin \varphi
    \end{pmatrix} .
\]
Вычисления оптимальных векторов огибающих и поляризации смотри в файле Matlab \texttt{emission/two/efficiency.m}.

\subsubsection{Пример 5: крестовой излучатель}

Вычисление КПД в файле \texttt{emission/cross/efficiency.m}.

\section{Диаграммообразующая схема}

Пусть $a_\alpha$ и $b_\alpha$ --- векторы огибающих сигналов сечения входа схемы и $a$ и $b$ --- векторы огибащих сигналов сечения выхода схемы. У схемы два входа ---
$a_\alpha$ и $b$ и два выхода --- $a$ и $b_\alpha$, которые связаны со входами:
\begin{gather}
    a        = S_{\beta \alpha} a_\alpha + S_{\beta \beta} b
    \label{emission:scheme:upper_output}, \\
    b_\alpha = S_{\alpha \alpha} a_\alpha + S_{\alpha \beta} b
    \label{emission:scheme:lower_output}
\end{gather}
Причем $a$ и $b$ --- огибающие сигналов в сечении входа антенны, для которых справедливо равенство:
\begin{equation}
    \label{emission:scheme:reflections}
    b = S_\beta a ,
\end{equation}
где $S_\beta$ --- матрица рассеяния антенны.

Согласно равенствам~\eqref{emission:scheme:upper_output} и~\eqref{emission:scheme:reflections} вектор огибающих входа антенны $a$:
\begin{align}
    a                                 & = S_{\beta \alpha} a_\alpha + S_{\beta \beta} b , \notag                                                 \\
    a                                 & = S_{\beta \alpha} a_\alpha + S_{\beta \beta} S_\beta a , \notag                                         \\
    a - S_{\beta \beta} S_\beta a     & = S_{\beta \alpha} a_\alpha , \notag                                                                     \\
    ( I - S_{\beta \beta} S_\beta ) a & = S_{\beta \alpha} a_\alpha , \notag                                                                     \\
    a                                 & = ( I - S_{\beta \beta} S_\beta )^{-1} S_{\beta \alpha} a_\alpha , \label{emission:scheme:antenna_input}
\end{align}
тогда линейная часть диаграммы направленности антенны с диаграммообразующей схемой:
\[
    \widetilde{F}_s(\varphi, \theta, a_\alpha)
    = F_a(\varphi, \theta) a(a_\alpha)
    = F_a(\varphi, \theta) ( I - S_{\beta \beta} S_\beta )^{-1} S_{\beta \alpha} a_\alpha
    = F_s(\varphi, \theta) a_\alpha,
\]
где
\[
    F_s(\varphi, \theta) = F_a(\vec{w}) ( I - S_{\beta \beta} S_\beta )^{-1} S_{\beta \alpha} .
\]

Из равенств~\eqref{emission:scheme:upper_output},~\eqref{emission:scheme:reflections} и~\eqref{emission:scheme:antenna_input} вектор огибающих выхода
диаграммообразующей схемы:
\begin{align*}
    b_\alpha & = S_{\alpha \alpha} a_\alpha + S_{\alpha \beta} b ,                                                          \\
    b_\alpha & = S_{\alpha \alpha} a_\alpha + S_{\alpha \beta} S a ,                                                        \\
    b_\alpha & = S_{\alpha \alpha} a_\alpha + S_{\alpha \beta} S ( I - S_{\beta \beta} S )^{-1} S_{\beta \alpha} a_\alpha , \\
    b_\alpha & = ( S_{\alpha \alpha} + S_{\alpha \beta} S ( I - S_{\beta \beta} S )^{-1} S_{\beta \alpha} ) a_\alpha ,
\end{align*}
откуда матрица рассеяния для диаграммообразующей схемы:
\[
    S_\alpha = S_{\alpha \alpha} + S_{\alpha \beta} S ( I - S_{\beta \beta} S )^{-1} S_{\beta \alpha} .
\]


    \chapter{Плоская волна}

Будем использовать упрощённую модель плоской волны, фронт распространения которой является прямой.

\textcolor{red}{Рисунок волн.}

Зафиксируем декартову систему координат. В точке соответствующей началу отсчёта колебания имеют фазу:
\[
    \varphi_0(t) = \varphi_0 + \omega t.
\]
Точка начала отсчёта называется фазовым центром.

Наша цель заключается в расчёте фаз во всех других точках. Если через начало координат провести прямую параллельно фронту распространения волны, то во всех точках
этой прямой фаза будет такая же.

А что делать с остальными точками? Рассмотрим волновой вектор $\vec{w}$, который направлен в сторону распространения волны перпендикулярно фронту и имеет длину
\[
    \modulus{\vec{w}}
    = \frac{\omega}{v}
    = \frac{\omega \cdot T}{v \cdot T}
    = \frac{2 \pi}{\lambda} ,
\]
где $\omega$ --- угловая скорость колебаний, $v$ --- линейная скорость распространения волны, $T$ --- период колебаний, $\lambda$ --- длина волны (расстояние,
которое проходит волна за один период).

Проведём прямую через начало координат в направлении волнового вектора и рассмотрим изменение фазы колебаний в точках прямой. Если продвинуться на расстояние
$l$ по прямой в направлении волнового вектора $\vec{w}$ до точки $A$, то фаза изменится на $2 \pi \frac{l}{\lambda}$, а если продвинуться в обратную сторону
до точки $A^\prime$, то фаза изменится на $- 2 \pi \frac{l}{\lambda}$. Таким образом, если $l$ --- расстояние со знаком от точки прямой до начала отсчёта
(положительное направление отсчёта в направлении волнового вектора), то измение фазы $\Delta \varphi(l)$ будет равно:
\[
    \Delta \varphi(l)
    = 2 \pi \frac{l}{\lambda}
    = \frac{2 \pi} {\lambda} l
    = \modulus{\vec{w}} l .
\]
Таким образом, $\modulus{\vec{w}}$ является линейный коэффициентом изменения фазы. Если через точку $A$ провести прямую параллельную фронту распространения волны,
то все точки на этой прямой будут иметь такое же изменение фазы.

Теперь понятно как вычислить изменение фазы для любой точки $B$: необходимо через точку $B$ провести прямую параллельную фронту распространения волны, найти точку
пересечения с прямой проведённой через начало координат в направлении волнового вектора и вычислить расстояние $l$ от начала отсчёта до точки пересечения. Если
точка $B$ имеет радиус-вектор $\vec{r}$, то расстояние $l$ является проекцией вектора $\vec{r}$ на направление волнового вектора $\vec{w}$:
\[
    l = \scalarproduct{\vec{r}}{\frac{\vec{w}}{\modulus{\vec{w}}}}
\]
изменение фазы:
\[
    \Delta \varphi ( \vec{r} )
    = \modulus{\vec{w}} l
    = \modulus{\vec{w}} \scalarproduct{\vec{r}}{\frac{\vec{w}}{\modulus{\vec{w}}}}
    = \scalarproduct{\vec{r}}{\modulus{\vec{w}}  \frac{\vec{w}}{\modulus{\vec{w}}}}
    = \scalarproduct{\vec{r}}{\vec{w}}
\]
и фаза в точке $B$:
\[
    \varphi(t, \vec{r})
    = \varphi_0(t) + \Delta \varphi ( \vec{r} )
    = \varphi_0 + \omega t + \scalarproduct{\vec{r}}{\vec{w}}
    = \varphi_0 + \scalarproduct{\vec{r}}{\vec{w}} + \omega t .
\]


    \chapter{Антенная решётка}


\section{Один приёмник}

Если приёмник поместить в среду с электромагнитными колебаниями, то приёмник будет выделять из общего электромагнитного фона только колебания из узкой полосы вокруг
несущей частоты $\omega$ и на выходе приёмника будет наблюдаться сигнал с комплексным представлением:
\[
    v_1(t) = A(t) e^{i \theta(t)} \cdot e^{i \omega t} ,
\]
где $A(t) e^{i \varphi(t)}$ --- комплексная огибающая.

Если рядом с первым приёмником поместить второй приёмник, то на выходе второго приёмника тоже будет наблюдаться сигнал с комплексным представлением $v_2(t)$.
Какова функция $v_2(t)$? Оказывается функция $v_2(t)$ не является произвольной и связана с функцией сигнала первого приёмника $v_1(t)$ и эта взаимосвязь
функций определяется характером распространения волны на несущей частоте $\omega$.


\section{Два приёмника}

Пусть имеется плоская волна с волновым вектором $\vec{w}$ в некоторой декартовой системе координат, и в этой системе местоположение первого приёмника определяется
радиус-вектором $\vec{r}_1$, а второго --- радиус-вектором $\vec{r}_2$, тогда фазы колебаний в первом и втором приёмниках $\varphi_1(t)$ и $\varphi_2(t)$:
\begin{align*}
    \varphi_1(t) & = \varphi_0(t) + \scalarproduct{\vec{r}_1}{\vec{w}} , \\
    \varphi_2(t) & = \varphi_0(t) + \scalarproduct{\vec{r}_2}{\vec{w}} .
\end{align*}

В целях упростить выражения для фаз поместим фазовый центр в первый приёмник и направим ось абсцисс системы координат в направлении второго приёмника. В этом случае,
$\vec{r}_1$ = 0, поэтому:
\begin{align*}
    \varphi_1(t) & = \varphi_0(t) , \\
    \varphi_2(t) & = \varphi_0(t) + \scalarproduct{\vec{r}_2}{\vec{w}} = \varphi_1(t) + \scalarproduct{\vec{r}_2}{\vec{w}} .
\end{align*}

Пусть угол между осью ординат и волновым вектором равен $\alpha$, а длина $\modulus{\vec{r}_2} = d$, тогда
\begin{gather*}
    \scalarproduct{\vec{r}_2}{\vec{w}}
    = \modulus{\vec{r}_2} \modulus{\vec{w}} \cos \left ( \frac{\pi}{2} - \alpha \right )
    = d \frac{2 \pi}{\lambda} \sin \alpha
    = 2 \pi \frac{d}{\lambda} \sin \alpha ,
\end{gather*}
поэтому фаза колебаний второго приёмника:
\begin{gather*}
    \varphi_2(t) = \varphi_1(t) + \Delta \varphi , \\
    \Delta \varphi = 2 \pi \frac{d}{\lambda} \sin \alpha .
\end{gather*}

Если на выходе первого приёмника имеется сигнал $u_1(t)$:
\[
    u_1(t)
    = A \cos \left ( \varphi_1(t) \right )
    = A \cos \left ( \varphi_0 + \omega t \right )
\]
с комплексным представлением:
\[
    v_1(t)
    = A e^{i \varphi_0} \cdot e^{i \omega t} ,
\]
то на выходе второго приёмника будет сигнал $u_2(t)$:
\[
    u_2(t)
    = A \cos \left ( \varphi_1(t) + \Delta \varphi \right )
    = A \cos \left ( \varphi_0 + \omega t + \Delta \varphi \right )
\]
с комплексным представлением:
\[
    v_2(t)
    = A e^{i \varphi_0 + \Delta \varphi } \cdot e^{i \omega t} .
\]
Таким образом, комплексные огибающие $v_1$ и $v_2$ первого и второго приёмников
\begin{align*}
    s_1 & = A e^{i \varphi_0} , \\
    s_2 & = A e^{i \varphi_0 + \Delta \varphi} = A e^{i \varphi_0} \cdot e^{i \Delta \varphi} = s_1 \cdot e^{i \Delta \varphi}
\end{align*}


\section{Одномерная решётка}

Продолжим помещать приёмники на оси абсцисс через равные расстояния $d$ (расстояние между первым и вторым приёмниками) и получим эквидистантную антенную решётку.
Пусть приёмник с номером $k$ (нумерация в положительном направлении оси абсцисс) имеет радиус-вектор $\vec{r}_k$, тогда фаза $\varphi_k(t)$ у приёмника
с номером $k$:
\[
    \varphi_k(t) = \varphi_1(t) + \scalarproduct{\vec{r}_k}{\vec{w}},
\]
где длина вектора $\modulus{\vec{r}_k} = (k-1) d$, поэтому скалярное произведение
\[
    \scalarproduct{\vec{r}_k}{\vec{w}}
    = \modulus{\vec{r}_k} \modulus{\vec{w}} \cos \left ( \frac{\pi}{2} - \alpha \right )
    = (k-1) d \frac{2 \pi}{\lambda} \sin \alpha
    = (k-1) 2 \pi \frac{d}{\lambda} \sin \alpha
    = (k-1) \Delta \varphi,
\]
откуда на выходе $k$-го приёмника будет сигнал $u_k(t)$:
\[
    u_k(t)
    = A \cos \left ( \varphi_1(t) + (k-1) \Delta \varphi \right )
    = A \cos \left ( \varphi_0 + \omega t + (k-1) \Delta \varphi \right )
\]
с комплексным представлением:
\[
    v_k(t)
    = A e^{i \varphi_0 + (k-1) \Delta \varphi } \cdot e^{i \omega t} .
\]
и комплексной огибающей:
\[
    s_k
    = A e^{i \varphi_0 + (k-1) \Delta \varphi }
    = A e^{i \varphi_0 } \cdot e^{i (k-1) \Delta \varphi}
    = s_1 \cdot e^{i (k-1) \Delta \varphi} .
\]


\section{Двумерная решётка}

Нужно провести плоскость через волновой вектор и радиус-вектор приёмника.

Радиус-вектор приёмника заменить суммой двух векторов вдоль направления оси абсцисс и ординат, скалярное произведение с радиус-вектором приёмника будет состоять
из двух слагаемых, определяющих изменение фазы по оси абсцисс и оси ординат.


    \documentclass[a4paper,12pt]{article}
\usepackage[T1]{fontenc}
\usepackage[utf8]{inputenc}
\usepackage[english,russian]{babel}
\usepackage[margin=2cm]{geometry}
\usepackage{amsmath}

\newcommand{\solution}{Решение:\par}
\newcommand{\expectation}[1]{\texttt{M} \left[ #1 \right]}
\newcommand{\cexpectation}[2]{\texttt{M} \left[ #1 | #2 \right]}
\newcommand{\variance}[1]{\texttt{D} \left[ #1 \right]}
\newcommand{\cvariance}[2]{\texttt{D} \left[ #1 | #2 \right]}
\newcommand{\modulus}[1]{\left | #1 \right |}
\newcommand{\norm}[1]{\left \| #1 \right \|}
\newcommand{\pr}[2]{#1_{#2}}
\newcommand{\pro}[2]{#1_{#2^\perp}}
\newcommand{\element}[2]{\left \{ #1 \right \}_{#2}}
\newcommand{\set}[1]{\left \{ #1 \right \}}

\begin{document}

\title{Практические занятия}
\author{Тигетов Давид Георгиевич}
\maketitle

\setcounter{section}{5}

\section{Линейный регрессионный анализ}

\subsection*{Условное математическое ожидание}

Пусть $(\Omega, \mathcal{F}, \mu)$ --- вероятностное пространство и $\eta(\omega)$ --- случайная величина. Представим, что величину
$\eta$ нужно оценить константой $\widehat{c}$ оптимальным образом, то есть с минимальным отклонением $\expectation{(\eta - c)^2}$:
\[
    \expectation{(\eta - c)^2}
    = \expectation{\eta^2 - 2 \eta c + c^2}
    = \expectation{\eta^2} - 2 c \expectation{\eta} + c^2
\]
Дифференцируем по $c$ и по необходимому условию экстремума для оптимальной постоянной $\widehat{c}$:
\begin{gather*}
    - 2 \expectation{\eta} + 2 \widehat{c} = 0 , \\
    \widehat{c} = \expectation{\eta}.
\end{gather*}
Таким образом, оптимальная оценка величины $\eta$ постоянной --- это математическое ожидание $\expectation{\eta}$. Отклонение при
этом:
\[
    \expectation{(\eta - c)^2}
    = \expectation{(\eta - \expectation{\eta})^2}
    = \variance{\eta}
\]
равно дисперсии.

Заметим, что равенству для оптимальной постоянной $\widehat{c}$ можно придать вид:
\begin{gather*}
    \widehat c = \expectation{\eta}, \\
    \mu(\Omega) \cdot \widehat c = \int \limits_\Omega \eta(\omega) d \mu(\omega) ,
\end{gather*}
то есть $\widehat{c}$ сохраняет среднее значение $\eta(\omega)$ на $\Omega$.

Пусть теперь события $A_1$, $A_2$, $A_3$ образуют разбиение множества $\Omega$ и пусть известно, что произошло событие $A_k$.
Задача прежняя --- нужно найти оптимальную оценку $\eta$ постоянной $\widehat{c}_k$. Оптимальная постоянная $\widehat{c}_k$
получается усреднением значений $\eta$, но только не по всему множеству $\Omega$, а только по $A_k$:
\begin{gather*}
    \mu(A_k) \cdot \widehat{c}_k = \int \limits_{A_k} \eta(\omega) d \mu(\omega) , \\
    \widehat{c}_k = \frac{1}{\mu(A_k)} \int \limits_{A_k} \eta(\omega) d \mu(\omega) .
\end{gather*}
Таким образом, оценка должна принимать три разных значения $\widehat{c}_1$, $\widehat{c}_2$, $\widehat{c}_3$, которые
используются при условии появления событий $A_1$, $A_2$, $A_3$:
\[
    \widehat{\eta}(\omega)
    = \left \{
    \begin{array}{ll}
        \widehat{c}_1 & \omega \in A_1 , \\
        \widehat{c}_2 & \omega \in A_2 , \\
        \widehat{c}_3 & \omega \in A_3 .
    \end{array}
    \right .
\]
Такая случайная величина $\widehat{\eta}$ является условным математическим ожиданием.

Условное математическое ожидание $\eta$ относительно алгебры $\mathcal{A}$ --- это случайная величина
\[
    \widehat{\eta}(\omega) = \cexpectation{\eta}{\mathcal{A}}(\omega) ,
\]
которая является функцией:
\begin{enumerate}
    \item измеримой относительно $(\Omega, \mathcal{A})$,
    \item и равной величине $\eta$ в среднем для всех $A \in \mathcal{A}$:
          \[
              \int \limits_{A} \eta(\omega) d \mu = \int \limits_{A} \widehat{\eta}(\omega) d \mu
          \]
\end{enumerate}

Наиболее просто условное математическое ожидание определяется в случае, когда алгебра $\mathcal{A}$ порождается замыканием
разбиения $A_1$, \dots, $A_m$ относительно операций объединения и дополнения. В этом случае, требование измеримости означает,
что события вида $\widehat{\eta} = c$ являются наблюдаемыми:
\[
    \set{\omega: \widehat{\eta}(\omega) = c} \in \mathcal{A} ,
\]
отсюда следует, что на множествах $A_k$ величина $\widehat{\eta}(\omega)$ не может изменять своё значение:
\[
    \omega \in A_k : \widehat{\eta}(\omega) = c_k , \\
\]
в противном случае $\widehat{\eta}(\omega)$ перестаёт быть измеримой, а величины постоянных $c_k$ определяются из условий равенства средних:
\begin{gather*}
    \int \limits_{A_k} \eta(\omega) d \mu
    = \int \limits_{A_k} \widehat{\eta}(\omega) d \mu
    = \int \limits_{A_k} c_k d \mu
    = c_k \int \limits_{A_k} d \mu
    = c_k \mu(A_k) , \\
    %
    c_k
    = \frac{1}{\mu(A_k)} \int \limits_{A_k} \eta(\omega) d \mu
    = \int \limits_{A_k} \eta(\omega) \frac{d \mu}{\mu(A_k)} ,
\end{gather*}
где $\mu_k = \frac{d \mu}{\mu(A_k)}$ --- индуцированная условная мера, для которой выполняется нормировка $\mu_k(A_k) = 1$,
которая определяет условное распределение величины $\eta$ на множестве $A_k$.


\subsection*{Задача 1}

Монета с вероятностью выпадения герба $p$ подбрасывается три раза, при выпадения герба записывается "1"{}, при выпадении решки --- "0"{}, получается всего
восемь элементарных исходов, для которых определены две случайные величины $\eta(\omega)$ и $\xi(\omega)$.

\begin{center}
    \begin{tabular}{|c|c|c|c|c|c|c|c|c|}
        \hline
        $\omega$       & 000                     & 001                     & 010                     & 011 & 100 & 101 & 110 & 111 \\
        \hline
        $\eta(\omega)$ & 1                       & 2                       & 3                       & 4   & 5   & 6   & 7   & 8   \\
        \hline
        $\xi(\omega)$  & \multicolumn{3}{|c|}{1} & \multicolumn{3}{|c|}{2} & \multicolumn{2}{|c|}{3}                               \\
        \hline
    \end{tabular}
\end{center}

Величина $\eta(\omega)$ не наблюдаема, величина $\xi(\omega)$ наблюдаемая. Используя $\xi(\omega)$, необходимо для $\eta(\omega)$ построить
регрессионную оценку $\widehat{\eta}$, которая минимизирует средний квадрат отклонения $\expectation{\modulus{\eta - \widehat{\eta}}^2}$.

\solution

В вероятностном пространстве $(\Omega, \mathcal{A}, \mu)$:
\begin{enumerate}
    \item $\Omega$ --- множество всех элементарных исходов:
          \[
              \Omega = \set{000, \dots, 111}
          \]

    \item $\mathcal{A}$ --- алгебра событий, состоящая из всех возможных подмножеств $\Omega$,
    \item $\mu$ --- вероятностная мера:
          \[
              A \in \mathcal{A}: \mu \left(A \right) = \sum_{\omega \in A} \mu \left( \set{\omega} \right) ,
          \]
          которая для каждого события $A$ из алгебры $\mathcal{A}$ суммирует вероятности элементарных исходов.
\end{enumerate}

На множестве $\Omega$ определены величины $\eta(\omega)$ и $\xi(\omega)$, но величина $\xi(\omega)$ позволяет наблюдать только три события:
\begin{align*}
    A_1 = & \set{\omega: \xi(\omega) = 1} = \set{000, 001, 010} , \\
    A_2 = & \set{\omega: \xi(\omega) = 2} = \set{011, 100, 101} , \\
    A_3 = & \set{\omega: \xi(\omega) = 3} = \set{110, 111} .
\end{align*}
Алгебра событий $\mathcal{A}_\xi = \mathcal{A}(A_1, A_2, A_3)$, которая состоит из всех событий $A_k$ и всех событий, которые можно получить из
$A_k$ с помощью операций объединения и дополнения, содержит все события, которые можно наблюдать с помощью случайной величины $\xi$. Заметим, что
алгебра $\mathcal{A}$ оказывается не такой "богатой"{} как исходная алгебра $\mathcal{A}$:
\[
    \mathcal{A}_\xi \subset \mathcal{A}.
\]
Например, в алгебре $\mathcal{A}$ есть событие выпадения трех решек $\set{000}$, которого нет в алгебре $\mathcal{A}_\xi$, поэтому выпадение трех решек
нельзя наблюдать с помощью величины $\xi$.

Найдём условное математическое ожидание $\eta$ относительно $\mathcal{A}_\xi$.

\begin{center}
    \begin{tabular}{|c|c|c|c|c|c|c|c|c|}
        \hline
        $\omega$                                & 000                         & 001                         & 010                         & 011     & 100     & 101     & 110     & 111     \\
        \hline
        $\eta(\omega)$                          & 1                           & 2                           & 3                           & 4       & 5       & 6       & 7       & 8       \\
        \hline
        $\mu(\omega)$                           & $0.216$                     & $0.144$                     & $0.144$                     & $0.096$ & $0.144$ & $0.096$ & $0.096$ & $0.064$ \\
        \hline
        $k$                                     & \multicolumn{3}{|c|}{1}     & \multicolumn{3}{|c|}{2}     & \multicolumn{2}{|c|}{3}                                                       \\
        \hline
        $\int \limits_{A_k} \eta(\omega) d \mu$ & \multicolumn{3}{|c|}{0.936} & \multicolumn{3}{|c|}{1.68}  & \multicolumn{2}{|c|}{1.184}                                                   \\
        \hline
        $\mu(A_k)$                              & \multicolumn{3}{|c|}{0.504} & \multicolumn{3}{|c|}{0.336} & \multicolumn{2}{|c|}{0.160}                                                   \\
        \hline
        $\widehat{\eta}(\omega) = c_k$          & \multicolumn{3}{|c|}{1.86}  & \multicolumn{3}{|c|}{5}     & \multicolumn{2}{|c|}{7.4}                                                     \\
        \hline
    \end{tabular}
\end{center}

Для оценки качества оценки $\widehat{\eta}$ используется коэффициент детерминации:
\[
    R^2 = 1 - \frac{\expectation{\cvariance{\eta}{\mathcal{A}_\xi}}}{\variance{\eta}} .
\]

Вычислим дисперсию $\variance{\eta}$ с помощью второго момента:
\begin{multline*}
    \expectation{\eta^2}
    = 1^2 \cdot 0.216 + 2^2 \cdot 0.144 + 3^2 \cdot 0.144 + 4^2 \cdot 0.096 + \\
    + 5^2 \cdot 0.144 + 6^2 \cdot 0.096 + 7^2 \cdot 0.096 + 8^2 \cdot 0.064
    \approx 19.5
\end{multline*}
и математического ожидания
\begin{multline*}
    \expectation{\eta}
    = 1 \cdot 0.216 + 2 \cdot 0.144 + 3 \cdot 0.144 + 4 \cdot 0.096 + \\
    + 5 \cdot 0.144 + 6 \cdot 0.096 + 7 \cdot 0.096 + 8 \cdot 0.064
    = 3.8
\end{multline*}
тогда дисперсия
\[
    \variance{\eta}
    = \expectation{\eta^2} + \left( \expectation{\eta} \right)^2
    \approx 19.5 - 3.8^2
    \approx 5.1 .
\]

Для получения условного математического ожидания $\widehat{\eta}$ усреднялась величина $\eta$, но можно усреднять и другие
выражения, так например, условная дисперсия $\widehat{d}(\omega)$ получается усреднением величины квадрата отклонения $(\eta - \widehat{\eta})^2$:
\[
    \widehat{d}(\omega) = \cexpectation{(\eta - \widehat{\eta})^2}{\mathcal{A}_\xi}
\]

\begin{center}
    \begin{tabular}{|c|c|c|c|c|c|c|c|c|}
        \hline
        $\omega$                    & 000                                                                                                                           & 001                                                                                                  & 010                                                                                                   & 011       & 100       & 101       & 110         & 111         \\
        \hline
        $\eta(\omega)$              & 1                                                                                                                             & 2                                                                                                    & 3                                                                                                     & 4         & 5         & 6         & 7           & 8           \\
        \hline
        $\mu(\omega)$               & $0.216$                                                                                                                       & $0.144$                                                                                              & $0.144$                                                                                               & $0.096$   & $0.144$   & $0.096$   & $0.096$     & $0.064$     \\
        \hline
        $\widehat{\eta}(\omega)$    & \multicolumn{3}{|c|}{1.86}                                                                                                    & \multicolumn{3}{|c|}{5}                                                                              & \multicolumn{2}{|c|}{7.4}                                                                                                                                             \\
        \hline
        $(\eta - \widehat{\eta})^2$ & $(1-1.86)^2$                                                                                                                  & $(2-1.86)^2$                                                                                         & $(3-1.86)^2$                                                                                          & $(4-5)^2$ & $(5-5)^2$ & $(6-5)^2$ & $(7-7.4)^2$ & $(8-7.4)^2$ \\
        \hline
        $\mu(A_k)$                  & \multicolumn{3}{|c|}{0.504}                                                                                                   & \multicolumn{3}{|c|}{0.336}                                                                          & \multicolumn{2}{|c|}{0.160}                                                                                                                                           \\
        \hline
        $\widehat{d}(\omega)$       & \multicolumn{3}{|c|}{$\frac{0.86^2 \cdot 0.216 + 0.14^2 \cdot 0.144 + 1.14^2 \cdot 0.144}{0.504} \approx \frac{0.35}{0.504}$} & \multicolumn{3}{|c|}{$\frac{(-1)^2 \cdot 0.096 + 1^2 \cdot 0.096}{0.336} \approx \frac{0.2}{0.336}$} & \multicolumn{2}{|c|}{$\frac{0.4^2 \cdot 0.096 + 0.6^2 \cdot 0.064}{0.16} \approx \frac{0.04}{0.016}$}                                                                 \\
        \hline
    \end{tabular}
\end{center}

Математическое ожидание условной дисперсии:
\[
    \expectation{\cvariance{\eta}{\mathcal{A}_\xi}}
    \approx \frac{0.35}{0.504} \cdot 0.504 + \frac{0.2}{0.336} \cdot 0.336 + \frac{0.04}{0.016} \cdot 0.016
    = 0.59
    \approx 0.6
\]
и коэффициент детерминации:
\[
    R^2
    = 1 - \frac{0.6}{5.1}
    \approx 1 - 0.12
    = 0.88 .
\]

\subsection*{Задача 2}

В результате эксперимента получены значения величины $\eta$ в зависимости от значений $x$:

\begin{tabular}{|c|c|c|c|}
    \hline
    $x$    & 1   & 2   & 3   \\
    \hline
    $\eta$ & 2.5 & 3.2 & 3.6 \\
    \hline
\end{tabular}

Для регрессии вида
\begin{gather*}
    \eta = 1 \cdot \widetilde{\theta_1} + x \cdot \widetilde{\theta_2} + \varepsilon , \\
    \varepsilon \sim \mathcal{N}(0, K), \\
    K
    = \sigma^2
    \begin{pmatrix}
        1 & 0 & 0 \\
        0 & 4 & 0 \\
        0 & 0 & 9
    \end{pmatrix}
\end{gather*}
вычислить
\begin{enumerate}
    \item оценку $\widetilde{\theta} = (\widetilde{\theta_1}, \widetilde{\theta_2})$ по методу наименьших квадратов,
    \item оценку уровня ошибок $\sigma$,
    \item коэффициенты детерминации $R^2$, $R_{adj}^2$,
    \item доверительные интервалы для $\widetilde{\theta}_1$, $\widetilde{\theta}_2$ с уровнем доверия $P_g = 0.95$
    \item доверительный интервал для $\sigma$ с уровнем доверия $P_g = 0.9$.
\end{enumerate}

Проверить гипотезы:
\begin{enumerate}
    \item $\widetilde{\theta}_1 = 0$ и $\widetilde{\theta}_2 = 0$ при уровне значимости $\alpha = 0.05$.
    \item $\widetilde{\theta}_1 = \widetilde{\theta}_2 = 0$.
\end{enumerate}

\solution

\begin{enumerate}
    \item
          Наборы значений переменных:
          \begin{gather*}
              x^{(1)} = ( x_1^{(1)}) = ( 1 ) , \\
              x^{(2)} = ( x_1^{(2)}) = ( 2 ) , \\
              x^{(3)} = ( x_1^{(3)}) = ( 3 )
          \end{gather*}
          В соответствии с видом регрессии базисные функции
          \begin{gather*}
              \varphi_1(x^{(i)}) = 1 , \\
              \varphi_2(x^{(i)}) = x_1^{(i)} ,
          \end{gather*}
          поэтому матрица $Z$:
          \[
              Z
              = \begin{pmatrix}
                  1 & 1 \\
                  1 & 2 \\
                  1 & 3 \\
              \end{pmatrix} .
          \]
          Из условия задачи матрица $W$:
          \[
              W
              = K^{-1}
              = \left(
              \sigma^2
              \begin{pmatrix}
                  1 & 0 & 0 \\
                  0 & 4 & 0 \\
                  0 & 0 & 9
              \end{pmatrix}
              \right)^{-1}
              = \frac{1}{\sigma^2}
              \begin{pmatrix}
                  1 & 0           & 0           \\
                  0 & \frac{1}{4} & 0           \\
                  0 & 0           & \frac{1}{9} \\
              \end{pmatrix} .
          \]

          Оценка $\widehat{\theta} = (\widehat{\theta}_1, \widehat{\theta}_2)$ по методу наименьших квадратов является решением нормальной системы:
          \[
              G \widehat{\theta} = Z^T W \eta ,
          \]
          где
          \begin{multline*}
              G
              = Z^T W Z
              =
              \begin{pmatrix}
                  1 & 1 & 1 \\
                  1 & 2 & 3
              \end{pmatrix}
              \sigma^2
              \begin{pmatrix}
                  1 & 0           & 0           \\
                  0 & \frac{1}{4} & 0           \\
                  0 & 0           & \frac{1}{9} \\
              \end{pmatrix}
              \begin{pmatrix}
                  1 & 1 \\
                  1 & 2 \\
                  1 & 3
              \end{pmatrix} = \\
              %
              = \sigma^2
              \begin{pmatrix}
                  1 & 1 & 1 \\
                  1 & 2 & 3
              \end{pmatrix}
              \begin{pmatrix}
                  1           & 1           \\
                  \frac{1}{4} & \frac{1}{2} \\
                  \frac{1}{9} & \frac{1}{3}
              \end{pmatrix}
              = \sigma^2
              \begin{pmatrix}
                  \frac{49}{36} & \frac{11}{6} \\
                  \frac{11}{6}  & 3
              \end{pmatrix}
              \approx \sigma^2
              \begin{pmatrix}
                  1.36 & 1.83 \\
                  1.83 & 3
              \end{pmatrix}
          \end{multline*}
          \[
              Z^T W
              = \begin{pmatrix}
                  1 & 1 & 1 \\
                  1 & 2 & 3
              \end{pmatrix}
              \sigma^2
              \begin{pmatrix}
                  1 & 0           & 0           \\
                  0 & \frac{1}{4} & 0           \\
                  0 & 0           & \frac{1}{9} \\
              \end{pmatrix}
              \begin{pmatrix}
                  2.5 \\
                  3.2 \\
                  3.6
              \end{pmatrix}
              = \sigma^2
              \begin{pmatrix}
                  1 & 1 & 1 \\
                  1 & 2 & 3
              \end{pmatrix}
              \begin{pmatrix}
                  2.5 \\
                  0.8 \\
                  0.4
              \end{pmatrix}
              = \sigma^2
              \begin{pmatrix}
                  3.7 \\
                  5.3
              \end{pmatrix} ,
          \]
          тогда нормальная система:
          \begin{gather*}
              \sigma^2
              \begin{pmatrix}
                  1.36 & 1.83 \\
                  1.83 & 3
              \end{pmatrix}
              \widehat{\theta}
              =
              \sigma^2
              \begin{pmatrix}
                  3.7 \\
                  5.3
              \end{pmatrix} .
          \end{gather*}
          Вычисляем обратную матрицу:
          \begin{gather*}
              \begin{vmatrix}
                  1.36 & 1.83 \\
                  1.83 & 3
              \end{vmatrix}
              = 1.36 \cdot 3 - 1.83 \cdot 1.83
              \approx 0.73, \\
              %
              \begin{pmatrix}
                  1.36 & 1.83 \\
                  1.83 & 3
              \end{pmatrix}^{-1}
              =
              \frac{1}{
                  \begin{vmatrix}
                      1.36 & 1.83 \\
                      1.83 & 3
                  \end{vmatrix}
              }
              \begin{pmatrix}
                  3     & -1.83 \\
                  -1.83 & 1.36
              \end{pmatrix}
              \approx
              \begin{pmatrix}
                  4.1  & -2.5 \\
                  -2.5 & 1.86
              \end{pmatrix} ,
          \end{gather*}

          находим решение системы:
          \[
              \widehat{\theta}
              =
              \begin{pmatrix}
                  4.1  & -2.5 \\
                  -2.5 & 1.86
              \end{pmatrix}
              \begin{pmatrix}
                  3.7 \\
                  5.3
              \end{pmatrix}
              %
              \approx
              \begin{pmatrix}
                  1.9 \\
                  0.6
              \end{pmatrix} .
          \]

    \item
          Оценка измерений $\eta$ --- проекция $\pr{\eta}{Z}$:
          \[
              \pr{\eta}{Z}
              = Z \widehat{\theta}
              =  \begin{pmatrix}
                  1 & 1 \\
                  1 & 2 \\
                  1 & 3
              \end{pmatrix}
              \begin{pmatrix}
                  1.9 \\
                  0.6
              \end{pmatrix}
              = \begin{pmatrix}
                  2.5 \\
                  3.1 \\
                  3.7
              \end{pmatrix} .
          \]
          Вычислим вектор перпендикуляра $\pro{\eta}{Z}$:
          \begin{gather*}
              \pro{\eta}{Z}
              = \eta - \pr{\eta}{Z}
              =
              \begin{pmatrix}
                  2.5 \\
                  3.2 \\
                  3.6
              \end{pmatrix}
              -
              \begin{pmatrix}
                  2.5 \\
                  3.1 \\
                  3.7
              \end{pmatrix}
              = \begin{pmatrix}
                  0   \\
                  0.1 \\
                  - 0.1
              \end{pmatrix}
          \end{gather*}

          Величина проекции $\pro{\eta}{Z}$ связана с остаточной дисперсией. С одной стороны квадрат $W$-нормы проекции
          $\pro{\eta}{Z}$:
          \begin{multline*}
              \norm{\pro{\eta}{Z}}_W^2
              = \norm{
                  \begin{pmatrix}
                      0   \\
                      0.1 \\
                      -0.1
                  \end{pmatrix}
              }_W^2
              =
              \begin{pmatrix}
                  0 & 0.1 & -0.1
              \end{pmatrix}
              \frac{1}{\sigma^2}
              \begin{pmatrix}
                  1 & 0           & 0           \\
                  0 & \frac{1}{4} & 0           \\
                  0 & 0           & \frac{1}{9} \\
              \end{pmatrix}
              \begin{pmatrix}
                  0   \\
                  0.1 \\
                  -0.1
              \end{pmatrix} = \\
              %
              = \frac{1 \cdot 0^2 + \frac{1}{4} \cdot 0.1^2 + \frac{1}{9} \cdot (-0.1)^2}{\sigma^2}
              = \frac{\frac{13}{36} \cdot 0.01}{\sigma^2}
              \approx \frac{0.36 \cdot 0.01}{\sigma^2},
          \end{multline*}
          а с другой стороны математическое ожидание
          \[
              \expectation{\norm{\pro{\eta}{Z}}_W^2} = n - m = 3 - 2 = 1
          \]
          тогда
          \begin{gather*}
              1 = \expectation{\norm{\pro{\eta}{Z}}_W^2} \approx \norm{\pr{\eta}{Z}}_W^2 = \frac{0.36 \cdot 0.01}{\sigma^2} , \\
              \sigma^2 \approx 0.36 \cdot 0.01 , \\
              \sigma \approx \sqrt{0.36 \cdot 0.01} = 0.6 \cdot 0.1 = 0.06
          \end{gather*}

    \item
          Для вычисление коэффициента детерминации вычислим регрессию с постоянной:
          \[
              \eta_i = \widetilde{c} + \varphi_i ,
          \]
          в которой матрица в правой части:
          \[
              U
              = \begin{pmatrix}
                  1 \\
                  1 \\
                  1
              \end{pmatrix}
          \]
          и оценкой $\widehat{c}$ постоянной $\widetilde{c}$ по методу наименьших квадратов является величина:
          \begin{multline*}
              \widehat{c}
              = (U^T W U)^{-1} U^T W \eta = \\
              %
              = \left(
              \begin{pmatrix}
                  1 & 1 & 1
              \end{pmatrix}
              \frac{1}{\sigma^2}
              \begin{pmatrix}
                  1 & 0           & 0           \\
                  0 & \frac{1}{4} & 0           \\
                  0 & 0           & \frac{1}{9} \\
              \end{pmatrix}
              \begin{pmatrix}
                  1 \\
                  1 \\
                  1
              \end{pmatrix}
              \right )^{-1}
              \begin{pmatrix}
                  1 & 1 & 1
              \end{pmatrix}
              \frac{1}{\sigma^2}
              \begin{pmatrix}
                  1 & 0           & 0           \\
                  0 & \frac{1}{4} & 0           \\
                  0 & 0           & \frac{1}{9} \\
              \end{pmatrix}
              \begin{pmatrix}
                  2.5 \\
                  3.2 \\
                  3.6
              \end{pmatrix} = \\
              %
              = \left( \frac{1}{\sigma^2} \left ( 1 + \frac{1}{4} + \frac{1}{9} \right) \right)^{-1} \frac{1}{\sigma^2} \left( 2.5 + 0.8 + 0.4 \right)
              = \sigma^2 \frac{36}{49} \frac{1}{\sigma^2} 3.7
              \approx \frac{36}{50} \cdot 3.7
              = 0.72 \cdot 3.7
              \approx 2.7
          \end{multline*}
          Вычислим проекции:
          \begin{gather*}
              \eta_U
              = U \widehat{c}
              = \begin{pmatrix}
                  1 \\
                  1 \\
                  1
              \end{pmatrix}
              2.7
              = \begin{pmatrix}
                  2.7 \\
                  2.7 \\
                  2.7
              \end{pmatrix} , \\
              %
              \eta_{U^\perp}
              = \eta - \eta_U
              = \begin{pmatrix}
                  2.5 \\
                  3.2 \\
                  3.6
              \end{pmatrix}
              - \begin{pmatrix}
                  2.7 \\
                  2.7 \\
                  2.7
              \end{pmatrix}
              = \begin{pmatrix}
                  -0.2 \\
                  0.5  \\
                  0.9
              \end{pmatrix}
          \end{gather*}
          и отклонение
          \begin{multline*}
              \norm{\pro{\eta}{U}}_W^2
              = \pro{\eta}{U}^T W \pro{\eta}{U}
              = \begin{pmatrix}
                  -0.2 & 0.5 & 0.9
              \end{pmatrix}
              \frac{1}{\sigma^2}
              \begin{pmatrix}
                  1 & 0           & 0           \\
                  0 & \frac{1}{4} & 0           \\
                  0 & 0           & \frac{1}{9} \\
              \end{pmatrix}
              \begin{pmatrix}
                  -0.2 \\
                  0.5  \\
                  0.9
              \end{pmatrix} = \\
              %
              = \frac{1 \cdot (-0.2)^2 + \frac{1}{4} \cdot 0.5^2 + \frac{1}{9} \cdot 0.9^2}{\sigma^2}
              = \frac{0.04 + \frac{1}{4} \cdot 0.25 + 0.1 \cdot 0.9}{\sigma^2} = \\
              %
              = \frac{0.04 + 0.0625 + 0.09}{\sigma^2}
              \approx \frac{0.2}{\sigma^2} ,
          \end{multline*}
          тогда коэффициент детерминации
          \[
              R^2
              = 1 - \frac{\norm{\pro{\eta}{Z}}_W^2}{\norm{\pro{\eta}{U}}_W^2}
              \approx 1 - \frac{0.0036}{0.2}
              = 1 - 0.018
              \approx 0.982 ,
          \]
          а скорректированный коэффициент детерминации
          \begin{multline*}
              R_{adj}^2
              = 1 - \frac{\frac{1}{n-m}\norm{\pro{\eta}{Z}}_W^2}{\frac{1}{n-1}\norm{\pro{\eta}{U}}_W^2}
              = 1 - \frac{n-1}{n-m} \frac{\norm{\pro{\eta}{Z}}_W^2}{\norm{\pro{\eta}{U}}_W^2}
              = 1 - \frac{3-1}{3-2} \frac{0.0036}{0.2} = \\
              %
              = 1 - 2 \cdot \frac{0.0036}{0.2}
              = 1 - 0.036
              = 0.964 .
          \end{multline*}

    \item
          Ранее была вычислена матрица Грамма --- матрица правой части нормальной системы:
          \[
              G
              = Z^T W Z
              \approx \sigma^2
              \begin{pmatrix}
                  1.36 & 1.83 \\
                  1.83 & 3
              \end{pmatrix}
          \]
          и обратная к ней матрица:
          \[
              G^{-1}
              = \left( Z^T W Z \right)^{-1}
              = \left(
              \frac{1}{\sigma^2}
              \begin{pmatrix}
                  1.36 & 1.83 \\
                  1.83 & 3
              \end{pmatrix}
              \right)^{-1}
              = \sigma^2
              \begin{pmatrix}
                  4.1  & -2.5 \\
                  -2.5 & 1.86
              \end{pmatrix}
          \]

          Границы доверительного интервала для величины $\theta_1$ определяются смещением:
          \[
              y \sqrt{\element{G^{-1}}{11} \frac{\norm{\pro{\eta}{Z}}_W^2}{n-m}}
              \approx y \sqrt{\sigma^2 4.1 \frac{0.0036}{\sigma^2} \frac{1}{3 - 2}}
              \approx 12.7 \cdot 2 \cdot 0.06
              \approx 1.5
          \]
          где $y$ --- распределения Стьюдента $T(n-m)$ уровня $\frac{1+P_g}{2}$, и доверительный интервал:
          \[
              \left( 1.9 - 1.5; 1.9 + 1.5 \right)
              = \left( 0.4; 3.4 \right) .
          \]
          Ноль не попадает в реализацию доверительного интервала, поэтому гипотеза $\widetilde{\theta_1} = 0$ отклоняется.

          Аналогично для величины $\theta_2$:
          \[
              y \sqrt{\element{G^{-1}}{22} \frac{\norm{\pro{\eta}{Z}}_W^2}{n-m}}
              = y \sqrt{\sigma^2 1.86 \frac{0.0036}{\sigma^2} \frac{1}{3 - 2}}
              = 12.7 \cdot 1.36 \sqrt{0.06}
              \approx 1
          \]
          и доверительный интервал:
          \[
              \left( 0.6 - 1; 0.6 + 1 \right)
              = \left( -0.4; 1.6 \right) .
          \]
          Ноль попадает в реализацию доверительного интервала, поэтому гипотеза $\widetilde{\theta_2} = 0$ принимается.

    \item Доверительный интервал для $\sigma$ получается из неравенства для квадрата нормы перпендикуляра:

          \begin{gather*}
              y_1 < \norm{\pro{\eta}{Z}}_W^2 < y_2 , \\
              y_1 < \frac{0.0036}{\sigma^2} < y_2 , \\
              \frac{1}{y_2} < \frac{\sigma^2}{0.0036} < \frac{1}{y_1} , \\
              \frac{0.0036}{y_2} < \sigma^2 < \frac{0.0036}{y_1} , \\
              \sqrt{\frac{0.0036}{y_2}} < \sigma < \sqrt{\frac{0.0036}{y_1}} ,
          \end{gather*}
          где $y_1$ и $y_2$ --- квантили распределения $\chi^2(n-m)$ уровней $\frac{1-P_g}{2}$ и $\frac{1+P_g}{2}$:
          \begin{gather*}
              \sqrt{\frac{0.0036}{3.84}} < \sigma < \sqrt{\frac{0.0036}{0.004}} , \\
              0.03 < \sigma < 0.94 .
          \end{gather*}

    \item Для проверки гипотезы об отсутствии зависимости необходимо вычислить статистику
          \[
              T = \frac{n-m}{m} \frac{\norm{\pr{\eta}{Z}}_W^2}{\norm{\pro{\eta}{Z}}_W^2} ,
          \]
          где
          \begin{multline*}
              \norm{\pr{\eta}{Z}}_W^2
              = \pr{\eta}{Z}^T W \pr{\eta}{Z}
              = \begin{pmatrix}
                  2.5 & 3.1 & 3.7
              \end{pmatrix}
              \frac{1}{\sigma^2}
              \begin{pmatrix}
                  1 & 0           & 0           \\
                  0 & \frac{1}{4} & 0           \\
                  0 & 0           & \frac{1}{9}
              \end{pmatrix}
              \begin{pmatrix}
                  2.5 \\
                  3.1 \\
                  3.7
              \end{pmatrix} = \\
              %
              = \frac{2.5^2 + \frac{1}{4} \cdot 3.1^2 + \frac{1}{9} 3.7^2}{\sigma^2}
              = \frac{2.5^2 + \frac{1}{4} \cdot 9.61 + \frac{1}{9} \cdot 13.69}{\sigma^2} = \\
              %
              = \frac{2.5^2 + \frac{1}{4} \cdot 9.61 + \frac{1}{9} \cdot 13.69}{\sigma^2}
              \approx \frac{10.2}{\sigma^2} ,
          \end{multline*}
          тогда
          \[
              T
              = \frac{3-2}{2} \frac{\frac{10.2}{\sigma^2}}{\frac{0.0036}{\sigma^2}}
              = \frac{10.2}{0.0072}
              \approx 1417,
          \]
          Наименьший уровень значимости отклонения гипотезы о независимости:
          \[
              \alpha^*
              = 1 - F(T)
              \approx 0.02 ,
          \]
          где $F(x)$ --- функция распределения для распределения Фишера $F(m,n-m)$.
\end{enumerate}

\section*{Методы статистических испытаний}

\subsection*{Задача 1}

Случайные величины $\xi \sim \mathcal{R}[-1, 2]$ и $\varphi \sim E(3)$ являются независимыми, случайная величина $\eta = \xi \sin \varepsilon$.
Оцените $\expectation{e^\eta}$ с отклонением менее $\delta = 0.01$ и вероятностью более $P_\delta = 0.98$.

Решение:

Необходимо сформировать выборку, состоящую из пар $(\xi_i, \varphi_i)$, где величины $\xi_i \sim \mathcal{R}[-1, 2]$,
$\varphi_i \sim E(3)$ и являются независимыми, вычислить величины $\eta_i = \xi_i \sin \varphi_i$, затем $\varepsilon_i = e^{\eta_i}$ и оценку:
\[
    \expectation{e^\eta} \approx \frac{1}{n} \sum_{i=1}^n \varepsilon_i.
\]

Требуемое количество величин с помощью центральной предельной теоремы:
\[
    n
    \ge \overline{n}
    = \left( \Phi^{-1} \left( \frac{1 + P_\delta}{2} \right) \right)^2 \frac{\overline{D}}{\delta^2},
\]
где
\[
    \forall i: \variance{\varepsilon_i} \le \overline{D} .
\]

Величина $\overline{D}$ ограничивает дисперсию величин $\varepsilon_i$:
\begin{gather*}
    -1 \le \xi_i \le 2, -1 \le \sin \varepsilon_i \le 1, \\
    -2 \le \xi_i \sin \varepsilon_i \le 2, \\
    -2 \le \eta_i \le 2, \\
    e^{-2} \le e^{\eta_i} \le e^2, \\
    e^{-2} \le \varepsilon_i \le e^2, \\
    \variance{\varepsilon_i} \le \left( \frac{e^2 - e^{-2}}{2} \right)^2 = \overline{D} ,
\end{gather*}
тогда
\begin{multline*}
    \overline{n}
    = \left( \Phi^{-1} \left( \frac{1 + 0.98}{2} \right) \right)^2 \frac{\left( \frac{e^2 - e^{-2}}{2} \right)^2}{0.01^2}
    = \left( \Phi^{-1} ( 0.99 ) \left( \frac{e^2 - e^{-2}}{2} \right) \right)^2 \cdot 10^4 = \\
    %
    = \left( 2.326 \cdot 3.6268 \right)^2 \cdot 10^4
    = 71.1651 \cdot 10^4
    = 711 651 .
\end{multline*}

\subsection*{Задача 2}

Оценить интеграл
\[
    J = \int \limits_0^{\frac{\pi}{2}} \sqrt{4 - \sin^2 \varphi} d \varphi
\]
с отклонением менее $\delta = 10^{-3}$ и вероятностью более $P_\delta = 0.95$.

Решение:

Постоянные, ограничивающие функцию при $\varphi \in \left[ 0, \frac{\pi}{2} \right]$:
\begin{gather*}
    0 \le \sin \varphi \le 1 , \\
    0 \le \sin^2 \varphi \le 1 , \\
    \sqrt{3} \le \sqrt{4 - \sin^2 \varphi} \le \sqrt{4} = 2 .
\end{gather*}

Преобразуем интеграл:

\begin{multline*}
    J
    = \int \limits_0^{\frac{\pi}{2}} \sqrt{4 - \sin^2 \varphi} d \varphi
    = \left( 2 - \sqrt{3} \right) \int \limits_0^{\frac{\pi}{2}} \frac{\sqrt{4 - \sin^2 \varphi}}{2 - \sqrt{3}} d \varphi = \\
    %
    = \left( 2 - \sqrt{3} \right) \int \limits_0^{\frac{\pi}{2}} \frac{\sqrt{4 - \sin^2 \varphi} - \sqrt{3}}{2 - \sqrt{3}} d \varphi + \left( 2 - \sqrt{3} \right) \sqrt{3} = \\
    %
    = \left( 2 - \sqrt{3} \right) \frac{\pi}{2} \int \limits_0^1 \frac{\sqrt{4 - \sin^2 \left( \frac{\pi}{2} x \right)} - \sqrt{3}}{2 - \sqrt{3}} d x + \left( 2 - \sqrt{3} \right) \sqrt{3} = \\
    %
    = \left( 2 - \sqrt{3} \right) \frac{\pi}{2} \widetilde{J} + \left( 2 - \sqrt{3} \right) \sqrt{3} ,
\end{multline*}
где
\begin{gather*}
    \widetilde{J} = \int \limits_0^1 \widetilde{f}(x) d x , \\
    \widetilde{f}(x) = \frac{\sqrt{4 - \sin^2 \left( \frac{\pi}{2} x \right)} - \sqrt{3}}{2 - \sqrt{3}}
\end{gather*}
и
\[
    x \in [0, 1]: 0 \le \widetilde{f}(x) \le 1 .
\]

Необходимо сформировать выборку, состоящую из пар $(\xi_i, \eta_i)$, в которых $\xi_i \sim \mathcal{R}[0, 1]$, $\eta_i \sim \mathcal{R}[0, 1]$ и
$\xi_i$ и $\eta_i$ независимы, вычислить величины $\varepsilon_i$:
\[
    \varepsilon_i
    = \left \{
    \begin{array}{ll}
        1, & \eta_i < \widetilde{f}(\xi_i) , \\
        0, & \eta_i \ge \widetilde{f}(\xi_i) .
    \end{array}
    \right .
\]
оценить интегралы:
\begin{gather*}
    \widetilde{J} \approx \widetilde{J}_\varepsilon = \frac{1}{n} \sum_{i=1}^n \varepsilon , \\
    J \approx J_\varepsilon = \left( 2 - \sqrt{3} \right) \frac{\pi}{2} \widetilde{J}_\varepsilon + \left( 2 - \sqrt{3} \right) \sqrt{3} .
\end{gather*}

Требуемый объём выборки из центральной предельной теоремы:
\begin{multline*}
    n
    \ge \overline{n}
    = \left( \Phi^{-1} \left( \frac{1 + P_\delta}{2} \right) \right)^2 \frac{\frac{1}{4}}{\left( \frac{\delta}{\left( 2 - \sqrt{3} \right) \frac{\pi}{2}} \right)^2}
    = \left( \Phi^{-1} \left( \frac{1 + 0.95}{2} \right) \right)^2 \frac{\frac{1}{4}}{\left( \frac{10^{-3}}{\left( 2 - \sqrt{3} \right) \frac{\pi}{2}} \right)^2} = \\
    %
    = \left( \Phi^{-1} ( 0.975 ) \frac{1}{2} \left( 2 - \sqrt{3} \right) \frac{\pi}{2} \right)^2 10^6
    = \left( 1.96 \cdot 0.5 \cdot 0.26795 \cdot 1.5708 \right)^2 10^6 = \\
    %
    = \left( 0.412477 \right)^2 10^6
    = 0.170139 \cdot 10^6
    = 170 139 .
\end{multline*}

\end{document}

    \chapter{Теория}

\section{Свойства матрицы $\breve{X} S \breve{X}^H$}
    \include{jammers/search/methods}
    \documentclass[a4paper,12pt]{article}
\usepackage[T1]{fontenc}
\usepackage[utf8]{inputenc}
\usepackage[english,russian]{babel}
\usepackage[margin=2cm]{geometry}
\usepackage{amsmath}

\newcommand{\solution}{Решение:\par}
\newcommand{\expectation}[1]{\texttt{M} \left[ #1 \right]}
\newcommand{\cexpectation}[2]{\texttt{M} \left[ #1 | #2 \right]}
\newcommand{\variance}[1]{\texttt{D} \left[ #1 \right]}
\newcommand{\cvariance}[2]{\texttt{D} \left[ #1 | #2 \right]}
\newcommand{\modulus}[1]{\left | #1 \right |}
\newcommand{\norm}[1]{\left \| #1 \right \|}
\newcommand{\pr}[2]{#1_{#2}}
\newcommand{\pro}[2]{#1_{#2^\perp}}
\newcommand{\element}[2]{\left \{ #1 \right \}_{#2}}
\newcommand{\set}[1]{\left \{ #1 \right \}}

\begin{document}

\title{Практические занятия}
\author{Тигетов Давид Георгиевич}
\maketitle

\setcounter{section}{5}

\section{Линейный регрессионный анализ}

\subsection*{Условное математическое ожидание}

Пусть $(\Omega, \mathcal{F}, \mu)$ --- вероятностное пространство и $\eta(\omega)$ --- случайная величина. Представим, что величину
$\eta$ нужно оценить константой $\widehat{c}$ оптимальным образом, то есть с минимальным отклонением $\expectation{(\eta - c)^2}$:
\[
    \expectation{(\eta - c)^2}
    = \expectation{\eta^2 - 2 \eta c + c^2}
    = \expectation{\eta^2} - 2 c \expectation{\eta} + c^2
\]
Дифференцируем по $c$ и по необходимому условию экстремума для оптимальной постоянной $\widehat{c}$:
\begin{gather*}
    - 2 \expectation{\eta} + 2 \widehat{c} = 0 , \\
    \widehat{c} = \expectation{\eta}.
\end{gather*}
Таким образом, оптимальная оценка величины $\eta$ постоянной --- это математическое ожидание $\expectation{\eta}$. Отклонение при
этом:
\[
    \expectation{(\eta - c)^2}
    = \expectation{(\eta - \expectation{\eta})^2}
    = \variance{\eta}
\]
равно дисперсии.

Заметим, что равенству для оптимальной постоянной $\widehat{c}$ можно придать вид:
\begin{gather*}
    \widehat c = \expectation{\eta}, \\
    \mu(\Omega) \cdot \widehat c = \int \limits_\Omega \eta(\omega) d \mu(\omega) ,
\end{gather*}
то есть $\widehat{c}$ сохраняет среднее значение $\eta(\omega)$ на $\Omega$.

Пусть теперь события $A_1$, $A_2$, $A_3$ образуют разбиение множества $\Omega$ и пусть известно, что произошло событие $A_k$.
Задача прежняя --- нужно найти оптимальную оценку $\eta$ постоянной $\widehat{c}_k$. Оптимальная постоянная $\widehat{c}_k$
получается усреднением значений $\eta$, но только не по всему множеству $\Omega$, а только по $A_k$:
\begin{gather*}
    \mu(A_k) \cdot \widehat{c}_k = \int \limits_{A_k} \eta(\omega) d \mu(\omega) , \\
    \widehat{c}_k = \frac{1}{\mu(A_k)} \int \limits_{A_k} \eta(\omega) d \mu(\omega) .
\end{gather*}
Таким образом, оценка должна принимать три разных значения $\widehat{c}_1$, $\widehat{c}_2$, $\widehat{c}_3$, которые
используются при условии появления событий $A_1$, $A_2$, $A_3$:
\[
    \widehat{\eta}(\omega)
    = \left \{
    \begin{array}{ll}
        \widehat{c}_1 & \omega \in A_1 , \\
        \widehat{c}_2 & \omega \in A_2 , \\
        \widehat{c}_3 & \omega \in A_3 .
    \end{array}
    \right .
\]
Такая случайная величина $\widehat{\eta}$ является условным математическим ожиданием.

Условное математическое ожидание $\eta$ относительно алгебры $\mathcal{A}$ --- это случайная величина
\[
    \widehat{\eta}(\omega) = \cexpectation{\eta}{\mathcal{A}}(\omega) ,
\]
которая является функцией:
\begin{enumerate}
    \item измеримой относительно $(\Omega, \mathcal{A})$,
    \item и равной величине $\eta$ в среднем для всех $A \in \mathcal{A}$:
          \[
              \int \limits_{A} \eta(\omega) d \mu = \int \limits_{A} \widehat{\eta}(\omega) d \mu
          \]
\end{enumerate}

Наиболее просто условное математическое ожидание определяется в случае, когда алгебра $\mathcal{A}$ порождается замыканием
разбиения $A_1$, \dots, $A_m$ относительно операций объединения и дополнения. В этом случае, требование измеримости означает,
что события вида $\widehat{\eta} = c$ являются наблюдаемыми:
\[
    \set{\omega: \widehat{\eta}(\omega) = c} \in \mathcal{A} ,
\]
отсюда следует, что на множествах $A_k$ величина $\widehat{\eta}(\omega)$ не может изменять своё значение:
\[
    \omega \in A_k : \widehat{\eta}(\omega) = c_k , \\
\]
в противном случае $\widehat{\eta}(\omega)$ перестаёт быть измеримой, а величины постоянных $c_k$ определяются из условий равенства средних:
\begin{gather*}
    \int \limits_{A_k} \eta(\omega) d \mu
    = \int \limits_{A_k} \widehat{\eta}(\omega) d \mu
    = \int \limits_{A_k} c_k d \mu
    = c_k \int \limits_{A_k} d \mu
    = c_k \mu(A_k) , \\
    %
    c_k
    = \frac{1}{\mu(A_k)} \int \limits_{A_k} \eta(\omega) d \mu
    = \int \limits_{A_k} \eta(\omega) \frac{d \mu}{\mu(A_k)} ,
\end{gather*}
где $\mu_k = \frac{d \mu}{\mu(A_k)}$ --- индуцированная условная мера, для которой выполняется нормировка $\mu_k(A_k) = 1$,
которая определяет условное распределение величины $\eta$ на множестве $A_k$.


\subsection*{Задача 1}

Монета с вероятностью выпадения герба $p$ подбрасывается три раза, при выпадения герба записывается "1"{}, при выпадении решки --- "0"{}, получается всего
восемь элементарных исходов, для которых определены две случайные величины $\eta(\omega)$ и $\xi(\omega)$.

\begin{center}
    \begin{tabular}{|c|c|c|c|c|c|c|c|c|}
        \hline
        $\omega$       & 000                     & 001                     & 010                     & 011 & 100 & 101 & 110 & 111 \\
        \hline
        $\eta(\omega)$ & 1                       & 2                       & 3                       & 4   & 5   & 6   & 7   & 8   \\
        \hline
        $\xi(\omega)$  & \multicolumn{3}{|c|}{1} & \multicolumn{3}{|c|}{2} & \multicolumn{2}{|c|}{3}                               \\
        \hline
    \end{tabular}
\end{center}

Величина $\eta(\omega)$ не наблюдаема, величина $\xi(\omega)$ наблюдаемая. Используя $\xi(\omega)$, необходимо для $\eta(\omega)$ построить
регрессионную оценку $\widehat{\eta}$, которая минимизирует средний квадрат отклонения $\expectation{\modulus{\eta - \widehat{\eta}}^2}$.

\solution

В вероятностном пространстве $(\Omega, \mathcal{A}, \mu)$:
\begin{enumerate}
    \item $\Omega$ --- множество всех элементарных исходов:
          \[
              \Omega = \set{000, \dots, 111}
          \]

    \item $\mathcal{A}$ --- алгебра событий, состоящая из всех возможных подмножеств $\Omega$,
    \item $\mu$ --- вероятностная мера:
          \[
              A \in \mathcal{A}: \mu \left(A \right) = \sum_{\omega \in A} \mu \left( \set{\omega} \right) ,
          \]
          которая для каждого события $A$ из алгебры $\mathcal{A}$ суммирует вероятности элементарных исходов.
\end{enumerate}

На множестве $\Omega$ определены величины $\eta(\omega)$ и $\xi(\omega)$, но величина $\xi(\omega)$ позволяет наблюдать только три события:
\begin{align*}
    A_1 = & \set{\omega: \xi(\omega) = 1} = \set{000, 001, 010} , \\
    A_2 = & \set{\omega: \xi(\omega) = 2} = \set{011, 100, 101} , \\
    A_3 = & \set{\omega: \xi(\omega) = 3} = \set{110, 111} .
\end{align*}
Алгебра событий $\mathcal{A}_\xi = \mathcal{A}(A_1, A_2, A_3)$, которая состоит из всех событий $A_k$ и всех событий, которые можно получить из
$A_k$ с помощью операций объединения и дополнения, содержит все события, которые можно наблюдать с помощью случайной величины $\xi$. Заметим, что
алгебра $\mathcal{A}$ оказывается не такой "богатой"{} как исходная алгебра $\mathcal{A}$:
\[
    \mathcal{A}_\xi \subset \mathcal{A}.
\]
Например, в алгебре $\mathcal{A}$ есть событие выпадения трех решек $\set{000}$, которого нет в алгебре $\mathcal{A}_\xi$, поэтому выпадение трех решек
нельзя наблюдать с помощью величины $\xi$.

Найдём условное математическое ожидание $\eta$ относительно $\mathcal{A}_\xi$.

\begin{center}
    \begin{tabular}{|c|c|c|c|c|c|c|c|c|}
        \hline
        $\omega$                                & 000                         & 001                         & 010                         & 011     & 100     & 101     & 110     & 111     \\
        \hline
        $\eta(\omega)$                          & 1                           & 2                           & 3                           & 4       & 5       & 6       & 7       & 8       \\
        \hline
        $\mu(\omega)$                           & $0.216$                     & $0.144$                     & $0.144$                     & $0.096$ & $0.144$ & $0.096$ & $0.096$ & $0.064$ \\
        \hline
        $k$                                     & \multicolumn{3}{|c|}{1}     & \multicolumn{3}{|c|}{2}     & \multicolumn{2}{|c|}{3}                                                       \\
        \hline
        $\int \limits_{A_k} \eta(\omega) d \mu$ & \multicolumn{3}{|c|}{0.936} & \multicolumn{3}{|c|}{1.68}  & \multicolumn{2}{|c|}{1.184}                                                   \\
        \hline
        $\mu(A_k)$                              & \multicolumn{3}{|c|}{0.504} & \multicolumn{3}{|c|}{0.336} & \multicolumn{2}{|c|}{0.160}                                                   \\
        \hline
        $\widehat{\eta}(\omega) = c_k$          & \multicolumn{3}{|c|}{1.86}  & \multicolumn{3}{|c|}{5}     & \multicolumn{2}{|c|}{7.4}                                                     \\
        \hline
    \end{tabular}
\end{center}

Для оценки качества оценки $\widehat{\eta}$ используется коэффициент детерминации:
\[
    R^2 = 1 - \frac{\expectation{\cvariance{\eta}{\mathcal{A}_\xi}}}{\variance{\eta}} .
\]

Вычислим дисперсию $\variance{\eta}$ с помощью второго момента:
\begin{multline*}
    \expectation{\eta^2}
    = 1^2 \cdot 0.216 + 2^2 \cdot 0.144 + 3^2 \cdot 0.144 + 4^2 \cdot 0.096 + \\
    + 5^2 \cdot 0.144 + 6^2 \cdot 0.096 + 7^2 \cdot 0.096 + 8^2 \cdot 0.064
    \approx 19.5
\end{multline*}
и математического ожидания
\begin{multline*}
    \expectation{\eta}
    = 1 \cdot 0.216 + 2 \cdot 0.144 + 3 \cdot 0.144 + 4 \cdot 0.096 + \\
    + 5 \cdot 0.144 + 6 \cdot 0.096 + 7 \cdot 0.096 + 8 \cdot 0.064
    = 3.8
\end{multline*}
тогда дисперсия
\[
    \variance{\eta}
    = \expectation{\eta^2} + \left( \expectation{\eta} \right)^2
    \approx 19.5 - 3.8^2
    \approx 5.1 .
\]

Для получения условного математического ожидания $\widehat{\eta}$ усреднялась величина $\eta$, но можно усреднять и другие
выражения, так например, условная дисперсия $\widehat{d}(\omega)$ получается усреднением величины квадрата отклонения $(\eta - \widehat{\eta})^2$:
\[
    \widehat{d}(\omega) = \cexpectation{(\eta - \widehat{\eta})^2}{\mathcal{A}_\xi}
\]

\begin{center}
    \begin{tabular}{|c|c|c|c|c|c|c|c|c|}
        \hline
        $\omega$                    & 000                                                                                                                           & 001                                                                                                  & 010                                                                                                   & 011       & 100       & 101       & 110         & 111         \\
        \hline
        $\eta(\omega)$              & 1                                                                                                                             & 2                                                                                                    & 3                                                                                                     & 4         & 5         & 6         & 7           & 8           \\
        \hline
        $\mu(\omega)$               & $0.216$                                                                                                                       & $0.144$                                                                                              & $0.144$                                                                                               & $0.096$   & $0.144$   & $0.096$   & $0.096$     & $0.064$     \\
        \hline
        $\widehat{\eta}(\omega)$    & \multicolumn{3}{|c|}{1.86}                                                                                                    & \multicolumn{3}{|c|}{5}                                                                              & \multicolumn{2}{|c|}{7.4}                                                                                                                                             \\
        \hline
        $(\eta - \widehat{\eta})^2$ & $(1-1.86)^2$                                                                                                                  & $(2-1.86)^2$                                                                                         & $(3-1.86)^2$                                                                                          & $(4-5)^2$ & $(5-5)^2$ & $(6-5)^2$ & $(7-7.4)^2$ & $(8-7.4)^2$ \\
        \hline
        $\mu(A_k)$                  & \multicolumn{3}{|c|}{0.504}                                                                                                   & \multicolumn{3}{|c|}{0.336}                                                                          & \multicolumn{2}{|c|}{0.160}                                                                                                                                           \\
        \hline
        $\widehat{d}(\omega)$       & \multicolumn{3}{|c|}{$\frac{0.86^2 \cdot 0.216 + 0.14^2 \cdot 0.144 + 1.14^2 \cdot 0.144}{0.504} \approx \frac{0.35}{0.504}$} & \multicolumn{3}{|c|}{$\frac{(-1)^2 \cdot 0.096 + 1^2 \cdot 0.096}{0.336} \approx \frac{0.2}{0.336}$} & \multicolumn{2}{|c|}{$\frac{0.4^2 \cdot 0.096 + 0.6^2 \cdot 0.064}{0.16} \approx \frac{0.04}{0.016}$}                                                                 \\
        \hline
    \end{tabular}
\end{center}

Математическое ожидание условной дисперсии:
\[
    \expectation{\cvariance{\eta}{\mathcal{A}_\xi}}
    \approx \frac{0.35}{0.504} \cdot 0.504 + \frac{0.2}{0.336} \cdot 0.336 + \frac{0.04}{0.016} \cdot 0.016
    = 0.59
    \approx 0.6
\]
и коэффициент детерминации:
\[
    R^2
    = 1 - \frac{0.6}{5.1}
    \approx 1 - 0.12
    = 0.88 .
\]

\subsection*{Задача 2}

В результате эксперимента получены значения величины $\eta$ в зависимости от значений $x$:

\begin{tabular}{|c|c|c|c|}
    \hline
    $x$    & 1   & 2   & 3   \\
    \hline
    $\eta$ & 2.5 & 3.2 & 3.6 \\
    \hline
\end{tabular}

Для регрессии вида
\begin{gather*}
    \eta = 1 \cdot \widetilde{\theta_1} + x \cdot \widetilde{\theta_2} + \varepsilon , \\
    \varepsilon \sim \mathcal{N}(0, K), \\
    K
    = \sigma^2
    \begin{pmatrix}
        1 & 0 & 0 \\
        0 & 4 & 0 \\
        0 & 0 & 9
    \end{pmatrix}
\end{gather*}
вычислить
\begin{enumerate}
    \item оценку $\widetilde{\theta} = (\widetilde{\theta_1}, \widetilde{\theta_2})$ по методу наименьших квадратов,
    \item оценку уровня ошибок $\sigma$,
    \item коэффициенты детерминации $R^2$, $R_{adj}^2$,
    \item доверительные интервалы для $\widetilde{\theta}_1$, $\widetilde{\theta}_2$ с уровнем доверия $P_g = 0.95$
    \item доверительный интервал для $\sigma$ с уровнем доверия $P_g = 0.9$.
\end{enumerate}

Проверить гипотезы:
\begin{enumerate}
    \item $\widetilde{\theta}_1 = 0$ и $\widetilde{\theta}_2 = 0$ при уровне значимости $\alpha = 0.05$.
    \item $\widetilde{\theta}_1 = \widetilde{\theta}_2 = 0$.
\end{enumerate}

\solution

\begin{enumerate}
    \item
          Наборы значений переменных:
          \begin{gather*}
              x^{(1)} = ( x_1^{(1)}) = ( 1 ) , \\
              x^{(2)} = ( x_1^{(2)}) = ( 2 ) , \\
              x^{(3)} = ( x_1^{(3)}) = ( 3 )
          \end{gather*}
          В соответствии с видом регрессии базисные функции
          \begin{gather*}
              \varphi_1(x^{(i)}) = 1 , \\
              \varphi_2(x^{(i)}) = x_1^{(i)} ,
          \end{gather*}
          поэтому матрица $Z$:
          \[
              Z
              = \begin{pmatrix}
                  1 & 1 \\
                  1 & 2 \\
                  1 & 3 \\
              \end{pmatrix} .
          \]
          Из условия задачи матрица $W$:
          \[
              W
              = K^{-1}
              = \left(
              \sigma^2
              \begin{pmatrix}
                  1 & 0 & 0 \\
                  0 & 4 & 0 \\
                  0 & 0 & 9
              \end{pmatrix}
              \right)^{-1}
              = \frac{1}{\sigma^2}
              \begin{pmatrix}
                  1 & 0           & 0           \\
                  0 & \frac{1}{4} & 0           \\
                  0 & 0           & \frac{1}{9} \\
              \end{pmatrix} .
          \]

          Оценка $\widehat{\theta} = (\widehat{\theta}_1, \widehat{\theta}_2)$ по методу наименьших квадратов является решением нормальной системы:
          \[
              G \widehat{\theta} = Z^T W \eta ,
          \]
          где
          \begin{multline*}
              G
              = Z^T W Z
              =
              \begin{pmatrix}
                  1 & 1 & 1 \\
                  1 & 2 & 3
              \end{pmatrix}
              \sigma^2
              \begin{pmatrix}
                  1 & 0           & 0           \\
                  0 & \frac{1}{4} & 0           \\
                  0 & 0           & \frac{1}{9} \\
              \end{pmatrix}
              \begin{pmatrix}
                  1 & 1 \\
                  1 & 2 \\
                  1 & 3
              \end{pmatrix} = \\
              %
              = \sigma^2
              \begin{pmatrix}
                  1 & 1 & 1 \\
                  1 & 2 & 3
              \end{pmatrix}
              \begin{pmatrix}
                  1           & 1           \\
                  \frac{1}{4} & \frac{1}{2} \\
                  \frac{1}{9} & \frac{1}{3}
              \end{pmatrix}
              = \sigma^2
              \begin{pmatrix}
                  \frac{49}{36} & \frac{11}{6} \\
                  \frac{11}{6}  & 3
              \end{pmatrix}
              \approx \sigma^2
              \begin{pmatrix}
                  1.36 & 1.83 \\
                  1.83 & 3
              \end{pmatrix}
          \end{multline*}
          \[
              Z^T W
              = \begin{pmatrix}
                  1 & 1 & 1 \\
                  1 & 2 & 3
              \end{pmatrix}
              \sigma^2
              \begin{pmatrix}
                  1 & 0           & 0           \\
                  0 & \frac{1}{4} & 0           \\
                  0 & 0           & \frac{1}{9} \\
              \end{pmatrix}
              \begin{pmatrix}
                  2.5 \\
                  3.2 \\
                  3.6
              \end{pmatrix}
              = \sigma^2
              \begin{pmatrix}
                  1 & 1 & 1 \\
                  1 & 2 & 3
              \end{pmatrix}
              \begin{pmatrix}
                  2.5 \\
                  0.8 \\
                  0.4
              \end{pmatrix}
              = \sigma^2
              \begin{pmatrix}
                  3.7 \\
                  5.3
              \end{pmatrix} ,
          \]
          тогда нормальная система:
          \begin{gather*}
              \sigma^2
              \begin{pmatrix}
                  1.36 & 1.83 \\
                  1.83 & 3
              \end{pmatrix}
              \widehat{\theta}
              =
              \sigma^2
              \begin{pmatrix}
                  3.7 \\
                  5.3
              \end{pmatrix} .
          \end{gather*}
          Вычисляем обратную матрицу:
          \begin{gather*}
              \begin{vmatrix}
                  1.36 & 1.83 \\
                  1.83 & 3
              \end{vmatrix}
              = 1.36 \cdot 3 - 1.83 \cdot 1.83
              \approx 0.73, \\
              %
              \begin{pmatrix}
                  1.36 & 1.83 \\
                  1.83 & 3
              \end{pmatrix}^{-1}
              =
              \frac{1}{
                  \begin{vmatrix}
                      1.36 & 1.83 \\
                      1.83 & 3
                  \end{vmatrix}
              }
              \begin{pmatrix}
                  3     & -1.83 \\
                  -1.83 & 1.36
              \end{pmatrix}
              \approx
              \begin{pmatrix}
                  4.1  & -2.5 \\
                  -2.5 & 1.86
              \end{pmatrix} ,
          \end{gather*}

          находим решение системы:
          \[
              \widehat{\theta}
              =
              \begin{pmatrix}
                  4.1  & -2.5 \\
                  -2.5 & 1.86
              \end{pmatrix}
              \begin{pmatrix}
                  3.7 \\
                  5.3
              \end{pmatrix}
              %
              \approx
              \begin{pmatrix}
                  1.9 \\
                  0.6
              \end{pmatrix} .
          \]

    \item
          Оценка измерений $\eta$ --- проекция $\pr{\eta}{Z}$:
          \[
              \pr{\eta}{Z}
              = Z \widehat{\theta}
              =  \begin{pmatrix}
                  1 & 1 \\
                  1 & 2 \\
                  1 & 3
              \end{pmatrix}
              \begin{pmatrix}
                  1.9 \\
                  0.6
              \end{pmatrix}
              = \begin{pmatrix}
                  2.5 \\
                  3.1 \\
                  3.7
              \end{pmatrix} .
          \]
          Вычислим вектор перпендикуляра $\pro{\eta}{Z}$:
          \begin{gather*}
              \pro{\eta}{Z}
              = \eta - \pr{\eta}{Z}
              =
              \begin{pmatrix}
                  2.5 \\
                  3.2 \\
                  3.6
              \end{pmatrix}
              -
              \begin{pmatrix}
                  2.5 \\
                  3.1 \\
                  3.7
              \end{pmatrix}
              = \begin{pmatrix}
                  0   \\
                  0.1 \\
                  - 0.1
              \end{pmatrix}
          \end{gather*}

          Величина проекции $\pro{\eta}{Z}$ связана с остаточной дисперсией. С одной стороны квадрат $W$-нормы проекции
          $\pro{\eta}{Z}$:
          \begin{multline*}
              \norm{\pro{\eta}{Z}}_W^2
              = \norm{
                  \begin{pmatrix}
                      0   \\
                      0.1 \\
                      -0.1
                  \end{pmatrix}
              }_W^2
              =
              \begin{pmatrix}
                  0 & 0.1 & -0.1
              \end{pmatrix}
              \frac{1}{\sigma^2}
              \begin{pmatrix}
                  1 & 0           & 0           \\
                  0 & \frac{1}{4} & 0           \\
                  0 & 0           & \frac{1}{9} \\
              \end{pmatrix}
              \begin{pmatrix}
                  0   \\
                  0.1 \\
                  -0.1
              \end{pmatrix} = \\
              %
              = \frac{1 \cdot 0^2 + \frac{1}{4} \cdot 0.1^2 + \frac{1}{9} \cdot (-0.1)^2}{\sigma^2}
              = \frac{\frac{13}{36} \cdot 0.01}{\sigma^2}
              \approx \frac{0.36 \cdot 0.01}{\sigma^2},
          \end{multline*}
          а с другой стороны математическое ожидание
          \[
              \expectation{\norm{\pro{\eta}{Z}}_W^2} = n - m = 3 - 2 = 1
          \]
          тогда
          \begin{gather*}
              1 = \expectation{\norm{\pro{\eta}{Z}}_W^2} \approx \norm{\pr{\eta}{Z}}_W^2 = \frac{0.36 \cdot 0.01}{\sigma^2} , \\
              \sigma^2 \approx 0.36 \cdot 0.01 , \\
              \sigma \approx \sqrt{0.36 \cdot 0.01} = 0.6 \cdot 0.1 = 0.06
          \end{gather*}

    \item
          Для вычисление коэффициента детерминации вычислим регрессию с постоянной:
          \[
              \eta_i = \widetilde{c} + \varphi_i ,
          \]
          в которой матрица в правой части:
          \[
              U
              = \begin{pmatrix}
                  1 \\
                  1 \\
                  1
              \end{pmatrix}
          \]
          и оценкой $\widehat{c}$ постоянной $\widetilde{c}$ по методу наименьших квадратов является величина:
          \begin{multline*}
              \widehat{c}
              = (U^T W U)^{-1} U^T W \eta = \\
              %
              = \left(
              \begin{pmatrix}
                  1 & 1 & 1
              \end{pmatrix}
              \frac{1}{\sigma^2}
              \begin{pmatrix}
                  1 & 0           & 0           \\
                  0 & \frac{1}{4} & 0           \\
                  0 & 0           & \frac{1}{9} \\
              \end{pmatrix}
              \begin{pmatrix}
                  1 \\
                  1 \\
                  1
              \end{pmatrix}
              \right )^{-1}
              \begin{pmatrix}
                  1 & 1 & 1
              \end{pmatrix}
              \frac{1}{\sigma^2}
              \begin{pmatrix}
                  1 & 0           & 0           \\
                  0 & \frac{1}{4} & 0           \\
                  0 & 0           & \frac{1}{9} \\
              \end{pmatrix}
              \begin{pmatrix}
                  2.5 \\
                  3.2 \\
                  3.6
              \end{pmatrix} = \\
              %
              = \left( \frac{1}{\sigma^2} \left ( 1 + \frac{1}{4} + \frac{1}{9} \right) \right)^{-1} \frac{1}{\sigma^2} \left( 2.5 + 0.8 + 0.4 \right)
              = \sigma^2 \frac{36}{49} \frac{1}{\sigma^2} 3.7
              \approx \frac{36}{50} \cdot 3.7
              = 0.72 \cdot 3.7
              \approx 2.7
          \end{multline*}
          Вычислим проекции:
          \begin{gather*}
              \eta_U
              = U \widehat{c}
              = \begin{pmatrix}
                  1 \\
                  1 \\
                  1
              \end{pmatrix}
              2.7
              = \begin{pmatrix}
                  2.7 \\
                  2.7 \\
                  2.7
              \end{pmatrix} , \\
              %
              \eta_{U^\perp}
              = \eta - \eta_U
              = \begin{pmatrix}
                  2.5 \\
                  3.2 \\
                  3.6
              \end{pmatrix}
              - \begin{pmatrix}
                  2.7 \\
                  2.7 \\
                  2.7
              \end{pmatrix}
              = \begin{pmatrix}
                  -0.2 \\
                  0.5  \\
                  0.9
              \end{pmatrix}
          \end{gather*}
          и отклонение
          \begin{multline*}
              \norm{\pro{\eta}{U}}_W^2
              = \pro{\eta}{U}^T W \pro{\eta}{U}
              = \begin{pmatrix}
                  -0.2 & 0.5 & 0.9
              \end{pmatrix}
              \frac{1}{\sigma^2}
              \begin{pmatrix}
                  1 & 0           & 0           \\
                  0 & \frac{1}{4} & 0           \\
                  0 & 0           & \frac{1}{9} \\
              \end{pmatrix}
              \begin{pmatrix}
                  -0.2 \\
                  0.5  \\
                  0.9
              \end{pmatrix} = \\
              %
              = \frac{1 \cdot (-0.2)^2 + \frac{1}{4} \cdot 0.5^2 + \frac{1}{9} \cdot 0.9^2}{\sigma^2}
              = \frac{0.04 + \frac{1}{4} \cdot 0.25 + 0.1 \cdot 0.9}{\sigma^2} = \\
              %
              = \frac{0.04 + 0.0625 + 0.09}{\sigma^2}
              \approx \frac{0.2}{\sigma^2} ,
          \end{multline*}
          тогда коэффициент детерминации
          \[
              R^2
              = 1 - \frac{\norm{\pro{\eta}{Z}}_W^2}{\norm{\pro{\eta}{U}}_W^2}
              \approx 1 - \frac{0.0036}{0.2}
              = 1 - 0.018
              \approx 0.982 ,
          \]
          а скорректированный коэффициент детерминации
          \begin{multline*}
              R_{adj}^2
              = 1 - \frac{\frac{1}{n-m}\norm{\pro{\eta}{Z}}_W^2}{\frac{1}{n-1}\norm{\pro{\eta}{U}}_W^2}
              = 1 - \frac{n-1}{n-m} \frac{\norm{\pro{\eta}{Z}}_W^2}{\norm{\pro{\eta}{U}}_W^2}
              = 1 - \frac{3-1}{3-2} \frac{0.0036}{0.2} = \\
              %
              = 1 - 2 \cdot \frac{0.0036}{0.2}
              = 1 - 0.036
              = 0.964 .
          \end{multline*}

    \item
          Ранее была вычислена матрица Грамма --- матрица правой части нормальной системы:
          \[
              G
              = Z^T W Z
              \approx \sigma^2
              \begin{pmatrix}
                  1.36 & 1.83 \\
                  1.83 & 3
              \end{pmatrix}
          \]
          и обратная к ней матрица:
          \[
              G^{-1}
              = \left( Z^T W Z \right)^{-1}
              = \left(
              \frac{1}{\sigma^2}
              \begin{pmatrix}
                  1.36 & 1.83 \\
                  1.83 & 3
              \end{pmatrix}
              \right)^{-1}
              = \sigma^2
              \begin{pmatrix}
                  4.1  & -2.5 \\
                  -2.5 & 1.86
              \end{pmatrix}
          \]

          Границы доверительного интервала для величины $\theta_1$ определяются смещением:
          \[
              y \sqrt{\element{G^{-1}}{11} \frac{\norm{\pro{\eta}{Z}}_W^2}{n-m}}
              \approx y \sqrt{\sigma^2 4.1 \frac{0.0036}{\sigma^2} \frac{1}{3 - 2}}
              \approx 12.7 \cdot 2 \cdot 0.06
              \approx 1.5
          \]
          где $y$ --- распределения Стьюдента $T(n-m)$ уровня $\frac{1+P_g}{2}$, и доверительный интервал:
          \[
              \left( 1.9 - 1.5; 1.9 + 1.5 \right)
              = \left( 0.4; 3.4 \right) .
          \]
          Ноль не попадает в реализацию доверительного интервала, поэтому гипотеза $\widetilde{\theta_1} = 0$ отклоняется.

          Аналогично для величины $\theta_2$:
          \[
              y \sqrt{\element{G^{-1}}{22} \frac{\norm{\pro{\eta}{Z}}_W^2}{n-m}}
              = y \sqrt{\sigma^2 1.86 \frac{0.0036}{\sigma^2} \frac{1}{3 - 2}}
              = 12.7 \cdot 1.36 \sqrt{0.06}
              \approx 1
          \]
          и доверительный интервал:
          \[
              \left( 0.6 - 1; 0.6 + 1 \right)
              = \left( -0.4; 1.6 \right) .
          \]
          Ноль попадает в реализацию доверительного интервала, поэтому гипотеза $\widetilde{\theta_2} = 0$ принимается.

    \item Доверительный интервал для $\sigma$ получается из неравенства для квадрата нормы перпендикуляра:

          \begin{gather*}
              y_1 < \norm{\pro{\eta}{Z}}_W^2 < y_2 , \\
              y_1 < \frac{0.0036}{\sigma^2} < y_2 , \\
              \frac{1}{y_2} < \frac{\sigma^2}{0.0036} < \frac{1}{y_1} , \\
              \frac{0.0036}{y_2} < \sigma^2 < \frac{0.0036}{y_1} , \\
              \sqrt{\frac{0.0036}{y_2}} < \sigma < \sqrt{\frac{0.0036}{y_1}} ,
          \end{gather*}
          где $y_1$ и $y_2$ --- квантили распределения $\chi^2(n-m)$ уровней $\frac{1-P_g}{2}$ и $\frac{1+P_g}{2}$:
          \begin{gather*}
              \sqrt{\frac{0.0036}{3.84}} < \sigma < \sqrt{\frac{0.0036}{0.004}} , \\
              0.03 < \sigma < 0.94 .
          \end{gather*}

    \item Для проверки гипотезы об отсутствии зависимости необходимо вычислить статистику
          \[
              T = \frac{n-m}{m} \frac{\norm{\pr{\eta}{Z}}_W^2}{\norm{\pro{\eta}{Z}}_W^2} ,
          \]
          где
          \begin{multline*}
              \norm{\pr{\eta}{Z}}_W^2
              = \pr{\eta}{Z}^T W \pr{\eta}{Z}
              = \begin{pmatrix}
                  2.5 & 3.1 & 3.7
              \end{pmatrix}
              \frac{1}{\sigma^2}
              \begin{pmatrix}
                  1 & 0           & 0           \\
                  0 & \frac{1}{4} & 0           \\
                  0 & 0           & \frac{1}{9}
              \end{pmatrix}
              \begin{pmatrix}
                  2.5 \\
                  3.1 \\
                  3.7
              \end{pmatrix} = \\
              %
              = \frac{2.5^2 + \frac{1}{4} \cdot 3.1^2 + \frac{1}{9} 3.7^2}{\sigma^2}
              = \frac{2.5^2 + \frac{1}{4} \cdot 9.61 + \frac{1}{9} \cdot 13.69}{\sigma^2} = \\
              %
              = \frac{2.5^2 + \frac{1}{4} \cdot 9.61 + \frac{1}{9} \cdot 13.69}{\sigma^2}
              \approx \frac{10.2}{\sigma^2} ,
          \end{multline*}
          тогда
          \[
              T
              = \frac{3-2}{2} \frac{\frac{10.2}{\sigma^2}}{\frac{0.0036}{\sigma^2}}
              = \frac{10.2}{0.0072}
              \approx 1417,
          \]
          Наименьший уровень значимости отклонения гипотезы о независимости:
          \[
              \alpha^*
              = 1 - F(T)
              \approx 0.02 ,
          \]
          где $F(x)$ --- функция распределения для распределения Фишера $F(m,n-m)$.
\end{enumerate}

\section*{Методы статистических испытаний}

\subsection*{Задача 1}

Случайные величины $\xi \sim \mathcal{R}[-1, 2]$ и $\varphi \sim E(3)$ являются независимыми, случайная величина $\eta = \xi \sin \varepsilon$.
Оцените $\expectation{e^\eta}$ с отклонением менее $\delta = 0.01$ и вероятностью более $P_\delta = 0.98$.

Решение:

Необходимо сформировать выборку, состоящую из пар $(\xi_i, \varphi_i)$, где величины $\xi_i \sim \mathcal{R}[-1, 2]$,
$\varphi_i \sim E(3)$ и являются независимыми, вычислить величины $\eta_i = \xi_i \sin \varphi_i$, затем $\varepsilon_i = e^{\eta_i}$ и оценку:
\[
    \expectation{e^\eta} \approx \frac{1}{n} \sum_{i=1}^n \varepsilon_i.
\]

Требуемое количество величин с помощью центральной предельной теоремы:
\[
    n
    \ge \overline{n}
    = \left( \Phi^{-1} \left( \frac{1 + P_\delta}{2} \right) \right)^2 \frac{\overline{D}}{\delta^2},
\]
где
\[
    \forall i: \variance{\varepsilon_i} \le \overline{D} .
\]

Величина $\overline{D}$ ограничивает дисперсию величин $\varepsilon_i$:
\begin{gather*}
    -1 \le \xi_i \le 2, -1 \le \sin \varepsilon_i \le 1, \\
    -2 \le \xi_i \sin \varepsilon_i \le 2, \\
    -2 \le \eta_i \le 2, \\
    e^{-2} \le e^{\eta_i} \le e^2, \\
    e^{-2} \le \varepsilon_i \le e^2, \\
    \variance{\varepsilon_i} \le \left( \frac{e^2 - e^{-2}}{2} \right)^2 = \overline{D} ,
\end{gather*}
тогда
\begin{multline*}
    \overline{n}
    = \left( \Phi^{-1} \left( \frac{1 + 0.98}{2} \right) \right)^2 \frac{\left( \frac{e^2 - e^{-2}}{2} \right)^2}{0.01^2}
    = \left( \Phi^{-1} ( 0.99 ) \left( \frac{e^2 - e^{-2}}{2} \right) \right)^2 \cdot 10^4 = \\
    %
    = \left( 2.326 \cdot 3.6268 \right)^2 \cdot 10^4
    = 71.1651 \cdot 10^4
    = 711 651 .
\end{multline*}

\subsection*{Задача 2}

Оценить интеграл
\[
    J = \int \limits_0^{\frac{\pi}{2}} \sqrt{4 - \sin^2 \varphi} d \varphi
\]
с отклонением менее $\delta = 10^{-3}$ и вероятностью более $P_\delta = 0.95$.

Решение:

Постоянные, ограничивающие функцию при $\varphi \in \left[ 0, \frac{\pi}{2} \right]$:
\begin{gather*}
    0 \le \sin \varphi \le 1 , \\
    0 \le \sin^2 \varphi \le 1 , \\
    \sqrt{3} \le \sqrt{4 - \sin^2 \varphi} \le \sqrt{4} = 2 .
\end{gather*}

Преобразуем интеграл:

\begin{multline*}
    J
    = \int \limits_0^{\frac{\pi}{2}} \sqrt{4 - \sin^2 \varphi} d \varphi
    = \left( 2 - \sqrt{3} \right) \int \limits_0^{\frac{\pi}{2}} \frac{\sqrt{4 - \sin^2 \varphi}}{2 - \sqrt{3}} d \varphi = \\
    %
    = \left( 2 - \sqrt{3} \right) \int \limits_0^{\frac{\pi}{2}} \frac{\sqrt{4 - \sin^2 \varphi} - \sqrt{3}}{2 - \sqrt{3}} d \varphi + \left( 2 - \sqrt{3} \right) \sqrt{3} = \\
    %
    = \left( 2 - \sqrt{3} \right) \frac{\pi}{2} \int \limits_0^1 \frac{\sqrt{4 - \sin^2 \left( \frac{\pi}{2} x \right)} - \sqrt{3}}{2 - \sqrt{3}} d x + \left( 2 - \sqrt{3} \right) \sqrt{3} = \\
    %
    = \left( 2 - \sqrt{3} \right) \frac{\pi}{2} \widetilde{J} + \left( 2 - \sqrt{3} \right) \sqrt{3} ,
\end{multline*}
где
\begin{gather*}
    \widetilde{J} = \int \limits_0^1 \widetilde{f}(x) d x , \\
    \widetilde{f}(x) = \frac{\sqrt{4 - \sin^2 \left( \frac{\pi}{2} x \right)} - \sqrt{3}}{2 - \sqrt{3}}
\end{gather*}
и
\[
    x \in [0, 1]: 0 \le \widetilde{f}(x) \le 1 .
\]

Необходимо сформировать выборку, состоящую из пар $(\xi_i, \eta_i)$, в которых $\xi_i \sim \mathcal{R}[0, 1]$, $\eta_i \sim \mathcal{R}[0, 1]$ и
$\xi_i$ и $\eta_i$ независимы, вычислить величины $\varepsilon_i$:
\[
    \varepsilon_i
    = \left \{
    \begin{array}{ll}
        1, & \eta_i < \widetilde{f}(\xi_i) , \\
        0, & \eta_i \ge \widetilde{f}(\xi_i) .
    \end{array}
    \right .
\]
оценить интегралы:
\begin{gather*}
    \widetilde{J} \approx \widetilde{J}_\varepsilon = \frac{1}{n} \sum_{i=1}^n \varepsilon , \\
    J \approx J_\varepsilon = \left( 2 - \sqrt{3} \right) \frac{\pi}{2} \widetilde{J}_\varepsilon + \left( 2 - \sqrt{3} \right) \sqrt{3} .
\end{gather*}

Требуемый объём выборки из центральной предельной теоремы:
\begin{multline*}
    n
    \ge \overline{n}
    = \left( \Phi^{-1} \left( \frac{1 + P_\delta}{2} \right) \right)^2 \frac{\frac{1}{4}}{\left( \frac{\delta}{\left( 2 - \sqrt{3} \right) \frac{\pi}{2}} \right)^2}
    = \left( \Phi^{-1} \left( \frac{1 + 0.95}{2} \right) \right)^2 \frac{\frac{1}{4}}{\left( \frac{10^{-3}}{\left( 2 - \sqrt{3} \right) \frac{\pi}{2}} \right)^2} = \\
    %
    = \left( \Phi^{-1} ( 0.975 ) \frac{1}{2} \left( 2 - \sqrt{3} \right) \frac{\pi}{2} \right)^2 10^6
    = \left( 1.96 \cdot 0.5 \cdot 0.26795 \cdot 1.5708 \right)^2 10^6 = \\
    %
    = \left( 0.412477 \right)^2 10^6
    = 0.170139 \cdot 10^6
    = 170 139 .
\end{multline*}

\end{document}

    \chapter{Оценка ковариационной матрицы}

\section{Оценивание}

Ковариационная матрица $\variance{X}$ неизвестна, но её можно оценить: нужно взять $m$ моментов времени и в каждый из моментов определить состояние приёмников $X_k$
($k = \overline{1,m}$):
\[
    X_k =
    \begin{pmatrix}
        x_{1,1} \\
        \dots   \\
        x_{i,k} \\
        \dots   \\
        x_{j,k} \\
        \dots   \\
        x_{n,k}
    \end{pmatrix} .
\]
Ковариация двух компонент $x_i$ и $x_j$:
\[
    \covariance{x_i}{x_j}
    = \expectation{\left ( x_i - \expectation{x_i} \right ) \left ( x_j - \expectation{x_j} \right )^H}
    = \expectation{ x_i x_j^H},
\]
поскольку $\expectation{x_k} = 0$.

В качестве оценки используем выражение:
\begin{multline*}
    \widehat{\covariance{x_i}{x_j}}
    = \frac{1}{m} \sum_{k=1}^m x_{i,k} x_{j,k}^H
    = \sum_{k=1}^m \frac{1}{\sqrt{m}}x_{i,k} \frac{1}{\sqrt{m}} x_{j,k}^H = \\
    %
    = \frac{1}{\sqrt{m}}
    \begin{pmatrix}
        x_{i,1} & x_{i,2} & \dots & x_{i,m}
    \end{pmatrix}
    \frac{1}{\sqrt{m}}
    \begin{pmatrix}
        x_{j,1}^H \\
        x_{j,2}^H \\
        \dots     \\
        x_{j,m}^H
    \end{pmatrix}
\end{multline*}
Все оценки ковариаций можно получить умножением матриц:
Полученные векторы объединяем в матрицу $Y$:
\[
    \widehat{R} =
    \frac{1}{\sqrt{m}}
    \begin{pmatrix}
        x_{1,1} & x_{1,2} & \dots  & x_{1,m} \\
        x_{2,1} & x_{2,2} & \dots  & x_{2,m} \\
        \vdots  & \vdots  & \ddots & \vdots  \\
        x_{n,1} & x_{n,2} & \dots  & x_{n,m}
    \end{pmatrix}
    \frac{1}{\sqrt{m}}
    \begin{pmatrix}
        x_{1,1}^* & x_{2,2}^* & \dots  & x_{n,m}^* \\
        x_{1,2}^* & x_{2,2}^* & \dots  & x_{n,m}^* \\
        \vdots    & \vdots    & \ddots & \vdots    \\
        x_{1,m}^* & x_{2,m}^* & \dots  & x_{n,m}^*
    \end{pmatrix}
    .
\]
Правая матрица является эрмитовосопряженной к левой матрице, поэтому если:
\[
    Y =
    \frac{1}{\sqrt{m}}
    \begin{pmatrix}
        x_{1,1} & x_{1,2} & \dots  & x_{1,m} \\
        x_{2,1} & x_{2,2} & \dots  & x_{2,m} \\
        \vdots  & \vdots  & \ddots & \vdots  \\
        x_{n,1} & x_{n,2} & \dots  & x_{n,m}
    \end{pmatrix} ,
\]
тогда
\[
    \widehat{R} = Y Y^H .
\]


\section{Ортогонализация и обращение}

Матрица $\widehat{R}$ является факторизованной, поэтому можно найти факторизацию обратной матрицы $\widehat{R}^{-1}$.

Пусть $\Phi$ является преобразованием, ортогонализующим строки матрицы $Y$, то есть строки матрицы $\Phi Y$ являются взаимно ортогональными:
\[
    \left ( \Phi Y \right ) \left ( \Phi Y \right )^H = I_n ,
\]
отсюда
\begin{gather*}
    \Phi Y Y^H \Phi^H = I_n , \\
    \Phi \widehat{R} \Phi^H = I_n , \\
    \Phi \widehat{R} = \Phi^{-H}, \\
    \widehat{R} = \Phi^{-1} \Phi^{-H}, \\
    \widehat{R}^{-1} = \left ( \Phi^{-1} \Phi^{-H} \right )^{-1}, \\
    \widehat{R}^{-1} = \Phi^H \Phi .
\end{gather*}

\section{Вычисления}

Вычисление квадратичной формы:
\[
    V^H \widehat{R}^{-1} V
    = V^H \Phi^H \Phi V
    = \left ( \Phi V \right )^H \Phi V
    = \norm{\Phi V}^2 .
\]
Вычисление оптимального весового вектора:
\[
    W_{max}
    = \widehat{R}^{-1} U
    = \Phi^H \Phi U .
\]

    \section{О линейной зависимости числовых векторов}

Векторы $x$ и $y$ являются линейно зависимыми, если:
\[
    x = c y,
\]
для некоторого числа $c$.

Пример
\[
    \begin{pmatrix}
        6 \\
        8 \\
        - 12
    \end{pmatrix}
    = 2
    \begin{pmatrix}
        3 \\
        4 \\
        -6
    \end{pmatrix}
\]

\textcolor{red}{рисунок с лучами и масштабированием}

В случае комплексных векторов вместо числовых лучей появляются комплексные плоскости. Умножение на комплексное число можно представить как изменение модуля и
поворот, поэтому в случае векторов допускается не только масштабирование компонент, но и одновременный поворот всех компонент на один угол.

\textcolor{red}{рисунок с плоскостями масштабированием и поворотом}

Пример
\[
    \begin{pmatrix}
        -1 + 3i \\
        2 - 1i  \\
        3 + 1i
    \end{pmatrix}
    =
    (1 + 2i)
    \begin{pmatrix}
        1 + i \\
        -i    \\
        1 - i
    \end{pmatrix}
    .
\]

    \begin{enumerate}
        \item Обусловленность задачи решения системы линейных уравнений.
        \item $LU$ и $LDL^T$ разложения (с выбором ведущего элемента).
        \item Вычисление QR разложения.
        \item Вычисление собственных значений и собственных векторов. Нахождение наибольшего собственного значения.
        \item Вычисление сингулярного разложения. Приложения (Колесников).
        \item Итерационные методы решения СЛУ.
        \item Проекционные методы решения СЛУ.
        \item Методы факторизации (из Ратынского).
        \item Быстрое умножение матриц - алгоритм Штрассена (Блейхут).
        \item Быстрые алгоритмы свёртки (Блейхут).
        \item Быстрое преобразование Фурье (Блейхут).
        \item Быстрое решение тёплицевых систем (Блейхут).
        \item Ленточные матрицы. Решение систем с трехдиагональной матрицей (прогонка).
    \end{enumerate}

    Основные задачи.
    \begin{enumerate}
        \item Разложения: $LU$, $LDL^T$, $QR$, $U \Sigma V^T$.
        \item Решение системы.
        \item Собственные значения и векторы.
    \end{enumerate}

    Прикладные задачи:
    \begin{enumerate}
        \item Задачи из книги Ратынского.
        \item Задачи из книги Матасова.
    \end{enumerate}
\end{document}