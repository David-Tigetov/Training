\documentclass[a4paper,12pt]{book}

\usepackage[T1]{fontenc}
\usepackage[utf8]{inputenc}
\usepackage[english,russian]{babel}
\usepackage[margin=2cm]{geometry}
\usepackage{amsmath}
\usepackage{amssymb}
\usepackage{tikz}
\usepackage{color}
\usepackage{amsfonts}
\usepackage{stmaryrd}

% команды вывода первой частной производной
\newcommand{\fpd}[1]{\frac{\partial}{\partial #1}}
\newcommand{\fpda}[2]{\frac{\partial #1}{\partial #2}}
\newcommand{\fpdp}[2]{\fpd{#2} \left ( #1 \right )}

\newcommand{\expectation}[1]{\mathtt{M} \left [ #1 \right ]}
\newcommand{\conditionalexpectation}[2]{\expectation{ #1 \left | #2 \right .}}
\newcommand{\variance}[1]{\mathtt{D} \left [ #1 \right ]}
\newcommand{\covariance}[2]{\mathtt{cov} \left ( #1, #2 \right )}

\newcommand{\modulus}[1]{\left | #1 \right |}
\newcommand{\norm}[1]{\left \| {#1} \right \|}

\newcommand{\event}[1]{\left \{ #1 \right \} }
\newcommand{\probability}[1]{P \event{#1}}


\begin{document}

    \title{Прикладные методы линейной алгебры}
    \author{Тигетов Давид Георгиевич}
    \maketitle

    \tableofcontents

    % план занятий
    \chapter{План занятий}

Для отношения Релея.
\begin{itemize}
    \item[Занятие 1.] Унитарные пространства. Сопряжённый оператор. Свойства сопряженного оператора.
    \item[Занятие 2.] Теорема Шура. Нормальные матрицы. Эрмитовые матрицы.
    \item[Занятие 3.] Экстремумы отношения Релея.
    \item[Занятие 4.] Экстремумы отношения Релея. Вычисление в Matlab.
    \item[Занятие 5.] Комплексные огибающие. Двухканальный излучатель. Коэффициент усиления. Пример 2. Пример 3. Пример 4. Пример 5.
    \item[Занятие 6.] Энергетическое ограничение. Коэффициент полезного действия. Примеры. Антенна.
\end{itemize}

    % отношение Релея
    \chapter{Отношение Релея}


\section{Унитарные пространства}

Пусть $\mathbb{C}$ --- обозначает поле комплексных чисел, и $\mathcal{U}$ --- множество векторов, для которых определена операции сложения $+$ векторов и умножения
векторов на число из поля $\mathbb{C}$, обладающих свойствами для любых $u, v, w \in \mathcal{U}$ и $\alpha, \beta \in \mathbb{C}$:
\begin{enumerate}
    \item $u + ( v + w ) = ( u + v ) + w$,
    \item $\exists 0 \in U: u + 0 = 0 + u = u$,
    \item $\exists (-u) \in U: u + (-u) = (-u) + u = 0$,
    \item $u + v = v + u$,
    \item $(\alpha + \beta) u = \alpha u + \beta u$,
    \item $\alpha ( u + v ) = \alpha u + \alpha v$,
    \item $\alpha (\beta u) = (\alpha \beta) u$,
    \item $u = 1 \cdot u$.
\end{enumerate}
Множество $\mathcal{U}$ с определенными на нём операциями сложения векторов и умножения на число называется \definition{линейным (векторным) пространством}.

Пусть на множестве всех пар векторов пространства $\mathcal{U}$ определена функция $\scalarproduct{\cdot}{\cdot}$, называемая \definition{скалярным произведением}, обладающая
следующими свойствами для любых $u, v \in \mathcal{U}$ и числа $\lambda \in \mathbb{C}$:
\begin{enumerate}
    \item $\scalarproduct{u}{u} \ge 0$,
    \item $\scalarproduct{u}{u} = 0 \Leftrightarrow u = 0$,
    \item $\scalarproduct{u + w}{v} = \scalarproduct{u}{v} + \scalarproduct{w}{v}$,
    \item $\scalarproduct{\lambda u}{v} = \lambda \scalarproduct{u}{v}$,
    \item $\scalarproduct{u}{v} = \overline{\scalarproduct{v}{u}}$,
\end{enumerate}
где черта $\overline{\cdot}$ обозначает комплексное сопряжение. Из приведённых свойств скалярного произведения следует:
\begin{gather*}
    \scalarproduct{u}{v + w}
    = \overline{\scalarproduct{v + w}{u}}
    = \overline{\scalarproduct{v}{u} + \scalarproduct{w}{u}}
    = \overline{\scalarproduct{v}{u}} + \overline{\scalarproduct{w}{u}}
    = \scalarproduct{u}{v} + \scalarproduct{u}{w}, \\
    %
    \scalarproduct{u}{\lambda v}
    = \overline{\scalarproduct{\lambda v}{u}}
    = \overline{\lambda  \scalarproduct{v}{u}}
    = \overline{\lambda} \overline{\scalarproduct{v}{u}}
    = \overline{\lambda} \scalarproduct{u}{v} .
\end{gather*}

Если в линейном пространстве $\mathcal{U}$ задано скалярное произведение $\scalarproduct{\cdot}{\cdot}$, то такое пространство называется \definition{унитарным}.

С помощью скалярного произведения можно определить норму $\norm{\cdot}$ векторов:
\[
    \norm{u} = \sqrt{\scalarproduct{u}{u}}.
\]
Такое определение будет удовлетворять всем свойствам нормы:
\begin{enumerate}
    \item $\norm{u} \ge 0$,
    \item $\norm{u} = 0 \Leftrightarrow u = 0$,
    \item $\norm{\lambda u} = \modulus{\lambda} \norm{u}$,
    \item $\norm{u + v} \le \norm{u} + \norm{v}$.
\end{enumerate}
Пространство с определенной для его элементов нормой называют \definition{нормированным пространством}.


\section{Сопряженный оператор}

Пусть $\mathcal{U}$ является унитарным пространством со скалярным произведением $\scalarproduct{\cdot}{\cdot}_\mathcal{U}$ и $\mathcal{V}$ тоже унитарное пространство
со своим скалярным произведением $\scalarproduct{\cdot}{\cdot}_\mathcal{V}$.

Пусть $\mathcal{A} : \mathcal{U} \rightarrow \mathcal{V}$ --- оператор, действующий из пространства $\mathcal{U}$ в пространство $\mathcal{V}$.
Оператор $\mathcal{A}^*$ называется \definition{сопряженным} к $\mathcal{A}$, если:
\begin{gather}
    \scalarproduct{\mathcal{A} u}{v}_{\mathcal{V}} = \scalarproduct{u}{\mathcal{A}^* v}_{\mathcal{U}}
    \label{rayleigh:conjugation:scalars_equality}, \\
    \forall u \in \mathcal{U}, \forall v \in \mathcal{V}
    \notag.
\end{gather}

Если сопряженный оператор $\mathcal{A}^*$ существует, то он является единственным. Это следует непосредственно из определения
\ref{rayleigh:conjugation:scalars_equality} и следующего утверждения.

\begin{statement}~\label{rayleigh:conjugation:identity}
    Если для операторов $\mathcal{B}_1$ и $\mathcal{B}_2$ выполняется равенство:
    \begin{gather*}
        \scalarproduct{u}{\mathcal{B}_1 v}_{\mathcal{U}} = \scalarproduct{u}{\mathcal{B}_2 v}_{\mathcal{U}} , \\
        \forall u \in \mathcal{U}, \forall v \in \mathcal{V},
    \end{gather*}
    тогда
    \[
        \mathcal{B}_1 = \mathcal{B}_2 .
    \]
\end{statement}
\begin{proof}
    Преобразуем равенство к виду:
    \begin{gather*}
        \scalarproduct{u}{\mathcal{B}_1^* v}_{\mathcal{U}} = \scalarproduct{u}{\mathcal{B}_2^* v}_{\mathcal{U}} , \\
        \scalarproduct{u}{\mathcal{B}_1^* v}_{\mathcal{U}} - \scalarproduct{u}{\mathcal{B}_2^* v}_{\mathcal{U}} = 0 , \\
        \scalarproduct{u}{\mathcal{B}_1^* v - \mathcal{B}_2^* v}_{\mathcal{U}} = 0 , \\
        \scalarproduct{u}{\left( \mathcal{B}_1^* - \mathcal{B}_2^* \right) v}_{\mathcal{U}} = 0 .
    \end{gather*}
    Поскольку полученное равенство выполняется для любых векторов $u$ и $v$, то возьмём в качестве вектора $u$ вектор
    $\left( \mathcal{B}_1^* - \mathcal{B}_2^* \right) v$:
    \[
        \scalarproduct{\left( \mathcal{B}_1^* - \mathcal{B}_2^* \right) v}{\left( \mathcal{B}_1^* - \mathcal{B}_2^* \right) v}_{\mathcal{U}} = 0 .
    \]
    Из свойства 2 скалярного произведения следует:
    \begin{gather*}
        \left( \mathcal{B}_1^* - \mathcal{B}_2^* \right) v = 0 , \\
        \mathcal{B}_1^* v = \mathcal{B}_2^* v .
    \end{gather*}
    Полученное равенство выполняется для всех векторов $v$, поэтому операторы идентичны:
    \[
        \mathcal{B}_1 = \mathcal{B}_2 .
    \]
\end{proof}

Используя доказанное утверждение легко заметить, что если сопряженный оператор $\mathcal{A}$ существует, то он единственный. Действительно, представим,
что есть два сопряжённых оператора $\mathcal{A}_1^*$ и $\mathcal{A}_2^*$, которые удовлетворяют определению \eqref{rayleigh:conjugation:scalars_equality}:
\begin{gather*}
    \scalarproduct{u}{\mathcal{A}_1^* v}_{\mathcal{U}} = \scalarproduct{\mathcal{A} u}{v}_{\mathcal{V}} = \scalarproduct{u}{\mathcal{A}_2^* v}_{\mathcal{U}} , \\
    \scalarproduct{u}{\mathcal{A}_1^* v}_{\mathcal{U}} = \scalarproduct{u}{\mathcal{A}_2^* v}_{\mathcal{U}} .
\end{gather*}
Поскольку равенство выполняется при всех $u$ и $v$, то в силу утверждения \ref{rayleigh:conjugation:identity} операторы идентичны:
\[
    \mathcal{A}_1^* = \mathcal{A}_2^* .
\]

Остался вопрос об условиях существовании сопряженного оператора $\mathcal{A}^*$. Как будет показано далее, при некоторых условиях сопряженный оператор существует,
более того его можно построить в явном виде.

Пусть оператор $\mathcal{A}$ является \definition{линейным}, то есть для любых $u_1, u_2 \in \mathcal{U}$ и любого числа $\lambda \in \mathbb{C}$ выполняются равенства:
\begin{enumerate}
    \item $\mathcal{A}(u_1 + u_2) = \mathcal{A}(u_1) + \mathcal{A}(u_2)$,
    \item $\mathcal{A}(\lambda u_1) = \lambda \mathcal{A}(u_1)$.
\end{enumerate}

Линейный оператор можно задать следующим образом: выбрать базисный набор векторов $e_i$ пространства $\mathcal{U}$ и определить действие оператора на базисные векторы
$\mathcal{A} e_i$, затем используя линейность для всякого элемента $u \in \mathcal{U}$:
\[
    u = c_1 e_1 + c_2 e_2 + \dots,
\]
получим:
\[
    \mathcal{A} u
    = \mathcal{A} ( c_1 e_1 + c_2 e_2 + \dots )
    = c_1 \mathcal{A} e_1 + c_2 \mathcal{A} e_2 + \dots
\]

При построении сопряженного оператора $\mathcal{A}^*$ можно использовать такой же способ его определения.

\begin{statement}
    Пусть $\mathcal{U}$ и $\mathcal{V}$ --- унитарные конечномерные пространства и $\mathcal{A} : \mathcal{U} \rightarrow \mathcal{V}$ --- линейный оператор, тогда
    \begin{enumerate}
        \item существует сопряженный к $\mathcal{A}$ оператор $\mathcal{A}^*$,
        \item если $A$ --- матрица оператора $\mathcal{A}$ в ортогональных базисах пространств $\mathcal{U}$ и $\mathcal{V}$, тогда матрица $A^*$ сопряженного оператора
        $\mathcal{A^*}$:
        \[
            A^* = \overline{A}^T .
        \]
    \end{enumerate}
\end{statement}
\begin{proof}
    Пусть набор векторов $e = \{ e_1, \dots, e_n \}$ является ортонормированными базисом в пространстве $\mathcal{U}$, а набор векторов $f = \{f_1, \dots, f_m \}$
    является ортономированным базисом в пространстве $\mathcal{V}$. Пусть $A = [a_{ij}] \in \Cspace{m \times n}$ является матрицей оператора $\mathcal{A}$ в базисах
    $e$ и $f$, тогда:
    \[
        \mathcal{A} e_j = a_{1j} f_1 + \dots + a_{mj} f_m,
    \]
    тем самым определено действие оператора $\mathcal{A}$ на базисные векторы $e_j$.

    Попробуем определить оператор $\mathcal{B}$ так, чтобы равенство \eqref{rayleigh:conjugation:scalars_equality} выполнялось для набора векторов $e$ и $f$:
    \begin{equation}
        \label{rayleigh:conjugation:basis_scalars_equality}
        \scalarproduct{\mathcal{A} e_j}{f_i}_\mathcal{V} = \scalarproduct{e_j}{\mathcal{B} f_i}_\mathcal{U} .
    \end{equation}
    Поскольку $\mathcal{B} f_i \in \mathcal{U}$, то:
    \[
        \mathcal{B} f_i = b_{1i} e_1 + \dots + b_{ni} e_n ,
    \]
    причём коэффициенты $b_{ji}$ следует выбирать таким образом, чтобы выполнялось равенство \eqref{rayleigh:conjugation:basis_scalars_equality}, которое в силу
    ортонормированности наборов $e$ и $f$ приводит к равенству:
    \begin{align*}
        \scalarproduct{\mathcal{A} e_j}{f_i}_\mathcal{V} & = \scalarproduct{e_j}{\mathcal{B} f_i}_\mathcal{U} , \\
        \scalarproduct{a_{1j} f_1 + \dots + a_{mj} f_m}{f_i}_\mathcal{V} & = \scalarproduct{e_j}{b_{1i} e_1 + \dots + b_{ni} e_n}_\mathcal{U} , \\
        a_{1j} \scalarproduct{f_1}{f_i}_\mathcal{V} + \dots + a_{mj} \scalarproduct{f_m}{f_i}_\mathcal{V} & = \overline{b_{1i}} \scalarproduct{e_j}{e_1}_\mathcal{U} + \dots + \overline{b_{ni}} \scalarproduct{e_j}{e_n}_\mathcal{U} , \\
        a_{ij} \scalarproduct{f_i}{f_i}_\mathcal{V} & = \overline{b_{ji}} \scalarproduct{e_j}{e_j}_\mathcal{U} , \\
        a_{ij} & = \overline{b_{ji}} , \\
        \overline{a_{ij}} & = b_{ji} .
    \end{align*}
    Таким образом, определено действие оператора $\mathcal{B}$ на базисные векторы $f_i$:
    \begin{equation}
        \label{rayleigh:conjugation:basis_images}
        \mathcal{B} f_i = \overline{a_{i1}} e_1 + \dots + \overline{a_{in}} e_n .
    \end{equation}

    Теперь распространим действие оператора $\mathcal{B}$ на все векторы $v \in \mathcal{V}$, сделав оператор $\mathcal{B}$ линейным, пусть
    \[
        v = v_1 f_1 + \dots + v_m f_m,
    \]
    тогда
    \[
        \mathcal{B} v = v_1 \mathcal{B} f_1 + \dots + v_m \mathcal{B} f_m .
    \]
    Проверим, что при таком определении оператор $\mathcal{B}$ оказывается сопряженным к оператору $\mathcal{A}$. Пусть $u \in \mathcal{U}$ --- произвольный вектор
    пространства $U$:
    \[
        u = u_1 e_1 + \dots + u_n e_n,
    \]
    тогда
    \begin{multline*}
        \scalarproduct{\mathcal{A} u}{v}_\mathcal{V}
        = \scalarproduct{\mathcal{A} (u_1 e_1 + \dots + u_n e_n)}{v}_\mathcal{V}
        = \scalarproduct{u_1 \mathcal{A} e_1 + \dots + u_n \mathcal{A} e_n}{v}_\mathcal{V} = \\
        %
        = \scalarproduct{u_1 \mathcal{A} e_1 + \dots + u_n \mathcal{A} e_n}{v_1 f_1 + \dots + v_m f_m}_\mathcal{V} = \\
        %
        = \sum_{i=1}^n u_i \sum_{j=1}^m \overline{v_j} \scalarproduct{\mathcal{A} e_i}{f_j}_\mathcal{V}
        = \sum_{i=1}^n u_i \sum_{j=1}^m \overline{v_j} \scalarproduct{e_i}{\mathcal{B} f_j}_\mathcal{U} = \\
        %
        = \scalarproduct{u_1 e_i + \dots + u_n e_n}{v_1 \mathcal{B} f_1 + \dots + v_m \mathcal{B} f_m}_\mathcal{U} = \\
        %
        = \scalarproduct{u}{v_1 \mathcal{B} f_1 + \dots + v_m \mathcal{B} f_m}_\mathcal{U}
        = \scalarproduct{u}{\mathcal{B} (v_1 f_1 + \dots + v_m f_m)}_\mathcal{U}
        = \scalarproduct{u}{\mathcal{B} v}_\mathcal{U} .
    \end{multline*}
    Таким образом, $\mathcal{B}$ является сопряженным к $\mathcal{A}$:
    \[
        \mathcal{A}^* = \mathcal{B}.
    \]
    Кроме того, из равенства \eqref{rayleigh:conjugation:basis_images} следует, что матрица $B$ оператора $\mathcal{B}$ в базисах $f$ и $e$:
    \[
        B =
        \begin{pmatrix}
            \overline{a_{11}} & \overline{a_{21}} & \dots  & \overline{a_{m1}} \\
            \overline{a_{12}} & \overline{a_{22}} & \dots  & \overline{a_{m2}} \\
            \vdots            & \vdots            & \ddots & \vdots            \\
            \overline{a_{1n}} & \overline{a_{2n}} & \dots  & \overline{a_{mn}}
        \end{pmatrix}
        =
        \begin{pmatrix}
            \overline{a_{11}} & \overline{a_{12}} & \dots  & \overline{a_{1n}} \\
            \overline{a_{21}} & \overline{a_{22}} & \dots  & \overline{a_{2n}} \\
            \vdots            & \vdots            & \ddots & \vdots            \\
            \overline{a_{n1}} & \overline{a_{n2}} & \dots  & \overline{a_{nm}}
        \end{pmatrix}^T
        = \overline{A}^T .
    \]
    То есть матрица $A^*$ сопряженного оператора $\mathcal{A}^*$:
    \[
        A^* = \overline{A}^T .
    \]
\end{proof}

Выделим некоторые свойства сопряженного оператора.
\begin{enumerate}
    \item Сопряженным к оператору $\alpha \mathcal{A}$ является оператор $\overline{\alpha} \mathcal{A}^*$. Действительно:
    \begin{align*}
        \scalarproduct{\alpha \mathcal{A} u}{v}_{\mathcal{V}}
        & = \scalarproduct{u}{\left( \alpha \mathcal{A} \right)^* v}_{\mathcal{U}} , \\
        %
        \scalarproduct{\alpha \mathcal{A} u}{v}_{\mathcal{V}}
        & = \alpha \scalarproduct{\mathcal{A} u}{v}_{\mathcal{V}}
        = \alpha \scalarproduct{u}{\mathcal{A}^* v}_{\mathcal{U}}
        = \scalarproduct{u}{\overline{\alpha} \mathcal{A}^* v}_{\mathcal{U}} .
    \end{align*}
    В силу утверждения \ref{rayleigh:conjugation:identity}:
    \[
        \scalarproduct{u}{\left( \alpha \mathcal{A} \right)^* v}_{\mathcal{U}} = \scalarproduct{u}{\overline{\alpha} \mathcal{A}^* v}_{\mathcal{U}}
        \Rightarrow
        \left( \alpha \mathcal{A} \right)^* = \overline{\alpha} \mathcal{A}^*.
    \]

    \item Сопряженным к оператору $\mathcal{A} + \mathcal{B}$ является оператор $\mathcal{A}^* + \mathcal{B}^*$. Действительно
    \begin{align*}
        \scalarproduct{(\mathcal{A} + \mathcal{B}) u}{v}_{\mathcal{V}}
        & = \scalarproduct{u}{( \mathcal{A} + \mathcal{B})^* v}_{\mathcal{U}} , \\
        %
        \scalarproduct{(\mathcal{A} + \mathcal{B}) u}{v}_{\mathcal{V}}
        & = \scalarproduct{\mathcal{A} u + \mathcal{B} u}{v}_{\mathcal{V}}
        = \scalarproduct{\mathcal{A} u}{v}_{\mathcal{V}} + \scalarproduct{\mathcal{B} u}{v}_{\mathcal{V}}
        = \scalarproduct{u}{\mathcal{A}^* v}_{\mathcal{U}} + \scalarproduct{u}{\mathcal{B}^* v}_{\mathcal{U}} = \\
        & = \scalarproduct{u}{\mathcal{A}^* v + \mathcal{B}^* v}_{\mathcal{U}}
        = \scalarproduct{u}{(\mathcal{A}^* + \mathcal{B}^*) v}_{\mathcal{U}} .
    \end{align*}
    В силу утверждения \ref{rayleigh:conjugation:identity}:
    \[
        \scalarproduct{u}{(\mathcal{A} + \mathcal{B})^* v}_{\mathcal{U}} = \scalarproduct{u}{(\mathcal{A}^* + \mathcal{B}^*) v}_{\mathcal{U}}
        \Rightarrow
        ( \mathcal{A} + \mathcal{B})^* = \mathcal{A}^* + \mathcal{B}^*
    \]

    \item Сопряженным к оператору $\mathcal{A} \mathcal{B}$ является оператор $\mathcal{B}^* \mathcal{A}^*$. Действительно
    \begin{align*}
        \scalarproduct{\mathcal{A} \mathcal{B} u}{v}_{\mathcal{V}}
        & = \scalarproduct{u}{(\mathcal{A} \mathcal{B})^* v}_{\mathcal{U}} , \\
        %
        \scalarproduct{\mathcal{A} \mathcal{B} u}{v}_{\mathcal{V}}
        & = \scalarproduct{\mathcal{B} u}{\mathcal{A}^* v}_{\mathcal{W}}
        = \scalarproduct{u}{\mathcal{B}^* \mathcal{A}^* v}_{\mathcal{U}} .
    \end{align*}
    В силу утверждения \ref{rayleigh:conjugation:identity}:
    \[
        \scalarproduct{u}{(\mathcal{A} \mathcal{B})^* v}_{\mathcal{U}} = \scalarproduct{u}{\mathcal{B}^* \mathcal{A}^* v}_{\mathcal{U}}
        \Rightarrow
        (\mathcal{A} \mathcal{B})^* = \mathcal{B}^* \mathcal{A}^* .
    \]

    \item У сопряженного оператора $\mathcal{A}^*$ тоже есть сопряженный оператор $(\mathcal{A}^*)^*$, который совпадает с исходным оператором $\mathcal{A}$.
    Действительно:
    \begin{align*}
        \scalarproduct{\mathcal{A}^* v}{u}_{\mathcal{U}}
        & = \scalarproduct{v}{(\mathcal{A}^*)^* u}_{\mathcal{V}} , \\
        %
        \scalarproduct{\mathcal{A}^* v}{u}_{\mathcal{U}}
        & = \overline{\scalarproduct{u}{\mathcal{A}^* v}_{\mathcal{V}}}
        = \overline{\scalarproduct{\mathcal{A} u}{v}_{\mathcal{V}}}
        = \scalarproduct{v}{\mathcal{A} u}_{\mathcal{V}}.
    \end{align*}
    В силу утверждения \ref{rayleigh:conjugation:identity}:
    \[
        \scalarproduct{v}{(\mathcal{A}^*)^* u}_{\mathcal{V}} = \scalarproduct{v}{\mathcal{A} u}_{\mathcal{V}}
        \Rightarrow
        (\mathcal{A}^*)^* = \mathcal{A} .
    \]
\end{enumerate}

Из свойства 3 следует:
\[
    \left ( \mathcal{A}^n \right )^* = \left ( \mathcal{A}^* \right )^n ,
\]
действительно:
\[
    \left ( \mathcal{A}^n \right )^*
    = \left ( \mathcal{A}^{n-1} \mathcal{A} \right )^*
    = \mathcal{A}^* \left ( \mathcal{A}^{n-1} \right )^*
    = \dots
    = \mathcal{A}^* \mathcal{A}^* \dots \mathcal{A}^*
    = \left ( \mathcal{A}^* \right )^n .
\]

\begin{example}
    Найти сопряженный оператор к оператору $\alpha \mathcal{A}^3 + \beta \mathcal{B}^2 \mathcal{C}$.

    Используя доказанные свойства, получим:
    \[
        \left ( \alpha \mathcal{A}^3 + \beta \mathcal{B}^2 \mathcal{C} \right )^*
        = \left ( \alpha \mathcal{A}^3 \right )^* + \left ( \beta \mathcal{B}^2 \mathcal{C} \right )^*
        = \overline{\alpha} \left ( \mathcal{A}^* \right )^3 + \overline{\beta} \mathcal{C}^* \left ( \mathcal{B}^* \right )^2 .
    \]
\end{example}

Выделяют следующие классы операторов, которые имеют полезные свойства.

Оператор $\mathcal{A}$ называется \definition{нормальным}, если:
\[
    \mathcal{A} \mathcal{A}^* = \mathcal{A}^* \mathcal{A}.
\]

Оператор $\mathcal{A}$ называется \definition{самоспоряженным} или \definition{эрмитовым}, если:
\[
    \mathcal{A}^* = \mathcal{A}.
\]

Оператор $\mathcal{A}$ называется \definition{унитарным}, если:
\[
    \mathcal{A}^* = \mathcal{A}^{-1}.
\]

Легко видеть, что если $\mathcal{A}$ является эрмитовым оператором, то он является нормальным:
\[
    \mathcal{A} \mathcal{A}^*
    = \mathcal{A} \mathcal{A}
    = \mathcal{A}^* \mathcal{A},
\]
и унитарный оператор $\mathcal{A}$ тоже является нормальным:
\[
    \mathcal{A} \mathcal{A}^*
    = \mathcal{A} \mathcal{A}^{-1}
    = I
    = \mathcal{A}^{-1} \mathcal{A}
    = \mathcal{A}^* \mathcal{A}.
\]

Аналогичные определения вводятся и для матриц, которые называют \definition{нормальными}, \definition{самосопряженными} или \definition{эрмитовыми}, \definition{унитарными}.


\section{Теорема Шура}

Согласно теореме Шура любая квадратная матрица $A$ унитарно подобна верхнетреугольной матрице, то есть существует унитарная матрица $U$:
\[
    U^* U = I,
\]
такая что
\[
    U^* A U = R, \\
\]
где $R$ --- верхнетреугольная матрица. Умножая последнее равенство слева на $U$ и справа на $U^*$, получим
\begin{align}
    U^* A U & = R,
    \notag \\
    U U^* A U U^* & = U R U^*,
    \notag \\
    A & = U R U^*
    \label{rayleigh:schur:decomposition}
\end{align}

\begin{example}
    Для матрицы $A$:
    \[
        A = \begin{pmatrix}
                3 & 2 \\
                1 & 4
        \end{pmatrix}
    \]
    необходимо найти унитарную матрицу $U$ и верхнетреугольную матрицу $R$.

    Для матрицы $U$ выполняется равенство:
    \[
        U^* A U = R.
    \]
    В качестве $U$ возьмем матрицу вращения, пусть $c = \cos \alpha$ и $s = \sin \alpha$:
    \[
        U
        = \begin{pmatrix}
              c  & s \\
              -s & c
        \end{pmatrix} ,
    \]
    тогда
    \[
        U^* A U
        = \begin{pmatrix}
              c & - s \\
              s & c
        \end{pmatrix}
        \begin{pmatrix}
            3 & 2 \\
            1 & 4
        \end{pmatrix}
        \begin{pmatrix}
            c   & s \\
            - s & c
        \end{pmatrix}
        = \begin{pmatrix}
              r_{11} & r_{12} \\
              r_{21} & r_{22}
        \end{pmatrix} .
    \]
    Нужно выбрать величину $\alpha$ так, чтобы элемент $r_{12}$ оказался равен нулю. Вычисляем матрицу в правой части:
    \begin{multline*}
        \begin{pmatrix}
            3 c - 1 s & 2 c - 4 s \\
            3 s + 1 c & 2 s + 4 c
        \end{pmatrix}
        \begin{pmatrix}
            c   & s \\
            - s & c
        \end{pmatrix}
        = \\
        %
        = \begin{pmatrix}
              3 c^2 - c s - 2 c s + 4 s^2 & 3 c s - s^2 + 2 c^2 - 4 c s \\
              3 c s + c^2 - 2 s^2 - 4 c s & 3 s^2 + c s + 2 c s + 4 c^2
        \end{pmatrix}
        = \\
        %
        = \begin{pmatrix}
              3 c^2 - 3 c s + 4 s^2 & - s^2 + 2 c^2 - c s   \\
              c^2 - 2 s^2 - c s     & 3 s^2 + 3 c s + 4 c^2
        \end{pmatrix}
    \end{multline*}
    Необходимо выполнение равенства:
    \begin{gather*}
        c^2 - c s - 2 s^2 = 0 , \\
        \left ( \frac{c}{s} \right )^2 - \frac{c}{s} - 2 = 0 , \\
        \frac{c}{s} = \frac{1 \pm \sqrt{1 + 4 \cdot 2}}{2} , \\
        \frac{c}{s} = \frac{1 \pm 3}{2} , \\
        \frac{c}{s} = \frac{1 - 3}{2} = \frac{-2}{2} = -1 .
    \end{gather*}
    Пусть
    \begin{gather*}
        c = \frac{1}{\sqrt{2}}, s = - \frac{1}{\sqrt{2}} , \\
        c = \cos \left ( - \frac{\pi}{4} \right ), s = \sin \left ( - \frac{\pi}{4} \right ) .
    \end{gather*}
    Таким образом, матрица $U$:
    \[
        U
        = \begin{pmatrix}
              \frac{1}{\sqrt{2}} & - \frac{1}{\sqrt{2}} \\
              \frac{1}{\sqrt{2}} & \frac{1}{\sqrt{2}}
        \end{pmatrix}.
    \]
    и матрица $R$:
    \[
        R
        = \begin{pmatrix}
              3 \cdot \frac{1}{2} + 3 \cdot \frac{1}{2} + 4 \cdot \frac{1}{2} & - \frac{1}{2} + 2 \cdot \frac{1}{2} + \frac{1}{2}               \\
              \frac{1}{2} - 2 \cdot \frac{1}{2} + \frac{1}{2}                 & 3 \cdot \frac{1}{2} - 3 \cdot \frac{1}{2} + 4 \cdot \frac{1}{2}
        \end{pmatrix}
        = \begin{pmatrix}
              5 & 1 \\
              0 & 2
        \end{pmatrix} .
    \]
\end{example}


\section{Спектральное разложение}

Пусть $A$ --- нормальная матрица:
\[
    A^* A = A A^* .
\]
Используя в последнем равенстве разложение \eqref{rayleigh:schur:decomposition}, получим:
\begin{align}
    \left ( U R U^* \right ) ^* U R U^* & = U R U^* \left ( U R U^* \right )^* , \notag \\
    U R^* U^* U R U^* & = U R U^* U R U^* , \notag \\
    U R^* R U^* & = U R R^* U^* , \notag \\
    U^* U R^* R U^* U & = U^* U R R^* U^* U, \notag \\
    R^* R & = R R^* \label{rayleigh:spectral:conjugation_equality}.
\end{align}
Поскольку $R$ --- верхнетреугольная матрица, то последнее равенство возможно только в том случае, когда $R$ --- диагональная:
\[
    R = \Lambda,
\]
где $\Lambda$ --- диагональная матрица.

\begin{example}
    Пусть матрица $R$ имеет порядок 3:
    \[
        R
        = \begin{pmatrix}
              r_{11} & r_{12} & r_{13} \\
              0      & r_{22} & r_{23} \\
              0      & 0      & r_{33}
        \end{pmatrix},
    \]
    тогда равенство \eqref{rayleigh:spectral:conjugation_equality} принимает вид:
    \[
        \begin{pmatrix}
            \overline{r_{11}} & 0                 & 0                 \\
            \overline{r_{12}} & \overline{r_{22}} & 0                 \\
            \overline{r_{13}} & \overline{r_{23}} & \overline{r_{33}}
        \end{pmatrix}
        \begin{pmatrix}
            r_{11} & r_{12} & r_{13} \\
            0      & r_{22} & r_{23} \\
            0      & 0      & r_{33}
        \end{pmatrix}
        =
        \begin{pmatrix}
            r_{11} & r_{12} & r_{13} \\
            0      & r_{22} & r_{23} \\
            0      & 0      & r_{33}
        \end{pmatrix}
        \begin{pmatrix}
            \overline{r_{11}} & 0                 & 0                 \\
            \overline{r_{12}} & \overline{r_{22}} & 0                 \\
            \overline{r_{13}} & \overline{r_{23}} & \overline{r_{33}}
        \end{pmatrix}
    \]
    Вычислим только два диагональных элемента в левой и правой частях:
    \[
        \begin{pmatrix}
            \modulus{r_{11}}^2 & \dots                                   & \dots \\
            \dots              & \modulus{r_{12}}^2 + \modulus{r_{22}}^2 & \dots \\
            \dots              & \dots                                   & \dots
        \end{pmatrix}
        =
        \begin{pmatrix}
            \modulus{r_{11}}^2 + \modulus{r_{12}}^2 + \modulus{r_{13}}^2 & \dots                                   & \dots \\
            \dots                                                        & \modulus{r_{22}}^2 + \modulus{r_{23}}^2 & \dots \\
            \dots                                                        & \dots                                   & \dots
        \end{pmatrix}
    \]
    Сравнивая диагональные элементы в левой и правой частях, получим равенства:
    \begin{align*}
        \modulus{r_{11}}^2 = \modulus{r_{11}}^2 + \modulus{r_{12}}^2 + \modulus{r_{13}}^2
        & \Rightarrow \modulus{r_{12}}^2 = 0, \modulus{r_{13}}^2 = 0
        \Rightarrow r_{12} = 0, r_{13} = 0 , \\
        %
        \modulus{r_{12}}^2 + \modulus{r_{22}}^2 = \modulus{r_{22}}^2 + \modulus{r_{23}}^2
        & \Rightarrow \modulus{r_{22}}^2 = \modulus{r_{22}}^2 + \modulus{r_{23}}^2
        \Rightarrow \modulus{r_{23}}^2 = 0
        \Rightarrow r_{23} = 0
    \end{align*}
    Таким образом внедиагональные элементы матрицы $R$ равны нулю.
\end{example}

Таким образом, если $A$ --- нормальная матрица, то она диагонализуема:
\begin{equation}
    \label{rayleigh:spectral:decomposition}
    A = U \Lambda U^* .
\end{equation}
Такое разложение называется спектральным, и называется так потому, что, умножая справа на $U$, получим равенство:
\[
    A U = U \Lambda,
\]
откуда следует, что столбцы матрицы $U$ определяют собственные векторы, а элементы матрицы $\Lambda$ являются собственными числами. Действительно, если представить матрицу $U$
в виде набора столбцов $u_i$:
\[
    U = \begin{pmatrix}
            u_1 & u_2 & \dots & u_n
    \end{pmatrix} ,
\]
тогда
\begin{gather*}
    A \begin{pmatrix}
          u_1 & u_2 & \dots & u_n
    \end{pmatrix}
    =
    \begin{pmatrix}
        u_1 & u_2 & \dots & u_n
    \end{pmatrix}
    \begin{pmatrix}
        \lambda_1 & 0         & \dots  & 0         \\
        0         & \lambda_2 & \dots  & 0         \\
        \vdots    & \vdots    & \ddots & \vdots    \\
        0         & 0         & \dots  & \lambda_n
    \end{pmatrix} , \\
    %
    \begin{pmatrix}
        A u_1 & A u_2 & \dots & A u_n
    \end{pmatrix}
    =
    \begin{pmatrix}
        \lambda_1 u_1 & \lambda_2 u_2 & \dots & \lambda_n u_n
    \end{pmatrix}
\end{gather*}
откуда следует, что
\[
    A u_i = \lambda_i u_i .
\]

\begin{example}
    Найти спектральное разложение для матрицы $A$:
    \[
        A
        = \begin{pmatrix}
              -7          & -5 \sqrt{3} \\
              -5 \sqrt{3} & 3
        \end{pmatrix}
    \]
    Находим собственные числа, с помощью характеристического полинома:
    \begin{multline*}
        \det \left ( A - \lambda E \right )
        = \begin{vmatrix}
              -7 - \lambda & -5 \sqrt{3} \\
              -5 \sqrt{3}  & 3 - \lambda
        \end{vmatrix}
        = (-7 - \lambda)(3 - \lambda) - 25 \cdot 3 = \\
        %
        = \lambda^2 + 4 \lambda - 21 - 75
        = \lambda^2 + 4 \lambda - 96.
    \end{multline*}
    Откуда корни характеристического полинома:
    \begin{gather*}
        \lambda_{1,2} = -2 \pm \sqrt{4 + 96} = -2 \pm 10 , \\
        \lambda_1 = -12, \lambda_2 = 8.
    \end{gather*}
    Таким образом, матрица $\Lambda$:
    \[
        \Lambda
        = \begin{pmatrix}
              -12 & 0 \\
              0   & 8
        \end{pmatrix}
    \]
    Столбцы матрицы $U$ являются собственными векторами, пусть $u_{ij}$ обозначают элементы матрицы $U$:
    \[
        U
        = \begin{pmatrix}
              u_{11} & u_{12} \\
              u_{21} & u_{22}
        \end{pmatrix} ,
    \]
    тогда
    \begin{gather*}
        A
        \begin{pmatrix}
            u_{11} \\
            u_{21}
        \end{pmatrix}
        = \lambda_1
        \begin{pmatrix}
            u_{11} \\
            u_{21}
        \end{pmatrix} , \\
        %
        \left ( A - \lambda_1 E \right )
        \begin{pmatrix}
            u_{11} \\
            u_{21}
        \end{pmatrix}
        = 0 , \\
        %
        \begin{pmatrix}
            -7 + 12     & -5 \sqrt{3} \\
            -5 \sqrt{3} & 3 + 12
        \end{pmatrix}
        \begin{pmatrix}
            u_{11} \\
            u_{21}
        \end{pmatrix}
        = 0 , \\
        %
        \begin{pmatrix}
            5           & -5 \sqrt{3} \\
            -5 \sqrt{3} & 15
        \end{pmatrix}
        \begin{pmatrix}
            u_{11} \\
            u_{21}
        \end{pmatrix}
        = 0 , \\
        \begin{pmatrix}
            1          & - \sqrt{3} \\
            - \sqrt{3} & 3
        \end{pmatrix}
        \begin{pmatrix}
            u_{11} \\
            u_{21}
        \end{pmatrix}
        = 0 , \\
        \begin{pmatrix}
            u_{11} \\
            u_{21}
        \end{pmatrix}
        = \begin{pmatrix}
              \sqrt{3} c \\
              c
        \end{pmatrix} ,
    \end{gather*}
    где постоянная $c$ выбирается из условия нормировки:
    \begin{gather*}
        u_{11}^2 + u_{21}^2 = 1 , \\
        3 c^2 + c^2 = 1 , \\
        4 c^2 = 1 , \\
        c = \pm \frac{1}{2} .
    \end{gather*}
    Таким образом:
    \[
        \begin{pmatrix}
            u_{11} \\
            u_{21}
        \end{pmatrix}
        = \pm \begin{pmatrix}
                  \frac{\sqrt{3}}{2} \\
                  \frac{1}{2}
        \end{pmatrix}
    \]
    Аналогично
    \begin{gather*}
        A
        \begin{pmatrix}
            u_{12} \\
            u_{22}
        \end{pmatrix}
        = \lambda_2
        \begin{pmatrix}
            u_{12} \\
            u_{22}
        \end{pmatrix} , \\
        %
        \left ( A - \lambda_2 E \right )
        \begin{pmatrix}
            u_{12} \\
            u_{22}
        \end{pmatrix}
        = 0 , \\
        %
        \begin{pmatrix}
            -7 - 8      & -5 \sqrt{3} \\
            -5 \sqrt{3} & 3 - 8
        \end{pmatrix}
        \begin{pmatrix}
            u_{12} \\
            u_{21}
        \end{pmatrix}
        = 0 , \\
        %
        \begin{pmatrix}
            -15         & -5 \sqrt{3} \\
            -5 \sqrt{3} & -5
        \end{pmatrix}
        \begin{pmatrix}
            u_{12} \\
            u_{22}
        \end{pmatrix}
        = 0 , \\
        \begin{pmatrix}
            3        & \sqrt{3} \\
            \sqrt{3} & 1
        \end{pmatrix}
        \begin{pmatrix}
            u_{12} \\
            u_{22}
        \end{pmatrix}
        = 0 , \\
        \begin{pmatrix}
            u_{12} \\
            u_{22}
        \end{pmatrix}
        = \begin{pmatrix}
              c \\
              - \sqrt{3}  c
        \end{pmatrix} ,
    \end{gather*}
    где постоянная $c$ выбирается из условия нормировки:
    \begin{gather*}
        u_{12}^2 + u_{12}^2 = 1 , \\
        c^2 + 3 c^2 = 1 , \\
        4 c^2 = 1 , \\
        c = \pm \frac{1}{2} .
    \end{gather*}
    Таким образом:
    \[
        \begin{pmatrix}
            u_{12} \\
            u_{22}
        \end{pmatrix}
        = \pm \begin{pmatrix}
                  \frac{1}{2} \\
                  - \frac{\sqrt{3}}{2}
        \end{pmatrix}
    \]
    и один из вариантов матрицы $U$:
    \[
        U
        = \begin{pmatrix}
              \frac{\sqrt{3}}{2} & \frac{1}{2}          \\
              \frac{1}{2}        & - \frac{\sqrt{3}}{2}
        \end{pmatrix} .
    \]
\end{example}


\section{Эрмитовые матрицы}

Пусть $A$ --- эрмитовая матрица:
\[
    A^* = A.
\]
Поскольку эрмитовые матрицы являются нормальными матрицами, то для матрицы $A$ cуществует спектральное разложение
\eqref{rayleigh:spectral:decomposition}, используя которое, получим:
\begin{align*}
    \left ( U \Lambda U^* \right )^* & = U \Lambda U^* , \\
    U \Lambda^* U^* & = U \Lambda U^* , \\
    U^* U \Lambda^* U^* U & = U^* U \Lambda U^* U, \\
    \Lambda^* & = \Lambda .
\end{align*}
Если $\lambda_i$ --- диагональный элемент матрицы $\Lambda$, тогда:
\[
    \overline{\lambda_i} = \lambda_i
\]
а это возможно тогда и только тогда, когда число $\lambda_i$ --- вещественное:
\[
    \lambda_i \in \mathbb{R}.
\]
Таким образом, у эрмитовой матрицы все собственные значения вещественные.

Рассмотрим квадратичную форму с эрмитовой матрицей $A$ и произвольным вектором $u$:
\[
    \overline{u^* A u}
    = \left ( u^* A u \right )^*
    = u^* A^* (u^*)^*
    = u^* A^* u
    = u^* A u .
\]
Отсюда следует, что значения квадратичной формы $u^* A u$ являются вещественными при всех векторах $u$. Пусть дополнительно оператор $A$
является неотрицательно определенным, то есть для всех векторов $u$:
\[
    u^* A u \ge 0
\]
Если $u_i$ --- собственный вектор, соответствующей собственному значению $\lambda_i$, тогда:
\begin{align*}
    u_i^* A u_i & \ge 0 , \\
    u_i^* \lambda_i u_i & \ge 0 , \\
    \lambda_i u_i^* u_i & \ge 0 , \\
    \lambda_i \scalarproduct{u_i}{u_i} & \ge 0 , \\
    \lambda_i \norm{u_i}^2 & \ge 0 , \\
    \lambda_i \ge 0 ,
\end{align*}
поскольку собственный вектор $u_i \neq 0$ и $\norm{u_i} > 0$. Таким образом, если $A \ge 0$, то все собственные числа неотрицательны, и для диагональной
матрицы $\Lambda$:
\[
    \Lambda
    = \begin{pmatrix}
          \lambda_1 & 0         & \dots  & 0         \\
          0         & \lambda_2 & \dots  & 0         \\
          \vdots    & \vdots    & \ddots & \vdots    \\
          0         & 0         & \dots  & \lambda_n
    \end{pmatrix}
\]
определен квадратный корень:
\begin{gather*}
    \Lambda = \Lambda^{\frac{1}{2}} \Lambda^{\frac{1}{2}} , \\
    %
    \Lambda^\frac{1}{2}
    = \begin{pmatrix}
          \lambda_1^\frac{1}{2} & 0                     & \dots  & 0                     \\
          0                     & \lambda_2^\frac{1}{2} & \dots  & 0                     \\
          \vdots                & \vdots                & \ddots & \vdots                \\
          0                     & 0                     & \dots  & \lambda_n^\frac{1}{2}
    \end{pmatrix}
    .
\end{gather*}
Тогда из спектрального разложения матрицы $A$ оператора \eqref{rayleigh:spectral:decomposition}:
\[
    A
    = U \Lambda U^*
    = U \Lambda^\frac{1}{2} \Lambda^\frac{1}{2} U^*
    = \left ( U \Lambda^\frac{1}{2} \right ) \left ( U \Lambda^\frac{1}{2} \right )^*
    = A^\frac{1}{2} \left ( A^\frac{1}{2} \right )^*,
\]
где
\[
    A^\frac{1}{2} = U \Lambda^\frac{1}{2}
\]
квадратный корень из эрмитовой матрицы $A$.

\begin{example}
    Пусть матрица $A$ имеет вид:
    \[
        A
        = \begin{pmatrix}
              43          & -7 \sqrt{3} \\
              -7 \sqrt{3} & 57
        \end{pmatrix} .
    \]
    Спектральное разложение матрицы $A$ имеет вид:
    \[
        A
        =
        \underbrace{
            \begin{pmatrix}
                \frac{\sqrt{3}}{2} & \frac{1}{2}         \\
                \frac{1}{2}        & -\frac{\sqrt{3}}{2}
            \end{pmatrix}
        }_{U}
        \underbrace{
            \begin{pmatrix}
                36 & 0  \\
                0  & 64
            \end{pmatrix}
        }_{\Lambda}
        \underbrace{
            \begin{pmatrix}
                \frac{\sqrt{3}}{2} & \frac{1}{2}         \\
                \frac{1}{2}        & -\frac{\sqrt{3}}{2}
            \end{pmatrix}
        }_{U^*} .
    \]
    Откуда квадратный корень
    \[
        A^\frac{1}{2}
        =
        \underbrace{
            \begin{pmatrix}
                \frac{\sqrt{3}}{2} & \frac{1}{2}         \\
                \frac{1}{2}        & -\frac{\sqrt{3}}{2}
            \end{pmatrix}
        }_{U}
        \underbrace{
            \begin{pmatrix}
                6 & 0 \\
                0 & 8
            \end{pmatrix}
        }_{\Lambda^\frac{1}{2}} .
    \]
\end{example}


\section{Вещественные квадратичные формы}

Если матрица $A$ определена ($A > 0$, $A \ge 0$, $A \le 0$ или $A < 0$), то при всех $x$ квадратичную форму $x^* A x$ можно сравнивать с нулём, а значит она является
вещественным числом. Отсюда сразу следует, что $A$ является эрмитовой. Действительно:
\begin{gather*}
    x^* A x \in \mathbb{R} , \\
    x^* A x = ( x^* A x )^* , \\
    x^* A x = x^* A^* x .
\end{gather*}

Возьмем в качестве $x$ векторы вида $e_k = (0, \dots, 0, 1, 0, \dots, 0)$ с одной единицей, тогда из равенства квадратичных форм следует, что диагональные
элементы матриц $A$ и $A^*$ вещественны и одинаковы:
\begin{gather*}
    e_k^* A e_k = e_k^* A^* e_k , \\
    a_{kk} = a_{kk}^* .
\end{gather*}
Возьмем в качестве $x$ векторы $e_{kj} = (0, \dots, 0, 1, 0, \dots, 0, 1, 0, \dots, 0)$ с двумя единицами, тогда из равенства квадратичных форм следует,
что внедиагональные элементы сопряжены:
\begin{gather*}
    e_{kj}^* A e_{kj} = e_{kj}^* A^* e_{kj} , \\
    a_{kk} + a_{kj} + a_{jk} + a_{jj} = a_{kk}^* + a_{jk}^* + a_{kj}^* + a_{jj}^* , \\
    a_{kk} + a_{kj} + a_{jk} + a_{jj} = a_{kk} + a_{jk}^* + a_{kj}^* + a_{jj} , \\
    a_{kj} + a_{jk} = a_{jk}^* + a_{kj}^* , \\
    a_{kj} + a_{jk} = a_{kj}^* + a_{jk}^* , \\
    a_{kj} + a_{jk} = ( a_{kj} + a_{jk} )^* , \\
    \image{a_{kj} + a_{jk}} = 0 , \\
    \image{a_{kj}} = - \image{a_{jk}} .
\end{gather*}
Возьмем в качестве $x$ векторы $e_{kj} = (0, \dots, 0, 1, 0, \dots, 0, -1, 0, \dots, 0)$ с двумя единицами, тогда из равенства квадратичных форм следует,
что внедиагональные элементы сопряжены:
\begin{gather*}
    e_{kj}^* A e_{kj} = e_{kj}^* A^* e_{kj} , \\
    a_{kk} + i a_{kj} - i a_{jk} - i^2 a_{jj} = a_{kk}^* + i a_{jk}^* - i a_{kj}^* - i^2 a_{jj}^* , \\
    a_{kk} + i a_{kj} - i a_{jk} + a_{jj} = a_{kk}^* + i a_{jk}^* - i a_{kj}^* + a_{jj}^* , \\
    a_{kk} + i a_{kj} - i a_{jk} + a_{jj} = a_{kk} + i a_{jk}^* - i a_{kj}^* + a_{jj} , \\
    i a_{kj} - i a_{jk} = i a_{jk}^* - i a_{kj}^* , \\
    i a_{kj} - i a_{jk} = ( - i a_{jk} + i a_{kj} )^* , \\
    i a_{kj} - i a_{jk} = ( i a_{kj} - i a_{jk})^* , \\
    \image{i a_{kj} - i a_{jk}} = 0 , \\
    \real{a_{kj} - a_{jk}} = 0 , \\
    \real{a_{kj}} = \real{a_{jk}} .
\end{gather*}
Таким образом,
\[
    A = A^*.
\]


\section{Экстремумы}~\label{rayleigh:extrema}

В радиолокации физические колебательные процессы часто описываются с помощью вектора комплексных амплитуд $x \in \Cspace{n}$. Кроме того, выделение линейной части
преобразований приводит к векторам $Fx$. Далее, обычно, интересуются энергией, которая пропорциональна квадратам норм:
\begin{gather*}
    \norm{x}^2 = x^* x , \\
    \norm{F x}^2 = x^* F^* F x ,
\end{gather*}
и сравнением энергий, которое приводит к отношениям вида:
\begin{gather*}
    \rho(x) = \frac{x^* F^* F x}{x^* x}.
\end{gather*}
Легко видеть, что матрица $F^* F$ является эрмитовой:
\[
    ( F^* F )^* = F^* F .
\]

В более общем случае рассматривается отношение:
\[
    \rho(x) = \frac{x^* A x}{x^* B x},
\]
где $A$ и $B$ --- эрмитовы матрицы и $B > 0$.

Заметим, что отношение Релея зависит только от направления вектора $x$, но не зависит от величины вектора $x$, действительно:
\begin{equation}
    \label{rayleight:extrema:homogenity}
    \rho(\alpha x)
    = \frac{\alpha^* x^* A \alpha x}{ \alpha^* x^* B \alpha x}
    = \frac{\modulus{\alpha}^2 \cdot x^* A x}{ \modulus{\alpha}^2 \cdot x^* B x}
    = \frac{x^* A x}{x^* B x}
    = \rho(x) ,
\end{equation}
поэтому при анализе значений отношения Релея можно ограничится рассмотрением векторов $x$ единичной нормы $\norm{x} = 1$.

Для положительно определенной матрицы $B > 0$ существует квадратный корень $B^\frac{1}{2}$:
\begin{gather*}
    B = B^\frac{1}{2} ( B^\frac{1}{2} )^* , \\
    %
    B^\frac{1}{2} = U_B \Lambda_B^\frac{1}{2} ,
\end{gather*}
причём
\[
    \det B^\frac{1}{2}
    = \det ( U_B \Lambda_B^\frac{1}{2} )
    = \det U_B \cdot \det \Lambda_B^\frac{1}{2}
    = 1 \cdot \det \Lambda_B^\frac{1}{2}
    > 0 ,
\]
поэтому существует обратная матрица $B^{-\frac{1}{2}}$:
\[
    B^{-\frac{1}{2}}
    = \left ( U_B \Lambda_B^\frac{1}{2} \right )^{-1}
    = \left ( \Lambda_B^\frac{1}{2} \right )^{-1} U_B^{-1}
    = \left ( \Lambda_B^\frac{1}{2} \right )^{-1} U_B^* .
\]

Отношение Релея $\rho(x)$ можно представить в виде:
\begin{gather*}
    \rho(x)
    = \frac{x^* A x}{x^* B^\frac{1}{2} ( B^\frac{1}{2} )^* x}
    = \frac{x^* B^\frac{1}{2} B^{-\frac{1}{2}} A ( B^{-\frac{1}{2}} )^* ( B^\frac{1}{2} )^* x}{x^* B^\frac{1}{2} ( B^\frac{1}{2} )^* x} , \\
    %
    \rho(y)
    = \frac{y^* B^{-\frac{1}{2}} A ( B^{-\frac{1}{2}} )^* y}{y^* y}
    = \frac{y^* C y}{y^* y}, \\
    %
    C = B^{-\frac{1}{2}} A ( B^{-\frac{1}{2}} )^*, \\
    %
    y = ( B^\frac{1}{2} )^* x .
\end{gather*}

Заметим, что матрица $C$ является эрмитовой:
\[
    C^*
    = ( B^{-\frac{1}{2}} A ( B^{-\frac{1}{2}} )^* )^*
    = B^{-\frac{1}{2}} A^* ( B^{-\frac{1}{2}} )^*
    = B^{-\frac{1}{2}} A ( B^{-\frac{1}{2}} )^*
    = C,
\]
поэтому она подобна диагональной матрице:
\[
    C = U_C \Lambda_C U_C^* .
\]
Используя это представление матрицы $C$, преобразуем отношение Релея к виду:
\begin{gather*}
    \rho(y)
    = \frac{y^* U_C \Lambda_C U_C^* y}{y^* y}
    = \frac{y^* U_C \Lambda_C U_C^* y}{y^* U_C U_C^* y} , \\
    %
    \rho(z)
    = \frac{z^* \Lambda_C z}{z^* z} , \\
    %
    z = U_C^* y
\end{gather*}

Согласно равенству \eqref{rayleight:extrema:homogenity} отношение Релея не зависит от величины вектора $z$, а только от его направления, поэтому можно ограничится рассмотрением
векторов $z$, для которых:
\begin{gather*}
    \norm{z} = 1 , \\
    %
    \rho(z) = z^* \Lambda_C z .
\end{gather*}

Пусть
\begin{gather*}
    \Lambda_C
    = \begin{pmatrix}
          \lambda_1 & 0         & \dots  & 0         \\
          0         & \lambda_2 & \dots  & 0         \\
          \vdots    & \vdots    & \ddots & \vdots    \\
          0         & 0         & \dots  & \lambda_n
    \end{pmatrix} , \\
    %
    \lambda_1 \ge \lambda_2 \ge \dots \ge \lambda_n.
\end{gather*}
тогда
\begin{gather*}
    \rho(z)
    = \lambda_1 \modulus{z_1}^2 + \lambda_2 \modulus{z_2}^2 + \dots + \lambda_n \modulus{z_n}^2, \\
    %
    \modulus{z_1}^2 + \modulus{z_2}^2 + \dots + \modulus{z_n}^2 = 1.
\end{gather*}
Из последнего равенства следует, что
\[
    0 \le \modulus{z_i}^2 \le 1 ,
\]
поэтому
\begin{gather*}
    \lambda_n \le \rho(z) \le \lambda_1 .
\end{gather*}

Максимальное значение отношение Релея достигает при векторе $z_{max}$:
\[
    z_{max}
    = \begin{pmatrix}
          1     \\
          0     \\
          \dots \\
          0
    \end{pmatrix} ,
\]
которому соответствует вектор $y_{max}$:
\begin{align*}
    z_{max} & = U_C^* y_{max} , \\
    U_C z_{max} & = y_{max} ,
\end{align*}
которому соответствует вектор $x_{max}$:
\begin{align*}
    \left ( B^\frac{1}{2} \right )^* x_{max} & = y_{max} = U_C z_{max} , \\
    \left ( U_B \Lambda_B^\frac{1}{2} \right )^* x_{max} & = U_C z_{max} , \\
    \Lambda_B^\frac{1}{2} U_B^* x_{max} & = U_C z_{max} , \\
    U_B^* x_{max} & = \Lambda_B^{-\frac{1}{2}} U_C z_{max} , \\
    x_{max} & = U_B \Lambda_B^{-\frac{1}{2}} U_C z_{max} .
\end{align*}
Аналогично минимальное значение отношение Релея достигает при векторе $z_{min}$:
\[
    z_{min}
    = \begin{pmatrix}
          0     \\
          \dots \\
          0     \\
          1
    \end{pmatrix} ,
\]
которому соответствует вектор $x_{min}$:
\[
    x_{min} = U_B \Lambda_B^{-\frac{1}{2}} U_C z_{min} .
\]


    % операции в Matlab
    \chapter{Операции в Matlab}

\section{Вычислительные}

\subsection{Арифметические}

Вектор--строка:
\matlab{x = [1 2 3]}

Вектор--столбец:
\matlab{y = [1; 2; 3]}

Матрица:
\matlab{A = [1 2 3; 4 5 6]}

Сложение:
\begin{Matlab}
    \Mcommand{B = [-2 -5 8; 3 1 -8]}
    \Mcommand{C = A + B}
\end{Matlab}

Умножение вектора на матрицу:
\matlab{z = A * y}

Сопряжение (транспонирование)
\matlab{Ac = A'}

Вычисление собственных чисел и векторов:
\begin{Matlab}
    \Mcommand{A = [-3 2; 3 -5];}
    \Mcommand{[V, D] = eig(A);}
\end{Matlab}

\subsection{Поэлементные}

Поэлементное умножение матриц:
\begin{Matlab}
    \Mcommand{A = [1 2 3; 4 5 6];}
    \Mcommand{B = [-2 -5 8; 3 1 -8];}
    \Mcommand{C = A .* B;}
\end{Matlab}

Поэлементное возведение в квадрат:
\begin{Matlab}
    \Mcommand{x = [1 3 5 7];}
    \Mcommand{y = x.$\hat{}$ 2;}
\end{Matlab}

\section{Графические}

График функции в 2D:
\begin{Matlab}
    \Mcommand{x = -1:0.1:1;}
    \Mcommand{y = x.$\hat{}$ 3;}
    \Mcommand{plot(x,y)}
\end{Matlab}

График кривой в 3D:
\begin{Matlab}
    \Mcommand{a = [ 0:0.1:2*pi 2*pi];}
    \Mcommand{x = cos(2*a);}
    \Mcommand{y = sin(2*a);}
    \Mcommand{z = a;}
\end{Matlab}

    % сигналы
    \chapter{Сигналы}


\section{Комплексное представление сигналов}

\subsection{Комплексные числа}

Комплексные числа $z = a + i b$ состоят из двух частей: действительной
\[
    a = \real{z},
\]
и мнимой
\[
    b = \image{z},
\]
где $a$ и $b$ --- действительные числа.

Комплексные числа можно представлять векторами на комплексной плоскости.

Если комплексное число не нулевое, то можно получить тригонометрическую форму:
\begin{gather*}
    z
    = a + i b
    = \sqrt{a^2 + b^2} \left ( \frac{a}{\sqrt{a^2 + b^2}} + i \frac{b}{\sqrt{a^2 + b^2}} \right )
    = A \left ( \cos \varphi + i \sin \varphi \right )
    = A \cos \varphi + i A \sin \varphi , \\
    %
    A = \sqrt{a^2 + b^2} , \\
    \cos \varphi = \frac{a}{\sqrt{a^2 + b^2}} , \\
    \sin \varphi = \frac{b}{\sqrt{a^2 + b^2}} ,
\end{gather*}
где $A$ и $\varphi$ --- модуль и аргумент комплексного числа.

\subsection{Комплексная экспонента}

Для вещественных чисел $a$ определена функция $e^a$, а её продолжением на комплексную плоскость является функция:
\[
    e^{a + i b}
    = e^a \cdot e^{i b}
    = e^a \cdot \left ( \cos b + i \sin b \right ).
\]

\subsection{Комплексное представление сигналов}

Пусть функция $u(t)$ описывает напряжение сигнала:
\[
    u(t) = A(t) \cos \varphi(t) ,
\]
где $A(t)$ --- амплитуда, $\varphi(t)$ --- фаза.

Функция $u(t)$ соответствует действительной части комплексно-значной функции $v(t)$ вещественного переменного $t$:
\begin{gather*}
    u(t) = \real{v(t)} , \\
    %
    v(t)
    = A(t) \cos \varphi(t) + i A(t) \sin \varphi(t)
    = A(t) \left ( \cos \varphi(t) + i \sin \varphi(t) \right )
    = A(t) e^{i \varphi(t)} .
\end{gather*}

\subsection{Сложение сигналов}

Пусть имеются сигналы, которые описываются функциями $u_1(t)$ и $u_2(t)$:
\begin{gather*}
    u_1(t) = A_1(t) \cos \varphi_1(t) , \\
    u_2(t) = A_2(t) \cos \varphi_2(t) ,
\end{gather*}
которым соответствую комплексные представления:
\begin{gather*}
    v_1(t) = A_1(t) e^{i \varphi_1(t)} , \\
    v_2(t) = A_2(t) e^{i \varphi_2(t)} .
\end{gather*}
тогда в результате сложения получается функция $u(t)$:
\[
    u(t)
    = u_1(t) + u_2(t)
    = \real{v_1(t)} + \real{v_2(t)}
    = \real{v_1(t) + v_2(t)} ,
\]
которой соответствует комплексное представление:
\[
    v(t) = v_1(t) + v_2(t).
\]

\subsection{Преобразование}

Пусть сигнал с функцией $u(t)$ и комплексным представлением $v(t)$:
\begin{gather*}
    u(t) = A(t) \cos \varphi(t) , \\
    v(t) = A(t) e^{i \varphi(t)}
\end{gather*}
преобразуется в некотором устройстве, и в результате преобразования изменяются амплитуда или фаза:
\[
    \widetilde{u}(t) = B(t) \cdot A(t) \cos ( \varphi(t) + \theta(t) ) ,
\]
тогда комплексное представление $\widetilde{v}(t)$ сигнала $\widetilde{u}(t)$ имеет вид:
\[
    \widetilde{v}(t)
    = A(t) \cdot B(t) e^{i (\varphi(t) + \theta(t))}
    = A(t) e^{i \varphi(t)} \cdot B(t) e^{i \theta(t)} = v(t) \cdot s(t),
\]
где функция $s(t)$ является комплексным представлением преобразования:
\[
    s(t) = B(t) e^{i \theta(t)} .
\]

\subsection{Комплексная огибающая}

Наиболее часто в радиолокации встречаются узкополосные сигналы, представляющие собой суперпозицию колебаний с частотами из узкой полосы частот, вокруг несущей
частоты $\omega$ (частота $\omega$ обычно составляет несколько мегагерц, поскольку антенны излучают сигналы высоких частот):
\[
    u(t) = A(t) \cos ( \omega t + \theta(t) ),
\]
где функции $A(t)$ и $\varphi(t)$ представляют модуляцию сигнала. Такому сигналу соответствует комплексное представление:
\[
    v(t)
    = A(t) e^{i ( \omega t + \theta(t) )}
    = A(t) e^{i \theta(t)} \cdot e^{i \omega t} .
\]
Первый множитель
\[
    v_s(t) = A(t) e^{i \theta(t)}
\]
является комплексной огибающей.

\subsection{Квадратурный детектор}

Сигнал $u(t)$ можно представить в виде суммы:
\[
    u(t)
    = A(t) \cos ( \omega t + \varphi(t) )
    = A(t) \cos \varphi(t) \cos \omega t - A(t) \sin \varphi(t) \sin \omega t .
\]
в которой множители
\begin{gather*}
    I(t) = A(t) \cos \varphi(t) , \\
    Q(t) = - A(t) \sin \varphi(t) ,
\end{gather*}
называются синфазной и квадратурной составляющими соответственно.

Составляющие $I(t)$ и $Q(t)$ определяют действительную и мнимую части комплексной огибающей $v(t)$:
\begin{gather*}
    I(t) = \real{v(t)} , \\
    Q(t) = - \image{v(t)} .
\end{gather*}

Устройство, которое выделяет составляющие $I(t)$ и $Q(t)$, называется квадратурным детектором.


    % излучение
    \chapter{Излучение}


\section{Двухканальный излучатель}

\subsection{Модель}

Рассматривается излучатель с двумя каналами поляризации и двумя входами, соответствующим этим двум каналам. Возникающие физические явления описываются только
линейными выражениями, а нелинейные эффекты не учитываются.

Пусть $a$ --- вектор огибающих входов излучателя:
\[
    a
    = \begin{pmatrix}
          a_1 \\
          a_2
    \end{pmatrix} .
\]

В результате отражений и связи между каналами, наводятся отраженные сигналы с огибающими $b$:
\begin{gather*}
    b
    = \begin{pmatrix}
          b_1 \\
          b_2
    \end{pmatrix}
    = \begin{pmatrix}
          s_{11} a_1 + s_{12} a_2 \\
          s_{21} a_1 + s_{22} a_2
    \end{pmatrix}
    = S a, \\
    %
    S
    = \begin{pmatrix}
          s_{11} & s_{12} \\
          s_{21} & s_{22}
    \end{pmatrix},
\end{gather*}
где $S$ --- матрица рассеяния.

        {
    \color{red}
    Может это исключить?

    Неотраженные части сигналов формируют электромагнитные волны, электрические колебания которых характеризуются огибающими $e$:
    \begin{gather*}
        e
        = \begin{pmatrix}
              e_\theta \\
              e_\varphi
        \end{pmatrix}
        = \begin{pmatrix}
              t_{1, \theta} a_1 + t_{2, \theta} a_2 \\
              t_{1, \varphi} a_1 + t_{2, \varphi} a_2
        \end{pmatrix}
        = T a, \\
        %
        T
        = \begin{pmatrix}
              t_{1, \theta}  & t_{2, \theta}  \\
              t_{1, \varphi} & t_{2, \varphi}
        \end{pmatrix} .
    \end{gather*}
    где $T$ --- матрица коэффициентов передачи со входов в волноводы.
}

\subsection{Диаграмма направленности}

Из волноводов происходит излучение электромагнитной волны в пространство, излучение является неоднородным и его характеристики зависят от рассматриваемого направления.
Если с излучателем связана сферическая система координат (с началом отсчёта в конце волноводов), то направление можно задать с помощью волнового вектора $w$,
длина которого:
\[
    \modulus{w} = \frac{2 \pi}{\lambda},
\]
где $\lambda$ --- длина волны на несущей частоте. В выбранном направлении $w$ на расстоянии $R$ напряженность электрического поля $E$ является функцией
направления $w$, расстояния $R$ и огибающих волн $e$ в волноводах излучателя, которые зависят от огибающих $a$ сигналов на входах:
\[
    E = E(w, R, a).
\]
В дальней зоне при больших расстояниях $R$ напряженность $E$ можно приближенно представить линейной частью по огибающим сигналов $a$:
\begin{equation}
    \label{emission:emitter:diagram:tension}
    E(w,R)
    \approx F(w) a \cdot \frac{e^{i \modulus{w} R}}{R} ,
\end{equation}
где множитель $e^{i \modulus{w} R}$ определяет смещение по фазе на расстоянии $R$ от начала отсчёта, множитель $\frac{1}{R}$ показывает затухание амплитуды
вектора напряженности $E$, и $F(w)$ --- диаграмма направленности излучателя:
\[
    F(w)
    = \begin{pmatrix}
          f_{1, \theta}(w)  & f_{2, \theta}(w)  \\
          f_{1, \varphi}(w) & f_{2, \varphi}(w)
    \end{pmatrix} ,
\]
в которой столбцы задают парциальные диаграммы направленности по каналам поляризации:
\begin{equation}
    f_1(w)
    = \begin{pmatrix}
          f_{1,\theta}(w) \\
          f_{1,\varphi}(w)
    \end{pmatrix}
    , \;
    f_2(w)
    = \begin{pmatrix}
          f_{2,\theta}(w) \\
          f_{2,\varphi}(w)
    \end{pmatrix}
    \label{emission:emitter:diagram:partial}
    .
\end{equation}


\subsection{Энергетическое ограничение}

Для излучателя выполняется закон сохранения энергии --- суммарная мощность входных сигналов $P_{inp}$ совпадает с суммой мощности отраженных сигналов $P_{ref}$,
мощности излучения $P_{rad}$ и мощности $P_{dis}$ дисспативных потерь:
\begin{equation}
    \label{emission:emitter:power:equality}
    P_{ref} + P_{rad} + P_{dis} = P_{inp} ,
\end{equation}
где
\begin{align}
    P_{inp} & = \norm{a}^2 = a^* a , \label{emission:emitter:power:input}\\
    P_{ref} & = \norm{b}^2 = \norm{S a} = a^* S^* S a \notag,
\end{align}
и выражение для мощности излучения $P_{rad}$ имеет вид:
\begin{gather*}
    P_{rad}
    = \iiint \limits_S \norm{E(w)}^2 ds
    = \iint \limits_{4 \pi} \norm{F(w) a}^2 d \Omega , \\
    %
    \norm{F(w) a}^2
    = a^* F^*(w) F(w) a .
\end{gather*}
Таким образом, мощность излучения $P_{rad}$ можно представить в виде квадратичной формы:
\begin{equation}
    \label{emission:emitter:power:radiated}
    P_{rad}
    = a^* Q a ,
\end{equation}
где элементами матрицы $Q$ являются интегралы вида:
\[
    Q_{jk} = \iint \limits_{4 \pi} f_j^*(w) f_k(w) d \Omega .
\]

Таким образом, равенство \eqref{emission:emitter:power:equality} можно представить в виде:
\begin{gather*}
    a^* S^* S a + a^* Q a + P_{dis} = a^* a , \\
    a^* Q a = a^* a - a^* S^* S a - P_{dis}, \\
    a^* Q a = a^* ( I - S^* S ) a - P_{dis} .
\end{gather*}
Откуда
\begin{gather*}
    \label{emission:emitter:power:inequality}
    a^* Q a \le a^* ( I - S^* S ) a \le a^* a.
\end{gather*}


\subsection{Коэффициент усиления}

Реализованный коэффициент усиления $G(w)$ определяет относительную величину плотности потока $\Pi(\theta, \varphi)$ мощности в дальней зоне в
направлении $w$:
\[
    G(w)
    = \frac{4 \pi R^2}{P_{inp}} \cdot \Pi(w)
    = \frac{4 \pi R^2}{P_{inp}} \cdot \frac{1}{Z_0} \norm{E(w)}^2
    = \frac{4 \pi R^2}{P_{inp}} \cdot \frac{1}{Z_0} \frac{\norm{F(w) a}^2}{R^2}
    = \frac{4 \pi}{Z_0} \cdot \frac{\norm{F(w) a}^2}{P_{inp}} ,
\]
где $Z_0 = 120 \pi$ --- волновое сопротивление свободного пространства.

Пусть направление $w$ фиксировано, тогда коэффициент усиления пропорционален отношению:
\[
    \rho(a)
    = \frac{\norm{F(w) a}^2}{P_{inp}}
    = \frac{a^* F^*(w) F(w) a}{a^* a}
\]
и возникает вопрос, каким образом нужно сформировать огибающие входных сигналов $a$ чтобы отношение $\rho(a)$ и коэффициент усиления в заданном направлении $w$
оказался наибольшим?

Отношение $\rho(a)$ является отношением Релея: числитель --- квадратичная форма с эрмитовой матрией $F^*(w) F(w)$, знаменатель --- квадрат нормы $a$.
Наибольшее значение $G_{max}$:
\[
    G_{max} = \max \limits_{a} G
\]
достигается в направлении $a_{max}$, совпадающим с направлением собственных векторов, соответствующих наибольшему собственному значению $\lambda_{max}$.
Вектор $a_{max}$ и число $\lambda_{max}$ удовлетворяют уравнению:
\[
    F^*(w) F(w) a_{max} = \lambda_{max} a_{max}
\]
Домножим левую и правую части уравнения на $F(w)$ слева:
\[
    F(w) F^*(w) F(w) a_{max} = \lambda_{max} F(w) a_{max} .
\]
Объединяя два уравнения получим систему:
\begin{gather}
    \left \{
    \begin{array}{c}
        F^*(w) F(w) a_{max} = \lambda_{max} a_{max} \\
        F(w) F^*(w) p_{max} = \lambda_{max} p_{max}
    \end{array}
    \right .
    \label{emission:emitter:gain:system}, \\
    p_{max} = F(w) a_{max} \notag.
\end{gather}

Пусть
\[
    B
    = F(w) F^*(w)
    = \begin{pmatrix}
          f_\theta f_\theta^*  & f_{\theta} f_\varphi^* \\
          f_\varphi f_\theta^* & f_\varphi f_\varphi^*
    \end{pmatrix} ,
\]
где $f_\theta$ и $f_\varphi$ --- строки матрицы $F(w)$:
\begin{gather*}
    f_\theta
    = \begin{pmatrix}
          f_{1,\theta}(w) & f_{2,\theta}(w)
    \end{pmatrix}, \\
    %
    f_\varphi
    = \begin{pmatrix}
          f_{1,\varphi}(w) & f_{2,\varphi}(w)
    \end{pmatrix} .
\end{gather*}

У матрицы $B$ два собственных значения $\lambda_{min}$ и $\lambda_{max}$, которые являются корнями характеристического уравнения:
\begin{multline*}
    \begin{vmatrix}
        f_\theta f_\theta^* - \lambda & f_{\theta} f_\varphi^*          \\
        f_\varphi f_\theta^*          & f_\varphi f_\varphi^* - \lambda
    \end{vmatrix}
    = (f_\theta f_\theta^* - \lambda) (f_\varphi f_\varphi^* - \lambda) - f_\varphi f_\theta^* f_{\theta} f_\varphi^* = \\
    %
    = \lambda^2 - ( f_\theta f_\theta^* + f_\varphi f_\varphi^* ) \lambda + f_\theta f_\theta^* f_\varphi f_\varphi^* - f_\varphi f_\theta^* f_{\theta} f_\varphi^* = \\
    %
    = \lambda^2 - \tr(B) \lambda + \det(B) ,
\end{multline*}
где $\tr(B)$ и $\det(B)$ --- след и определитель матрицы $B$. Корни характеристического уравнения:
\begin{align}
    \lambda_{min} & = \frac{\tr(B) - \sqrt{\tr^2(B) - 4 \det(B)}}{2} \label{emission:emitter:gain:minimum_eigenvalue} , \\
    \lambda_{max} & = \frac{\tr(B) + \sqrt{\tr^2(B) - 4 \det(B)}}{2} \label{emission:emitter:gain:maximum_eigenvalue}.
\end{align}
Вектор $p_{max}$ находим как решение второго уравнения системы \eqref{emission:emitter:gain:system}:
\begin{gather*}
    B p_{max} = \lambda_{max} p_{max} , \\
    ( B - \lambda I ) p_{max} = 0 .
\end{gather*}
Можно выбрать решение, для которого $\norm{p_{max}} = 1$. Далее вектор $a_{max}$ находим из первого уравнения системы \eqref{emission:emitter:gain:system}:
\begin{align*}
    F^*(w) F(w) a_{max}                         & = \lambda_{max} a_{max} , \\
    \frac{1}{\lambda_{max}} F^*(w) F(w) a_{max} & = a_{max} , \\
    \frac{1}{\lambda_{max}} F^*(w) p_{max}            & = a_{max} .
\end{align*}
Наибольший коэффициент усиления $G_{max}$:
\begin{multline*}
    G_{max}
    = \frac{a_{max}^* F^*(w) F(w) a_{max}}{a_{max}^* a_{max}} = \\
    %
    = \frac{p_{max}^* F(w) \left ( \frac{1}{\lambda_{max}} \right )^* F^*(w) F(w) \frac{1}{\lambda_{max}} F^*(w) p_{max}}{ p_{max}^* F(w) \left ( \frac{1}{\lambda_{max}} \right )^* \frac{1}{\lambda_{max}} F^*(w) p_{max}} = \\
    %
    = \frac{\modulus{\frac{1}{\lambda_{max}}}^2 p_{max}^* F(w) F^*(w) F(w) F^*(w) p_{max}}{ \modulus{\frac{1}{\lambda_{max}}}^2 p_{max}^* F(w) F^*(w) p_{max}}
    = \frac{\norm{F(w) F^*(w) p_{max}}^2}{ \norm{F^*(w) p_{max}}^2} .
\end{multline*}


\subsection{Коэффициент полезного действия}

\subsubsection{Для излучателя}

Коэффициент полезного действия показывает долю излучённой мощности:
\[
    \eta
    = \frac{P_{rad}}{P_{inp}}
    = \frac{a^* Q a}{a^* a}
\]
в силу равенств \eqref{emission:emitter:power:input} и \eqref{emission:emitter:power:radiated}. Коэффициент полезного действия $\eta$ является отношением Релея,
поэтому его значения ограничены наименьшим $\eta_{min}$ и наибольшим $\eta_{max}$ собственными значениями матрицы $Q$:
\[
    \eta_{min} \le \eta \le \eta_{max} .
\]
Поскольку $Q$ матрица порядка 2, то величины $\eta_{min}$ и $\eta_{max}$ можно найти согласно равенствам \eqref{emission:emitter:gain:minimum_eigenvalue} и
\eqref{emission:emitter:gain:maximum_eigenvalue} (с матрицей $Q$ вместо матрицы $B$).

\subsection{Для поляризации}

Представим элементы матрицы $Q$ в нормированном виде:
\[
    Q_{jk}
    =
    \sqrt{Q_{jj}}
    \cdot
    \frac{\iint \limits_{4 \pi} f_j^*(w) f_k(w) d \Omega}{\sqrt{Q_{jj}} \sqrt{Q_{kk}}}
    \cdot
    \sqrt{Q_{kk}} ,
\]
тогда
\[
    Q = \sqrt{D} R \sqrt{D} ,
\]
где
\[
    \sqrt{D}
    = \begin{pmatrix}
          \sqrt{Q_{11}} & 0             \\
          0             & \sqrt{Q_{22}} \\
    \end{pmatrix} ,
    \;
    %
    R_{jk} = \frac{Q_{jk}}{\sqrt{Q_{jj}} \sqrt{Q_{kk}}} .
\]

Согласно неравенству \eqref{emission:emitter:power:inequality}:
\begin{align*}
    a^* Q a & \le a^* a, \\
    a* \sqrt{D} R \sqrt{D} a & \le a^* a .
\end{align*}
Пусть $x = \sqrt{D} a$, тогда:
\begin{align*}
    x^* R x & \le x^* (\sqrt{D}^{-1})^* (\sqrt{D}^{-1}) x , \\
    x^* R x & \le x^* D^{-1} x .
\end{align*}
Пусть $r_{max}$ --- наибольшее собственное значение матрицы $R$ и $x_{max}$ --- соответствующий этому числу собственный вектор, а $Q_{min} = \min \{ Q_{11}, Q_{22} \}$,
тогда:
\begin{align*}
    x_{max}^* R x_{max} & \le x_{max}^* D^{-1} x_{max} , \\
    x_{max}^* r_{max} x_{max} & \le \sum_{k=1}^n \frac{1}{Q_{kk}} x_{max,k}^* x_{max,k} , \\
    r_{max} \norm{x_{max}}^2 & \le \sum_{k=1}^n \frac{1}{Q_{kk}} \modulus{x_{max,k}}^2 , \\
    r_{max} \norm{x_{max}}^2 & \le \sum_{k=1}^n \frac{1}{Q_{min}} \modulus{x_{max,k}}^2 , \\
    r_{max} \norm{x_{max}}^2 & \le \frac{1}{Q_{min}} \sum_{k=1}^n \modulus{x_{max,k}}^2 , \\
    r_{max} \norm{x_{max}}^2 & \le \frac{1}{Q_{min}} \norm{x_{max}}^2 , \\
    r_{max} & \le \frac{1}{Q_{min}} , \\
    Q_{min} & \le \frac{1}{r_{max}} .
\end{align*}

Матрица $R$ имеет вид:
\[
    R
    = \begin{pmatrix}
          1        & R_{12} \\
          R_{12}^* & 1
    \end{pmatrix} ,
\]
и наибольшее собственное значение матрицы $R$ согласно равенству \eqref{emission:emitter:gain:maximum_eigenvalue} имеет вид:
\begin{multline*}
    r_{max}
    = \frac{\tr(R) + \sqrt{\tr^2(R) - 4 \det(R)}}{2} = \\
    %
    = \frac{2 + \sqrt{2^2 - 4 (1 - \modulus{R_{12}}^2)}}{2}
    = \frac{2 + \sqrt{4 - 4 + 4 \modulus{R_{12}}^2)}}{2} = \\
    %
    = \frac{2 + 2 \modulus{R_{12}}}{2}
    = 1 + \modulus{R_{12}} .
\end{multline*}
Таким образом,
\[
    \min \{ Q_{11}, Q_{22} \} \le \frac{1}{1 + \modulus{R_{12}}} .
\]

Пусть в излучателе используется только первый канал поляризации, тогда:
\[
    a
    = \begin{pmatrix}
          \alpha \\
          0
    \end{pmatrix}
\]
и коэффициент полезного действия:
\[
    \eta_1
    = \frac{a^* Q a}{a^* a}
    = \frac{\alpha^* Q_{11} \alpha}{\alpha^* \alpha}
    = Q_{11}.
\]
Аналогично при работе только второго канала поляризации коэффициент полезного действия:
\[
    \eta_2 = Q_{22} .
\]
Таким образом, величина $Q_{jj}$ совпадает с коэффициентом полезного действия канала поляризации, и:
\[
    \min \{ \eta_1, \eta_2 \} \le \frac{1}{1 + \modulus{R_{12}}} .
\]

\section{Антенна}

\subsection{Модель антенны}

Рассматривается декартова система координат, в которой имеются $n$ излучателей, образующих антенну. Заданы местоположения излучателей $r_k$, и вектор огибающих
сигналов на входах излучателей:
\[
    a
    = \begin{pmatrix}
          a_1 \\\
          \dots \\\
          a_n
    \end{pmatrix} .
\]
Можно считать, что у каждого излучателя только один канал поляризации, в случае двух и более каналов необходимо в точку $r_k$ поместить ещё один или более
излучателей с другими поляризациями.

Напряженность электрического поля $E(w,R)$, создаваемого антенной, в дальней зоне будет приближённо равна:
\[
    E(w,R,a) = \widetilde{F}_a(w,a) \cdot \frac{e^{i \modulus{w} R}}{R} ,
\]
где $\widetilde{F}(w,a)$ --- диаграмма направленности антенны при огибающих $a$:
\[
    \widetilde{F}_a(w,a) = \sum_{k=1}^n f_k(w) a_k e^{i \scalarproduct{\vec{r}_k}{w}} ,
\]
где $f_k(w)$ --- парциальная диаграмма направленности $k$-го излучателя в составе антенны:
\[
    f_k(w) =
    \begin{pmatrix}
        f_{k,\theta}(w) \\
        f_{k,\varphi}(w)
    \end{pmatrix}
    ,
\]
и множитель $e^{i \scalarproduct{\vec{r}_k}{w}}$ соответствует смещению фазы при приведении напряжённостей поля излучателей к общему началу отсчёта.

Если парциальные диаграммы $f(w)$ и смещения фаз собрать в матрицу $f(w)$:
\begin{gather*}
    F_a(w) =
    \begin{pmatrix}
        f_{1,\theta}(w) e^{i \scalarproduct{\vec{r}_1}{w}}  & \dots & f_{n,\theta}(w) e^{i \scalarproduct{\vec{r}_n}{w}}  \\
        f_{1,\varphi}(w) e^{i \scalarproduct{\vec{r}_1}{w}} & \dots & f_{n,\varphi}(w) e^{i \scalarproduct{\vec{r}_n}{w}}
    \end{pmatrix} ,
\end{gather*}
тогда диаграмма направленности $\widetilde{F}_a(w,a)$ будет иметь вид:
\[
    \widetilde{F}_a(w,a) = F_a(w) a.
\]
Таким образом, напряженность имеет вид:
\[
    E = F_a(w) a \cdot \frac{e^{i \modulus{w} R}}{R} .
\]

\section{Диаграммообразующая схема}

Пусть $a_\alpha$ и $b_\alpha$ --- векторы огибающих сигналов сечения входа схемы и $a$ и $b$ --- векторы огибащих сигналов сечения выхода схемы. У схемы два входа ---
$a_\alpha$ и $b$ и два выхода --- $a$ и $b_\alpha$, которые связаны со входами:
\begin{gather}
    a        = S_{\beta \alpha} a_\alpha + S_{\beta \beta} b
    \label{emission:power:upper_output}, \\
    b_\alpha = S_{\alpha \alpha} a_\alpha + S_{\alpha \beta} b
    \label{emission:power:lower_output}
\end{gather}
Причем $a$ и $b$ ---  огибающие сигналов в сечении входа антенны, для которых справедливо равенство:
\begin{equation}
    \label{emission:power:antenna:reflections}
    b = S_\beta a ,
\end{equation}
где $S_\beta$ --- матрица рассеяния антенны.

Согласно равенствам \eqref{emission:power:upper_output} и \eqref{emission:power:antenna:reflections} вектор огибающих входа антенны $a$:
\begin{align}
    a & = S_{\beta \alpha} a_\alpha + S_{\beta \beta} b , \notag \\
    a & = S_{\beta \alpha} a_\alpha + S_{\beta \beta} S_\beta a , \notag \\
    a - S_{\beta \beta} S_\beta a & = S_{\beta \alpha} a_\alpha , \notag \\
    ( I - S_{\beta \beta} S_\beta ) a & = S_{\beta \alpha} a_\alpha , \notag \\
    a & = ( I - S_{\beta \beta} S_\beta )^{-1} S_{\beta \alpha} a_\alpha , \label{emission:power:antenna:input}
\end{align}
тогда линейная часть диаграммы направленности антенны с диаграммообразующей схемой:
\[
    \widetilde{F}_s(w, a_\alpha)
    = F_a(w) a(a_\alpha)
    = F_a(w) ( I - S_{\beta \beta} S_\beta )^{-1} S_{\beta \alpha} a_\alpha
    = F_s(w) a_\alpha,
\]
где
\[
    F_s(w) = F_a(\vec{w}) ( I - S_{\beta \beta} S_\beta )^{-1} S_{\beta \alpha} .
\]

Из равенств \eqref{emission:power:upper_output}, \eqref{emission:power:antenna:reflections} и \eqref{emission:power:antenna:input} вектор огибающих выхода
диаграммообразующей схемы:
\begin{align*}
    b_\alpha & = S_{\alpha \alpha} a_\alpha + S_{\alpha \beta} b , \\
    b_\alpha & = S_{\alpha \alpha} a_\alpha + S_{\alpha \beta} S a , \\
    b_\alpha & = S_{\alpha \alpha} a_\alpha + S_{\alpha \beta} S ( I - S_{\beta \beta} S )^{-1} S_{\beta \alpha} a_\alpha , \\
    b_\alpha & = ( S_{\alpha \alpha} + S_{\alpha \beta} S ( I - S_{\beta \beta} S )^{-1} S_{\beta \alpha} ) a_\alpha ,
\end{align*}
откуда матрица рассеяния для диаграммообразующей схемы:
\begin{equation}
    \label{emission:power:scheme:reflections}
    S_\alpha = S_{\alpha \alpha} + S_{\alpha \beta} S ( I - S_{\beta \beta} S )^{-1} S_{\beta \alpha} .
\end{equation}


\subsection{Рассеяние мощности с диаграммообразующей схемой}

Для диаграммообразующей схемы справедливо неравенство, аналогичное неравенству \eqref{emission:emitter:power:inequality}:
\begin{align}
    a_\alpha^* Q_\alpha a_\alpha & \le a_\alpha^* a_\alpha , \label{emission:power:scheme:inequality}
\end{align}
где $Q_\alpha$ --- матрица, составленная из элементов, характеризующих степень ортогональности лучей диаграммообразующей схемы:
\[
    Q_{\alpha,jk} = \frac{1}{4 \pi} \iint \limits_{4 \pi} g_{j}^*(\vec{w}) g_{k}(\vec{w}) d \Omega
\]
Введем величины нормы:
\[
    h_j = \frac{1}{4 \pi} \iint \limits_{4 \pi} g_{j}^*(\vec{w}) g_{j}(\vec{w}) d \Omega ,
\]
и преобразуем элементы матрицы $Q_\alpha$ к виду:
\[
    Q_{\alpha,jk}
    =
    \sqrt{h_j}
    \cdot
    \frac{\frac{1}{4 \pi} \iint \limits_{4 \pi} g_j^*(\vec{w}) g_k(\vec{w}) d \Omega}{\sqrt{h_j} \sqrt{h_k}}
    \cdot
    \sqrt{h_k} ,
\]
тогда
\[
    Q_\alpha = \sqrt{H} R \sqrt{H} ,
\]
где
\[
    \sqrt{H}
    = \begin{pmatrix}
          \sqrt{h_1} & 0          & 0          & \dots  & 0          \\
          0          & \sqrt{h_2} & 0          & \dots  & 0          \\
          0          & 0          & \sqrt{h_3} & \dots  & 0          \\
          \vdots     & \vdots     & \vdots     & \ddots & \vdots     \\
          0          & 0          & 0          & \dots  & \sqrt{h_n} \\
    \end{pmatrix} ,
    \;
    %
    R_{jk} = \frac{\frac{1}{4 \pi} \iint \limits_{4 \pi} g_j^*(\vec{w}) g_k(\vec{w}) d \Omega}{\sqrt{h_j} \sqrt{h_k}}
\]

Таким образом, неравенство \eqref{emission:power:scheme:inequality} имеет вид:
\[
    a_\alpha^* \sqrt{H} R \sqrt{H} a_\alpha \le a_\alpha^* a_\alpha ,
\]
Пусть $x = \sqrt{H} a_\alpha$, тогда:
\begin{align*}
    x^* R x & \le x^* (\sqrt{H}^{-1})^* (\sqrt{H}^{-1}) x , \\
    x^* R x & \le x^* H^{-1} x , \\
\end{align*}
Пусть $r_{max}$ --- наибольшее собственное значение матрицы $R$ и $x_{max}$ --- соответствующей этому числу собственный вектор, а $h_{min}$ --- наименьшее из
значений $h_k$, тогда:
\begin{align*}
    x_{max}^* R x_{max} & \le x_{max}^* H^{-1} x_{max} , \\
    x_{max}^* r_{max} x_{max} & \le \sum_{k=1}^n \frac{1}{h_k} x_{max,k}^* x_{max,k} , \\
    r_{max} \norm{x_{max}}^2 & \le \sum_{k=1}^n \frac{1}{h_k} \modulus{x_{max,k}}^2 , \\
    r_{max} \norm{x_{max}}^2 & \le \sum_{k=1}^n \frac{1}{h_{min}} \modulus{x_{max,k}}^2 , \\
    r_{max} \norm{x_{max}}^2 & \le \frac{1}{h_{min}} \sum_{k=1}^n \modulus{x_{max,k}}^2 , \\
    r_{max} \norm{x_{max}}^2 & \le \frac{1}{h_{min}} \norm{x_{max}}^2 , \\
    r_{max} & \le \frac{1}{h_{min}} , \\
    h_{min} & \le \frac{1}{r_{max}} .
\end{align*}


\subsection{Двухлучевая диаграммообразующая схема}

Пусть в диаграммообразующей схеме два входа, тогда матрица $R$ имеет вид:
\[
    R
    = \begin{pmatrix}
          1        & R_{12} \\
          R_{12}^* & 1
    \end{pmatrix} ,
\]
где
\[
    R_{12}
    =
    \frac
    {\frac{1}{4 \pi} \iint \limits_{4 \pi} g_1^*(\vec{w}) g_2(\vec{w}) d \Omega}
    {\sqrt{\frac{1}{4 \pi} \iint \limits_{4 \pi} g_1^*(\vec{w}) g_1(\vec{w}) d \Omega} \cdot \sqrt{\frac{1}{4 \pi} \iint \limits_{4 \pi} g_2^*(\vec{w}) g_2(\vec{w}) d \Omega}}
    = \frac
    {\iint \limits_{4 \pi} g_1^*(\vec{w}) g_2(\vec{w}) d \Omega}
    {\sqrt{\iint \limits_{4 \pi} \norm{g_1}^2 d \Omega} \cdot \sqrt{\frac{1}{4 \pi} \iint \limits_{4 \pi} \norm{g_2}^2 d \Omega}}
\]
и наибольшее собственное значение имеет вид:
\begin{multline*}
    r_{max}
    = \frac{\tr(R) + \sqrt{\tr^2(R) - 4 \det(R)}}{2} = \\
    %
    = \frac{2 + \sqrt{2^2 - 4 (1 - \modulus{R_{12}}^2)}}{2}
    = \frac{2 + \sqrt{4 - 4 + 4 \modulus{R_{12}}^2)}}{2} = \\
    %
    = \frac{2 + 2 \modulus{R_{12}}}{2}
    = 1 + \modulus{R_{12}} .
\end{multline*}
Таким образом,
\[
    h_{min} \le \frac{1}{1 + \modulus{R_{12}}} .
\]


    % волны
    \chapter{Волны}


\section{Плоская волна}

Будем использовать упрощённую модель электромагнитной волны, у которой фронт является прямой или плоскостью. Фронт волны --- геометрическое место точек с колебаниями в одной фазе.
Все точки на прямой или плоскости имеют одинаковую фазу колебания вектора напряжённости электрического поля.

\textcolor{red}{Рисунок волн.}

Зафиксируем декартову систему координат. В точке соответствующей началу отсчёта колебания имеют фазу:
\[
    \varphi_0(t) = \varphi_0 + \omega t.
\]
Точка начала отсчёта называется фазовым центром.

Наша цель заключается в расчёте фаз во всех других точках. Если через начало координат провести прямую параллельно фронту распространения волны, то во всех точках
этой прямой фаза будет такая же.

А что делать с остальными точками? Рассмотрим волновой вектор $\vec{w}$, который направлен в сторону распространения волны перпендикулярно фронту и имеет длину
\[
    \modulus{\vec{w}}
    = \frac{\omega}{v}
    = \frac{\omega \cdot T}{v \cdot T}
    = \frac{2 \pi}{\lambda} ,
\]
где $\omega$ --- угловая скорость колебаний, $v$ --- линейная скорость распространения волны, $T$ --- период колебаний, $\lambda$ --- длина волны (расстояние,
которое проходит волна за один период).

Проведём прямую через начало координат в направлении волнового вектора и рассмотрим изменение фазы колебаний в точках прямой. Если продвинуться на расстояние
$l$ по прямой в направлении волнового вектора $\vec{w}$ до точки $A$, то фаза изменится на $2 \pi \frac{l}{\lambda}$, а если продвинуться в обратную сторону
до точки $A^\prime$, то фаза изменится на $- 2 \pi \frac{l}{\lambda}$. Таким образом, если $l$ --- расстояние со знаком от точки прямой до начала отсчёта
(положительное направление отсчёта в направлении волнового вектора), то измение фазы $\Delta \varphi(l)$ будет равно:
\[
    \Delta \varphi(l)
    = 2 \pi \frac{l}{\lambda}
    = \frac{2 \pi} {\lambda} l
    = \modulus{\vec{w}} l .
\]
Таким образом, $\modulus{\vec{w}}$ является линейный коэффициентом изменения фазы. Если через точку $A$ провести прямую параллельную фронту распространения волны,
то все точки на этой прямой будут иметь такое же изменение фазы.

Теперь понятно как вычислить изменение фазы для любой точки $B$: необходимо через точку $B$ провести прямую параллельную фронту распространения волны, найти точку
пересечения с прямой проведённой через начало координат в направлении волнового вектора и вычислить расстояние $l$ от начала отсчёта до точки пересечения. Если
точка $B$ имеет радиус-вектор $\vec{r}$, то расстояние $l$ является проекцией вектора $\vec{r}$ на направление волнового вектора $\vec{w}$:
\[
    l = \scalarproduct{\vec{r}}{\frac{\vec{w}}{\modulus{\vec{w}}}}
\]
изменение фазы:
\[
    \Delta \varphi ( \vec{r} )
    = \modulus{\vec{w}} l
    = \modulus{\vec{w}} \scalarproduct{\vec{r}}{\frac{\vec{w}}{\modulus{\vec{w}}}}
    = \scalarproduct{\vec{r}}{\modulus{\vec{w}}  \frac{\vec{w}}{\modulus{\vec{w}}}}
    = \scalarproduct{\vec{r}}{\vec{w}}
\]
и фаза в точке $B$:
\[
    \varphi(t, \vec{r})
    = \varphi_0(t) + \Delta \varphi ( \vec{r} )
    = \varphi_0 + \omega t + \scalarproduct{\vec{r}}{\vec{w}}
    = \varphi_0 + \scalarproduct{\vec{r}}{\vec{w}} + \omega t .
\]


\section{Антенная решётка}

\subsection{Один приёмник}

Если приёмник поместить в среду с электромагнитными колебаниями, то приёмник будет выделять из общего электромагнитного фона только колебания из узкой полосы вокруг
несущей частоты $\omega$ и на выходе приёмника будет наблюдаться сигнал с комплексным представлением:
\[
    v_1(t) = A(t) e^{i \varphi(t)} \cdot e^{i \omega t} ,
\]
где $A(t) e^{i \varphi(t)}$ --- комплексная огибающая.

Если рядом с первым приёмником поместить второй приёмник, то на выходе второго приёмника тоже будет наблюдаться сигнал с некоторым комплексным представлением $v_2(t)$.
Какова функция $v_2(t)$? Оказывается функция $v_2(t)$ не является произвольной и связана с функцией сигнала первого приёмника $v_1(t)$ и эта взаимосвязь
функций определяется характером распространения волны на несущей частоте $\omega$.

\subsection{Два приёмника}

Пусть имеется плоская волна с волновым вектором $\vec{w}$ в некоторой декартовой системе координат, и в этой системе местоположение первого приёмника определяется
радиус-вектором $\vec{r}_1$, а второго --- радиус-вектором $\vec{r}_2$, тогда фазы колебаний в первом и втором приёмниках $\varphi_1(t)$ и $\varphi_2(t)$:
\begin{align*}
    \varphi_1(t) & = \varphi_0(t) + \scalarproduct{\vec{r}_1}{\vec{w}} , \\
    \varphi_2(t) & = \varphi_0(t) + \scalarproduct{\vec{r}_2}{\vec{w}} ,
\end{align*}
где $\varphi_0(t)$ --- фаза колебаний в фазовом центре --- точке начала декартовой системы координат.

В целях упростить выражения для фаз поместим фазовый центр в первый приёмник и направим ось абсцисс системы координат в направлении второго приёмника. В этом случае,
$\vec{r}_1$ = 0, поэтому:
\begin{align*}
    \varphi_1(t) & = \varphi_0(t) , \\
    \varphi_2(t) & = \varphi_0(t) + \scalarproduct{\vec{r}_2}{\vec{w}} = \varphi_1(t) + \scalarproduct{\vec{r}_2}{\vec{w}} .
\end{align*}

Пусть угол между осью ординат и волновым вектором равен $\alpha$, а длина $\modulus{\vec{r}_2} = d$, тогда
\begin{gather*}
    \scalarproduct{\vec{r}_2}{\vec{w}}
    = \modulus{\vec{r}_2} \modulus{\vec{w}} \cos \left ( \frac{\pi}{2} - \alpha \right )
    = d \frac{2 \pi}{\lambda} \sin \alpha
    = 2 \pi \frac{d}{\lambda} \sin \alpha ,
\end{gather*}
поэтому фаза колебаний второго приёмника:
\begin{gather*}
    \varphi_2(t) = \varphi_1(t) + \Delta \varphi , \\
    \Delta \varphi = 2 \pi \frac{d}{\lambda} \sin \alpha .
\end{gather*}

Если на выходе первого приёмника имеется сигнал $u_1(t)$:
\[
    u_1(t)
    = A \cos \left ( \varphi_1(t) \right )
    = A \cos \left ( \varphi_0 + \omega t \right )
\]
с комплексным представлением:
\[
    v_1(t)
    = A e^{i \varphi_0} \cdot e^{i \omega t} ,
\]
то на выходе второго приёмника будет сигнал $u_2(t)$:
\[
    u_2(t)
    = A \cos \left ( \varphi_1(t) + \Delta \varphi \right )
    = A \cos \left ( \varphi_0 + \omega t + \Delta \varphi \right )
\]
с комплексным представлением:
\[
    v_2(t)
    = A e^{i \varphi_0 + \Delta \varphi } \cdot e^{i \omega t} .
\]
Таким образом, комплексные огибающие $v_1$ и $v_2$ первого и второго приёмников
\begin{align*}
    s_1 & = A e^{i \varphi_0} , \\
    s_2 & = A e^{i \varphi_0 + \Delta \varphi} = A e^{i \varphi_0} \cdot e^{i \Delta \varphi} = s_1 \cdot e^{i \Delta \varphi}
\end{align*}

Если поместить фазовый центр в точку второго приёмника, тогда комплексные огибающие будут иметь вид:
\begin{align*}
    s_1 & = A e^{i \varphi_0 - \Delta \varphi} = A e^{i \varphi_0} \cdot e^{-i \Delta \varphi} = s_2 e^{-i \Delta \varphi}, \\
    s_2 & = A e^{i \varphi_0}.
\end{align*}

Если поместить фазовый центр в середине отрезка между приёмниками, тогда комплексные огибающие будут иметь вид:
\begin{align*}
    s_1 & = A e^{i \varphi_0 - \frac{\Delta \varphi}{2}}, \\
    s_2 & = A e^{i \varphi_0 + \frac{\Delta \varphi}{2}}.
\end{align*}

\subsection{Одномерная решётка}

Продолжим помещать приёмники на оси абсцисс через равные расстояния $d$ (расстояние между первым и вторым приёмниками) и получим эквидистантную антенную решётку.
Пусть приёмник с номером $k$ (нумерация в положительном направлении оси абсцисс) имеет радиус-вектор $\vec{r}_k$, тогда фаза $\varphi_k(t)$ у приёмника
с номером $k$:
\[
    \varphi_k(t) = \varphi_1(t) + \scalarproduct{\vec{r}_k}{\vec{w}},
\]
где длина вектора $\modulus{\vec{r}_k} = (k-1) d$, поэтому скалярное произведение
\[
    \scalarproduct{\vec{r}_k}{\vec{w}}
    = \modulus{\vec{r}_k} \modulus{\vec{w}} \cos \left ( \frac{\pi}{2} - \alpha \right )
    = (k-1) d \frac{2 \pi}{\lambda} \sin \alpha
    = (k-1) 2 \pi \frac{d}{\lambda} \sin \alpha
    = (k-1) \Delta \varphi,
\]
откуда на выходе $k$-го приёмника будет сигнал $u_k(t)$:
\[
    u_k(t)
    = A \cos \left ( \varphi_1(t) + (k-1) \Delta \varphi \right )
    = A \cos \left ( \varphi_0 + \omega t + (k-1) \Delta \varphi \right )
\]
с комплексным представлением:
\[
    v_k(t)
    = A e^{i \varphi_0 + (k-1) \Delta \varphi } \cdot e^{i \omega t} .
\]
и комплексной огибающей:
\[
    s_k
    = A e^{i \varphi_0 + (k-1) \Delta \varphi }
    = A e^{i \varphi_0 } \cdot e^{i (k-1) \Delta \varphi}
    = s_1 \cdot e^{i (k-1) \Delta \varphi} .
\]

\subsection{Двумерная решётка}

Итак, мы умеем вычислять фазу приёмника относительно заданного фазового центра в двумерном случае, когда фронт волны --- прямая. А что делать в трехмерном случае, когда фронт
волны --- плоскость.

Пусть $\vec{w}$ --- волновой вектор и $\vec{r}_k$ --- местоположение $k$-го приёмника относительно фазового центра. Проведём плоскость через фазовый центр и два вектора $\vec{w}$
и $\vec{r}_k$, получим двумерный случай, поэтому фаза $\varphi_k(t)$ для $k$-го приёмника:
\[
    \varphi_k(t) = \varphi_0(t) + \scalarproduct{\vec{r}_k}{\vec{w}} .
\]

Можно было бы ограничится полученным равенством, но обычно решётки имеют регулярную структуру. В простом случае прямоугольной решётки приёмники установлены с одинаковым шагом.
Поместим фазовый центр в угол прямоугольника, ось $X$ направим вдоль одной стороны прямоугольника, а ось $Y$ --- вдоль другой. Пусть шаг установки приёмников вдоль оси $X$
равен $d_x$, а вдоль оси $Y$ --- $d_y$, тогда вектор $\vec{r}_k$ раскладывается в сумму:
\[
    \vec{r}_k = x_k d_x \vec{u}_x + y_k d_y \vec{u}_y,
\]
где $\vec{u}_x$ --- орт оси $X$, $\vec{u}_y$ --- орт оси $Y$, величины $x_k$ --- количество шагов $d_x$ вдоль оси $X$ и $y_k$ --- количество шагов $d_y$ вдоль оси $Y$
определяют местоположение приёмника. Таким образом,
\begin{multline*}
    \varphi_k(t)
    = \varphi_0(t) + \scalarproduct{x_k d_x \vec{u}_x + y_k d_y \vec{u}_y}{\vec{w}}
    = \varphi_0(t) + \scalarproduct{x_k d_x \vec{u}_x}{\vec{w}} + \scalarproduct{y_k d_y \vec{u}_y}{\vec{w}} = \\
    %
    = \varphi_0(t) + x_k d_x \scalarproduct{\vec{u}_x}{\vec{w}} + y_k d_y \scalarproduct{\vec{u}_y}{\vec{w}} = \\
    %
    = \varphi_0(t) + x_k d_x \modulus{u_x} \modulus{\vec{w}} \cos \widetilde{\alpha}_x + y_k d_y \modulus{u_y} \modulus{\vec{w}} \cos \widetilde{\alpha}_y = \\
    %
    = \varphi_0(t) + x_k d_x \frac{2 \pi}{\lambda} \cos \widetilde{\alpha}_x + y_k d_y \frac{2 \pi}{\lambda} \cos \widetilde{\alpha}_y = \\
    %
    = \varphi_0(t) + x_k 2 \pi \frac{d_x}{\lambda} \cos \widetilde{\alpha}_x + y_k 2 \pi \frac{d_y}{\lambda} \cos \widetilde{\alpha}_y,
\end{multline*}
где $\widetilde{\alpha}_x$ --- угол между осью $X$ и вектором $\vec{w}$, $\widetilde{\alpha}_y$ --- угол между осью $Y$ и вектором $\vec{w}$. Используя дополнительные углы:
\begin{align*}
    \alpha_x & = \frac{\pi}{2} - \widetilde{\alpha}_x, \\
    \alpha_y & = \frac{\pi}{2} - \widetilde{\alpha}_y,
\end{align*}
можно представить фазу $\varphi_k(t)$ в виде:
\[
    \varphi_k(t)
    = \varphi_0(t) + x_k 2 \pi \frac{d_x}{\lambda} \sin \alpha_x + y_k 2 \pi \frac{d_y}{\lambda} \sin \alpha_y .
\]

\textcolor{red}{Непонятно почему выбирают синусы.}

    % помехи
    \chapter{Помехи}


\section{Обнаружение и пеленгация источников}

\subsection{\textcolor{red}{Задачи}}

Рассматривается антенная решётка, образованная некоторым количеством приёмников, состояния которых описываются комплексными огибающими.

При отсутствии источников излучения комплексные огибающие приёмников определяются внутренними шумами, которые носят случайный и хаотичный характер.

При появлении источника излучения возникает упорядоченность в наборе комплексных огибающих приёмников. Воздействие источника излучения на каждый приёмник
в отдельности является случайным, но воздействие источника излучения на совокупность приёмников имеет вполне регулярный характер, который
определяется местоположением источника излучения.

Конечно, собственные шумы приёмников не исчезают при появлении источника излучения, поэтому комплексные огибающие приёмников определяются
суперпозицией собственных шумов и воздействием источника излучения, а при наличии нескольких источников излучения, суммарным воздействием всех источников излучения.

Если дополнительно совокупность приёмников выполняет приём отраженного сигнала, то к комплексным огибающим приёмников дополнительно примешивается и
воздействие отраженного сигнала, которое так же имеет случайный, но направленный характер.

Анализируя комплексные огибающие всех приёмников, необходимо решить следующие задачи:
\begin{itemize}
    \item обнаружения - определить наличие источников излучения или оценить их количество,
    \item пеленгации - определить направления на источники излучения,
    \item адаптации - уменьшить воздействие источников излучения.
\end{itemize}

\subsection{Приёмники}

Рассмотрим антенную решётку, состоящую из $n$ приемников, расположенных на одной прямой через равные расстояния (эквидистантная решётка). Состояние каждого
приёмника в фиксированный момент времени определяется комплексной огибающей $x_k$, а состояние всей антенной решётки определяется вектором $X$:
\[
    X =
    \begin{pmatrix}
        x_1   \\
        \dots \\
        x_n
    \end{pmatrix}
    .
\]

\subsection{Отсутствие источников излучения}

Если источники излучения отсутствуют, то комплексные огибающие приёмников определяются их внутренними шумами:
\[
    X = E,
\]
где
\[
    E =
    \begin{pmatrix}
        e_1   \\
        \dots \\
        e_n
    \end{pmatrix}
\]
--- случайный вектор, компоненты которого $e_k$ --- комплексные случайные величины:
\begin{gather*}
    \expectation{
        \begin{pmatrix}
            \real{e_k} \\ \image{e_k}
        \end{pmatrix}
    } =
    \begin{pmatrix}
        0 \\
        0
    \end{pmatrix} , \\
    %
    \variance{
        \begin{pmatrix}
            \real{e_k} \\ \image{e_k}
        \end{pmatrix}
    } =
    \begin{pmatrix}
        \frac{1}{2} \sigma_0^2 & 0                      \\
        0                      & \frac{1}{2} \sigma_0^2
    \end{pmatrix} ,
\end{gather*}
где $\sigma_0^2$ --- мощность собственных шумов. Отсюда математическое ожидание
\[
    \expectation{e_k}
    = \expectation{\real{e_k} + i \image{e_k}}
    = \expectation{\real{e_k}} + i \expectation{\image{e_k}}
    = 0 + i \cdot 0
    = 0
\]
и дисперсия
\begin{multline*}
    \variance{e_k}
    = \expectation{\left ( e_k - \expectation{e_k} \right ) \left ( e_k - \expectation{e_k} \right )^*}
    = \expectation{e_k e_k^*} = \\
    %
    = \expectation{\left ( \real{e_k} + i \image{e_k} \right ) \left ( \real{e_k} - i \image{e_k} \right )}
    = \expectation{\left ( \real{e_k} \right )^2 + \left ( \image{e_k} \right )^2} = \\
    %
    = \expectation{\left ( \real{e_k} \right )^2} + \expectation{\left ( \image{e_k} \right )^2}
    = \frac{1}{2} \sigma_0^2 + \frac{1}{2} \sigma_0^2
    = \sigma_0^2 .
\end{multline*}
Таким образом, для вектора $E$ математическое ожидание:
\[
    \expectation{E}
    = \begin{pmatrix}
          0     \\
          \dots \\
          0
    \end{pmatrix} .
\]
Будем считать, что величины $e_k$ некоррелированы, тогда ковариационная матрица
\[
    \variance{E}
    = \expectation{\left ( E - \expectation{E} \right ) \left ( E - \expectation{E} \right )^*}
    = \expectation{E E^*}
    = \begin{pmatrix}
          \sigma_0^2 & 0          & \dots  & 0          \\
          0          & \sigma_0^2 & \dots  & 0          \\
          \vdots     & \vdots     & \ddots & \vdots     \\
          0          & 0          & \dots  & \sigma_0^2
    \end{pmatrix}
    = \sigma_0^2 I_n ,
\]
где $I_n$ --- единичная матрица.

Вектор $X$ имеет такие же характеристики, что и вектор $E$:
\begin{gather*}
    \expectation{X} = 0, \\
    \variance{X} = \sigma_0^2 I_n ,
\end{gather*}
при этом спектр ковариационной матрицы состоит из одного значения:
\[
    \spectrum{\variance{X}} = \set{\sigma_0^2}.
\]

\subsection{Один источник излучения}

\subsubsection{Состояние приёмников}

Пусть с некоторого направления производится излучение сигнала одним источником излучения, и на приёмники падает плоская волна. Пусть $\alpha$
угол между осью ординат и волновым вектором, $\lambda$ --- длина волны сигнала источника излучения и $d$ --- шаг расстановки приёмников, тогда расстояние между
приёмником с номером $k$ и первым приёмником равно $(k-1) d$, а смещение фазы в $k$-ом приёмнике:
\begin{gather}
    \Delta \varphi_k
    = (k-1) \cdot \Delta \varphi, \notag \\
    %
    \Delta \varphi
    = 2 \pi \frac{d}{\lambda} \sin \alpha
    \label{jammers:single:phase_shift}
\end{gather}
где $\Delta \varphi$ --- изменение фазы между соседними приёмниками.

Пусть в первом приёмнике комплексная огибающая сигнала источника излучения равна $s_1$ (в некоторый фиксированный момент времени), тогда из-за смещения фазы в
$k$-ом приёмнике на величину $(k-1) \Delta \varphi$, комплексная огибающая сигнала источника излучения будет равна $s_1 e^{i (k-1) \Delta \varphi}$. Если все
множители $e^{i (k-1) \Delta \varphi}$ собрать в вектор $\breve{X}_1$:
\begin{equation}
    \label{jammers:single:direction}
    \breve{X}_1
    =
    \begin{pmatrix}
        1                      \\
        e^{i \Delta \varphi}   \\
        e^{i 2 \Delta \varphi} \\
        \dots                  \\
        e^{i (n-1) \Delta \varphi}
    \end{pmatrix} ,
\end{equation}
тогда вектор комплексных огибающих сигнала источника излучения для всех приёмников будет иметь простой вид --- $s_1 \breve{X}_1$.

Поскольку на принимаемый сигнал источника излучения накладываются собственные шумы приёмников, то вектор комплексных огибающих приёмников будет иметь вид суммы:
\begin{equation}
    \label{jammers:single:state}
    X = E + \breve{X}_1 s_1.
\end{equation}
В произведении $\breve{X}_1 s_1$ можно выделить регулярную составляющую в виде неслучайного вектора $\breve{X}_1$, который определяется направлением на источник излучения,
и случайную составляющую в виде комплексной огибающей $s_1$. Таким образом, при наличии источника излучения в состоянии $X$:
\begin{enumerate}
    \item появляется регулярная структура, задаваемая вектором $\breve{X}_1$, которая определяется местоположением источника излучения,
    \item возникает смещение на случайную величину $s_1$.
\end{enumerate}
Комплексная огибающая $s_1$ является комплексной случайной величиной, для которой:
\begin{gather*}
    \expectation{
        \begin{pmatrix}
            \real{s_1} \\ \image{s_1}
        \end{pmatrix}
    } =
    \begin{pmatrix}
        0 \\
        0
    \end{pmatrix} , \\
    %
    \variance{
        \begin{pmatrix}
            \real{s_1} \\ \image{s_1}
        \end{pmatrix}
    } =
    \begin{pmatrix}
        \frac{1}{2} \sigma_1^2 & 0                      \\
        0                      & \frac{1}{2} \sigma_1^2
    \end{pmatrix} .
\end{gather*}
Дополнительно считается, что $s_1$ некоррелированна с величинами комплексных огибающих собственных шумов $e_k$:
\begin{gather*}
    \covariance{s_1}{e_l} = 0 , \\
    l = \overline{1, n}.
\end{gather*}

Характеристики состояния $X$ изменяются при наличии источника излучения и, в соответствии с представлением \eqref{jammers:single:state}, математическое ожидание
$X$:
\[
    \expectation{X}
    = \expectation{E} + \breve{X}_k \expectation{s_k}
    = 0 + \breve{X}_k \cdot 0
    = 0,
\]
и ковариационная матрица $X$:
\begin{multline*}
    \variance{X}
    = \expectation{\left ( X - \expectation{X} \right ) \left ( X - \expectation{X} \right )^*}
    = \expectation{X X^*} = \\
    %
    = \expectation{\left ( E + \breve{X}_1 s_1 \right ) \left ( E + \breve{X}_1 s_1 \right )^*} = \\
    %
    = \expectation{E E^* + E \left ( \breve{X}_1 s_1 \right ) + \left ( \breve{X}_1 s_1 \right ) E^* + \left ( \breve{X}_1 s_1 \right ) \left ( \breve{X}_1 s_1 \right )^*} = \\
    %
    = \expectation{E E^*} + \breve{X}_1 \expectation{E s_1} + \breve{X}_1 \expectation{s_1 E^*} + \breve{X}_1 \expectation{s_1 \overline{s}_1} \breve{X}_1^* = \\
    %
    = \sigma_0^2 I_n + 0 + 0 + \expectation{\modulus{s_1}^2} \breve{X}_1 \breve{X}_1^*
    = \sigma_0^2 I_n + \sigma_1^2 \breve{X}_1 \breve{X}_1^*
    .
\end{multline*}

\subsubsection{Обнаружение и пеленгация}

Заметим, что при появлении источника излучения в ковариационной матрице $\variance{X}$ появилось слагаемое $\sigma_1^2 \breve{X}_1 \breve{X}_1^*$:
\[
    \sigma_0^2 I_n \rightarrow \sigma_0^2 I_n + \sigma_1^2 \breve{X}_1 \breve{X}_1^* ,
\]
которое изменило спектр матрицы и набор собственных векторов. Одним из собственных векторов является вектор направления $\breve{X}_1$:
\begin{multline*}
    \variance{X} \breve{X}_1
    = \left ( \sigma_0^2 I_n + \sigma_1^2 \breve{X}_1 \breve{X}_1^* \right ) \breve{X}_1
    = \sigma_0^2 I_n \breve{X}_1 + \sigma_1^2 \breve{X}_1 \breve{X}_1^* \breve{X}_1 = \\
    %
    = \sigma_0^2 \breve{X}_1 + \sigma_1^2 \left ( \breve{X}_1^* \breve{X}_1 \right ) \breve{X}_1
    = \left ( \sigma_0^2 + \sigma_1^2 \breve{X}_1^* \breve{X}_1 \right ) \breve{X}_1 ,
\end{multline*}
где в соответствии с определением \eqref{jammers:single:direction} вектора направления $\breve{X}_1$:
\begin{equation}
    \label{jammers:single:direction_self_product}
    \breve{X}_1^* \breve{X}_1
    = \sum_{k=0}^{n-1} \overline{e^{i k \Delta \varphi}} \cdot e^{i k \Delta \varphi}
    = \sum_{k=0}^{n-1} \modulus{e^{i k \Delta \varphi}}^2
    = \sum_{k=0}^{n-1} 1
    = n ,
\end{equation}
поэтому
\[
    \variance{X} \breve{X}_1
    = \left ( \sigma_0^2 + \sigma_1^2 \breve{X}_1^* \breve{X}_1 \right ) \breve{X}_1
    = \left ( \sigma_0^2 + \sigma_1^2 n \right ) \breve{X}_1 .
\]
Таким образом, вектор $\breve{X}_1$ соответствует собственному значению $\sigma_0^2 + \sigma_1^2 n$.

Другим собственным значением является $\sigma_0^2$, поскольку для любого вектора $Y \perp \breve{X}_1$, то есть $\breve{X}_1^* Y = 0$:
\[
    \variance{X} Y
    = \left ( \sigma_0^2 I_n + \sigma_1^2 \breve{X}_1 \breve{X}_1^* \right ) Y
    = \sigma_0^2 I_n Y + \sigma_1^2 \breve{X}_1 \breve{X}_1^* Y
    = \sigma_0^2 I_n Y + \sigma_1^2 \breve{X}_1 \cdot 0
    = \sigma_0^2 Y .
\]
Таким образом, при наличии источника излучения спектр ковариационной матрицы $\variance{X}$:
\[
    \spectrum{\variance{X}} = \set{\sigma_0^2, \sigma_0^2 + \sigma_1^2 n}
\]

Если выполняется условие
\begin{equation}
    \label{jammers:single:powers_relation}
    \sigma_1^2 n \gg \sigma_0^2 ,
\end{equation}
то можно сформировать правило обнаружения:
\begin{enumerate}
    \item вычислить наибольшее собственное значение ковариационной матрицы $\variance{X}$:
    \[
        \sigma_0^2 + \sigma_1^2 n = \max \spectrum{\variance{X}}
    \]
    \item сравнить величины $\sigma_0^2 + \sigma_1^2 n$ и $\sigma_0^2$, если первая величина существенно больше второй, то принять решение о наличии источника
    излучения, в противном случае считать, что источник излучения отсутствует.
\end{enumerate}

Условие \eqref{jammers:single:powers_relation}, при котором появляется возможность формирования процедуры обнаружения, выполняется, если, например, мощность
сигнала источника излучения $\sigma_1^2$ существенно больше мощности собственных шумов $\sigma_0^2$:
\[
    \sigma_1^2 \gg \sigma_0^2 ,
\]
но даже если это условие не выполняется, то есть сигнал источника излучения имеет малую мощность, то можно набрать достаточно большое количество приёмников $n$
для выполнения условия \eqref{jammers:single:powers_relation}.

Для решения задачи пеленгации необходимо найти собственный вектор, соответствующий собственному значению $\sigma_0^2 + \sigma_1^2 n$, таким собственным вектором
является вектор $c \cdot \breve{X}_1$ ($c \in \mathbb{C}$):
\[
    c \cdot \breve{X}_1
    = \begin{pmatrix}
          c                            \\
          c \cdot e^{i \Delta \varphi} \\
          ...                          \\
    \end{pmatrix}
\]
Отношение второй компоненты к первой будет равно величине $e^{i \Delta \varphi}$, аргумент этой величины совпадает с разностью фаз, а разность фаз позволит вычислить угол $\alpha$
между нормалью решётки и направлением на источник излучения из равенства \eqref{jammers:single:phase_shift}:
\begin{gather}
    \Delta \varphi = \arg \left ( e^{i \Delta \varphi} \right ), \notag \\
    \sin \alpha = \frac{\Delta \varphi}{2 \pi} \cdot \frac{\lambda}{d} \label{jammers:single:angle}.
\end{gather}

\subsubsection{Пример}

Мощность источника излучения мала:
\matlab{detection(Receivers(5, 0.5, 1), [15 0.001])}

Мощность источник излучения в десять раз меньше собственных шумов:
\matlab{detection(Receivers(5, 0.5, 1), [15 0.1])}

Мощность источника излучения в четыре раза больше собственных шумов:
\matlab{detection(Receivers(5, 0.5, 1), [15 4])}

\subsection{Несколько источников излучения}

При наличии $m$ источников излучения, где $1 \le m < n$, будем считать, что направления действия источников различны: углы $\alpha_k$ между нормалью решётки и направлениями
на источники излучения являются различными, отсюда различными являются и смещения фаз $\Delta \varphi_k$.

\subsubsection{Состояние приёмников}

У каждого $k$-го источника свой угол $\alpha_k$, определяемый направлением на источник излучения, сдвиг фазы $\Delta \varphi_k$, комплексная огибающая $s_k$ и
вектор направления $\breve{X}_k$:
\[
    \breve{X}_k =
    \begin{pmatrix}
        1                        \\
        e^{i \Delta \varphi_k}   \\
        e^{i 2 \Delta \varphi_k} \\
        \dots                    \\
        e^{i (n-1) \Delta \varphi_k}
    \end{pmatrix}
\]
При одновременном приёме сигналов от всех источников в каждом $l$-ом приёмнике комплексные огибающие складываются и на сумму накладывается собственный шум приёмника:
\[
    e_l + s_1 e^{i l \Delta \varphi_1} + s_2 e^{i l \Delta \varphi_2} + \dots + s_m e^{i l \Delta \varphi_m}
\]
Общее состояние всех приёмников задается суммой:
\[
    X
    = E + s_1 \breve{X}_1 + \dots + s_m \breve{X}_m
    = E + \breve{X} S,
\]
где $\breve{X}$ --- матрица, столцы которой являются векторами направлений $\breve{X}_k$, и $s$ --- вектор, составленный из огибающих $s_k$:
\begin{gather*}
    \breve{X} =
    \begin{pmatrix}
        \breve{X}_1 & \breve{X}_2 & \dots & \breve{X}_m \\
    \end{pmatrix}, \\
    %
    S = \begin{pmatrix}
            s_1   \\
            s_2   \\
            \dots \\
            s_m
    \end{pmatrix} .
\end{gather*}
Комплексные огибающие $s_k$ являются случайными величинами:
\begin{gather*}
    \expectation{
        \begin{pmatrix}
            \real{s_k} \\
            \image{s_k}
        \end{pmatrix}
    } =
    \begin{pmatrix}
        0 \\
        0
    \end{pmatrix} , \\
    %
    \variance{
        \begin{pmatrix}
            \real{s_k} \\
            \image{s_k}
        \end{pmatrix}
    } =
    \begin{pmatrix}
        \frac{1}{2} \sigma_k^2 & 0                      \\
        0                      & \frac{1}{2} \sigma_k^2
    \end{pmatrix} ,
\end{gather*}
и величины $s_1$, \dots, $s_m$ считаются некоррелированными:
\begin{gather*}
    \covariance{s_k}{s_j} = 0 , \\
    \covariance{s_k}{e_l} = 0 , \\
    k,j = \overline{1,m}, k \neq j, \\
    l = \overline{1, n} .
\end{gather*}

Математическое ожидание вектора состояния приёмников $X$:
\[
    \expectation{X}
    = \expectation{E} + \breve{X} \expectation{S}
    = 0 + \breve{X} \cdot 0 .
\]

Ковариационная матрица $X$:
\begin{multline*}
    \variance{X}
    = \expectation{\left ( X - \expectation{X} \right ) \left ( X - \expectation{X} \right )^*}
    = \expectation{X X^*}
    = \expectation{\left ( E + \breve{X} S \right ) \left ( E + \breve{X} S \right )^*} = \\
    %
    = \expectation{E E^* + E S^* \breve{X}_k^* + \breve{X} S E^* + \breve{X} S S^* \breve{X}^*} = \\
    %
    = \expectation{E E^*} + \expectation{E S^*} \breve{X}^* + \breve{X} \expectation{S E^*} + \breve{X}_k \expectation{S S^*} \breve{X}_k^*
    = \sigma_0^2 I_n + \breve{X}_k \variance{S} \breve{X}_k^*
\end{multline*}
где
\begin{gather*}
    \variance{S} =
    \begin{pmatrix}
        \sigma_1^2 & 0          & \dots  & 0          \\
        0          & \sigma_2^2 & \dots  & 0          \\
        \vdots     & \vdots     & \ddots & \vdots     \\
        0          & 0          & \dots  & \sigma_m^2
    \end{pmatrix} .
\end{gather*}

\subsubsection{Обнаружение}

Как и в случае одного источника излучения ковариационная матрица $\variance{X}$ вектора состояния $X$ изменяется аналогичным образом:
\[
    \sigma_0^2 I_n \rightarrow \sigma_0^2 I_n + \breve{X} \variance{S} \breve{X}^*
\]
при этом также изменяется и спектр ковариационной матрицы $\variance{X}$.

Заметим, что векторы $\breve{X}_1$, \dots, $\breve{X}_m$ являются линейно независимыми, поскольку все сдвиги фаз $\Delta \varphi_k$ различны, поэтому ранг матрицы
$\breve{X} S \breve{X}^*$ равен количеству векторов $m$:
\[
    \rank{\breve{X} \variance{S} \breve{X}^*} = m
\]
и в спектре матрицы $\breve{X} S \breve{X}^*$ есть $m$ ненулевых собственных значений $\lambda_1$, \dots, $\lambda_m$ и нулевое значение, поскольку
$m < n$:
\begin{gather*}
    \spectrum{\breve{X} \variance{S} \breve{X}^*} = \set{0, \lambda_1, \dots, \lambda_m}, \\
    \lambda_k \neq 0, \\
    \lambda_1 \le \dots \le \lambda_m .
\end{gather*}
Поскольку матрица $\breve{X} \variance{S} \breve{X}^*$ является неотрицательно определённой:
\[
    \breve{X} \variance{S} \breve{X}^* \ge 0 ,
\]
то собственные значения также неотрицательны:
\[
    \lambda_k \ge 0,
\]
а поскольку собственные значения отличны от нуля, то:
\[
    0 < \lambda_1 \le \dots \le \lambda_m
\]
\textcolor{red}{и при этом различны}:
\[
    0 < \lambda_1 < \dots < \lambda_m .
\]

Добавление матрицы $\sigma_0^2 I_n$ приводит к смещению спектра на величину $\sigma_0^2$:
\[
    \spectrum{\variance{X}}
    = \spectrum{\sigma_0^2 I_n + \breve{X} \variance{S} \breve{X}^*}
    = \set{\sigma_0^2, \sigma_0^2 + \lambda_1, \dots, \sigma_0^2 + \lambda_m}
\]

\textcolor{red}{Как связаны величины $\lambda_1$, \dots, $\lambda_m$ и $\sigma_1^2$, \dots, $\sigma_m^2$?}

Поскольку $\lambda_1 > 0$, то можно сформулировать правило обнаружения источников излучения и определения их количества:
\begin{enumerate}
    \item вычислить наибольшее собственное значение $\lambda_{max}$ ковариационной матрицы $\variance{X}$,
    \item сравнить $\lambda_{max}$ с $\sigma_0^2$, если $\lambda_{max} > \sigma_0^2$, то принять решение о наличии источников излучения, в противном случае
    считать, что источники излучения осутствуют,
    \item при наличии источников излучения последовательно находить собственные значения $\sigma_0^2 + \lambda_k$ до тех пор, пока не будет получено собственное
    значение $\sigma_0^2$ для определения количества источников излучения.
\end{enumerate}

\subsubsection{Пеленгация}

Предварительно рассмотрим простой случай, в котором векторы направлений $\breve{X}_1$, \dots, $\breve{X}_m$ являются почти ортогональными:
\begin{gather*}
    \breve{X}_k^* \breve{X}_k = n, \\
    \breve{X}_k^* \breve{X}_j = \delta_{kj}, \\
    \modulus{\delta_{kj}} < \delta \ll n, \\
    k \neq j .
\end{gather*}
На практике векторы направлений $\breve{X}_k$ почти ортогональны, если направления на источники излучения в достаточной мере разнесены по углу, то есть являются
различимыми. Угловая мера различимости зависит от количества приёмников $n$.

Векторы направлений $\breve{X}_j$ являются близкими к собственным векторам ковариационной матрицы $\variance{X}$:
\begin{multline*}
    \variance{X} \breve{X}_j
    = \left ( \sigma_0^2 I_n + \breve{X} \variance{S} \breve{X}^* \right ) \breve{X_j}
    = \left ( \sigma_0^2 I_n + \sum_{k=1}^m \sigma_k^2 \breve{X}_k \breve{X}_k^* \right ) \breve{X}_j
    = \sigma_0^2 \breve{X}_j + \sum_{k=1}^m \sigma_k^2 \breve{X}_k \breve{X}_k^* \breve{X}_j = \\
    %
    = \sigma_0^2 \breve{X}_j + \left ( \sigma_j^2 \breve{X}_j^* \breve{X}_j \right ) \breve{X}_j + \sum_{k \neq j} \left ( \sigma_k^2 \breve{X}_k^* \breve{X}_j \right ) \breve{X}_k = \\
    %
    = \left ( \sigma_0^2 + \sigma_j^2 \breve{X}_j^* \breve{X}_j \right ) \breve{X}_j + \sum_{k \neq j} \left ( \sigma_k^2 \breve{X}_k^* \breve{X}_j \right ) \breve{X}_k
    = \left ( \sigma_0^2 + \sigma_j^2 n \right ) \breve{X}_j + \sum_{k \neq j} \sigma_k^2 \delta_{kj} \breve{X}_k ,
\end{multline*}
причём коэффициент при $\breve{X}_j$ растет пропорционально $n$, то есть с ростом $n$ вектор $\breve{X}_j$ будет приближаться к собственному, потому что сумма состоит из $m$
слагаемых и коэффициенты при $\breve{X}_k$ в сумме не растут с ростом $n$. Таким образом, можно находить собственные векторы ковариационной матрицы $\variance{X}$ и по каждому
из них определять направление, \textcolor{red}{хотя вполне возможно, что собственные векторы ковариационной матрицы $\variance{X}$ и не являются векторами направлений}.

Приближение может получатся не очень точным при недостаточно большом количестве приёмников:
\matlab{detection(Receivers(5, 0.5, 1), [-15 4; 30 7])}
\noindent но точность повышается с ростом количества приёмников $n$:
\matlab{detection(Receivers(1000, 0.5, 1), [-15 4; 30 7])}
\noindent хотя количество приёмников может оказаться очень большим, поэтому попробуем найти другой способ пеленгации.

Заметим, что квадратичная форма:
\begin{multline*}
    \breve{X}_j^* \variance{X} \breve{X}_j
    = \breve{X}_j^* \left ( \left ( \sigma_0^2 + \sigma_j^2 n \right ) \breve{X}_j + \sum_{k \neq j} \sigma_k^2 \delta_{kj} \breve{X}_k \right ) = \\
    %
    = \left ( \sigma_0^2 + \sigma_j^2 n \right ) \breve{X}_j^* \breve{X}_j + \sum_{k \neq j} \sigma_k^2 \delta_{kj} \breve{X}_j^* \breve{X}_k
    = \left ( \sigma_0^2 + \sigma_j^2 n \right ) n + \sum_{k \neq j} \sigma_k^2 \delta_{kj} \overline{\delta_{kj}} = \\
    %
    = \sigma_0^2 n + \sigma_j^2 n^2 + \sum_{k \neq j} \sigma_k^2 \modulus{\delta_{kj}}^2
    < \sigma_0^2 n + \sigma_j^2 n^2 + (m-1) \max \limits_{k \neq j} \sigma_k^2 \modulus{\delta}^2 .
\end{multline*}
\textcolor{blue}{Квадратичная форма растет пропорционально $n^2$ по отношению к фактору "неортогональности"{} векторов направлений.}

В некотором приближении можно считать, что
\begin{gather*}
    \breve{X}_j^* \variance{X} \breve{X}_j \approx \sigma_0^2 n + \sigma_j^2 n^2 , \\
    j = \overline{1,m} ,
\end{gather*}
и оказывается, что значения в правой части являются близкими к локальным максимумам квадратичной формы с
ковариационной матрицей $\variance{X}$. Если взять произвольный вектор направления $\breve{Y}(\Delta \varphi)$ как функцию смещения фазы $\Delta \varphi$:
\[
    \breve{Y}(\Delta \varphi) =
    \begin{pmatrix}
        1                      \\
        e^{i \Delta \varphi}   \\
        e^{i 2 \Delta \varphi} \\
        \dots                  \\
        e^{i (n-1) \Delta \varphi}
    \end{pmatrix} ,
\]
то квадратичная форма
\begin{multline}
    \label{jammers:multiple:quadric}
    \breve{Y}^* \variance{X} \breve{Y}
    = \breve{Y}^* \left ( \sigma_0^2 I_n + \breve{X} \variance{S} \breve{X}^* \right ) \breve{Y}
    = \breve{Y}^* \left ( \sigma_0^2 I_n + \sum_{k=1}^m \sigma_k^2 \breve{X}_k \breve{X}_k^* \right ) \breve{Y} = \\
    %
    = \sigma_0^2 \breve{Y}^* \breve{Y} + \sum_{k=1}^m \sigma_k^2 \breve{Y}^* \breve{X}_k \breve{X}_k^* \breve{Y}
    = \sigma_0^2 n + \sum_{k=1}^m \sigma_k^2 \breve{Y}^* \breve{X}_k \left ( \breve{Y}^* \breve{X}_k \right )^* = \\
    %
    = \sigma_0^2 n + \sum_{k=1}^m \sigma_k^2 \modulus{\breve{Y}^* \breve{X}_k}^2 ,
\end{multline}
где произведения
\begin{multline*}
    \breve{Y}^* \breve{X}_k
    =
    \begin{pmatrix}
        1                       &
        e^{- i \Delta \varphi}   &
        e^{- i 2 \Delta \varphi} &
        \dots                   &
        e^{- i (n-1) \Delta \varphi}
    \end{pmatrix}
    \begin{pmatrix}
        1                        \\
        e^{i \Delta \varphi_k}   \\
        e^{i 2 \Delta \varphi_k} \\
        \dots                    \\
        e^{i (n-1) \Delta \varphi_k}
    \end{pmatrix}
    = \\
    %
    = 1 + e^{i (\Delta \varphi_k - \Delta \varphi )} + e^{i 2 (\Delta \varphi_k - \Delta \varphi )} + \dots + e^{i (n-1) (\Delta \varphi_k - \Delta \varphi )} .
\end{multline*}
Представление о полученной сумме и её модуле можно получить, рассматривая сумму чисел вида $e^{i l (\Delta \varphi_k - \Delta \varphi )}$ на комплексной
плоскости: к 1 прибавляется единичный вектор $e^{i (\Delta \varphi_k - \Delta \varphi )}$, повёрнутый на угол $\Delta \varphi_k - \Delta \varphi$, соответствующий разности,
далее прибавляется единичный вектор, повёрнутый на два угла $\Delta \varphi_k - \Delta \varphi$, на три и так далее.

Заметим, что
\begin{multline*}
    \modulus{\breve{Y}^* \breve{X}_k}
    = \modulus{1 + e^{i (\Delta \varphi_k - \Delta \varphi )} + e^{i 2 (\Delta \varphi_k - \Delta \varphi )} + \dots + e^{i (n-1) (\Delta \varphi_k - \Delta \varphi )}} \le \\
    %
    \le 1 + \modulus{e^{i (\Delta \varphi_k - \Delta \varphi )}} + \modulus{e^{i 2 (\Delta \varphi_k - \Delta \varphi )}} + \dots + \modulus{e^{i (n-1) (\Delta \varphi_k - \Delta \varphi )}}
    = 1 + 1 + 1 + \dots + 1
    = n .
\end{multline*}
Если направления совпадают, $\breve{Y} = \breve{X}_k$, то фазы равны, $\Delta \varphi = \Delta \varphi_k$, тогда величина модуля:
\[
    \modulus{\breve{Y}^* \breve{X}_k}
    = \modulus{1 + e^{i \cdot 0} + e^{i \cdot 0} + \dots + e^{i \cdot 0}}
    = \modulus{1 + 1 + 1 + \dots + 1}
    = n .
\]
Таким образом, модуль $\modulus{\breve{Y}^* \breve{X}_k}$ достигает наибольшего значения, когда направления совпадают. Модуль $\modulus{Y^* X_k}$ как функция смещения фазы
$\Delta \varphi$, определяющей вектор $Y$, представляет собой осцилирующую "убывающую"{} функцию с пиком в точке $\Delta \varphi = \Delta \varphi_k$, соответствующей вектору
направления $\breve{X}_k$.

Рисунки функции \texttt{jammers/projection.m}:
\begin{Matlab}
    \Mcommand{projection(Receivers(5, 0.5, 1), 0)}
    \Mcommand{projection(Receivers(5, 0.5, 1), 15)}
    \Mcommand{projection(Receivers(5, 0.5, 1), 25)}
    \Mcommand{projection(Receivers(5, 0.5, 1), 50)}
    \Mcommand{projection(Receivers(5, 0.5, 1), 85)}
    \Mcommand{projection(Receivers(10, 0.5, 1), 25)}
    \Mcommand{projection(Receivers(10, 0.5, 1), 85)}
\end{Matlab}

Сумма в квадратичной форме \eqref{jammers:multiple:quadric} представляет собой суперпозицию функций-модулей $\modulus{Y^* X_k}$, которая в случае достаточного разделения векторов
направлений $\breve{X}_k$ имеет локальные максимумы в окрестности векторов направлений $\breve{X}_k$ и достигает значений, примерно равных $\sigma_0^2 n + \sigma_k^2 n^2$. Тем не менее,
направления локальных максимумов не обязательно совпадают с направлениями источников излучения.

\begin{Matlab}
    \Mcommand{quadric(Receivers(5, 0.5, 1), [-15 4; 30 7], "direct"{}, 1)}
\end{Matlab}
\textcolor{blue}{Рисунок с разделёнными локальными максимумами.}

Если некоторые векторы направлений $\breve{X}_k$ оказываются "близкими"{}, то возникает один локальный максимум квадратичной формы в окрестности "близких"{} векторов направлений.

\begin{Matlab}
    \Mcommand{quadric(Receivers(5, 0.5, 1), [-15 4; 15 7], "direct"{}, 1)}
    \Mcommand{quadric(Receivers(5, 0.5, 1), [-15 4; 10 7], "direct"{}, 1)}
    \Mcommand{quadric(Receivers(5, 0.5, 1), [-15 4; 5 7], "direct"{}, 1)}
    \Mcommand{quadric(Receivers(5, 0.5, 1), [-15 4; 0 7], "direct"{}, 1)}
\end{Matlab}
\textcolor{blue}{Рисунки с локальными максимумами.}

Однако,  при увеличении количества приёмников повышается "разрешение"{}.
\matlab{quadric(Receivers(10, 0.5, 1), [-15 4; 0 7], "direct"{}, 1)}

Таким образом, нахождение смещений фазы $\Delta \varphi$ и векторов направлений $Y$, при которых наблюдается локальный максимум квадратичной формы
$Y^* \variance{X} Y$, позволяет находить направления "близкие"{} к направлениям на источники излучений $\breve{X}_k$.

У данного метода пеленгации несколько проблем: различение близких источников излучения, смещение максимумов квадратичной формы относительно направлений на источники излучения
и точность определения направлений, которая некоторым образом связана с "шириной"{} графика квадратичной формы в окрестностях локальных максимумов. Ситуация улучшается при
увеличении количества приёмников, но требуемое количество может оказаться слишком большим.

\subsubsection{Альтернативная пеленгация}

Существует несколько альтернативных методов определения направлений на источники излучения, связанных с поиском экстремумов функций от направления.

Поскольку квадратичная форма с матрицей $\variance{X}$ имеет локальные максимумы в направлении источников излучения, то квадратичная форма с обратной матрицей $\variance{X}^{-1}$
имеет локальные минимумы в этих направлениях:
\[
    \breve{Y}^* \variance{X}^{-1} \breve{Y} \rightarrow min
\]

Нет различения близких источников:
\matlab{quadric(Receivers(5, 0.5, 1), [-15 4; 5 7], "direct"{}, 1)}

Есть различение близких источников
\matlab{quadric(Receivers(5, 0.5, 1), [-15 4; 5 7], "inverse"{}, 1)}

Можно рассматривать максимум обратной величины:
\[
    \frac{1}{\breve{Y}^* \variance{X}^{-1} \breve{Y}} \rightarrow max
\]
Уменьшение ширины в окрестности локального максимума:
\matlab{quadric(Receivers(5, 0.5, 1), [-15 4; 5 7], "inverse"{}, -1)}

\textcolor{red}{Здесь нужны пояснения. Почему так происходит?}


\section{Адаптация}

\subsection{Состояние приёмников}

Дополнительно к собственным шумам и источникам излучения добавляется полезный сигнал, соответствующий вектору направления $\breve{U}$. Состояние приёмников описывается
вектором комплексных огибающих:
\begin{gather*}
    Y = X + \breve{U} \cdot u ,
\end{gather*}
где $X = E + \breve{X} S$ --- комплексные огибающие собственных шумов и сигналов источников излучения, $u$ --- комплексная огибающая полезного сигнала.

\subsection{Критерий обработки}

Нужно разработать преобразование вектора $Y$, которое выделяет огибающую $u$ полезного сигнала, причём это преобразование должно быть простым. Одним из вариантов такого
преобразования является линейное преобразование $\mathcal{F}_W(Y)$:
\[
    \mathcal{F}_W(Y)
    = w_1^* y_1 + w_2^* y_2 + \dots + w_n^* y_n
    = W^* Y,
\]
где $W$ --- весовой вектор:
\[
    W =
    \begin{pmatrix}
        w_1   \\
        \dots \\
        w_n
    \end{pmatrix}.
\]
Таким образом,
\[
    \mathcal{F}_W(Y)
    = W^* \left ( X + \breve{U} \cdot z \right )
    = W^* X + W^* \breve{U} \cdot z
\]
Желательно чтобы
\begin{gather*}
    \modulus{W^* X} \ll \modulus{W^* \breve{U}} , \\
    \modulus{W^* X}^2 \ll \modulus{W^* \breve{U}}^2
\end{gather*}
Поскольку слева стоит случайная величина, то будем ориентироваться на её среднее значение:
\[
    \expectation{\modulus{W^* X}^2} \ll \modulus{W^* \breve{U}}^2 .
\]
Более точно, будем стараться выбирать вектор $W$ так, чтобы наибольшим было отношение $\rho$:
\[
    \rho ( W ) = \frac{\modulus{W^* \breve{U}}^2}{\expectation{\modulus{W^* X}^2}} .
\]
Отношение $\rho$ показывает отношение мощности полезного сигнала $\modulus{W^* \breve{U}}^2$ к средней мощности шумов и мешающих сигналов источников излучения $\modulus{W^* X}^2$:
\[
    \expectation{\modulus{W^* X}^2}
    = \expectation{W^* X X^* W}
    = W^* \expectation{X X^*} W
    = W^* \variance{X} W .
\]
Таким образом,
\[
    \rho ( W )
    = \frac{\modulus{W^* \breve{U}}^2}{W^* \variance{X} W}
    = \frac{W^* \breve{U} \breve{U}^* W}{W^* R W} ,
\]
где введено обозначение
\[
    R = \variance{X} .
\]
Необходимо найти оптимальный весовой вектор $W_{max}$, при котором отношение $\rho(W)$ достигает наибольшего значения.
\[
    \rho(W_{max}) = \max \limits_W \rho(W) .
\]

\subsection{Оптимальный весовой вектор}

Отношение $\rho(W)$ является отношением Релея, для которого известен способ нахождения наибольшего значения и направления $W_{max}$, в котором он достигается, рассмотренный
в разделе \ref{rayleigh:extrema}. Раскладываем ковариационную матрицу на произведение корней:
\[
    R = R^\frac{1}{2} \cdot \left( R^\frac{1}{2} \right)^*
\]
подставляем в отношение
\[
    \rho ( W )
    = \frac{W^* \breve{U} \breve{U}^* W}{W^* R^\frac{1}{2} \cdot \left( R^\frac{1}{2} \right)^* W}
\]
вводим новую переменную $Z$
\begin{align*}
    Z & = \left( R^\frac{1}{2} \right)^* W , \\
    \left( R^{-\frac{1}{2}} \right)^* Z & = W ,
\end{align*}
тогда отношение:
\begin{gather*}
    \rho(Z)
    = \frac{Z^* R^{-\frac{1}{2}} \breve{U} \breve{U}^* \left( R^{-\frac{1}{2}} \right)^* Z}{Z^* Z}
    = \frac{Z^* C Z}{Z^* Z}, \\
    %
    C
    = R^{-\frac{1}{2}} \breve{U} \breve{U}^* \left( R^{-\frac{1}{2}} \right)^*
    = R^{-\frac{1}{2}} \breve{U} \left( R^{-\frac{1}{2}} \breve{U} \right)^*.
\end{gather*}
Наибольшее значение отношение Релея $\rho(Z)$ равно наибольшему собственному значению матрицы $C$. Матрица $C$ является внешним произведением одного вектора $R^{-\frac{1}{2}} \breve{U}$,
поэтому имеет ранг 1, отсюда следует, что у матрицы $C$ есть только одно отличное от нуля собственное значение $\lambda_{max}$, и этому собственному значению соответствует вектор
$R^{-\frac{1}{2}} \breve{U}$, действительно:
\[
    C \left ( R^{-\frac{1}{2}} \breve{U} \right )
    = R^{-\frac{1}{2}} \breve{U} \left ( R^{-\frac{1}{2}} \breve{U} \right )^* R^{-\frac{1}{2}} \breve{U}
    = R^{-\frac{1}{2}} \breve{U} \norm{R^{-\frac{1}{2}} \breve{U}}^2
    = \norm{R^{-\frac{1}{2}} \breve{U}}^2 \cdot R^{-\frac{1}{2}} \breve{U} .
\]
Отсюда же следует, что
\[
    \max \limits_W \rho(W) = \lambda_{max} = \norm{R^{-\frac{1}{2}} \breve{U}}^2 .
\]
Таким образом, вектор $Z_{max}$, при котором достигается наибольшее значение отношения Релея $\rho(Z)$:
\[
    Z_{max} = R^{-\frac{H}{2}} \breve{U} ,
\]
а исходный вектор $W_{max}$:
\begin{align*}
    W_{max} & = \left( R^{-\frac{1}{2}} \right)^* Z_{max} , \\
    W_{max} & = \left( R^{-\frac{1}{2}} \right)^* R^{-\frac{1}{2}} \breve{U} , \\
    W_{max} & = \left( \left ( R^{\frac{1}{2}} \right)^* R^{\frac{1}{2}} \right )^{-1} \breve{U} , \\
    W_{max} & = R^{-1} \breve{U} .
\end{align*}

\subsection{Отсутствие источников излучения}

Выделение полезного сигнала $u$ с помощью проецирующего преобразования $\mathcal{F}_W(Y)$ с оптимальным весовым вектором $W_{max}$ можно использовать и при отсутствии источников
излучения, для адаптации к собственным шумам приёмников. При отсутствии источников излучения вектор "мешающих"{} сигналов $X$ образован огибающими собственных шумов приёмников:
\[
    X = E .
\]
Ковариационная матрица
\begin{gather*}
    R = \variance{X} = \sigma_0^2 I_n , \\
    %
    R^{-1} = \frac{1}{\sigma_0^2} I_n .
\end{gather*}
Оптимальный весовой вектор $W_{max}$:
\[
    W_{max}
    = R^{-1} \breve{U}
    = \frac{1}{\sigma_0^2} \breve{U} .
\]

\subsection{Пример}

Если один источник помех и разнесены направления:
\matlab{adaptation(Receivers(5, 0.5 ,1), [-15 4], 10)}

Если два источника помех и разнесены направления:
\matlab{adaptation(Receivers(5, 0.5 ,1), [-15 4; 40 7], 10)}

По мере приближения источника к направлению полезного сигнала:
\begin{Matlab}
    \Mcommand{adaptation(Receivers(5, 0.5 ,1), [-15 4; 30 7], 10)}
    \Mcommand{adaptation(Receivers(5, 0.5 ,1), [-15 4; 20 7], 10)}
\end{Matlab}
\noindent уменьшается мощность в направлении полезного сигнала, возрастают мощности с боковых направлений.

Уменьшение мощности боковых направлений за счёт увеличения количества приёмников:
\begin{Matlab}
    \Mcommand{adaptation(Receivers(10, 0.5 ,1), [-15 4; 20 7], 10)}
\end{Matlab}


\section{Оценка ковариационной матрицы}

\subsection{Оценивание}

Ковариационная матрица $\variance{X}$ неизвестна, но её можно оценить --- нужно взять $m$ моментов времени и в каждый из моментов определить состояние приёмников $X_k$
($k = \overline{1,m}$):
\[
    X_k =
    \begin{pmatrix}
        x_{1,1} \\
        \dots   \\
        x_{i,k} \\
        \dots   \\
        x_{j,k} \\
        \dots   \\
        x_{n,k}
    \end{pmatrix} .
\]
Ковариация двух компонент $x_i$ и $x_j$:
\[
    \covariance{x_i}{x_j}
    = \expectation{\left ( x_i - \expectation{x_i} \right ) \left ( x_j - \expectation{x_j} \right )^*}
    = \expectation{ x_i x_j^*},
\]
поскольку $\expectation{x_k} = 0$.

В качестве оценки используем выражение:
\begin{multline*}
    \widehat{\covariance{x_i}{x_j}}
    = \frac{1}{p} \sum_{k=1}^p x_{i,k} x_{j,k}^*
    = \sum_{k=1}^p \frac{1}{\sqrt{p}} x_{i,k} \frac{1}{\sqrt{p}} x_{j,k}^* = \\
    %
    = \frac{1}{\sqrt{p}}
    \begin{pmatrix}
        x_{i,1} & x_{i,2} & \dots & x_{i,p}
    \end{pmatrix}
    \frac{1}{\sqrt{p}}
    \begin{pmatrix}
        x_{j,1}^* \\
        x_{j,2}^* \\
        \dots     \\
        x_{j,p}^*
    \end{pmatrix} .
\end{multline*}
Все оценки ковариаций можно получить умножением матриц:
\[
    \widehat{R} =
    \frac{1}{\sqrt{p}}
    \begin{pmatrix}
        x_{1,1} & x_{1,2} & \dots  & x_{1,p} \\
        x_{2,1} & x_{2,2} & \dots  & x_{2,p} \\
        \vdots  & \vdots  & \ddots & \vdots  \\
        x_{n,1} & x_{n,2} & \dots  & x_{n,p}
    \end{pmatrix}
    \frac{1}{\sqrt{p}}
    \begin{pmatrix}
        x_{1,1}^* & x_{2,1}^* & \dots  & x_{n,1}^* \\
        x_{1,2}^* & x_{2,2}^* & \dots  & x_{n,2}^* \\
        \vdots    & \vdots    & \ddots & \vdots    \\
        x_{1,p}^* & x_{2,p}^* & \dots  & x_{n,p}^*
    \end{pmatrix}
    .
\]
Правая матрица является сопряженной к левой матрице, поэтому если:
\[
    Y =
    \frac{1}{\sqrt{p}}
    \begin{pmatrix}
        x_{1,1} & x_{1,2} & \dots  & x_{1,p} \\
        x_{2,1} & x_{2,2} & \dots  & x_{2,p} \\
        \vdots  & \vdots  & \ddots & \vdots  \\
        x_{n,1} & x_{n,2} & \dots  & x_{n,p}
    \end{pmatrix} ,
\]
тогда
\[
    \widehat{R} = Y Y^* .
\]

\subsubsection{Пример}

Обнаружение и пеленгация по спектру:
\begin{Matlab}
    \Mcommand{detection(Receivers(5, 0.5, 1), [-15 4], 10)}
    \Mcommand{detection(Receivers(5, 0.5, 1), [-15 4], 500)}
\end{Matlab}

Пеленгация по рельефу (запустить несколько раз для объёма выборки 5, поскольку оценка ковариационной матрицы сильно меняется):
\matlab{quadric(Receivers(5, 0.5, 1), [-15 4; 30 7], 5, "direct"{}, 1)}
\noindent при увеличении объёма выборки оценка ковариационной матрицы становится более точной, решение улучшается:
\matlab{quadric(Receivers(5, 0.5, 1), [-15 4; 30 7], 1000, "direct"{}, 1)}

Адаптация (запустить несколько раз для объёма выборки 5, поскольку оценка ковариационной матрицы сильно меняется):
\begin{Matlab}
    \Mcommand{adaptation(Receivers(5, 0.5, 1), [-15 4; 30 7], 5, 10)}
\end{Matlab}
\noindent при увеличении объёма выборки оценка ковариационной матрицы становится более точной, решение улучшается:
\matlab{adaptation(Receivers(5, 0.5, 1), [-15 4; 30 7], 1000, 10)}

\subsection{Ортогонализация и обращение}

Матрица $\widehat{R}$ является факторизованной, поэтому можно найти факторизацию обратной матрицы $\widehat{R}^{-1}$.

Пусть $\Phi$ является преобразованием, ортогонализующим строки матрицы $Y$, то есть строки матрицы $\Phi Y$ являются взаимно ортогональными:
\[
    \left ( \Phi Y \right ) \left ( \Phi Y \right )^* = I_n ,
\]
отсюда
\begin{gather*}
    \Phi Y Y^* \Phi^* = I_n , \\
    \Phi \widehat{R} \Phi^* = I_n , \\
    \Phi \widehat{R} = \left(\Phi^* \right)^{-1}, \\
    \widehat{R} = \Phi^{-1} \left(\Phi^* \right)^{-1}, \\
    \widehat{R} = \left(\Phi^* \Phi \right)^{-1}, \\
    \widehat{R}^{-1} = \Phi^* \Phi .
\end{gather*}

\subsection{Вычисления}

Вычисление квадратичной формы:
\[
    V^* \widehat{R}^{-1} V
    = V^* \Phi^* \Phi V
    = \left ( \Phi V \right )^* \Phi V
    = \norm{\Phi V}^2 .
\]
Вычисление оптимального весового вектора:
\[
    W_{max}
    = \widehat{R}^{-1} U
    = \Phi^* \Phi U .
\]

    % линейные системы
    \chapter{Стационарные линейные системы}


\section{Системы}

Дискретные сигналы будем представлять в виде бесконечных последовательностей $x$. Элемент дискретного сигнала $x$ с номером $k \in \mathbb{Z}$ будет обозначать $x[k]$.
Номера $k$ будет считать моментами дискретного времени.

На основе множестве дискретных сигналов построим линейное пространство с поэлементным сложением сигналов и умножением на вещественные числа:
\[
    z = x + h \cdot y \Rightarrow z[k] = x[k] + h \cdot y[k].
\]

Дискретные сигналы будем преобразовывать с помощью операторов $\mathcal{L}$:
\[
    y = \mathcal{L}(x).
\]
Будем считать, что оператор соответствует действию системы, на вход которой последовательно подаются компоненты $x[k]$ сигнала $x$, и с выхода системы записываются компоненты
$y[k]$ сигнала $y$.

Далее рассматриваются только физически реализуемые системы, у которых на компонент $y[k]$ выходного сигнала могут влиять только компоненты $x[m]$, которые были получены системой
до момента $k$ включительно, то есть $m \le k$.

Рассматриваемые системы считаем стационарными: характеристики систем не изменяются во времени. Если входной сигнал $x$ сдвинуть на несколько моментов $s$ вправо или влево
на оси дискретного времени, то выходной сигнал системы также сдвинется на такое же количество моментов $s$:
\begin{align*}
    \widetilde{x}[k+s]              & = x[k] , \\
    \mathcal{L}(\widetilde{x})[k+s] & = \mathcal{L}(\widetilde{x})[k]
\end{align*}

Предполагаем, что системы являются линейными:
\begin{gather*}
    \mathcal{L}(x + h \cdot y)
    = \mathcal{L}(x) + \mathcal{L}(h \cdot y)
    = \mathcal{L}(x) + h \cdot \mathcal{L}(y) , \\
    %
    \mathcal{L}(x + h \cdot y)[k]
    = \mathcal{L}(x)[k] + h \cdot \mathcal{L}(y)[k] .
\end{gather*}

\subsection{Декомпозиция}

Свойство линейности позволяет анализировать действие оператора $\mathcal{L}$ на "сложные"{} сигналы, если "сложный"{} сигнал $x$ разложить на "простые"{} сигналы $e_i$:
\[
    x = h_1 \cdot e_1 + \dots + h_n \cdot e_n ,
\]
тогда
\[
    \mathcal{L}(x) = h_1 \cdot \mathcal{L}(e_1) + \dots + h_n \cdot \mathcal{L}(e_n) .
\]

В качестве элементов $e_i$ можно выбирать единичные импульсы (но это не единственный возможный вариант). Рассмотрим сигнал $\delta$ в виде единичного импульса, помещенного в
момент времени 0:
\[
    \delta[k]
    = \left \{
    \begin{array}{rr}
        1 & \text{, если } k = 0 ,  \\
        0 & \text{, если } k \neq 0
    \end{array}
    \right .
\]
В качестве элементов $e_i$ выберем сдвинутые единичные импульсы:
\[
    e_i[k+i]
    = \delta[k] ,
\]
тогда любой сигнал $x$ можно представить в виде:
\[
    x
    = \sum_{i} x[i] e_i .
\]

\subsection{Импульсная характеристика}

Реакция системы на сигнал $x$ оказывается равной:
\[
    y
    = \mathcal{L}(x)
    = \mathcal{L} \left ( \sum_{i} x[i] e_i \right )
    = \sum_{i} \mathcal{L} \left ( x[i] e_i \right )
    = \sum_{i} x[i] \mathcal{L} \left ( e_i \right )
\]
Значит необходимо выяснить выходные сигналы системы при подаче на вход системы импульсов $e_i$:
\[
    \mathcal{L} \left ( e_i \right )[k+i] = \mathcal{L}(\delta)[k]
\]
и эти выходные сигналы оказываются сдвигами выходного сигнала системы, если на вход системы подаётся единичный импульс $\delta$. Обозначим этот сигнал $h$:
\[
    h = \mathcal{L}(\delta).
\]
Сигнал $h$ называется импульсной характеристикой системы $\mathcal{L}$.

Таким образом, элемент $s$ выходного сигнала системы при входном сигнале $x$ представляется суммой:
\begin{equation}
    ~\label{linear:impulse:convolution}
    y[s]
    = \mathcal{L}(x)[s]
    = \sum_{i} x[i] \mathcal{L} \left ( e_i \right )[s]
    = \sum_{i} x[i] \mathcal{L} \left ( \delta \right )[s-i]
    = \sum_{i} x[i] h[s-i] .
\end{equation}
Выходной сигнал представляет собой свёртку входного сигнала $x$ и импульсной характеристики $h$ системы $\mathcal{L}$.

\textcolor{red}{Рисунок: перевёрнутая импульсная характеристика "едет"{} вдоль сигнала.}

\subsection{Свёртка}

В силу свойства физической реализуемости системы $\mathcal{L}$:
\[
    k < 0 : h[k] = 0 .
\]
Будем дополнительно считать, что импульсная характеристика является конечной, то есть существует число $n>0$:
\[
    k > n-1 : h[k] = 0 ,
\]
тогда в свёртке \eqref{linear:impulse:convolution} останется конечное число слагаемых, поскольку одновременно должны выполняться равенства:
\begin{gather*}
    0 \le s - i \le n-1 , \\
    -(n - 1) \le i - s \le 0 , \\
    s - (n - 1) \le i \le s .
\end{gather*}
Таким образом,
\[
    y[s]
    = \mathcal{L}(x)[s]
    = \sum_{i=s-(n-1)}^s x[i] h[s-i] .
\]
Выпишем компонент $y[n-1]$:
\begin{multline*}
    y[n-1]
    = \mathcal{L}(x)[n-1]
    = \sum_{i=0}^{n-1} x[i] h[s-i] = \\
    %
    = x[0] h[n-1] + x[1] h[n-2] + x[2] h[n-3] + \dots + x[n-2] h[1] + x[n-1] h[0] .
\end{multline*}
Можно представить в виде умножения матрицы и столбца:
\[
    \begin{pmatrix}
        \vdots \\
        y[n-1]
    \end{pmatrix}
    = \begin{pmatrix}
          \vdots & \vdots & \vdots & \vdots & \vdots & \vdots \\
          h[n-1] & h[n-2] & h[n-3] & \dots  & h[1]   & h[0]
    \end{pmatrix}
    \begin{pmatrix}
        x[0]   \\
        x[1]   \\
        x[2]   \\
        \vdots \\
        x[n-2] \\
        x[n-1]
    \end{pmatrix}
\]

Выпишем компонент $y[n-2]$:
\begin{multline*}
    y[n-2]
    = \mathcal{L}(x)[n-2]
    = \sum_{i=-1}^{n-2} x[i] h[s-i] = \\
    %
    = x[-1] h[n-1] + x[0] h[n-2] + x[1] h[n-3] + \dots + x[n-3] h[1] + x[n-2] h[0] .
\end{multline*}
Появился компонент $x[-1]$. Поскольку далее будут использоваться периодические дискретные сигналы, то будем считать, что на отрезке от 0 до $n-1$ помещается натуральное количество
периодов сигнала $x$ (число $n$ всегда можно увеличить до ближайшей границы периода справа) и для всех $s$ выполняется
\[
    x[s] = x[s+n] ,
\]
тогда в силу принятой периодичности:
\[
    x[-1] = x[-1+n] = x[n-1].
\]
И в выражении для $y[n-2]$ можно заменить $x[-1]$:
\[
    y[n-1]
    = x[n-1] h[n-1] + x[0] h[n-2] + x[1] h[n-3] + \dots + x[n-3] h[1] + x[n-2] h[0] .
\]
тогда в матричной форме:
\[
    \begin{pmatrix}
        \vdots \\
        y[n-2] \\
        y[n-1]
    \end{pmatrix}
    = \begin{pmatrix}
          \vdots & \vdots & \vdots & \vdots & \vdots & \vdots \\
          h[n-2] & h[n-3] & h[n-4] & \dots  & h[0]   & h[n-1] \\
          h[n-1] & h[n-2] & h[n-3] & \dots  & h[1]   & h[0]
    \end{pmatrix}
    \begin{pmatrix}
        x[0]   \\
        x[1]   \\
        x[2]   \\
        \vdots \\
        x[n-2] \\
        x[n-1]
    \end{pmatrix}
\]
Продолжая аналогичным образом, с учётом периодичности сигнала $x$, получим:
\[
    \begin{pmatrix}
        y[0]   \\
        y[1]   \\
        y[2]   \\
        \vdots \\
        y[n-2] \\
        y[n-1]
    \end{pmatrix}
    = \begin{pmatrix}
          h[0]   & h[n-1] & h[n-2] & \dots  & h[2]   & h[1]   \\
          h[1]   & h[0]   & h[n-1] & \dots  & h[3]   & h[2]   \\
          h[2]   & h[1]   & h[0]   & \dots  & h[4]   & h[3]   \\
          \vdots & \vdots & \vdots & \vdots & \vdots & \vdots \\
          h[n-2] & h[n-3] & h[n-4] & \dots  & h[0]   & h[n-1] \\
          h[n-1] & h[n-2] & h[n-3] & \dots  & h[1]   & h[0]
    \end{pmatrix}
    \begin{pmatrix}
        x[0]   \\
        x[1]   \\
        x[2]   \\
        \vdots \\
        x[n-2] \\
        x[n-1] \\
    \end{pmatrix} .
\]

\subsection{Циркулянты}

Далее в силу принятой периодичности сигналов будем рассматривать не сами сигналы (бесконечные последовательности), а векторы длины $n$. Введём обозначения для значений сигналов:
\begin{gather*}
    y_k = y[k] , \\
    h_k = h[k] , \\
    x_k = x[k]
\end{gather*}
и векторов:
\[
    y
    = \begin{pmatrix}
          y_0    \\
          y_1    \\
          \vdots \\
          y_{n-1}
    \end{pmatrix}
    , \;
    h
    = \begin{pmatrix}
          h_0    \\
          h_1    \\
          \vdots \\
          h_{n-1}
    \end{pmatrix}
    , \;
    x
    = \begin{pmatrix}
          x_0    \\
          x_1    \\
          \vdots \\
          x_{n-1}
    \end{pmatrix}
    .
\]
Таким образом,
\begin{equation}
    \label{linear:transformation}
    y = C(h) x ,
\end{equation}
где матрица $C(h)$:
\begin{equation}
    \label{linear:circulant}
    C(h)
    = \begin{pmatrix}
          h_0     & h_{n-1} & h_{n-2} & \dots  & h_2    & h_1     \\
          h_1     & h_0     & h_{n}   & \dots  & h_3    & h_2     \\
          h_2     & h_1     & h_{0}   & \dots  & h_4    & h_3     \\
          \vdots  & \vdots  & \vdots  & \vdots & \vdots & \vdots  \\
          h_{n-2} & h_{n-3} & h_{n-4} & \dots  & h_0    & h_{n-1} \\
          h_{n-1} & h_{n-2} & h_{n-3} & \dots  & h_1    & h_0
    \end{pmatrix}
    .
\end{equation}
Строки матрицы $C(h)$ образованы циклическим сдвигом первой строки и столбцы образованы циклическим сдвигом первого столбца. Такие матрицы называются циркулянтными матрицами или просто
циркулянтами.

У циркулянтов имеется интересное свойство. Пусть $\delta \in \mathbb{C}$ --- комплексный корень из единицы степени $n$:
\[
    \delta^n = 1 .
\]
Образуем строку из степеней числа $\delta$ и умножим на циркулянт $C(h)$ слева:
\[
    \begin{pmatrix}
        \delta^0 & \delta^1 & \delta^2 & \dots & \delta^{n-1}
    \end{pmatrix}
    \begin{pmatrix}
        h_0     & h_{n-1} & h_{n-2} & \dots  & h_2    & h_1     \\
        h_1     & h_0     & h_{n}   & \dots  & h_3    & h_2     \\
        h_2     & h_1     & h_{0}   & \dots  & h_4    & h_3     \\
        \vdots  & \vdots  & \vdots  & \vdots & \vdots & \vdots  \\
        h_{n-2} & h_{n-3} & h_{n-4} & \dots  & h_0    & h_{n-1} \\
        h_{n-1} & h_{n-2} & h_{n-3} & \dots  & h_1    & h_0
    \end{pmatrix}
\]

В результате умножения строки на первый столбец $C(h)$ получим выражение:
\[
    \begin{pmatrix}
        \delta^0 & \delta^1 & \delta^2 & \dots & \delta^{n-1}
    \end{pmatrix}
    \begin{pmatrix}
        h_0   \\
        h_1   \\
        h_2   \\
        \dots \\
        h_{n-1}
    \end{pmatrix}
    =
    h_0 \delta^0 + h_1 \delta^1 + h_2 \delta^2 + \dots + h_{n-1} \delta^{n-1} = \lambda(\delta) ,
\]
представляющее собой значение полинома $\lambda(z)$ при $z = \delta$:
\[
    \lambda(z) = h_0 z^0 + h_1 z^1 + h_2 z^2 + \dots + h_{n-1} z^{n-1} .
\]
При умножении строки на второй столбец получим:
\begin{multline*}
    \begin{pmatrix}
        \delta^0 & \delta^1 & \delta^2 & \dots & \delta^{n-1}
    \end{pmatrix}
    \begin{pmatrix}
        h_{n-1} \\
        h_0     \\
        h_1     \\
        h_2     \\
        \dots   \\
        h_{n-2}
    \end{pmatrix} = \\
    %
    = h_{n-1} \delta^0 + h_0 \delta^1 + h_1 \delta^2 + \dots + h_{n-2} \delta^{n-1} = \\
    %
    = h_{n-1} 1 + h_0 \delta^1 + h_1 \delta^2 + \dots + h_{n-2} \delta^{n-1} = \\
    %
    = h_{n-1} \delta^n + h_0 \delta^1 + h_1 \delta^2 + \dots + h_{n-2} \delta^{n-1} = \\
    %
    = h_0 \delta^1 + h_1 \delta^2 + \dots + h_{n-2} \delta^{n-1} + h_{n-1} \delta^n = \\
    %
    = \delta \left( h_0 \delta^0 + h_1 \delta^1 + \dots + h_{n-2} \delta^{n-2} + h_{n-1} \delta^{n-1} \right) = \\
    %
    = \delta \lambda(\delta) .
\end{multline*}

Аналогичным образом получаются результаты умножения строки на все остальные столбцы матрицы $C$, в итоге получим:
\begin{multline*}
    \begin{pmatrix}
        \delta^0 & \delta^1 & \delta^2 & \dots & \delta^{n-1}
    \end{pmatrix}
    \begin{pmatrix}
        h_0     & h_{n-1} & h_{n-2} & \dots  & h_3    & h_2    & h_1    \\
        h_1     & h_0     & h_{n-1} & \dots  & h_4    & h_3    & h_2    \\
        h_2     & h_1     & h_0     & \dots  & h_5    & h_4    & h_3    \\
        \vdots  & \vdots  & \vdots  & \ddots & \vdots & \vdots & \vdots \\
        h_{n-1} & h_{n-2} & h_{n-3} & \dots  & h_2    & h_1    & h_0
    \end{pmatrix} = \\
    %
    = \begin{pmatrix}
          \lambda(\delta) & \delta \lambda(\delta) & \delta^2 \lambda(\delta) & \dots & \delta^{n-1} \lambda(\delta)
    \end{pmatrix} = \\
    %
    = \lambda(\delta)
    \begin{pmatrix}
        \delta^0 & \delta^1 & \delta^2 & \dots & \delta^{n-1}
    \end{pmatrix}
    .
\end{multline*}


\section{Дискретное преобразование Фурье}

\subsection{Базис}

Ранее для представления сигналов использовались единичные импульсы $e_i$:
\[
    e_i[k]
    =
    \left\{
    \begin{array}{ll}
        1, & k = i    \\
        0, & k \neq i
    \end{array}
    \right.
    ,
\]
которые образовывали базис, по которому раскладывались сигналы. Это не единственный вариант базиса, есть алтернативные варианты, и этих вариантов достаточно много.

Поскольку теперь рассматриваются периодические сигналы, которые повторяются через каждые $n$ моментов времени, то вполне допустимым было бы ввести базис, образованный периодическими
сигналами. Например, можно определить сигнал $\xi_1$, у которого на протяжении $n$ моментов укладывается один период гармонического колебания:
\[
    x_1[k]
    = \cos \frac{2 \pi}{n} k
\]
Далее будет удобно работать с комплексными величинами, поэтому дополним сигнал мнимой частью:
\[
    z_1[k]
    = \cos \frac{2 \pi}{n} k + i \sin \frac{2 \pi}{n} k
    = e^{i \frac{2 \pi}{n} k} .
\]
и заметим, что все значения сигнала являются степенями одного числа $\varepsilon$:
\begin{gather*}
    z_1[k]
    = \left( e^{i \frac{2 \pi}{n}} \right)^k
    = \varepsilon^k, \\
    %
    \varepsilon = e^{i \frac{2 \pi}{n}} .
\end{gather*}

Соберём первые $n$ значений сигнала $z_1$ в вектор:
\[
    \xi_1
    = \begin{pmatrix}
          \varepsilon^0 \\
          \varepsilon^1 \\
          \varepsilon^2 \\
          \vdots        \\
          \varepsilon^{n-1}
    \end{pmatrix} .
\]

Возьмем теперь сигнал, у которого на протяжении $n$ моментов времени укладывается два периода гармонических колебаний:
\begin{gather*}
    x_2[k]
    = \cos \frac{2 \cdot 2 \pi}{n} k , \\
    %
    z_2[k]
    = \cos \frac{2 \cdot 2 \pi}{n} k + i \sin \frac{2 \cdot 2 \pi}{n} k
    = e^{i \frac{2 \cdot 2 \pi}{n} k}, \\
    %
    z_2[k]
    = \left( e^{- i \frac{2 \cdot 2 \pi}{n}} \right)^k
    = \left( \left( e^{- i \frac{2 \pi}{n}} \right)^2 \right)^k
    = \left( \varepsilon^2 \right)^k
\end{gather*}
Первые $n$ значений соберем в вектор
\[
    \xi_2
    = \begin{pmatrix}
          \varepsilon^{2 \cdot 0} \\
          \varepsilon^{2 \cdot 1} \\
          \varepsilon^{2 \cdot 2} \\
          \vdots                  \\
          \varepsilon^{2 \cdot (n-1)}
    \end{pmatrix}
\]

Аналогично, можно рассматривать сигнал, у которого на протяжении $n$ моментов времени укладывается $m$ периодов гармонических колебаний:
\begin{gather*}
    x_m[k]
    = \cos \frac{m \cdot 2 \pi}{n} k , \\
    %
    z_m[k]
    = \cos \frac{m \cdot 2 \pi}{n} k + i \sin \frac{m \cdot 2 \pi}{n} k
    = e^{i \frac{m \cdot 2 \pi}{n} k}, \\
    %
    z_m[k]
    = \left( e^{i \frac{m \cdot 2 \pi}{n}} \right)^k
    = \left( \left( e^{i \frac{2 \pi}{n}} \right)^m \right)^k
    = \left( \varepsilon^m \right)^k
\end{gather*}
и сформируем вектор
\[
    \xi_m
    = \begin{pmatrix}
          \varepsilon^{m \cdot 0} \\
          \varepsilon^{m \cdot 1} \\
          \varepsilon^{m \cdot 2} \\
          \vdots                  \\
          \varepsilon^{m \cdot (n-1)}
    \end{pmatrix} .
\]

До каких пор можно увеличивать $m$? Оказывается, что при $m = n$:
\[
    x_n[k]
    = \cos \frac{n \cdot 2 \pi}{n} k
    = \cos 2 \pi k
    = 1
    = \cos 0 \cdot k
    = \cos \frac{0 \cdot 2 \pi}{n} k
    = x_0[k]
    , \\
\]
поэтому вектор
\[
    \xi_n
    = \begin{pmatrix}
          1      \\
          1      \\
          1      \\
          \vdots \\
          1
    \end{pmatrix}
    = \xi_0.
\]
Если же $m > n$, то сигналы повторяются:
\begin{multline*}
    x_m[k]
    = \cos \frac{m \cdot 2 \pi}{n} k
    = \cos \frac{((m-n) + n) \cdot 2 \pi}{n} k = \\
    %
    = \cos \left( \frac{(m-n) \cdot 2 \pi}{n} + 2 \pi  \right) k
    = \cos \frac{(m-n) \cdot 2 \pi}{n} k
    = x_{m-n}[k].
\end{multline*}
Таким образом, мы исчерпали все возможные варианты и получили набор векторов $\xi_m$, образованные степенями $\varepsilon^{m \cdot k}$:
\[
    \xi_0
    = \begin{pmatrix}
          \varepsilon^{0 \cdot 0} \\
          \varepsilon^{0 \cdot 1} \\
          \varepsilon^{0 \cdot 2} \\
          \vdots                  \\
          \varepsilon^{0 \cdot (n-1)}
    \end{pmatrix} ,
    \;
    %
    \xi_1
    = \begin{pmatrix}
          \varepsilon^{1 \cdot 0} \\
          \varepsilon^{1 \cdot 1} \\
          \varepsilon^{1 \cdot 2} \\
          \vdots                  \\
          \varepsilon^{1 \cdot (n-1)}
    \end{pmatrix} ,
    \;
    \xi_2
    = \begin{pmatrix}
          \varepsilon^{2 \cdot 0} \\
          \varepsilon^{2 \cdot 1} \\
          \varepsilon^{2 \cdot 2} \\
          \vdots                  \\
          \varepsilon^{2 \cdot (n-1)}
    \end{pmatrix} ,
    \;
    \dots ,
    \;
    \xi_{n-1}
    = \begin{pmatrix}
          \varepsilon^{(n-1) \cdot 0} \\
          \varepsilon^{(n-1) \cdot 1} \\
          \varepsilon^{(n-1) \cdot 2} \\
          \vdots                      \\
          \varepsilon^{(n-1) \cdot (n-1)}
    \end{pmatrix} .
\]
Векторы $\xi_m$ можно рассматривать как элементы линейного пространства $\mathbb{C}^n$, в котором введём скалярное прозведение:
\[
    \scalarproduct{x}{y} = y^* x .
\]
Вычислим скалярное произведение двух векторов:
\[
    \scalarproduct{\xi_j}{\xi_k}
    = \xi_k^* \xi_j
    = \sum_{l=0}^{n-1} \overline{\varepsilon^{k \cdot l}} \varepsilon^{j \cdot l}
\]
Заметим, что для любой степени $p$
\begin{equation}
    ~\label{linear:Fourier:conjugation}
    \overline{\varepsilon^{p}}
    = \frac{\overline{\varepsilon^{p}} \varepsilon^{p}}{\varepsilon^{p}}
    = \frac{\modulus{\varepsilon^{p}}^2}{\varepsilon^{p}}
    = \frac{\left ( \modulus{\varepsilon}^{p} \right )^2}{\varepsilon^{p}}
    = \frac{\left ( 1^{p} \right )^2}{\varepsilon^{p}}
    = \frac{1}{\varepsilon^{p}}
    = \varepsilon^{- p}
    .
\end{equation}
Таким образом,
\[
    \scalarproduct{\xi_j}{\xi_k}
    = \sum_{l=0}^{n-1} \varepsilon^{- k \cdot l} \varepsilon^{j \cdot l}
    = \sum_{l=0}^{n-1} \varepsilon^{(j - k) \cdot l}
\]
Если $j = k$, то сразу получается норма векторов $\xi_j$, порождаемая скалярным произведением:
\[
    \norm{\xi_j}^2
    = \scalarproduct{\xi_j}{\xi_j}
    = \sum_{l=0}^{n-1} \varepsilon^{( j - j ) \cdot l}
    = \sum_{l=0}^{n-1} \varepsilon^0
    = \sum_{j=0}^{n-1} 1
    = n .
\]
Если $j \neq k$, то в сумме, стоящей справа легко увидеть геометрическую прогрессию со знаменателем $\varepsilon^{j - k}$, её сумма:
\[
    \scalarproduct{\xi_j}{\xi_k}
    = \frac{1 - \left( \varepsilon^{(j-k)} \right)^{n-1+1}}{1 - \varepsilon^{j - k}}
    = \frac{1 - \left( \varepsilon^{(j-k)} \right)^n}{1 - \varepsilon^{j - k}}
    = \frac{1 - \left( \varepsilon^n \right)^{j-k}}{1 - \varepsilon^{j - k}}
    = \frac{1 - 1^{j-k}}{1 - \varepsilon^{j - k}}
    = \frac{1 - 1}{1 - \varepsilon^{j - k}}
    = 0 ,
\]
поскольку
\[
    \varepsilon^n
    = \left( e^{- i \frac{2 \pi}{n}} \right)^n
    = e^{- i \frac{2 \pi}{n} \cdot n}
    = e^{- i 2 \pi}
    = 1.
\]

\subsection{Спектр}

Таким образом, векторы $\xi_j$ является ортогональными, поэтому они являются линейно независимыми. Их количество $n$ совпадает с размерностью пространства $\mathbb{C}^n$,
поэтому векторы $\xi_j$ образуют базис в пространстве $\mathbb{C}^n$ и любой вектор $x \in \mathbb{C}^n$ можно разложить по базису $\xi_j$:
\[
    x = \sum_{j=0}^{n-1} X_j \xi_j.
\]
Координаты $X_k$ легко вычислить, поскольку векторы $\xi_k$ ортогональны, то нужно вычислить скалярное произведение:
\[
    \scalarproduct{x}{\xi_k}
    = \scalarproduct{\sum_{j=0}^{n-1} X_j \xi_j}{\xi_k}
    = \sum_{j=0}^{n-1} X_j \scalarproduct{\xi_j}{\xi_k}
    = X_k \scalarproduct{\xi_k}{\xi_k}
    = X_k \norm{\xi_k}^2
    = X_k n
    .
\]
Вектор $X = \left ( n X_0, n X_1, \dots, n X_{n-1} \right )$ называют спектром сигнала $x$.

\subsection{Прямое преобразование}

Вектор $X$ можно получить умножением на матрицу $F_n$:
\[
    X
    = \begin{pmatrix}
          \scalarproduct{x}{\xi_0}     \\
          \scalarproduct{x}{\xi_1}     \\
          \vdots                       \\
          \scalarproduct{x}{\xi_{n-1}} \\
    \end{pmatrix}
    = \begin{pmatrix}
          \xi_0^* x     \\
          \xi_1^* x     \\
          \vdots        \\
          \xi_{n-1}^* x \\
    \end{pmatrix}
    = F_n x ,
\]
где
\begin{multline*}
    F_n =
    \begin{pmatrix}
        \xi_0^* \\
        \xi_1^* \\
        \dots   \\
        \xi_{n-1}^*
    \end{pmatrix}
    =
    \begin{pmatrix}
        \overline{\varepsilon^{0 \cdot 0}}     & \overline{\varepsilon^{0 \cdot 1}}     & \overline{\varepsilon^{0 \cdot 2}}     & \dots  & \overline{\varepsilon^{0 \cdot (n-1)}} \\
        \overline{\varepsilon^{1 \cdot 0}}     & \overline{\varepsilon^{1 \cdot 1}}     & \overline{\varepsilon^{1 \cdot 2}}     & \dots  & \overline{\varepsilon^{1 \cdot (n-1)}} \\
        \vdots                                 & \vdots                                 & \vdots                                 & \ddots & \vdots                                     \\
        \overline{\varepsilon^{(n-1) \cdot 0}} & \overline{\varepsilon^{(n-1) \cdot 1}} & \overline{\varepsilon^{(n-1) \cdot 2}} & \dots  & \overline{\varepsilon^{(n-1) \cdot (n-1)}} \\
    \end{pmatrix} = \\
    %
    = \begin{pmatrix}
          \varepsilon^{- 0 \cdot 0}     & \varepsilon^{- 0 \cdot 1}     & \varepsilon^{- 0 \cdot 2}     & \dots  & \varepsilon^{- 0 \cdot (n-1)}     \\
          \varepsilon^{- 1 \cdot 0}     & \varepsilon^{- 1 \cdot 1}     & \varepsilon^{- 1 \cdot 2}     & \dots  & \varepsilon^{- 1 \cdot (n-1)}     \\
          \vdots                        & \vdots                        & \vdots                        & \ddots & \vdots                            \\
          \varepsilon^{- (n-1) \cdot 0} & \varepsilon^{- (n-1) \cdot 1} & \varepsilon^{- (n-1) \cdot 2} & \dots  & \varepsilon^{- (n-1) \cdot (n-1)} \\
    \end{pmatrix} = \\
    %
    = \begin{pmatrix}
          1      & 1                             & 1                             & \dots  & 1                                 \\
          1      & \varepsilon^{- 1 \cdot 1}     & \varepsilon^{- 2 \cdot 1}     & \dots  & \varepsilon^{- (n-1) \cdot 1}     \\
          \vdots & \vdots                        & \vdots                        & \ddots & \vdots                            \\
          1      & \varepsilon^{- 1 \cdot (n-1)} & \varepsilon^{- 2 \cdot (n-1)} & \dots  & \varepsilon^{- (n-1) \cdot (n-1)}
    \end{pmatrix}
    ,
\end{multline*}
последнее равенство получено в силу сопряжения \eqref{linear:Fourier:conjugation}.

Матрица $F_n$ является матрицей Фурье (матрицей прямого дискретного преобразования Фурье).

\subsection{Обратное преобразование}

В силу ортогональности векторов $\xi_k$ для матрицы $F_n$ правой обратной матрицей будет $\frac{1}{n} F_n^*$:
\[
    \frac{1}{n} F_n^*
    =
    \frac{1}{n}
    \begin{pmatrix}
        \xi_0 & \xi_1 & \dots & \xi_{n-1}
    \end{pmatrix}
    =
    \frac{1}{n}
    \begin{pmatrix}
        \varepsilon^{0 \cdot 0}     & \varepsilon^{1 \cdot 0}     & \dots  & \varepsilon^{(n-1) \cdot 0}     \\
        \varepsilon^{0 \cdot 1}     & \varepsilon^{1 \cdot 1}     & \dots  & \varepsilon^{(n-1) \cdot 1}     \\
        \vdots                      & \vdots                      & \ddots & \vdots                          \\
        \varepsilon^{0 \cdot (n-1)} & \varepsilon^{1 \cdot (n-1)} & \dots  & \varepsilon^{(n-1) \cdot (n-1)}
    \end{pmatrix}
    ,
\]
действительно:
\begin{multline*}
    F_n \frac{1}{n} F_n^*
    =
    \begin{pmatrix}
        \xi_0^* \\
        \xi_1^* \\
        \dots   \\
        \xi_{n-1}^*
    \end{pmatrix}
    \frac{1}{n}
    \begin{pmatrix}
        \xi_0 & \xi_1 & \dots & \xi_{n-1}
    \end{pmatrix} = \\
    %
    = \frac{1}{n}
    \begin{pmatrix}
        \scalarproduct{\xi_0}{\xi_0}     & \scalarproduct{\xi_0}{\xi_1}     & \dots  & \scalarproduct{\xi_0}{\xi_{n-1}}     \\
        \scalarproduct{\xi_1}{\xi_0}     & \scalarproduct{\xi_1}{\xi_1}     & \dots  & \scalarproduct{\xi_1}{\xi_{n-1}}     \\
        \vdots                           & \vdots                           & \ddots & \vdots                               \\
        \scalarproduct{\xi_{n-1}}{\xi_0} & \scalarproduct{\xi_{n-1}}{\xi_1} & \dots  & \scalarproduct{\xi_{n-1}}{\xi_{n-1}}
    \end{pmatrix}
    = \frac{1}{n}
    \begin{pmatrix}
        \norm{\xi_0}^2 & 0              & \dots  & 0              \\
        0              & \norm{\xi_1}^2 & \dots  & 0              \\
        \vdots         & \vdots         & \ddots & \vdots         \\
        0              & 0              & \dots  & \norm{\xi_n}^2
    \end{pmatrix}
    = \\
    %
    = \frac{1}{n}
    \begin{pmatrix}
        n      & 0      & \dots  & 0      \\
        0      & n      & \dots  & 0      \\
        \vdots & \vdots & \ddots & \vdots \\
        0      & 0      & \dots  & n
    \end{pmatrix}
    =
    \begin{pmatrix}
        1      & 0      & \dots  & 0      \\
        0      & 1      & \dots  & 0      \\
        \vdots & \vdots & \ddots & \vdots \\
        0      & 0      & \dots  & 1
    \end{pmatrix}
    .
\end{multline*}

Известно, что правая обратная матрица является и левой обратной.

\subsection{Преобразование свёртки}

Теперь обратим внимание, что строки матрицы прямого преобразования Фурье $F_n$ являются векторами $\xi_k^*$:
\[
    \xi_k^*
    = \begin{pmatrix}
          \varepsilon^{- 0 \cdot k} & \varepsilon^{- 1 \cdot k} & \varepsilon^{- 2 \cdot k} & \dots & \varepsilon^{- (n-1) \cdot k}
    \end{pmatrix}
\]
которые образованы степенями корней из единицы $\varepsilon^{-k}$ (действительно, $\left ( \varepsilon^{-k} \right )^n = \left ( \varepsilon^n \right )^{-k} = 1^{-k} = 1$).

Пусть $h$ --- произвольный вектор и $C(h)$ --- циркулянт \eqref{linear:circulant}, тогда в результате умножения одной строки матрицы Фурье $F_n$:
\[
    \xi_k^* C(h)
    = \lambda ( \varepsilon^{-k} ) \begin{pmatrix}
                                       \varepsilon^{- 0 \cdot k} & \varepsilon^{- 1 \cdot k} & \varepsilon^{- 2 \cdot k} & \dots & \varepsilon^{- (n-1) \cdot k}
    \end{pmatrix}
    = \lambda ( \varepsilon^{-k} ) \xi_k^*
\]
откуда для произведения
\begin{multline*}
    F_n C(h)
    =
    \begin{pmatrix}
        \lambda ( \varepsilon^{-0} ) \xi_0^* \\
        \lambda ( \varepsilon^{-1} ) \xi_1^* \\
        \lambda ( \varepsilon^{-2} ) \xi_2^* \\
        \dots                                \\
        \lambda ( \varepsilon^{-(n-1)} ) \xi_{n-1}^*
    \end{pmatrix} = \\
    %
    =
    \begin{pmatrix}
        \lambda ( \varepsilon^{-0} ) & 0                            & 0                            & \dots  & 0                                \\
        0                            & \lambda ( \varepsilon^{-1} ) & 0                            & \dots  & 0                                \\
        0                            & 0                            & \lambda ( \varepsilon^{-2} ) & \dots  & 0                                \\
        \vdots                       & \vdots                       & \ddots                       & \ddots & 0                                \\
        0                            & 0                            & 0                            & \dots  & \lambda ( \varepsilon^{-(n-1)} )
    \end{pmatrix}
    \begin{pmatrix}
        \xi_0^* \\
        \xi_1^* \\
        \xi_2^* \\
        \dots   \\
        \xi_{n-1}^*
    \end{pmatrix}
    = \\
    %
    =
    \begin{pmatrix}
        \xi_0^* h & 0         & 0         & \dots  & 0             \\
        0         & \xi_1^* h & 0         & \dots  & 0             \\
        0         & 0         & \xi_2^* h & \dots  & 0             \\
        \vdots    & \vdots    & \ddots    & \ddots & 0             \\
        0         & 0         & 0         & \dots  & \xi_{n-1}^* h
    \end{pmatrix}
    \begin{pmatrix}
        \xi_0^* \\
        \xi_1^* \\
        \xi_2^* \\
        \dots   \\
        \xi_{n-1}^*
    \end{pmatrix}
\end{multline*}
поскольку
\[
    \lambda ( \varepsilon^{-k} )
    = h_0 \varepsilon^{- 0 \cdot k} + h_1 \varepsilon^{- 1 \cdot k} + h_2 \varepsilon^{- 2 \cdot k} + \dots + h_{n-1} \varepsilon^{- (n-1) \cdot k} .
\]

Далее легко видеть, что на диагонали располагаются элементы спектра $H$ вектора $h$:
\[
    H
    =
    \begin{pmatrix}
        \xi_0^* h \\
        \xi_1^* h \\
        \xi_2^* h \\
        \dots     \\
        \xi_{n-1}^* h
    \end{pmatrix}
    =
    \begin{pmatrix}
        \xi_0^* \\
        \xi_1^* \\
        \xi_2^* \\
        \dots   \\
        \xi_{n-1}^*
    \end{pmatrix}
    h
    = F_n h
\]
Таким образом,
\[
    F_n C(h) = diag \left( H \right) F_n.
\]
где $diag \left( H \right)$ обозначает диагональную матрицу с элементами $H_k$ на главной диагонали.

Пусть теперь некоторый сигнал $x$ со спектром $X$ преобразуется в сигнал $y$ линейной системой с импульсной характеристикой $h$ \eqref{linear:transformation}:
\[
    y = C(h) x ,
\]
тогда спектр $Y$ сигнала $y$:
\[
    Y
    = F_n y
    = F_n C(h) x
    = diag \left( H \right) F_n x
    = diag \left( H \right) X,
\]
или
\[
    Y_k = H_k \cdot X_k .
\]
Используя спектр $Y$, с помощью обратного преобразования получим сигнал $y$:
\[
    y = \frac{1}{n} F_n^* Y .
\]

Таким образом, для вычисления выходного сигнала $y$ существует два способа:
\begin{enumerate}
    \item прямое вычисление:
    \[
        y = C(h) x,
    \]
    \item через вычисление спектров:
    \begin{gather*}
        H = F_n h, \\
        X = F_n x, \\
        Y_k = H_k \cdot X_k, \\
        y = \frac{1}{n} F_n^* Y_k .
    \end{gather*}
\end{enumerate}
Второй способ требует больше вычислительных операций, но в силу особой структуры матрицы Фурье для вычисления спектров есть быстрый метод, поэтому в некоторых случаях второй способ
оказывается быстрее первого.


\section{Быстрое преобразование Фурье}

Легко видеть, что прямое и обратное преобразование Фурье связаны с матрицами вида:
\[
    F_n(\delta)
    = \begin{pmatrix}
          1      & 1            & 1               & 1               & \dots  & 1                   \\
          1      & \delta       & \delta^2        & \delta^3        & \dots  & \delta^{n-1}        \\
          1      & \delta^2     & \delta^4        & \delta^6        & \dots  & \delta^{2(n-1)}     \\
          1      & \delta^3     & \delta^6        & \delta^9        & \dots  & \delta^{3(n-1)}     \\
          \vdots & \vdots       & \vdots          & \vdots          & \ddots & \vdots              \\
          1      & \delta^{n-1} & \delta^{2(n-1)} & \delta^{3(n-1)} & \dots  & \delta^{(n-1)(n-1)}
    \end{pmatrix}.
\]
В случае умножения на матрицу $F_n(\delta)$ требуется примерно $(n-1) \cdot (n-1)$ умножений, но в силу особого вида матриц $F_n(\delta)$ количество умножений можно сократить,
если $n$ --- составное число.

\subsection{6 и 3, факторизация}

Матрица степеней
\[
    F_6
    = \begin{pmatrix}
          1 & 1        & 1           & 1           & 1           & 1           \\
          1 & \delta^1 & \delta^2    & \delta^3    & \delta^4    & \delta^5    \\
          1 & \delta^2 & \delta^4    & \delta^6    & \delta^8    & \delta^{10} \\
          1 & \delta^3 & \delta^6    & \delta^9    & \delta^{12} & \delta^{15} \\
          1 & \delta^4 & \delta^8    & \delta^{12} & \delta^{16} & \delta^{20} \\
          1 & \delta^5 & \delta^{10} & \delta^{15} & \delta^{20} & \delta^{25} \\
    \end{pmatrix}
\]
Перестановка строк ($P_6$ --- обратная перестановка строк):
\[
    F_6
    =
    \underbrace{
        \begin{pmatrix}
            1 & 0 & 0 & 0 & 0 & 0 \\
            0 & 0 & 1 & 0 & 0 & 0 \\
            0 & 0 & 0 & 0 & 1 & 0 \\
            0 & 1 & 0 & 0 & 0 & 0 \\
            0 & 0 & 0 & 1 & 0 & 0 \\
            0 & 0 & 0 & 0 & 0 & 1 \\
        \end{pmatrix}
    }_{P_6}
    \begin{pmatrix}
        1 & 1        & 1           & 1           & 1           & 1           \\
        1 & \delta^3 & \delta^6    & \delta^9    & \delta^{12} & \delta^{15} \\
        1 & \delta^1 & \delta^2    & \delta^3    & \delta^4    & \delta^5    \\
        1 & \delta^4 & \delta^8    & \delta^{12} & \delta^{16} & \delta^{20} \\
        1 & \delta^2 & \delta^4    & \delta^6    & \delta^8    & \delta^{10} \\
        1 & \delta^5 & \delta^{10} & \delta^{15} & \delta^{20} & \delta^{25} \\
    \end{pmatrix}
\]
Выделение блоков:
\[
    F_6
    = P_6
    \begin{pmatrix}
        \begin{pmatrix}
            1 & 1        \\
            1 & \delta^3
        \end{pmatrix}
        &
        \begin{pmatrix}
            1        & 1        \\
            \delta^6 & \delta^9
        \end{pmatrix}
        &
        \begin{pmatrix}
            1           & 1           \\
            \delta^{12} & \delta^{15}
        \end{pmatrix}
        \\
        \begin{pmatrix}
            1 & 1        \\
            1 & \delta^3
        \end{pmatrix}
        \begin{pmatrix}
            1 & 0      \\
            0 & \delta
        \end{pmatrix}
        &
        \begin{pmatrix}
            1        & 1        \\
            \delta^6 & \delta^9
        \end{pmatrix}
        \begin{pmatrix}
            \delta^2 & 0        \\
            0        & \delta^3
        \end{pmatrix}
        &
        \begin{pmatrix}
            1           & 1           \\
            \delta^{12} & \delta^{15}
        \end{pmatrix}
        \begin{pmatrix}
            \delta^4 & 0        \\
            0        & \delta^5
        \end{pmatrix}
        \\
        \begin{pmatrix}
            1 & 1        \\
            1 & \delta^3
        \end{pmatrix}
        \begin{pmatrix}
            1 & 0        \\
            0 & \delta^2
        \end{pmatrix}
        &
        \begin{pmatrix}
            1        & 1        \\
            \delta^6 & \delta^9
        \end{pmatrix}
        \begin{pmatrix}
            \delta^4 & 0        \\
            0        & \delta^6
        \end{pmatrix}
        &
        \begin{pmatrix}
            1           & 1           \\
            \delta^{12} & \delta^{15}
        \end{pmatrix}
        \begin{pmatrix}
            \delta^8 & 0           \\
            0        & \delta^{10}
        \end{pmatrix}
    \end{pmatrix}
\]
Сокращение степени:
\[
    F_6
    = P_6
    \begin{pmatrix}
        \begin{pmatrix}
            1 & 1        \\
            1 & \delta^3
        \end{pmatrix}
        &
        \begin{pmatrix}
            1 & 1        \\
            1 & \delta^3
        \end{pmatrix}
        &
        \begin{pmatrix}
            1 & 1        \\
            1 & \delta^3
        \end{pmatrix}
        \\
        \begin{pmatrix}
            1 & 1        \\
            1 & \delta^3
        \end{pmatrix}
        \begin{pmatrix}
            1 & 0      \\
            0 & \delta
        \end{pmatrix}
        &
        \begin{pmatrix}
            1 & 1        \\
            1 & \delta^3
        \end{pmatrix}
        \begin{pmatrix}
            \delta^2 & 0        \\
            0        & \delta^3
        \end{pmatrix}
        &
        \begin{pmatrix}
            1 & 1        \\
            1 & \delta^3
        \end{pmatrix}
        \begin{pmatrix}
            \delta^4 & 0        \\
            0        & \delta^5
        \end{pmatrix}
        \\
        \begin{pmatrix}
            1 & 1        \\
            1 & \delta^3
        \end{pmatrix}
        \begin{pmatrix}
            1 & 0        \\
            0 & \delta^2
        \end{pmatrix}
        &
        \begin{pmatrix}
            1 & 1        \\
            1 & \delta^3
        \end{pmatrix}
        \begin{pmatrix}
            \delta^4 & 0        \\
            0        & \delta^6
        \end{pmatrix}
        &
        \begin{pmatrix}
            1 & 1        \\
            1 & \delta^3
        \end{pmatrix}
        \begin{pmatrix}
            \delta^8 & 0           \\
            0        & \delta^{10}
        \end{pmatrix}
    \end{pmatrix}
\]
Пусть
\[
    F_2(\delta)
    = \begin{pmatrix}
          1 & 1      \\
          1 & \delta
    \end{pmatrix}
\]
--- матрица Фурье размера 2, тогда:
\[
    F_6
    = P_6
    \begin{pmatrix}
        F_2(\delta^3) & 0             & 0             \\
        0             & F_2(\delta^3) & 0             \\
        0             & 0             & F_2(\delta^3)
    \end{pmatrix}
    \begin{pmatrix}
        1 & 0        & 1        & 0        & 1        & 0           \\
        0 & 1        & 0        & 1        & 0        & 1           \\
        1 & 0        & \delta^2 & 0        & \delta^4 & 0           \\
        0 & \delta   & 0        & \delta^3 & 0        & \delta^5    \\
        1 & 0        & \delta^4 & 0        & \delta^8 & 0           \\
        0 & \delta^2 & 0        & \delta^6 & 0        & \delta^{10}
    \end{pmatrix}
\]
Матрица слева преобразуется перестановкой строк и столбцов к матрице (причём той же матрицей $P_6$!):
\begin{multline*}
    \begin{pmatrix}
        1 & 0        & 1        & 0        & 1        & 0           \\
        0 & 1        & 0        & 1        & 0        & 1           \\
        1 & 0        & \delta^2 & 0        & \delta^4 & 0           \\
        0 & \delta   & 0        & \delta^3 & 0        & \delta^5    \\
        1 & 0        & \delta^4 & 0        & \delta^8 & 0           \\
        0 & \delta^2 & 0        & \delta^6 & 0        & \delta^{10}
    \end{pmatrix}
    = P_6^T
    \begin{pmatrix}
        1 & 1        & 1        & 0        & 0        & 0           \\
        1 & \delta^2 & \delta^4 & 0        & 0        & 0           \\
        1 & \delta^4 & \delta^8 & 0        & 0        & 0           \\
        0 & 0        & 0        & 1        & 1        & 1           \\
        0 & 0        & 0        & \delta   & \delta^3 & \delta^5    \\
        0 & 0        & 0        & \delta^2 & \delta^6 & \delta^{10}
    \end{pmatrix}
    P_6 = \\
    %
    = P_6^T
    \begin{pmatrix}
        1 & 0 & 0 & 0 & 0      & 0        \\
        0 & 1 & 0 & 0 & 0      & 0        \\
        0 & 0 & 1 & 0 & 0      & 0        \\
        0 & 0 & 0 & 1 & 0      & 0        \\
        0 & 0 & 0 & 0 & \delta & 0        \\
        0 & 0 & 0 & 0 & 0      & \delta^2 \\
    \end{pmatrix}
    \begin{pmatrix}
        F_3(\delta^2) & 0             \\
        0             & F_3(\delta^2) \\
    \end{pmatrix}
    P_6
\end{multline*}
Таким образом,
\[
    F_6
    = P_6
    \begin{pmatrix}
        F_2(\delta^3) & 0             & 0             \\
        0             & F_2(\delta^3) & 0             \\
        0             & 0             & F_2(\delta^3)
    \end{pmatrix}
    P_6^T
    \begin{pmatrix}
        I_3 & 0   \\
        0   & D_3
    \end{pmatrix}
    \begin{pmatrix}
        F_3(\delta^2) & 0             \\
        0             & F_3(\delta^2) \\
    \end{pmatrix}
    P_6
\]

\subsection{6 и 3, с таблицами}

$H$ --- спектр сигнала $h$:
\begin{gather*}
    H = F_6(\delta) h , \\
    %
    \begin{pmatrix}
        H_0 \\
        H_1 \\
        H_2 \\
        H_3 \\
        H_4 \\
        H_5
    \end{pmatrix}
    = \begin{pmatrix}
          1 & 1        & 1           & 1           & 1           & 1           \\
          1 & \delta^1 & \delta^2    & \delta^3    & \delta^4    & \delta^5    \\
          1 & \delta^2 & \delta^4    & \delta^6    & \delta^8    & \delta^{10} \\
          1 & \delta^3 & \delta^6    & \delta^9    & \delta^{12} & \delta^{15} \\
          1 & \delta^4 & \delta^8    & \delta^{12} & \delta^{16} & \delta^{20} \\
          1 & \delta^5 & \delta^{10} & \delta^{15} & \delta^{20} & \delta^{25} \\
    \end{pmatrix}
    \begin{pmatrix}
        h_0 \\
        h_1 \\
        h_2 \\
        h_3 \\
        h_4 \\
        h_5
    \end{pmatrix}
\end{gather*}

Нумеруем строки от $0$, \dots, $5$, группируем строки по остаткам: сперва строки с остатком 0, потом строки с остатком 1, и так далее. Перестановка строк приводит к перестановке
элементов вектора спектра $H$:
\[
    \begin{pmatrix}
        H_0 \\
        H_3 \\
        H_1 \\
        H_4 \\
        H_2 \\
        H_5
    \end{pmatrix}
    =
    \underbrace{
        \begin{pmatrix}
            1 & 1        & 1           & 1           & 1           & 1           \\
            1 & \delta^3 & \delta^6    & \delta^9    & \delta^{12} & \delta^{15} \\
            1 & \delta^2 & \delta^4    & \delta^6    & \delta^8    & \delta^{10} \\
            1 & \delta^1 & \delta^2    & \delta^3    & \delta^4    & \delta^5    \\
            1 & \delta^4 & \delta^8    & \delta^{12} & \delta^{16} & \delta^{20} \\
            1 & \delta^5 & \delta^{10} & \delta^{15} & \delta^{20} & \delta^{25} \\
        \end{pmatrix}
    }_{\widetilde{F}_6}
    \begin{pmatrix}
        h_0 \\
        h_1 \\
        h_2 \\
        h_3 \\
        h_4 \\
        h_5
    \end{pmatrix}
\]
Выделение блоков, сокращение степени, вынесение общих множителей в строках:
\begin{multline*}
    \widetilde{F}_6
    =
    \begin{pmatrix}
        \begin{pmatrix}
            1 & 1        \\
            1 & \delta^3
        \end{pmatrix}
        &
        \begin{pmatrix}
            1        & 1        \\
            \delta^6 & \delta^9
        \end{pmatrix}
        &
        \begin{pmatrix}
            1           & 1           \\
            \delta^{12} & \delta^{15}
        \end{pmatrix}
        \\
        \begin{pmatrix}
            1 & 1        \\
            1 & \delta^3
        \end{pmatrix}
        \begin{pmatrix}
            1 & 0      \\
            0 & \delta
        \end{pmatrix}
        &
        \begin{pmatrix}
            1        & 1        \\
            \delta^6 & \delta^9
        \end{pmatrix}
        \begin{pmatrix}
            \delta^2 & 0        \\
            0        & \delta^3
        \end{pmatrix}
        &
        \begin{pmatrix}
            1           & 1           \\
            \delta^{12} & \delta^{15}
        \end{pmatrix}
        \begin{pmatrix}
            \delta^4 & 0        \\
            0        & \delta^5
        \end{pmatrix}
        \\
        \begin{pmatrix}
            1 & 1        \\
            1 & \delta^3
        \end{pmatrix}
        \begin{pmatrix}
            1 & 0        \\
            0 & \delta^2
        \end{pmatrix}
        &
        \begin{pmatrix}
            1        & 1        \\
            \delta^6 & \delta^9
        \end{pmatrix}
        \begin{pmatrix}
            \delta^4 & 0        \\
            0        & \delta^6
        \end{pmatrix}
        &
        \begin{pmatrix}
            1           & 1           \\
            \delta^{12} & \delta^{15}
        \end{pmatrix}
        \begin{pmatrix}
            \delta^8 & 0           \\
            0        & \delta^{10}
        \end{pmatrix}
    \end{pmatrix} = \\
    %
    =
    \begin{pmatrix}
        \begin{pmatrix}
            1 & 1        \\
            1 & \delta^3
        \end{pmatrix}
        &
        \begin{pmatrix}
            1 & 1        \\
            1 & \delta^3
        \end{pmatrix}
        &
        \begin{pmatrix}
            1 & 1        \\
            1 & \delta^3
        \end{pmatrix}
        \\
        \begin{pmatrix}
            1 & 1        \\
            1 & \delta^3
        \end{pmatrix}
        \begin{pmatrix}
            1 & 0      \\
            0 & \delta
        \end{pmatrix}
        &
        \begin{pmatrix}
            1 & 1        \\
            1 & \delta^3
        \end{pmatrix}
        \begin{pmatrix}
            \delta^2 & 0        \\
            0        & \delta^3
        \end{pmatrix}
        &
        \begin{pmatrix}
            1 & 1        \\
            1 & \delta^3
        \end{pmatrix}
        \begin{pmatrix}
            \delta^4 & 0        \\
            0        & \delta^5
        \end{pmatrix}
        \\
        \begin{pmatrix}
            1 & 1        \\
            1 & \delta^3
        \end{pmatrix}
        \begin{pmatrix}
            1 & 0        \\
            0 & \delta^2
        \end{pmatrix}
        &
        \begin{pmatrix}
            1 & 1        \\
            1 & \delta^3
        \end{pmatrix}
        \begin{pmatrix}
            \delta^4 & 0        \\
            0        & \delta^6
        \end{pmatrix}
        &
        \begin{pmatrix}
            1 & 1        \\
            1 & \delta^3
        \end{pmatrix}
        \begin{pmatrix}
            \delta^8 & 0           \\
            0        & \delta^{10}
        \end{pmatrix}
    \end{pmatrix} = \\
    %
    =
    \begin{pmatrix}
        \begin{pmatrix}
            1 & 1        \\
            1 & \delta^3
        \end{pmatrix}
        &
        0
        &
        0
        \\
        0
        &
        \begin{pmatrix}
            1 & 1        \\
            1 & \delta^3
        \end{pmatrix}
        &
        0
        \\
        0
        &
        0
        &
        \begin{pmatrix}
            1 & 1        \\
            1 & \delta^3
        \end{pmatrix}
    \end{pmatrix}
    \begin{pmatrix}
        1 & 0        & 1        & 0        & 1        & 0           \\
        0 & 1        & 0        & 1        & 0        & 1           \\
        1 & 0        & \delta^2 & 0        & \delta^4 & 0           \\
        0 & \delta   & 0        & \delta^3 & 0        & \delta^5    \\
        1 & 0        & \delta^4 & 0        & \delta^8 & 0           \\
        0 & \delta^2 & 0        & \delta^6 & 0        & \delta^{10} \\
    \end{pmatrix} = \\
    %
    =
    \begin{pmatrix}
        F_2(\delta^3) & 0             & 0             \\
        0             & F_2(\delta^3) & 0             \\
        0             & 0             & F_2(\delta^3)
    \end{pmatrix}
    \begin{pmatrix}
        1 & 0        & 1        & 0        & 1        & 0           \\
        0 & 1        & 0        & 1        & 0        & 1           \\
        1 & 0        & \delta^2 & 0        & \delta^4 & 0           \\
        0 & \delta   & 0        & \delta^3 & 0        & \delta^5    \\
        1 & 0        & \delta^4 & 0        & \delta^8 & 0           \\
        0 & \delta^2 & 0        & \delta^6 & 0        & \delta^{10} \\
    \end{pmatrix} ,
\end{multline*}
где $F_2(\delta^3)$ --- матрица Фурье размера 2:
\[
    F_2(\delta^3)
    = \begin{pmatrix}
          1 & 1        \\
          1 & \delta^3
    \end{pmatrix}
\]
Таким образом,
\[
    \begin{pmatrix}
        H_0 \\
        H_3 \\
        H_1 \\
        H_4 \\
        H_2 \\
        H_5
    \end{pmatrix}
    =
    \begin{pmatrix}
        F_2(\delta^3) & 0             & 0             \\
        0             & F_2(\delta^3) & 0             \\
        0             & 0             & F_2(\delta^3)
    \end{pmatrix}
    \begin{pmatrix}
        1 & 0        & 1        & 0        & 1        & 0           \\
        0 & 1        & 0        & 1        & 0        & 1           \\
        1 & 0        & \delta^2 & 0        & \delta^4 & 0           \\
        0 & \delta   & 0        & \delta^3 & 0        & \delta^5    \\
        1 & 0        & \delta^4 & 0        & \delta^8 & 0           \\
        0 & \delta^2 & 0        & \delta^6 & 0        & \delta^{10} \\
    \end{pmatrix}
    \begin{pmatrix}
        h_0 \\
        h_1 \\
        h_2 \\
        h_3 \\
        h_4 \\
        h_5
    \end{pmatrix}
\]
Пусть вектор справа
\[
    \begin{pmatrix}
        \widetilde{H}_0 \\
        \widetilde{H}_1 \\
        \widetilde{H}_2 \\
        \widetilde{H}_3 \\
        \widetilde{H}_4 \\
        \widetilde{H}_5
    \end{pmatrix}
    =
    \begin{pmatrix}
        1 & 0        & 1        & 0        & 1        & 0           \\
        0 & 1        & 0        & 1        & 0        & 1           \\
        1 & 0        & \delta^2 & 0        & \delta^4 & 0           \\
        0 & \delta   & 0        & \delta^3 & 0        & \delta^5    \\
        1 & 0        & \delta^4 & 0        & \delta^8 & 0           \\
        0 & \delta^2 & 0        & \delta^6 & 0        & \delta^{10} \\
    \end{pmatrix}
    \begin{pmatrix}
        h_0 \\
        h_1 \\
        h_2 \\
        h_3 \\
        h_4 \\
        h_5
    \end{pmatrix}
\]
тогда
\[
    \begin{pmatrix}
        H_0 & H_1 & H_2 \\
        H_3 & H_4 & H_5
    \end{pmatrix}
    =
    F_2(\delta^3)
    \begin{pmatrix}
        \widetilde{H}_0 & \widetilde{H}_2 & \widetilde{H}_4 \\
        \widetilde{H}_1 & \widetilde{H}_3 & \widetilde{H}_5
    \end{pmatrix} .
\]
Вектор слева записан по строкам, а вектор справа по столбцам. Столбцы вектора слева являются преобразованиями Фурье векторов справа.

Нам осталось вычислить вектор $\widetilde{H}$. В его вычислении тоже можно увидеть преобразования Фурье, если переставить строки
\[
    \begin{pmatrix}
        \widetilde{H}_0 \\
        \widetilde{H}_2 \\
        \widetilde{H}_4 \\
        \widetilde{H}_1 \\
        \widetilde{H}_3 \\
        \widetilde{H}_5
    \end{pmatrix}
    =
    \begin{pmatrix}
        1 & 0        & 1        & 0        & 1        & 0           \\
        1 & 0        & \delta^2 & 0        & \delta^4 & 0           \\
        1 & 0        & \delta^4 & 0        & \delta^8 & 0           \\
        0 & 1        & 0        & 1        & 0        & 1           \\
        0 & \delta   & 0        & \delta^3 & 0        & \delta^5    \\
        0 & \delta^2 & 0        & \delta^6 & 0        & \delta^{10} \\
    \end{pmatrix}
    \begin{pmatrix}
        h_0 \\
        h_1 \\
        h_2 \\
        h_3 \\
        h_4 \\
        h_5
    \end{pmatrix}
\]
и переставить столбцы:
\[
    \begin{pmatrix}
        \widetilde{H}_0 \\
        \widetilde{H}_2 \\
        \widetilde{H}_4 \\
        \widetilde{H}_1 \\
        \widetilde{H}_3 \\
        \widetilde{H}_5
    \end{pmatrix}
    =
    \begin{pmatrix}
        1 & 1        & 1        & 0        & 0        & 0           \\
        1 & \delta^2 & \delta^4 & 0        & 0        & 0           \\
        1 & \delta^4 & \delta^8 & 0        & 0        & 0           \\
        0 & 0        & 0        & 1        & 1        & 1           \\
        0 & 0        & 0        & \delta   & \delta^3 & \delta^5    \\
        0 & 0        & 0        & \delta^2 & \delta^6 & \delta^{10} \\
    \end{pmatrix}
    \begin{pmatrix}
        h_0 \\
        h_2 \\
        h_4 \\
        h_1 \\
        h_3 \\
        h_5
    \end{pmatrix} .
\]
Выносим множители из последних двух строк:
\[
    \begin{pmatrix}
        \widetilde{H}_0 \\
        \widetilde{H}_2 \\
        \widetilde{H}_4 \\
        \widetilde{H}_1 \\
        \widetilde{H}_3 \\
        \widetilde{H}_5
    \end{pmatrix}
    =
    \begin{pmatrix}
        1 & 0 & 0 & 0 & 0      & 0        \\
        0 & 1 & 0 & 0 & 0      & 0        \\
        0 & 0 & 1 & 0 & 0      & 0        \\
        0 & 0 & 0 & 1 & 0      & 0        \\
        0 & 0 & 0 & 0 & \delta & 0        \\
        0 & 0 & 0 & 0 & 0      & \delta^2
    \end{pmatrix}
    \begin{pmatrix}
        1 & 1        & 1        & 0 & 0        & 0        \\
        1 & \delta^2 & \delta^4 & 0 & 0        & 0        \\
        1 & \delta^4 & \delta^8 & 0 & 0        & 0        \\
        0 & 0        & 0        & 1 & 1        & 1        \\
        0 & 0        & 0        & 1 & \delta^2 & \delta^4 \\
        0 & 0        & 0        & 1 & \delta^4 & \delta^8 \\
    \end{pmatrix}
    \begin{pmatrix}
        h_0 \\
        h_2 \\
        h_4 \\
        h_1 \\
        h_3 \\
        h_5
    \end{pmatrix} .
\]
Видно, что в матрице справа получились два одинаковых блока, которая являются матрицами Фурье размера 3:
\[
    F_3(\delta^2)
    = \begin{pmatrix}
          1 & 1        & 1        \\
          1 & \delta^2 & \delta^4 \\
          1 & \delta^4 & \delta^8 \\
    \end{pmatrix} .
\]
Таким образом,
\[
    \begin{pmatrix}
        \widetilde{H}_0 & \widetilde{H}_1 \\
        \widetilde{H}_2 & \widetilde{H}_3 \\
        \widetilde{H}_4 & \widetilde{H}_5
    \end{pmatrix}
    =
    \begin{pmatrix}
        1 & 1        \\
        1 & \delta   \\
        1 & \delta^2
    \end{pmatrix}
    \circ
    F_3(\delta^2)
    \begin{pmatrix}
        h_0 & h_1 \\
        h_2 & h_3 \\
        h_4 & h_5
    \end{pmatrix} ,
\]
где $\circ$ обозначает поэлементное умножение. Транспонируем для того, чтобы получилось как в правой части вычисления спектра:
\[
    \begin{pmatrix}
        \widetilde{H}_0 & \widetilde{H}_2 & \widetilde{H}_4 \\
        \widetilde{H}_1 & \widetilde{H}_3 & \widetilde{H}_5
    \end{pmatrix}
    =
    \begin{pmatrix}
        1 & 1      & 1        \\
        1 & \delta & \delta^2
    \end{pmatrix}
    \circ
    \begin{pmatrix}
        h_0 & h_2 & h_4 \\
        h_1 & h_3 & h_5
    \end{pmatrix}
    F_3(\delta^2)
\]
и всё вместе:
\[
    \begin{pmatrix}
        H_0 & H_1 & H_2 \\
        H_3 & H_4 & H_5
    \end{pmatrix}
    =
    F_2(\delta^3)
    \left(
    \begin{pmatrix}
        1 & 1      & 1        \\
        1 & \delta & \delta^2
    \end{pmatrix}
    \circ
    \begin{pmatrix}
        h_0 & h_2 & h_4 \\
        h_1 & h_3 & h_5
    \end{pmatrix}
    F_3(\delta^2)
    \right)
\]

\subsection{6 и 2}

Матрица степеней
\[
    F_6
    = \begin{pmatrix}
          1 & 1        & 1           & 1           & 1           & 1           \\
          1 & \delta^1 & \delta^2    & \delta^3    & \delta^4    & \delta^5    \\
          1 & \delta^2 & \delta^4    & \delta^6    & \delta^8    & \delta^{10} \\
          1 & \delta^3 & \delta^6    & \delta^9    & \delta^{12} & \delta^{15} \\
          1 & \delta^4 & \delta^8    & \delta^{12} & \delta^{16} & \delta^{20} \\
          1 & \delta^5 & \delta^{10} & \delta^{15} & \delta^{20} & \delta^{25} \\
    \end{pmatrix}
\]
Перестановка строк ($P_6$ --- обратная перестановка строк):
\[
    F_6
    = P_6
    \begin{pmatrix}
        1 & 1        & 1           & 1           & 1           & 1           \\
        1 & \delta^2 & \delta^4    & \delta^6    & \delta^8    & \delta^{10} \\
        1 & \delta^4 & \delta^8    & \delta^{12} & \delta^{16} & \delta^{20} \\
        1 & \delta^1 & \delta^2    & \delta^3    & \delta^4    & \delta^5    \\
        1 & \delta^3 & \delta^6    & \delta^9    & \delta^{12} & \delta^{15} \\
        1 & \delta^5 & \delta^{10} & \delta^{15} & \delta^{20} & \delta^{25} \\
    \end{pmatrix}
\]
Выделение блоков:
\[
    F_6
    = P_6
    \begin{pmatrix}
        \begin{pmatrix}
            1 & 1        & 1        \\
            1 & \delta^2 & \delta^4 \\
            1 & \delta^4 & \delta^8
        \end{pmatrix}
        &
        \begin{pmatrix}
            1           & 1           & 1           \\
            \delta^6    & \delta^8    & \delta^{10} \\
            \delta^{12} & \delta^{16} & \delta^{20}
        \end{pmatrix}
        \\
        \begin{pmatrix}
            1 & 1        & 1        \\
            1 & \delta^2 & \delta^4 \\
            1 & \delta^4 & \delta^8
        \end{pmatrix}
        \begin{pmatrix}
            1 & 0      & 0        \\
            0 & \delta & 0        \\
            0 & 0      & \delta^2
        \end{pmatrix}
        &
        \begin{pmatrix}
            1           & 1           & 1           \\
            \delta^6    & \delta^8    & \delta^{10} \\
            \delta^{12} & \delta^{16} & \delta^{20}
        \end{pmatrix}
        \begin{pmatrix}
            \delta^3 & 0        & 0        \\
            0        & \delta^4 & 0        \\
            0        & 0        & \delta^5
        \end{pmatrix}
    \end{pmatrix}
\]
Сокращение степени в силу $\delta^6 = 1$:
\[
    F_6
    = P_6
    \begin{pmatrix}
        \begin{pmatrix}
            1 & 1        & 1        \\
            1 & \delta^2 & \delta^4 \\
            1 & \delta^4 & \delta^8
        \end{pmatrix}
        &
        \begin{pmatrix}
            1 & 1        & 1        \\
            1 & \delta^2 & \delta^4 \\
            1 & \delta^4 & \delta^8
        \end{pmatrix}
        \\
        \begin{pmatrix}
            1 & 1        & 1        \\
            1 & \delta^2 & \delta^4 \\
            1 & \delta^4 & \delta^8
        \end{pmatrix}
        \begin{pmatrix}
            1 & 0      & 0        \\
            0 & \delta & 0        \\
            0 & 0      & \delta^2
        \end{pmatrix}
        &
        \begin{pmatrix}
            1 & 1        & 1        \\
            1 & \delta^2 & \delta^4 \\
            1 & \delta^4 & \delta^8
        \end{pmatrix}
        \begin{pmatrix}
            \delta^3 & 0        & 0        \\
            0        & \delta^4 & 0        \\
            0        & 0        & \delta^5
        \end{pmatrix}
    \end{pmatrix}
\]
Матрица
\[
    \begin{pmatrix}
        \delta^3 & 0        & 0        \\
        0        & \delta^4 & 0        \\
        0        & 0        & \delta^5
    \end{pmatrix}
    = \delta^3
    \begin{pmatrix}
        1 & 0        & 0        \\
        0 & \delta^1 & 0        \\
        0 & 0        & \delta^2
    \end{pmatrix}
    = (-1)
    \begin{pmatrix}
        1 & 0      & 0        \\
        0 & \delta & 0        \\
        0 & 0      & \delta^2
    \end{pmatrix}
\]
Таким образом,
\[
    F_6
    = P_6
    \begin{pmatrix}
        \begin{pmatrix}
            1 & 1        & 1        \\
            1 & \delta^2 & \delta^4 \\
            1 & \delta^4 & \delta^8
        \end{pmatrix}
        &
        \begin{pmatrix}
            1 & 1        & 1        \\
            1 & \delta^2 & \delta^4 \\
            1 & \delta^4 & \delta^8
        \end{pmatrix}
        \\
        \begin{pmatrix}
            1 & 1        & 1        \\
            1 & \delta^2 & \delta^4 \\
            1 & \delta^4 & \delta^8
        \end{pmatrix}
        \begin{pmatrix}
            1 & 0      & 0        \\
            0 & \delta & 0        \\
            0 & 0      & \delta^2
        \end{pmatrix}
        &
        -
        \begin{pmatrix}
            1 & 1        & 1        \\
            1 & \delta^2 & \delta^4 \\
            1 & \delta^4 & \delta^8
        \end{pmatrix}
        \begin{pmatrix}
            1 & 0      & 0        \\
            0 & \delta & 0        \\
            0 & 0      & \delta^2
        \end{pmatrix}
    \end{pmatrix}
\]
Пусть
\[
    F_3(\delta^2)
    = \begin{pmatrix}
          1 & 1        & 1        \\
          1 & \delta^2 & \delta^4 \\
          1 & \delta^4 & \delta^8
    \end{pmatrix},
    \;
    D_3
    = \begin{pmatrix}
          1 & 0      & 0        \\
          0 & \delta & 0        \\
          0 & 0      & \delta^2
    \end{pmatrix} ,
\]
где $F_3(\delta)$ --- матрица Фурье размера 3, тогда
\begin{multline*}
    F_6(\delta)
    = P_6
    \begin{pmatrix}
        F_3(\delta^2)     & F_3(\delta^2)      \\
        F_3(\delta^2) D_3 & -F_3(\delta^2) D_3 \\
    \end{pmatrix} = \\
    %
    = P_6
    \begin{pmatrix}
        F_3(\delta^2) & 0             \\
        0             & F_3(\delta^2) \\
    \end{pmatrix}
    \begin{pmatrix}
        I_3 & I_3  \\
        D_3 & -D_3 \\
    \end{pmatrix} = \\
    %
    = P_6
    \begin{pmatrix}
        F_3(\delta^2) & 0             \\
        0             & F_3(\delta^2) \\
    \end{pmatrix}
    \begin{pmatrix}
        I_3 & 0   \\
        0   & D_3 \\
    \end{pmatrix}
    \begin{pmatrix}
        I_3 & I_3  \\
        I_3 & -I_3 \\
    \end{pmatrix}
\end{multline*}

\subsection{4 и 2}

Исходная матрица Фурье:
\[
    F_4(\varepsilon)
    = \begin{pmatrix}
          1 & 1                & 1                & 1                \\
          1 & \varepsilon^{-1} & \varepsilon^{-2} & \varepsilon^{-3} \\
          1 & \varepsilon^{-2} & \varepsilon^{-4} & \varepsilon^{-6} \\
          1 & \varepsilon^{-3} & \varepsilon^{-6} & \varepsilon^{-9}
    \end{pmatrix}
    = \begin{pmatrix}
          1 & 1                & 1                                 & 1                                 \\
          1 & \varepsilon^{-1} & \varepsilon^{-2}                  & \varepsilon^{-3}                  \\
          1 & \varepsilon^{-2} & \varepsilon^{-4}                  & \varepsilon^{-4} \varepsilon^{-2} \\
          1 & \varepsilon^{-3} & \varepsilon^{-4} \varepsilon^{-2} & \varepsilon^{-4} \varepsilon^{-5}
    \end{pmatrix}
\]
Сокращение степени с учётом $\varepsilon^{-4} = 1$:
\[
    F_4(\varepsilon)
    = \begin{pmatrix}
          1 & 1                & 1                & 1                \\
          1 & \varepsilon^{-1} & \varepsilon^{-2} & \varepsilon^{-3} \\
          1 & \varepsilon^{-2} & 1                & \varepsilon^{-2} \\
          1 & \varepsilon^{-3} & \varepsilon^{-2} & \varepsilon^{-5}
    \end{pmatrix}
\]
Перестановка строк, сперва все нечётные строки, затем все чётные:
\[
    F_4(\varepsilon)
    =
    P_4
    \begin{pmatrix}
        1 & 1                & 1                & 1                \\
        1 & \varepsilon^{-2} & 1                & \varepsilon^{-2} \\
        1 & \varepsilon^{-1} & \varepsilon^{-2} & \varepsilon^{-3} \\
        1 & \varepsilon^{-3} & \varepsilon^{-2} & \varepsilon^{-5}
    \end{pmatrix} ,
\]
где $P_4$ --- матрица, выполняющая обратную перестановку строк.

Выделение подматриц:
\[
    F_4(\varepsilon)
    = P_4 \begin{pmatrix}
              \begin{pmatrix}
                  1 & 1                \\
                  1 & \varepsilon^{-2}
              \end{pmatrix}
              &
              \begin{pmatrix}
                  1 & 1                \\
                  1 & \varepsilon^{-2}
              \end{pmatrix} \\
              \begin{pmatrix}
                  1 & 1                \\
                  1 & \varepsilon^{-2}
              \end{pmatrix}
              \begin{pmatrix}
                  1 & 0                \\
                  0 & \varepsilon^{-1}
              \end{pmatrix}
              &
              \varepsilon^{-2}
              \begin{pmatrix}
                  1 & 1                \\
                  1 & \varepsilon^{-2}
              \end{pmatrix}
              \begin{pmatrix}
                  1 & 0                \\
                  0 & \varepsilon^{-1}
              \end{pmatrix}
    \end{pmatrix} ,
\]
где
\[
    \varepsilon^{-2}
    = \left( e^{i \frac{2 \pi}{4}} \right)^2
    = e^{i \frac{2 \pi}{2}}
    = e^{i \pi}
    = -1 ,
\]
поэтому
\[
    F_4(\varepsilon)
    = P_4
    \begin{pmatrix}
        F_2(\varepsilon^2)     & F_2(\varepsilon^2)       \\
        F_2(\varepsilon^2) D_2 & - F_2(\varepsilon^2) D_2
    \end{pmatrix} ,
\]
где $F_2(\delta)$ --- матрица Фурье порядка 2 и $D_2$ --- диагональная матрица:
\[
    F_2(\delta)
    = \begin{pmatrix}
          1 & 1           \\
          1 & \delta^{-1}
    \end{pmatrix} ,
    \;
    D_2
    = \begin{pmatrix}
          1 & 0                \\
          0 & \varepsilon^{-1}
    \end{pmatrix} .
\]
Представим, что нужно умножить матрицу $F_4(\varepsilon)$ на некоторый вектор $h$ (как для вычисления спектра), и посмотрим как будет выглядеть умножение:
\begin{multline*}
    F_4(\varepsilon) h
    = P_4
    \begin{pmatrix}
        F_2(\varepsilon^2)     & F_2(\varepsilon^2)       \\
        F_2(\varepsilon^2) D_2 & - F_2(\varepsilon^2) D_2
    \end{pmatrix}
    \begin{pmatrix}
        h_0 \\
        h_1 \\
        h_2 \\
        h_3
    \end{pmatrix}
    = P_4
    \begin{pmatrix}
        F_2(\varepsilon^2)
        \begin{pmatrix}
            h_0 \\
            h_1
        \end{pmatrix}
        +
        F_2(\varepsilon^2)
        \begin{pmatrix}
            h_2 \\
            h_3
        \end{pmatrix} \\
        F_2(\varepsilon^2) D_2
        \begin{pmatrix}
            h_0 \\
            h_1
        \end{pmatrix}
        - F_2(\varepsilon^2) D_2
        \begin{pmatrix}
            h_2 \\
            h_3
        \end{pmatrix}
    \end{pmatrix}
    = \\
%
    = P_4
    \begin{pmatrix}
        F_2(\varepsilon^2)
        \left(
        \begin{pmatrix}
            h_0 \\
            h_1
        \end{pmatrix}
        +
        \begin{pmatrix}
            h_2 \\
            h_3
        \end{pmatrix}
        \right) \\
        F_2(\varepsilon^2) D_2
        \left(
        \begin{pmatrix}
            h_0 \\
            h_1
        \end{pmatrix}
        -
        \begin{pmatrix}
            h_2 \\
            h_3
        \end{pmatrix}
        \right)
    \end{pmatrix} = \\
%
    = P_4
    \begin{pmatrix}
        F_2(\varepsilon^2) \\
        F_2(\varepsilon^2) D_2
    \end{pmatrix}
    \begin{pmatrix}
        I_2 & I_2  \\
        I_2 & -I_2
    \end{pmatrix}
    \begin{pmatrix}
        h_0 \\
        h_1 \\
        h_2 \\
        h_3
    \end{pmatrix}
    = P_4
    \begin{pmatrix}
        F_2(\varepsilon^2) & 0                  \\
        0                  & F_2(\varepsilon^2)
    \end{pmatrix}
    \begin{pmatrix}
        I_2 & 0   \\
        0   & D_2
    \end{pmatrix}
    \begin{pmatrix}
        I_2 & I_2  \\
        I_2 & -I_2
    \end{pmatrix}
    \begin{pmatrix}
        h_0 \\
        h_1 \\
        h_2 \\
        h_3
    \end{pmatrix}
\end{multline*}
Для умножения на матрицу $F_4(\varepsilon)$ необходимо сделать $\frac{n}{2}$ сложений ($I_2 + I_2$), $\frac{n}{2}$ вычитаний ($I_2 - I_2$), $\frac{n}{2}$ умножений на элементы
диагональной матрицы $D_2$ и потом дважды умножить на матрицу $F_2(\varepsilon^2)$ размерности $\frac{n}{2}$.

Пусть $n = 2^L$, тогда $L = \log_2 n$, пусть $S_\pm(n)$ --- общее количество сложений и вычитаний и $S_*(n)$ --- общее количество умножений, тогда:
\begin{gather*}
    S_\pm(n)
    = n + 2 \frac{n}{2} + 4 \frac{n}{4} + \dots + 2^{L-1} \frac{n}{2^{L-1}}
    = n + n + n + \dots + n
    = n \cdot L
    = n \log_2 n, \\
%
    S_*(n)
    = \frac{n}{2} + 2 \frac{n}{2 \cdot 2} + 4 \frac{n}{4 \cdot 2} + \dots + 2^{L-1} \frac{n}{2^{L-1} \cdot 2}
    = \frac{n}{2} + \frac{n}{2} + \frac{n}{2} + \dots + \frac{n}{2}
    = \frac{n}{2} \cdot L
    = \frac{1}{2} n \log_2 n .
\end{gather*}

Выражение для $S_\pm(n)$ получается следующим образом: для редукции задачи размера $n$ нужно сделать $\frac{n}{2} + \frac{n}{2} = n$ сложений-вычитаний,
в результате получатся 2 задачи размера $\frac{n}{2}$, каждая из которых при редукции потребует $\frac{\frac{n}{2}}{2} + \frac{\frac{n}{2}}{2} = \frac{n}{2}$, поэтому для редукции
двух задач размера $\frac{n}{2}$ потребуется $2 \frac{n}{2}$ сложений-вычитаний, далее образуется уже 4 задачи размера $\frac{n}{4}$, редукция которых требует $4 \frac{n}{4}$
сложений-вычитаний и так далее пока не достигнем размера задачи 2, таких задачи будет $2^{L-1}$ и они будут заключаться в умножении на матрицу Фурье $F_2$, при котором нужно будет
сделать два сложения, или $\frac{n}{2^{L-1}} = \frac{2^L}{2^{L-1}} = 2$.

Аналогично для $S_*(n)$: редукция задачи размера $n$ требует $\frac{n}{2}$ умножений, получаются 2 задачи, редукция каждой из которых требует $\frac{\frac{n}{2}}{2}$ умножений, далее
получаются 4 задачи, редукция каждой из которых требует $\frac{\frac{n}{4}}{2}$ умножений, и так далее пока не будет достигнут размер задачи 2, таких задач $2^{L-1}$ и в силу
вида матрицы $F_2$ потребуется только одно умножение, $\frac{n}{2^{L-1} \cdot 2} = \frac{2^L}{2^{L-1} \cdot 2} = 1$.

\subsection{8 и 2}

Матрица Фурье 8.
\[
    F_8
    = \begin{pmatrix}
          1 & 1                & 1                 & 1                 & 1                 & 1                 & 1                 & 1                 \\
          1 & \varepsilon^{-1} & \varepsilon^{-2}  & \varepsilon^{-3}  & \varepsilon^{-4}  & \varepsilon^{-5}  & \varepsilon^{-6}  & \varepsilon^{-7} \\
          1 & \varepsilon^{-2} & \varepsilon^{-4}  & \varepsilon^{-6}  & \varepsilon^{-8}  & \varepsilon^{-10} & \varepsilon^{-12} & \varepsilon^{-14} \\
          1 & \varepsilon^{-3} & \varepsilon^{-6}  & \varepsilon^{-9}  & \varepsilon^{-12} & \varepsilon^{-15} & \varepsilon^{-18} & \varepsilon^{-21} \\
          1 & \varepsilon^{-4} & \varepsilon^{-8}  & \varepsilon^{-12} & \varepsilon^{-16} & \varepsilon^{-20} & \varepsilon^{-24} & \varepsilon^{-28} \\
          1 & \varepsilon^{-5} & \varepsilon^{-10} & \varepsilon^{-15} & \varepsilon^{-20} & \varepsilon^{-25} & \varepsilon^{-30} & \varepsilon^{-35} \\
          1 & \varepsilon^{-6} & \varepsilon^{-12} & \varepsilon^{-18} & \varepsilon^{-24} & \varepsilon^{-30} & \varepsilon^{-36} & \varepsilon^{-42} \\
          1 & \varepsilon^{-7} & \varepsilon^{-14} & \varepsilon^{-21} & \varepsilon^{-28} & \varepsilon^{-35} & \varepsilon^{-42} & \varepsilon^{-49}
    \end{pmatrix}
\]

Сокращение степени.
\[
    F_8
    = \begin{pmatrix}
          1 & 1                & 1                 & 1                 & 1                & 1                 & 1                 & 1                 \\
          1 & \varepsilon^{-1} & \varepsilon^{-2}  & \varepsilon^{-3}  & \varepsilon^{-4} & \varepsilon^{-5}  & \varepsilon^{-6}  & \varepsilon^{-7} \\
          1 & \varepsilon^{-2} & \varepsilon^{-4}  & \varepsilon^{-6}  & 1                & \varepsilon^{-2}  & \varepsilon^{-4}  & \varepsilon^{-6} \\
          1 & \varepsilon^{-3} & \varepsilon^{-6}  & \varepsilon^{-9}  & \varepsilon^{-4} & \varepsilon^{-7}  & \varepsilon^{-10} & \varepsilon^{-13} \\
          1 & \varepsilon^{-4} & \varepsilon^{-8}  & \varepsilon^{-12} & 1                & \varepsilon^{-4}  & \varepsilon^{-8}  & \varepsilon^{-12} \\
          1 & \varepsilon^{-5} & \varepsilon^{-10} & \varepsilon^{-15} & \varepsilon^{-4} & \varepsilon^{-9}  & \varepsilon^{-14} & \varepsilon^{-19} \\
          1 & \varepsilon^{-6} & \varepsilon^{-12} & \varepsilon^{-18} & 1                & \varepsilon^{-6}  & \varepsilon^{-12} & \varepsilon^{-18} \\
          1 & \varepsilon^{-7} & \varepsilon^{-14} & \varepsilon^{-21} & \varepsilon^{-4} & \varepsilon^{-11} & \varepsilon^{-18} & \varepsilon^{-25}
    \end{pmatrix}
\]

Перестановка строк: сперва нечётные, затем чётные ($P_8$ --- матрица обратной перестановки строк).
\[
    F_8
    = P_8
    \begin{pmatrix}
        1 & 1                & 1                 & 1                 & 1                & 1                 & 1                 & 1                 \\
        1 & \varepsilon^{-2} & \varepsilon^{-4}  & \varepsilon^{-6}  & 1                & \varepsilon^{-2}  & \varepsilon^{-4}  & \varepsilon^{-6} \\
        1 & \varepsilon^{-4} & \varepsilon^{-8}  & \varepsilon^{-12} & 1                & \varepsilon^{-4}  & \varepsilon^{-8}  & \varepsilon^{-12} \\
        1 & \varepsilon^{-6} & \varepsilon^{-12} & \varepsilon^{-18} & 1                & \varepsilon^{-6}  & \varepsilon^{-12} & \varepsilon^{-18} \\
        1 & \varepsilon^{-1} & \varepsilon^{-2}  & \varepsilon^{-3}  & \varepsilon^{-4} & \varepsilon^{-5}  & \varepsilon^{-6}  & \varepsilon^{-7} \\
        1 & \varepsilon^{-3} & \varepsilon^{-6}  & \varepsilon^{-9}  & \varepsilon^{-4} & \varepsilon^{-7}  & \varepsilon^{-10} & \varepsilon^{-13} \\
        1 & \varepsilon^{-5} & \varepsilon^{-10} & \varepsilon^{-15} & \varepsilon^{-4} & \varepsilon^{-9}  & \varepsilon^{-14} & \varepsilon^{-19} \\
        1 & \varepsilon^{-7} & \varepsilon^{-14} & \varepsilon^{-21} & \varepsilon^{-4} & \varepsilon^{-11} & \varepsilon^{-18} & \varepsilon^{-25}
    \end{pmatrix}
\]

Выделение подматриц.
\[
    F_8
    = \begin{pmatrix}
          \begin{pmatrix}
              1 & 1                & 1                 & 1                 \\
              1 & \varepsilon^{-2} & \varepsilon^{-4}  & \varepsilon^{-6}  \\
              1 & \varepsilon^{-4} & \varepsilon^{-8}  & \varepsilon^{-12} \\
              1 & \varepsilon^{-6} & \varepsilon^{-12} & \varepsilon^{-18} \\
          \end{pmatrix}
          &
          \begin{pmatrix}
              1 & 1                & 1                 & 1                 \\
              1 & \varepsilon^{-2} & \varepsilon^{-4}  & \varepsilon^{-6}  \\
              1 & \varepsilon^{-4} & \varepsilon^{-8}  & \varepsilon^{-12} \\
              1 & \varepsilon^{-6} & \varepsilon^{-12} & \varepsilon^{-18} \\
          \end{pmatrix} \\
%
          \begin{pmatrix}
              1 & 1                & 1                 & 1                 \\
              1 & \varepsilon^{-2} & \varepsilon^{-4}  & \varepsilon^{-6}  \\
              1 & \varepsilon^{-4} & \varepsilon^{-8}  & \varepsilon^{-12} \\
              1 & \varepsilon^{-6} & \varepsilon^{-12} & \varepsilon^{-18} \\
          \end{pmatrix}
          \begin{pmatrix}
              1 & 0                & 0                & 0                \\
              0 & \varepsilon^{-1} & 0                & 0                \\
              0 & 0                & \varepsilon^{-2} & 0                \\
              0 & 0                & 0                & \varepsilon^{-3} \\
          \end{pmatrix}
          &
          \varepsilon^{-4}
          \begin{pmatrix}
              1 & 1                & 1                 & 1                 \\
              1 & \varepsilon^{-2} & \varepsilon^{-4}  & \varepsilon^{-6}  \\
              1 & \varepsilon^{-4} & \varepsilon^{-8}  & \varepsilon^{-12} \\
              1 & \varepsilon^{-6} & \varepsilon^{-12} & \varepsilon^{-18} \\
          \end{pmatrix}
          \begin{pmatrix}
              1 & 0                & 0                & 0                \\
              0 & \varepsilon^{-1} & 0                & 0                \\
              0 & 0                & \varepsilon^{-2} & 0                \\
              0 & 0                & 0                & \varepsilon^{-3} \\
          \end{pmatrix}

    \end{pmatrix}
\]
Матрица Фурье порядка 4
\[
    F_4(\varepsilon^2)
    = \begin{pmatrix}
          1 & 1                & 1                 & 1                 \\
          1 & \varepsilon^{-2} & \varepsilon^{-4}  & \varepsilon^{-6}  \\
          1 & \varepsilon^{-4} & \varepsilon^{-8}  & \varepsilon^{-12} \\
          1 & \varepsilon^{-6} & \varepsilon^{-12} & \varepsilon^{-18} \\
    \end{pmatrix}
\]
Сокращение степени:
\[
    F_4(\varepsilon^2)
    = \begin{pmatrix}
          1 & 1                & 1                & 1                 \\
          1 & \varepsilon^{-2} & \varepsilon^{-4} & \varepsilon^{-6}  \\
          1 & \varepsilon^{-4} & 1                & \varepsilon^{-4}  \\
          1 & \varepsilon^{-6} & \varepsilon^{-4} & \varepsilon^{-10} \\
    \end{pmatrix}
\]
Перестановка строк ($P_4$ --- матрица обратной перестановки).
\[
    F_4(\varepsilon^2)
    =
    P_4
    \begin{pmatrix}
        1 & 1                & 1                & 1                 \\
        1 & \varepsilon^{-4} & 1                & \varepsilon^{-4}  \\
        1 & \varepsilon^{-2} & \varepsilon^{-4} & \varepsilon^{-6}  \\
        1 & \varepsilon^{-6} & \varepsilon^{-4} & \varepsilon^{-10} \\
    \end{pmatrix}
\]
Выделение подматриц:
\begin{multline*}
    F_4(\varepsilon^2)
    =
    P_4
    \begin{pmatrix}
        \begin{pmatrix}
            1 & 1                \\
            1 & \varepsilon^{-4}
        \end{pmatrix}
        &
        \begin{pmatrix}
            1 & 1                \\
            1 & \varepsilon^{-4}
        \end{pmatrix}
        \\
        \begin{pmatrix}
            1 & \varepsilon^{-2} \\
            1 & \varepsilon^{-6}
        \end{pmatrix}
        &
        \varepsilon^{-4}
        \begin{pmatrix}
            1 & \varepsilon^{-2} \\
            1 & \varepsilon^{-6} \\
        \end{pmatrix}
    \end{pmatrix} = \\
%
    = P_4
    \begin{pmatrix}
        \begin{pmatrix}
            1 & 1                \\
            1 & \varepsilon^{-4}
        \end{pmatrix}
        &
        \begin{pmatrix}
            1 & 1                \\
            1 & \varepsilon^{-4}
        \end{pmatrix}
        \\
        \begin{pmatrix}
            1 & 1                \\
            1 & \varepsilon^{-4}
        \end{pmatrix}
        \begin{pmatrix}
            1 & 0                \\
            0 & \varepsilon^{-2}
        \end{pmatrix}
        &
        \varepsilon^{-4}
        \begin{pmatrix}
            1 & 1                \\
            1 & \varepsilon^{-4}
        \end{pmatrix}
        \begin{pmatrix}
            1 & 0                \\
            0 & \varepsilon^{-2}
        \end{pmatrix}
    \end{pmatrix} .
\end{multline*}


\section{Фильтр с конечной импульсной характеристикой}

В частном случае линейная система может выполнять преобразование свертки дискретного сигнала $x$ с некоторым вектором $h = ( h_0, h_1, \dots, h_{m-1})$ длины $m$
(как правило, $m < n$). Получающийся на выходе сигнал $y_i$ формируется по правилу:
$$
y_i = \sum_{j=0}^{m-1} h_j x_{i - j}.
$$
Например, для $n=7$ и $h=3$:
\begin{align*}
    y_0 & = h_0 x_0 + h_1 x_{-1}  + h_2 x_{-2} = h_0 x_0 + h_1 x_{n-1}  + h_2 x_{n-2}, \\
    y_1 & = h_0 x_1 + h_1 x_0 + h_2 x_{-1} = h_0 x_1 + h_1 x_0 + h_2 x_{n-1}, \\
    y_2 & = h_0 x_2 + h_1 x_1 + h_2 x_0, \\
    \dots \\
    y_n & = h_0 x_n + h_1 x_{n-1} + h_2 x_{n-2}.
\end{align*}

Удобно считать, что вектор $h$ имеет длину $n$ и ненулевыми только первые $m$ элементов. В этом случае
$$
y = L x ,
$$
где матрица системы $L$:
$$
L =
\begin{pmatrix}
    h_0     & h_{n-1} & h_{n-2} & \dots  & h_3    & h_2    & h_1    \\
    h_1     & h_0     & h_{n-1} & \dots  & h_4    & h_3    & h_2    \\
    h_2     & h_1     & h_0     & \dots  & h_5    & h_4    & h_3    \\
    \vdots  & \vdots  & \vdots  & \ddots & \vdots & \vdots & \vdots \\
    h_{n-1} & h_{n-2} & h_{n-3} & \dots  & h_2    & h_1    & h_0
\end{pmatrix}
.
$$
Например, для $n=7$ и $h=3$:
$$
L =
\begin{pmatrix}
    h_0 & 0   & 0   & 0   & 0   & h_2 & h_1 \\
    h_1 & h_0 & 0   & 0   & 0   & 0   & h_2 \\
    h_2 & h_1 & h_0 & 0   & 0   & 0   & 0   \\
    0   & h_2 & h_1 & h_0 & 0   & 0   & 0   \\
    0   & 0   & h_2 & h_1 & h_0 & 0   & 0   \\
    0   & 0   & 0   & h_2 & h_1 & h_0 & 0   \\
    0   & 0   & 0   & 0   & h_2 & h_1 & h_0
\end{pmatrix}
.
$$

Легко видеть, что матрица $L$ является циркулянтом, поэтому если $\vec{Y}$,  $\vec{X}$ и $\vec{H}$ обозначают спектры сигналов $y$, $x$ и вектора $h$:
$$
\vec{Y} = F y, \; \vec{X} = F x, \; \vec{H} = F h,
$$
тогда из равенства \eqref{equation:Fourier:spectra-product}:
\begin{equation}
    \label{equation:finite-filter:output-spectrum}
    Y_k = H_k \cdot X_k.
\end{equation}
Равенство \eqref{equation:finite-filter:output-spectrum} показывает, что спектр входного сигнала $\vec{X}$ может быть модифицирован произвольным образом, путем умножения на
разные числа $H_i$. После формирования вектора $\vec{H}$ путем обратного дискретного преобразования Фурье \eqref{equation:Fourier:inverse} можно вычислить вектор $h$:
$$
h = \frac{1}{n} F_n^* \vec{H}.
$$

\subsection{Фильтр нижних частот}

Равенство \eqref{equation:finite-filter:output-spectrum} показывает, что если элемент спектра $X_k$ входного сигнала не нулевой, то путем умножения на $H_k$ в спектре
выходного сигнала элемент $Y_k$ можно сделать любым. Это означает, что если во входном сигнале есть гармоника с частотой $k \frac{f_s}{n}$, то её аплитуду можно увеличить и
уменьшить в произвольное число раз (и, конечно, обнулить).

Предположим, что нужно спроектировать фильтр нижних частот, то есть такой фильтр, который пропускает (увеличивает или незначительно уменьшает амплитуды гармоник с нижними
частотами), и не пропускает (значительно уменьшает амплитуды с верхними частотами). Во избежание наложений спектров, вызваемых периодичностью дискретных спектров, частоту
дискретизации $f_s$ нужно выбрать такой большой, чтобы самая верхняя частота сигнала была меньше половины частоты дискретизации $f_s$. Далее нужно отделить "нижние"{}
частоты от "верхних"{}, пусть частота среза (cut frequency) $f_c = k_c \frac{f_s}{n}$, все частоты не больше $f_c$ будем считать нижними, а все частоты больше $f_c$ будем
считать верхними.

Теперь можно задать спектр $\vec{H}$ таким образом (спектр действительного вектора $h$ всегда симметричен):
\begin{enumerate}
    \item $H_0 = \dots = H_{k_c} = 1$,
    \item $H_{k_c + 1} = \dots = H_{n - k_c - 1} = 0$,
    \item $H_{n - k_c - 1} = \dots = H_{n-1} = 1$
\end{enumerate}
Спектр $\vec{H}$ равен единице по "краям"{} и нулю в "середине"{} (как на рисунке \ref{figure:finite-filter:low-frequencies:system-spectrum}).

\begin{figure}[h]
    \centering
    \begin{tikzpicture}
% ось X
        \draw [->] ( -0.1, 0 ) -- ( 7, 0 ) node [below] ( 7, 0 ) {$N$};
% ось Y
        \draw [->] ( 0, -0.1 ) -- ( 0, 1.5 ) node [left] ( 0, 1.5 ) {$\vec{H}$};

% осчеты
        \draw [-] ( 0, 0.1 ) -- ( 0, -0.1 ) node [below ] ( 0, -0.1 ) {$0$};
        \draw [-] ( 1, 0.1 ) -- ( 1, -0.1 ) node [below ] ( 1, -0.1 ) {$1$};
        \draw [-] ( 2, 0.1 ) -- ( 2, -0.1 ) node [below ] ( 2, -0.1 ) {$2$};
        \draw [-] ( 3, 0.1 ) -- ( 3, -0.1 ) node [below ] ( 3, -0.1 ) {$3$};
        \draw [-] ( 4, 0.1 ) -- ( 4, -0.1 ) node [below ] ( 4, -0.1 ) {$4$};
        \draw [-] ( 5, 0.1 ) -- ( 5, -0.1 ) node [below ] ( 5, -0.1 ) {$5$};
        \draw [-] ( 6, 0.1 ) -- ( 6, -0.1 ) node [below ] ( 6, -0.1 ) {$6$};

% спектр
        \draw [fill=black] ( 0, 1 ) circle [radius=0.05];
        \draw [fill=black] ( 1, 1 ) circle [radius=0.05];
        \draw [fill=black] ( 2, 0 ) circle [radius=0.05];
        \draw [fill=black] ( 3, 0 ) circle [radius=0.05];
        \draw [fill=black] ( 4, 0 ) circle [radius=0.05];
        \draw [fill=black] ( 5, 1 ) circle [radius=0.05];
        \draw [fill=black] ( 6, 1 ) circle [radius=0.05];
    \end{tikzpicture}
    \caption{Спектр $\vec{H}$ для $n=7$ и $k_c=1$.}
    \label{figure:finite-filter:low-frequencies:system-spectrum}
\end{figure}

В результате обратного дискретного преобразования Фурье получится импульсная характеристика $h$:
$$
h = \frac{1}{n} F_n^* \vec{H}
$$
которая будет иметь вид как на рисунке \ref{figure:finite-filter:low-frequences:impulse}.

\begin{figure}[h]
    \centering
    \begin{tikzpicture}
        \draw [domain=0.01:3.14,samples=100] plot (\x, {2 * sin(7 * \x r)/ ( 7 * \x )});
        \draw [-] ( 3.14, 0 ) -- ( 5, 0);
        \draw [domain=5:8.13,samples=100] plot (\x, {2 * sin(7 * ( 8.14 - \x ) r)/ ( 7 * ( 8.14 - \x ))});

    \end{tikzpicture}
    \caption{Импульсная характеристика $h$.}
    \label{figure:finite-filter:low-frequences:impulse}
\end{figure}


\section{Декомпозиция}

Существует много вариантов декомпозиции --- разложение по базису.

Разложение по базису требует решения системы линейных уравнений.
Если базис ортогонализован, то коэффициенты разложения легко вычисляются.


\section{Декомпозиция Фурье}

Базис Фурье --- частный случай базиса степеней первообразного корня из единицы.
Базис Фурье имеет простую физическую интерпретацию.
Линейные системы могут только изменить амлитуду и фазу.
Базис Фурье ортогональный.

    % зависимость
    \section{О линейной зависимости числовых векторов}

Векторы $x$ и $y$ являются линейно зависимыми, если:
\[
    x = c y,
\]
для некоторого числа $c$.

Пример
\[
    \begin{pmatrix}
        6 \\
        8 \\
        - 12
    \end{pmatrix}
    = 2
    \begin{pmatrix}
        3 \\
        4 \\
        -6
    \end{pmatrix}
\]

\textcolor{red}{рисунок с лучами и масштабированием}

В случае комплексных векторов вместо числовых лучей появляются комплексные плоскости. Умножение на комплексное число можно представить как изменение модуля и
поворот, поэтому в случае векторов допускается не только масштабирование компонент, но и одновременный поворот всех компонент на один угол.

\textcolor{red}{рисунок с плоскостями масштабированием и поворотом}

Пример
\[
    \begin{pmatrix}
        -1 + 3i \\
        2 - 1i  \\
        3 + 1i
    \end{pmatrix}
    =
    (1 + 2i)
    \begin{pmatrix}
        1 + i \\
        -i    \\
        1 - i
    \end{pmatrix}
    .
\]

    \begin{enumerate}
        \item Обусловленность задачи решения системы линейных уравнений.
        \item $LU$ и $LDL^T$ разложения (с выбором ведущего элемента).
        \item Вычисление QR разложения.
        \item Вычисление собственных значений и собственных векторов. Нахождение наибольшего собственного значения.
        \item Вычисление сингулярного разложения. Приложения (Колесников).
        \item Итерационные методы решения СЛУ.
        \item Проекционные методы решения СЛУ.
        \item Методы факторизации (из Ратынского).
        \item Быстрое умножение матриц - алгоритм Штрассена (Блейхут).
        \item Быстрые алгоритмы свёртки (Блейхут).
        \item Быстрое преобразование Фурье (Блейхут).
        \item Быстрое решение тёплицевых систем (Блейхут).
        \item Ленточные матрицы. Решение систем с трехдиагональной матрицей (прогонка).
    \end{enumerate}

    Основные задачи.
    \begin{enumerate}
        \item Разложения: $LU$, $LDL^T$, $QR$, $U \Sigma V^T$.
        \item Решение системы.
        \item Собственные значения и векторы.
    \end{enumerate}

    Прикладные задачи:
    \begin{enumerate}
        \item Задачи из книги Ратынского.
        \item Задачи из книги Матасова.
    \end{enumerate}
\end{document}