\documentclass[a4paper,12pt]{book}

\usepackage[T1]{fontenc}
\usepackage[utf8]{inputenc}
\usepackage[english,russian]{babel}
\usepackage[margin=2cm]{geometry}
\usepackage{amsmath}
\usepackage{amssymb}
\usepackage{tikz}
\usepackage{color}
\usepackage{amsfonts}

% команды вывода первой частной производной
\newcommand{\fpd}[1]{\frac{\partial}{\partial #1}}
\newcommand{\fpda}[2]{\frac{\partial #1}{\partial #2}}
\newcommand{\fpdp}[2]{\fpd{#2} \left ( #1 \right )}

\newcommand{\expectation}[1]{\mathtt{M} \left [ #1 \right ]}
\newcommand{\conditionalexpectation}[2]{\expectation{ #1 \left | #2 \right .}}
\newcommand{\variance}[1]{\mathtt{D} \left [ #1 \right ]}
\newcommand{\covariance}[2]{\mathtt{cov} \left ( #1, #2 \right )}

\newcommand{\modulus}[1]{\left | #1 \right |}
\newcommand{\norm}[1]{\left \| {#1} \right \|}

\newcommand{\event}[1]{\left \{ #1 \right \} }
\newcommand{\probability}[1]{P \event{#1}}


\begin{document}

    \title{Прикладные методы линейной алгебры}
    \author{Тигетов Давид Георгиевич}
    \maketitle

    \tableofcontents

    % план занятий
    \chapter{План занятий}

Для отношения Релея.
\begin{itemize}
    \item[Занятие 1.] Унитарные пространства. Сопряжённый оператор. Свойства сопряженного оператора.
    \item[Занятие 2.] Теорема Шура. Нормальные матрицы. Эрмитовые матрицы.
    \item[Занятие 3.] Экстремумы отношения Релея.
    \item[Занятие 4.] Экстремумы отношения Релея. Вычисление в Matlab.
    \item[Занятие 5.] Комплексные огибающие. Двухканальный излучатель. Коэффициент усиления. Пример 2. Пример 3. Пример 4. Пример 5.
    \item[Занятие 6.] Энергетическое ограничение. Коэффициент полезного действия. Примеры. Антенна.
\end{itemize}

    % отношение Релея
    \chapter{Отношение Релея}


\section{Унитарные пространства}

Пусть $\mathbb{C}$ --- обозначает поле комплексных чисел, и $\mathcal{U}$ --- множество векторов, для которых определена операции сложения $+$ векторов и умножения
векторов на число из поля $\mathbb{C}$, обладающих свойствами для любых $u, v, w \in \mathcal{U}$ и $\alpha, \beta \in \mathbb{C}$:
\begin{enumerate}
    \item $u + ( v + w ) = ( u + v ) + w$,
    \item $\exists 0 \in U: u + 0 = 0 + u = u$,
    \item $\exists (-u) \in U: u + (-u) = (-u) + u = 0$,
    \item $u + v = v + u$,
    \item $(\alpha + \beta) u = \alpha u + \beta u$,
    \item $\alpha ( u + v ) = \alpha u + \alpha v$,
    \item $\alpha (\beta u) = (\alpha \beta) u$,
    \item $u = 1 \cdot u$.
\end{enumerate}
Множество $\mathcal{U}$ с определенными на нём операциями сложения векторов и умножения на число называется \definition{линейным (векторным) пространством}.

Пусть на множестве всех пар векторов пространства $\mathcal{U}$ определена функция $\scalarproduct{\cdot}{\cdot}$, называемая \definition{скалярным произведением}, обладающая
следующими свойствами для любых $u, v \in \mathcal{U}$ и числа $\lambda \in \mathbb{C}$:
\begin{enumerate}
    \item $\scalarproduct{u}{u} \ge 0$,
    \item $\scalarproduct{u}{u} = 0 \Leftrightarrow u = 0$,
    \item $\scalarproduct{u + w}{v} = \scalarproduct{u}{v} + \scalarproduct{w}{v}$,
    \item $\scalarproduct{\lambda u}{v} = \lambda \scalarproduct{u}{v}$,
    \item $\scalarproduct{u}{v} = \overline{\scalarproduct{v}{u}}$,
\end{enumerate}
где черта $\overline{\cdot}$ обозначает комплексное сопряжение. Из приведённых свойств скалярного произведения следует:
\begin{gather*}
    \scalarproduct{u}{v + w}
    = \overline{\scalarproduct{v + w}{u}}
    = \overline{\scalarproduct{v}{u} + \scalarproduct{w}{u}}
    = \overline{\scalarproduct{v}{u}} + \overline{\scalarproduct{w}{u}}
    = \scalarproduct{u}{v} + \scalarproduct{u}{w}, \\
    %
    \scalarproduct{u}{\lambda v}
    = \overline{\scalarproduct{\lambda v}{u}}
    = \overline{\lambda  \scalarproduct{v}{u}}
    = \overline{\lambda} \overline{\scalarproduct{v}{u}}
    = \overline{\lambda} \scalarproduct{u}{v} .
\end{gather*}

Если в линейном пространстве $\mathcal{U}$ задано скалярное произведение $\scalarproduct{\cdot}{\cdot}$, то такое пространство называется \definition{унитарным}.

С помощью скалярного произведения можно определить норму $\norm{\cdot}$ векторов:
\[
    \norm{u} = \sqrt{\scalarproduct{u}{u}}.
\]
Такое определение будет удовлетворять всем свойствам нормы:
\begin{enumerate}
    \item $\norm{u} \ge 0$,
    \item $\norm{u} = 0 \Leftrightarrow u = 0$,
    \item $\norm{\lambda u} = \modulus{\lambda} \norm{u}$,
    \item $\norm{u + v} \le \norm{u} + \norm{v}$.
\end{enumerate}
Пространство с определенной для его элементов нормой называют \definition{нормированным пространством}.


\section{Сопряженный оператор}

Пусть $\mathcal{U}$ является унитарным пространством со скалярным произведением $\scalarproduct{\cdot}{\cdot}_\mathcal{U}$ и $\mathcal{V}$ тоже унитарное пространство
со своим скалярным произведением $\scalarproduct{\cdot}{\cdot}_\mathcal{V}$.

Пусть $\mathcal{A} : \mathcal{U} \rightarrow \mathcal{V}$ --- оператор, действующий из пространства $\mathcal{U}$ в пространство $\mathcal{V}$.
Оператор $\mathcal{A}^*$ называется \definition{сопряженным} к $\mathcal{A}$, если:
\begin{gather}
    \scalarproduct{\mathcal{A} u}{v}_{\mathcal{V}} = \scalarproduct{u}{\mathcal{A}^* v}_{\mathcal{U}}
    \label{rayleigh:conjugation:scalars_equality}, \\
    \forall u \in \mathcal{U}, \forall v \in \mathcal{V}
    \notag.
\end{gather}

Если сопряженный оператор $\mathcal{A}^*$ существует, то он является единственным. Это следует непосредственно из определения
\ref{rayleigh:conjugation:scalars_equality} и следующего утверждения.

\begin{statement}~\label{rayleigh:conjugation:identity}
    Если для операторов $\mathcal{B}_1$ и $\mathcal{B}_2$ выполняется равенство:
    \begin{gather*}
        \scalarproduct{u}{\mathcal{B}_1 v}_{\mathcal{U}} = \scalarproduct{u}{\mathcal{B}_2 v}_{\mathcal{U}} , \\
        \forall u \in \mathcal{U}, \forall v \in \mathcal{V},
    \end{gather*}
    тогда
    \[
        \mathcal{B}_1 = \mathcal{B}_2 .
    \]
\end{statement}
\begin{proof}
    Преобразуем равенство к виду:
    \begin{gather*}
        \scalarproduct{u}{\mathcal{B}_1^* v}_{\mathcal{U}} = \scalarproduct{u}{\mathcal{B}_2^* v}_{\mathcal{U}} , \\
        \scalarproduct{u}{\mathcal{B}_1^* v}_{\mathcal{U}} - \scalarproduct{u}{\mathcal{B}_2^* v}_{\mathcal{U}} = 0 , \\
        \scalarproduct{u}{\mathcal{B}_1^* v - \mathcal{B}_2^* v}_{\mathcal{U}} = 0 , \\
        \scalarproduct{u}{\left( \mathcal{B}_1^* - \mathcal{B}_2^* \right) v}_{\mathcal{U}} = 0 .
    \end{gather*}
    Поскольку полученное равенство выполняется для любых векторов $u$ и $v$, то возьмём в качестве вектора $u$ вектор
    $\left( \mathcal{B}_1^* - \mathcal{B}_2^* \right) v$:
    \[
        \scalarproduct{\left( \mathcal{B}_1^* - \mathcal{B}_2^* \right) v}{\left( \mathcal{B}_1^* - \mathcal{B}_2^* \right) v}_{\mathcal{U}} = 0 .
    \]
    Из свойства 2 скалярного произведения следует:
    \begin{gather*}
        \left( \mathcal{B}_1^* - \mathcal{B}_2^* \right) v = 0 , \\
        \mathcal{B}_1^* v = \mathcal{B}_2^* v .
    \end{gather*}
    Полученное равенство выполняется для всех векторов $v$, поэтому операторы идентичны:
    \[
        \mathcal{B}_1 = \mathcal{B}_2 .
    \]
\end{proof}

Используя доказанное утверждение легко заметить, что если сопряженный оператор $\mathcal{A}$ существует, то он единственный. Действительно, представим,
что есть два сопряжённых оператора $\mathcal{A}_1^*$ и $\mathcal{A}_2^*$, которые удовлетворяют определению \eqref{rayleigh:conjugation:scalars_equality}:
\begin{gather*}
    \scalarproduct{u}{\mathcal{A}_1^* v}_{\mathcal{U}} = \scalarproduct{\mathcal{A} u}{v}_{\mathcal{V}} = \scalarproduct{u}{\mathcal{A}_2^* v}_{\mathcal{U}} , \\
    \scalarproduct{u}{\mathcal{A}_1^* v}_{\mathcal{U}} = \scalarproduct{u}{\mathcal{A}_2^* v}_{\mathcal{U}} .
\end{gather*}
Поскольку равенство выполняется при всех $u$ и $v$, то в силу утверждения \ref{rayleigh:conjugation:identity} операторы идентичны:
\[
    \mathcal{A}_1^* = \mathcal{A}_2^* .
\]

Остался вопрос об условиях существовании сопряженного оператора $\mathcal{A}^*$. Как будет показано далее, при некоторых условиях сопряженный оператор существует,
более того его можно построить в явном виде.

Пусть оператор $\mathcal{A}$ является \definition{линейным}, то есть для любых $u_1, u_2 \in \mathcal{U}$ и любого числа $\lambda \in \mathbb{C}$ выполняются равенства:
\begin{enumerate}
    \item $\mathcal{A}(u_1 + u_2) = \mathcal{A}(u_1) + \mathcal{A}(u_2)$,
    \item $\mathcal{A}(\lambda u_1) = \lambda \mathcal{A}(u_1)$.
\end{enumerate}

Линейный оператор можно задать следующим образом: выбрать базисный набор векторов $e_i$ пространства $\mathcal{U}$ и определить действие оператора на базисные векторы
$\mathcal{A} e_i$, затем используя линейность для всякого элемента $u \in \mathcal{U}$:
\[
    u = c_1 e_1 + c_2 e_2 + \dots,
\]
получим:
\[
    \mathcal{A} u
    = \mathcal{A} ( c_1 e_1 + c_2 e_2 + \dots )
    = c_1 \mathcal{A} e_1 + c_2 \mathcal{A} e_2 + \dots
\]

При построении сопряженного оператора $\mathcal{A}^*$ можно использовать такой же способ его определения.

\begin{statement}
    Пусть $\mathcal{U}$ и $\mathcal{V}$ --- унитарные конечномерные пространства и $\mathcal{A} : \mathcal{U} \rightarrow \mathcal{V}$ --- линейный оператор, тогда
    \begin{enumerate}
        \item существует сопряженный к $\mathcal{A}$ оператор $\mathcal{A}^*$,
        \item если $A$ --- матрица оператора $\mathcal{A}$ в ортогональных базисах пространств $\mathcal{U}$ и $\mathcal{V}$, тогда матрица $A^*$ сопряженного оператора
        $\mathcal{A^*}$:
        \[
            A^* = \overline{A}^T .
        \]
    \end{enumerate}
\end{statement}
\begin{proof}
    Пусть набор векторов $e = \{ e_1, \dots, e_n \}$ является ортонормированными базисом в пространстве $\mathcal{U}$, а набор векторов $f = \{f_1, \dots, f_m \}$
    является ортономированным базисом в пространстве $\mathcal{V}$. Пусть $A = [a_{ij}] \in \Cspace{m \times n}$ является матрицей оператора $\mathcal{A}$ в базисах
    $e$ и $f$, тогда:
    \[
        \mathcal{A} e_j = a_{1j} f_1 + \dots + a_{mj} f_m,
    \]
    тем самым определено действие оператора $\mathcal{A}$ на базисные векторы $e_j$.

    Попробуем определить оператор $\mathcal{B}$ так, чтобы равенство \eqref{rayleigh:conjugation:scalars_equality} выполнялось для набора векторов $e$ и $f$:
    \begin{equation}
        \label{rayleigh:conjugation:basis_scalars_equality}
        \scalarproduct{\mathcal{A} e_j}{f_i}_\mathcal{V} = \scalarproduct{e_j}{\mathcal{B} f_i}_\mathcal{U} .
    \end{equation}
    Поскольку $\mathcal{B} f_i \in \mathcal{U}$, то:
    \[
        \mathcal{B} f_i = b_{1i} e_1 + \dots + b_{ni} e_n ,
    \]
    причём коэффициенты $b_{ji}$ следует выбирать таким образом, чтобы выполнялось равенство \eqref{rayleigh:conjugation:basis_scalars_equality}, которое в силу
    ортонормированности наборов $e$ и $f$ приводит к равенству:
    \begin{align*}
        \scalarproduct{\mathcal{A} e_j}{f_i}_\mathcal{V} & = \scalarproduct{e_j}{\mathcal{B} f_i}_\mathcal{U} , \\
        \scalarproduct{a_{1j} f_1 + \dots + a_{mj} f_m}{f_i}_\mathcal{V} & = \scalarproduct{e_j}{b_{1i} e_1 + \dots + b_{ni} e_n}_\mathcal{U} , \\
        a_{1j} \scalarproduct{f_1}{f_i}_\mathcal{V} + \dots + a_{mj} \scalarproduct{f_m}{f_i}_\mathcal{V} & = \overline{b_{1i}} \scalarproduct{e_j}{e_1}_\mathcal{U} + \dots + \overline{b_{ni}} \scalarproduct{e_j}{e_n}_\mathcal{U} , \\
        a_{ij} \scalarproduct{f_i}{f_i}_\mathcal{V} & = \overline{b_{ji}} \scalarproduct{e_j}{e_j}_\mathcal{U} , \\
        a_{ij} & = \overline{b_{ji}} , \\
        \overline{a_{ij}} & = b_{ji} .
    \end{align*}
    Таким образом, определено действие оператора $\mathcal{B}$ на базисные векторы $f_i$:
    \begin{equation}
        \label{rayleigh:conjugation:basis_images}
        \mathcal{B} f_i = \overline{a_{i1}} e_1 + \dots + \overline{a_{in}} e_n .
    \end{equation}

    Теперь распространим действие оператора $\mathcal{B}$ на все векторы $v \in \mathcal{V}$, сделав оператор $\mathcal{B}$ линейным, пусть
    \[
        v = v_1 f_1 + \dots + v_m f_m,
    \]
    тогда
    \[
        \mathcal{B} v = v_1 \mathcal{B} f_1 + \dots + v_m \mathcal{B} f_m .
    \]
    Проверим, что при таком определении оператор $\mathcal{B}$ оказывается сопряженным к оператору $\mathcal{A}$. Пусть $u \in \mathcal{U}$ --- произвольный вектор
    пространства $U$:
    \[
        u = u_1 e_1 + \dots + u_n e_n,
    \]
    тогда
    \begin{multline*}
        \scalarproduct{\mathcal{A} u}{v}_\mathcal{V}
        = \scalarproduct{\mathcal{A} (u_1 e_1 + \dots + u_n e_n)}{v}_\mathcal{V}
        = \scalarproduct{u_1 \mathcal{A} e_1 + \dots + u_n \mathcal{A} e_n}{v}_\mathcal{V} = \\
        %
        = \scalarproduct{u_1 \mathcal{A} e_1 + \dots + u_n \mathcal{A} e_n}{v_1 f_1 + \dots + v_m f_m}_\mathcal{V} = \\
        %
        = \sum_{i=1}^n u_i \sum_{j=1}^m \overline{v_j} \scalarproduct{\mathcal{A} e_i}{f_j}_\mathcal{V}
        = \sum_{i=1}^n u_i \sum_{j=1}^m \overline{v_j} \scalarproduct{e_i}{\mathcal{B} f_j}_\mathcal{U} = \\
        %
        = \scalarproduct{u_1 e_i + \dots + u_n e_n}{v_1 \mathcal{B} f_1 + \dots + v_m \mathcal{B} f_m}_\mathcal{U} = \\
        %
        = \scalarproduct{u}{v_1 \mathcal{B} f_1 + \dots + v_m \mathcal{B} f_m}_\mathcal{U}
        = \scalarproduct{u}{\mathcal{B} (v_1 f_1 + \dots + v_m f_m)}_\mathcal{U}
        = \scalarproduct{u}{\mathcal{B} v}_\mathcal{U} .
    \end{multline*}
    Таким образом, $\mathcal{B}$ является сопряженным к $\mathcal{A}$:
    \[
        \mathcal{A}^* = \mathcal{B}.
    \]
    Кроме того, из равенства \eqref{rayleigh:conjugation:basis_images} следует, что матрица $B$ оператора $\mathcal{B}$ в базисах $f$ и $e$:
    \[
        B =
        \begin{pmatrix}
            \overline{a_{11}} & \overline{a_{21}} & \dots  & \overline{a_{m1}} \\
            \overline{a_{12}} & \overline{a_{22}} & \dots  & \overline{a_{m2}} \\
            \vdots            & \vdots            & \ddots & \vdots            \\
            \overline{a_{1n}} & \overline{a_{2n}} & \dots  & \overline{a_{mn}}
        \end{pmatrix}
        =
        \begin{pmatrix}
            \overline{a_{11}} & \overline{a_{12}} & \dots  & \overline{a_{1n}} \\
            \overline{a_{21}} & \overline{a_{22}} & \dots  & \overline{a_{2n}} \\
            \vdots            & \vdots            & \ddots & \vdots            \\
            \overline{a_{n1}} & \overline{a_{n2}} & \dots  & \overline{a_{nm}}
        \end{pmatrix}^T
        = \overline{A}^T .
    \]
    То есть матрица $A^*$ сопряженного оператора $\mathcal{A}^*$:
    \[
        A^* = \overline{A}^T .
    \]
\end{proof}

Выделим некоторые свойства сопряженного оператора.
\begin{enumerate}
    \item Сопряженным к оператору $\alpha \mathcal{A}$ является оператор $\overline{\alpha} \mathcal{A}^*$. Действительно:
    \begin{align*}
        \scalarproduct{\alpha \mathcal{A} u}{v}_{\mathcal{V}}
        & = \scalarproduct{u}{\left( \alpha \mathcal{A} \right)^* v}_{\mathcal{U}} , \\
        %
        \scalarproduct{\alpha \mathcal{A} u}{v}_{\mathcal{V}}
        & = \alpha \scalarproduct{\mathcal{A} u}{v}_{\mathcal{V}}
        = \alpha \scalarproduct{u}{\mathcal{A}^* v}_{\mathcal{U}}
        = \scalarproduct{u}{\overline{\alpha} \mathcal{A}^* v}_{\mathcal{U}} .
    \end{align*}
    В силу утверждения \ref{rayleigh:conjugation:identity}:
    \[
        \scalarproduct{u}{\left( \alpha \mathcal{A} \right)^* v}_{\mathcal{U}} = \scalarproduct{u}{\overline{\alpha} \mathcal{A}^* v}_{\mathcal{U}}
        \Rightarrow
        \left( \alpha \mathcal{A} \right)^* = \overline{\alpha} \mathcal{A}^*.
    \]

    \item Сопряженным к оператору $\mathcal{A} + \mathcal{B}$ является оператор $\mathcal{A}^* + \mathcal{B}^*$. Действительно
    \begin{align*}
        \scalarproduct{(\mathcal{A} + \mathcal{B}) u}{v}_{\mathcal{V}}
        & = \scalarproduct{u}{( \mathcal{A} + \mathcal{B})^* v}_{\mathcal{U}} , \\
        %
        \scalarproduct{(\mathcal{A} + \mathcal{B}) u}{v}_{\mathcal{V}}
        & = \scalarproduct{\mathcal{A} u + \mathcal{B} u}{v}_{\mathcal{V}}
        = \scalarproduct{\mathcal{A} u}{v}_{\mathcal{V}} + \scalarproduct{\mathcal{B} u}{v}_{\mathcal{V}}
        = \scalarproduct{u}{\mathcal{A}^* v}_{\mathcal{U}} + \scalarproduct{u}{\mathcal{B}^* v}_{\mathcal{U}} = \\
        & = \scalarproduct{u}{\mathcal{A}^* v + \mathcal{B}^* v}_{\mathcal{U}}
        = \scalarproduct{u}{(\mathcal{A}^* + \mathcal{B}^*) v}_{\mathcal{U}} .
    \end{align*}
    В силу утверждения \ref{rayleigh:conjugation:identity}:
    \[
        \scalarproduct{u}{(\mathcal{A} + \mathcal{B})^* v}_{\mathcal{U}} = \scalarproduct{u}{(\mathcal{A}^* + \mathcal{B}^*) v}_{\mathcal{U}}
        \Rightarrow
        ( \mathcal{A} + \mathcal{B})^* = \mathcal{A}^* + \mathcal{B}^*
    \]

    \item Сопряженным к оператору $\mathcal{A} \mathcal{B}$ является оператор $\mathcal{B}^* \mathcal{A}^*$. Действительно
    \begin{align*}
        \scalarproduct{\mathcal{A} \mathcal{B} u}{v}_{\mathcal{V}}
        & = \scalarproduct{u}{(\mathcal{A} \mathcal{B})^* v}_{\mathcal{U}} , \\
        %
        \scalarproduct{\mathcal{A} \mathcal{B} u}{v}_{\mathcal{V}}
        & = \scalarproduct{\mathcal{B} u}{\mathcal{A}^* v}_{\mathcal{W}}
        = \scalarproduct{u}{\mathcal{B}^* \mathcal{A}^* v}_{\mathcal{U}} .
    \end{align*}
    В силу утверждения \ref{rayleigh:conjugation:identity}:
    \[
        \scalarproduct{u}{(\mathcal{A} \mathcal{B})^* v}_{\mathcal{U}} = \scalarproduct{u}{\mathcal{B}^* \mathcal{A}^* v}_{\mathcal{U}}
        \Rightarrow
        (\mathcal{A} \mathcal{B})^* = \mathcal{B}^* \mathcal{A}^* .
    \]

    \item У сопряженного оператора $\mathcal{A}^*$ тоже есть сопряженный оператор $(\mathcal{A}^*)^*$, который совпадает с исходным оператором $\mathcal{A}$.
    Действительно:
    \begin{align*}
        \scalarproduct{\mathcal{A}^* v}{u}_{\mathcal{U}}
        & = \scalarproduct{v}{(\mathcal{A}^*)^* u}_{\mathcal{V}} , \\
        %
        \scalarproduct{\mathcal{A}^* v}{u}_{\mathcal{U}}
        & = \overline{\scalarproduct{u}{\mathcal{A}^* v}_{\mathcal{V}}}
        = \overline{\scalarproduct{\mathcal{A} u}{v}_{\mathcal{V}}}
        = \scalarproduct{v}{\mathcal{A} u}_{\mathcal{V}}.
    \end{align*}
    В силу утверждения \ref{rayleigh:conjugation:identity}:
    \[
        \scalarproduct{v}{(\mathcal{A}^*)^* u}_{\mathcal{V}} = \scalarproduct{v}{\mathcal{A} u}_{\mathcal{V}}
        \Rightarrow
        (\mathcal{A}^*)^* = \mathcal{A} .
    \]
\end{enumerate}

Из свойства 3 следует:
\[
    \left ( \mathcal{A}^n \right )^* = \left ( \mathcal{A}^* \right )^n ,
\]
действительно:
\[
    \left ( \mathcal{A}^n \right )^*
    = \left ( \mathcal{A}^{n-1} \mathcal{A} \right )^*
    = \mathcal{A}^* \left ( \mathcal{A}^{n-1} \right )^*
    = \dots
    = \mathcal{A}^* \mathcal{A}^* \dots \mathcal{A}^*
    = \left ( \mathcal{A}^* \right )^n .
\]

\begin{example}
    Найти сопряженный оператор к оператору $\alpha \mathcal{A}^3 + \beta \mathcal{B}^2 \mathcal{C}$.

    Используя доказанные свойства, получим:
    \[
        \left ( \alpha \mathcal{A}^3 + \beta \mathcal{B}^2 \mathcal{C} \right )^*
        = \left ( \alpha \mathcal{A}^3 \right )^* + \left ( \beta \mathcal{B}^2 \mathcal{C} \right )^*
        = \overline{\alpha} \left ( \mathcal{A}^* \right )^3 + \overline{\beta} \mathcal{C}^* \left ( \mathcal{B}^* \right )^2 .
    \]
\end{example}

Выделяют следующие классы операторов, которые имеют полезные свойства.

Оператор $\mathcal{A}$ называется \definition{нормальным}, если:
\[
    \mathcal{A} \mathcal{A}^* = \mathcal{A}^* \mathcal{A}.
\]

Оператор $\mathcal{A}$ называется \definition{самоспоряженным} или \definition{эрмитовым}, если:
\[
    \mathcal{A}^* = \mathcal{A}.
\]

Оператор $\mathcal{A}$ называется \definition{унитарным}, если:
\[
    \mathcal{A}^* = \mathcal{A}^{-1}.
\]

Легко видеть, что если $\mathcal{A}$ является эрмитовым оператором, то он является нормальным:
\[
    \mathcal{A} \mathcal{A}^*
    = \mathcal{A} \mathcal{A}
    = \mathcal{A}^* \mathcal{A},
\]
и унитарный оператор $\mathcal{A}$ тоже является нормальным:
\[
    \mathcal{A} \mathcal{A}^*
    = \mathcal{A} \mathcal{A}^{-1}
    = I
    = \mathcal{A}^{-1} \mathcal{A}
    = \mathcal{A}^* \mathcal{A}.
\]

Аналогичные определения вводятся и для матриц, которые называют \definition{нормальными}, \definition{самосопряженными} или \definition{эрмитовыми}, \definition{унитарными}.


\section{Теорема Шура}

Согласно теореме Шура любая квадратная матрица $A$ унитарно подобна верхнетреугольной матрице, то есть существует унитарная матрица $U$:
\[
    U^* U = I,
\]
такая что
\[
    U^* A U = R, \\
\]
где $R$ --- верхнетреугольная матрица. Умножая последнее равенство слева на $U$ и справа на $U^*$, получим
\begin{align}
    U^* A U & = R,
    \notag \\
    U U^* A U U^* & = U R U^*,
    \notag \\
    A & = U R U^*
    \label{rayleigh:schur:decomposition}
\end{align}

\begin{example}
    Для матрицы $A$:
    \[
        A = \begin{pmatrix}
                3 & 2 \\
                1 & 4
        \end{pmatrix}
    \]
    необходимо найти унитарную матрицу $U$ и верхнетреугольную матрицу $R$.

    Для матрицы $U$ выполняется равенство:
    \[
        U^* A U = R.
    \]
    В качестве $U$ возьмем матрицу вращения, пусть $c = \cos \alpha$ и $s = \sin \alpha$:
    \[
        U
        = \begin{pmatrix}
              c  & s \\
              -s & c
        \end{pmatrix} ,
    \]
    тогда
    \[
        U^* A U
        = \begin{pmatrix}
              c & - s \\
              s & c
        \end{pmatrix}
        \begin{pmatrix}
            3 & 2 \\
            1 & 4
        \end{pmatrix}
        \begin{pmatrix}
            c   & s \\
            - s & c
        \end{pmatrix}
        = \begin{pmatrix}
              r_{11} & r_{12} \\
              r_{21} & r_{22}
        \end{pmatrix} .
    \]
    Нужно выбрать величину $\alpha$ так, чтобы элемент $r_{12}$ оказался равен нулю. Вычисляем матрицу в правой части:
    \begin{multline*}
        \begin{pmatrix}
            3 c - 1 s & 2 c - 4 s \\
            3 s + 1 c & 2 s + 4 c
        \end{pmatrix}
        \begin{pmatrix}
            c   & s \\
            - s & c
        \end{pmatrix}
        = \\
        %
        = \begin{pmatrix}
              3 c^2 - c s - 2 c s + 4 s^2 & 3 c s - s^2 + 2 c^2 - 4 c s \\
              3 c s + c^2 - 2 s^2 - 4 c s & 3 s^2 + c s + 2 c s + 4 c^2
        \end{pmatrix}
        = \\
        %
        = \begin{pmatrix}
              3 c^2 - 3 c s + 4 s^2 & - s^2 + 2 c^2 - c s   \\
              c^2 - 2 s^2 - c s     & 3 s^2 + 3 c s + 4 c^2
        \end{pmatrix}
    \end{multline*}
    Необходимо выполнение равенства:
    \begin{gather*}
        c^2 - c s - 2 s^2 = 0 , \\
        \left ( \frac{c}{s} \right )^2 - \frac{c}{s} - 2 = 0 , \\
        \frac{c}{s} = \frac{1 \pm \sqrt{1 + 4 \cdot 2}}{2} , \\
        \frac{c}{s} = \frac{1 \pm 3}{2} , \\
        \frac{c}{s} = \frac{1 - 3}{2} = \frac{-2}{2} = -1 .
    \end{gather*}
    Пусть
    \begin{gather*}
        c = \frac{1}{\sqrt{2}}, s = - \frac{1}{\sqrt{2}} , \\
        c = \cos \left ( - \frac{\pi}{4} \right ), s = \sin \left ( - \frac{\pi}{4} \right ) .
    \end{gather*}
    Таким образом, матрица $U$:
    \[
        U
        = \begin{pmatrix}
              \frac{1}{\sqrt{2}} & - \frac{1}{\sqrt{2}} \\
              \frac{1}{\sqrt{2}} & \frac{1}{\sqrt{2}}
        \end{pmatrix}.
    \]
    и матрица $R$:
    \[
        R
        = \begin{pmatrix}
              3 \cdot \frac{1}{2} + 3 \cdot \frac{1}{2} + 4 \cdot \frac{1}{2} & - \frac{1}{2} + 2 \cdot \frac{1}{2} + \frac{1}{2}               \\
              \frac{1}{2} - 2 \cdot \frac{1}{2} + \frac{1}{2}                 & 3 \cdot \frac{1}{2} - 3 \cdot \frac{1}{2} + 4 \cdot \frac{1}{2}
        \end{pmatrix}
        = \begin{pmatrix}
              5 & 1 \\
              0 & 2
        \end{pmatrix} .
    \]
\end{example}


\section{Спектральное разложение}

Пусть $A$ --- нормальная матрица:
\[
    A^* A = A A^* .
\]
Используя в последнем равенстве разложение \eqref{rayleigh:schur:decomposition}, получим:
\begin{align}
    \left ( U R U^* \right ) ^* U R U^* & = U R U^* \left ( U R U^* \right )^* , \notag \\
    U R^* U^* U R U^* & = U R U^* U R U^* , \notag \\
    U R^* R U^* & = U R R^* U^* , \notag \\
    U^* U R^* R U^* U & = U^* U R R^* U^* U, \notag \\
    R^* R & = R R^* \label{rayleigh:spectral:conjugation_equality}.
\end{align}
Поскольку $R$ --- верхнетреугольная матрица, то последнее равенство возможно только в том случае, когда $R$ --- диагональная:
\[
    R = \Lambda,
\]
где $\Lambda$ --- диагональная матрица.

\begin{example}
    Пусть матрица $R$ имеет порядок 3:
    \[
        R
        = \begin{pmatrix}
              r_{11} & r_{12} & r_{13} \\
              0      & r_{22} & r_{23} \\
              0      & 0      & r_{33}
        \end{pmatrix},
    \]
    тогда равенство \eqref{rayleigh:spectral:conjugation_equality} принимает вид:
    \[
        \begin{pmatrix}
            \overline{r_{11}} & 0                 & 0                 \\
            \overline{r_{12}} & \overline{r_{22}} & 0                 \\
            \overline{r_{13}} & \overline{r_{23}} & \overline{r_{33}}
        \end{pmatrix}
        \begin{pmatrix}
            r_{11} & r_{12} & r_{13} \\
            0      & r_{22} & r_{23} \\
            0      & 0      & r_{33}
        \end{pmatrix}
        =
        \begin{pmatrix}
            r_{11} & r_{12} & r_{13} \\
            0      & r_{22} & r_{23} \\
            0      & 0      & r_{33}
        \end{pmatrix}
        \begin{pmatrix}
            \overline{r_{11}} & 0                 & 0                 \\
            \overline{r_{12}} & \overline{r_{22}} & 0                 \\
            \overline{r_{13}} & \overline{r_{23}} & \overline{r_{33}}
        \end{pmatrix}
    \]
    Вычислим только два диагональных элемента в левой и правой частях:
    \[
        \begin{pmatrix}
            \modulus{r_{11}}^2 & \dots                                   & \dots \\
            \dots              & \modulus{r_{12}}^2 + \modulus{r_{22}}^2 & \dots \\
            \dots              & \dots                                   & \dots
        \end{pmatrix}
        =
        \begin{pmatrix}
            \modulus{r_{11}}^2 + \modulus{r_{12}}^2 + \modulus{r_{13}}^2 & \dots                                   & \dots \\
            \dots                                                        & \modulus{r_{22}}^2 + \modulus{r_{23}}^2 & \dots \\
            \dots                                                        & \dots                                   & \dots
        \end{pmatrix}
    \]
    Сравнивая диагональные элементы в левой и правой частях, получим равенства:
    \begin{align*}
        \modulus{r_{11}}^2 = \modulus{r_{11}}^2 + \modulus{r_{12}}^2 + \modulus{r_{13}}^2
        & \Rightarrow \modulus{r_{12}}^2 = 0, \modulus{r_{13}}^2 = 0
        \Rightarrow r_{12} = 0, r_{13} = 0 , \\
        %
        \modulus{r_{12}}^2 + \modulus{r_{22}}^2 = \modulus{r_{22}}^2 + \modulus{r_{23}}^2
        & \Rightarrow \modulus{r_{22}}^2 = \modulus{r_{22}}^2 + \modulus{r_{23}}^2
        \Rightarrow \modulus{r_{23}}^2 = 0
        \Rightarrow r_{23} = 0
    \end{align*}
    Таким образом внедиагональные элементы матрицы $R$ равны нулю.
\end{example}

Таким образом, если $A$ --- нормальная матрица, то она диагонализуема:
\begin{equation}
    \label{rayleigh:spectral:decomposition}
    A = U \Lambda U^* .
\end{equation}
Такое разложение называется спектральным, и называется так потому, что, умножая справа на $U$, получим равенство:
\[
    A U = U \Lambda,
\]
откуда следует, что столбцы матрицы $U$ определяют собственные векторы, а элементы матрицы $\Lambda$ являются собственными числами. Действительно, если представить матрицу $U$
в виде набора столбцов $u_i$:
\[
    U = \begin{pmatrix}
            u_1 & u_2 & \dots & u_n
    \end{pmatrix} ,
\]
тогда
\begin{gather*}
    A \begin{pmatrix}
          u_1 & u_2 & \dots & u_n
    \end{pmatrix}
    =
    \begin{pmatrix}
        u_1 & u_2 & \dots & u_n
    \end{pmatrix}
    \begin{pmatrix}
        \lambda_1 & 0         & \dots  & 0         \\
        0         & \lambda_2 & \dots  & 0         \\
        \vdots    & \vdots    & \ddots & \vdots    \\
        0         & 0         & \dots  & \lambda_n
    \end{pmatrix} , \\
    %
    \begin{pmatrix}
        A u_1 & A u_2 & \dots & A u_n
    \end{pmatrix}
    =
    \begin{pmatrix}
        \lambda_1 u_1 & \lambda_2 u_2 & \dots & \lambda_n u_n
    \end{pmatrix}
\end{gather*}
откуда следует, что
\[
    A u_i = \lambda_i u_i .
\]

\begin{example}
    Найти спектральное разложение для матрицы $A$:
    \[
        A
        = \begin{pmatrix}
              -7          & -5 \sqrt{3} \\
              -5 \sqrt{3} & 3
        \end{pmatrix}
    \]
    Находим собственные числа, с помощью характеристического полинома:
    \begin{multline*}
        \det \left ( A - \lambda E \right )
        = \begin{vmatrix}
              -7 - \lambda & -5 \sqrt{3} \\
              -5 \sqrt{3}  & 3 - \lambda
        \end{vmatrix}
        = (-7 - \lambda)(3 - \lambda) - 25 \cdot 3 = \\
        %
        = \lambda^2 + 4 \lambda - 21 - 75
        = \lambda^2 + 4 \lambda - 96.
    \end{multline*}
    Откуда корни характеристического полинома:
    \begin{gather*}
        \lambda_{1,2} = -2 \pm \sqrt{4 + 96} = -2 \pm 10 , \\
        \lambda_1 = -12, \lambda_2 = 8.
    \end{gather*}
    Таким образом, матрица $\Lambda$:
    \[
        \Lambda
        = \begin{pmatrix}
              -12 & 0 \\
              0   & 8
        \end{pmatrix}
    \]
    Столбцы матрицы $U$ являются собственными векторами, пусть $u_{ij}$ обозначают элементы матрицы $U$:
    \[
        U
        = \begin{pmatrix}
              u_{11} & u_{12} \\
              u_{21} & u_{22}
        \end{pmatrix} ,
    \]
    тогда
    \begin{gather*}
        A
        \begin{pmatrix}
            u_{11} \\
            u_{21}
        \end{pmatrix}
        = \lambda_1
        \begin{pmatrix}
            u_{11} \\
            u_{21}
        \end{pmatrix} , \\
        %
        \left ( A - \lambda_1 E \right )
        \begin{pmatrix}
            u_{11} \\
            u_{21}
        \end{pmatrix}
        = 0 , \\
        %
        \begin{pmatrix}
            -7 + 12     & -5 \sqrt{3} \\
            -5 \sqrt{3} & 3 + 12
        \end{pmatrix}
        \begin{pmatrix}
            u_{11} \\
            u_{21}
        \end{pmatrix}
        = 0 , \\
        %
        \begin{pmatrix}
            5           & -5 \sqrt{3} \\
            -5 \sqrt{3} & 15
        \end{pmatrix}
        \begin{pmatrix}
            u_{11} \\
            u_{21}
        \end{pmatrix}
        = 0 , \\
        \begin{pmatrix}
            1          & - \sqrt{3} \\
            - \sqrt{3} & 3
        \end{pmatrix}
        \begin{pmatrix}
            u_{11} \\
            u_{21}
        \end{pmatrix}
        = 0 , \\
        \begin{pmatrix}
            u_{11} \\
            u_{21}
        \end{pmatrix}
        = \begin{pmatrix}
              \sqrt{3} c \\
              c
        \end{pmatrix} ,
    \end{gather*}
    где постоянная $c$ выбирается из условия нормировки:
    \begin{gather*}
        u_{11}^2 + u_{21}^2 = 1 , \\
        3 c^2 + c^2 = 1 , \\
        4 c^2 = 1 , \\
        c = \pm \frac{1}{2} .
    \end{gather*}
    Таким образом:
    \[
        \begin{pmatrix}
            u_{11} \\
            u_{21}
        \end{pmatrix}
        = \pm \begin{pmatrix}
                  \frac{\sqrt{3}}{2} \\
                  \frac{1}{2}
        \end{pmatrix}
    \]
    Аналогично
    \begin{gather*}
        A
        \begin{pmatrix}
            u_{12} \\
            u_{22}
        \end{pmatrix}
        = \lambda_2
        \begin{pmatrix}
            u_{12} \\
            u_{22}
        \end{pmatrix} , \\
        %
        \left ( A - \lambda_2 E \right )
        \begin{pmatrix}
            u_{12} \\
            u_{22}
        \end{pmatrix}
        = 0 , \\
        %
        \begin{pmatrix}
            -7 - 8      & -5 \sqrt{3} \\
            -5 \sqrt{3} & 3 - 8
        \end{pmatrix}
        \begin{pmatrix}
            u_{12} \\
            u_{21}
        \end{pmatrix}
        = 0 , \\
        %
        \begin{pmatrix}
            -15         & -5 \sqrt{3} \\
            -5 \sqrt{3} & -5
        \end{pmatrix}
        \begin{pmatrix}
            u_{12} \\
            u_{22}
        \end{pmatrix}
        = 0 , \\
        \begin{pmatrix}
            3        & \sqrt{3} \\
            \sqrt{3} & 1
        \end{pmatrix}
        \begin{pmatrix}
            u_{12} \\
            u_{22}
        \end{pmatrix}
        = 0 , \\
        \begin{pmatrix}
            u_{12} \\
            u_{22}
        \end{pmatrix}
        = \begin{pmatrix}
              c \\
              - \sqrt{3}  c
        \end{pmatrix} ,
    \end{gather*}
    где постоянная $c$ выбирается из условия нормировки:
    \begin{gather*}
        u_{12}^2 + u_{12}^2 = 1 , \\
        c^2 + 3 c^2 = 1 , \\
        4 c^2 = 1 , \\
        c = \pm \frac{1}{2} .
    \end{gather*}
    Таким образом:
    \[
        \begin{pmatrix}
            u_{12} \\
            u_{22}
        \end{pmatrix}
        = \pm \begin{pmatrix}
                  \frac{1}{2} \\
                  - \frac{\sqrt{3}}{2}
        \end{pmatrix}
    \]
    и один из вариантов матрицы $U$:
    \[
        U
        = \begin{pmatrix}
              \frac{\sqrt{3}}{2} & \frac{1}{2}          \\
              \frac{1}{2}        & - \frac{\sqrt{3}}{2}
        \end{pmatrix} .
    \]
\end{example}


\section{Эрмитовые матрицы}

Пусть $A$ --- эрмитовая матрица:
\[
    A^* = A.
\]
Поскольку эрмитовые матрицы являются нормальными матрицами, то для матрицы $A$ cуществует спектральное разложение
\eqref{rayleigh:spectral:decomposition}, используя которое, получим:
\begin{align*}
    \left ( U \Lambda U^* \right )^* & = U \Lambda U^* , \\
    U \Lambda^* U^* & = U \Lambda U^* , \\
    U^* U \Lambda^* U^* U & = U^* U \Lambda U^* U, \\
    \Lambda^* & = \Lambda .
\end{align*}
Если $\lambda_i$ --- диагональный элемент матрицы $\Lambda$, тогда:
\[
    \overline{\lambda_i} = \lambda_i
\]
а это возможно тогда и только тогда, когда число $\lambda_i$ --- вещественное:
\[
    \lambda_i \in \mathbb{R}.
\]
Таким образом, у эрмитовой матрицы все собственные значения вещественные.

Рассмотрим квадратичную форму с эрмитовой матрицей $A$ и произвольным вектором $u$:
\[
    \overline{u^* A u}
    = \left ( u^* A u \right )^*
    = u^* A^* (u^*)^*
    = u^* A^* u
    = u^* A u .
\]
Отсюда следует, что значения квадратичной формы $u^* A u$ являются вещественными при всех векторах $u$. Пусть дополнительно оператор $A$
является неотрицательно определенным, то есть для всех векторов $u$:
\[
    u^* A u \ge 0
\]
Если $u_i$ --- собственный вектор, соответствующей собственному значению $\lambda_i$, тогда:
\begin{align*}
    u_i^* A u_i & \ge 0 , \\
    u_i^* \lambda_i u_i & \ge 0 , \\
    \lambda_i u_i^* u_i & \ge 0 , \\
    \lambda_i \scalarproduct{u_i}{u_i} & \ge 0 , \\
    \lambda_i \norm{u_i}^2 & \ge 0 , \\
    \lambda_i \ge 0 ,
\end{align*}
поскольку собственный вектор $u_i \neq 0$ и $\norm{u_i} > 0$. Таким образом, если $A \ge 0$, то все собственные числа неотрицательны, и для диагональной
матрицы $\Lambda$:
\[
    \Lambda
    = \begin{pmatrix}
          \lambda_1 & 0         & \dots  & 0         \\
          0         & \lambda_2 & \dots  & 0         \\
          \vdots    & \vdots    & \ddots & \vdots    \\
          0         & 0         & \dots  & \lambda_n
    \end{pmatrix}
\]
определен квадратный корень:
\begin{gather*}
    \Lambda = \Lambda^{\frac{1}{2}} \Lambda^{\frac{1}{2}} , \\
    %
    \Lambda^\frac{1}{2}
    = \begin{pmatrix}
          \lambda_1^\frac{1}{2} & 0                     & \dots  & 0                     \\
          0                     & \lambda_2^\frac{1}{2} & \dots  & 0                     \\
          \vdots                & \vdots                & \ddots & \vdots                \\
          0                     & 0                     & \dots  & \lambda_n^\frac{1}{2}
    \end{pmatrix}
    .
\end{gather*}
Тогда из спектрального разложения матрицы $A$ оператора \eqref{rayleigh:spectral:decomposition}:
\[
    A
    = U \Lambda U^*
    = U \Lambda^\frac{1}{2} \Lambda^\frac{1}{2} U^*
    = \left ( U \Lambda^\frac{1}{2} \right ) \left ( U \Lambda^\frac{1}{2} \right )^*
    = A^\frac{1}{2} \left ( A^\frac{1}{2} \right )^*,
\]
где
\[
    A^\frac{1}{2} = U \Lambda^\frac{1}{2}
\]
квадратный корень из эрмитовой матрицы $A$.

\begin{example}
    Пусть матрица $A$ имеет вид:
    \[
        A
        = \begin{pmatrix}
              43          & -7 \sqrt{3} \\
              -7 \sqrt{3} & 57
        \end{pmatrix} .
    \]
    Спектральное разложение матрицы $A$ имеет вид:
    \[
        A
        =
        \underbrace{
            \begin{pmatrix}
                \frac{\sqrt{3}}{2} & \frac{1}{2}         \\
                \frac{1}{2}        & -\frac{\sqrt{3}}{2}
            \end{pmatrix}
        }_{U}
        \underbrace{
            \begin{pmatrix}
                36 & 0  \\
                0  & 64
            \end{pmatrix}
        }_{\Lambda}
        \underbrace{
            \begin{pmatrix}
                \frac{\sqrt{3}}{2} & \frac{1}{2}         \\
                \frac{1}{2}        & -\frac{\sqrt{3}}{2}
            \end{pmatrix}
        }_{U^*} .
    \]
    Откуда квадратный корень
    \[
        A^\frac{1}{2}
        =
        \underbrace{
            \begin{pmatrix}
                \frac{\sqrt{3}}{2} & \frac{1}{2}         \\
                \frac{1}{2}        & -\frac{\sqrt{3}}{2}
            \end{pmatrix}
        }_{U}
        \underbrace{
            \begin{pmatrix}
                6 & 0 \\
                0 & 8
            \end{pmatrix}
        }_{\Lambda^\frac{1}{2}} .
    \]
\end{example}


\section{Вещественные квадратичные формы}

Если матрица $A$ определена ($A > 0$, $A \ge 0$, $A \le 0$ или $A < 0$), то при всех $x$ квадратичную форму $x^* A x$ можно сравнивать с нулём, а значит она является
вещественным числом. Отсюда сразу следует, что $A$ является эрмитовой. Действительно:
\begin{gather*}
    x^* A x \in \mathbb{R} , \\
    x^* A x = ( x^* A x )^* , \\
    x^* A x = x^* A^* x .
\end{gather*}

Возьмем в качестве $x$ векторы вида $e_k = (0, \dots, 0, 1, 0, \dots, 0)$ с одной единицей, тогда из равенства квадратичных форм следует, что диагональные
элементы матриц $A$ и $A^*$ вещественны и одинаковы:
\begin{gather*}
    e_k^* A e_k = e_k^* A^* e_k , \\
    a_{kk} = a_{kk}^* .
\end{gather*}
Возьмем в качестве $x$ векторы $e_{kj} = (0, \dots, 0, 1, 0, \dots, 0, 1, 0, \dots, 0)$ с двумя единицами, тогда из равенства квадратичных форм следует,
что внедиагональные элементы сопряжены:
\begin{gather*}
    e_{kj}^* A e_{kj} = e_{kj}^* A^* e_{kj} , \\
    a_{kk} + a_{kj} + a_{jk} + a_{jj} = a_{kk}^* + a_{jk}^* + a_{kj}^* + a_{jj}^* , \\
    a_{kk} + a_{kj} + a_{jk} + a_{jj} = a_{kk} + a_{jk}^* + a_{kj}^* + a_{jj} , \\
    a_{kj} + a_{jk} = a_{jk}^* + a_{kj}^* , \\
    a_{kj} + a_{jk} = a_{kj}^* + a_{jk}^* , \\
    a_{kj} + a_{jk} = ( a_{kj} + a_{jk} )^* , \\
    \image{a_{kj} + a_{jk}} = 0 , \\
    \image{a_{kj}} = - \image{a_{jk}} .
\end{gather*}
Возьмем в качестве $x$ векторы $e_{kj} = (0, \dots, 0, 1, 0, \dots, 0, -1, 0, \dots, 0)$ с двумя единицами, тогда из равенства квадратичных форм следует,
что внедиагональные элементы сопряжены:
\begin{gather*}
    e_{kj}^* A e_{kj} = e_{kj}^* A^* e_{kj} , \\
    a_{kk} + i a_{kj} - i a_{jk} - i^2 a_{jj} = a_{kk}^* + i a_{jk}^* - i a_{kj}^* - i^2 a_{jj}^* , \\
    a_{kk} + i a_{kj} - i a_{jk} + a_{jj} = a_{kk}^* + i a_{jk}^* - i a_{kj}^* + a_{jj}^* , \\
    a_{kk} + i a_{kj} - i a_{jk} + a_{jj} = a_{kk} + i a_{jk}^* - i a_{kj}^* + a_{jj} , \\
    i a_{kj} - i a_{jk} = i a_{jk}^* - i a_{kj}^* , \\
    i a_{kj} - i a_{jk} = ( - i a_{jk} + i a_{kj} )^* , \\
    i a_{kj} - i a_{jk} = ( i a_{kj} - i a_{jk})^* , \\
    \image{i a_{kj} - i a_{jk}} = 0 , \\
    \real{a_{kj} - a_{jk}} = 0 , \\
    \real{a_{kj}} = \real{a_{jk}} .
\end{gather*}
Таким образом,
\[
    A = A^*.
\]


\section{Экстремумы}~\label{rayleigh:extrema}

В радиолокации физические колебательные процессы часто описываются с помощью вектора комплексных амплитуд $x \in \Cspace{n}$. Кроме того, выделение линейной части
преобразований приводит к векторам $Fx$. Далее, обычно, интересуются энергией, которая пропорциональна квадратам норм:
\begin{gather*}
    \norm{x}^2 = x^* x , \\
    \norm{F x}^2 = x^* F^* F x ,
\end{gather*}
и сравнением энергий, которое приводит к отношениям вида:
\begin{gather*}
    \rho(x) = \frac{x^* F^* F x}{x^* x}.
\end{gather*}
Легко видеть, что матрица $F^* F$ является эрмитовой:
\[
    ( F^* F )^* = F^* F .
\]

В более общем случае рассматривается отношение:
\[
    \rho(x) = \frac{x^* A x}{x^* B x},
\]
где $A$ и $B$ --- эрмитовы матрицы и $B > 0$.

Заметим, что отношение Релея зависит только от направления вектора $x$, но не зависит от величины вектора $x$, действительно:
\begin{equation}
    \label{rayleight:extrema:homogenity}
    \rho(\alpha x)
    = \frac{\alpha^* x^* A \alpha x}{ \alpha^* x^* B \alpha x}
    = \frac{\modulus{\alpha}^2 \cdot x^* A x}{ \modulus{\alpha}^2 \cdot x^* B x}
    = \frac{x^* A x}{x^* B x}
    = \rho(x) ,
\end{equation}
поэтому при анализе значений отношения Релея можно ограничится рассмотрением векторов $x$ единичной нормы $\norm{x} = 1$.

Для положительно определенной матрицы $B > 0$ существует квадратный корень $B^\frac{1}{2}$:
\begin{gather*}
    B = B^\frac{1}{2} ( B^\frac{1}{2} )^* , \\
    %
    B^\frac{1}{2} = U_B \Lambda_B^\frac{1}{2} ,
\end{gather*}
причём
\[
    \det B^\frac{1}{2}
    = \det ( U_B \Lambda_B^\frac{1}{2} )
    = \det U_B \cdot \det \Lambda_B^\frac{1}{2}
    = 1 \cdot \det \Lambda_B^\frac{1}{2}
    > 0 ,
\]
поэтому существует обратная матрица $B^{-\frac{1}{2}}$:
\[
    B^{-\frac{1}{2}}
    = \left ( U_B \Lambda_B^\frac{1}{2} \right )^{-1}
    = \left ( \Lambda_B^\frac{1}{2} \right )^{-1} U_B^{-1}
    = \left ( \Lambda_B^\frac{1}{2} \right )^{-1} U_B^* .
\]

Отношение Релея $\rho(x)$ можно представить в виде:
\begin{gather*}
    \rho(x)
    = \frac{x^* A x}{x^* B^\frac{1}{2} ( B^\frac{1}{2} )^* x}
    = \frac{x^* B^\frac{1}{2} B^{-\frac{1}{2}} A ( B^{-\frac{1}{2}} )^* ( B^\frac{1}{2} )^* x}{x^* B^\frac{1}{2} ( B^\frac{1}{2} )^* x} , \\
    %
    \rho(y)
    = \frac{y^* B^{-\frac{1}{2}} A ( B^{-\frac{1}{2}} )^* y}{y^* y}
    = \frac{y^* C y}{y^* y}, \\
    %
    C = B^{-\frac{1}{2}} A ( B^{-\frac{1}{2}} )^*, \\
    %
    y = ( B^\frac{1}{2} )^* x .
\end{gather*}

Заметим, что матрица $C$ является эрмитовой:
\[
    C^*
    = ( B^{-\frac{1}{2}} A ( B^{-\frac{1}{2}} )^* )^*
    = B^{-\frac{1}{2}} A^* ( B^{-\frac{1}{2}} )^*
    = B^{-\frac{1}{2}} A ( B^{-\frac{1}{2}} )^*
    = C,
\]
поэтому она подобна диагональной матрице:
\[
    C = U_C \Lambda_C U_C^* .
\]
Используя это представление матрицы $C$, преобразуем отношение Релея к виду:
\begin{gather*}
    \rho(y)
    = \frac{y^* U_C \Lambda_C U_C^* y}{y^* y}
    = \frac{y^* U_C \Lambda_C U_C^* y}{y^* U_C U_C^* y} , \\
    %
    \rho(z)
    = \frac{z^* \Lambda_C z}{z^* z} , \\
    %
    z = U_C^* y
\end{gather*}

Согласно равенству \eqref{rayleight:extrema:homogenity} отношение Релея не зависит от величины вектора $z$, а только от его направления, поэтому можно ограничится рассмотрением
векторов $z$, для которых:
\begin{gather*}
    \norm{z} = 1 , \\
    %
    \rho(z) = z^* \Lambda_C z .
\end{gather*}

Пусть
\begin{gather*}
    \Lambda_C
    = \begin{pmatrix}
          \lambda_1 & 0         & \dots  & 0         \\
          0         & \lambda_2 & \dots  & 0         \\
          \vdots    & \vdots    & \ddots & \vdots    \\
          0         & 0         & \dots  & \lambda_n
    \end{pmatrix} , \\
    %
    \lambda_1 \ge \lambda_2 \ge \dots \ge \lambda_n.
\end{gather*}
тогда
\begin{gather*}
    \rho(z)
    = \lambda_1 \modulus{z_1}^2 + \lambda_2 \modulus{z_2}^2 + \dots + \lambda_n \modulus{z_n}^2, \\
    %
    \modulus{z_1}^2 + \modulus{z_2}^2 + \dots + \modulus{z_n}^2 = 1.
\end{gather*}
Из последнего равенства следует, что
\[
    0 \le \modulus{z_i}^2 \le 1 ,
\]
поэтому
\begin{gather*}
    \lambda_n \le \rho(z) \le \lambda_1 .
\end{gather*}

Максимальное значение отношение Релея достигает при векторе $z_{max}$:
\[
    z_{max}
    = \begin{pmatrix}
          1     \\
          0     \\
          \dots \\
          0
    \end{pmatrix} ,
\]
которому соответствует вектор $y_{max}$:
\begin{align*}
    z_{max} & = U_C^* y_{max} , \\
    U_C z_{max} & = y_{max} ,
\end{align*}
которому соответствует вектор $x_{max}$:
\begin{align*}
    \left ( B^\frac{1}{2} \right )^* x_{max} & = y_{max} = U_C z_{max} , \\
    \left ( U_B \Lambda_B^\frac{1}{2} \right )^* x_{max} & = U_C z_{max} , \\
    \Lambda_B^\frac{1}{2} U_B^* x_{max} & = U_C z_{max} , \\
    U_B^* x_{max} & = \Lambda_B^{-\frac{1}{2}} U_C z_{max} , \\
    x_{max} & = U_B \Lambda_B^{-\frac{1}{2}} U_C z_{max} .
\end{align*}
Аналогично минимальное значение отношение Релея достигает при векторе $z_{min}$:
\[
    z_{min}
    = \begin{pmatrix}
          0     \\
          \dots \\
          0     \\
          1
    \end{pmatrix} ,
\]
которому соответствует вектор $x_{min}$:
\[
    x_{min} = U_B \Lambda_B^{-\frac{1}{2}} U_C z_{min} .
\]


    % операции в Matlab
    \chapter{Операции в Matlab}

\section{Вычислительные}

\subsection{Арифметические}

Вектор--строка:
\matlab{x = [1 2 3]}

Вектор--столбец:
\matlab{y = [1; 2; 3]}

Матрица:
\matlab{A = [1 2 3; 4 5 6]}

Сложение:
\begin{Matlab}
    \Mcommand{B = [-2 -5 8; 3 1 -8]}
    \Mcommand{C = A + B}
\end{Matlab}

Умножение вектора на матрицу:
\matlab{z = A * y}

Сопряжение (транспонирование)
\matlab{Ac = A'}

Вычисление собственных чисел и векторов:
\begin{Matlab}
    \Mcommand{A = [-3 2; 3 -5];}
    \Mcommand{[V, D] = eig(A);}
\end{Matlab}

\subsection{Поэлементные}

Поэлементное умножение матриц:
\begin{Matlab}
    \Mcommand{A = [1 2 3; 4 5 6];}
    \Mcommand{B = [-2 -5 8; 3 1 -8];}
    \Mcommand{C = A .* B;}
\end{Matlab}

Поэлементное возведение в квадрат:
\begin{Matlab}
    \Mcommand{x = [1 3 5 7];}
    \Mcommand{y = x.$\hat{}$ 2;}
\end{Matlab}

\section{Графические}

График функции в 2D:
\begin{Matlab}
    \Mcommand{x = -1:0.1:1;}
    \Mcommand{y = x.$\hat{}$ 3;}
    \Mcommand{plot(x,y)}
\end{Matlab}

График кривой в 3D:
\begin{Matlab}
    \Mcommand{a = [ 0:0.1:2*pi 2*pi];}
    \Mcommand{x = cos(2*a);}
    \Mcommand{y = sin(2*a);}
    \Mcommand{z = a;}
\end{Matlab}

    % сигналы
    \chapter{Сигналы}


\section{Комплексное представление сигналов}

\subsection{Комплексные числа}

Комплексные числа $z = a + i b$ состоят из двух частей: действительной
\[
    a = \real{z},
\]
и мнимой
\[
    b = \image{z},
\]
где $a$ и $b$ --- действительные числа.

Комплексные числа можно представлять векторами на комплексной плоскости.

Если комплексное число не нулевое, то можно получить тригонометрическую форму:
\begin{gather*}
    z
    = a + i b
    = \sqrt{a^2 + b^2} \left ( \frac{a}{\sqrt{a^2 + b^2}} + i \frac{b}{\sqrt{a^2 + b^2}} \right )
    = A \left ( \cos \varphi + i \sin \varphi \right )
    = A \cos \varphi + i A \sin \varphi , \\
    %
    A = \sqrt{a^2 + b^2} , \\
    \cos \varphi = \frac{a}{\sqrt{a^2 + b^2}} , \\
    \sin \varphi = \frac{b}{\sqrt{a^2 + b^2}} ,
\end{gather*}
где $A$ и $\varphi$ --- модуль и аргумент комплексного числа.

\subsection{Комплексная экспонента}

Для вещественных чисел $a$ определена функция $e^a$, а её продолжением на комплексную плоскость является функция:
\[
    e^{a + i b}
    = e^a \cdot e^{i b}
    = e^a \cdot \left ( \cos b + i \sin b \right ).
\]

\subsection{Комплексное представление сигналов}

Пусть функция $u(t)$ описывает напряжение сигнала:
\[
    u(t) = A(t) \cos \varphi(t) ,
\]
где $A(t)$ --- амплитуда, $\varphi(t)$ --- фаза.

Функция $u(t)$ соответствует действительной части комплексно-значной функции $v(t)$ вещественного переменного $t$:
\begin{gather*}
    u(t) = \real{v(t)} , \\
    %
    v(t)
    = A(t) \cos \varphi(t) + i A(t) \sin \varphi(t)
    = A(t) \left ( \cos \varphi(t) + i \sin \varphi(t) \right )
    = A(t) e^{i \varphi(t)} .
\end{gather*}

\subsection{Сложение сигналов}

Пусть имеются сигналы, которые описываются функциями $u_1(t)$ и $u_2(t)$:
\begin{gather*}
    u_1(t) = A_1(t) \cos \varphi_1(t) , \\
    u_2(t) = A_2(t) \cos \varphi_2(t) ,
\end{gather*}
которым соответствую комплексные представления:
\begin{gather*}
    v_1(t) = A_1(t) e^{i \varphi_1(t)} , \\
    v_2(t) = A_2(t) e^{i \varphi_2(t)} .
\end{gather*}
тогда в результате сложения получается функция $u(t)$:
\[
    u(t)
    = u_1(t) + u_2(t)
    = \real{v_1(t)} + \real{v_2(t)}
    = \real{v_1(t) + v_2(t)} ,
\]
которой соответствует комплексное представление:
\[
    v(t) = v_1(t) + v_2(t).
\]

\subsection{Преобразование}

Пусть сигнал с функцией $u(t)$ и комплексным представлением $v(t)$:
\begin{gather*}
    u(t) = A(t) \cos \varphi(t) , \\
    v(t) = A(t) e^{i \varphi(t)}
\end{gather*}
преобразуется в некотором устройстве, и в результате преобразования изменяются амплитуда или фаза:
\[
    \widetilde{u}(t) = B(t) \cdot A(t) \cos ( \varphi(t) + \theta(t) ) ,
\]
тогда комплексное представление $\widetilde{v}(t)$ сигнала $\widetilde{u}(t)$ имеет вид:
\[
    \widetilde{v}(t)
    = A(t) \cdot B(t) e^{i (\varphi(t) + \theta(t))}
    = A(t) e^{i \varphi(t)} \cdot B(t) e^{i \theta(t)} = v(t) \cdot s(t),
\]
где функция $s(t)$ является комплексным представлением преобразования:
\[
    s(t) = B(t) e^{i \theta(t)} .
\]

\subsection{Комплексная огибающая}

Наиболее часто в радиолокации встречаются узкополосные сигналы, представляющие собой суперпозицию колебаний с частотами из узкой полосы частот, вокруг несущей
частоты $\omega$ (частота $\omega$ обычно составляет несколько мегагерц, поскольку антенны излучают сигналы высоких частот):
\[
    u(t) = A(t) \cos ( \omega t + \theta(t) ),
\]
где функции $A(t)$ и $\varphi(t)$ представляют модуляцию сигнала. Такому сигналу соответствует комплексное представление:
\[
    v(t)
    = A(t) e^{i ( \omega t + \theta(t) )}
    = A(t) e^{i \theta(t)} \cdot e^{i \omega t} .
\]
Первый множитель
\[
    v_s(t) = A(t) e^{i \theta(t)}
\]
является комплексной огибающей.

\subsection{Квадратурный детектор}

Сигнал $u(t)$ можно представить в виде суммы:
\[
    u(t)
    = A(t) \cos ( \omega t + \varphi(t) )
    = A(t) \cos \varphi(t) \cos \omega t - A(t) \sin \varphi(t) \sin \omega t .
\]
в которой множители
\begin{gather*}
    I(t) = A(t) \cos \varphi(t) , \\
    Q(t) = - A(t) \sin \varphi(t) ,
\end{gather*}
называются синфазной и квадратурной составляющими соответственно.

Составляющие $I(t)$ и $Q(t)$ определяют действительную и мнимую части комплексной огибающей $v(t)$:
\begin{gather*}
    I(t) = \real{v(t)} , \\
    Q(t) = - \image{v(t)} .
\end{gather*}

Устройство, которое выделяет составляющие $I(t)$ и $Q(t)$, называется квадратурным детектором.


    % излучение
    \chapter{Излучение}


\section{Двухканальный излучатель}

\subsection{Модель}

Рассматривается излучатель с двумя каналами поляризации и двумя входами, соответствующим этим двум каналам. Возникающие физические явления описываются только
линейными выражениями, а нелинейные эффекты не учитываются.

Пусть $a$ --- вектор огибающих входов излучателя:
\[
    a
    = \begin{pmatrix}
          a_1 \\
          a_2
    \end{pmatrix} .
\]

В результате отражений и связи между каналами, наводятся отраженные сигналы с огибающими $b$:
\begin{gather*}
    b
    = \begin{pmatrix}
          b_1 \\
          b_2
    \end{pmatrix}
    = \begin{pmatrix}
          s_{11} a_1 + s_{12} a_2 \\
          s_{21} a_1 + s_{22} a_2
    \end{pmatrix}
    = S a, \\
    %
    S
    = \begin{pmatrix}
          s_{11} & s_{12} \\
          s_{21} & s_{22}
    \end{pmatrix},
\end{gather*}
где $S$ --- матрица рассеяния.

        {
    \color{red}
    Может это исключить?

    Неотраженные части сигналов формируют электромагнитные волны, электрические колебания которых характеризуются огибающими $e$:
    \begin{gather*}
        e
        = \begin{pmatrix}
              e_\theta \\
              e_\varphi
        \end{pmatrix}
        = \begin{pmatrix}
              t_{1, \theta} a_1 + t_{2, \theta} a_2 \\
              t_{1, \varphi} a_1 + t_{2, \varphi} a_2
        \end{pmatrix}
        = T a, \\
        %
        T
        = \begin{pmatrix}
              t_{1, \theta}  & t_{2, \theta}  \\
              t_{1, \varphi} & t_{2, \varphi}
        \end{pmatrix} .
    \end{gather*}
    где $T$ --- матрица коэффициентов передачи со входов в волноводы.
}

\subsection{Диаграмма направленности}

Из волноводов происходит излучение электромагнитной волны в пространство, излучение является неоднородным и его характеристики зависят от рассматриваемого направления.
Если с излучателем связана сферическая система координат (с началом отсчёта в конце волноводов), то направление можно задать с помощью волнового вектора $w$,
длина которого:
\[
    \modulus{w} = \frac{2 \pi}{\lambda},
\]
где $\lambda$ --- длина волны на несущей частоте. В выбранном направлении $w$ на расстоянии $R$ напряженность электрического поля $E$ является функцией
направления $w$, расстояния $R$ и огибающих волн $e$ в волноводах излучателя, которые зависят от огибающих $a$ сигналов на входах:
\[
    E = E(w, R, a).
\]
В дальней зоне при больших расстояниях $R$ напряженность $E$ можно приближенно представить линейной частью по огибающим сигналов $a$:
\begin{equation}
    \label{emission:emitter:diagram:tension}
    E(w,R)
    \approx F(w) a \cdot \frac{e^{i \modulus{w} R}}{R} ,
\end{equation}
где множитель $e^{i \modulus{w} R}$ определяет смещение по фазе на расстоянии $R$ от начала отсчёта, множитель $\frac{1}{R}$ показывает затухание амплитуды
вектора напряженности $E$, и $F(w)$ --- диаграмма направленности излучателя:
\[
    F(w)
    = \begin{pmatrix}
          f_{1, \theta}(w)  & f_{2, \theta}(w)  \\
          f_{1, \varphi}(w) & f_{2, \varphi}(w)
    \end{pmatrix} ,
\]
в которой столбцы задают парциальные диаграммы направленности по каналам поляризации:
\begin{equation}
    f_1(w)
    = \begin{pmatrix}
          f_{1,\theta}(w) \\
          f_{1,\varphi}(w)
    \end{pmatrix}
    , \;
    f_2(w)
    = \begin{pmatrix}
          f_{2,\theta}(w) \\
          f_{2,\varphi}(w)
    \end{pmatrix}
    \label{emission:emitter:diagram:partial}
    .
\end{equation}


\subsection{Энергетическое ограничение}

Для излучателя выполняется закон сохранения энергии --- суммарная мощность входных сигналов $P_{inp}$ совпадает с суммой мощности отраженных сигналов $P_{ref}$,
мощности излучения $P_{rad}$ и мощности $P_{dis}$ дисспативных потерь:
\begin{equation}
    \label{emission:emitter:power:equality}
    P_{ref} + P_{rad} + P_{dis} = P_{inp} ,
\end{equation}
где
\begin{align}
    P_{inp} & = \norm{a}^2 = a^* a , \label{emission:emitter:power:input}\\
    P_{ref} & = \norm{b}^2 = \norm{S a} = a^* S^* S a \notag,
\end{align}
и выражение для мощности излучения $P_{rad}$ имеет вид:
\begin{gather*}
    P_{rad}
    = \iiint \limits_S \norm{E(w)}^2 ds
    = \iint \limits_{4 \pi} \norm{F(w) a}^2 d \Omega , \\
    %
    \norm{F(w) a}^2
    = a^* F^*(w) F(w) a .
\end{gather*}
Таким образом, мощность излучения $P_{rad}$ можно представить в виде квадратичной формы:
\begin{equation}
    \label{emission:emitter:power:radiated}
    P_{rad}
    = a^* Q a ,
\end{equation}
где элементами матрицы $Q$ являются интегралы вида:
\[
    Q_{jk} = \iint \limits_{4 \pi} f_j^*(w) f_k(w) d \Omega .
\]

Таким образом, равенство \eqref{emission:emitter:power:equality} можно представить в виде:
\begin{gather*}
    a^* S^* S a + a^* Q a + P_{dis} = a^* a , \\
    a^* Q a = a^* a - a^* S^* S a - P_{dis}, \\
    a^* Q a = a^* ( I - S^* S ) a - P_{dis} .
\end{gather*}
Откуда
\begin{gather*}
    \label{emission:emitter:power:inequality}
    a^* Q a \le a^* ( I - S^* S ) a \le a^* a.
\end{gather*}


\subsection{Коэффициент усиления}

Реализованный коэффициент усиления $G(w)$ определяет относительную величину плотности потока $\Pi(\theta, \varphi)$ мощности в дальней зоне в
направлении $w$:
\[
    G(w)
    = \frac{4 \pi R^2}{P_{inp}} \cdot \Pi(w)
    = \frac{4 \pi R^2}{P_{inp}} \cdot \frac{1}{Z_0} \norm{E(w)}^2
    = \frac{4 \pi R^2}{P_{inp}} \cdot \frac{1}{Z_0} \frac{\norm{F(w) a}^2}{R^2}
    = \frac{4 \pi}{Z_0} \cdot \frac{\norm{F(w) a}^2}{P_{inp}} ,
\]
где $Z_0 = 120 \pi$ --- волновое сопротивление свободного пространства.

Пусть направление $w$ фиксировано, тогда коэффициент усиления пропорционален отношению:
\[
    \rho(a)
    = \frac{\norm{F(w) a}^2}{P_{inp}}
    = \frac{a^* F^*(w) F(w) a}{a^* a}
\]
и возникает вопрос, каким образом нужно сформировать огибающие входных сигналов $a$ чтобы отношение $\rho(a)$ и коэффициент усиления в заданном направлении $w$
оказался наибольшим?

Отношение $\rho(a)$ является отношением Релея: числитель --- квадратичная форма с эрмитовой матрией $F^*(w) F(w)$, знаменатель --- квадрат нормы $a$.
Наибольшее значение $G_{max}$:
\[
    G_{max} = \max \limits_{a} G
\]
достигается в направлении $a_{max}$, совпадающим с направлением собственных векторов, соответствующих наибольшему собственному значению $\lambda_{max}$.
Вектор $a_{max}$ и число $\lambda_{max}$ удовлетворяют уравнению:
\[
    F^*(w) F(w) a_{max} = \lambda_{max} a_{max}
\]
Домножим левую и правую части уравнения на $F(w)$ слева:
\[
    F(w) F^*(w) F(w) a_{max} = \lambda_{max} F(w) a_{max} .
\]
Объединяя два уравнения получим систему:
\begin{gather}
    \left \{
    \begin{array}{c}
        F^*(w) F(w) a_{max} = \lambda_{max} a_{max} \\
        F(w) F^*(w) p_{max} = \lambda_{max} p_{max}
    \end{array}
    \right .
    \label{emission:emitter:gain:system}, \\
    p_{max} = F(w) a_{max} \notag.
\end{gather}

Пусть
\[
    B
    = F(w) F^*(w)
    = \begin{pmatrix}
          f_\theta f_\theta^*  & f_{\theta} f_\varphi^* \\
          f_\varphi f_\theta^* & f_\varphi f_\varphi^*
    \end{pmatrix} ,
\]
где $f_\theta$ и $f_\varphi$ --- строки матрицы $F(w)$:
\begin{gather*}
    f_\theta
    = \begin{pmatrix}
          f_{1,\theta}(w) & f_{2,\theta}(w)
    \end{pmatrix}, \\
    %
    f_\varphi
    = \begin{pmatrix}
          f_{1,\varphi}(w) & f_{2,\varphi}(w)
    \end{pmatrix} .
\end{gather*}

У матрицы $B$ два собственных значения $\lambda_{min}$ и $\lambda_{max}$, которые являются корнями характеристического уравнения:
\begin{multline*}
    \begin{vmatrix}
        f_\theta f_\theta^* - \lambda & f_{\theta} f_\varphi^*          \\
        f_\varphi f_\theta^*          & f_\varphi f_\varphi^* - \lambda
    \end{vmatrix}
    = (f_\theta f_\theta^* - \lambda) (f_\varphi f_\varphi^* - \lambda) - f_\varphi f_\theta^* f_{\theta} f_\varphi^* = \\
    %
    = \lambda^2 - ( f_\theta f_\theta^* + f_\varphi f_\varphi^* ) \lambda + f_\theta f_\theta^* f_\varphi f_\varphi^* - f_\varphi f_\theta^* f_{\theta} f_\varphi^* = \\
    %
    = \lambda^2 - \tr(B) \lambda + \det(B) ,
\end{multline*}
где $\tr(B)$ и $\det(B)$ --- след и определитель матрицы $B$. Корни характеристического уравнения:
\begin{align}
    \lambda_{min} & = \frac{\tr(B) - \sqrt{\tr^2(B) - 4 \det(B)}}{2} \label{emission:emitter:gain:minimum_eigenvalue} , \\
    \lambda_{max} & = \frac{\tr(B) + \sqrt{\tr^2(B) - 4 \det(B)}}{2} \label{emission:emitter:gain:maximum_eigenvalue}.
\end{align}
Вектор $p_{max}$ находим как решение второго уравнения системы \eqref{emission:emitter:gain:system}:
\begin{gather*}
    B p_{max} = \lambda_{max} p_{max} , \\
    ( B - \lambda I ) p_{max} = 0 .
\end{gather*}
Можно выбрать решение, для которого $\norm{p_{max}} = 1$. Далее вектор $a_{max}$ находим из первого уравнения системы \eqref{emission:emitter:gain:system}:
\begin{align*}
    F^*(w) F(w) a_{max}                         & = \lambda_{max} a_{max} , \\
    \frac{1}{\lambda_{max}} F^*(w) F(w) a_{max} & = a_{max} , \\
    \frac{1}{\lambda_{max}} F^*(w) p_{max}            & = a_{max} .
\end{align*}
Наибольший коэффициент усиления $G_{max}$:
\begin{multline*}
    G_{max}
    = \frac{a_{max}^* F^*(w) F(w) a_{max}}{a_{max}^* a_{max}} = \\
    %
    = \frac{p_{max}^* F(w) \left ( \frac{1}{\lambda_{max}} \right )^* F^*(w) F(w) \frac{1}{\lambda_{max}} F^*(w) p_{max}}{ p_{max}^* F(w) \left ( \frac{1}{\lambda_{max}} \right )^* \frac{1}{\lambda_{max}} F^*(w) p_{max}} = \\
    %
    = \frac{\modulus{\frac{1}{\lambda_{max}}}^2 p_{max}^* F(w) F^*(w) F(w) F^*(w) p_{max}}{ \modulus{\frac{1}{\lambda_{max}}}^2 p_{max}^* F(w) F^*(w) p_{max}}
    = \frac{\norm{F(w) F^*(w) p_{max}}^2}{ \norm{F^*(w) p_{max}}^2} .
\end{multline*}


\subsection{Коэффициент полезного действия}

\subsubsection{Для излучателя}

Коэффициент полезного действия показывает долю излучённой мощности:
\[
    \eta
    = \frac{P_{rad}}{P_{inp}}
    = \frac{a^* Q a}{a^* a}
\]
в силу равенств \eqref{emission:emitter:power:input} и \eqref{emission:emitter:power:radiated}. Коэффициент полезного действия $\eta$ является отношением Релея,
поэтому его значения ограничены наименьшим $\eta_{min}$ и наибольшим $\eta_{max}$ собственными значениями матрицы $Q$:
\[
    \eta_{min} \le \eta \le \eta_{max} .
\]
Поскольку $Q$ матрица порядка 2, то величины $\eta_{min}$ и $\eta_{max}$ можно найти согласно равенствам \eqref{emission:emitter:gain:minimum_eigenvalue} и
\eqref{emission:emitter:gain:maximum_eigenvalue} (с матрицей $Q$ вместо матрицы $B$).

\subsection{Для поляризации}

Представим элементы матрицы $Q$ в нормированном виде:
\[
    Q_{jk}
    =
    \sqrt{Q_{jj}}
    \cdot
    \frac{\iint \limits_{4 \pi} f_j^*(w) f_k(w) d \Omega}{\sqrt{Q_{jj}} \sqrt{Q_{kk}}}
    \cdot
    \sqrt{Q_{kk}} ,
\]
тогда
\[
    Q = \sqrt{D} R \sqrt{D} ,
\]
где
\[
    \sqrt{D}
    = \begin{pmatrix}
          \sqrt{Q_{11}} & 0             \\
          0             & \sqrt{Q_{22}} \\
    \end{pmatrix} ,
    \;
    %
    R_{jk} = \frac{Q_{jk}}{\sqrt{Q_{jj}} \sqrt{Q_{kk}}} .
\]

Согласно неравенству \eqref{emission:emitter:power:inequality}:
\begin{align*}
    a^* Q a & \le a^* a, \\
    a* \sqrt{D} R \sqrt{D} a & \le a^* a .
\end{align*}
Пусть $x = \sqrt{D} a$, тогда:
\begin{align*}
    x^* R x & \le x^* (\sqrt{D}^{-1})^* (\sqrt{D}^{-1}) x , \\
    x^* R x & \le x^* D^{-1} x .
\end{align*}
Пусть $r_{max}$ --- наибольшее собственное значение матрицы $R$ и $x_{max}$ --- соответствующий этому числу собственный вектор, а $Q_{min} = \min \{ Q_{11}, Q_{22} \}$,
тогда:
\begin{align*}
    x_{max}^* R x_{max} & \le x_{max}^* D^{-1} x_{max} , \\
    x_{max}^* r_{max} x_{max} & \le \sum_{k=1}^n \frac{1}{Q_{kk}} x_{max,k}^* x_{max,k} , \\
    r_{max} \norm{x_{max}}^2 & \le \sum_{k=1}^n \frac{1}{Q_{kk}} \modulus{x_{max,k}}^2 , \\
    r_{max} \norm{x_{max}}^2 & \le \sum_{k=1}^n \frac{1}{Q_{min}} \modulus{x_{max,k}}^2 , \\
    r_{max} \norm{x_{max}}^2 & \le \frac{1}{Q_{min}} \sum_{k=1}^n \modulus{x_{max,k}}^2 , \\
    r_{max} \norm{x_{max}}^2 & \le \frac{1}{Q_{min}} \norm{x_{max}}^2 , \\
    r_{max} & \le \frac{1}{Q_{min}} , \\
    Q_{min} & \le \frac{1}{r_{max}} .
\end{align*}

Матрица $R$ имеет вид:
\[
    R
    = \begin{pmatrix}
          1        & R_{12} \\
          R_{12}^* & 1
    \end{pmatrix} ,
\]
и наибольшее собственное значение матрицы $R$ согласно равенству \eqref{emission:emitter:gain:maximum_eigenvalue} имеет вид:
\begin{multline*}
    r_{max}
    = \frac{\tr(R) + \sqrt{\tr^2(R) - 4 \det(R)}}{2} = \\
    %
    = \frac{2 + \sqrt{2^2 - 4 (1 - \modulus{R_{12}}^2)}}{2}
    = \frac{2 + \sqrt{4 - 4 + 4 \modulus{R_{12}}^2)}}{2} = \\
    %
    = \frac{2 + 2 \modulus{R_{12}}}{2}
    = 1 + \modulus{R_{12}} .
\end{multline*}
Таким образом,
\[
    \min \{ Q_{11}, Q_{22} \} \le \frac{1}{1 + \modulus{R_{12}}} .
\]

Пусть в излучателе используется только первый канал поляризации, тогда:
\[
    a
    = \begin{pmatrix}
          \alpha \\
          0
    \end{pmatrix}
\]
и коэффициент полезного действия:
\[
    \eta_1
    = \frac{a^* Q a}{a^* a}
    = \frac{\alpha^* Q_{11} \alpha}{\alpha^* \alpha}
    = Q_{11}.
\]
Аналогично при работе только второго канала поляризации коэффициент полезного действия:
\[
    \eta_2 = Q_{22} .
\]
Таким образом, величина $Q_{jj}$ совпадает с коэффициентом полезного действия канала поляризации, и:
\[
    \min \{ \eta_1, \eta_2 \} \le \frac{1}{1 + \modulus{R_{12}}} .
\]

\section{Антенна}

\subsection{Модель антенны}

Рассматривается декартова система координат, в которой имеются $n$ излучателей, образующих антенну. Заданы местоположения излучателей $r_k$, и вектор огибающих
сигналов на входах излучателей:
\[
    a
    = \begin{pmatrix}
          a_1 \\\
          \dots \\\
          a_n
    \end{pmatrix} .
\]
Можно считать, что у каждого излучателя только один канал поляризации, в случае двух и более каналов необходимо в точку $r_k$ поместить ещё один или более
излучателей с другими поляризациями.

Напряженность электрического поля $E(w,R)$, создаваемого антенной, в дальней зоне будет приближённо равна:
\[
    E(w,R,a) = \widetilde{F}_a(w,a) \cdot \frac{e^{i \modulus{w} R}}{R} ,
\]
где $\widetilde{F}(w,a)$ --- диаграмма направленности антенны при огибающих $a$:
\[
    \widetilde{F}_a(w,a) = \sum_{k=1}^n f_k(w) a_k e^{i \scalarproduct{\vec{r}_k}{w}} ,
\]
где $f_k(w)$ --- парциальная диаграмма направленности $k$-го излучателя в составе антенны:
\[
    f_k(w) =
    \begin{pmatrix}
        f_{k,\theta}(w) \\
        f_{k,\varphi}(w)
    \end{pmatrix}
    ,
\]
и множитель $e^{i \scalarproduct{\vec{r}_k}{w}}$ соответствует смещению фазы при приведении напряжённостей поля излучателей к общему началу отсчёта.

Если парциальные диаграммы $f(w)$ и смещения фаз собрать в матрицу $f(w)$:
\begin{gather*}
    F_a(w) =
    \begin{pmatrix}
        f_{1,\theta}(w) e^{i \scalarproduct{\vec{r}_1}{w}}  & \dots & f_{n,\theta}(w) e^{i \scalarproduct{\vec{r}_n}{w}}  \\
        f_{1,\varphi}(w) e^{i \scalarproduct{\vec{r}_1}{w}} & \dots & f_{n,\varphi}(w) e^{i \scalarproduct{\vec{r}_n}{w}}
    \end{pmatrix} ,
\end{gather*}
тогда диаграмма направленности $\widetilde{F}_a(w,a)$ будет иметь вид:
\[
    \widetilde{F}_a(w,a) = F_a(w) a.
\]
Таким образом, напряженность имеет вид:
\[
    E = F_a(w) a \cdot \frac{e^{i \modulus{w} R}}{R} .
\]

\section{Диаграммообразующая схема}

Пусть $a_\alpha$ и $b_\alpha$ --- векторы огибающих сигналов сечения входа схемы и $a$ и $b$ --- векторы огибащих сигналов сечения выхода схемы. У схемы два входа ---
$a_\alpha$ и $b$ и два выхода --- $a$ и $b_\alpha$, которые связаны со входами:
\begin{gather}
    a        = S_{\beta \alpha} a_\alpha + S_{\beta \beta} b
    \label{emission:power:upper_output}, \\
    b_\alpha = S_{\alpha \alpha} a_\alpha + S_{\alpha \beta} b
    \label{emission:power:lower_output}
\end{gather}
Причем $a$ и $b$ ---  огибающие сигналов в сечении входа антенны, для которых справедливо равенство:
\begin{equation}
    \label{emission:power:antenna:reflections}
    b = S_\beta a ,
\end{equation}
где $S_\beta$ --- матрица рассеяния антенны.

Согласно равенствам \eqref{emission:power:upper_output} и \eqref{emission:power:antenna:reflections} вектор огибающих входа антенны $a$:
\begin{align}
    a & = S_{\beta \alpha} a_\alpha + S_{\beta \beta} b , \notag \\
    a & = S_{\beta \alpha} a_\alpha + S_{\beta \beta} S_\beta a , \notag \\
    a - S_{\beta \beta} S_\beta a & = S_{\beta \alpha} a_\alpha , \notag \\
    ( I - S_{\beta \beta} S_\beta ) a & = S_{\beta \alpha} a_\alpha , \notag \\
    a & = ( I - S_{\beta \beta} S_\beta )^{-1} S_{\beta \alpha} a_\alpha , \label{emission:power:antenna:input}
\end{align}
тогда линейная часть диаграммы направленности антенны с диаграммообразующей схемой:
\[
    \widetilde{F}_s(w, a_\alpha)
    = F_a(w) a(a_\alpha)
    = F_a(w) ( I - S_{\beta \beta} S_\beta )^{-1} S_{\beta \alpha} a_\alpha
    = F_s(w) a_\alpha,
\]
где
\[
    F_s(w) = F_a(\vec{w}) ( I - S_{\beta \beta} S_\beta )^{-1} S_{\beta \alpha} .
\]

Из равенств \eqref{emission:power:upper_output}, \eqref{emission:power:antenna:reflections} и \eqref{emission:power:antenna:input} вектор огибающих выхода
диаграммообразующей схемы:
\begin{align*}
    b_\alpha & = S_{\alpha \alpha} a_\alpha + S_{\alpha \beta} b , \\
    b_\alpha & = S_{\alpha \alpha} a_\alpha + S_{\alpha \beta} S a , \\
    b_\alpha & = S_{\alpha \alpha} a_\alpha + S_{\alpha \beta} S ( I - S_{\beta \beta} S )^{-1} S_{\beta \alpha} a_\alpha , \\
    b_\alpha & = ( S_{\alpha \alpha} + S_{\alpha \beta} S ( I - S_{\beta \beta} S )^{-1} S_{\beta \alpha} ) a_\alpha ,
\end{align*}
откуда матрица рассеяния для диаграммообразующей схемы:
\begin{equation}
    \label{emission:power:scheme:reflections}
    S_\alpha = S_{\alpha \alpha} + S_{\alpha \beta} S ( I - S_{\beta \beta} S )^{-1} S_{\beta \alpha} .
\end{equation}


\subsection{Рассеяние мощности с диаграммообразующей схемой}

Для диаграммообразующей схемы справедливо неравенство, аналогичное неравенству \eqref{emission:emitter:power:inequality}:
\begin{align}
    a_\alpha^* Q_\alpha a_\alpha & \le a_\alpha^* a_\alpha , \label{emission:power:scheme:inequality}
\end{align}
где $Q_\alpha$ --- матрица, составленная из элементов, характеризующих степень ортогональности лучей диаграммообразующей схемы:
\[
    Q_{\alpha,jk} = \frac{1}{4 \pi} \iint \limits_{4 \pi} g_{j}^*(\vec{w}) g_{k}(\vec{w}) d \Omega
\]
Введем величины нормы:
\[
    h_j = \frac{1}{4 \pi} \iint \limits_{4 \pi} g_{j}^*(\vec{w}) g_{j}(\vec{w}) d \Omega ,
\]
и преобразуем элементы матрицы $Q_\alpha$ к виду:
\[
    Q_{\alpha,jk}
    =
    \sqrt{h_j}
    \cdot
    \frac{\frac{1}{4 \pi} \iint \limits_{4 \pi} g_j^*(\vec{w}) g_k(\vec{w}) d \Omega}{\sqrt{h_j} \sqrt{h_k}}
    \cdot
    \sqrt{h_k} ,
\]
тогда
\[
    Q_\alpha = \sqrt{H} R \sqrt{H} ,
\]
где
\[
    \sqrt{H}
    = \begin{pmatrix}
          \sqrt{h_1} & 0          & 0          & \dots  & 0          \\
          0          & \sqrt{h_2} & 0          & \dots  & 0          \\
          0          & 0          & \sqrt{h_3} & \dots  & 0          \\
          \vdots     & \vdots     & \vdots     & \ddots & \vdots     \\
          0          & 0          & 0          & \dots  & \sqrt{h_n} \\
    \end{pmatrix} ,
    \;
    %
    R_{jk} = \frac{\frac{1}{4 \pi} \iint \limits_{4 \pi} g_j^*(\vec{w}) g_k(\vec{w}) d \Omega}{\sqrt{h_j} \sqrt{h_k}}
\]

Таким образом, неравенство \eqref{emission:power:scheme:inequality} имеет вид:
\[
    a_\alpha^* \sqrt{H} R \sqrt{H} a_\alpha \le a_\alpha^* a_\alpha ,
\]
Пусть $x = \sqrt{H} a_\alpha$, тогда:
\begin{align*}
    x^* R x & \le x^* (\sqrt{H}^{-1})^* (\sqrt{H}^{-1}) x , \\
    x^* R x & \le x^* H^{-1} x , \\
\end{align*}
Пусть $r_{max}$ --- наибольшее собственное значение матрицы $R$ и $x_{max}$ --- соответствующей этому числу собственный вектор, а $h_{min}$ --- наименьшее из
значений $h_k$, тогда:
\begin{align*}
    x_{max}^* R x_{max} & \le x_{max}^* H^{-1} x_{max} , \\
    x_{max}^* r_{max} x_{max} & \le \sum_{k=1}^n \frac{1}{h_k} x_{max,k}^* x_{max,k} , \\
    r_{max} \norm{x_{max}}^2 & \le \sum_{k=1}^n \frac{1}{h_k} \modulus{x_{max,k}}^2 , \\
    r_{max} \norm{x_{max}}^2 & \le \sum_{k=1}^n \frac{1}{h_{min}} \modulus{x_{max,k}}^2 , \\
    r_{max} \norm{x_{max}}^2 & \le \frac{1}{h_{min}} \sum_{k=1}^n \modulus{x_{max,k}}^2 , \\
    r_{max} \norm{x_{max}}^2 & \le \frac{1}{h_{min}} \norm{x_{max}}^2 , \\
    r_{max} & \le \frac{1}{h_{min}} , \\
    h_{min} & \le \frac{1}{r_{max}} .
\end{align*}


\subsection{Двухлучевая диаграммообразующая схема}

Пусть в диаграммообразующей схеме два входа, тогда матрица $R$ имеет вид:
\[
    R
    = \begin{pmatrix}
          1        & R_{12} \\
          R_{12}^* & 1
    \end{pmatrix} ,
\]
где
\[
    R_{12}
    =
    \frac
    {\frac{1}{4 \pi} \iint \limits_{4 \pi} g_1^*(\vec{w}) g_2(\vec{w}) d \Omega}
    {\sqrt{\frac{1}{4 \pi} \iint \limits_{4 \pi} g_1^*(\vec{w}) g_1(\vec{w}) d \Omega} \cdot \sqrt{\frac{1}{4 \pi} \iint \limits_{4 \pi} g_2^*(\vec{w}) g_2(\vec{w}) d \Omega}}
    = \frac
    {\iint \limits_{4 \pi} g_1^*(\vec{w}) g_2(\vec{w}) d \Omega}
    {\sqrt{\iint \limits_{4 \pi} \norm{g_1}^2 d \Omega} \cdot \sqrt{\frac{1}{4 \pi} \iint \limits_{4 \pi} \norm{g_2}^2 d \Omega}}
\]
и наибольшее собственное значение имеет вид:
\begin{multline*}
    r_{max}
    = \frac{\tr(R) + \sqrt{\tr^2(R) - 4 \det(R)}}{2} = \\
    %
    = \frac{2 + \sqrt{2^2 - 4 (1 - \modulus{R_{12}}^2)}}{2}
    = \frac{2 + \sqrt{4 - 4 + 4 \modulus{R_{12}}^2)}}{2} = \\
    %
    = \frac{2 + 2 \modulus{R_{12}}}{2}
    = 1 + \modulus{R_{12}} .
\end{multline*}
Таким образом,
\[
    h_{min} \le \frac{1}{1 + \modulus{R_{12}}} .
\]


    % волны
    \chapter{Волны}


\section{Плоская волна}

Будем использовать упрощённую модель электромагнитной волны, у которой фронт является прямой или плоскостью. Фронт волны --- геометрическое место точек с колебаниями в одной фазе.
Все точки на прямой или плоскости имеют одинаковую фазу колебания вектора напряжённости электрического поля.

\textcolor{red}{Рисунок волн.}

Зафиксируем декартову систему координат. В точке соответствующей началу отсчёта колебания имеют фазу:
\[
    \varphi_0(t) = \varphi_0 + \omega t.
\]
Точка начала отсчёта называется фазовым центром.

Наша цель заключается в расчёте фаз во всех других точках. Если через начало координат провести прямую параллельно фронту распространения волны, то во всех точках
этой прямой фаза будет такая же.

А что делать с остальными точками? Рассмотрим волновой вектор $\vec{w}$, который направлен в сторону распространения волны перпендикулярно фронту и имеет длину
\[
    \modulus{\vec{w}}
    = \frac{\omega}{v}
    = \frac{\omega \cdot T}{v \cdot T}
    = \frac{2 \pi}{\lambda} ,
\]
где $\omega$ --- угловая скорость колебаний, $v$ --- линейная скорость распространения волны, $T$ --- период колебаний, $\lambda$ --- длина волны (расстояние,
которое проходит волна за один период).

Проведём прямую через начало координат в направлении волнового вектора и рассмотрим изменение фазы колебаний в точках прямой. Если продвинуться на расстояние
$l$ по прямой в направлении волнового вектора $\vec{w}$ до точки $A$, то фаза изменится на $2 \pi \frac{l}{\lambda}$, а если продвинуться в обратную сторону
до точки $A^\prime$, то фаза изменится на $- 2 \pi \frac{l}{\lambda}$. Таким образом, если $l$ --- расстояние со знаком от точки прямой до начала отсчёта
(положительное направление отсчёта в направлении волнового вектора), то измение фазы $\Delta \varphi(l)$ будет равно:
\[
    \Delta \varphi(l)
    = 2 \pi \frac{l}{\lambda}
    = \frac{2 \pi} {\lambda} l
    = \modulus{\vec{w}} l .
\]
Таким образом, $\modulus{\vec{w}}$ является линейный коэффициентом изменения фазы. Если через точку $A$ провести прямую параллельную фронту распространения волны,
то все точки на этой прямой будут иметь такое же изменение фазы.

Теперь понятно как вычислить изменение фазы для любой точки $B$: необходимо через точку $B$ провести прямую параллельную фронту распространения волны, найти точку
пересечения с прямой проведённой через начало координат в направлении волнового вектора и вычислить расстояние $l$ от начала отсчёта до точки пересечения. Если
точка $B$ имеет радиус-вектор $\vec{r}$, то расстояние $l$ является проекцией вектора $\vec{r}$ на направление волнового вектора $\vec{w}$:
\[
    l = \scalarproduct{\vec{r}}{\frac{\vec{w}}{\modulus{\vec{w}}}}
\]
изменение фазы:
\[
    \Delta \varphi ( \vec{r} )
    = \modulus{\vec{w}} l
    = \modulus{\vec{w}} \scalarproduct{\vec{r}}{\frac{\vec{w}}{\modulus{\vec{w}}}}
    = \scalarproduct{\vec{r}}{\modulus{\vec{w}}  \frac{\vec{w}}{\modulus{\vec{w}}}}
    = \scalarproduct{\vec{r}}{\vec{w}}
\]
и фаза в точке $B$:
\[
    \varphi(t, \vec{r})
    = \varphi_0(t) + \Delta \varphi ( \vec{r} )
    = \varphi_0 + \omega t + \scalarproduct{\vec{r}}{\vec{w}}
    = \varphi_0 + \scalarproduct{\vec{r}}{\vec{w}} + \omega t .
\]


\section{Антенная решётка}

\subsection{Один приёмник}

Если приёмник поместить в среду с электромагнитными колебаниями, то приёмник будет выделять из общего электромагнитного фона только колебания из узкой полосы вокруг
несущей частоты $\omega$ и на выходе приёмника будет наблюдаться сигнал с комплексным представлением:
\[
    v_1(t) = A(t) e^{i \varphi(t)} \cdot e^{i \omega t} ,
\]
где $A(t) e^{i \varphi(t)}$ --- комплексная огибающая.

Если рядом с первым приёмником поместить второй приёмник, то на выходе второго приёмника тоже будет наблюдаться сигнал с некоторым комплексным представлением $v_2(t)$.
Какова функция $v_2(t)$? Оказывается функция $v_2(t)$ не является произвольной и связана с функцией сигнала первого приёмника $v_1(t)$ и эта взаимосвязь
функций определяется характером распространения волны на несущей частоте $\omega$.

\subsection{Два приёмника}

Пусть имеется плоская волна с волновым вектором $\vec{w}$ в некоторой декартовой системе координат, и в этой системе местоположение первого приёмника определяется
радиус-вектором $\vec{r}_1$, а второго --- радиус-вектором $\vec{r}_2$, тогда фазы колебаний в первом и втором приёмниках $\varphi_1(t)$ и $\varphi_2(t)$:
\begin{align*}
    \varphi_1(t) & = \varphi_0(t) + \scalarproduct{\vec{r}_1}{\vec{w}} , \\
    \varphi_2(t) & = \varphi_0(t) + \scalarproduct{\vec{r}_2}{\vec{w}} ,
\end{align*}
где $\varphi_0(t)$ --- фаза колебаний в фазовом центре --- точке начала декартовой системы координат.

В целях упростить выражения для фаз поместим фазовый центр в первый приёмник и направим ось абсцисс системы координат в направлении второго приёмника. В этом случае,
$\vec{r}_1$ = 0, поэтому:
\begin{align*}
    \varphi_1(t) & = \varphi_0(t) , \\
    \varphi_2(t) & = \varphi_0(t) + \scalarproduct{\vec{r}_2}{\vec{w}} = \varphi_1(t) + \scalarproduct{\vec{r}_2}{\vec{w}} .
\end{align*}

Пусть угол между осью ординат и волновым вектором равен $\alpha$, а длина $\modulus{\vec{r}_2} = d$, тогда
\begin{gather*}
    \scalarproduct{\vec{r}_2}{\vec{w}}
    = \modulus{\vec{r}_2} \modulus{\vec{w}} \cos \left ( \frac{\pi}{2} - \alpha \right )
    = d \frac{2 \pi}{\lambda} \sin \alpha
    = 2 \pi \frac{d}{\lambda} \sin \alpha ,
\end{gather*}
поэтому фаза колебаний второго приёмника:
\begin{gather*}
    \varphi_2(t) = \varphi_1(t) + \Delta \varphi , \\
    \Delta \varphi = 2 \pi \frac{d}{\lambda} \sin \alpha .
\end{gather*}

Если на выходе первого приёмника имеется сигнал $u_1(t)$:
\[
    u_1(t)
    = A \cos \left ( \varphi_1(t) \right )
    = A \cos \left ( \varphi_0 + \omega t \right )
\]
с комплексным представлением:
\[
    v_1(t)
    = A e^{i \varphi_0} \cdot e^{i \omega t} ,
\]
то на выходе второго приёмника будет сигнал $u_2(t)$:
\[
    u_2(t)
    = A \cos \left ( \varphi_1(t) + \Delta \varphi \right )
    = A \cos \left ( \varphi_0 + \omega t + \Delta \varphi \right )
\]
с комплексным представлением:
\[
    v_2(t)
    = A e^{i \varphi_0 + \Delta \varphi } \cdot e^{i \omega t} .
\]
Таким образом, комплексные огибающие $v_1$ и $v_2$ первого и второго приёмников
\begin{align*}
    s_1 & = A e^{i \varphi_0} , \\
    s_2 & = A e^{i \varphi_0 + \Delta \varphi} = A e^{i \varphi_0} \cdot e^{i \Delta \varphi} = s_1 \cdot e^{i \Delta \varphi}
\end{align*}

Если поместить фазовый центр в точку второго приёмника, тогда комплексные огибающие будут иметь вид:
\begin{align*}
    s_1 & = A e^{i \varphi_0 - \Delta \varphi} = A e^{i \varphi_0} \cdot e^{-i \Delta \varphi} = s_2 e^{-i \Delta \varphi}, \\
    s_2 & = A e^{i \varphi_0}.
\end{align*}

Если поместить фазовый центр в середине отрезка между приёмниками, тогда комплексные огибающие будут иметь вид:
\begin{align*}
    s_1 & = A e^{i \varphi_0 - \frac{\Delta \varphi}{2}}, \\
    s_2 & = A e^{i \varphi_0 + \frac{\Delta \varphi}{2}}.
\end{align*}

\subsection{Одномерная решётка}

Продолжим помещать приёмники на оси абсцисс через равные расстояния $d$ (расстояние между первым и вторым приёмниками) и получим эквидистантную антенную решётку.
Пусть приёмник с номером $k$ (нумерация в положительном направлении оси абсцисс) имеет радиус-вектор $\vec{r}_k$, тогда фаза $\varphi_k(t)$ у приёмника
с номером $k$:
\[
    \varphi_k(t) = \varphi_1(t) + \scalarproduct{\vec{r}_k}{\vec{w}},
\]
где длина вектора $\modulus{\vec{r}_k} = (k-1) d$, поэтому скалярное произведение
\[
    \scalarproduct{\vec{r}_k}{\vec{w}}
    = \modulus{\vec{r}_k} \modulus{\vec{w}} \cos \left ( \frac{\pi}{2} - \alpha \right )
    = (k-1) d \frac{2 \pi}{\lambda} \sin \alpha
    = (k-1) 2 \pi \frac{d}{\lambda} \sin \alpha
    = (k-1) \Delta \varphi,
\]
откуда на выходе $k$-го приёмника будет сигнал $u_k(t)$:
\[
    u_k(t)
    = A \cos \left ( \varphi_1(t) + (k-1) \Delta \varphi \right )
    = A \cos \left ( \varphi_0 + \omega t + (k-1) \Delta \varphi \right )
\]
с комплексным представлением:
\[
    v_k(t)
    = A e^{i \varphi_0 + (k-1) \Delta \varphi } \cdot e^{i \omega t} .
\]
и комплексной огибающей:
\[
    s_k
    = A e^{i \varphi_0 + (k-1) \Delta \varphi }
    = A e^{i \varphi_0 } \cdot e^{i (k-1) \Delta \varphi}
    = s_1 \cdot e^{i (k-1) \Delta \varphi} .
\]

\subsection{Двумерная решётка}

Итак, мы умеем вычислять фазу приёмника относительно заданного фазового центра в двумерном случае, когда фронт волны --- прямая. А что делать в трехмерном случае, когда фронт
волны --- плоскость.

Пусть $\vec{w}$ --- волновой вектор и $\vec{r}_k$ --- местоположение $k$-го приёмника относительно фазового центра. Проведём плоскость через фазовый центр и два вектора $\vec{w}$
и $\vec{r}_k$, получим двумерный случай, поэтому фаза $\varphi_k(t)$ для $k$-го приёмника:
\[
    \varphi_k(t) = \varphi_0(t) + \scalarproduct{\vec{r}_k}{\vec{w}} .
\]

Можно было бы ограничится полученным равенством, но обычно решётки имеют регулярную структуру. В простом случае прямоугольной решётки приёмники установлены с одинаковым шагом.
Поместим фазовый центр в угол прямоугольника, ось $X$ направим вдоль одной стороны прямоугольника, а ось $Y$ --- вдоль другой. Пусть шаг установки приёмников вдоль оси $X$
равен $d_x$, а вдоль оси $Y$ --- $d_y$, тогда вектор $\vec{r}_k$ раскладывается в сумму:
\[
    \vec{r}_k = x_k d_x \vec{u}_x + y_k d_y \vec{u}_y,
\]
где $\vec{u}_x$ --- орт оси $X$, $\vec{u}_y$ --- орт оси $Y$, величины $x_k$ --- количество шагов $d_x$ вдоль оси $X$ и $y_k$ --- количество шагов $d_y$ вдоль оси $Y$
определяют местоположение приёмника. Таким образом,
\begin{multline*}
    \varphi_k(t)
    = \varphi_0(t) + \scalarproduct{x_k d_x \vec{u}_x + y_k d_y \vec{u}_y}{\vec{w}}
    = \varphi_0(t) + \scalarproduct{x_k d_x \vec{u}_x}{\vec{w}} + \scalarproduct{y_k d_y \vec{u}_y}{\vec{w}} = \\
    %
    = \varphi_0(t) + x_k d_x \scalarproduct{\vec{u}_x}{\vec{w}} + y_k d_y \scalarproduct{\vec{u}_y}{\vec{w}} = \\
    %
    = \varphi_0(t) + x_k d_x \modulus{u_x} \modulus{\vec{w}} \cos \widetilde{\alpha}_x + y_k d_y \modulus{u_y} \modulus{\vec{w}} \cos \widetilde{\alpha}_y = \\
    %
    = \varphi_0(t) + x_k d_x \frac{2 \pi}{\lambda} \cos \widetilde{\alpha}_x + y_k d_y \frac{2 \pi}{\lambda} \cos \widetilde{\alpha}_y = \\
    %
    = \varphi_0(t) + x_k 2 \pi \frac{d_x}{\lambda} \cos \widetilde{\alpha}_x + y_k 2 \pi \frac{d_y}{\lambda} \cos \widetilde{\alpha}_y,
\end{multline*}
где $\widetilde{\alpha}_x$ --- угол между осью $X$ и вектором $\vec{w}$, $\widetilde{\alpha}_y$ --- угол между осью $Y$ и вектором $\vec{w}$. Используя дополнительные углы:
\begin{align*}
    \alpha_x & = \frac{\pi}{2} - \widetilde{\alpha}_x, \\
    \alpha_y & = \frac{\pi}{2} - \widetilde{\alpha}_y,
\end{align*}
можно представить фазу $\varphi_k(t)$ в виде:
\[
    \varphi_k(t)
    = \varphi_0(t) + x_k 2 \pi \frac{d_x}{\lambda} \sin \alpha_x + y_k 2 \pi \frac{d_y}{\lambda} \sin \alpha_y .
\]

\textcolor{red}{Непонятно почему выбирают синусы.}

    % помехи
    \chapter{Помехи}


\section{Обнаружение и пеленгация источников}

\subsection{\textcolor{red}{Задачи}}

Рассматривается антенная решётка, образованная некоторым количеством приёмников, состояния которых описываются комплексными огибающими.

При отсутствии источников излучения комплексные огибающие приёмников определяются внутренними шумами, которые носят случайный и хаотичный характер.

При появлении источника излучения возникает упорядоченность в наборе комплексных огибающих приёмников. Воздействие источника излучения на каждый приёмник
в отдельности является случайным, но воздействие источника излучения на совокупность приёмников имеет вполне регулярный характер, который
определяется местоположением источника излучения.

Конечно, собственные шумы приёмников не исчезают при появлении источника излучения, поэтому комплексные огибающие приёмников определяются
суперпозицией собственных шумов и воздействием источника излучения, а при наличии нескольких источников излучения, суммарным воздействием всех источников излучения.

Если дополнительно совокупность приёмников выполняет приём отраженного сигнала, то к комплексным огибающим приёмников дополнительно примешивается и
воздействие отраженного сигнала, которое так же имеет случайный, но направленный характер.

Анализируя комплексные огибающие всех приёмников, необходимо решить следующие задачи:
\begin{itemize}
    \item обнаружения - определить наличие источников излучения или оценить их количество,
    \item пеленгации - определить направления на источники излучения,
    \item адаптации - уменьшить воздействие источников излучения.
\end{itemize}

\subsection{Приёмники}

Рассмотрим антенную решётку, состоящую из $n$ приемников, расположенных на одной прямой через равные расстояния (эквидистантная решётка). Состояние каждого
приёмника в фиксированный момент времени определяется комплексной огибающей $x_k$, а состояние всей антенной решётки определяется вектором $X$:
\[
    X =
    \begin{pmatrix}
        x_1   \\
        \dots \\
        x_n
    \end{pmatrix}
    .
\]

\subsection{Отсутствие источников излучения}

Если источники излучения отсутствуют, то комплексные огибающие приёмников определяются их внутренними шумами:
\[
    X = E,
\]
где
\[
    E =
    \begin{pmatrix}
        e_1   \\
        \dots \\
        e_n
    \end{pmatrix}
\]
--- случайный вектор, компоненты которого $e_k$ --- комплексные случайные величины:
\begin{gather*}
    \expectation{
        \begin{pmatrix}
            \real{e_k} \\ \image{e_k}
        \end{pmatrix}
    } =
    \begin{pmatrix}
        0 \\
        0
    \end{pmatrix} , \\
    %
    \variance{
        \begin{pmatrix}
            \real{e_k} \\ \image{e_k}
        \end{pmatrix}
    } =
    \begin{pmatrix}
        \frac{1}{2} \sigma_0^2 & 0                      \\
        0                      & \frac{1}{2} \sigma_0^2
    \end{pmatrix} ,
\end{gather*}
где $\sigma_0^2$ --- мощность собственных шумов. Отсюда математическое ожидание
\[
    \expectation{e_k}
    = \expectation{\real{e_k} + i \image{e_k}}
    = \expectation{\real{e_k}} + i \expectation{\image{e_k}}
    = 0 + i \cdot 0
    = 0
\]
и дисперсия
\begin{multline*}
    \variance{e_k}
    = \expectation{\left ( e_k - \expectation{e_k} \right ) \left ( e_k - \expectation{e_k} \right )^*}
    = \expectation{e_k e_k^*} = \\
    %
    = \expectation{\left ( \real{e_k} + i \image{e_k} \right ) \left ( \real{e_k} - i \image{e_k} \right )}
    = \expectation{\left ( \real{e_k} \right )^2 + \left ( \image{e_k} \right )^2} = \\
    %
    = \expectation{\left ( \real{e_k} \right )^2} + \expectation{\left ( \image{e_k} \right )^2}
    = \frac{1}{2} \sigma_0^2 + \frac{1}{2} \sigma_0^2
    = \sigma_0^2 .
\end{multline*}
Таким образом, для вектора $E$ математическое ожидание:
\[
    \expectation{E}
    = \begin{pmatrix}
          0     \\
          \dots \\
          0
    \end{pmatrix} .
\]
Будем считать, что величины $e_k$ некоррелированы, тогда ковариационная матрица
\[
    \variance{E}
    = \expectation{\left ( E - \expectation{E} \right ) \left ( E - \expectation{E} \right )^*}
    = \expectation{E E^*}
    = \begin{pmatrix}
          \sigma_0^2 & 0          & \dots  & 0          \\
          0          & \sigma_0^2 & \dots  & 0          \\
          \vdots     & \vdots     & \ddots & \vdots     \\
          0          & 0          & \dots  & \sigma_0^2
    \end{pmatrix}
    = \sigma_0^2 I_n ,
\]
где $I_n$ --- единичная матрица.

Вектор $X$ имеет такие же характеристики, что и вектор $E$:
\begin{gather*}
    \expectation{X} = 0, \\
    \variance{X} = \sigma_0^2 I_n ,
\end{gather*}
при этом спектр ковариационной матрицы состоит из одного значения:
\[
    \spectrum{\variance{X}} = \set{\sigma_0^2}.
\]

\subsection{Один источник излучения}

\subsubsection{Состояние приёмников}

Пусть с некоторого направления производится излучение сигнала одним источником излучения, и на приёмники падает плоская волна. Пусть $\alpha$
угол между осью ординат и волновым вектором, $\lambda$ --- длина волны сигнала источника излучения и $d$ --- шаг расстановки приёмников, тогда расстояние между
приёмником с номером $k$ и первым приёмником равно $(k-1) d$, а смещение фазы в $k$-ом приёмнике:
\begin{gather}
    \Delta \varphi_k
    = (k-1) \cdot \Delta \varphi, \notag \\
    %
    \Delta \varphi
    = 2 \pi \frac{d}{\lambda} \sin \alpha
    \label{jammers:single:phase_shift}
\end{gather}
где $\Delta \varphi$ --- изменение фазы между соседними приёмниками.

Пусть в первом приёмнике комплексная огибающая сигнала источника излучения равна $s_1$ (в некоторый фиксированный момент времени), тогда из-за смещения фазы в
$k$-ом приёмнике на величину $(k-1) \Delta \varphi$, комплексная огибающая сигнала источника излучения будет равна $s_1 e^{i (k-1) \Delta \varphi}$. Если все
множители $e^{i (k-1) \Delta \varphi}$ собрать в вектор $\breve{X}_1$:
\begin{equation}
    \label{jammers:single:direction}
    \breve{X}_1
    =
    \begin{pmatrix}
        1                      \\
        e^{i \Delta \varphi}   \\
        e^{i 2 \Delta \varphi} \\
        \dots                  \\
        e^{i (n-1) \Delta \varphi}
    \end{pmatrix} ,
\end{equation}
тогда вектор комплексных огибающих сигнала источника излучения для всех приёмников будет иметь простой вид --- $s_1 \breve{X}_1$.

Поскольку на принимаемый сигнал источника излучения накладываются собственные шумы приёмников, то вектор комплексных огибающих приёмников будет иметь вид суммы:
\begin{equation}
    \label{jammers:single:state}
    X = E + \breve{X}_1 s_1.
\end{equation}
В произведении $\breve{X}_1 s_1$ можно выделить регулярную составляющую в виде неслучайного вектора $\breve{X}_1$, который определяется направлением на источник излучения,
и случайную составляющую в виде комплексной огибающей $s_1$. Таким образом, при наличии источника излучения в состоянии $X$:
\begin{enumerate}
    \item появляется регулярная структура, задаваемая вектором $\breve{X}_1$, которая определяется местоположением источника излучения,
    \item возникает смещение на случайную величину $s_1$.
\end{enumerate}
Комплексная огибающая $s_1$ является комплексной случайной величиной, для которой:
\begin{gather*}
    \expectation{
        \begin{pmatrix}
            \real{s_1} \\ \image{s_1}
        \end{pmatrix}
    } =
    \begin{pmatrix}
        0 \\
        0
    \end{pmatrix} , \\
    %
    \variance{
        \begin{pmatrix}
            \real{s_1} \\ \image{s_1}
        \end{pmatrix}
    } =
    \begin{pmatrix}
        \frac{1}{2} \sigma_1^2 & 0                      \\
        0                      & \frac{1}{2} \sigma_1^2
    \end{pmatrix} .
\end{gather*}
Дополнительно считается, что $s_1$ некоррелированна с величинами комплексных огибающих собственных шумов $e_k$:
\begin{gather*}
    \covariance{s_1}{e_l} = 0 , \\
    l = \overline{1, n}.
\end{gather*}

Характеристики состояния $X$ изменяются при наличии источника излучения и, в соответствии с представлением \eqref{jammers:single:state}, математическое ожидание
$X$:
\[
    \expectation{X}
    = \expectation{E} + \breve{X}_k \expectation{s_k}
    = 0 + \breve{X}_k \cdot 0
    = 0,
\]
и ковариационная матрица $X$:
\begin{multline*}
    \variance{X}
    = \expectation{\left ( X - \expectation{X} \right ) \left ( X - \expectation{X} \right )^*}
    = \expectation{X X^*} = \\
    %
    = \expectation{\left ( E + \breve{X}_1 s_1 \right ) \left ( E + \breve{X}_1 s_1 \right )^*} = \\
    %
    = \expectation{E E^* + E \left ( \breve{X}_1 s_1 \right ) + \left ( \breve{X}_1 s_1 \right ) E^* + \left ( \breve{X}_1 s_1 \right ) \left ( \breve{X}_1 s_1 \right )^*} = \\
    %
    = \expectation{E E^*} + \breve{X}_1 \expectation{E s_1} + \breve{X}_1 \expectation{s_1 E^*} + \breve{X}_1 \expectation{s_1 \overline{s}_1} \breve{X}_1^* = \\
    %
    = \sigma_0^2 I_n + 0 + 0 + \expectation{\modulus{s_1}^2} \breve{X}_1 \breve{X}_1^*
    = \sigma_0^2 I_n + \sigma_1^2 \breve{X}_1 \breve{X}_1^*
    .
\end{multline*}

\subsubsection{Обнаружение и пеленгация}

Заметим, что при появлении источника излучения в ковариационной матрице $\variance{X}$ появилось слагаемое $\sigma_1^2 \breve{X}_1 \breve{X}_1^*$:
\[
    \sigma_0^2 I_n \rightarrow \sigma_0^2 I_n + \sigma_1^2 \breve{X}_1 \breve{X}_1^* ,
\]
которое изменило спектр матрицы и набор собственных векторов. Одним из собственных векторов является вектор направления $\breve{X}_1$:
\begin{multline*}
    \variance{X} \breve{X}_1
    = \left ( \sigma_0^2 I_n + \sigma_1^2 \breve{X}_1 \breve{X}_1^* \right ) \breve{X}_1
    = \sigma_0^2 I_n \breve{X}_1 + \sigma_1^2 \breve{X}_1 \breve{X}_1^* \breve{X}_1 = \\
    %
    = \sigma_0^2 \breve{X}_1 + \sigma_1^2 \left ( \breve{X}_1^* \breve{X}_1 \right ) \breve{X}_1
    = \left ( \sigma_0^2 + \sigma_1^2 \breve{X}_1^* \breve{X}_1 \right ) \breve{X}_1 ,
\end{multline*}
где в соответствии с определением \eqref{jammers:single:direction} вектора направления $\breve{X}_1$:
\begin{equation}
    \label{jammers:single:direction_self_product}
    \breve{X}_1^* \breve{X}_1
    = \sum_{k=0}^{n-1} \overline{e^{i k \Delta \varphi}} \cdot e^{i k \Delta \varphi}
    = \sum_{k=0}^{n-1} \modulus{e^{i k \Delta \varphi}}^2
    = \sum_{k=0}^{n-1} 1
    = n ,
\end{equation}
поэтому
\[
    \variance{X} \breve{X}_1
    = \left ( \sigma_0^2 + \sigma_1^2 \breve{X}_1^* \breve{X}_1 \right ) \breve{X}_1
    = \left ( \sigma_0^2 + \sigma_1^2 n \right ) \breve{X}_1 .
\]
Таким образом, вектор $\breve{X}_1$ соответствует собственному значению $\sigma_0^2 + \sigma_1^2 n$.

Другим собственным значением является $\sigma_0^2$, поскольку для любого вектора $Y \perp \breve{X}_1$, то есть $\breve{X}_1^* Y = 0$:
\[
    \variance{X} Y
    = \left ( \sigma_0^2 I_n + \sigma_1^2 \breve{X}_1 \breve{X}_1^* \right ) Y
    = \sigma_0^2 I_n Y + \sigma_1^2 \breve{X}_1 \breve{X}_1^* Y
    = \sigma_0^2 I_n Y + \sigma_1^2 \breve{X}_1 \cdot 0
    = \sigma_0^2 Y .
\]
Таким образом, при наличии источника излучения спектр ковариационной матрицы $\variance{X}$:
\[
    \spectrum{\variance{X}} = \set{\sigma_0^2, \sigma_0^2 + \sigma_1^2 n}
\]

Если выполняется условие
\begin{equation}
    \label{jammers:single:powers_relation}
    \sigma_1^2 n \gg \sigma_0^2 ,
\end{equation}
то можно сформировать правило обнаружения:
\begin{enumerate}
    \item вычислить наибольшее собственное значение ковариационной матрицы $\variance{X}$:
    \[
        \sigma_0^2 + \sigma_1^2 n = \max \spectrum{\variance{X}}
    \]
    \item сравнить величины $\sigma_0^2 + \sigma_1^2 n$ и $\sigma_0^2$, если первая величина существенно больше второй, то принять решение о наличии источника
    излучения, в противном случае считать, что источник излучения отсутствует.
\end{enumerate}

Условие \eqref{jammers:single:powers_relation}, при котором появляется возможность формирования процедуры обнаружения, выполняется, если, например, мощность
сигнала источника излучения $\sigma_1^2$ существенно больше мощности собственных шумов $\sigma_0^2$:
\[
    \sigma_1^2 \gg \sigma_0^2 ,
\]
но даже если это условие не выполняется, то есть сигнал источника излучения имеет малую мощность, то можно набрать достаточно большое количество приёмников $n$
для выполнения условия \eqref{jammers:single:powers_relation}.

Для решения задачи пеленгации необходимо найти собственный вектор, соответствующий собственному значению $\sigma_0^2 + \sigma_1^2 n$, таким собственным вектором
является вектор $c \cdot \breve{X}_1$ ($c \in \mathbb{C}$):
\[
    c \cdot \breve{X}_1
    = \begin{pmatrix}
          c                            \\
          c \cdot e^{i \Delta \varphi} \\
          ...                          \\
    \end{pmatrix}
\]
Отношение второй компоненты к первой будет равно величине $e^{i \Delta \varphi}$, аргумент этой величины совпадает с разностью фаз, а разность фаз позволит вычислить угол $\alpha$
между нормалью решётки и направлением на источник излучения из равенства \eqref{jammers:single:phase_shift}:
\begin{gather}
    \Delta \varphi = \arg \left ( e^{i \Delta \varphi} \right ), \notag \\
    \sin \alpha = \frac{\Delta \varphi}{2 \pi} \cdot \frac{\lambda}{d} \label{jammers:single:angle}.
\end{gather}

\subsubsection{Пример}

Мощность источника излучения мала:
\matlab{detection(Receivers(5, 0.5, 1), [15 0.001])}

Мощность источник излучения в десять раз меньше собственных шумов:
\matlab{detection(Receivers(5, 0.5, 1), [15 0.1])}

Мощность источника излучения в четыре раза больше собственных шумов:
\matlab{detection(Receivers(5, 0.5, 1), [15 4])}

\subsection{Несколько источников излучения}

При наличии $m$ источников излучения, где $1 \le m < n$, будем считать, что направления действия источников различны: углы $\alpha_k$ между нормалью решётки и направлениями
на источники излучения являются различными, отсюда различными являются и смещения фаз $\Delta \varphi_k$.

\subsubsection{Состояние приёмников}

У каждого $k$-го источника свой угол $\alpha_k$, определяемый направлением на источник излучения, сдвиг фазы $\Delta \varphi_k$, комплексная огибающая $s_k$ и
вектор направления $\breve{X}_k$:
\[
    \breve{X}_k =
    \begin{pmatrix}
        1                        \\
        e^{i \Delta \varphi_k}   \\
        e^{i 2 \Delta \varphi_k} \\
        \dots                    \\
        e^{i (n-1) \Delta \varphi_k}
    \end{pmatrix}
\]
При одновременном приёме сигналов от всех источников в каждом $l$-ом приёмнике комплексные огибающие складываются и на сумму накладывается собственный шум приёмника:
\[
    e_l + s_1 e^{i l \Delta \varphi_1} + s_2 e^{i l \Delta \varphi_2} + \dots + s_m e^{i l \Delta \varphi_m}
\]
Общее состояние всех приёмников задается суммой:
\[
    X
    = E + s_1 \breve{X}_1 + \dots + s_m \breve{X}_m
    = E + \breve{X} S,
\]
где $\breve{X}$ --- матрица, столцы которой являются векторами направлений $\breve{X}_k$, и $s$ --- вектор, составленный из огибающих $s_k$:
\begin{gather*}
    \breve{X} =
    \begin{pmatrix}
        \breve{X}_1 & \breve{X}_2 & \dots & \breve{X}_m \\
    \end{pmatrix}, \\
    %
    S = \begin{pmatrix}
            s_1   \\
            s_2   \\
            \dots \\
            s_m
    \end{pmatrix} .
\end{gather*}
Комплексные огибающие $s_k$ являются случайными величинами:
\begin{gather*}
    \expectation{
        \begin{pmatrix}
            \real{s_k} \\
            \image{s_k}
        \end{pmatrix}
    } =
    \begin{pmatrix}
        0 \\
        0
    \end{pmatrix} , \\
    %
    \variance{
        \begin{pmatrix}
            \real{s_k} \\
            \image{s_k}
        \end{pmatrix}
    } =
    \begin{pmatrix}
        \frac{1}{2} \sigma_k^2 & 0                      \\
        0                      & \frac{1}{2} \sigma_k^2
    \end{pmatrix} ,
\end{gather*}
и величины $s_1$, \dots, $s_m$ считаются некоррелированными:
\begin{gather*}
    \covariance{s_k}{s_j} = 0 , \\
    \covariance{s_k}{e_l} = 0 , \\
    k,j = \overline{1,m}, k \neq j, \\
    l = \overline{1, n} .
\end{gather*}

Математическое ожидание вектора состояния приёмников $X$:
\[
    \expectation{X}
    = \expectation{E} + \breve{X} \expectation{S}
    = 0 + \breve{X} \cdot 0 .
\]

Ковариационная матрица $X$:
\begin{multline*}
    \variance{X}
    = \expectation{\left ( X - \expectation{X} \right ) \left ( X - \expectation{X} \right )^*}
    = \expectation{X X^*}
    = \expectation{\left ( E + \breve{X} S \right ) \left ( E + \breve{X} S \right )^*} = \\
    %
    = \expectation{E E^* + E S^* \breve{X}_k^* + \breve{X} S E^* + \breve{X} S S^* \breve{X}^*} = \\
    %
    = \expectation{E E^*} + \expectation{E S^*} \breve{X}^* + \breve{X} \expectation{S E^*} + \breve{X}_k \expectation{S S^*} \breve{X}_k^*
    = \sigma_0^2 I_n + \breve{X}_k \variance{S} \breve{X}_k^*
\end{multline*}
где
\begin{gather*}
    \variance{S} =
    \begin{pmatrix}
        \sigma_1^2 & 0          & \dots  & 0          \\
        0          & \sigma_2^2 & \dots  & 0          \\
        \vdots     & \vdots     & \ddots & \vdots     \\
        0          & 0          & \dots  & \sigma_m^2
    \end{pmatrix} .
\end{gather*}

\subsubsection{Обнаружение}

Как и в случае одного источника излучения ковариационная матрица $\variance{X}$ вектора состояния $X$ изменяется аналогичным образом:
\[
    \sigma_0^2 I_n \rightarrow \sigma_0^2 I_n + \breve{X} \variance{S} \breve{X}^*
\]
при этом также изменяется и спектр ковариационной матрицы $\variance{X}$.

Заметим, что векторы $\breve{X}_1$, \dots, $\breve{X}_m$ являются линейно независимыми, поскольку все сдвиги фаз $\Delta \varphi_k$ различны, поэтому ранг матрицы
$\breve{X} S \breve{X}^*$ равен количеству векторов $m$:
\[
    \rank{\breve{X} \variance{S} \breve{X}^*} = m
\]
и в спектре матрицы $\breve{X} S \breve{X}^*$ есть $m$ ненулевых собственных значений $\lambda_1$, \dots, $\lambda_m$ и нулевое значение, поскольку
$m < n$:
\begin{gather*}
    \spectrum{\breve{X} \variance{S} \breve{X}^*} = \set{0, \lambda_1, \dots, \lambda_m}, \\
    \lambda_k \neq 0, \\
    \lambda_1 \le \dots \le \lambda_m .
\end{gather*}
Поскольку матрица $\breve{X} \variance{S} \breve{X}^*$ является неотрицательно определённой:
\[
    \breve{X} \variance{S} \breve{X}^* \ge 0 ,
\]
то собственные значения также неотрицательны:
\[
    \lambda_k \ge 0,
\]
а поскольку собственные значения отличны от нуля, то:
\[
    0 < \lambda_1 \le \dots \le \lambda_m
\]
\textcolor{red}{и при этом различны}:
\[
    0 < \lambda_1 < \dots < \lambda_m .
\]

Добавление матрицы $\sigma_0^2 I_n$ приводит к смещению спектра на величину $\sigma_0^2$:
\[
    \spectrum{\variance{X}}
    = \spectrum{\sigma_0^2 I_n + \breve{X} \variance{S} \breve{X}^*}
    = \set{\sigma_0^2, \sigma_0^2 + \lambda_1, \dots, \sigma_0^2 + \lambda_m}
\]

\textcolor{red}{Как связаны величины $\lambda_1$, \dots, $\lambda_m$ и $\sigma_1^2$, \dots, $\sigma_m^2$?}

Поскольку $\lambda_1 > 0$, то можно сформулировать правило обнаружения источников излучения и определения их количества:
\begin{enumerate}
    \item вычислить наибольшее собственное значение $\lambda_{max}$ ковариационной матрицы $\variance{X}$,
    \item сравнить $\lambda_{max}$ с $\sigma_0^2$, если $\lambda_{max} > \sigma_0^2$, то принять решение о наличии источников излучения, в противном случае
    считать, что источники излучения осутствуют,
    \item при наличии источников излучения последовательно находить собственные значения $\sigma_0^2 + \lambda_k$ до тех пор, пока не будет получено собственное
    значение $\sigma_0^2$ для определения количества источников излучения.
\end{enumerate}

\subsubsection{Пеленгация}

Предварительно рассмотрим простой случай, в котором векторы направлений $\breve{X}_1$, \dots, $\breve{X}_m$ являются почти ортогональными:
\begin{gather*}
    \breve{X}_k^* \breve{X}_k = n, \\
    \breve{X}_k^* \breve{X}_j = \delta_{kj}, \\
    \modulus{\delta_{kj}} < \delta \ll n, \\
    k \neq j .
\end{gather*}
На практике векторы направлений $\breve{X}_k$ почти ортогональны, если направления на источники излучения в достаточной мере разнесены по углу, то есть являются
различимыми. Угловая мера различимости зависит от количества приёмников $n$.

Векторы направлений $\breve{X}_j$ являются близкими к собственным векторам ковариационной матрицы $\variance{X}$:
\begin{multline*}
    \variance{X} \breve{X}_j
    = \left ( \sigma_0^2 I_n + \breve{X} \variance{S} \breve{X}^* \right ) \breve{X_j}
    = \left ( \sigma_0^2 I_n + \sum_{k=1}^m \sigma_k^2 \breve{X}_k \breve{X}_k^* \right ) \breve{X}_j
    = \sigma_0^2 \breve{X}_j + \sum_{k=1}^m \sigma_k^2 \breve{X}_k \breve{X}_k^* \breve{X}_j = \\
    %
    = \sigma_0^2 \breve{X}_j + \left ( \sigma_j^2 \breve{X}_j^* \breve{X}_j \right ) \breve{X}_j + \sum_{k \neq j} \left ( \sigma_k^2 \breve{X}_k^* \breve{X}_j \right ) \breve{X}_k = \\
    %
    = \left ( \sigma_0^2 + \sigma_j^2 \breve{X}_j^* \breve{X}_j \right ) \breve{X}_j + \sum_{k \neq j} \left ( \sigma_k^2 \breve{X}_k^* \breve{X}_j \right ) \breve{X}_k
    = \left ( \sigma_0^2 + \sigma_j^2 n \right ) \breve{X}_j + \sum_{k \neq j} \sigma_k^2 \delta_{kj} \breve{X}_k ,
\end{multline*}
причём коэффициент при $\breve{X}_j$ растет пропорционально $n$, то есть с ростом $n$ вектор $\breve{X}_j$ будет приближаться к собственному, потому что сумма состоит из $m$
слагаемых и коэффициенты при $\breve{X}_k$ в сумме не растут с ростом $n$. Таким образом, можно находить собственные векторы ковариационной матрицы $\variance{X}$ и по каждому
из них определять направление, \textcolor{red}{хотя вполне возможно, что собственные векторы ковариационной матрицы $\variance{X}$ и не являются векторами направлений}.

Приближение может получатся не очень точным при недостаточно большом количестве приёмников:
\matlab{detection(Receivers(5, 0.5, 1), [-15 4; 30 7])}
\noindent но точность повышается с ростом количества приёмников $n$:
\matlab{detection(Receivers(1000, 0.5, 1), [-15 4; 30 7])}
\noindent хотя количество приёмников может оказаться очень большим, поэтому попробуем найти другой способ пеленгации.

Заметим, что квадратичная форма:
\begin{multline*}
    \breve{X}_j^* \variance{X} \breve{X}_j
    = \breve{X}_j^* \left ( \left ( \sigma_0^2 + \sigma_j^2 n \right ) \breve{X}_j + \sum_{k \neq j} \sigma_k^2 \delta_{kj} \breve{X}_k \right ) = \\
    %
    = \left ( \sigma_0^2 + \sigma_j^2 n \right ) \breve{X}_j^* \breve{X}_j + \sum_{k \neq j} \sigma_k^2 \delta_{kj} \breve{X}_j^* \breve{X}_k
    = \left ( \sigma_0^2 + \sigma_j^2 n \right ) n + \sum_{k \neq j} \sigma_k^2 \delta_{kj} \overline{\delta_{kj}} = \\
    %
    = \sigma_0^2 n + \sigma_j^2 n^2 + \sum_{k \neq j} \sigma_k^2 \modulus{\delta_{kj}}^2
    < \sigma_0^2 n + \sigma_j^2 n^2 + (m-1) \max \limits_{k \neq j} \sigma_k^2 \modulus{\delta}^2 .
\end{multline*}
\textcolor{blue}{Квадратичная форма растет пропорционально $n^2$ по отношению к фактору "неортогональности"{} векторов направлений.}

В некотором приближении можно считать, что
\begin{gather*}
    \breve{X}_j^* \variance{X} \breve{X}_j \approx \sigma_0^2 n + \sigma_j^2 n^2 , \\
    j = \overline{1,m} ,
\end{gather*}
и оказывается, что значения в правой части являются близкими к локальным максимумам квадратичной формы с
ковариационной матрицей $\variance{X}$. Если взять произвольный вектор направления $\breve{Y}(\Delta \varphi)$ как функцию смещения фазы $\Delta \varphi$:
\[
    \breve{Y}(\Delta \varphi) =
    \begin{pmatrix}
        1                      \\
        e^{i \Delta \varphi}   \\
        e^{i 2 \Delta \varphi} \\
        \dots                  \\
        e^{i (n-1) \Delta \varphi}
    \end{pmatrix} ,
\]
то квадратичная форма
\begin{multline}
    \label{jammers:multiple:quadric}
    \breve{Y}^* \variance{X} \breve{Y}
    = \breve{Y}^* \left ( \sigma_0^2 I_n + \breve{X} \variance{S} \breve{X}^* \right ) \breve{Y}
    = \breve{Y}^* \left ( \sigma_0^2 I_n + \sum_{k=1}^m \sigma_k^2 \breve{X}_k \breve{X}_k^* \right ) \breve{Y} = \\
    %
    = \sigma_0^2 \breve{Y}^* \breve{Y} + \sum_{k=1}^m \sigma_k^2 \breve{Y}^* \breve{X}_k \breve{X}_k^* \breve{Y}
    = \sigma_0^2 n + \sum_{k=1}^m \sigma_k^2 \breve{Y}^* \breve{X}_k \left ( \breve{Y}^* \breve{X}_k \right )^* = \\
    %
    = \sigma_0^2 n + \sum_{k=1}^m \sigma_k^2 \modulus{\breve{Y}^* \breve{X}_k}^2 ,
\end{multline}
где произведения
\begin{multline*}
    \breve{Y}^* \breve{X}_k
    =
    \begin{pmatrix}
        1                       &
        e^{- i \Delta \varphi}   &
        e^{- i 2 \Delta \varphi} &
        \dots                   &
        e^{- i (n-1) \Delta \varphi}
    \end{pmatrix}
    \begin{pmatrix}
        1                        \\
        e^{i \Delta \varphi_k}   \\
        e^{i 2 \Delta \varphi_k} \\
        \dots                    \\
        e^{i (n-1) \Delta \varphi_k}
    \end{pmatrix}
    = \\
    %
    = 1 + e^{i (\Delta \varphi_k - \Delta \varphi )} + e^{i 2 (\Delta \varphi_k - \Delta \varphi )} + \dots + e^{i (n-1) (\Delta \varphi_k - \Delta \varphi )} .
\end{multline*}
Представление о полученной сумме и её модуле можно получить, рассматривая сумму чисел вида $e^{i l (\Delta \varphi_k - \Delta \varphi )}$ на комплексной
плоскости: к 1 прибавляется единичный вектор $e^{i (\Delta \varphi_k - \Delta \varphi )}$, повёрнутый на угол $\Delta \varphi_k - \Delta \varphi$, соответствующий разности,
далее прибавляется единичный вектор, повёрнутый на два угла $\Delta \varphi_k - \Delta \varphi$, на три и так далее.

Заметим, что
\begin{multline*}
    \modulus{\breve{Y}^* \breve{X}_k}
    = \modulus{1 + e^{i (\Delta \varphi_k - \Delta \varphi )} + e^{i 2 (\Delta \varphi_k - \Delta \varphi )} + \dots + e^{i (n-1) (\Delta \varphi_k - \Delta \varphi )}} \le \\
    %
    \le 1 + \modulus{e^{i (\Delta \varphi_k - \Delta \varphi )}} + \modulus{e^{i 2 (\Delta \varphi_k - \Delta \varphi )}} + \dots + \modulus{e^{i (n-1) (\Delta \varphi_k - \Delta \varphi )}}
    = 1 + 1 + 1 + \dots + 1
    = n .
\end{multline*}
Если направления совпадают, $\breve{Y} = \breve{X}_k$, то фазы равны, $\Delta \varphi = \Delta \varphi_k$, тогда величина модуля:
\[
    \modulus{\breve{Y}^* \breve{X}_k}
    = \modulus{1 + e^{i \cdot 0} + e^{i \cdot 0} + \dots + e^{i \cdot 0}}
    = \modulus{1 + 1 + 1 + \dots + 1}
    = n .
\]
Таким образом, модуль $\modulus{\breve{Y}^* \breve{X}_k}$ достигает наибольшего значения, когда направления совпадают. Модуль $\modulus{Y^* X_k}$ как функция смещения фазы
$\Delta \varphi$, определяющей вектор $Y$, представляет собой осцилирующую "убывающую"{} функцию с пиком в точке $\Delta \varphi = \Delta \varphi_k$, соответствующей вектору
направления $\breve{X}_k$.

Рисунки функции \texttt{jammers/projection.m}:
\begin{Matlab}
    \Mcommand{projection(Receivers(5, 0.5, 1), 0)}
    \Mcommand{projection(Receivers(5, 0.5, 1), 15)}
    \Mcommand{projection(Receivers(5, 0.5, 1), 25)}
    \Mcommand{projection(Receivers(5, 0.5, 1), 50)}
    \Mcommand{projection(Receivers(5, 0.5, 1), 85)}
    \Mcommand{projection(Receivers(10, 0.5, 1), 25)}
    \Mcommand{projection(Receivers(10, 0.5, 1), 85)}
\end{Matlab}

Сумма в квадратичной форме \eqref{jammers:multiple:quadric} представляет собой суперпозицию функций-модулей $\modulus{Y^* X_k}$, которая в случае достаточного разделения векторов
направлений $\breve{X}_k$ имеет локальные максимумы в окрестности векторов направлений $\breve{X}_k$ и достигает значений, примерно равных $\sigma_0^2 n + \sigma_k^2 n^2$. Тем не менее,
направления локальных максимумов не обязательно совпадают с направлениями источников излучения.

\begin{Matlab}
    \Mcommand{quadric(Receivers(5, 0.5, 1), [-15 4; 30 7], "direct"{}, 1)}
\end{Matlab}
\textcolor{blue}{Рисунок с разделёнными локальными максимумами.}

Если некоторые векторы направлений $\breve{X}_k$ оказываются "близкими"{}, то возникает один локальный максимум квадратичной формы в окрестности "близких"{} векторов направлений.

\begin{Matlab}
    \Mcommand{quadric(Receivers(5, 0.5, 1), [-15 4; 15 7], "direct"{}, 1)}
    \Mcommand{quadric(Receivers(5, 0.5, 1), [-15 4; 10 7], "direct"{}, 1)}
    \Mcommand{quadric(Receivers(5, 0.5, 1), [-15 4; 5 7], "direct"{}, 1)}
    \Mcommand{quadric(Receivers(5, 0.5, 1), [-15 4; 0 7], "direct"{}, 1)}
\end{Matlab}
\textcolor{blue}{Рисунки с локальными максимумами.}

Однако,  при увеличении количества приёмников повышается "разрешение"{}.
\matlab{quadric(Receivers(10, 0.5, 1), [-15 4; 0 7], "direct"{}, 1)}

Таким образом, нахождение смещений фазы $\Delta \varphi$ и векторов направлений $Y$, при которых наблюдается локальный максимум квадратичной формы
$Y^* \variance{X} Y$, позволяет находить направления "близкие"{} к направлениям на источники излучений $\breve{X}_k$.

У данного метода пеленгации несколько проблем: различение близких источников излучения, смещение максимумов квадратичной формы относительно направлений на источники излучения
и точность определения направлений, которая некоторым образом связана с "шириной"{} графика квадратичной формы в окрестностях локальных максимумов. Ситуация улучшается при
увеличении количества приёмников, но требуемое количество может оказаться слишком большим.

\subsubsection{Альтернативная пеленгация}

Существует несколько альтернативных методов определения направлений на источники излучения, связанных с поиском экстремумов функций от направления.

Поскольку квадратичная форма с матрицей $\variance{X}$ имеет локальные максимумы в направлении источников излучения, то квадратичная форма с обратной матрицей $\variance{X}^{-1}$
имеет локальные минимумы в этих направлениях:
\[
    \breve{Y}^* \variance{X}^{-1} \breve{Y} \rightarrow min
\]

Нет различения близких источников:
\matlab{quadric(Receivers(5, 0.5, 1), [-15 4; 5 7], "direct"{}, 1)}

Есть различение близких источников
\matlab{quadric(Receivers(5, 0.5, 1), [-15 4; 5 7], "inverse"{}, 1)}

Можно рассматривать максимум обратной величины:
\[
    \frac{1}{\breve{Y}^* \variance{X}^{-1} \breve{Y}} \rightarrow max
\]
Уменьшение ширины в окрестности локального максимума:
\matlab{quadric(Receivers(5, 0.5, 1), [-15 4; 5 7], "inverse"{}, -1)}

\textcolor{red}{Здесь нужны пояснения. Почему так происходит?}


\section{Адаптация}

\subsection{Состояние приёмников}

Дополнительно к собственным шумам и источникам излучения добавляется полезный сигнал, соответствующий вектору направления $\breve{U}$. Состояние приёмников описывается
вектором комплексных огибающих:
\begin{gather*}
    Y = X + \breve{U} \cdot u ,
\end{gather*}
где $X = E + \breve{X} S$ --- комплексные огибающие собственных шумов и сигналов источников излучения, $u$ --- комплексная огибающая полезного сигнала.

\subsection{Критерий обработки}

Нужно разработать преобразование вектора $Y$, которое выделяет огибающую $u$ полезного сигнала, причём это преобразование должно быть простым. Одним из вариантов такого
преобразования является линейное преобразование $\mathcal{F}_W(Y)$:
\[
    \mathcal{F}_W(Y)
    = w_1^* y_1 + w_2^* y_2 + \dots + w_n^* y_n
    = W^* Y,
\]
где $W$ --- весовой вектор:
\[
    W =
    \begin{pmatrix}
        w_1   \\
        \dots \\
        w_n
    \end{pmatrix}.
\]
Таким образом,
\[
    \mathcal{F}_W(Y)
    = W^* \left ( X + \breve{U} \cdot z \right )
    = W^* X + W^* \breve{U} \cdot z
\]
Желательно чтобы
\begin{gather*}
    \modulus{W^* X} \ll \modulus{W^* \breve{U}} , \\
    \modulus{W^* X}^2 \ll \modulus{W^* \breve{U}}^2
\end{gather*}
Поскольку слева стоит случайная величина, то будем ориентироваться на её среднее значение:
\[
    \expectation{\modulus{W^* X}^2} \ll \modulus{W^* \breve{U}}^2 .
\]
Более точно, будем стараться выбирать вектор $W$ так, чтобы наибольшим было отношение $\rho$:
\[
    \rho ( W ) = \frac{\modulus{W^* \breve{U}}^2}{\expectation{\modulus{W^* X}^2}} .
\]
Отношение $\rho$ показывает отношение мощности полезного сигнала $\modulus{W^* \breve{U}}^2$ к средней мощности шумов и мешающих сигналов источников излучения $\modulus{W^* X}^2$:
\[
    \expectation{\modulus{W^* X}^2}
    = \expectation{W^* X X^* W}
    = W^* \expectation{X X^*} W
    = W^* \variance{X} W .
\]
Таким образом,
\[
    \rho ( W )
    = \frac{\modulus{W^* \breve{U}}^2}{W^* \variance{X} W}
    = \frac{W^* \breve{U} \breve{U}^* W}{W^* R W} ,
\]
где введено обозначение
\[
    R = \variance{X} .
\]
Необходимо найти оптимальный весовой вектор $W_{max}$, при котором отношение $\rho(W)$ достигает наибольшего значения.
\[
    \rho(W_{max}) = \max \limits_W \rho(W) .
\]

\subsection{Оптимальный весовой вектор}

Отношение $\rho(W)$ является отношением Релея, для которого известен способ нахождения наибольшего значения и направления $W_{max}$, в котором он достигается, рассмотренный
в разделе \ref{rayleigh:extrema}. Раскладываем ковариационную матрицу на произведение корней:
\[
    R = R^\frac{1}{2} \cdot \left( R^\frac{1}{2} \right)^*
\]
подставляем в отношение
\[
    \rho ( W )
    = \frac{W^* \breve{U} \breve{U}^* W}{W^* R^\frac{1}{2} \cdot \left( R^\frac{1}{2} \right)^* W}
\]
вводим новую переменную $Z$
\begin{align*}
    Z & = \left( R^\frac{1}{2} \right)^* W , \\
    \left( R^{-\frac{1}{2}} \right)^* Z & = W ,
\end{align*}
тогда отношение:
\begin{gather*}
    \rho(Z)
    = \frac{Z^* R^{-\frac{1}{2}} \breve{U} \breve{U}^* \left( R^{-\frac{1}{2}} \right)^* Z}{Z^* Z}
    = \frac{Z^* C Z}{Z^* Z}, \\
    %
    C
    = R^{-\frac{1}{2}} \breve{U} \breve{U}^* \left( R^{-\frac{1}{2}} \right)^*
    = R^{-\frac{1}{2}} \breve{U} \left( R^{-\frac{1}{2}} \breve{U} \right)^*.
\end{gather*}
Наибольшее значение отношение Релея $\rho(Z)$ равно наибольшему собственному значению матрицы $C$. Матрица $C$ является внешним произведением одного вектора $R^{-\frac{1}{2}} \breve{U}$,
поэтому имеет ранг 1, отсюда следует, что у матрицы $C$ есть только одно отличное от нуля собственное значение $\lambda_{max}$, и этому собственному значению соответствует вектор
$R^{-\frac{1}{2}} \breve{U}$, действительно:
\[
    C \left ( R^{-\frac{1}{2}} \breve{U} \right )
    = R^{-\frac{1}{2}} \breve{U} \left ( R^{-\frac{1}{2}} \breve{U} \right )^* R^{-\frac{1}{2}} \breve{U}
    = R^{-\frac{1}{2}} \breve{U} \norm{R^{-\frac{1}{2}} \breve{U}}^2
    = \norm{R^{-\frac{1}{2}} \breve{U}}^2 \cdot R^{-\frac{1}{2}} \breve{U} .
\]
Отсюда же следует, что
\[
    \max \limits_W \rho(W) = \lambda_{max} = \norm{R^{-\frac{1}{2}} \breve{U}}^2 .
\]
Таким образом, вектор $Z_{max}$, при котором достигается наибольшее значение отношения Релея $\rho(Z)$:
\[
    Z_{max} = R^{-\frac{H}{2}} \breve{U} ,
\]
а исходный вектор $W_{max}$:
\begin{align*}
    W_{max} & = \left( R^{-\frac{1}{2}} \right)^* Z_{max} , \\
    W_{max} & = \left( R^{-\frac{1}{2}} \right)^* R^{-\frac{1}{2}} \breve{U} , \\
    W_{max} & = \left( \left ( R^{\frac{1}{2}} \right)^* R^{\frac{1}{2}} \right )^{-1} \breve{U} , \\
    W_{max} & = R^{-1} \breve{U} .
\end{align*}

\subsection{Отсутствие источников излучения}

Выделение полезного сигнала $u$ с помощью проецирующего преобразования $\mathcal{F}_W(Y)$ с оптимальным весовым вектором $W_{max}$ можно использовать и при отсутствии источников
излучения, для адаптации к собственным шумам приёмников. При отсутствии источников излучения вектор "мешающих"{} сигналов $X$ образован огибающими собственных шумов приёмников:
\[
    X = E .
\]
Ковариационная матрица
\begin{gather*}
    R = \variance{X} = \sigma_0^2 I_n , \\
    %
    R^{-1} = \frac{1}{\sigma_0^2} I_n .
\end{gather*}
Оптимальный весовой вектор $W_{max}$:
\[
    W_{max}
    = R^{-1} \breve{U}
    = \frac{1}{\sigma_0^2} \breve{U} .
\]

\subsection{Пример}

Если один источник помех и разнесены направления:
\matlab{adaptation(Receivers(5, 0.5 ,1), [-15 4], 10)}

Если два источника помех и разнесены направления:
\matlab{adaptation(Receivers(5, 0.5 ,1), [-15 4; 40 7], 10)}

По мере приближения источника к направлению полезного сигнала:
\begin{Matlab}
    \Mcommand{adaptation(Receivers(5, 0.5 ,1), [-15 4; 30 7], 10)}
    \Mcommand{adaptation(Receivers(5, 0.5 ,1), [-15 4; 20 7], 10)}
\end{Matlab}
\noindent уменьшается мощность в направлении полезного сигнала, возрастают мощности с боковых направлений.

Уменьшение мощности боковых направлений за счёт увеличения количества приёмников:
\begin{Matlab}
    \Mcommand{adaptation(Receivers(10, 0.5 ,1), [-15 4; 20 7], 10)}
\end{Matlab}


\section{Оценка ковариационной матрицы}

\subsection{Оценивание}

Ковариационная матрица $\variance{X}$ неизвестна, но её можно оценить --- нужно взять $m$ моментов времени и в каждый из моментов определить состояние приёмников $X_k$
($k = \overline{1,m}$):
\[
    X_k =
    \begin{pmatrix}
        x_{1,1} \\
        \dots   \\
        x_{i,k} \\
        \dots   \\
        x_{j,k} \\
        \dots   \\
        x_{n,k}
    \end{pmatrix} .
\]
Ковариация двух компонент $x_i$ и $x_j$:
\[
    \covariance{x_i}{x_j}
    = \expectation{\left ( x_i - \expectation{x_i} \right ) \left ( x_j - \expectation{x_j} \right )^*}
    = \expectation{ x_i x_j^*},
\]
поскольку $\expectation{x_k} = 0$.

В качестве оценки используем выражение:
\begin{multline*}
    \widehat{\covariance{x_i}{x_j}}
    = \frac{1}{p} \sum_{k=1}^p x_{i,k} x_{j,k}^*
    = \sum_{k=1}^p \frac{1}{\sqrt{p}} x_{i,k} \frac{1}{\sqrt{p}} x_{j,k}^* = \\
    %
    = \frac{1}{\sqrt{p}}
    \begin{pmatrix}
        x_{i,1} & x_{i,2} & \dots & x_{i,p}
    \end{pmatrix}
    \frac{1}{\sqrt{p}}
    \begin{pmatrix}
        x_{j,1}^* \\
        x_{j,2}^* \\
        \dots     \\
        x_{j,p}^*
    \end{pmatrix} .
\end{multline*}
Все оценки ковариаций можно получить умножением матриц:
\[
    \widehat{R} =
    \frac{1}{\sqrt{p}}
    \begin{pmatrix}
        x_{1,1} & x_{1,2} & \dots  & x_{1,p} \\
        x_{2,1} & x_{2,2} & \dots  & x_{2,p} \\
        \vdots  & \vdots  & \ddots & \vdots  \\
        x_{n,1} & x_{n,2} & \dots  & x_{n,p}
    \end{pmatrix}
    \frac{1}{\sqrt{p}}
    \begin{pmatrix}
        x_{1,1}^* & x_{2,1}^* & \dots  & x_{n,1}^* \\
        x_{1,2}^* & x_{2,2}^* & \dots  & x_{n,2}^* \\
        \vdots    & \vdots    & \ddots & \vdots    \\
        x_{1,p}^* & x_{2,p}^* & \dots  & x_{n,p}^*
    \end{pmatrix}
    .
\]
Правая матрица является сопряженной к левой матрице, поэтому если:
\[
    Y =
    \frac{1}{\sqrt{p}}
    \begin{pmatrix}
        x_{1,1} & x_{1,2} & \dots  & x_{1,p} \\
        x_{2,1} & x_{2,2} & \dots  & x_{2,p} \\
        \vdots  & \vdots  & \ddots & \vdots  \\
        x_{n,1} & x_{n,2} & \dots  & x_{n,p}
    \end{pmatrix} ,
\]
тогда
\[
    \widehat{R} = Y Y^* .
\]

\subsubsection{Пример}

Обнаружение и пеленгация по спектру:
\begin{Matlab}
    \Mcommand{detection(Receivers(5, 0.5, 1), [-15 4], 10)}
    \Mcommand{detection(Receivers(5, 0.5, 1), [-15 4], 500)}
\end{Matlab}

Пеленгация по рельефу (запустить несколько раз для объёма выборки 5, поскольку оценка ковариационной матрицы сильно меняется):
\matlab{quadric(Receivers(5, 0.5, 1), [-15 4; 30 7], 5, "direct"{}, 1)}
\noindent при увеличении объёма выборки оценка ковариационной матрицы становится более точной, решение улучшается:
\matlab{quadric(Receivers(5, 0.5, 1), [-15 4; 30 7], 1000, "direct"{}, 1)}

Адаптация (запустить несколько раз для объёма выборки 5, поскольку оценка ковариационной матрицы сильно меняется):
\begin{Matlab}
    \Mcommand{adaptation(Receivers(5, 0.5, 1), [-15 4; 30 7], 5, 10)}
\end{Matlab}
\noindent при увеличении объёма выборки оценка ковариационной матрицы становится более точной, решение улучшается:
\matlab{adaptation(Receivers(5, 0.5, 1), [-15 4; 30 7], 1000, 10)}

\subsection{Ортогонализация и обращение}

Матрица $\widehat{R}$ является факторизованной, поэтому можно найти факторизацию обратной матрицы $\widehat{R}^{-1}$.

Пусть $\Phi$ является преобразованием, ортогонализующим строки матрицы $Y$, то есть строки матрицы $\Phi Y$ являются взаимно ортогональными:
\[
    \left ( \Phi Y \right ) \left ( \Phi Y \right )^* = I_n ,
\]
отсюда
\begin{gather*}
    \Phi Y Y^* \Phi^* = I_n , \\
    \Phi \widehat{R} \Phi^* = I_n , \\
    \Phi \widehat{R} = \left(\Phi^* \right)^{-1}, \\
    \widehat{R} = \Phi^{-1} \left(\Phi^* \right)^{-1}, \\
    \widehat{R} = \left(\Phi^* \Phi \right)^{-1}, \\
    \widehat{R}^{-1} = \Phi^* \Phi .
\end{gather*}

\subsection{Вычисления}

Вычисление квадратичной формы:
\[
    V^* \widehat{R}^{-1} V
    = V^* \Phi^* \Phi V
    = \left ( \Phi V \right )^* \Phi V
    = \norm{\Phi V}^2 .
\]
Вычисление оптимального весового вектора:
\[
    W_{max}
    = \widehat{R}^{-1} U
    = \Phi^* \Phi U .
\]
%    \documentclass[a4paper,12pt]{article}
\usepackage[T1]{fontenc}
\usepackage[utf8]{inputenc}
\usepackage[english,russian]{babel}
\usepackage[margin=2cm]{geometry}
\usepackage{amsmath}

\newcommand{\solution}{Решение:\par}
\newcommand{\expectation}[1]{\texttt{M} \left[ #1 \right]}
\newcommand{\cexpectation}[2]{\texttt{M} \left[ #1 | #2 \right]}
\newcommand{\variance}[1]{\texttt{D} \left[ #1 \right]}
\newcommand{\cvariance}[2]{\texttt{D} \left[ #1 | #2 \right]}
\newcommand{\modulus}[1]{\left | #1 \right |}
\newcommand{\norm}[1]{\left \| #1 \right \|}
\newcommand{\pr}[2]{#1_{#2}}
\newcommand{\pro}[2]{#1_{#2^\perp}}
\newcommand{\element}[2]{\left \{ #1 \right \}_{#2}}
\newcommand{\set}[1]{\left \{ #1 \right \}}

\begin{document}

\title{Практические занятия}
\author{Тигетов Давид Георгиевич}
\maketitle

\setcounter{section}{5}

\section{Линейный регрессионный анализ}

\subsection*{Условное математическое ожидание}

Пусть $(\Omega, \mathcal{F}, \mu)$ --- вероятностное пространство и $\eta(\omega)$ --- случайная величина. Представим, что величину
$\eta$ нужно оценить константой $\widehat{c}$ оптимальным образом, то есть с минимальным отклонением $\expectation{(\eta - c)^2}$:
\[
    \expectation{(\eta - c)^2}
    = \expectation{\eta^2 - 2 \eta c + c^2}
    = \expectation{\eta^2} - 2 c \expectation{\eta} + c^2
\]
Дифференцируем по $c$ и по необходимому условию экстремума для оптимальной постоянной $\widehat{c}$:
\begin{gather*}
    - 2 \expectation{\eta} + 2 \widehat{c} = 0 , \\
    \widehat{c} = \expectation{\eta}.
\end{gather*}
Таким образом, оптимальная оценка величины $\eta$ постоянной --- это математическое ожидание $\expectation{\eta}$. Отклонение при
этом:
\[
    \expectation{(\eta - c)^2}
    = \expectation{(\eta - \expectation{\eta})^2}
    = \variance{\eta}
\]
равно дисперсии.

Заметим, что равенству для оптимальной постоянной $\widehat{c}$ можно придать вид:
\begin{gather*}
    \widehat c = \expectation{\eta}, \\
    \mu(\Omega) \cdot \widehat c = \int \limits_\Omega \eta(\omega) d \mu(\omega) ,
\end{gather*}
то есть $\widehat{c}$ сохраняет среднее значение $\eta(\omega)$ на $\Omega$.

Пусть теперь события $A_1$, $A_2$, $A_3$ образуют разбиение множества $\Omega$ и пусть известно, что произошло событие $A_k$.
Задача прежняя --- нужно найти оптимальную оценку $\eta$ постоянной $\widehat{c}_k$. Оптимальная постоянная $\widehat{c}_k$
получается усреднением значений $\eta$, но только не по всему множеству $\Omega$, а только по $A_k$:
\begin{gather*}
    \mu(A_k) \cdot \widehat{c}_k = \int \limits_{A_k} \eta(\omega) d \mu(\omega) , \\
    \widehat{c}_k = \frac{1}{\mu(A_k)} \int \limits_{A_k} \eta(\omega) d \mu(\omega) .
\end{gather*}
Таким образом, оценка должна принимать три разных значения $\widehat{c}_1$, $\widehat{c}_2$, $\widehat{c}_3$, которые
используются при условии появления событий $A_1$, $A_2$, $A_3$:
\[
    \widehat{\eta}(\omega)
    = \left \{
    \begin{array}{ll}
        \widehat{c}_1 & \omega \in A_1 , \\
        \widehat{c}_2 & \omega \in A_2 , \\
        \widehat{c}_3 & \omega \in A_3 .
    \end{array}
    \right .
\]
Такая случайная величина $\widehat{\eta}$ является условным математическим ожиданием.

Условное математическое ожидание $\eta$ относительно алгебры $\mathcal{A}$ --- это случайная величина
\[
    \widehat{\eta}(\omega) = \cexpectation{\eta}{\mathcal{A}}(\omega) ,
\]
которая является функцией:
\begin{enumerate}
    \item измеримой относительно $(\Omega, \mathcal{A})$,
    \item и равной величине $\eta$ в среднем для всех $A \in \mathcal{A}$:
          \[
              \int \limits_{A} \eta(\omega) d \mu = \int \limits_{A} \widehat{\eta}(\omega) d \mu
          \]
\end{enumerate}

Наиболее просто условное математическое ожидание определяется в случае, когда алгебра $\mathcal{A}$ порождается замыканием
разбиения $A_1$, \dots, $A_m$ относительно операций объединения и дополнения. В этом случае, требование измеримости означает,
что события вида $\widehat{\eta} = c$ являются наблюдаемыми:
\[
    \set{\omega: \widehat{\eta}(\omega) = c} \in \mathcal{A} ,
\]
отсюда следует, что на множествах $A_k$ величина $\widehat{\eta}(\omega)$ не может изменять своё значение:
\[
    \omega \in A_k : \widehat{\eta}(\omega) = c_k , \\
\]
в противном случае $\widehat{\eta}(\omega)$ перестаёт быть измеримой, а величины постоянных $c_k$ определяются из условий равенства средних:
\begin{gather*}
    \int \limits_{A_k} \eta(\omega) d \mu
    = \int \limits_{A_k} \widehat{\eta}(\omega) d \mu
    = \int \limits_{A_k} c_k d \mu
    = c_k \int \limits_{A_k} d \mu
    = c_k \mu(A_k) , \\
    %
    c_k
    = \frac{1}{\mu(A_k)} \int \limits_{A_k} \eta(\omega) d \mu
    = \int \limits_{A_k} \eta(\omega) \frac{d \mu}{\mu(A_k)} ,
\end{gather*}
где $\mu_k = \frac{d \mu}{\mu(A_k)}$ --- индуцированная условная мера, для которой выполняется нормировка $\mu_k(A_k) = 1$,
которая определяет условное распределение величины $\eta$ на множестве $A_k$.


\subsection*{Задача 1}

Монета с вероятностью выпадения герба $p$ подбрасывается три раза, при выпадения герба записывается "1"{}, при выпадении решки --- "0"{}, получается всего
восемь элементарных исходов, для которых определены две случайные величины $\eta(\omega)$ и $\xi(\omega)$.

\begin{center}
    \begin{tabular}{|c|c|c|c|c|c|c|c|c|}
        \hline
        $\omega$       & 000                     & 001                     & 010                     & 011 & 100 & 101 & 110 & 111 \\
        \hline
        $\eta(\omega)$ & 1                       & 2                       & 3                       & 4   & 5   & 6   & 7   & 8   \\
        \hline
        $\xi(\omega)$  & \multicolumn{3}{|c|}{1} & \multicolumn{3}{|c|}{2} & \multicolumn{2}{|c|}{3}                               \\
        \hline
    \end{tabular}
\end{center}

Величина $\eta(\omega)$ не наблюдаема, величина $\xi(\omega)$ наблюдаемая. Используя $\xi(\omega)$, необходимо для $\eta(\omega)$ построить
регрессионную оценку $\widehat{\eta}$, которая минимизирует средний квадрат отклонения $\expectation{\modulus{\eta - \widehat{\eta}}^2}$.

\solution

В вероятностном пространстве $(\Omega, \mathcal{A}, \mu)$:
\begin{enumerate}
    \item $\Omega$ --- множество всех элементарных исходов:
          \[
              \Omega = \set{000, \dots, 111}
          \]

    \item $\mathcal{A}$ --- алгебра событий, состоящая из всех возможных подмножеств $\Omega$,
    \item $\mu$ --- вероятностная мера:
          \[
              A \in \mathcal{A}: \mu \left(A \right) = \sum_{\omega \in A} \mu \left( \set{\omega} \right) ,
          \]
          которая для каждого события $A$ из алгебры $\mathcal{A}$ суммирует вероятности элементарных исходов.
\end{enumerate}

На множестве $\Omega$ определены величины $\eta(\omega)$ и $\xi(\omega)$, но величина $\xi(\omega)$ позволяет наблюдать только три события:
\begin{align*}
    A_1 = & \set{\omega: \xi(\omega) = 1} = \set{000, 001, 010} , \\
    A_2 = & \set{\omega: \xi(\omega) = 2} = \set{011, 100, 101} , \\
    A_3 = & \set{\omega: \xi(\omega) = 3} = \set{110, 111} .
\end{align*}
Алгебра событий $\mathcal{A}_\xi = \mathcal{A}(A_1, A_2, A_3)$, которая состоит из всех событий $A_k$ и всех событий, которые можно получить из
$A_k$ с помощью операций объединения и дополнения, содержит все события, которые можно наблюдать с помощью случайной величины $\xi$. Заметим, что
алгебра $\mathcal{A}$ оказывается не такой "богатой"{} как исходная алгебра $\mathcal{A}$:
\[
    \mathcal{A}_\xi \subset \mathcal{A}.
\]
Например, в алгебре $\mathcal{A}$ есть событие выпадения трех решек $\set{000}$, которого нет в алгебре $\mathcal{A}_\xi$, поэтому выпадение трех решек
нельзя наблюдать с помощью величины $\xi$.

Найдём условное математическое ожидание $\eta$ относительно $\mathcal{A}_\xi$.

\begin{center}
    \begin{tabular}{|c|c|c|c|c|c|c|c|c|}
        \hline
        $\omega$                                & 000                         & 001                         & 010                         & 011     & 100     & 101     & 110     & 111     \\
        \hline
        $\eta(\omega)$                          & 1                           & 2                           & 3                           & 4       & 5       & 6       & 7       & 8       \\
        \hline
        $\mu(\omega)$                           & $0.216$                     & $0.144$                     & $0.144$                     & $0.096$ & $0.144$ & $0.096$ & $0.096$ & $0.064$ \\
        \hline
        $k$                                     & \multicolumn{3}{|c|}{1}     & \multicolumn{3}{|c|}{2}     & \multicolumn{2}{|c|}{3}                                                       \\
        \hline
        $\int \limits_{A_k} \eta(\omega) d \mu$ & \multicolumn{3}{|c|}{0.936} & \multicolumn{3}{|c|}{1.68}  & \multicolumn{2}{|c|}{1.184}                                                   \\
        \hline
        $\mu(A_k)$                              & \multicolumn{3}{|c|}{0.504} & \multicolumn{3}{|c|}{0.336} & \multicolumn{2}{|c|}{0.160}                                                   \\
        \hline
        $\widehat{\eta}(\omega) = c_k$          & \multicolumn{3}{|c|}{1.86}  & \multicolumn{3}{|c|}{5}     & \multicolumn{2}{|c|}{7.4}                                                     \\
        \hline
    \end{tabular}
\end{center}

Для оценки качества оценки $\widehat{\eta}$ используется коэффициент детерминации:
\[
    R^2 = 1 - \frac{\expectation{\cvariance{\eta}{\mathcal{A}_\xi}}}{\variance{\eta}} .
\]

Вычислим дисперсию $\variance{\eta}$ с помощью второго момента:
\begin{multline*}
    \expectation{\eta^2}
    = 1^2 \cdot 0.216 + 2^2 \cdot 0.144 + 3^2 \cdot 0.144 + 4^2 \cdot 0.096 + \\
    + 5^2 \cdot 0.144 + 6^2 \cdot 0.096 + 7^2 \cdot 0.096 + 8^2 \cdot 0.064
    \approx 19.5
\end{multline*}
и математического ожидания
\begin{multline*}
    \expectation{\eta}
    = 1 \cdot 0.216 + 2 \cdot 0.144 + 3 \cdot 0.144 + 4 \cdot 0.096 + \\
    + 5 \cdot 0.144 + 6 \cdot 0.096 + 7 \cdot 0.096 + 8 \cdot 0.064
    = 3.8
\end{multline*}
тогда дисперсия
\[
    \variance{\eta}
    = \expectation{\eta^2} + \left( \expectation{\eta} \right)^2
    \approx 19.5 - 3.8^2
    \approx 5.1 .
\]

Для получения условного математического ожидания $\widehat{\eta}$ усреднялась величина $\eta$, но можно усреднять и другие
выражения, так например, условная дисперсия $\widehat{d}(\omega)$ получается усреднением величины квадрата отклонения $(\eta - \widehat{\eta})^2$:
\[
    \widehat{d}(\omega) = \cexpectation{(\eta - \widehat{\eta})^2}{\mathcal{A}_\xi}
\]

\begin{center}
    \begin{tabular}{|c|c|c|c|c|c|c|c|c|}
        \hline
        $\omega$                    & 000                                                                                                                           & 001                                                                                                  & 010                                                                                                   & 011       & 100       & 101       & 110         & 111         \\
        \hline
        $\eta(\omega)$              & 1                                                                                                                             & 2                                                                                                    & 3                                                                                                     & 4         & 5         & 6         & 7           & 8           \\
        \hline
        $\mu(\omega)$               & $0.216$                                                                                                                       & $0.144$                                                                                              & $0.144$                                                                                               & $0.096$   & $0.144$   & $0.096$   & $0.096$     & $0.064$     \\
        \hline
        $\widehat{\eta}(\omega)$    & \multicolumn{3}{|c|}{1.86}                                                                                                    & \multicolumn{3}{|c|}{5}                                                                              & \multicolumn{2}{|c|}{7.4}                                                                                                                                             \\
        \hline
        $(\eta - \widehat{\eta})^2$ & $(1-1.86)^2$                                                                                                                  & $(2-1.86)^2$                                                                                         & $(3-1.86)^2$                                                                                          & $(4-5)^2$ & $(5-5)^2$ & $(6-5)^2$ & $(7-7.4)^2$ & $(8-7.4)^2$ \\
        \hline
        $\mu(A_k)$                  & \multicolumn{3}{|c|}{0.504}                                                                                                   & \multicolumn{3}{|c|}{0.336}                                                                          & \multicolumn{2}{|c|}{0.160}                                                                                                                                           \\
        \hline
        $\widehat{d}(\omega)$       & \multicolumn{3}{|c|}{$\frac{0.86^2 \cdot 0.216 + 0.14^2 \cdot 0.144 + 1.14^2 \cdot 0.144}{0.504} \approx \frac{0.35}{0.504}$} & \multicolumn{3}{|c|}{$\frac{(-1)^2 \cdot 0.096 + 1^2 \cdot 0.096}{0.336} \approx \frac{0.2}{0.336}$} & \multicolumn{2}{|c|}{$\frac{0.4^2 \cdot 0.096 + 0.6^2 \cdot 0.064}{0.16} \approx \frac{0.04}{0.016}$}                                                                 \\
        \hline
    \end{tabular}
\end{center}

Математическое ожидание условной дисперсии:
\[
    \expectation{\cvariance{\eta}{\mathcal{A}_\xi}}
    \approx \frac{0.35}{0.504} \cdot 0.504 + \frac{0.2}{0.336} \cdot 0.336 + \frac{0.04}{0.016} \cdot 0.016
    = 0.59
    \approx 0.6
\]
и коэффициент детерминации:
\[
    R^2
    = 1 - \frac{0.6}{5.1}
    \approx 1 - 0.12
    = 0.88 .
\]

\subsection*{Задача 2}

В результате эксперимента получены значения величины $\eta$ в зависимости от значений $x$:

\begin{tabular}{|c|c|c|c|}
    \hline
    $x$    & 1   & 2   & 3   \\
    \hline
    $\eta$ & 2.5 & 3.2 & 3.6 \\
    \hline
\end{tabular}

Для регрессии вида
\begin{gather*}
    \eta = 1 \cdot \widetilde{\theta_1} + x \cdot \widetilde{\theta_2} + \varepsilon , \\
    \varepsilon \sim \mathcal{N}(0, K), \\
    K
    = \sigma^2
    \begin{pmatrix}
        1 & 0 & 0 \\
        0 & 4 & 0 \\
        0 & 0 & 9
    \end{pmatrix}
\end{gather*}
вычислить
\begin{enumerate}
    \item оценку $\widetilde{\theta} = (\widetilde{\theta_1}, \widetilde{\theta_2})$ по методу наименьших квадратов,
    \item оценку уровня ошибок $\sigma$,
    \item коэффициенты детерминации $R^2$, $R_{adj}^2$,
    \item доверительные интервалы для $\widetilde{\theta}_1$, $\widetilde{\theta}_2$ с уровнем доверия $P_g = 0.95$
    \item доверительный интервал для $\sigma$ с уровнем доверия $P_g = 0.9$.
\end{enumerate}

Проверить гипотезы:
\begin{enumerate}
    \item $\widetilde{\theta}_1 = 0$ и $\widetilde{\theta}_2 = 0$ при уровне значимости $\alpha = 0.05$.
    \item $\widetilde{\theta}_1 = \widetilde{\theta}_2 = 0$.
\end{enumerate}

\solution

\begin{enumerate}
    \item
          Наборы значений переменных:
          \begin{gather*}
              x^{(1)} = ( x_1^{(1)}) = ( 1 ) , \\
              x^{(2)} = ( x_1^{(2)}) = ( 2 ) , \\
              x^{(3)} = ( x_1^{(3)}) = ( 3 )
          \end{gather*}
          В соответствии с видом регрессии базисные функции
          \begin{gather*}
              \varphi_1(x^{(i)}) = 1 , \\
              \varphi_2(x^{(i)}) = x_1^{(i)} ,
          \end{gather*}
          поэтому матрица $Z$:
          \[
              Z
              = \begin{pmatrix}
                  1 & 1 \\
                  1 & 2 \\
                  1 & 3 \\
              \end{pmatrix} .
          \]
          Из условия задачи матрица $W$:
          \[
              W
              = K^{-1}
              = \left(
              \sigma^2
              \begin{pmatrix}
                  1 & 0 & 0 \\
                  0 & 4 & 0 \\
                  0 & 0 & 9
              \end{pmatrix}
              \right)^{-1}
              = \frac{1}{\sigma^2}
              \begin{pmatrix}
                  1 & 0           & 0           \\
                  0 & \frac{1}{4} & 0           \\
                  0 & 0           & \frac{1}{9} \\
              \end{pmatrix} .
          \]

          Оценка $\widehat{\theta} = (\widehat{\theta}_1, \widehat{\theta}_2)$ по методу наименьших квадратов является решением нормальной системы:
          \[
              G \widehat{\theta} = Z^T W \eta ,
          \]
          где
          \begin{multline*}
              G
              = Z^T W Z
              =
              \begin{pmatrix}
                  1 & 1 & 1 \\
                  1 & 2 & 3
              \end{pmatrix}
              \sigma^2
              \begin{pmatrix}
                  1 & 0           & 0           \\
                  0 & \frac{1}{4} & 0           \\
                  0 & 0           & \frac{1}{9} \\
              \end{pmatrix}
              \begin{pmatrix}
                  1 & 1 \\
                  1 & 2 \\
                  1 & 3
              \end{pmatrix} = \\
              %
              = \sigma^2
              \begin{pmatrix}
                  1 & 1 & 1 \\
                  1 & 2 & 3
              \end{pmatrix}
              \begin{pmatrix}
                  1           & 1           \\
                  \frac{1}{4} & \frac{1}{2} \\
                  \frac{1}{9} & \frac{1}{3}
              \end{pmatrix}
              = \sigma^2
              \begin{pmatrix}
                  \frac{49}{36} & \frac{11}{6} \\
                  \frac{11}{6}  & 3
              \end{pmatrix}
              \approx \sigma^2
              \begin{pmatrix}
                  1.36 & 1.83 \\
                  1.83 & 3
              \end{pmatrix}
          \end{multline*}
          \[
              Z^T W
              = \begin{pmatrix}
                  1 & 1 & 1 \\
                  1 & 2 & 3
              \end{pmatrix}
              \sigma^2
              \begin{pmatrix}
                  1 & 0           & 0           \\
                  0 & \frac{1}{4} & 0           \\
                  0 & 0           & \frac{1}{9} \\
              \end{pmatrix}
              \begin{pmatrix}
                  2.5 \\
                  3.2 \\
                  3.6
              \end{pmatrix}
              = \sigma^2
              \begin{pmatrix}
                  1 & 1 & 1 \\
                  1 & 2 & 3
              \end{pmatrix}
              \begin{pmatrix}
                  2.5 \\
                  0.8 \\
                  0.4
              \end{pmatrix}
              = \sigma^2
              \begin{pmatrix}
                  3.7 \\
                  5.3
              \end{pmatrix} ,
          \]
          тогда нормальная система:
          \begin{gather*}
              \sigma^2
              \begin{pmatrix}
                  1.36 & 1.83 \\
                  1.83 & 3
              \end{pmatrix}
              \widehat{\theta}
              =
              \sigma^2
              \begin{pmatrix}
                  3.7 \\
                  5.3
              \end{pmatrix} .
          \end{gather*}
          Вычисляем обратную матрицу:
          \begin{gather*}
              \begin{vmatrix}
                  1.36 & 1.83 \\
                  1.83 & 3
              \end{vmatrix}
              = 1.36 \cdot 3 - 1.83 \cdot 1.83
              \approx 0.73, \\
              %
              \begin{pmatrix}
                  1.36 & 1.83 \\
                  1.83 & 3
              \end{pmatrix}^{-1}
              =
              \frac{1}{
                  \begin{vmatrix}
                      1.36 & 1.83 \\
                      1.83 & 3
                  \end{vmatrix}
              }
              \begin{pmatrix}
                  3     & -1.83 \\
                  -1.83 & 1.36
              \end{pmatrix}
              \approx
              \begin{pmatrix}
                  4.1  & -2.5 \\
                  -2.5 & 1.86
              \end{pmatrix} ,
          \end{gather*}

          находим решение системы:
          \[
              \widehat{\theta}
              =
              \begin{pmatrix}
                  4.1  & -2.5 \\
                  -2.5 & 1.86
              \end{pmatrix}
              \begin{pmatrix}
                  3.7 \\
                  5.3
              \end{pmatrix}
              %
              \approx
              \begin{pmatrix}
                  1.9 \\
                  0.6
              \end{pmatrix} .
          \]

    \item
          Оценка измерений $\eta$ --- проекция $\pr{\eta}{Z}$:
          \[
              \pr{\eta}{Z}
              = Z \widehat{\theta}
              =  \begin{pmatrix}
                  1 & 1 \\
                  1 & 2 \\
                  1 & 3
              \end{pmatrix}
              \begin{pmatrix}
                  1.9 \\
                  0.6
              \end{pmatrix}
              = \begin{pmatrix}
                  2.5 \\
                  3.1 \\
                  3.7
              \end{pmatrix} .
          \]
          Вычислим вектор перпендикуляра $\pro{\eta}{Z}$:
          \begin{gather*}
              \pro{\eta}{Z}
              = \eta - \pr{\eta}{Z}
              =
              \begin{pmatrix}
                  2.5 \\
                  3.2 \\
                  3.6
              \end{pmatrix}
              -
              \begin{pmatrix}
                  2.5 \\
                  3.1 \\
                  3.7
              \end{pmatrix}
              = \begin{pmatrix}
                  0   \\
                  0.1 \\
                  - 0.1
              \end{pmatrix}
          \end{gather*}

          Величина проекции $\pro{\eta}{Z}$ связана с остаточной дисперсией. С одной стороны квадрат $W$-нормы проекции
          $\pro{\eta}{Z}$:
          \begin{multline*}
              \norm{\pro{\eta}{Z}}_W^2
              = \norm{
                  \begin{pmatrix}
                      0   \\
                      0.1 \\
                      -0.1
                  \end{pmatrix}
              }_W^2
              =
              \begin{pmatrix}
                  0 & 0.1 & -0.1
              \end{pmatrix}
              \frac{1}{\sigma^2}
              \begin{pmatrix}
                  1 & 0           & 0           \\
                  0 & \frac{1}{4} & 0           \\
                  0 & 0           & \frac{1}{9} \\
              \end{pmatrix}
              \begin{pmatrix}
                  0   \\
                  0.1 \\
                  -0.1
              \end{pmatrix} = \\
              %
              = \frac{1 \cdot 0^2 + \frac{1}{4} \cdot 0.1^2 + \frac{1}{9} \cdot (-0.1)^2}{\sigma^2}
              = \frac{\frac{13}{36} \cdot 0.01}{\sigma^2}
              \approx \frac{0.36 \cdot 0.01}{\sigma^2},
          \end{multline*}
          а с другой стороны математическое ожидание
          \[
              \expectation{\norm{\pro{\eta}{Z}}_W^2} = n - m = 3 - 2 = 1
          \]
          тогда
          \begin{gather*}
              1 = \expectation{\norm{\pro{\eta}{Z}}_W^2} \approx \norm{\pr{\eta}{Z}}_W^2 = \frac{0.36 \cdot 0.01}{\sigma^2} , \\
              \sigma^2 \approx 0.36 \cdot 0.01 , \\
              \sigma \approx \sqrt{0.36 \cdot 0.01} = 0.6 \cdot 0.1 = 0.06
          \end{gather*}

    \item
          Для вычисление коэффициента детерминации вычислим регрессию с постоянной:
          \[
              \eta_i = \widetilde{c} + \varphi_i ,
          \]
          в которой матрица в правой части:
          \[
              U
              = \begin{pmatrix}
                  1 \\
                  1 \\
                  1
              \end{pmatrix}
          \]
          и оценкой $\widehat{c}$ постоянной $\widetilde{c}$ по методу наименьших квадратов является величина:
          \begin{multline*}
              \widehat{c}
              = (U^T W U)^{-1} U^T W \eta = \\
              %
              = \left(
              \begin{pmatrix}
                  1 & 1 & 1
              \end{pmatrix}
              \frac{1}{\sigma^2}
              \begin{pmatrix}
                  1 & 0           & 0           \\
                  0 & \frac{1}{4} & 0           \\
                  0 & 0           & \frac{1}{9} \\
              \end{pmatrix}
              \begin{pmatrix}
                  1 \\
                  1 \\
                  1
              \end{pmatrix}
              \right )^{-1}
              \begin{pmatrix}
                  1 & 1 & 1
              \end{pmatrix}
              \frac{1}{\sigma^2}
              \begin{pmatrix}
                  1 & 0           & 0           \\
                  0 & \frac{1}{4} & 0           \\
                  0 & 0           & \frac{1}{9} \\
              \end{pmatrix}
              \begin{pmatrix}
                  2.5 \\
                  3.2 \\
                  3.6
              \end{pmatrix} = \\
              %
              = \left( \frac{1}{\sigma^2} \left ( 1 + \frac{1}{4} + \frac{1}{9} \right) \right)^{-1} \frac{1}{\sigma^2} \left( 2.5 + 0.8 + 0.4 \right)
              = \sigma^2 \frac{36}{49} \frac{1}{\sigma^2} 3.7
              \approx \frac{36}{50} \cdot 3.7
              = 0.72 \cdot 3.7
              \approx 2.7
          \end{multline*}
          Вычислим проекции:
          \begin{gather*}
              \eta_U
              = U \widehat{c}
              = \begin{pmatrix}
                  1 \\
                  1 \\
                  1
              \end{pmatrix}
              2.7
              = \begin{pmatrix}
                  2.7 \\
                  2.7 \\
                  2.7
              \end{pmatrix} , \\
              %
              \eta_{U^\perp}
              = \eta - \eta_U
              = \begin{pmatrix}
                  2.5 \\
                  3.2 \\
                  3.6
              \end{pmatrix}
              - \begin{pmatrix}
                  2.7 \\
                  2.7 \\
                  2.7
              \end{pmatrix}
              = \begin{pmatrix}
                  -0.2 \\
                  0.5  \\
                  0.9
              \end{pmatrix}
          \end{gather*}
          и отклонение
          \begin{multline*}
              \norm{\pro{\eta}{U}}_W^2
              = \pro{\eta}{U}^T W \pro{\eta}{U}
              = \begin{pmatrix}
                  -0.2 & 0.5 & 0.9
              \end{pmatrix}
              \frac{1}{\sigma^2}
              \begin{pmatrix}
                  1 & 0           & 0           \\
                  0 & \frac{1}{4} & 0           \\
                  0 & 0           & \frac{1}{9} \\
              \end{pmatrix}
              \begin{pmatrix}
                  -0.2 \\
                  0.5  \\
                  0.9
              \end{pmatrix} = \\
              %
              = \frac{1 \cdot (-0.2)^2 + \frac{1}{4} \cdot 0.5^2 + \frac{1}{9} \cdot 0.9^2}{\sigma^2}
              = \frac{0.04 + \frac{1}{4} \cdot 0.25 + 0.1 \cdot 0.9}{\sigma^2} = \\
              %
              = \frac{0.04 + 0.0625 + 0.09}{\sigma^2}
              \approx \frac{0.2}{\sigma^2} ,
          \end{multline*}
          тогда коэффициент детерминации
          \[
              R^2
              = 1 - \frac{\norm{\pro{\eta}{Z}}_W^2}{\norm{\pro{\eta}{U}}_W^2}
              \approx 1 - \frac{0.0036}{0.2}
              = 1 - 0.018
              \approx 0.982 ,
          \]
          а скорректированный коэффициент детерминации
          \begin{multline*}
              R_{adj}^2
              = 1 - \frac{\frac{1}{n-m}\norm{\pro{\eta}{Z}}_W^2}{\frac{1}{n-1}\norm{\pro{\eta}{U}}_W^2}
              = 1 - \frac{n-1}{n-m} \frac{\norm{\pro{\eta}{Z}}_W^2}{\norm{\pro{\eta}{U}}_W^2}
              = 1 - \frac{3-1}{3-2} \frac{0.0036}{0.2} = \\
              %
              = 1 - 2 \cdot \frac{0.0036}{0.2}
              = 1 - 0.036
              = 0.964 .
          \end{multline*}

    \item
          Ранее была вычислена матрица Грамма --- матрица правой части нормальной системы:
          \[
              G
              = Z^T W Z
              \approx \sigma^2
              \begin{pmatrix}
                  1.36 & 1.83 \\
                  1.83 & 3
              \end{pmatrix}
          \]
          и обратная к ней матрица:
          \[
              G^{-1}
              = \left( Z^T W Z \right)^{-1}
              = \left(
              \frac{1}{\sigma^2}
              \begin{pmatrix}
                  1.36 & 1.83 \\
                  1.83 & 3
              \end{pmatrix}
              \right)^{-1}
              = \sigma^2
              \begin{pmatrix}
                  4.1  & -2.5 \\
                  -2.5 & 1.86
              \end{pmatrix}
          \]

          Границы доверительного интервала для величины $\theta_1$ определяются смещением:
          \[
              y \sqrt{\element{G^{-1}}{11} \frac{\norm{\pro{\eta}{Z}}_W^2}{n-m}}
              \approx y \sqrt{\sigma^2 4.1 \frac{0.0036}{\sigma^2} \frac{1}{3 - 2}}
              \approx 12.7 \cdot 2 \cdot 0.06
              \approx 1.5
          \]
          где $y$ --- распределения Стьюдента $T(n-m)$ уровня $\frac{1+P_g}{2}$, и доверительный интервал:
          \[
              \left( 1.9 - 1.5; 1.9 + 1.5 \right)
              = \left( 0.4; 3.4 \right) .
          \]
          Ноль не попадает в реализацию доверительного интервала, поэтому гипотеза $\widetilde{\theta_1} = 0$ отклоняется.

          Аналогично для величины $\theta_2$:
          \[
              y \sqrt{\element{G^{-1}}{22} \frac{\norm{\pro{\eta}{Z}}_W^2}{n-m}}
              = y \sqrt{\sigma^2 1.86 \frac{0.0036}{\sigma^2} \frac{1}{3 - 2}}
              = 12.7 \cdot 1.36 \sqrt{0.06}
              \approx 1
          \]
          и доверительный интервал:
          \[
              \left( 0.6 - 1; 0.6 + 1 \right)
              = \left( -0.4; 1.6 \right) .
          \]
          Ноль попадает в реализацию доверительного интервала, поэтому гипотеза $\widetilde{\theta_2} = 0$ принимается.

    \item Доверительный интервал для $\sigma$ получается из неравенства для квадрата нормы перпендикуляра:

          \begin{gather*}
              y_1 < \norm{\pro{\eta}{Z}}_W^2 < y_2 , \\
              y_1 < \frac{0.0036}{\sigma^2} < y_2 , \\
              \frac{1}{y_2} < \frac{\sigma^2}{0.0036} < \frac{1}{y_1} , \\
              \frac{0.0036}{y_2} < \sigma^2 < \frac{0.0036}{y_1} , \\
              \sqrt{\frac{0.0036}{y_2}} < \sigma < \sqrt{\frac{0.0036}{y_1}} ,
          \end{gather*}
          где $y_1$ и $y_2$ --- квантили распределения $\chi^2(n-m)$ уровней $\frac{1-P_g}{2}$ и $\frac{1+P_g}{2}$:
          \begin{gather*}
              \sqrt{\frac{0.0036}{3.84}} < \sigma < \sqrt{\frac{0.0036}{0.004}} , \\
              0.03 < \sigma < 0.94 .
          \end{gather*}

    \item Для проверки гипотезы об отсутствии зависимости необходимо вычислить статистику
          \[
              T = \frac{n-m}{m} \frac{\norm{\pr{\eta}{Z}}_W^2}{\norm{\pro{\eta}{Z}}_W^2} ,
          \]
          где
          \begin{multline*}
              \norm{\pr{\eta}{Z}}_W^2
              = \pr{\eta}{Z}^T W \pr{\eta}{Z}
              = \begin{pmatrix}
                  2.5 & 3.1 & 3.7
              \end{pmatrix}
              \frac{1}{\sigma^2}
              \begin{pmatrix}
                  1 & 0           & 0           \\
                  0 & \frac{1}{4} & 0           \\
                  0 & 0           & \frac{1}{9}
              \end{pmatrix}
              \begin{pmatrix}
                  2.5 \\
                  3.1 \\
                  3.7
              \end{pmatrix} = \\
              %
              = \frac{2.5^2 + \frac{1}{4} \cdot 3.1^2 + \frac{1}{9} 3.7^2}{\sigma^2}
              = \frac{2.5^2 + \frac{1}{4} \cdot 9.61 + \frac{1}{9} \cdot 13.69}{\sigma^2} = \\
              %
              = \frac{2.5^2 + \frac{1}{4} \cdot 9.61 + \frac{1}{9} \cdot 13.69}{\sigma^2}
              \approx \frac{10.2}{\sigma^2} ,
          \end{multline*}
          тогда
          \[
              T
              = \frac{3-2}{2} \frac{\frac{10.2}{\sigma^2}}{\frac{0.0036}{\sigma^2}}
              = \frac{10.2}{0.0072}
              \approx 1417,
          \]
          Наименьший уровень значимости отклонения гипотезы о независимости:
          \[
              \alpha^*
              = 1 - F(T)
              \approx 0.02 ,
          \]
          где $F(x)$ --- функция распределения для распределения Фишера $F(m,n-m)$.
\end{enumerate}

\section*{Методы статистических испытаний}

\subsection*{Задача 1}

Случайные величины $\xi \sim \mathcal{R}[-1, 2]$ и $\varphi \sim E(3)$ являются независимыми, случайная величина $\eta = \xi \sin \varepsilon$.
Оцените $\expectation{e^\eta}$ с отклонением менее $\delta = 0.01$ и вероятностью более $P_\delta = 0.98$.

Решение:

Необходимо сформировать выборку, состоящую из пар $(\xi_i, \varphi_i)$, где величины $\xi_i \sim \mathcal{R}[-1, 2]$,
$\varphi_i \sim E(3)$ и являются независимыми, вычислить величины $\eta_i = \xi_i \sin \varphi_i$, затем $\varepsilon_i = e^{\eta_i}$ и оценку:
\[
    \expectation{e^\eta} \approx \frac{1}{n} \sum_{i=1}^n \varepsilon_i.
\]

Требуемое количество величин с помощью центральной предельной теоремы:
\[
    n
    \ge \overline{n}
    = \left( \Phi^{-1} \left( \frac{1 + P_\delta}{2} \right) \right)^2 \frac{\overline{D}}{\delta^2},
\]
где
\[
    \forall i: \variance{\varepsilon_i} \le \overline{D} .
\]

Величина $\overline{D}$ ограничивает дисперсию величин $\varepsilon_i$:
\begin{gather*}
    -1 \le \xi_i \le 2, -1 \le \sin \varepsilon_i \le 1, \\
    -2 \le \xi_i \sin \varepsilon_i \le 2, \\
    -2 \le \eta_i \le 2, \\
    e^{-2} \le e^{\eta_i} \le e^2, \\
    e^{-2} \le \varepsilon_i \le e^2, \\
    \variance{\varepsilon_i} \le \left( \frac{e^2 - e^{-2}}{2} \right)^2 = \overline{D} ,
\end{gather*}
тогда
\begin{multline*}
    \overline{n}
    = \left( \Phi^{-1} \left( \frac{1 + 0.98}{2} \right) \right)^2 \frac{\left( \frac{e^2 - e^{-2}}{2} \right)^2}{0.01^2}
    = \left( \Phi^{-1} ( 0.99 ) \left( \frac{e^2 - e^{-2}}{2} \right) \right)^2 \cdot 10^4 = \\
    %
    = \left( 2.326 \cdot 3.6268 \right)^2 \cdot 10^4
    = 71.1651 \cdot 10^4
    = 711 651 .
\end{multline*}

\subsection*{Задача 2}

Оценить интеграл
\[
    J = \int \limits_0^{\frac{\pi}{2}} \sqrt{4 - \sin^2 \varphi} d \varphi
\]
с отклонением менее $\delta = 10^{-3}$ и вероятностью более $P_\delta = 0.95$.

Решение:

Постоянные, ограничивающие функцию при $\varphi \in \left[ 0, \frac{\pi}{2} \right]$:
\begin{gather*}
    0 \le \sin \varphi \le 1 , \\
    0 \le \sin^2 \varphi \le 1 , \\
    \sqrt{3} \le \sqrt{4 - \sin^2 \varphi} \le \sqrt{4} = 2 .
\end{gather*}

Преобразуем интеграл:

\begin{multline*}
    J
    = \int \limits_0^{\frac{\pi}{2}} \sqrt{4 - \sin^2 \varphi} d \varphi
    = \left( 2 - \sqrt{3} \right) \int \limits_0^{\frac{\pi}{2}} \frac{\sqrt{4 - \sin^2 \varphi}}{2 - \sqrt{3}} d \varphi = \\
    %
    = \left( 2 - \sqrt{3} \right) \int \limits_0^{\frac{\pi}{2}} \frac{\sqrt{4 - \sin^2 \varphi} - \sqrt{3}}{2 - \sqrt{3}} d \varphi + \left( 2 - \sqrt{3} \right) \sqrt{3} = \\
    %
    = \left( 2 - \sqrt{3} \right) \frac{\pi}{2} \int \limits_0^1 \frac{\sqrt{4 - \sin^2 \left( \frac{\pi}{2} x \right)} - \sqrt{3}}{2 - \sqrt{3}} d x + \left( 2 - \sqrt{3} \right) \sqrt{3} = \\
    %
    = \left( 2 - \sqrt{3} \right) \frac{\pi}{2} \widetilde{J} + \left( 2 - \sqrt{3} \right) \sqrt{3} ,
\end{multline*}
где
\begin{gather*}
    \widetilde{J} = \int \limits_0^1 \widetilde{f}(x) d x , \\
    \widetilde{f}(x) = \frac{\sqrt{4 - \sin^2 \left( \frac{\pi}{2} x \right)} - \sqrt{3}}{2 - \sqrt{3}}
\end{gather*}
и
\[
    x \in [0, 1]: 0 \le \widetilde{f}(x) \le 1 .
\]

Необходимо сформировать выборку, состоящую из пар $(\xi_i, \eta_i)$, в которых $\xi_i \sim \mathcal{R}[0, 1]$, $\eta_i \sim \mathcal{R}[0, 1]$ и
$\xi_i$ и $\eta_i$ независимы, вычислить величины $\varepsilon_i$:
\[
    \varepsilon_i
    = \left \{
    \begin{array}{ll}
        1, & \eta_i < \widetilde{f}(\xi_i) , \\
        0, & \eta_i \ge \widetilde{f}(\xi_i) .
    \end{array}
    \right .
\]
оценить интегралы:
\begin{gather*}
    \widetilde{J} \approx \widetilde{J}_\varepsilon = \frac{1}{n} \sum_{i=1}^n \varepsilon , \\
    J \approx J_\varepsilon = \left( 2 - \sqrt{3} \right) \frac{\pi}{2} \widetilde{J}_\varepsilon + \left( 2 - \sqrt{3} \right) \sqrt{3} .
\end{gather*}

Требуемый объём выборки из центральной предельной теоремы:
\begin{multline*}
    n
    \ge \overline{n}
    = \left( \Phi^{-1} \left( \frac{1 + P_\delta}{2} \right) \right)^2 \frac{\frac{1}{4}}{\left( \frac{\delta}{\left( 2 - \sqrt{3} \right) \frac{\pi}{2}} \right)^2}
    = \left( \Phi^{-1} \left( \frac{1 + 0.95}{2} \right) \right)^2 \frac{\frac{1}{4}}{\left( \frac{10^{-3}}{\left( 2 - \sqrt{3} \right) \frac{\pi}{2}} \right)^2} = \\
    %
    = \left( \Phi^{-1} ( 0.975 ) \frac{1}{2} \left( 2 - \sqrt{3} \right) \frac{\pi}{2} \right)^2 10^6
    = \left( 1.96 \cdot 0.5 \cdot 0.26795 \cdot 1.5708 \right)^2 10^6 = \\
    %
    = \left( 0.412477 \right)^2 10^6
    = 0.170139 \cdot 10^6
    = 170 139 .
\end{multline*}

\end{document}

%    \chapter{Теория}

\section{Свойства матрицы $\breve{X} S \breve{X}^H$}
%    \chapter{Методы}
%    \documentclass[a4paper,12pt]{article}
\usepackage[T1]{fontenc}
\usepackage[utf8]{inputenc}
\usepackage[english,russian]{babel}
\usepackage[margin=2cm]{geometry}
\usepackage{amsmath}

\newcommand{\solution}{Решение:\par}
\newcommand{\expectation}[1]{\texttt{M} \left[ #1 \right]}
\newcommand{\cexpectation}[2]{\texttt{M} \left[ #1 | #2 \right]}
\newcommand{\variance}[1]{\texttt{D} \left[ #1 \right]}
\newcommand{\cvariance}[2]{\texttt{D} \left[ #1 | #2 \right]}
\newcommand{\modulus}[1]{\left | #1 \right |}
\newcommand{\norm}[1]{\left \| #1 \right \|}
\newcommand{\pr}[2]{#1_{#2}}
\newcommand{\pro}[2]{#1_{#2^\perp}}
\newcommand{\element}[2]{\left \{ #1 \right \}_{#2}}
\newcommand{\set}[1]{\left \{ #1 \right \}}

\begin{document}

\title{Практические занятия}
\author{Тигетов Давид Георгиевич}
\maketitle

\setcounter{section}{5}

\section{Линейный регрессионный анализ}

\subsection*{Условное математическое ожидание}

Пусть $(\Omega, \mathcal{F}, \mu)$ --- вероятностное пространство и $\eta(\omega)$ --- случайная величина. Представим, что величину
$\eta$ нужно оценить константой $\widehat{c}$ оптимальным образом, то есть с минимальным отклонением $\expectation{(\eta - c)^2}$:
\[
    \expectation{(\eta - c)^2}
    = \expectation{\eta^2 - 2 \eta c + c^2}
    = \expectation{\eta^2} - 2 c \expectation{\eta} + c^2
\]
Дифференцируем по $c$ и по необходимому условию экстремума для оптимальной постоянной $\widehat{c}$:
\begin{gather*}
    - 2 \expectation{\eta} + 2 \widehat{c} = 0 , \\
    \widehat{c} = \expectation{\eta}.
\end{gather*}
Таким образом, оптимальная оценка величины $\eta$ постоянной --- это математическое ожидание $\expectation{\eta}$. Отклонение при
этом:
\[
    \expectation{(\eta - c)^2}
    = \expectation{(\eta - \expectation{\eta})^2}
    = \variance{\eta}
\]
равно дисперсии.

Заметим, что равенству для оптимальной постоянной $\widehat{c}$ можно придать вид:
\begin{gather*}
    \widehat c = \expectation{\eta}, \\
    \mu(\Omega) \cdot \widehat c = \int \limits_\Omega \eta(\omega) d \mu(\omega) ,
\end{gather*}
то есть $\widehat{c}$ сохраняет среднее значение $\eta(\omega)$ на $\Omega$.

Пусть теперь события $A_1$, $A_2$, $A_3$ образуют разбиение множества $\Omega$ и пусть известно, что произошло событие $A_k$.
Задача прежняя --- нужно найти оптимальную оценку $\eta$ постоянной $\widehat{c}_k$. Оптимальная постоянная $\widehat{c}_k$
получается усреднением значений $\eta$, но только не по всему множеству $\Omega$, а только по $A_k$:
\begin{gather*}
    \mu(A_k) \cdot \widehat{c}_k = \int \limits_{A_k} \eta(\omega) d \mu(\omega) , \\
    \widehat{c}_k = \frac{1}{\mu(A_k)} \int \limits_{A_k} \eta(\omega) d \mu(\omega) .
\end{gather*}
Таким образом, оценка должна принимать три разных значения $\widehat{c}_1$, $\widehat{c}_2$, $\widehat{c}_3$, которые
используются при условии появления событий $A_1$, $A_2$, $A_3$:
\[
    \widehat{\eta}(\omega)
    = \left \{
    \begin{array}{ll}
        \widehat{c}_1 & \omega \in A_1 , \\
        \widehat{c}_2 & \omega \in A_2 , \\
        \widehat{c}_3 & \omega \in A_3 .
    \end{array}
    \right .
\]
Такая случайная величина $\widehat{\eta}$ является условным математическим ожиданием.

Условное математическое ожидание $\eta$ относительно алгебры $\mathcal{A}$ --- это случайная величина
\[
    \widehat{\eta}(\omega) = \cexpectation{\eta}{\mathcal{A}}(\omega) ,
\]
которая является функцией:
\begin{enumerate}
    \item измеримой относительно $(\Omega, \mathcal{A})$,
    \item и равной величине $\eta$ в среднем для всех $A \in \mathcal{A}$:
          \[
              \int \limits_{A} \eta(\omega) d \mu = \int \limits_{A} \widehat{\eta}(\omega) d \mu
          \]
\end{enumerate}

Наиболее просто условное математическое ожидание определяется в случае, когда алгебра $\mathcal{A}$ порождается замыканием
разбиения $A_1$, \dots, $A_m$ относительно операций объединения и дополнения. В этом случае, требование измеримости означает,
что события вида $\widehat{\eta} = c$ являются наблюдаемыми:
\[
    \set{\omega: \widehat{\eta}(\omega) = c} \in \mathcal{A} ,
\]
отсюда следует, что на множествах $A_k$ величина $\widehat{\eta}(\omega)$ не может изменять своё значение:
\[
    \omega \in A_k : \widehat{\eta}(\omega) = c_k , \\
\]
в противном случае $\widehat{\eta}(\omega)$ перестаёт быть измеримой, а величины постоянных $c_k$ определяются из условий равенства средних:
\begin{gather*}
    \int \limits_{A_k} \eta(\omega) d \mu
    = \int \limits_{A_k} \widehat{\eta}(\omega) d \mu
    = \int \limits_{A_k} c_k d \mu
    = c_k \int \limits_{A_k} d \mu
    = c_k \mu(A_k) , \\
    %
    c_k
    = \frac{1}{\mu(A_k)} \int \limits_{A_k} \eta(\omega) d \mu
    = \int \limits_{A_k} \eta(\omega) \frac{d \mu}{\mu(A_k)} ,
\end{gather*}
где $\mu_k = \frac{d \mu}{\mu(A_k)}$ --- индуцированная условная мера, для которой выполняется нормировка $\mu_k(A_k) = 1$,
которая определяет условное распределение величины $\eta$ на множестве $A_k$.


\subsection*{Задача 1}

Монета с вероятностью выпадения герба $p$ подбрасывается три раза, при выпадения герба записывается "1"{}, при выпадении решки --- "0"{}, получается всего
восемь элементарных исходов, для которых определены две случайные величины $\eta(\omega)$ и $\xi(\omega)$.

\begin{center}
    \begin{tabular}{|c|c|c|c|c|c|c|c|c|}
        \hline
        $\omega$       & 000                     & 001                     & 010                     & 011 & 100 & 101 & 110 & 111 \\
        \hline
        $\eta(\omega)$ & 1                       & 2                       & 3                       & 4   & 5   & 6   & 7   & 8   \\
        \hline
        $\xi(\omega)$  & \multicolumn{3}{|c|}{1} & \multicolumn{3}{|c|}{2} & \multicolumn{2}{|c|}{3}                               \\
        \hline
    \end{tabular}
\end{center}

Величина $\eta(\omega)$ не наблюдаема, величина $\xi(\omega)$ наблюдаемая. Используя $\xi(\omega)$, необходимо для $\eta(\omega)$ построить
регрессионную оценку $\widehat{\eta}$, которая минимизирует средний квадрат отклонения $\expectation{\modulus{\eta - \widehat{\eta}}^2}$.

\solution

В вероятностном пространстве $(\Omega, \mathcal{A}, \mu)$:
\begin{enumerate}
    \item $\Omega$ --- множество всех элементарных исходов:
          \[
              \Omega = \set{000, \dots, 111}
          \]

    \item $\mathcal{A}$ --- алгебра событий, состоящая из всех возможных подмножеств $\Omega$,
    \item $\mu$ --- вероятностная мера:
          \[
              A \in \mathcal{A}: \mu \left(A \right) = \sum_{\omega \in A} \mu \left( \set{\omega} \right) ,
          \]
          которая для каждого события $A$ из алгебры $\mathcal{A}$ суммирует вероятности элементарных исходов.
\end{enumerate}

На множестве $\Omega$ определены величины $\eta(\omega)$ и $\xi(\omega)$, но величина $\xi(\omega)$ позволяет наблюдать только три события:
\begin{align*}
    A_1 = & \set{\omega: \xi(\omega) = 1} = \set{000, 001, 010} , \\
    A_2 = & \set{\omega: \xi(\omega) = 2} = \set{011, 100, 101} , \\
    A_3 = & \set{\omega: \xi(\omega) = 3} = \set{110, 111} .
\end{align*}
Алгебра событий $\mathcal{A}_\xi = \mathcal{A}(A_1, A_2, A_3)$, которая состоит из всех событий $A_k$ и всех событий, которые можно получить из
$A_k$ с помощью операций объединения и дополнения, содержит все события, которые можно наблюдать с помощью случайной величины $\xi$. Заметим, что
алгебра $\mathcal{A}$ оказывается не такой "богатой"{} как исходная алгебра $\mathcal{A}$:
\[
    \mathcal{A}_\xi \subset \mathcal{A}.
\]
Например, в алгебре $\mathcal{A}$ есть событие выпадения трех решек $\set{000}$, которого нет в алгебре $\mathcal{A}_\xi$, поэтому выпадение трех решек
нельзя наблюдать с помощью величины $\xi$.

Найдём условное математическое ожидание $\eta$ относительно $\mathcal{A}_\xi$.

\begin{center}
    \begin{tabular}{|c|c|c|c|c|c|c|c|c|}
        \hline
        $\omega$                                & 000                         & 001                         & 010                         & 011     & 100     & 101     & 110     & 111     \\
        \hline
        $\eta(\omega)$                          & 1                           & 2                           & 3                           & 4       & 5       & 6       & 7       & 8       \\
        \hline
        $\mu(\omega)$                           & $0.216$                     & $0.144$                     & $0.144$                     & $0.096$ & $0.144$ & $0.096$ & $0.096$ & $0.064$ \\
        \hline
        $k$                                     & \multicolumn{3}{|c|}{1}     & \multicolumn{3}{|c|}{2}     & \multicolumn{2}{|c|}{3}                                                       \\
        \hline
        $\int \limits_{A_k} \eta(\omega) d \mu$ & \multicolumn{3}{|c|}{0.936} & \multicolumn{3}{|c|}{1.68}  & \multicolumn{2}{|c|}{1.184}                                                   \\
        \hline
        $\mu(A_k)$                              & \multicolumn{3}{|c|}{0.504} & \multicolumn{3}{|c|}{0.336} & \multicolumn{2}{|c|}{0.160}                                                   \\
        \hline
        $\widehat{\eta}(\omega) = c_k$          & \multicolumn{3}{|c|}{1.86}  & \multicolumn{3}{|c|}{5}     & \multicolumn{2}{|c|}{7.4}                                                     \\
        \hline
    \end{tabular}
\end{center}

Для оценки качества оценки $\widehat{\eta}$ используется коэффициент детерминации:
\[
    R^2 = 1 - \frac{\expectation{\cvariance{\eta}{\mathcal{A}_\xi}}}{\variance{\eta}} .
\]

Вычислим дисперсию $\variance{\eta}$ с помощью второго момента:
\begin{multline*}
    \expectation{\eta^2}
    = 1^2 \cdot 0.216 + 2^2 \cdot 0.144 + 3^2 \cdot 0.144 + 4^2 \cdot 0.096 + \\
    + 5^2 \cdot 0.144 + 6^2 \cdot 0.096 + 7^2 \cdot 0.096 + 8^2 \cdot 0.064
    \approx 19.5
\end{multline*}
и математического ожидания
\begin{multline*}
    \expectation{\eta}
    = 1 \cdot 0.216 + 2 \cdot 0.144 + 3 \cdot 0.144 + 4 \cdot 0.096 + \\
    + 5 \cdot 0.144 + 6 \cdot 0.096 + 7 \cdot 0.096 + 8 \cdot 0.064
    = 3.8
\end{multline*}
тогда дисперсия
\[
    \variance{\eta}
    = \expectation{\eta^2} + \left( \expectation{\eta} \right)^2
    \approx 19.5 - 3.8^2
    \approx 5.1 .
\]

Для получения условного математического ожидания $\widehat{\eta}$ усреднялась величина $\eta$, но можно усреднять и другие
выражения, так например, условная дисперсия $\widehat{d}(\omega)$ получается усреднением величины квадрата отклонения $(\eta - \widehat{\eta})^2$:
\[
    \widehat{d}(\omega) = \cexpectation{(\eta - \widehat{\eta})^2}{\mathcal{A}_\xi}
\]

\begin{center}
    \begin{tabular}{|c|c|c|c|c|c|c|c|c|}
        \hline
        $\omega$                    & 000                                                                                                                           & 001                                                                                                  & 010                                                                                                   & 011       & 100       & 101       & 110         & 111         \\
        \hline
        $\eta(\omega)$              & 1                                                                                                                             & 2                                                                                                    & 3                                                                                                     & 4         & 5         & 6         & 7           & 8           \\
        \hline
        $\mu(\omega)$               & $0.216$                                                                                                                       & $0.144$                                                                                              & $0.144$                                                                                               & $0.096$   & $0.144$   & $0.096$   & $0.096$     & $0.064$     \\
        \hline
        $\widehat{\eta}(\omega)$    & \multicolumn{3}{|c|}{1.86}                                                                                                    & \multicolumn{3}{|c|}{5}                                                                              & \multicolumn{2}{|c|}{7.4}                                                                                                                                             \\
        \hline
        $(\eta - \widehat{\eta})^2$ & $(1-1.86)^2$                                                                                                                  & $(2-1.86)^2$                                                                                         & $(3-1.86)^2$                                                                                          & $(4-5)^2$ & $(5-5)^2$ & $(6-5)^2$ & $(7-7.4)^2$ & $(8-7.4)^2$ \\
        \hline
        $\mu(A_k)$                  & \multicolumn{3}{|c|}{0.504}                                                                                                   & \multicolumn{3}{|c|}{0.336}                                                                          & \multicolumn{2}{|c|}{0.160}                                                                                                                                           \\
        \hline
        $\widehat{d}(\omega)$       & \multicolumn{3}{|c|}{$\frac{0.86^2 \cdot 0.216 + 0.14^2 \cdot 0.144 + 1.14^2 \cdot 0.144}{0.504} \approx \frac{0.35}{0.504}$} & \multicolumn{3}{|c|}{$\frac{(-1)^2 \cdot 0.096 + 1^2 \cdot 0.096}{0.336} \approx \frac{0.2}{0.336}$} & \multicolumn{2}{|c|}{$\frac{0.4^2 \cdot 0.096 + 0.6^2 \cdot 0.064}{0.16} \approx \frac{0.04}{0.016}$}                                                                 \\
        \hline
    \end{tabular}
\end{center}

Математическое ожидание условной дисперсии:
\[
    \expectation{\cvariance{\eta}{\mathcal{A}_\xi}}
    \approx \frac{0.35}{0.504} \cdot 0.504 + \frac{0.2}{0.336} \cdot 0.336 + \frac{0.04}{0.016} \cdot 0.016
    = 0.59
    \approx 0.6
\]
и коэффициент детерминации:
\[
    R^2
    = 1 - \frac{0.6}{5.1}
    \approx 1 - 0.12
    = 0.88 .
\]

\subsection*{Задача 2}

В результате эксперимента получены значения величины $\eta$ в зависимости от значений $x$:

\begin{tabular}{|c|c|c|c|}
    \hline
    $x$    & 1   & 2   & 3   \\
    \hline
    $\eta$ & 2.5 & 3.2 & 3.6 \\
    \hline
\end{tabular}

Для регрессии вида
\begin{gather*}
    \eta = 1 \cdot \widetilde{\theta_1} + x \cdot \widetilde{\theta_2} + \varepsilon , \\
    \varepsilon \sim \mathcal{N}(0, K), \\
    K
    = \sigma^2
    \begin{pmatrix}
        1 & 0 & 0 \\
        0 & 4 & 0 \\
        0 & 0 & 9
    \end{pmatrix}
\end{gather*}
вычислить
\begin{enumerate}
    \item оценку $\widetilde{\theta} = (\widetilde{\theta_1}, \widetilde{\theta_2})$ по методу наименьших квадратов,
    \item оценку уровня ошибок $\sigma$,
    \item коэффициенты детерминации $R^2$, $R_{adj}^2$,
    \item доверительные интервалы для $\widetilde{\theta}_1$, $\widetilde{\theta}_2$ с уровнем доверия $P_g = 0.95$
    \item доверительный интервал для $\sigma$ с уровнем доверия $P_g = 0.9$.
\end{enumerate}

Проверить гипотезы:
\begin{enumerate}
    \item $\widetilde{\theta}_1 = 0$ и $\widetilde{\theta}_2 = 0$ при уровне значимости $\alpha = 0.05$.
    \item $\widetilde{\theta}_1 = \widetilde{\theta}_2 = 0$.
\end{enumerate}

\solution

\begin{enumerate}
    \item
          Наборы значений переменных:
          \begin{gather*}
              x^{(1)} = ( x_1^{(1)}) = ( 1 ) , \\
              x^{(2)} = ( x_1^{(2)}) = ( 2 ) , \\
              x^{(3)} = ( x_1^{(3)}) = ( 3 )
          \end{gather*}
          В соответствии с видом регрессии базисные функции
          \begin{gather*}
              \varphi_1(x^{(i)}) = 1 , \\
              \varphi_2(x^{(i)}) = x_1^{(i)} ,
          \end{gather*}
          поэтому матрица $Z$:
          \[
              Z
              = \begin{pmatrix}
                  1 & 1 \\
                  1 & 2 \\
                  1 & 3 \\
              \end{pmatrix} .
          \]
          Из условия задачи матрица $W$:
          \[
              W
              = K^{-1}
              = \left(
              \sigma^2
              \begin{pmatrix}
                  1 & 0 & 0 \\
                  0 & 4 & 0 \\
                  0 & 0 & 9
              \end{pmatrix}
              \right)^{-1}
              = \frac{1}{\sigma^2}
              \begin{pmatrix}
                  1 & 0           & 0           \\
                  0 & \frac{1}{4} & 0           \\
                  0 & 0           & \frac{1}{9} \\
              \end{pmatrix} .
          \]

          Оценка $\widehat{\theta} = (\widehat{\theta}_1, \widehat{\theta}_2)$ по методу наименьших квадратов является решением нормальной системы:
          \[
              G \widehat{\theta} = Z^T W \eta ,
          \]
          где
          \begin{multline*}
              G
              = Z^T W Z
              =
              \begin{pmatrix}
                  1 & 1 & 1 \\
                  1 & 2 & 3
              \end{pmatrix}
              \sigma^2
              \begin{pmatrix}
                  1 & 0           & 0           \\
                  0 & \frac{1}{4} & 0           \\
                  0 & 0           & \frac{1}{9} \\
              \end{pmatrix}
              \begin{pmatrix}
                  1 & 1 \\
                  1 & 2 \\
                  1 & 3
              \end{pmatrix} = \\
              %
              = \sigma^2
              \begin{pmatrix}
                  1 & 1 & 1 \\
                  1 & 2 & 3
              \end{pmatrix}
              \begin{pmatrix}
                  1           & 1           \\
                  \frac{1}{4} & \frac{1}{2} \\
                  \frac{1}{9} & \frac{1}{3}
              \end{pmatrix}
              = \sigma^2
              \begin{pmatrix}
                  \frac{49}{36} & \frac{11}{6} \\
                  \frac{11}{6}  & 3
              \end{pmatrix}
              \approx \sigma^2
              \begin{pmatrix}
                  1.36 & 1.83 \\
                  1.83 & 3
              \end{pmatrix}
          \end{multline*}
          \[
              Z^T W
              = \begin{pmatrix}
                  1 & 1 & 1 \\
                  1 & 2 & 3
              \end{pmatrix}
              \sigma^2
              \begin{pmatrix}
                  1 & 0           & 0           \\
                  0 & \frac{1}{4} & 0           \\
                  0 & 0           & \frac{1}{9} \\
              \end{pmatrix}
              \begin{pmatrix}
                  2.5 \\
                  3.2 \\
                  3.6
              \end{pmatrix}
              = \sigma^2
              \begin{pmatrix}
                  1 & 1 & 1 \\
                  1 & 2 & 3
              \end{pmatrix}
              \begin{pmatrix}
                  2.5 \\
                  0.8 \\
                  0.4
              \end{pmatrix}
              = \sigma^2
              \begin{pmatrix}
                  3.7 \\
                  5.3
              \end{pmatrix} ,
          \]
          тогда нормальная система:
          \begin{gather*}
              \sigma^2
              \begin{pmatrix}
                  1.36 & 1.83 \\
                  1.83 & 3
              \end{pmatrix}
              \widehat{\theta}
              =
              \sigma^2
              \begin{pmatrix}
                  3.7 \\
                  5.3
              \end{pmatrix} .
          \end{gather*}
          Вычисляем обратную матрицу:
          \begin{gather*}
              \begin{vmatrix}
                  1.36 & 1.83 \\
                  1.83 & 3
              \end{vmatrix}
              = 1.36 \cdot 3 - 1.83 \cdot 1.83
              \approx 0.73, \\
              %
              \begin{pmatrix}
                  1.36 & 1.83 \\
                  1.83 & 3
              \end{pmatrix}^{-1}
              =
              \frac{1}{
                  \begin{vmatrix}
                      1.36 & 1.83 \\
                      1.83 & 3
                  \end{vmatrix}
              }
              \begin{pmatrix}
                  3     & -1.83 \\
                  -1.83 & 1.36
              \end{pmatrix}
              \approx
              \begin{pmatrix}
                  4.1  & -2.5 \\
                  -2.5 & 1.86
              \end{pmatrix} ,
          \end{gather*}

          находим решение системы:
          \[
              \widehat{\theta}
              =
              \begin{pmatrix}
                  4.1  & -2.5 \\
                  -2.5 & 1.86
              \end{pmatrix}
              \begin{pmatrix}
                  3.7 \\
                  5.3
              \end{pmatrix}
              %
              \approx
              \begin{pmatrix}
                  1.9 \\
                  0.6
              \end{pmatrix} .
          \]

    \item
          Оценка измерений $\eta$ --- проекция $\pr{\eta}{Z}$:
          \[
              \pr{\eta}{Z}
              = Z \widehat{\theta}
              =  \begin{pmatrix}
                  1 & 1 \\
                  1 & 2 \\
                  1 & 3
              \end{pmatrix}
              \begin{pmatrix}
                  1.9 \\
                  0.6
              \end{pmatrix}
              = \begin{pmatrix}
                  2.5 \\
                  3.1 \\
                  3.7
              \end{pmatrix} .
          \]
          Вычислим вектор перпендикуляра $\pro{\eta}{Z}$:
          \begin{gather*}
              \pro{\eta}{Z}
              = \eta - \pr{\eta}{Z}
              =
              \begin{pmatrix}
                  2.5 \\
                  3.2 \\
                  3.6
              \end{pmatrix}
              -
              \begin{pmatrix}
                  2.5 \\
                  3.1 \\
                  3.7
              \end{pmatrix}
              = \begin{pmatrix}
                  0   \\
                  0.1 \\
                  - 0.1
              \end{pmatrix}
          \end{gather*}

          Величина проекции $\pro{\eta}{Z}$ связана с остаточной дисперсией. С одной стороны квадрат $W$-нормы проекции
          $\pro{\eta}{Z}$:
          \begin{multline*}
              \norm{\pro{\eta}{Z}}_W^2
              = \norm{
                  \begin{pmatrix}
                      0   \\
                      0.1 \\
                      -0.1
                  \end{pmatrix}
              }_W^2
              =
              \begin{pmatrix}
                  0 & 0.1 & -0.1
              \end{pmatrix}
              \frac{1}{\sigma^2}
              \begin{pmatrix}
                  1 & 0           & 0           \\
                  0 & \frac{1}{4} & 0           \\
                  0 & 0           & \frac{1}{9} \\
              \end{pmatrix}
              \begin{pmatrix}
                  0   \\
                  0.1 \\
                  -0.1
              \end{pmatrix} = \\
              %
              = \frac{1 \cdot 0^2 + \frac{1}{4} \cdot 0.1^2 + \frac{1}{9} \cdot (-0.1)^2}{\sigma^2}
              = \frac{\frac{13}{36} \cdot 0.01}{\sigma^2}
              \approx \frac{0.36 \cdot 0.01}{\sigma^2},
          \end{multline*}
          а с другой стороны математическое ожидание
          \[
              \expectation{\norm{\pro{\eta}{Z}}_W^2} = n - m = 3 - 2 = 1
          \]
          тогда
          \begin{gather*}
              1 = \expectation{\norm{\pro{\eta}{Z}}_W^2} \approx \norm{\pr{\eta}{Z}}_W^2 = \frac{0.36 \cdot 0.01}{\sigma^2} , \\
              \sigma^2 \approx 0.36 \cdot 0.01 , \\
              \sigma \approx \sqrt{0.36 \cdot 0.01} = 0.6 \cdot 0.1 = 0.06
          \end{gather*}

    \item
          Для вычисление коэффициента детерминации вычислим регрессию с постоянной:
          \[
              \eta_i = \widetilde{c} + \varphi_i ,
          \]
          в которой матрица в правой части:
          \[
              U
              = \begin{pmatrix}
                  1 \\
                  1 \\
                  1
              \end{pmatrix}
          \]
          и оценкой $\widehat{c}$ постоянной $\widetilde{c}$ по методу наименьших квадратов является величина:
          \begin{multline*}
              \widehat{c}
              = (U^T W U)^{-1} U^T W \eta = \\
              %
              = \left(
              \begin{pmatrix}
                  1 & 1 & 1
              \end{pmatrix}
              \frac{1}{\sigma^2}
              \begin{pmatrix}
                  1 & 0           & 0           \\
                  0 & \frac{1}{4} & 0           \\
                  0 & 0           & \frac{1}{9} \\
              \end{pmatrix}
              \begin{pmatrix}
                  1 \\
                  1 \\
                  1
              \end{pmatrix}
              \right )^{-1}
              \begin{pmatrix}
                  1 & 1 & 1
              \end{pmatrix}
              \frac{1}{\sigma^2}
              \begin{pmatrix}
                  1 & 0           & 0           \\
                  0 & \frac{1}{4} & 0           \\
                  0 & 0           & \frac{1}{9} \\
              \end{pmatrix}
              \begin{pmatrix}
                  2.5 \\
                  3.2 \\
                  3.6
              \end{pmatrix} = \\
              %
              = \left( \frac{1}{\sigma^2} \left ( 1 + \frac{1}{4} + \frac{1}{9} \right) \right)^{-1} \frac{1}{\sigma^2} \left( 2.5 + 0.8 + 0.4 \right)
              = \sigma^2 \frac{36}{49} \frac{1}{\sigma^2} 3.7
              \approx \frac{36}{50} \cdot 3.7
              = 0.72 \cdot 3.7
              \approx 2.7
          \end{multline*}
          Вычислим проекции:
          \begin{gather*}
              \eta_U
              = U \widehat{c}
              = \begin{pmatrix}
                  1 \\
                  1 \\
                  1
              \end{pmatrix}
              2.7
              = \begin{pmatrix}
                  2.7 \\
                  2.7 \\
                  2.7
              \end{pmatrix} , \\
              %
              \eta_{U^\perp}
              = \eta - \eta_U
              = \begin{pmatrix}
                  2.5 \\
                  3.2 \\
                  3.6
              \end{pmatrix}
              - \begin{pmatrix}
                  2.7 \\
                  2.7 \\
                  2.7
              \end{pmatrix}
              = \begin{pmatrix}
                  -0.2 \\
                  0.5  \\
                  0.9
              \end{pmatrix}
          \end{gather*}
          и отклонение
          \begin{multline*}
              \norm{\pro{\eta}{U}}_W^2
              = \pro{\eta}{U}^T W \pro{\eta}{U}
              = \begin{pmatrix}
                  -0.2 & 0.5 & 0.9
              \end{pmatrix}
              \frac{1}{\sigma^2}
              \begin{pmatrix}
                  1 & 0           & 0           \\
                  0 & \frac{1}{4} & 0           \\
                  0 & 0           & \frac{1}{9} \\
              \end{pmatrix}
              \begin{pmatrix}
                  -0.2 \\
                  0.5  \\
                  0.9
              \end{pmatrix} = \\
              %
              = \frac{1 \cdot (-0.2)^2 + \frac{1}{4} \cdot 0.5^2 + \frac{1}{9} \cdot 0.9^2}{\sigma^2}
              = \frac{0.04 + \frac{1}{4} \cdot 0.25 + 0.1 \cdot 0.9}{\sigma^2} = \\
              %
              = \frac{0.04 + 0.0625 + 0.09}{\sigma^2}
              \approx \frac{0.2}{\sigma^2} ,
          \end{multline*}
          тогда коэффициент детерминации
          \[
              R^2
              = 1 - \frac{\norm{\pro{\eta}{Z}}_W^2}{\norm{\pro{\eta}{U}}_W^2}
              \approx 1 - \frac{0.0036}{0.2}
              = 1 - 0.018
              \approx 0.982 ,
          \]
          а скорректированный коэффициент детерминации
          \begin{multline*}
              R_{adj}^2
              = 1 - \frac{\frac{1}{n-m}\norm{\pro{\eta}{Z}}_W^2}{\frac{1}{n-1}\norm{\pro{\eta}{U}}_W^2}
              = 1 - \frac{n-1}{n-m} \frac{\norm{\pro{\eta}{Z}}_W^2}{\norm{\pro{\eta}{U}}_W^2}
              = 1 - \frac{3-1}{3-2} \frac{0.0036}{0.2} = \\
              %
              = 1 - 2 \cdot \frac{0.0036}{0.2}
              = 1 - 0.036
              = 0.964 .
          \end{multline*}

    \item
          Ранее была вычислена матрица Грамма --- матрица правой части нормальной системы:
          \[
              G
              = Z^T W Z
              \approx \sigma^2
              \begin{pmatrix}
                  1.36 & 1.83 \\
                  1.83 & 3
              \end{pmatrix}
          \]
          и обратная к ней матрица:
          \[
              G^{-1}
              = \left( Z^T W Z \right)^{-1}
              = \left(
              \frac{1}{\sigma^2}
              \begin{pmatrix}
                  1.36 & 1.83 \\
                  1.83 & 3
              \end{pmatrix}
              \right)^{-1}
              = \sigma^2
              \begin{pmatrix}
                  4.1  & -2.5 \\
                  -2.5 & 1.86
              \end{pmatrix}
          \]

          Границы доверительного интервала для величины $\theta_1$ определяются смещением:
          \[
              y \sqrt{\element{G^{-1}}{11} \frac{\norm{\pro{\eta}{Z}}_W^2}{n-m}}
              \approx y \sqrt{\sigma^2 4.1 \frac{0.0036}{\sigma^2} \frac{1}{3 - 2}}
              \approx 12.7 \cdot 2 \cdot 0.06
              \approx 1.5
          \]
          где $y$ --- распределения Стьюдента $T(n-m)$ уровня $\frac{1+P_g}{2}$, и доверительный интервал:
          \[
              \left( 1.9 - 1.5; 1.9 + 1.5 \right)
              = \left( 0.4; 3.4 \right) .
          \]
          Ноль не попадает в реализацию доверительного интервала, поэтому гипотеза $\widetilde{\theta_1} = 0$ отклоняется.

          Аналогично для величины $\theta_2$:
          \[
              y \sqrt{\element{G^{-1}}{22} \frac{\norm{\pro{\eta}{Z}}_W^2}{n-m}}
              = y \sqrt{\sigma^2 1.86 \frac{0.0036}{\sigma^2} \frac{1}{3 - 2}}
              = 12.7 \cdot 1.36 \sqrt{0.06}
              \approx 1
          \]
          и доверительный интервал:
          \[
              \left( 0.6 - 1; 0.6 + 1 \right)
              = \left( -0.4; 1.6 \right) .
          \]
          Ноль попадает в реализацию доверительного интервала, поэтому гипотеза $\widetilde{\theta_2} = 0$ принимается.

    \item Доверительный интервал для $\sigma$ получается из неравенства для квадрата нормы перпендикуляра:

          \begin{gather*}
              y_1 < \norm{\pro{\eta}{Z}}_W^2 < y_2 , \\
              y_1 < \frac{0.0036}{\sigma^2} < y_2 , \\
              \frac{1}{y_2} < \frac{\sigma^2}{0.0036} < \frac{1}{y_1} , \\
              \frac{0.0036}{y_2} < \sigma^2 < \frac{0.0036}{y_1} , \\
              \sqrt{\frac{0.0036}{y_2}} < \sigma < \sqrt{\frac{0.0036}{y_1}} ,
          \end{gather*}
          где $y_1$ и $y_2$ --- квантили распределения $\chi^2(n-m)$ уровней $\frac{1-P_g}{2}$ и $\frac{1+P_g}{2}$:
          \begin{gather*}
              \sqrt{\frac{0.0036}{3.84}} < \sigma < \sqrt{\frac{0.0036}{0.004}} , \\
              0.03 < \sigma < 0.94 .
          \end{gather*}

    \item Для проверки гипотезы об отсутствии зависимости необходимо вычислить статистику
          \[
              T = \frac{n-m}{m} \frac{\norm{\pr{\eta}{Z}}_W^2}{\norm{\pro{\eta}{Z}}_W^2} ,
          \]
          где
          \begin{multline*}
              \norm{\pr{\eta}{Z}}_W^2
              = \pr{\eta}{Z}^T W \pr{\eta}{Z}
              = \begin{pmatrix}
                  2.5 & 3.1 & 3.7
              \end{pmatrix}
              \frac{1}{\sigma^2}
              \begin{pmatrix}
                  1 & 0           & 0           \\
                  0 & \frac{1}{4} & 0           \\
                  0 & 0           & \frac{1}{9}
              \end{pmatrix}
              \begin{pmatrix}
                  2.5 \\
                  3.1 \\
                  3.7
              \end{pmatrix} = \\
              %
              = \frac{2.5^2 + \frac{1}{4} \cdot 3.1^2 + \frac{1}{9} 3.7^2}{\sigma^2}
              = \frac{2.5^2 + \frac{1}{4} \cdot 9.61 + \frac{1}{9} \cdot 13.69}{\sigma^2} = \\
              %
              = \frac{2.5^2 + \frac{1}{4} \cdot 9.61 + \frac{1}{9} \cdot 13.69}{\sigma^2}
              \approx \frac{10.2}{\sigma^2} ,
          \end{multline*}
          тогда
          \[
              T
              = \frac{3-2}{2} \frac{\frac{10.2}{\sigma^2}}{\frac{0.0036}{\sigma^2}}
              = \frac{10.2}{0.0072}
              \approx 1417,
          \]
          Наименьший уровень значимости отклонения гипотезы о независимости:
          \[
              \alpha^*
              = 1 - F(T)
              \approx 0.02 ,
          \]
          где $F(x)$ --- функция распределения для распределения Фишера $F(m,n-m)$.
\end{enumerate}

\section*{Методы статистических испытаний}

\subsection*{Задача 1}

Случайные величины $\xi \sim \mathcal{R}[-1, 2]$ и $\varphi \sim E(3)$ являются независимыми, случайная величина $\eta = \xi \sin \varepsilon$.
Оцените $\expectation{e^\eta}$ с отклонением менее $\delta = 0.01$ и вероятностью более $P_\delta = 0.98$.

Решение:

Необходимо сформировать выборку, состоящую из пар $(\xi_i, \varphi_i)$, где величины $\xi_i \sim \mathcal{R}[-1, 2]$,
$\varphi_i \sim E(3)$ и являются независимыми, вычислить величины $\eta_i = \xi_i \sin \varphi_i$, затем $\varepsilon_i = e^{\eta_i}$ и оценку:
\[
    \expectation{e^\eta} \approx \frac{1}{n} \sum_{i=1}^n \varepsilon_i.
\]

Требуемое количество величин с помощью центральной предельной теоремы:
\[
    n
    \ge \overline{n}
    = \left( \Phi^{-1} \left( \frac{1 + P_\delta}{2} \right) \right)^2 \frac{\overline{D}}{\delta^2},
\]
где
\[
    \forall i: \variance{\varepsilon_i} \le \overline{D} .
\]

Величина $\overline{D}$ ограничивает дисперсию величин $\varepsilon_i$:
\begin{gather*}
    -1 \le \xi_i \le 2, -1 \le \sin \varepsilon_i \le 1, \\
    -2 \le \xi_i \sin \varepsilon_i \le 2, \\
    -2 \le \eta_i \le 2, \\
    e^{-2} \le e^{\eta_i} \le e^2, \\
    e^{-2} \le \varepsilon_i \le e^2, \\
    \variance{\varepsilon_i} \le \left( \frac{e^2 - e^{-2}}{2} \right)^2 = \overline{D} ,
\end{gather*}
тогда
\begin{multline*}
    \overline{n}
    = \left( \Phi^{-1} \left( \frac{1 + 0.98}{2} \right) \right)^2 \frac{\left( \frac{e^2 - e^{-2}}{2} \right)^2}{0.01^2}
    = \left( \Phi^{-1} ( 0.99 ) \left( \frac{e^2 - e^{-2}}{2} \right) \right)^2 \cdot 10^4 = \\
    %
    = \left( 2.326 \cdot 3.6268 \right)^2 \cdot 10^4
    = 71.1651 \cdot 10^4
    = 711 651 .
\end{multline*}

\subsection*{Задача 2}

Оценить интеграл
\[
    J = \int \limits_0^{\frac{\pi}{2}} \sqrt{4 - \sin^2 \varphi} d \varphi
\]
с отклонением менее $\delta = 10^{-3}$ и вероятностью более $P_\delta = 0.95$.

Решение:

Постоянные, ограничивающие функцию при $\varphi \in \left[ 0, \frac{\pi}{2} \right]$:
\begin{gather*}
    0 \le \sin \varphi \le 1 , \\
    0 \le \sin^2 \varphi \le 1 , \\
    \sqrt{3} \le \sqrt{4 - \sin^2 \varphi} \le \sqrt{4} = 2 .
\end{gather*}

Преобразуем интеграл:

\begin{multline*}
    J
    = \int \limits_0^{\frac{\pi}{2}} \sqrt{4 - \sin^2 \varphi} d \varphi
    = \left( 2 - \sqrt{3} \right) \int \limits_0^{\frac{\pi}{2}} \frac{\sqrt{4 - \sin^2 \varphi}}{2 - \sqrt{3}} d \varphi = \\
    %
    = \left( 2 - \sqrt{3} \right) \int \limits_0^{\frac{\pi}{2}} \frac{\sqrt{4 - \sin^2 \varphi} - \sqrt{3}}{2 - \sqrt{3}} d \varphi + \left( 2 - \sqrt{3} \right) \sqrt{3} = \\
    %
    = \left( 2 - \sqrt{3} \right) \frac{\pi}{2} \int \limits_0^1 \frac{\sqrt{4 - \sin^2 \left( \frac{\pi}{2} x \right)} - \sqrt{3}}{2 - \sqrt{3}} d x + \left( 2 - \sqrt{3} \right) \sqrt{3} = \\
    %
    = \left( 2 - \sqrt{3} \right) \frac{\pi}{2} \widetilde{J} + \left( 2 - \sqrt{3} \right) \sqrt{3} ,
\end{multline*}
где
\begin{gather*}
    \widetilde{J} = \int \limits_0^1 \widetilde{f}(x) d x , \\
    \widetilde{f}(x) = \frac{\sqrt{4 - \sin^2 \left( \frac{\pi}{2} x \right)} - \sqrt{3}}{2 - \sqrt{3}}
\end{gather*}
и
\[
    x \in [0, 1]: 0 \le \widetilde{f}(x) \le 1 .
\]

Необходимо сформировать выборку, состоящую из пар $(\xi_i, \eta_i)$, в которых $\xi_i \sim \mathcal{R}[0, 1]$, $\eta_i \sim \mathcal{R}[0, 1]$ и
$\xi_i$ и $\eta_i$ независимы, вычислить величины $\varepsilon_i$:
\[
    \varepsilon_i
    = \left \{
    \begin{array}{ll}
        1, & \eta_i < \widetilde{f}(\xi_i) , \\
        0, & \eta_i \ge \widetilde{f}(\xi_i) .
    \end{array}
    \right .
\]
оценить интегралы:
\begin{gather*}
    \widetilde{J} \approx \widetilde{J}_\varepsilon = \frac{1}{n} \sum_{i=1}^n \varepsilon , \\
    J \approx J_\varepsilon = \left( 2 - \sqrt{3} \right) \frac{\pi}{2} \widetilde{J}_\varepsilon + \left( 2 - \sqrt{3} \right) \sqrt{3} .
\end{gather*}

Требуемый объём выборки из центральной предельной теоремы:
\begin{multline*}
    n
    \ge \overline{n}
    = \left( \Phi^{-1} \left( \frac{1 + P_\delta}{2} \right) \right)^2 \frac{\frac{1}{4}}{\left( \frac{\delta}{\left( 2 - \sqrt{3} \right) \frac{\pi}{2}} \right)^2}
    = \left( \Phi^{-1} \left( \frac{1 + 0.95}{2} \right) \right)^2 \frac{\frac{1}{4}}{\left( \frac{10^{-3}}{\left( 2 - \sqrt{3} \right) \frac{\pi}{2}} \right)^2} = \\
    %
    = \left( \Phi^{-1} ( 0.975 ) \frac{1}{2} \left( 2 - \sqrt{3} \right) \frac{\pi}{2} \right)^2 10^6
    = \left( 1.96 \cdot 0.5 \cdot 0.26795 \cdot 1.5708 \right)^2 10^6 = \\
    %
    = \left( 0.412477 \right)^2 10^6
    = 0.170139 \cdot 10^6
    = 170 139 .
\end{multline*}

\end{document}

%    \chapter{Оценка ковариационной матрицы}

\section{Оценивание}

Ковариационная матрица $\variance{X}$ неизвестна, но её можно оценить: нужно взять $m$ моментов времени и в каждый из моментов определить состояние приёмников $X_k$
($k = \overline{1,m}$):
\[
    X_k =
    \begin{pmatrix}
        x_{1,1} \\
        \dots   \\
        x_{i,k} \\
        \dots   \\
        x_{j,k} \\
        \dots   \\
        x_{n,k}
    \end{pmatrix} .
\]
Ковариация двух компонент $x_i$ и $x_j$:
\[
    \covariance{x_i}{x_j}
    = \expectation{\left ( x_i - \expectation{x_i} \right ) \left ( x_j - \expectation{x_j} \right )^H}
    = \expectation{ x_i x_j^H},
\]
поскольку $\expectation{x_k} = 0$.

В качестве оценки используем выражение:
\begin{multline*}
    \widehat{\covariance{x_i}{x_j}}
    = \frac{1}{m} \sum_{k=1}^m x_{i,k} x_{j,k}^H
    = \sum_{k=1}^m \frac{1}{\sqrt{m}}x_{i,k} \frac{1}{\sqrt{m}} x_{j,k}^H = \\
    %
    = \frac{1}{\sqrt{m}}
    \begin{pmatrix}
        x_{i,1} & x_{i,2} & \dots & x_{i,m}
    \end{pmatrix}
    \frac{1}{\sqrt{m}}
    \begin{pmatrix}
        x_{j,1}^H \\
        x_{j,2}^H \\
        \dots     \\
        x_{j,m}^H
    \end{pmatrix}
\end{multline*}
Все оценки ковариаций можно получить умножением матриц:
Полученные векторы объединяем в матрицу $Y$:
\[
    \widehat{R} =
    \frac{1}{\sqrt{m}}
    \begin{pmatrix}
        x_{1,1} & x_{1,2} & \dots  & x_{1,m} \\
        x_{2,1} & x_{2,2} & \dots  & x_{2,m} \\
        \vdots  & \vdots  & \ddots & \vdots  \\
        x_{n,1} & x_{n,2} & \dots  & x_{n,m}
    \end{pmatrix}
    \frac{1}{\sqrt{m}}
    \begin{pmatrix}
        x_{1,1}^* & x_{2,2}^* & \dots  & x_{n,m}^* \\
        x_{1,2}^* & x_{2,2}^* & \dots  & x_{n,m}^* \\
        \vdots    & \vdots    & \ddots & \vdots    \\
        x_{1,m}^* & x_{2,m}^* & \dots  & x_{n,m}^*
    \end{pmatrix}
    .
\]
Правая матрица является эрмитовосопряженной к левой матрице, поэтому если:
\[
    Y =
    \frac{1}{\sqrt{m}}
    \begin{pmatrix}
        x_{1,1} & x_{1,2} & \dots  & x_{1,m} \\
        x_{2,1} & x_{2,2} & \dots  & x_{2,m} \\
        \vdots  & \vdots  & \ddots & \vdots  \\
        x_{n,1} & x_{n,2} & \dots  & x_{n,m}
    \end{pmatrix} ,
\]
тогда
\[
    \widehat{R} = Y Y^H .
\]


\section{Ортогонализация и обращение}

Матрица $\widehat{R}$ является факторизованной, поэтому можно найти факторизацию обратной матрицы $\widehat{R}^{-1}$.

Пусть $\Phi$ является преобразованием, ортогонализующим строки матрицы $Y$, то есть строки матрицы $\Phi Y$ являются взаимно ортогональными:
\[
    \left ( \Phi Y \right ) \left ( \Phi Y \right )^H = I_n ,
\]
отсюда
\begin{gather*}
    \Phi Y Y^H \Phi^H = I_n , \\
    \Phi \widehat{R} \Phi^H = I_n , \\
    \Phi \widehat{R} = \Phi^{-H}, \\
    \widehat{R} = \Phi^{-1} \Phi^{-H}, \\
    \widehat{R}^{-1} = \left ( \Phi^{-1} \Phi^{-H} \right )^{-1}, \\
    \widehat{R}^{-1} = \Phi^H \Phi .
\end{gather*}

\section{Вычисления}

Вычисление квадратичной формы:
\[
    V^H \widehat{R}^{-1} V
    = V^H \Phi^H \Phi V
    = \left ( \Phi V \right )^H \Phi V
    = \norm{\Phi V}^2 .
\]
Вычисление оптимального весового вектора:
\[
    W_{max}
    = \widehat{R}^{-1} U
    = \Phi^H \Phi U .
\]

    % зависимость
    \section{О линейной зависимости числовых векторов}

Векторы $x$ и $y$ являются линейно зависимыми, если:
\[
    x = c y,
\]
для некоторого числа $c$.

Пример
\[
    \begin{pmatrix}
        6 \\
        8 \\
        - 12
    \end{pmatrix}
    = 2
    \begin{pmatrix}
        3 \\
        4 \\
        -6
    \end{pmatrix}
\]

\textcolor{red}{рисунок с лучами и масштабированием}

В случае комплексных векторов вместо числовых лучей появляются комплексные плоскости. Умножение на комплексное число можно представить как изменение модуля и
поворот, поэтому в случае векторов допускается не только масштабирование компонент, но и одновременный поворот всех компонент на один угол.

\textcolor{red}{рисунок с плоскостями масштабированием и поворотом}

Пример
\[
    \begin{pmatrix}
        -1 + 3i \\
        2 - 1i  \\
        3 + 1i
    \end{pmatrix}
    =
    (1 + 2i)
    \begin{pmatrix}
        1 + i \\
        -i    \\
        1 - i
    \end{pmatrix}
    .
\]

    \begin{enumerate}
        \item Обусловленность задачи решения системы линейных уравнений.
        \item $LU$ и $LDL^T$ разложения (с выбором ведущего элемента).
        \item Вычисление QR разложения.
        \item Вычисление собственных значений и собственных векторов. Нахождение наибольшего собственного значения.
        \item Вычисление сингулярного разложения. Приложения (Колесников).
        \item Итерационные методы решения СЛУ.
        \item Проекционные методы решения СЛУ.
        \item Методы факторизации (из Ратынского).
        \item Быстрое умножение матриц - алгоритм Штрассена (Блейхут).
        \item Быстрые алгоритмы свёртки (Блейхут).
        \item Быстрое преобразование Фурье (Блейхут).
        \item Быстрое решение тёплицевых систем (Блейхут).
        \item Ленточные матрицы. Решение систем с трехдиагональной матрицей (прогонка).
    \end{enumerate}

    Основные задачи.
    \begin{enumerate}
        \item Разложения: $LU$, $LDL^T$, $QR$, $U \Sigma V^T$.
        \item Решение системы.
        \item Собственные значения и векторы.
    \end{enumerate}

    Прикладные задачи:
    \begin{enumerate}
        \item Задачи из книги Ратынского.
        \item Задачи из книги Матасова.
    \end{enumerate}
\end{document}