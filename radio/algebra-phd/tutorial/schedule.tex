\chapter{План занятий}

Первый год.

{
\center
\begin{tabular}{|p{3cm}|p{12cm}|}
    \hline
    Занятие 1  & Унитарные пространства. Сопряжённый оператор. Свойства сопряженного оператора.                                 \\
    \hline
    Занятие 2  & Теорема Шура. Нормальные матрицы. Эрмитовые матрицы.                                                           \\
    \hline
    Занятие 3  & Экстремумы отношения Релея.                                                                                    \\
    \hline
    Занятие 4  & Экстремумы отношения Релея. Вычисление в Matlab.                                                               \\
    \hline
    Занятие 5  & Комплексные огибающие. Двухканальный излучатель. Коэффициент усиления. Пример 2. Пример 3. Пример 4. Пример 5. \\
    \hline
    Занятие 6  & Энергетическое ограничение. Коэффициент полезного действия. Примеры. Антенна.                                  \\
    \hline
    Занятие 7  & Плоская волна, набеги фаз в решётке. Обнаружение и пеленгация одного источника.                                \\
    \hline
    Занятие 8  & Обнаружение и пеленгация нескольких источников. Альтернативная пеленгация.                                     \\
    \hline
    Занятие 9  & Задача адаптации. Оценка ковариационной матрицы. Ортогонализатор.                                              \\
    \hline
    Занятие 10 & Сигналы. Циклическая свёртка. Циркулянты.                                                                      \\
    \hline
    Занятие 11 & Матрица Фурье. Спектр циклической свёртки.                                                                     \\
    \hline
    Занятие 12 & Быстрое преобразование Фурье. Фильтр нижних частот.                                                            \\
    \hline
\end{tabular}
\par
}

Второй год.

{
\center
\begin{tabular}{|p{3cm}|p{12cm}|}
    \hline
    Занятие 1  & Унитарные пространства. Сопряжённый оператор. Свойства сопряженного оператора. Теорема Шура. \\
    \hline
    Занятие 2  & Нормальные матрицы. Эрмитовые матрицы. Экстремумы отношения Релея.                           \\
    \hline
    Занятие 3  & Решение задачи 1 в Matlab.                                                                   \\
    \hline
    Занятие 4  & Комплексные огибающие. Диаграмма направленности. Коэффициент усиления. Примеры 2 -- 5.       \\
    \hline
    Занятие 5  & Энергетическое ограничение. Коэффициент полезного действия. Примеры 2 -- 5.                  \\
    \hline
    Занятие 6  & Решение задачи 2 в Matlab.                                                                   \\
    \hline
    Занятие 7  & Решение задачи 2 в Matlab.                                                                   \\
    \hline
    Занятие 8  & Плоская волна, набеги фаз в решётке. Обнаружение и пеленгация одного источника.              \\
    \hline
    Занятие 9  & Обнаружение и пеленгация нескольких источников. Альтернативная пеленгация.                   \\
    \hline
    Занятие 10  & Задача адаптации. Оценка ковариационной матрицы. Ортогонализатор.                            \\
    \hline
    Занятие 11 & Решение задачи 3 в Matlab.                                                                   \\
    \hline
    Занятие 12 & Сигналы. Циклическая свёртка. Циркулянты.                                                    \\
    \hline
    Занятие 13 & Матрица Фурье. Спектр циклической свёртки.                                                   \\
    \hline
    Занятие 14 & Быстрое преобразование Фурье. Фильтр нижних частот.                                          \\
    \hline
\end{tabular}
\par
}
