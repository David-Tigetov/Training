\chapter{План занятий}

Для отношения Релея.
\begin{itemize}
    \item[Занятие 1.] Унитарные пространства. Сопряжённый оператор. Свойства сопряженного оператора.
    \item[Занятие 2.] Теорема Шура. Нормальные матрицы. Эрмитовые матрицы.
    \item[Занятие 3.] Экстремумы отношения Релея.
    \item[Занятие 4.] Экстремумы отношения Релея. Вычисление в Matlab.
    \item[Занятие 5.] Комплексные огибающие. Двухканальный излучатель. Коэффициент усиления. Пример 2. Пример 3. Пример 4. Пример 5.
    \item[Занятие 6.] Энергетическое ограничение. Коэффициент полезного действия. Примеры. Антенна.
    \item[Занятие 7.] Плоская волна, набеги фаз в решётке. Обнаружение и пеленгация одного источника.
    \item[Занятие 8.] Обнаружение и пеленгация нескольких источников. Альтернативная пеленгация.
    \item[Занятие 9.] Задача адаптации. Оценка ковариационной матрицы. Ортогонализатор.
    \item[Занятие 10.] Сигналы. Циклическая свёртка. Циркулянты.
    \item[Занятие 11.] Матрица Фурье. Спектр циклической свёртки.
    \item[Занятие 12.] Быстрое преобразование Фурье. Фильтр нижних частот.
\end{itemize}
