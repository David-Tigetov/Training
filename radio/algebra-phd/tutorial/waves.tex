\chapter{Волны}


\section{Плоская волна}

Будем использовать упрощённую модель электромагнитной волны, у которой фронт является прямой или плоскостью. Фронт волны --- геометрическое место точек с колебаниями в одной фазе.
Все точки на прямой или плоскости имеют одинаковую фазу колебания вектора напряжённости электрического поля.

\textcolor{red}{Рисунок волн.}

Зафиксируем декартову систему координат. В точке соответствующей началу отсчёта колебания имеют фазу:
\[
    \varphi_0(t) = \varphi_0 + \omega t.
\]
Точка начала отсчёта называется фазовым центром.

Наша цель заключается в расчёте фаз во всех других точках. Если через начало координат провести прямую параллельно фронту распространения волны, то во всех точках
этой прямой фаза будет такая же.

А что делать с остальными точками? Рассмотрим волновой вектор $\vec{w}$, который направлен в сторону распространения волны перпендикулярно фронту и имеет длину
\[
    \modulus{\vec{w}}
    = \frac{\omega}{v}
    = \frac{\omega \cdot T}{v \cdot T}
    = \frac{2 \pi}{\lambda} ,
\]
где $\omega$ --- угловая скорость колебаний, $v$ --- линейная скорость распространения волны, $T$ --- период колебаний, $\lambda$ --- длина волны (расстояние,
которое проходит волна за один период).

Проведём прямую через начало координат в направлении волнового вектора и рассмотрим изменение фазы колебаний в точках прямой. Если продвинуться на расстояние
$l$ по прямой в направлении волнового вектора $\vec{w}$ до точки $A$, то фаза изменится на $2 \pi \frac{l}{\lambda}$, а если продвинуться в обратную сторону
до точки $A^\prime$, то фаза изменится на $- 2 \pi \frac{l}{\lambda}$. Таким образом, если $l$ --- расстояние со знаком от точки прямой до начала отсчёта
(положительное направление отсчёта в направлении волнового вектора), то измение фазы $\Delta \varphi(l)$ будет равно:
\[
    \Delta \varphi(l)
    = 2 \pi \frac{l}{\lambda}
    = \frac{2 \pi} {\lambda} l
    = \modulus{\vec{w}} l .
\]
Таким образом, $\modulus{\vec{w}}$ является линейный коэффициентом изменения фазы. Если через точку $A$ провести прямую параллельную фронту распространения волны,
то все точки на этой прямой будут иметь такое же изменение фазы.

Теперь понятно как вычислить изменение фазы для любой точки $B$: необходимо через точку $B$ провести прямую параллельную фронту распространения волны, найти точку
пересечения с прямой проведённой через начало координат в направлении волнового вектора и вычислить расстояние $l$ от начала отсчёта до точки пересечения. Если
точка $B$ имеет радиус-вектор $\vec{r}$, то расстояние $l$ является проекцией вектора $\vec{r}$ на направление волнового вектора $\vec{w}$:
\[
    l = \scalarproduct{\vec{r}}{\frac{\vec{w}}{\modulus{\vec{w}}}}
\]
изменение фазы:
\[
    \Delta \varphi ( \vec{r} )
    = \modulus{\vec{w}} l
    = \modulus{\vec{w}} \scalarproduct{\vec{r}}{\frac{\vec{w}}{\modulus{\vec{w}}}}
    = \scalarproduct{\vec{r}}{\modulus{\vec{w}}  \frac{\vec{w}}{\modulus{\vec{w}}}}
    = \scalarproduct{\vec{r}}{\vec{w}}
\]
и фаза в точке $B$:
\[
    \varphi(t, \vec{r})
    = \varphi_0(t) + \Delta \varphi ( \vec{r} )
    = \varphi_0 + \omega t + \scalarproduct{\vec{r}}{\vec{w}}
    = \varphi_0 + \scalarproduct{\vec{r}}{\vec{w}} + \omega t .
\]


\section{Антенная решётка}

\subsection{Один приёмник}

Если приёмник поместить в среду с электромагнитными колебаниями, то приёмник будет выделять из общего электромагнитного фона только колебания из узкой полосы вокруг
несущей частоты $\omega$ и на выходе приёмника будет наблюдаться сигнал с комплексным представлением:
\[
    v_1(t) = A(t) e^{i \varphi(t)} \cdot e^{i \omega t} ,
\]
где $A(t) e^{i \varphi(t)}$ --- комплексная огибающая.

Если рядом с первым приёмником поместить второй приёмник, то на выходе второго приёмника тоже будет наблюдаться сигнал с некоторым комплексным представлением $v_2(t)$.
Какова функция $v_2(t)$? Оказывается функция $v_2(t)$ не является произвольной и связана с функцией сигнала первого приёмника $v_1(t)$ и эта взаимосвязь
функций определяется характером распространения волны на несущей частоте $\omega$.

\subsection{Два приёмника}

Пусть имеется плоская волна с волновым вектором $\vec{w}$ в некоторой декартовой системе координат, и в этой системе местоположение первого приёмника определяется
радиус-вектором $\vec{r}_1$, а второго --- радиус-вектором $\vec{r}_2$, тогда фазы колебаний в первом и втором приёмниках $\varphi_1(t)$ и $\varphi_2(t)$:
\begin{align*}
    \varphi_1(t) & = \varphi_0(t) + \scalarproduct{\vec{r}_1}{\vec{w}} , \\
    \varphi_2(t) & = \varphi_0(t) + \scalarproduct{\vec{r}_2}{\vec{w}} ,
\end{align*}
где $\varphi_0(t)$ --- фаза колебаний в фазовом центре --- точке начала декартовой системы координат.

В целях упростить выражения для фаз поместим фазовый центр в первый приёмник и направим ось абсцисс системы координат в направлении второго приёмника. В этом случае,
$\vec{r}_1$ = 0, поэтому:
\begin{align*}
    \varphi_1(t) & = \varphi_0(t) , \\
    \varphi_2(t) & = \varphi_0(t) + \scalarproduct{\vec{r}_2}{\vec{w}} = \varphi_1(t) + \scalarproduct{\vec{r}_2}{\vec{w}} .
\end{align*}

Пусть угол между осью ординат и волновым вектором равен $\alpha$, а длина $\modulus{\vec{r}_2} = d$, тогда
\begin{gather*}
    \scalarproduct{\vec{r}_2}{\vec{w}}
    = \modulus{\vec{r}_2} \modulus{\vec{w}} \cos \left ( \frac{\pi}{2} - \alpha \right )
    = d \frac{2 \pi}{\lambda} \sin \alpha
    = 2 \pi \frac{d}{\lambda} \sin \alpha ,
\end{gather*}
поэтому фаза колебаний второго приёмника:
\begin{gather*}
    \varphi_2(t) = \varphi_1(t) + \Delta \varphi , \\
    \Delta \varphi = 2 \pi \frac{d}{\lambda} \sin \alpha .
\end{gather*}

Если на выходе первого приёмника имеется сигнал $u_1(t)$:
\[
    u_1(t)
    = A \cos \left ( \varphi_1(t) \right )
    = A \cos \left ( \varphi_0 + \omega t \right )
\]
с комплексным представлением:
\[
    v_1(t)
    = A e^{i \varphi_0} \cdot e^{i \omega t} ,
\]
то на выходе второго приёмника будет сигнал $u_2(t)$:
\[
    u_2(t)
    = A \cos \left ( \varphi_1(t) + \Delta \varphi \right )
    = A \cos \left ( \varphi_0 + \omega t + \Delta \varphi \right )
\]
с комплексным представлением:
\[
    v_2(t)
    = A e^{i \varphi_0 + \Delta \varphi } \cdot e^{i \omega t} .
\]
Таким образом, комплексные огибающие $v_1$ и $v_2$ первого и второго приёмников
\begin{align*}
    s_1 & = A e^{i \varphi_0} , \\
    s_2 & = A e^{i \varphi_0 + \Delta \varphi} = A e^{i \varphi_0} \cdot e^{i \Delta \varphi} = s_1 \cdot e^{i \Delta \varphi}
\end{align*}

Если поместить фазовый центр в точку второго приёмника, тогда комплексные огибающие будут иметь вид:
\begin{align*}
    s_1 & = A e^{i \varphi_0 - \Delta \varphi} = A e^{i \varphi_0} \cdot e^{-i \Delta \varphi} = s_2 e^{-i \Delta \varphi}, \\
    s_2 & = A e^{i \varphi_0}.
\end{align*}

Если поместить фазовый центр в середине отрезка между приёмниками, тогда комплексные огибающие будут иметь вид:
\begin{align*}
    s_1 & = A e^{i \varphi_0 - \frac{\Delta \varphi}{2}}, \\
    s_2 & = A e^{i \varphi_0 + \frac{\Delta \varphi}{2}}.
\end{align*}

\subsection{Одномерная решётка}

Продолжим помещать приёмники на оси абсцисс через равные расстояния $d$ (расстояние между первым и вторым приёмниками) и получим эквидистантную антенную решётку.
Пусть приёмник с номером $k$ (нумерация в положительном направлении оси абсцисс) имеет радиус-вектор $\vec{r}_k$, тогда фаза $\varphi_k(t)$ у приёмника
с номером $k$:
\[
    \varphi_k(t) = \varphi_1(t) + \scalarproduct{\vec{r}_k}{\vec{w}},
\]
где длина вектора $\modulus{\vec{r}_k} = (k-1) d$, поэтому скалярное произведение
\[
    \scalarproduct{\vec{r}_k}{\vec{w}}
    = \modulus{\vec{r}_k} \modulus{\vec{w}} \cos \left ( \frac{\pi}{2} - \alpha \right )
    = (k-1) d \frac{2 \pi}{\lambda} \sin \alpha
    = (k-1) 2 \pi \frac{d}{\lambda} \sin \alpha
    = (k-1) \Delta \varphi,
\]
откуда на выходе $k$-го приёмника будет сигнал $u_k(t)$:
\[
    u_k(t)
    = A \cos \left ( \varphi_1(t) + (k-1) \Delta \varphi \right )
    = A \cos \left ( \varphi_0 + \omega t + (k-1) \Delta \varphi \right )
\]
с комплексным представлением:
\[
    v_k(t)
    = A e^{i \varphi_0 + (k-1) \Delta \varphi } \cdot e^{i \omega t} .
\]
и комплексной огибающей:
\[
    s_k
    = A e^{i \varphi_0 + (k-1) \Delta \varphi }
    = A e^{i \varphi_0 } \cdot e^{i (k-1) \Delta \varphi}
    = s_1 \cdot e^{i (k-1) \Delta \varphi} .
\]

\subsection{Двумерная решётка}

Итак, мы умеем вычислять фазу приёмника относительно заданного фазового центра в двумерном случае, когда фронт волны --- прямая. А что делать в трехмерном случае, когда фронт
волны --- плоскость?

Пусть $\vec{w}$ --- волновой вектор и $\vec{r}_k$ --- местоположение $k$-го приёмника относительно фазового центра. Проведём плоскость через фазовый центр и два вектора $\vec{w}$
и $\vec{r}_k$, получим двумерный случай, поэтому фаза $\varphi_k(t)$ для $k$-го приёмника:
\[
    \varphi_k(t) = \varphi_0(t) + \scalarproduct{\vec{r}_k}{\vec{w}} .
\]

Можно было бы ограничится полученным равенством, но обычно решётки имеют регулярную структуру. В простом случае прямоугольной решётки приёмники установлены с одинаковым шагом.
Поместим фазовый центр в угол прямоугольника, ось $X$ направим вдоль одной стороны прямоугольника, а ось $Y$ --- вдоль другой. Пусть шаг установки приёмников вдоль оси $X$
равен $d_x$, а вдоль оси $Y$ --- $d_y$, тогда вектор $\vec{r}_k$ раскладывается в сумму:
\[
    \vec{r}_k = x_k d_x \vec{u}_x + y_k d_y \vec{u}_y,
\]
где $\vec{u}_x$ --- орт оси $X$, $\vec{u}_y$ --- орт оси $Y$, величины $x_k$ --- количество шагов $d_x$ вдоль оси $X$ и $y_k$ --- количество шагов $d_y$ вдоль оси $Y$
определяют местоположение приёмника. Таким образом,
\begin{multline*}
    \varphi_k(t)
    = \varphi_0(t) + \scalarproduct{x_k d_x \vec{u}_x + y_k d_y \vec{u}_y}{\vec{w}}
    = \varphi_0(t) + \scalarproduct{x_k d_x \vec{u}_x}{\vec{w}} + \scalarproduct{y_k d_y \vec{u}_y}{\vec{w}} = \\
    %
    = \varphi_0(t) + x_k d_x \scalarproduct{\vec{u}_x}{\vec{w}} + y_k d_y \scalarproduct{\vec{u}_y}{\vec{w}} = \\
    %
    = \varphi_0(t) + x_k d_x \modulus{u_x} \modulus{\vec{w}} \cos \widetilde{\alpha}_x + y_k d_y \modulus{u_y} \modulus{\vec{w}} \cos \widetilde{\alpha}_y = \\
    %
    = \varphi_0(t) + x_k d_x \frac{2 \pi}{\lambda} \cos \widetilde{\alpha}_x + y_k d_y \frac{2 \pi}{\lambda} \cos \widetilde{\alpha}_y = \\
    %
    = \varphi_0(t) + x_k 2 \pi \frac{d_x}{\lambda} \cos \widetilde{\alpha}_x + y_k 2 \pi \frac{d_y}{\lambda} \cos \widetilde{\alpha}_y,
\end{multline*}
где $\widetilde{\alpha}_x$ --- угол между осью $X$ и вектором $\vec{w}$, $\widetilde{\alpha}_y$ --- угол между осью $Y$ и вектором $\vec{w}$. Используя дополнительные углы:
\begin{align*}
    \alpha_x & = \frac{\pi}{2} - \widetilde{\alpha}_x, \\
    \alpha_y & = \frac{\pi}{2} - \widetilde{\alpha}_y,
\end{align*}
можно представить фазу $\varphi_k(t)$ в виде:
\[
    \varphi_k(t)
    = \varphi_0(t) + x_k 2 \pi \frac{d_x}{\lambda} \sin \alpha_x + y_k 2 \pi \frac{d_y}{\lambda} \sin \alpha_y .
\]

\textcolor{red}{Непонятно почему выбирают синусы.}
