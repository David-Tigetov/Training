\documentclass[a4paper,12pt]{article}
\usepackage[T1]{fontenc}
\usepackage[utf8]{inputenc}
\usepackage[english,russian]{babel}
\usepackage[margin=2cm]{geometry}
\usepackage{amsmath}

% окружение Matlab
\newenvironment{Matlab}{\par \vspace{0.2cm}}{\vspace{0.2cm} \par}
% команда внутри окружения
\newcommand{\Mcommand}[1]{\noindent \texttt{> #1} \par}
% команда в Matlab
\newcommand{\matlab}[1]{\begin{Matlab} \Mcommand{#1} \end{Matlab}}
% норма
\newcommand{\snorm}[1]{\left\| #1 \right\|_2}

\begin{document}

\title{Программа зачёта по курсу \\ "Прикладные методы линейной алгебры"}
\date{}
\author{}
\maketitle

\section{Отношение Релея}

\subsection{Задача}

\begin{enumerate}
    \item Сформируйте эрмитову матрицу $H$ размера 5 -- 10:
          \matlab{H = hermite(7, "any");}
          Найдите векторы, при которых квадратичная форма $z^* H z$ достигает наименьшего и наибольшего значений если $\snorm{z}=1$.

    \item Сформируйте произвольную эрмитову матрицу $A$ и положительную матрицу $B$ порядка 5 -- 10:
          \begin{Matlab}
              \Mcommand{A = hermite(9, "any");}
              \Mcommand{B = hermite(9, "positive");}
          \end{Matlab}
          Найдите векторы, при которых отношение Релея $\rho(z) = \frac{z^* A z}{z^* B z}$ достигает наименьшего и наибольшего значений.
\end{enumerate}

\subsection{Вопросы}

\begin{enumerate}
    \item Сопряжённый оператор.
    \item Эрмитов оператор и эрмитова матрица.
    \item Спектральное разложение эрмитовой матрицы.
    \item Квадратный корень из эрмитовой матрицы.
    \item Экстремумы квадратичной формы с эрмитовой матрицей.
    \item Экстремумы отношения Релея.
\end{enumerate}

\section{Излучение}

\subsection{Задача}

В начало отсчёта сферической системы координат помещены три излучателя, $F(\varphi, \theta)$ --- диаграмма направленности системы этих трёх излучателей:
\[
    F(\varphi, \theta)
    = \begin{pmatrix}
        f_1(\varphi, \theta) & f_2(\varphi, \theta) & f_3(\varphi, \theta)
    \end{pmatrix} ,
\]
где $\varphi$ и $\theta$ --- углы сферической системы и $f_1$, $f_2$, $f_3$ --- парциальные диаграммы направленности:
\begin{gather*}
    f_1(\varphi, \theta) =
    0.6 \cdot A \left( \varphi + \frac{\pi}{6}, \theta + \frac{\pi}{4} \right)
    \begin{pmatrix}
        e^{i \frac{\pi}{3}} \\
        e^{i \frac{\pi}{4}}
    \end{pmatrix}, \\
    %
    f_2(\varphi, \theta) =
    0.8 \cdot A \left( \varphi, \theta \right)
    \begin{pmatrix}
        e^{i \frac{\pi}{4}} \\
        e^{i \frac{\pi}{4}}
    \end{pmatrix}, \\
    %
    f_3(\varphi, \theta) =
    0.7 \cdot A \left( \varphi - \frac{\pi}{6}, \theta - \frac{\pi}{4} \right)
    \begin{pmatrix}
        e^{-i \frac{\pi}{3}} \\
        e^{-i \frac{\pi}{3}}
    \end{pmatrix} , \\
    %
    A \left( \varphi, \theta \right)
    =
    \left \{
    \begin{array}{ll}
        \cos \left( \frac{3}{2} \varphi \right) \cos \left( 2 \theta \right), & -\frac{\pi}{3} \le \varphi \le \frac{\pi}{3}, -\frac{\pi}{4} \le \theta \le \frac{\pi}{4}, \\
        0,                                                                    & \text{иначе}
    \end{array}
    \right .
\end{gather*}

\begin{enumerate}
    \item Постройте поверхности коэффициента усиления каждого излучателя по отдельности и всей системы излучателей
        в координатах углов $\varphi$, $\theta$.
    \item Найдите коэффициенты полезного действия каждого излучателя и коэффициенты пересечения диаграмм.
    \item Для направления $\varphi = \frac{\pi}{12}$, $\theta = -\frac{\pi}{6}$ найдите вектор оптимальных входных сигналов $a_{max}$,
        при котором достигается наибольший коэффициент усиления. Вычислите достигнутый коэффициент полезного действия при входных сигналах $a_{max}$.
        Постройте поверхность излучения $\snorm{F(\varphi, \theta) a_{max}}^2$ в координатах углов $\varphi$, $\theta$. 
    \item Найдите наибольший коэффициент полезного действия системы трёх излучателей и вектор оптимальных входных сигналов $a_{max}$,
        при котором достигается наибольший коэффициент полезного действия. Постройте поверхность излучения $\snorm{F(\varphi, \theta) a_{max}}^2$
        в координатах углов $\varphi$, $\theta$.
\end{enumerate}

\subsection{Вопросы}

\begin{enumerate}
    \item Комплексная огибающая.
    \item Диаграмма направленности системы излучателей.
    \item Коэффициент усиления системы излучателей. Наибольший коэффициент усиления для заданного направления.
    \item Коэффициент полезного действия системы излучателей, наименьшее и наибольшее значение коэффициента полезного действия.
\end{enumerate}
\end{document}
