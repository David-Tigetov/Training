\documentclass[a4paper,12pt]{article}
\usepackage[T1]{fontenc}
\usepackage[utf8]{inputenc}
\usepackage[english,russian]{babel}
\usepackage[margin=2cm]{geometry}
\usepackage{amsmath}

% окружение Matlab
\newenvironment{Matlab}{\par \vspace{0.2cm}}{\vspace{0.2cm} \par}
% команда внутри окружения
\newcommand{\Mcommand}[1]{\noindent \texttt{> #1} \par}
% команда в Matlab
\newcommand{\matlab}[1]{\begin{Matlab} \Mcommand{#1} \end{Matlab}}
% норма
\newcommand{\snorm}[1]{\left\| #1 \right\|_2}

\begin{document}

\title{Программа зачёта по курсу \\ "Прикладные методы линейной алгебры"}
\date{}
\author{}
\maketitle

\section{Отношение Релея}

\subsection{Задача}

\begin{enumerate}
    \item Сформируйте эрмитову матрицу $H$ размера 5 -- 10:
          \matlab{H = hermite(7, "any");}
          Найдите векторы, при которых квадратичная форма $z^* H z$ достигает наименьшего и наибольшего значений если $\snorm{z}=1$.

    \item Сформируйте произвольную эрмитову матрицу $A$ и положительную матрицу $B$ порядка 5 -- 10:
          \begin{Matlab}
              \Mcommand{A = hermite(9, "any");}
              \Mcommand{B = hermite(9, "positive");}
          \end{Matlab}
          Найдите векторы, при которых отношение Релея $\rho(z) = \frac{z^* A z}{z^* B z}$ достигает наименьшего и наибольшего значений.
\end{enumerate}

\subsection{Вопросы}

\begin{enumerate}
    \item Сопряжённый оператор.
    \item Эрмитов оператор и эрмитова матрица.
    \item Спектральное разложение эрмитовой матрицы.
    \item Квадратный корень из эрмитовой матрицы.
    \item Экстремумы квадратичной формы с эрмитовой матрицей.
    \item Экстремумы отношения Релея.
\end{enumerate}

\section{Излучение}

\subsection{Задача}

В начало отсчёта сферической системы координат помещены три излучателя, $F(\varphi, \theta)$ --- диаграмма направленности системы этих трёх излучателей:
\[
    F(\varphi, \theta)
    = \begin{pmatrix}
        f_1(\varphi, \theta) & f_2(\varphi, \theta) & f_3(\varphi, \theta)
    \end{pmatrix} ,
\]
где $\varphi$ и $\theta$ --- углы сферической системы и $f_1$, $f_2$, $f_3$ --- парциальные диаграммы направленности:
\begin{gather*}
    f_1(\varphi, \theta) =
    0.6 \cdot A \left( \varphi + \frac{\pi}{6}, \theta + \frac{\pi}{4} \right)
    \begin{pmatrix}
        e^{i \frac{\pi}{3}} \\
        e^{i \frac{\pi}{4}}
    \end{pmatrix}, \\
    %
    f_2(\varphi, \theta) =
    0.8 \cdot A \left( \varphi, \theta \right)
    \begin{pmatrix}
        e^{i \frac{\pi}{4}} \\
        e^{i \frac{\pi}{4}}
    \end{pmatrix}, \\
    %
    f_3(\varphi, \theta) =
    0.7 \cdot A \left( \varphi - \frac{\pi}{6}, \theta - \frac{\pi}{4} \right)
    \begin{pmatrix}
        e^{-i \frac{\pi}{3}} \\
        e^{-i \frac{\pi}{3}}
    \end{pmatrix} , \\
    %
    A \left( \varphi, \theta \right)
    =
    \left \{
    \begin{array}{ll}
        \cos \left( \frac{3}{2} \varphi \right) \cos \left( 2 \theta \right), & -\frac{\pi}{3} \le \varphi \le \frac{\pi}{3}, -\frac{\pi}{4} \le \theta \le \frac{\pi}{4}, \\
        0,                                                                    & \text{иначе}
    \end{array}
    \right .
\end{gather*}

\begin{enumerate}
    \item Постройте поверхности коэффициента усиления каждого излучателя по отдельности и всей системы излучателей
          в координатах углов $\varphi$, $\theta$.
    \item Найдите коэффициенты полезного действия каждого излучателя и коэффициенты пересечения диаграмм.
    \item Для направления $\varphi = \frac{\pi}{12}$, $\theta = -\frac{\pi}{6}$ найдите вектор оптимальных входных сигналов $a_{max}$,
          при котором достигается наибольший коэффициент усиления. Вычислите достигнутый коэффициент полезного действия при входных сигналах $a_{max}$.
          Постройте поверхность излучения $\snorm{F(\varphi, \theta) a_{max}}^2$ в координатах углов $\varphi$, $\theta$.
    \item Найдите наибольший коэффициент полезного действия системы трёх излучателей и вектор оптимальных входных сигналов $a_{max}$,
          при котором достигается наибольший коэффициент полезного действия. Постройте поверхность излучения $\snorm{F(\varphi, \theta) a_{max}}^2$
          в координатах углов $\varphi$, $\theta$.
\end{enumerate}

\subsection{Вопросы}

\begin{enumerate}
    \item Комплексная огибающая.
    \item Диаграмма направленности системы излучателей.
    \item Коэффициент усиления системы излучателей. Наибольший коэффициент усиления для заданного направления.
    \item Коэффициент полезного действия системы излучателей, наименьшее и наибольшее значение коэффициента полезного действия.
\end{enumerate}

\section{Помехи}

\subsection{Задача}

Задайте линейную эквидистантную решётку, выбрав произвольным образом количество, шаг установки и мощность собственных шумов приёмников.

\begin{enumerate}
    \item Напишите функцию, которая возвращает реализацию огибающих сигналов приёмников решетки при отсутствии и наличии источников излучения.
    \item Задайте проивольным образом несколько источников излучения, выбрав направления и мощности излучения. Реализуйте метод пеленгации
          всех источников излучения, который использует реализации огибающих сигналов приёмников. Примените реализованный метод пеленгации
          при отсутствии источников излучения, убедитесь в отсутствии "ложных"{} пеленгаций.
    \item Задайте проивольным образом несколько источников излучения и направление приёма полезного сигнала. Реализуйте метод
          адаптации: вычисления оптимального весового вектора для приёма полезного сигнала, на основе реализаций огибающих сигналов приёмников.
          Для вычисленного оптимального весового вектора постройте график мощности принимаемых сигналов в зависимости от направления.
          Сравните мощности выходных сигналов, получаемых от источников излучения и полезного сигнала.
\end{enumerate}


% Сформируйте линейную эквидистантную решётку, выбрав количество приёмников от 2 до 20, шаг установки приёмников от 0.1 до 0.9,
% и произвольный уровень мощности собственных шумов, используя класс Receivers:
% \matlab{receivers = Receivers(5, 0.5, 1);}
% В конструкторе Receivers первый аргумент --- количество приёмников, второй --- шаг установки приёмников, измеряемый в длинах излучаемых волн,
% третий --- мощность собственных шумов.

% С помощью объекта receivers сформируйте выборку комплексных огибающих сигналов приёмников:
% \matlab{signals = receivers.sampling(jammers, volume);}
% \noindent где jammers --- матрица параметров источников излучения, в каждой строке которой два элемента, первый --- угол на источник в градусах
% относительно нормали, второй --- мощность излучения источника. Например, матрица:
% \matlab{jammers = [-15 3; 30 7; 5 4];}
% \noindent задаёт три источника излучения, первый расположен в направлении -15 градусов мощностью 3, второй --- в направлении 30 градусов мощностью 7,
% третий --- в направлении 5 градусов мощностью 4.

% Аргумент volume задаёт объём выборки.

% В матрице signals каждый столбец представляет собой значения огибающих в приёмниках.

% Используя матрицу signals необходимо решить задачи:
% \begin{itemize}
%     \item пеленгации: определить направления всех источников излучения,
%     \item адаптации: определить оптимальный весовой вектор, выбрав произвольным образом направление приёма полезного сигнала, и построить
%           график модуля .
% \end{itemize}

% Объект receivers позволяет вычислять векторы направлений с помощью метода directions:
% \matlab{direction = receivers.directions(10);}
% \noindent аргументом которого является угол, измеряемый в градусах относительно нормали.


\subsection{Вопросы}

\begin{enumerate}
    \item Случайный вектор огибающих сигналов приёмников линейной эквидистантной решётки при наличии и отсутствии источников излучения.
        Математическое ожидание и ковариационная матрица вектора огибающих.
    \item Задача обнаружения и пеленгации одного источника излучения.
    \item Задача обнаружения и пеленгации нескольких источников излучения.
    \item Задача адаптации: выбор оптимального весового вектора для приёма полезного сигнала.
    \item Оценивание ковариационной матрицы вектора огибающих.
    \item Обращение ковариационной матрицы вектора огибающих с помощью ортогонализации.
\end{enumerate}

\section{Линейные системы}

\subsection{Задача}

Выберите произвольным образом длительность исходного дискретного сигнала и сформируйте его как сумму нескольких гармонических дискретных
сигналов. Спроектируйте и реализуйте полосовой фильтр, который выделяет набор гармоник из узкой полосы спектра, например, из нижней, верхней
или средней части. Определите импульсную характеристику фильтра и используйте её для преобразования исходного сигнала с помощью циклической
свёртки.

Постройте графики:
\begin{enumerate}
    \item исходного дискретного сигнала и его спектра,
    \item импульсной характеристики фильтра и её спектра,
    \item выходного сигнала и его спектра.
\end{enumerate}

\subsection{Вопросы}

\begin{enumerate}
    \item Дискретные сигналы и линейные стационарные системы.
    \item Декомпозиция дискретных сигналов с помощью единичных импульсов, импульсная характеристика линейной системы.
    \item Периодические дискретные сигналы, представление действия линеной системы в виде циклической свёртки. Циркулянты и свойство
        умножения циркулянта на строку степеней корня из единицы.
    \item Гармонический базис дискретных сигналов. Спектр дискретного сигнала, дискретное преобразование Фурье сигнала в спектр с помощью
        матрицы Фурье и обратное преобразование спектра в сигнал.
    \item Вычисление сигнала, полученного в результате циклической свёртки, с помощью прямого и обратного дискретных преобразований Фурье.
    \item Схема быстрого дискретного преобразования Фурье, оценка количества операций.
    \item Фильтр нижних частот.
\end{enumerate}

\end{document}
