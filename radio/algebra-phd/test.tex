\documentclass[a4paper,12pt]{article}
\usepackage[T1]{fontenc}
\usepackage[utf8]{inputenc}
\usepackage[english,russian]{babel}
\usepackage[margin=2cm]{geometry}

% окружение Matlab
\newenvironment{Matlab}{\par \vspace{0.2cm}}{\vspace{0.2cm} \par}
% команда внутри окружения
\newcommand{\Mcommand}[1]{\noindent \texttt{> #1} \par}
% команда в Matlab
\newcommand{\matlab}[1]{\begin{Matlab} \Mcommand{#1} \end{Matlab}}

\begin{document}

\title{Программа зачёта по курсу \\ "Прикладные методы линейной алгебры"}
\date{}
\author{}
\maketitle

\section{Отношение Релея}

\subsection{Задача}

\begin{enumerate}
    \item Сформируйте эрмитову матрицу $H$ размера 5 -- 10:
        \matlab{H = hermite(7, "any");}
        Найдите векторы, при которых квадратичная форма $z^* H z$ достигает наименьшего и наибольшего значений.

    \item Сформируйте произвольную эрмитову матрицу $A$ и положительную матрицу $B$ порядка 5 -- 10:
    \begin{Matlab}
        \Mcommand{A = hermite(9, "any");}
        \Mcommand{B = hermite(9, "positive");}
    \end{Matlab}
        Найдите векторы, при которых отношение Релея $\rho(z) = \frac{z^* A z}{z^* B z}$ достигает наименьшего и наибольшего значений.
\end{enumerate}

\subsection{Вопросы}

\begin{enumerate}
    \item Сопряжённый оператор.
    \item Эрмитов оператор и эрмитова матрица.
    \item Спектральное разложение эрмитовой матрицы.
    \item Квадратный корень из эрмитовой матрицы.
    \item Экстремумы квадратичной формы с эрмитовой матрицей.
    \item Экстремумы отношения Релея.
\end{enumerate}
\end{document}
