\section{Как это работает?}

Радиолокация --- область науки и техники, объединяющая методы и средства измерения координат объектов и определения их свойств
с помощью радиоволн.

Радиолокационная станция --- это измерительный прибор (что-то вроде рулетки, только немного сложнее).

В качестве примера будем рассматривать радиолокацию на плоскости: радиолокационная станция находится в начале полярной системы координат,
в некоторой точке $A$ с координатами $(\varphi_A, \rho_A)$ находится объект.

\subsection{Излучение и приём сигналов}

Станция излучает сигнал в направлении, соответствующем углу $\varphi_A$. Сигнал доходит до объекта и отражается обратно в сторону станции,
которая засекает время $\Delta t$ между отправкой сигнала и приёмом отражённого сигнала (ещё его называют эхо-сигналом). За время $\Delta t$ сигнал должен
пройти до объекта расстояние $\rho_A$ и вернуться обратно, проделав такое же расстояние $\rho_A$. Учитывая, что сигналы распространяются со
скоростью света $c$:
\begin{gather*}
    2 \rho_A = c \Delta t , \\
    \rho_A = \frac{c \Delta t}{2} .
\end{gather*}
Станция смогла вычислить расстояние $\rho_A$ до объекта $A$ и знает направление, в котором излучался сигнал $\varphi_A$.

Если объект находится далеко, то отраженный сигнал оказывается очень слабым, поэтому станции требуется излучать несколько сигналов в одном
направлении, их называют \term{пачкой}, принимать несколько отраженных сигналов и специальным образом их накапливать (складывать).

\subsection{Цифровая обработка сигналов (ЦОС)}

Обработку отраженных сигналов выполняют специальными методами, которые позволяют сфорировать набор \term{обнаружений} (приближенных
координат объекта), набор которых образует "облако"{} вокруг объекта.

\subsection{Первичная обработка информации (ПОИ)}

"Облако"{} обнаружений стягиватся в одну точку, которая называется \term{координатной точкой}. Поскольку обнаружения содержат
приближенные координаты объекта, то и координатная точка определяется с ошибкой относительно расположения объекта, поэтому
правильно говорить, что объект находится не в точке, а в некоторой области, центр которой --- координатная точка, а размеры определяются
величинами ошибок. Станция мыслит областями (объёмами) пространства, в которых с некоторой вероятностью находится объект.

\subsection{Вторичная обработка информации (ВОИ)}

Если объект движется, то в последующие моменты времени станции нужно искать его уже в другом месте, и для того чтобы понять в каком
именно, нужно следить за объектом (сопровождать), анализировать его движение в предшествующий период времени и формировать наиболее
вероятную область его нахождения в будущем. При анализе координатные точки связываются в \term{траекторию} (\term{трассу}) движения
объекта.

\subsection{Обработка}

Вся обработка сводится к нескольким последовательно выполняемым шагам, представленным на рисунке

\begin{figure}[!h]
    \centering
    \begin{tikzpicture}[stage/.style = {draw, rectangle}]
        \node[stage] (1) {Излучение и приём};
        \node[stage] (2) [below of=1] {ЦОС};
        \node[stage] (3) [below of=2] {ПОИ};
        \node[stage] (4) [below of=3] {ВОИ};

        \draw [->] (1) -- node[right] {отраженные сигналы} (2);
        \draw [->] (2) -- node [right] {обнаружения} (3);
        \draw [->] (3) -- node [right] {координатные точки} (4);
    \end{tikzpicture}
    \caption{Обработка информации.}
\end{figure}

\subsection{Виды станций}

Вообще говоря, станция не знает в каком направлении находится объект, поэтому перебирает возможные направления в секторе,
такие станции называются \term{секторными}, или по кругу, тогда их называют станциями \term{кругового обзора}.

Так работают \term{импульсные} станции, а способ локации называется \term{активным}, поскольку станция сама излучает сигнал. Есть
и \term{пассивная} локация, когда станция не излучает сигналы, а только принимает, например, когда объет сам является источником сигнала,
или когда его облучает другая станция, такие системы из нескольких станций называют \term{многопозиционными}.
