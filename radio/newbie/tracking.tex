\section{Вторичная обработка информации}

\subsection{Модели}

Для сопровождения объектов нужно иметь представление о том:
\begin{itemize}
    \item как движется объект --- \term{модель движения},
    \item что измеряет станция --- \term{модель измерений}.
\end{itemize}

Основная задача вторичной обработки --- определить движение по измерениям.

Пусть состояние объекта во времени описывается функцией $\xi(t)$, широко распространены две модели движения:
\begin{itemize}
    \item $\xi(t)$ --- детерминированная функция,
    \item $\xi(t)$ --- случайная функция.
\end{itemize}

Модель измерений --- набор случайных величин $\eta = (\eta_1, \dots, \eta_n)$ --- как правило, координат объекта с ошибками, полученными
в моменты времени $\tau = (\tau_1, \dots, \tau_n)$.

\subsection{Методы}

\subsubsection{МП-оценка}

Модель движения: $\xi(t) = \mathcal{F}(t; \theta)$, где $\mathcal{F}$ --- известная функция, $\theta$ --- неизвестный параметр.

Модель измерений: условное распределение $p_\eta(y; \theta)$ величин $\eta$.

Оценка \term{максимального правдоподобия}:
\[
    \widehat{\theta} = arg \sup \limits_\theta p_\eta(\eta; \theta) .
\]

\subsubsection{МНК-оценка}

Модель движения: $\xi(t) = \mathcal{F}(t; \theta)$, где $\mathcal{F}$ --- известная функция, $\theta$ --- неизвестный параметр.

Модель измерений: $\eta = \mathcal{F}(\tau; \theta) + \varepsilon$.

\term{МНК-оценка}:
\begin{gather*}
    \widehat{\theta} = arg \sup \limits_\theta \Phi(\eta; \theta) , \\
    \Phi(\eta; \theta) = \Anorm{\eta - \mathcal{F}(\tau, \theta)}{K_\varepsilon^{-1}}^2 ,
\end{gather*}
$K_\varepsilon$ --- ковариация $\varepsilon$.

\subsubsection{Байесовские оценки}

Модель движения: $\xi(t) = \mathcal{F}(t; \theta)$, $\mathcal{F}$ --- известная функция, $\theta$ --- параметр с распределением $p_\theta(u)$.

Модель измерений: $p_\eta(y; \theta)$ --- условное распределение величин $\eta$.

\term{Байесовская оценка}:
\[
    \widehat{\theta} = arg \inf \limits_{\widehat{u}} \int R(u, \widehat{u}) p_\eta(\eta; u) p_\theta(u) du ,
\]
c функцией штрафа $R(u, \widehat{u})$.
\begin{itemize}
    \item $R(u,\widehat{u}) = (u - \widehat{u})^2$, $\widehat{\theta}$ --- условное математическое ожидание $\theta$ относительно $\eta$,
    \item $R(u,\widehat{u}) = \delta(u - \widehat{u})$, $\widehat{\theta}$ --- мода апостериорного распределения,
    \item $R(u,\widehat{u}) = \modulus{u - \widehat{u}}$, $\widehat{\theta}$ --- медиана апостериорного распределения.
\end{itemize}

\subsubsection{Линейная оценка}

Модель движения: $\xi(t) = \xi$.

Модель измерений: $\eta$.

\term{Линейная оценка}:
\begin{gather*}
    \widehat{\xi} = arg \min \limits_{\xi^* \sim \linearspan{\eta}} \norm{\xi - \xi^*}^2 , \\
    \widehat{\xi} = \expectation{\xi} + K_{\xi,\eta} K_{\eta,\eta}^{-1} \left ( \eta - \expectation{\eta} \right ) , \\
    K_{\xi,\eta} = \covariance{\xi}{\eta} , \\
    K_{\eta,\eta} = \covariance{\eta}{\eta} .
\end{gather*}

\subsubsection{Линейный фильтр}

Модель движения: $\xi_1$, $\xi_2$, \dots --- последовательность случайных величин.

Модель измерений: $\eta_1$, $\eta_2$, \dots --- последовательность случайных величин.

Линейный фильтр вычисляет оптимальную оценку $\widehat{\xi}_{k+1}$ по измерениям $\eta_1$, \dots, $\eta_{k+1}$.

Прогноз состояния $\xi_{k+1}^*$:
\begin{gather*}
    \xi_{k+1}^* = \expectation{\xi_{k+1}} + R_{k+1} \overline{Q}_k^{-1} \left ( \overline{\eta}_k - \expectation{\overline{\eta}_k}\right ) , \\
    R_{k+1} = \covariance{\xi_{k+1}}{\overline{\eta}_k} ,
\end{gather*}
и его ошибка $\widetilde{\xi}_{k+1}$:
\begin{gather*}
    \widetilde{\xi}_{k+1}^* = \xi_{k+1} - \xi_{k+1}^* , \\
    \widetilde{P}_{k+1}^* = P_{k+1} - R_{k+1} \overline{Q}_k^{-1} R_{k+1}^T , \\
    \widetilde{P}_{k+1}^* = \covariance{\widetilde{\xi}_{k+1}^*}{\widetilde{\xi}_{k+1}^*} , \\
    P_{k+1} = \covariance{\xi_{k+1}}{\xi_{k+1}} .
\end{gather*}

Прогноз измерения $\eta_{k+1}^*$:
\begin{gather*}
    \eta_{k+1}^* = \expectation{\eta_{k+1}} + S_{k+1} \overline{Q}_k^{-1} \left ( \overline{\eta}_k - \expectation{\overline{\eta}_k} \right ) , \\
    \overline{\eta}_k = \left ( \eta_1, \dots, \eta_k \right ) , \\
    \overline{Q}_k = \covariance{\overline{\eta}_k}{\overline{\eta}_k} , \\
    S_{k+1} = \covariance{\eta_{k+1}}{\overline{\eta}_k} ,
\end{gather*}
и его ошибка:
\begin{gather*}
    \widetilde{\eta}_{k+1}^* = \eta_{k+1} - \eta_{k+1}^* , \\
    \widetilde{Q}_{k+1}^* = Q_{k+1} - S_{k+1} \overline{Q}_k^{-1} S_{k+1}^T , \\
    \widetilde{Q}_{k+1}^* = \covariance{\widetilde{\eta}_{k+1}^*}{\widetilde{\eta}_{k+1}^*} , \\
    Q_{k+1} = \covariance{\eta_{k+1}}{\eta_{k+1}} .
\end{gather*}

Оценка:
\begin{gather*}
    \widehat{\xi}_{k+1} = \xi_{k+1}^* + \widetilde{R}_{k+1}^* \widetilde{Q}_{k+1}^* \widetilde{\eta}_{k+1} , \\
    \widetilde{R}_{k+1}^* = \covariance{\xi_{k+1}}{\widetilde{\eta}_{k+1}^*}
\end{gather*}
и ошибка оценки:
\begin{gather*}
    \widetilde{\xi}_{k+1} = \xi_{k+1} - \widehat{\xi}_{k+1} , \\
    \widetilde{P}_{k+1} = \widetilde{P}_{k+1}^* - \widetilde{R}_{k+1}^* \widetilde{Q}_{k+1}^* \transposition{\widetilde{R}_{k+1}^*}, \\
    \widetilde{P}_{k+1} = \covariance{\widetilde{\xi}_{k+1}}{\widetilde{\xi}_{k+1}} , \\
\end{gather*}

\subsubsection{Фильтр Калмана}

Модель движения: $\xi_{k+1} = F_k \xi_k + G_k u_k + v_k$, $u_k$ --- детерминированые, $v_k$ --- случайные.

Модель измерений: $\eta_{k} = H_k \xi_k + w_k$, $w_k$ --- случайные.

Частный случай линейного фильтра.

Прогноз состояния:
\begin{gather*}
    \xi_{k+1}^* = F_k \widehat{\xi}_k + G_k u_k , \\
    \widetilde{P}_{k+1}^* = F_k \widetilde{P}_k F_k^T + V_k , \\
    V_k = \covariance{v_k}{v_k} .
\end{gather*}

Прогноз измерения:
\begin{gather*}
    \eta_{k+1}^* = H_{k+1} \xi_{k+1}^* , \\
    \widetilde{Q}_{k+1}^* = H_{k+1} \widetilde{P}_{k+1}^* H_{k+1}^T + W_{k+1} , \\
    W_{k+1} = \covariance{w_k}{w_k} .
\end{gather*}

Оценка:
\begin{gather*}
    \widehat{\xi}_{k+1} = \xi_{k+1}^* + \widetilde{P}_{k+1}^* H_{k+1}^T \widetilde{Q}_{k+1}^* \widetilde{\eta}_{k+1}, \\
    \widetilde{P}_{k+1} = \widetilde{P}_{k+1}^* - \widetilde{P}_{k+1}^* H_{k+1}^T \inversion{\widetilde{Q}_{k+1}^*} H_{k+1} \widetilde{P}_{k+1}^* .
\end{gather*}
