\chapter{Точность пеленгации источника сигнала}
\section{Постановка}

Состояние системы приёмников во времени описывается векторным процессом $x(t)$:
\[
    x(t) = e(t) + s_1(t) \breve{x}_1(t) + \dots + s_m(t) \breve{x}_m(t) ,
\]
где $e(t)$ --- вектор шумов, $s_k(t)$ --- огибающие, принимаемых сигналов, и $\breve{x}_k(t)$ --- векторы направлений,
которые определяются известной функцией фазовых набегов $\mathcal{F}(u,v)$, где $u$ и $v$ --- углы выбранной системы:
\[
    \breve{x}_k(t) = \mathcal{F}(u_k(t), v_k(t)) .
\]
В результате наблюдения процесса $X(t)$ в дискретные моменты времени $t_1$, \dots, $t_n$ формируются величины:
\[
    X_i = X(t_i) .
\]
Необходимо по величинам $X_i$ определить направления $(u_k(t), v_k(t))$.

\section{Один постоянный сигнал}

Пусть принимается один сигнал, параметры которого не изменяются во времени:
\[
    X(t)
    = E(t) + s_1 \breve{X}_1
    = E(t) + s_1 \mathcal{F}(u_1, v_1) .
\]
По наблюдениям
\begin{gather*}
    X_i = E_i + s_1 \mathcal{F}(u_1, v_1), \\
    E_i = E(t_i)
\end{gather*}
необходимо оценить неизвестные $s_1$, $u_1$, $v_1$.

При оценке по методу наименьших квадратов все векторы $X_i$ соберем в один вектор $Y$, введем вектор сигнала $\mathcal{D}$:
\[
    Y =
    \begin{pmatrix}
        X_1    \\
        X_2    \\
        \vdots \\
        X_n
    \end{pmatrix}
    , \;
    \mathcal{D}(s, u, v) =
    \begin{pmatrix}
        s \mathcal{F}(u, v) \\
        s \mathcal{F}(u, v) \\
        \vdots              \\
        s \mathcal{F}(u, v)
    \end{pmatrix}
\]
и определим функционал отклонения:
\[
    \Phi(s,u,v)
    = \norm{Y - \mathcal{D}(s,u,v)}{R^{-1}}^2 ,
\]
где $R$ --- ковариационная матрица вектора $Y$:
\[
    R
    = \covariance{Y}{Y}
    = \covariance{E}{E}
\]

При линеаризации задачи в окрестности точки $(s_1, u_1, v_1)$:
\[
    \widetilde{\Phi}(s,u,v)
    = \norm{Y - \mathcal{D}(s_1,u_1,v_1) - \jacobi{D}{s,u,v} \Delta \theta}{R^{-1}}^2
    = \norm{E - \jacobi{D}{s,u,v} \Delta \theta}{R^{-1}}^2 ,
\]
смещение $\Delta \theta$ является проекцией вектора ошибок $E$:
\[
    \Delta \theta = \left( \jacobi{D}{s,u,v}^T R^{-1}\jacobi{D}{s,u,v} \right)^{-1} \jacobi{D}{s,u,v}^T R^{-\frac{1}{2}} E ,
\]
откуда ковариация ошибки:
\[
    \variance{\Delta \theta} = \left( \jacobi{D}{s,u,v}^T R^{-1}\jacobi{D}{s,u,v} \right)^{-1} .
\]
