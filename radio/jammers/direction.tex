\chapter{Направление приёма}

\section{Центральноэрмитовы матрицы}

По материалам статьи \cite{Lee}.

\subsection{Определение}

Комплексная матрица $A=[a_{jk}]$ порядка $n \times m$ называется центральноэрмитовой, если
\begin{gather}
    a_{n+1-j, m+1-k} = \overline{a}_{jk},
    \label{direction:centralhermitian:elementwise} \\
    1 \le j \le n,
    \notag \\
    1 \le k \le m.
    \notag
\end{gather}
где $\overline{a}$ обозначает комплексное сопряжение $a$. У центральноэрмитовой матрицы элементы сопряжены относительно центра.

Пусть $P_n$ --- матрица порядка $n$ с единицами на побочной диагонали. С помощью матрицы $P_n$ поэлементное равенство
\eqref{direction:centralhermitian:elementwise} можно представить в виде:
\begin{equation}~\label{direction:centralhermitian:matrix}
    P_n A P_m = \overline{A} .
\end{equation}
Умножение на матрицу $P_n$ слева переставляет строки в обратном порядке, умножение на матрицу $P_m$ справа переставляет столбцы в
обратном порядке. Центральная симметрия --- это отражение по горизонтали и вертикали, проходящих через центр матрицы.

Заметим, что:
\begin{equation}~\label{direction:V:idempotence}
    P_n P_n = I_n ,
\end{equation}
где $I_n$ --- единичная матрица порядка $n$. Отсюда следует, что
\begin{equation}~\label{direction:V:inversion}
    P_n = P_n^{-1}
\end{equation}

\subsection{Левые $P$-действительные матрицы}

Матрица $Q$ называется левой $P$--действительной, если
\[
    P Q = \overline{Q}.
\]
Такие матрицы существуют, например при чётном порядке $2n$:
\begin{equation}~\label{direction:Q:even}
    Q_{2n}
    = \frac{1}{\sqrt{2}}
    \begin{pmatrix}
        I_n & i I_n  \\
        P_n & -i P_n
    \end{pmatrix}
    =  \frac{1}{\sqrt{2}}
    \begin{pmatrix}
        1      & \dots  & 0      & i      & \dots  & 0      \\
        \vdots & \ddots & \vdots & \vdots & \ddots & \vdots \\
        0      & \dots  & 1      & 0      & \dots  & i      \\
        0      & \dots  & 1      & 0      & \dots  & -i     \\
        \vdots & \ddots & \vdots & \vdots & \ddots & \vdots \\
        1      & \dots  & 0      & -i     & \dots  & 0
    \end{pmatrix} ,
\end{equation}
поскольку с учётом равенства \eqref{direction:V:idempotence}:
\begin{multline*}
    P_{2n} Q_{2n}
    = P_{2n}
    \frac{1}{\sqrt{2}}
    \begin{pmatrix}
        I_n & i I_n  \\
        P_n & -i P_n
    \end{pmatrix}
    = \frac{1}{\sqrt{2}}
    P_{2n}
    \begin{pmatrix}
        I_n & i I_n  \\
        P_n & -i P_n
    \end{pmatrix}
    = \frac{1}{\sqrt{2}}
    \begin{pmatrix}
        0   & P_n \\
        P_n & 0
    \end{pmatrix}
    \begin{pmatrix}
        I_n & i I_n  \\
        P_n & -i P_n
    \end{pmatrix} = \\
    %
    = \frac{1}{\sqrt{2}}
    \begin{pmatrix}
        P_n P_n & - i P_n P_n \\
        P_n I_n & i P_n I_n
    \end{pmatrix}
    = \frac{1}{\sqrt{2}}
    \begin{pmatrix}
        I_n & - i I_n \\
        P_n & i P_n
    \end{pmatrix}
    = \overline{
        \frac{1}{\sqrt{2}}
        \begin{pmatrix}
            I_n & i I_n  \\
            P_n & -i P_n
        \end{pmatrix}
    }
    = \overline{Q}_{2n} ,
\end{multline*}

При нечётном порядке $2n + 1$, примером является матрица:
\begin{equation}~\label{direction:Q:odd}
    Q_{2n+1}
    = \frac{1}{\sqrt{2}}
    \begin{pmatrix}
        I_n & 0        & i I_n   \\
        0   & \sqrt{2} & 0       \\
        P_n & 0        & - i P_n
    \end{pmatrix}
    =  \frac{1}{\sqrt{2}}
    \begin{pmatrix}
        1      & \dots  & 0      & 0        & i      & \dots  & 0      \\
        \vdots & \ddots & \vdots & \vdots   & \vdots & \ddots & \vdots \\
        0      & \dots  & 1      & 0        & 0      & \dots  & i      \\
        0      & \dots  & 0      & \sqrt{2} & 0      & \dots  & 0      \\
        0      & \dots  & 1      & 0        & 0      & \dots  & -i     \\
        \vdots & \ddots & \vdots & \vdots   & \vdots & \ddots & \vdots \\
        1      & \dots  & 0      & 0        & -i     & \dots  & 0
    \end{pmatrix} ,
\end{equation}
поскольку
\begin{multline*}
    P_{2n+1} Q_{2n + 1}
    = P_{2n+1}
    \frac{1}{\sqrt{2}}
    \begin{pmatrix}
        I_n & 0        & i I_n   \\
        0   & \sqrt{2} & 0       \\
        P_n & 0        & - i P_n
    \end{pmatrix}
    = \frac{1}{\sqrt{2}}
    P_{2n+1}
    \begin{pmatrix}
        I_n & 0        & i I_n   \\
        0   & \sqrt{2} & 0       \\
        P_n & 0        & - i P_n
    \end{pmatrix} = \\
    %
    = \frac{1}{\sqrt{2}}
    \begin{pmatrix}
        0   & 0 & P_n \\
        0   & 1 & 0   \\
        P_n & 0 & 0
    \end{pmatrix}
    \begin{pmatrix}
        I_n & 0        & i I_n   \\
        0   & \sqrt{2} & 0       \\
        P_n & 0        & - i P_n
    \end{pmatrix}
    = \frac{1}{\sqrt{2}}
    \begin{pmatrix}
        P_n P_n & 0        & - i P_n P_n \\
        0       & \sqrt{2} & 0           \\
        P_n I_n & 0        & i P_n I_n
    \end{pmatrix} = \\
    %
    = \frac{1}{\sqrt{2}}
    \begin{pmatrix}
        I_n & 0        & - i I_n \\
        0   & \sqrt{2} & 0       \\
        P_n & 0        & i P_n
    \end{pmatrix}
    = \overline{
        \frac{1}{\sqrt{2}}
        \begin{pmatrix}
            I_n & 0        & i I_n   \\
            0   & \sqrt{2} & 0       \\
            P_n & 0        & - i P_n
        \end{pmatrix}
    }
    = \overline{Q}_{2n+1} ,
\end{multline*}
опять же в силу равенства \eqref{direction:V:idempotence}.

Введённые матрицы являются унитарными:
\begin{multline*}
    Q_{2n} Q_{2n}^*
    = \frac{1}{\sqrt{2}}
    \begin{pmatrix}
        I_n & i I_n   \\
        P_n & - i P_n
    \end{pmatrix}
    \frac{1}{\sqrt{2}}
    \begin{pmatrix}
        I_n     & P_n   \\
        - i I_n & i P_n
    \end{pmatrix}
    = \frac{1}{2}
    \begin{pmatrix}
        I_n I_n - i^2 I_n I_n & I_n P_n + i^2 I_n P_n \\
        P_n I_n + i^2 P_n I_n & P_n P_n - i^2 P_n P_n
    \end{pmatrix} = \\
    %
    = \frac{1}{2}
    \begin{pmatrix}
        I_n + I_n & P_n - P_n  \\
        P_n - P_n & I_n  + I_n
    \end{pmatrix}
    = \frac{1}{2}
    \begin{pmatrix}
        2 I_n & 0     \\
        0     & 2 I_n
    \end{pmatrix}
    = \begin{pmatrix}
        I_n & 0   \\
        0   & I_n
    \end{pmatrix} .
\end{multline*}

\begin{multline*}
    Q_{2n+1} Q_{2n+1}^*
    = \frac{1}{\sqrt{2}}
    \begin{pmatrix}
        I_n & 0        & i I_n   \\
        0   & \sqrt{2} & 0       \\
        P_n & 0        & - i P_n
    \end{pmatrix}
    \frac{1}{\sqrt{2}}
    \begin{pmatrix}
        I_n    & 0        & P_n   \\
        0      & \sqrt{2} & 0     \\
        -i I_n & 0        & i P_n
    \end{pmatrix} = \\
    %
    = \frac{1}{2}
    \begin{pmatrix}
        I_n I_n - i^2 I_n I_n & 0 & I_n P_n + i^2 I_n P_n \\
        0                     & 2 & 0                     \\
        P_n I_n + i^2 P_n I_n & 0 & P_n P_n - i^2 P_n P_n
    \end{pmatrix}
    = \frac{1}{2}
    \begin{pmatrix}
        I_n + I_n & 0 & P_n - P_n \\
        0         & 2 & 0         \\
        P_n - P_n & 0 & I_n + I_n
    \end{pmatrix} = \\
    %
    = \frac{1}{2}
    \begin{pmatrix}
        2 I_n & 0 & 0     \\
        0     & 2 & 0     \\
        0     & 0 & 2 I_n
    \end{pmatrix}
    = \begin{pmatrix}
        I_n & 0 & 0   \\
        0   & 1 & 0   \\
        0   & 0 & I_n
    \end{pmatrix} .
\end{multline*}


\subsection{Отражение на действительные матрицы}

Пусть
\begin{itemize}
    \item[] $\mathbb{C}_{n \times m}$ --- множество всех комплексных матриц порядка $n \times m$,
    \item[] $\mathbb{R}_{n \times m}$ --- множество всех действительных матриц порядка $n \times m$,
    \item[] $CH_{n \times m}$ --- множество всех центральноэрмитовых матриц.
\end{itemize}

Пусть $T_n$ и $U_m$ --- невырожденные, левые $P$--действительные, тогда соответствие
$\varphi : \mathbb{C}_{n \times m} \rightarrow \mathbb{R}_{n \times m} $:
\[
    \varphi(A) = T_n^{-1} A U_m .
\]
является отображением на подмножестве $CH_{n \times m} \subseteq \mathbb{C}_{n \times m}$.

Пусть $A \in CH_{n \times m}$ центральноэрмитовая, покажем, что матрица $\varphi(A)$ является действительной, что эквивалентно условию:
\[
    \overline{\varphi(A)} = \varphi(A) .
\]
Поскольку $A$ --- центральноэрмитовая матрица, то используя определение \eqref{direction:centralhermitian:matrix}, получим:
\[
    \overline{\varphi(A)}
    = \overline{T_n^{-1} A U_m}
    = \overline{T}_n^{-1} \cdot \overline{A} \cdot \overline{U}_m
    = \overline{T}_n^{-1} P_n A P_m \overline{U}_m .
\]
Матрица $T_n$ левая $P$--действительная и невырожденная:
\begin{gather*}
    P_n T_n = \overline{T}_n , \\
    P_n \overline{T_n} = T_n , \\
    \left ( P_n \overline{T_n} \right)^{-1} = T_n^{-1} , \\
    \overline{T}_n^{-1} P_n^{-1} = T_n^{-1} , \\
    \overline{T}_n^{-1} P_n = T_n^{-1} ,
\end{gather*}
матрица $U_m$ левая $P$--действительная, согласно определению:
\begin{gather*}
    P_m U_m = \overline{U}_m , \\
    P_m \overline{U_m} = U_m ,
\end{gather*}
тогда
\[
    \overline{\varphi(A)}
    = T_n^{-1} A U_m
    = \varphi(A) .
\]

Поскольку $T_n$ и $U_m$ невырожденные, то существует обратное соответствие:
\[
    \varphi^{-1}(R) = T_n R U_m^{-1} .
\]

Пусть $R \in \mathbb{R}_{n \times m}$ действительная матрица, покажем, что $\varphi^{-1}(R)$ центральноэрмитова, рассмотрим:
\[
    P_n \varphi^{-1}(R) P_m
    = P_n T_n R U_m^{-1} P_m ,
\]
где $T_n$ левая $P$--действительная:
\[
    P_n T_n = \overline{T}_n ,
\]
а матрица $U_m$ левая $P$--действительная:
\begin{gather*}
    P_m U_m = \overline{U}_m , \\
    \left( P_m U_m \right)^{-1} = \overline{U}_m^{-1} , \\
    U_m^{-1} P_m^{-1}  = \overline{U}_m^{-1} , \\
    U_m^{-1} P_m  = \overline{U}_m^{-1} ,
\end{gather*}
тогда:
\[
    P_n \varphi^{-1}(R) P_m
    = \overline{T}_n R \overline{U}_m^{-1}
    = \overline{T}_n \cdot \overline{R} \cdot \overline{U}_m^{-1}
    = \overline{T_n R U_m^{-1}}
    = \overline{\varphi^{-1}(R)}
\]
и матрица $\varphi^{-1}(R)$ является центральноэрмитовой в силу определения \eqref{direction:centralhermitian:matrix}.

\subsection{Подобие}

Если $n = m$, то в качестве $U_m$ можно взять $T_n$, опуская индекс $n$, получим:
\[
    \varphi(A) = T^{-1} A T ,
\]
тогда $\varphi$ --- подобие (сохраняет собственные и сингулярные числа).

В качестве $T$ можно взять матрицы $Q$ \eqref{direction:Q:even}, \eqref{direction:Q:odd}:
\[
    \varphi(A) = Q^{-1} A Q .
\]
Поскольку матрицы $Q$ являются унитарными, то:
\[
    \varphi(A) = Q^* A Q .
\]

\subsection{SVD--разложение}

Пусть вычислено SVD--разложение матрицы $\varphi(A)$:
\begin{gather*}
    Q^* A Q = \varphi(A) = U \Sigma V^* , \\
    Q^* A Q = U \Sigma V^* , \\
    A = Q U \Sigma V^* Q^* , \\
    A = \left( Q U \right ) \Sigma \left( Q V \right)^* .
\end{gather*}
Получили SVD--разложение центральноэрмитовой матрицы $A$.

Исходная матрица $\varphi(A) \in \mathbb{R}_{n \times m}$ действительная, поэтому для неё быстрее вычислить SVD--разложение, чем для матрицы $A$.

\section{Направление приёма в линейной решётке}

По материалам статьи \cite{Cao_Liu}.

Пусть $X$ --- случайный вектор огибающих в приёмниках и $R$ --- ковариационная матрица вектора $X$:
\[
    R = \sigma_0^2 I + \breve{X} \variance{S} \breve{X}^* ,
\]
где $S$ --- диагональная матрица:
\[
    \variance{S}
    = \begin{pmatrix}
        d_1^2  & \dots  & 0      \\
        \vdots & \ddots & \vdots \\
        0      & \dots  & d_m^2
    \end{pmatrix} ,
\]
и $\breve{X}$ --- матрица векторов направлений:
\begin{gather*}
    \breve{X}
    = \begin{pmatrix}
        \breve{X}_1 & \breve{X}_2 & \dots & \breve{X}_m
    \end{pmatrix} ,
    \notag \\
    %
    \breve{X}_1
    = \begin{pmatrix}
        1             \\
        \varphi_1     \\
        \varphi^2     \\
        \vdots        \\
        \varphi^{n-2} \\
        \varphi^{n-1}
    \end{pmatrix}
    ,
    \breve{X}_2
    = \begin{pmatrix}
        1               \\
        \varphi_2       \\
        \varphi_2^2     \\
        \vdots          \\
        \varphi_2^{n-2} \\
        \varphi_2^{n-1}
    \end{pmatrix}
    ,
    \dots
    ,
    \breve{X}_m
    = \begin{pmatrix}
        1               \\
        \varphi_m       \\
        \varphi_m^2     \\
        \vdots          \\
        \varphi_m^{n-2} \\
        \varphi_m^{n-1}
    \end{pmatrix} , \\
    %
    \varphi_k = e^{i \Delta \varphi_k} .
\end{gather*}
Будем считать комплексные сдвиги $\varphi_k$ различными:
\begin{gather*}
    \varphi_k \neq \varphi_j , \\
    k \neq j .
\end{gather*}
Для комплексных сдвигов:
\begin{gather*}
    \overline{\varphi_k}
    = \overline{e^{i \Delta \varphi_k}}
    = e^{- i \Delta \varphi_k}
    = \left( e^{i \Delta \varphi_k} \right)^{-1}
    = \varphi_k^{-1} , \\
    %
    \overline{\varphi_k^p}
    = \overline{\varphi_k \cdot ... \cdot \varphi_k}
    = \overline{\varphi_k} \cdot ... \cdot \overline{\varphi_k}
    = \varphi_k^{-1} \cdot ... \cdot \varphi_k^{-1}
    = \varphi_k^{-p} .
\end{gather*}

\subsection{Центральноэрмитовость}

Матрица $R$ является центральноэрмитовой в силу особого вида векторов направлений $\breve{X}$. Покажем справедливость определения
\eqref{direction:centralhermitian:matrix}:
\[
    P R P = \overline{R} .
\]
Рассмотрим правую часть
\begin{multline*}
    P R P
    = P \left ( \sigma_0^2 I + \breve{X} S \breve{X}^* \right ) P
    = \sigma_0^2 P P + P \breve{X} S \breve{X}^* P
    = \sigma_0^2 I + P \breve{X} S \left( P \breve{X} \right)^*,
\end{multline*}
где
\[
    P \breve{X}
    = \begin{pmatrix}
        P \breve{X}_1 & P \breve{X}_2 & \dots & P \breve{X}_m
    \end{pmatrix}
\]
и каждый столбец
\[
    P \breve{X}_k
    = P \begin{pmatrix}
        1               \\
        \varphi_k       \\
        \varphi_k^2     \\
        \vdots          \\
        \varphi_k^{n-2} \\
        \varphi_k^{n-1}
    \end{pmatrix}
    = \begin{pmatrix}
        \varphi_k^{n-1} \\
        \varphi_k^{n-2} \\
        \vdots          \\
        \varphi_k^2     \\
        \varphi_k       \\
        1
    \end{pmatrix}
    = \varphi_k^{n-1}
    \begin{pmatrix}
        1                  \\
        \varphi_k^{-1}     \\
        \vdots             \\
        \varphi_k^{-(n-3)} \\
        \varphi_k^{-(n-2)} \\
        \varphi_k^{-(n-1)} \\
    \end{pmatrix}
    = \varphi_k^{n-1} \overline{\breve{X}}_k ,
\]
поэтому:
\begin{multline*}
    P \breve{X}
    = \begin{pmatrix}
        \varphi_1^{n-1} \overline{\breve{X}}_1 & \varphi_2^{n-1} \overline{\breve{X}}_2 & \dots & \varphi_m^{n-1} \overline{\breve{X}}_m
    \end{pmatrix} = \\
    %
    = \begin{pmatrix}
        \overline{\breve{X}}_1 & \overline{\breve{X}}_2 & \dots & \overline{\breve{X}}_m
    \end{pmatrix}
    \begin{pmatrix}
        \varphi_1^{n-1} & 0               & \dots  & 0               \\
        0               & \varphi_2^{n-1} & \dots  & 0               \\
        \vdots          & \vdots          & \ddots & \vdots          \\
        0               & 0               & \dots  & \varphi_m^{n-1}
    \end{pmatrix} = \\
    %
    = \overline{\breve{X}}
    \begin{pmatrix}
        \varphi_1^{n-1} & 0               & \dots  & 0               \\
        0               & \varphi_2^{n-1} & \dots  & 0               \\
        \vdots          & \vdots          & \ddots & \vdots          \\
        0               & 0               & \dots  & \varphi_m^{n-1}
    \end{pmatrix} .
\end{multline*}
Отсюда
\[
    \left( P \breve{X} \right)^*
    =
    \begin{pmatrix}
        \varphi_1^{-(n-1)} & 0                  & \dots  & 0                  \\
        0                  & \varphi_2^{-(n-1)} & \dots  & 0                  \\
        \vdots             & \vdots             & \ddots & \vdots             \\
        0                  & 0                  & \dots  & \varphi_m^{-(n-1)}
    \end{pmatrix}
    \overline{\breve{X}^*} .
\]

Таким образом,
\begin{multline*}
    P R P = \\
    %
    = \sigma_0^2 I
    + \overline{\breve{X}}
    \begin{pmatrix}
        \varphi_1^{n-1} & \dots  & 0               \\
        \vdots          & \ddots & \vdots          \\
        0               & \dots  & \varphi_m^{n-1}
    \end{pmatrix}
    S
    \begin{pmatrix}
        \varphi_1^{-(n-1)} & \dots  & 0                  \\
        \vdots             & \ddots & \vdots             \\
        0                  & \dots  & \varphi_m^{-(n-1)}
    \end{pmatrix}
    \overline{\breve{X}^*} = \\
    %
    = \sigma_0^2 I
    + \overline{\breve{X}}
    \begin{pmatrix}
        \varphi_1^{n-1} & \dots  & 0               \\
        \vdots          & \ddots & \vdots          \\
        0               & \dots  & \varphi_m^{n-1}
    \end{pmatrix}
    \begin{pmatrix}
        \varphi_1^{-(n-1)} & \dots  & 0                  \\
        \vdots             & \ddots & \vdots             \\
        0                  & \dots  & \varphi_m^{-(n-1)}
    \end{pmatrix}
    S
    \overline{\breve{X}^*} = \\
    %
    = \sigma_0^2 I + \overline{\breve{X}} S \overline{\breve{X}^*}
    = \sigma_0^2 I + \overline{\breve{X}} \overline{S} \overline{\breve{X}^*}
    = \overline{\left( \sigma_0^2 I + \breve{X} S \breve{X}^* \right)}
    = \overline{R} ,
\end{multline*}
где $S$ коммутирует, поскольку $S$ --- диагональная, а также $S = \overline{S}$, поскольку $S$ --- действительная.

\subsection{SVD--разложение}

Образуем матрицу $C$
\[
    C
    = Q^* R Q .
\]
Почему-то в статье \cite{Cao_Liu} матрица $C$ определяется в эквивалентном виде:
\[
    C
    = \frac{1}{2} Q^* \left( R + P \overline{R} P \right) Q ,
\]
где $R$ --- центральноэрмитовая, поэтому:
\begin{gather*}
    P R P = \overline{R} , \\
    \overline{P R P} = R , \\
    P \overline{R} P = R ,
\end{gather*}
и $C$ преобразуется к первоначальному виду:
\[
    C
    = \frac{1}{2} Q^* \left( R + R \right) Q
    = \frac{1}{2} Q^* \left( 2 R \right) Q
    = Q^* R Q .
\]

Заметим, что матрица $C$ действительная, поскольку $R$ --- центральноэрмитовая.

Пусть вычислено SVD--разложение матрицы $C$:
\[
    C = U_s \Lambda_s V_s^* + U_n \Lambda_n V_n^* ,
\]
где индекс $s$ соответствует сигнальному, а $n$ --- шумовому подпространствам, тогда SVD--разложение матрицы $R$:
\[
    R
    = Q C Q^*
    = Q \left( U_s \Lambda_s V_s^* + U_n \Lambda_n V_n^* \right) Q^*
    = Q U_s \Lambda_s V_s^* Q^* + Q U_n \Lambda_n V_n^* Q^* .
\]
Сигнальному подпространству соответствует матрица:
\begin{equation}~\label{direction:svd_signal_columns}
    U_c = Q U_s .
\end{equation}

\subsection{Без шума}

\subsubsection{Система}

Если шума нет, $\sigma_0 = 0$, тогда в матрице $U_c$ ровно $m$ ненулевых столбцов по количеству векторов направлений $\breve{X}$:
\begin{equation}~\label{direction:U_columns}
    U_c
    = \begin{pmatrix}
        U_1 & \dots & U_m & 0 & \dots & 0
    \end{pmatrix} ,
\end{equation}
при этом линейные оболочки столбцов $U_k$ и $\breve{X}_k$ совпадают:
\[
    \linear{U_1, \dots, U_m} = \linear{\breve{X}_1, \dots, \breve{X}_m} ,
\]
значит любой столбец $U_k$ является линейной комбинацией $\breve{X}_k$ при некоторых $x_k$:
\begin{gather}
    U_k = \breve{X}_1 x_1 + \dots + \breve{X}_m x_m , \notag \\
    %
    \begin{pmatrix}
        u_1    \\
        u_2    \\
        \vdots \\
        u_n
    \end{pmatrix}
    = \begin{pmatrix}
        1         \\
        \varphi_1 \\
        \vdots    \\
        \varphi_1^{n-1}
    \end{pmatrix}
    x_1
    + \dots
    + \begin{pmatrix}
        1         \\
        \varphi_m \\
        \vdots    \\
        \varphi_m^{n-1}
    \end{pmatrix}
    x_m
    \label{direction:u_system}
\end{gather}
В матрице правой части:
\begin{equation}~\label{direction:shifts_powers}
    \begin{pmatrix}
        1               & 1               & \dots  & 1               \\
        \varphi_1       & \varphi_2       & \dots  & \varphi_m       \\
        \vdots          & \vdots          & \ddots & \vdots          \\
        \varphi_1^{n-1} & \varphi_2^{n-1} & \dots  & \varphi_m^{n-1}
    \end{pmatrix}
\end{equation}
рассмотрим первые $m+1$ строку:
\[
    \begin{pmatrix}
        1           & 1           & \dots  & 1           \\
        \varphi_1   & \varphi_2   & \dots  & \varphi_m   \\
        \vdots      & \vdots      & \ddots & \vdots      \\
        \varphi_1^m & \varphi_2^m & \dots  & \varphi_m^m
    \end{pmatrix}
\]
В этой матрице $m+1$ строка и $m$ столбцов, поэтому строки линейнозависимы, и найдутся числа $c_k$ при которых линейная комбинация строк даёт нулевую
строку:
\[
    \begin{pmatrix}
        c_1 & c_2 & \dots & c_{m+1}
    \end{pmatrix}
    \begin{pmatrix}
        1           & 1           & \dots  & 1           \\
        \varphi_1   & \varphi_2   & \dots  & \varphi_m   \\
        \vdots      & \vdots      & \ddots & \vdots      \\
        \varphi_1^m & \varphi_2^m & \dots  & \varphi_m^m
    \end{pmatrix}
    = \begin{pmatrix}
        0 & 0 & \dots & 0
    \end{pmatrix}
    .
\]
Теперь рассмотрим $m+1$ строку матрицы \eqref{direction:shifts_powers}, начиная со строки с номером $p$ и умножим слева на тот же вектор чисел $c_k$:
\begin{multline}~\label{direction:c_zeroing}
    \begin{pmatrix}
        c_1 & c_2 & \dots & c_{m+1}
    \end{pmatrix}
    \begin{pmatrix}
        \varphi_1^{p-1}   & \varphi_2^{p-1}   & \dots  & \varphi_m^{p-1}   \\
        \varphi_1^{p-1+1} & \varphi_2^{p-1+1} & \dots  & \varphi_m^{p-1+1} \\
        \vdots            & \vdots            & \vdots & \vdots            \\
        \varphi_1^{p-1+m} & \varphi_2^{p-1+m} & \dots  & \varphi_m^{p-1+m}
    \end{pmatrix} = \\
    %
    = \begin{pmatrix}
        c_1 & c_2 & \dots & c_{m+1}
    \end{pmatrix}
    \begin{pmatrix}
        1           & 1           & \dots  & 1           \\
        \varphi_1   & \varphi_2   & \dots  & \varphi_m   \\
        \vdots      & \vdots      & \vdots & \vdots      \\
        \varphi_1^m & \varphi_2^m & \dots  & \varphi_m^m
    \end{pmatrix}
    \begin{pmatrix}
        \varphi_1^{p-1} & 0               & \dots  & 0               \\
        0               & \varphi_2^{p-1} & \dots  & 0               \\
        \vdots          & \vdots          & \vdots & \vdots          \\
        0               & 0               & \dots  & \varphi_m^{p-1}
    \end{pmatrix} = \\
    %
    = \begin{pmatrix}
        0 & 0 & \dots & 0
    \end{pmatrix}
    \begin{pmatrix}
        \varphi_1^{p-1} & 0               & \dots  & 0               \\
        0               & \varphi_2^{p-1} & \dots  & 0               \\
        \vdots          & \vdots          & \vdots & \vdots          \\
        0               & 0               & \dots  & \varphi_m^{p-1}
    \end{pmatrix}
    %
    = \begin{pmatrix}
        0 & 0 & \dots 0
    \end{pmatrix} .
\end{multline}

Рассмотрим $m+1$ равенство системы \eqref{direction:u_system}, начиная со строки с номером $p$:
\[
    \begin{pmatrix}
        u_p     \\
        u_{p+1} \\
        \vdots  \\
        u_{p+m}
    \end{pmatrix}
    =
    \begin{pmatrix}
        \varphi_1^{p-1}   & \varphi_2^{p-1}   & \dots  & \varphi_m^{p-1}   \\
        \varphi_1^{p-1+1} & \varphi_2^{p-1+1} & \dots  & \varphi_m^{p-1+1} \\
        \vdots            & \vdots            & \vdots & \vdots            \\
        \varphi_1^{p-1+m} & \varphi_2^{p-1+m} & \dots  & \varphi_m^{p-1+m}
    \end{pmatrix}
    \begin{pmatrix}
        x_1    \\
        \vdots \\
        x_m
    \end{pmatrix}
\]
Умножим слева на вектор чисел $c_k$:
\begin{multline*}
    \begin{pmatrix}
        c_1 & c_2 & \dots & c_{m+1}
    \end{pmatrix}
    \begin{pmatrix}
        u_p     \\
        u_{p+1} \\
        \vdots  \\
        u_{p+m}
    \end{pmatrix} = \\
    %
    = \begin{pmatrix}
        c_1 & c_2 & \dots & c_{m+1}
    \end{pmatrix}
    \begin{pmatrix}
        \varphi_1^{p-1}   & \varphi_2^{p-1}   & \dots  & \varphi_m^{p-1}   \\
        \varphi_1^{p-1+1} & \varphi_2^{p-1+1} & \dots  & \varphi_m^{p-1+1} \\
        \vdots            & \vdots            & \vdots & \vdots            \\
        \varphi_1^{p-1+m} & \varphi_2^{p-1+m} & \dots  & \varphi_m^{p-1+m}
    \end{pmatrix}
    \begin{pmatrix}
        x_1    \\
        \vdots \\
        x_m
    \end{pmatrix} = \\
    %
    = \begin{pmatrix}
        0 & 0 & \dots & 0
    \end{pmatrix}
    \begin{pmatrix}
        x_1    \\
        \vdots \\
        x_m
    \end{pmatrix}
    = 0 ,
\end{multline*}
в силу обнуляющего свойства \eqref{direction:c_zeroing} вектора чисел $c_k$.

Таким образом:
\[
    \begin{pmatrix}
        c_1 & c_2 & \dots & c_{m+1}
    \end{pmatrix}
    \begin{pmatrix}
        u_p     \\
        u_{p+1} \\
        \vdots  \\
        u_{p+m}
    \end{pmatrix}
    = 0 ,
\]
транспонируя, получим:
\[
    \begin{pmatrix}
        u_p & u_{p+1} & \dots & u_{p+m} \\
    \end{pmatrix}
    \begin{pmatrix}
        c_1    \\
        c_2    \\
        \vdots \\
        c_{m+1}
    \end{pmatrix}
    = 0 .
\]
Объединяя все такие равенства при $p=1, 2, \dots, n-m$ в систему, получим:
\[
    \underbrace{
        \begin{pmatrix}
            u_1     & u_2       & \dots  & u_{m+1} \\
            u_2     & u_3       & \dots  & u_{m+2} \\
            \vdots  & \vdots    & \vdots & \vdots  \\
            u_{n-m} & u_{n-m+1} & \dots  & u_n
        \end{pmatrix}
    }_{D[U_k]}
    \begin{pmatrix}
        c_1    \\
        c_2    \\
        \vdots \\
        c_{m+1}
    \end{pmatrix}
    = \begin{pmatrix}
        0      \\
        0      \\
        \vdots \\
        0
    \end{pmatrix} .
\]
Такая система получается для одного столбца $U_k$ матрицы $U_c$ из равенства \eqref{direction:U_columns}. Пусть $D[U_k]$ обозначает матрицу системы,
тогда в матричном виде система принимает вид:
\[
    D[U_k]
    \begin{pmatrix}
        c_1    \\
        c_2    \\
        \vdots \\
        c_{m+1}
    \end{pmatrix}
    =
    \begin{pmatrix}
        0      \\
        0      \\
        \vdots \\
        0
    \end{pmatrix} .
\]
Объединяя системы для всех векторов $U_k$ получим систему:
\begin{equation}~\label{direction:c_system}
    \begin{pmatrix}
        D[U_1] \\
        D[U_2] \\
        \vdots \\
        D[U_m] \\
    \end{pmatrix}
    \begin{pmatrix}
        c_1    \\
        c_2    \\
        \vdots \\
        c_{m+1}
    \end{pmatrix}
    =
    \begin{pmatrix}
        0      \\
        0      \\
        \vdots \\
        0
    \end{pmatrix} ,
\end{equation}
которой удовлетворяет вектор чисел $c_k$.

\subsubsection{Уравнение}

Из равенства \eqref{direction:c_zeroing} следует, что при любом $k$:
\[
    c_1 \cdot 1 + c_2 \cdot \varphi_k + \dots + c_{m+1} \cdot \varphi_k^{m} = 0 .
\]
Значит, для полинома:
\begin{equation}~\label{direction:c_polynomial}
    P(z) = c_1 + c_2 z + \dots + c_{m+1} z^m .
\end{equation}
числа $\varphi_k$ являются корнями:
\[
    P \left( \varphi_k \right) = 0 ,
\]
причем количество чисел $\varphi_k$ равно $m$ и полином $P(z)$ имеет степень $m$ и у него $m$ корней, поэтому каждый корень полинома $P(z)$
соответствует некоторому числу $\varphi_k$ и наоборот, и все корни располагаются на единичной окружности в комплексной плоскости.

\subsection{С шумом}~\label{direction:linear:estimation}

При наличии шума, $\sigma_0 \neq 0$, система \eqref{direction:c_system} может оказаться несовместной, поэтому нужно искать обобщённое решение:
\[
    \widehat{c} = arg \min_c \norm{D c}{W} ,
\]
где $W$ --- матрица весов (как выбрать неизвестно).

Затем необходимо искать корни полинома $\widehat{P}(z)$, аналогичного полиному \eqref{direction:c_polynomial}:
\[
    \widehat{P}(z)
    = \widehat{c}_1 + \widehat{c}_2 z + \dots \widehat{c}_m z^m ,
\]
каждый корень которого $\widehat{\varphi}_k$ имеет вид:
\[
    \widehat{\varphi}_k
    = a_k e^{i \Delta \widehat{\varphi}_k} ,
\]
где аргумент $\widehat{z}_k$ даёт оценку смещения:
\[
    \Delta \widehat{\varphi}_k
    = arg \left( \widehat{\varphi}_k \right).
\]

\section{Направление приёма в прямоугольной решётке}

Пусть $Z$ --- случайный вектор огибающих в приёмниках и $R$ --- ковариационная матрица вектора $Z$:
\[
    R = \sigma_0^2 I + \breve{Z} \variance{S} \breve{Z}^* ,
\]
где $\variance{S}$ --- диагональная матрица:
\[
    \variance{S}
    = \begin{pmatrix}
        d_1^2  & \dots  & 0      \\
        \vdots & \ddots & \vdots \\
        0      & \dots  & d_m^2
    \end{pmatrix} ,
\]
и $\breve{Z}$ --- матрица векторов направлений:
\[
    \breve{Z}
    = \begin{pmatrix}
        \breve{Z}_1 & \breve{Z}_2 & \dots & \breve{Z}_m
    \end{pmatrix} , \\
\]
Векторы $\breve{Z}_k$ можно формировать в направлении строк и в направлении столбцов прямоугольной решётки. Пусть в каждой строке решётки $n_x$ приёмников, и в каждом
столбце решётки $n_y$ приёмников. Пусть для $k$-го источника $\breve{X}_k$ обозначает вектор направления для одной строки решётки и $\breve{Y}_k$ --- вектор
направления для одного столбца решётки:
\begin{gather*}
    \breve{X}_k
    = \begin{pmatrix}
        1           \\
        \varphi_k   \\
        \varphi_k^2 \\
        \vdots      \\
        \varphi_k^{n_x-1}
    \end{pmatrix} ,
    %
    \breve{Y}_k
    = \begin{pmatrix}
        1          \\
        \theta_k   \\
        \theta_k^2 \\
        \vdots     \\
        \theta_k^{n_y - 1}
    \end{pmatrix} , \\
    %
    \varphi_k = e^{i \Delta \varphi_k} ,
    \theta_k = e^{i \Delta \theta_k} ,
\end{gather*}
где $\varphi_k$ и $\theta_k$ --- горизонтальные и вертикальные комплексные сдвиги $k$-го источника, которые считаются различными:
\begin{gather*}
    \theta_k \neq \theta_j , \varphi_k \neq \varphi_j , \\
    k \neq j .
\end{gather*}

\subsection{По строкам решётки}

Пусть векторы $\breve{Z}_k$ формируются в направлении строк решётки, сперва огибающие первой строки, затем огибающие второй строки и так далее,
\[
    \breve{Z}_k
    = \breve{Y}_k \otimes \breve{X}_k
    = \begin{pmatrix}
        \breve{X}_k            \\
        \theta_k \breve{X}_k   \\
        \theta_k^2 \breve{X}_k \\
        \vdots                 \\
        \theta_k^{n_y - 1} \breve{X}_k
    \end{pmatrix} .
\]
В матрице
\[
    \breve{Z} \variance{S} \breve{Z}^*
    = \breve{Z}_1 d_1^2 \breve{Z}_1^* + \dots + \breve{Z}_m d_m^2 \breve{Z}_m^* ,
\]
каждое слагаемое:
\begin{multline*}
    \breve{Z}_k d_k^2 \breve{Z}_k^*
    = \begin{pmatrix}
        \breve{X}_k                    \\
        \theta_k \breve{X}_k           \\
        \theta_k^2 \breve{X}_k         \\
        \vdots                         \\
        \theta_k^{n_y - 1} \breve{X}_k \\
    \end{pmatrix}
    d_k^2
    \begin{pmatrix}
        \breve{X}_k^* & \theta_k^{-1} \breve{X}_k^* & \theta_k^{-2} \breve{X}_k^* & \dots & \theta_k^{-(n_y - 1)} \breve{X}_k^*
    \end{pmatrix} = \\
    %
    = d_k^2
    \begin{pmatrix}
        1                & \theta_k^{-1}    & \theta_k^{-2}    & \dots  & \theta_k^{-(n_y-1)} \\
        \theta_k         & 1                & \theta_k^{-1}    & \dots  & \theta_k^{-(n_y-2)} \\
        \theta_k^2       & \theta_k^1       & 1                & \dots  & \theta_k^{-(n_y-3)} \\
        \vdots           & \vdots           & \vdots           & \ddots & \vdots              \\
        \theta_k^{n_y-1} & \theta_k^{n_y-2} & \theta_k^{n_y-3} & \dots  & 1
    \end{pmatrix}
    \otimes
    \breve{X}_k \breve{X}_k^*
    =
    d_k^2
    \cdot
    \breve{Y}_k \breve{Y}_k^*
    \otimes
    \breve{X}_k \breve{X}_k^*
    .
\end{multline*}

Матрица $R$ по-прежнему центральноэрмитовая. Рассмотрим:
\[
    P R P
    = P \left ( \sigma_0^2 I + \breve{Z} S \breve{Z}^* \right ) P
    = \sigma_0^2 P P + P \breve{Z} S \breve{Z}^* P
    = \sigma_0^2 I + P \breve{Z} S \left( P \breve{Z} \right)^* ,
\]
где
\[
    P \breve{Z}
    = \begin{pmatrix}
        P \breve{Z}_1 & P \breve{Z}_2 & \dots & P \breve{Z}_m
    \end{pmatrix}
\]
и каждый столбец
\begin{multline*}
    P \breve{Z}_k
    = \begin{pmatrix}
        0      & \dots & 0      & P      \\
        0      & \dots & P      & 0      \\
        \vdots & \dots & \vdots & \vdots \\
        P      & \dots & 0      & 0
    \end{pmatrix}
    \begin{pmatrix}
        1 \cdot \breve{X}_k                \\
        \theta_k \cdot \breve{X}_k         \\
        \vdots                             \\
        \theta_k^{n_y-1} \cdot \breve{X}_k \\
    \end{pmatrix}
    =
    \begin{pmatrix}
        \theta_k^{n_y-1} \cdot P \breve{X}_k \\
        \vdots                               \\
        \theta_k \cdot P \breve{X}_k         \\
        1 \cdot P \breve{X}_k                \\
    \end{pmatrix} =    \\
    %
    = \theta_k^{n_y - 1}
    \begin{pmatrix}
        1 \cdot P \breve{X}_k                     \\
        \vdots                                    \\
        \theta_k^{-(n_y - 2)} \cdot P \breve{X}_k \\
        \theta_k^{-(n_y - 1)} \cdot P \breve{X}_k \\
    \end{pmatrix}
    = \theta_k^{n_y - 1}
    \begin{pmatrix}
        1 \cdot \varphi_k^{n_x - 1} \overline{\breve{X}}_k                     \\
        \vdots                                                                 \\
        \theta_k^{-(n_y - 2)} \cdot \varphi_k^{n_x - 1} \overline{\breve{X}}_k \\
        \theta_k^{-(n_y - 1)} \cdot \varphi_k^{n_x - 1} \overline{\breve{X}}_k \\
    \end{pmatrix} = \\
    %
    = \theta_k^{n_y - 1} \varphi_k^{n_x - 1}
    \begin{pmatrix}
        1 \cdot \overline{\breve{X}}_k                     \\
        \vdots                                             \\
        \theta_k^{-(n_y - 2)} \cdot \overline{\breve{X}}_k \\
        \theta_k^{-(n_y - 1)} \cdot \overline{\breve{X}}_k \\
    \end{pmatrix}
    = \theta_k^{n_y - 1} \varphi_k^{n_x - 1}
    \begin{pmatrix}
        \overline{1 \cdot \breve{X}}_k                    \\
        \vdots                                            \\
        \overline{\theta_k^{(n_y - 2)} \cdot \breve{X}}_k \\
        \overline{\theta_k^{(n_y - 1)} \cdot \breve{X}}_k \\
    \end{pmatrix}
    = \theta_k^{n_y - 1} \varphi_k^{n_x - 1} \overline{\breve{Z}}_k .
\end{multline*}
Для всех столбцов,
\begin{multline*}
    P \breve{Z}
    =
    \begin{pmatrix}
        \theta_1^{n_y - 1} \varphi_1^{n_x - 1} \overline{\breve{Z}}_1 & \dots & \theta_m^{n_y - 1} \varphi_m^{n_x - 1} \overline{\breve{Z}}_m
    \end{pmatrix} = \\
    %
    =
    \begin{pmatrix}
        \overline{\breve{Z}}_1 & \dots & \overline{\breve{Z}}_m
    \end{pmatrix}
    \underbrace{
        \begin{pmatrix}
            \theta_1^{n_y - 1} \varphi_1^{n_x - 1} & \dots  & 0                                      \\
            \vdots                                 & \ddots & \vdots                                 \\
            0                                      & \dots  & \theta_m^{n_y - 1} \varphi_m^{n_x - 1}
        \end{pmatrix}
    }_N
    =
    \overline{\breve{Z}} N .
\end{multline*}
Отсюда
\[
    \left( P \breve{Z} \right)^*
    = \left( \overline{\breve{Z}} N \right)^*
    = N^* \overline{\breve{Z}^*}
\]
Таким образом,
\[
    P R P
    = \sigma_0^2 I + \overline{\breve{Z}} N S N^* \overline{\breve{Z}^*}
    = \sigma_0^2 I + \overline{\breve{Z}} N N^* S \overline{\breve{Z}^*}
    = \sigma_0^2 I + \overline{\breve{Z}} S \overline{\breve{Z}^*}
    = \overline{\sigma_0^2 I + \breve{Z} S \breve{Z}^*}
    = \overline{R} ,
\]
где $S$ коммутирует c $N^*$, поскольку $S$ и $N^*$ диагональные, а из определения $N$:
\[
    N N^*
    =
    \begin{pmatrix}
        \theta_1^{n_y - 1} \varphi_1^{n_x - 1} & \dots  & 0                                      \\
        \vdots                                 & \ddots & \vdots                                 \\
        0                                      & \dots  & \theta_m^{n_y - 1} \varphi_m^{n_x - 1}
    \end{pmatrix}
    \begin{pmatrix}
        \theta_1^{-(n_y - 1)} \varphi_1^{-(n_x - 1)} & \dots  & 0                                            \\
        \vdots                                       & \ddots & \vdots                                       \\
        0                                            & \dots  & \theta_m^{-(n_y - 1)} \varphi_m^{-(n_x - 1)}
    \end{pmatrix}
    = I .
\]

Поскольку $R$ --- центральноэрмитовая матрица, то матрица:
\[
    C = Q^* R Q
\]
действительная и SVD--разложение вычисляется для действительной матрицы $C$ как и в случае линейной решётки.

Как и ранее, любой столбец $U_k$ из равенства \eqref{direction:U_columns} может быть получен линейной комбинацией столбцов $\breve{Z}_k$, поэтому
существуют числа $z_k$, при которых справедливо равенство аналогичное \eqref{direction:u_system}:
\begin{gather*}
    U_k = \breve{Z}_1 z_1 + \dots + \breve{Z}_m z_m , \notag \\
    %
    \begin{pmatrix}
        U_k^{(1)} \\
        U_k^{(2)} \\
        \vdots    \\
        U_k^{(n_y)}
    \end{pmatrix}
    =
    \begin{pmatrix}
        \breve{X}_1                \\
        \theta_1 \cdot \breve{X}_1 \\
        \vdots                     \\
        \theta_1^{n_y-1} \cdot \breve{X}_1
    \end{pmatrix}
    z_1
    + \dots
    + \begin{pmatrix}
        \breve{X}_m                \\
        \theta_m \cdot \breve{X}_m \\
        \vdots                     \\
        \theta_m^{n_y-1} \cdot \breve{X}_m
    \end{pmatrix}
    z_m ,
\end{gather*}
откуда
\[
    U_k^{(p)}
    = \theta_1^{p-1} \breve{X}_1 z_1 + \dots + \theta_m^{p-1} \breve{X}_m z_m
    .
\]
При $p=1$ получается равенство
\[
    U_k^{(1)}
    = \breve{X}_1 z_1 + \dots + \breve{X}_m z_m
    ,
\]
из которого следует существование вектора чисел $c_k$ с анулирующим свойством \eqref{direction:c_zeroing}. Вектор чисел $c_k$ обладает таким же свойством
для блоков с $p=\overline{2,n_y}$, поскольку блоки представляются в виде:
\[
    U_k^{(p)}
    = \begin{pmatrix}
        \theta_1^{p-1} \breve{X}_1 & \dots & \theta_m^{p-1} \breve{X}_m
    \end{pmatrix}
    \begin{pmatrix}
        z_1    \\
        z_2    \\
        \vdots \\
        z_m
    \end{pmatrix}
    = \begin{pmatrix}
        \breve{X}_1 & \dots & \breve{X}_m
    \end{pmatrix}
    \begin{pmatrix}
        \theta_1^{p-1} & \dots  & 0              \\
        \vdots         & \ddots & \vdots         \\
        0              & \dots  & \theta_m^{p-1}
    \end{pmatrix}
    \begin{pmatrix}
        z_1    \\
        z_2    \\
        \vdots \\
        z_m
    \end{pmatrix}
    ,
\]
так что при умножении на вектор чисел $c_k$ слева будет происходить обнуление строк в самом левом множителе.

Из обнуляющего свойства вектора чисел $c_k$ для каждого блока будет следовать система равенств вида \eqref{direction:c_system}, поэтому для всех блоков
получим систему вида:
\[
    \begin{pmatrix}
        D[U_1^{(1)}] \\
        \vdots       \\
        D[U_m^{(1)}] \\
        D[U_1^{(2)}] \\
        \vdots       \\
        D[U_m^{(2)}] \\
        \vdots       \\
        D[U_1^{(m)}] \\
        \vdots       \\
        D[U_m^{(m)}] \\
    \end{pmatrix}
    \begin{pmatrix}
        c_1    \\
        c_2    \\
        \vdots \\
        c_{m+1}
    \end{pmatrix}
    =
    \begin{pmatrix}
        0      \\
        \vdots \\
        0      \\
        0      \\
        \vdots \\
        0      \\
        \vdots \\
        0      \\
        \vdots \\
        0
    \end{pmatrix} ,
\]
Далее следует проводить оценку способом, аналогичным изложенному в разделе \ref{direction:linear:estimation}, в результате будут получены оценки комплексных
сдвигов $\widehat{\varphi}_k$.

\subsection{По столбцам решётки}

Для получения оценок вертикальных комплексных сдвигов $\theta_k$ необходимо формировать векторы направлений по столбцам:
\[
    \begin{pmatrix}
        \breve{Y}_k           \\
        \varphi_k \breve{Y}_k \\
        \vdots                \\
        \varphi_k^{n_x-1} \breve{Y}_k
    \end{pmatrix}
    = \breve{X}_k \otimes \breve{Y}_k
    = \Pi \breve{Z}_k ,
\]
что может быть выполнено путём перестановки $\Pi$ вектора $\breve{Z}_k$. В блок $\varphi_k^p \cdot \breve{Y}_k$ необходимо сгруппировать строки вектора
$\breve{Z}_k$, в которых есть $\varphi_k^p$, то есть необходимо в векторе $\breve{Z}_k$ взять строку с номером $p+1$ и от неё двигаться в шагом $n_y$ до
конца вектора $\breve{Z}_k$.

При формировании векторов направлений по стролбцам решётки матрица направлений будет иметь вид:
\[
    \begin{pmatrix}
        \Pi \breve{Z}_1 & \dots & \Pi \breve{Z}_m
    \end{pmatrix}
    = \Pi \begin{pmatrix}
        \breve{Z}_1 & \dots & \breve{Z}_m
    \end{pmatrix}
    = \Pi \breve{Z} .
\]
Ковариационная матрица будет иметь вид:
\[
    \sigma_0^2 I + \Pi \breve{Z} \variance{S} \breve{Z}^* \Pi^*
    = \sigma_0^2 \Pi \Pi^* + \Pi \breve{Z} \variance{S} \breve{Z}^* \Pi^*
    = \Pi \left( \sigma_0^2 I + \breve{Z} \variance{S} \breve{Z}^* \right) \Pi^*
    = \Pi R \Pi^*
\]
Пусть SVD--разложение матрицы $R$ имеет вид:
\[
    R = U_R \Sigma_R V_R^* ,
\]
тогда
\[
    \Pi R \Pi^*
    = \left( \Pi U_R \right) \Sigma_R \left(V_R^* \Pi^* \right) .
\]
Таким образом, после вычисления столбцов, образующих сигнальное подпространство \eqref{direction:svd_signal_columns}, необходимо переставить в них строки,
в соответствии с перестановкой $\Pi$:
\[
    \Pi U_c = \Pi Q U_s .
\]
Далее для столбцов в левой части повторить метод оценивания: вычислить оценки $\widehat{c}_k$ и корни полинома с коэффициентами $\widehat{c}_k$, по которым
построить оценки $\widehat{\theta}_k$.