\chapter{Направление приёма}

\[
    \begin{pmatrix}
        u_1 \\
        u_2 \\
        u_3 \\
        u_4 \\
        u_5
    \end{pmatrix}
    \begin{array}{l}
        c_1 \\
        c_2 \\
        c_3 \\
        \\
        \\
    \end{array}
    \overbrace{
    \begin{pmatrix}
        a_1   & a_2   & a_3   \\
        a_1^2 & a_2^2 & a_3^2 \\
        a_1^3 & a_2^3 & a_3^3 \\
        a_1^4 & a_2^4 & a_3^4 \\
        a_1^5 & a_2^5 & a_3^5
    \end{pmatrix}
    }^{\begin{array}{ccc} \xi_1 & \xi_2 & \xi_3 \end{array}}
\]

\begin{align*}
    u_1 = & a_1 \xi_1 + a_2 \xi_2 + a_3 \xi_3 , \\
    u_2 = & a_1^2 \xi_1 + a_2^2 \xi_2 + a_3^2 \xi_3 , \\
    u_3 = & a_1^3 \xi_1 + a_2^3 \xi_2 + a_3^3 \xi_3 
\end{align*}

\begin{gather}
    u_k = \sum_{j=1}^3 \xi_j e^{i \mu_j k} , \\
\end{gather}
\begin{multline*}
    u_n = a_1 u_{n-1} + a_2 u_{n-2} + a_3 u_{n-3} = \\
    %
    = a_1 \sum_{j=1}^3 \xi_j e^{i \mu_j (n-1)} + a_2 \sum_{j=1}^3 \xi_j e^{i \mu_j (n-2)} + a_3 \sum_{j=1}^3 \xi_j e^{i \mu_j (n-3)} = \\
    %
    \shoveleft{= \xi_1 e^{i \mu_1 n} \left ( a_1 e^{- i \mu_1} + a_2 e^{- i 2 \mu_1} + a_3 e^{- i 3 \mu_1} \right) +} \\
    + \xi_2 e^{i \mu_2 n} \left ( a_1 e^{- i \mu_2} + a_2 e^{- i 2 \mu_2} + a_3 e^{- i 3 \mu_2} \right) + \\
    \shoveright{+ \xi_3 e^{i \mu_3 n} \left ( a_1 e^{- i \mu_3} + a_2 e^{- i 2 \mu_3} + a_3 e^{- i 3 \mu_3} \right)} \\
\end{multline*}
Будет вылнено, если величины $a_1$, $a_2$, $a_3$:
\[
    \begin{pmatrix}
        e^{- i \mu_1} & e^{- i 2 \mu_1} & e^{- i 3 \mu_1} \\
        e^{- i \mu_2} & e^{- i 2 \mu_2} & e^{- i 3 \mu_2} \\
        e^{- i \mu_3} & e^{- i 2 \mu_3} & e^{- i 3 \mu_3}
    \end{pmatrix}
    \begin{pmatrix}
        a_1 \\
        a_2 \\
        a_3
    \end{pmatrix}
    = \begin{pmatrix}
        1 \\
        1 \\
        1
    \end{pmatrix}
\]
Матрица невырождена, поскольку определитель является определителем Вандермонда.
